
\documentclass[twoside,11pt]{article}
\title{嘉绒-汉-法词典}
\author{Guillaume Jacques}
\usepackage[paperwidth=185mm,paperheight=260mm,top=16mm,bottom=16mm,left=15mm,right=20mm]{geometry}
\usepackage{multicol}
\setlength{\columnseprule}{1pt}
\setlength{\columnsep}{1.5cm}
\usepackage{changepage}
\usepackage[dvipsnames,table]{xcolor}
\usepackage{fancyhdr}
\pagestyle{fancy}
\fancyheadoffset{3.4em}
\fancyhead[LE,LO]{\rightmark}
\fancyhead[RE,RO]{\leftmark}
\usepackage{hyperref}
\hypersetup{pdftex,bookmarks=true,bookmarksnumbered,bookmarksopenlevel=5,bookmarksdepth=5,xetex,colorlinks=true,linkcolor=blue,citecolor=blue}
\usepackage[all]{hypcap}
\usepackage{fontspec}
\usepackage{natbib}
\usepackage{booktabs}
\usepackage{polyglossia}
\usepackage{media9}
\setdefaultlanguage{french}
\setotherlanguages{french,english}
\setmainfont{Charis SIL}
\usepackage{media9}
\usepackage{graphicx}
\usepackage{totcount}
\newcounter{compteur}
\setcounter{compteur}{0}
\regtotcounter{compteur}
\newfontfamily{\prin}[Mapping=tex-text,Ligatures=Common,Scale=MatchUppercase]{Liberation Serif}
\newfontfamily{\jya}[Mapping=tex-text,Ligatures=Common,Scale=MatchUppercase]{Charis SIL}
\newfontfamily{\fra}[Mapping=tex-text,Ligatures=Common,Scale=MatchUppercase]{EB Garamond}
\newfontfamily{\cmn}[Mapping=tex-text,Ligatures=Common,Scale=MatchUppercase]{AR PL UMing CN}
\newfontfamily{\eng}[Mapping=tex-text,Ligatures=Common,Scale=MatchUppercase]{Liberation Serif}
\newcommand{\pprin}[1]{\begin{cmn}{\prin #1}\end{cmn}}
\newcommand{\pjya}[1]{{\jya\textcolor{Blue}{#1}}}
\newcommand{\pfra}[1]{\begin{french}{\fra\textcolor{OliveGreen}{#1}}\end{french}}
\newcommand{\pcmn}[1]{{\cmn\textcolor{black}{#1}}}
\newcommand{\peng}[1]{\begin{english}{\eng\textcolor{Sepia}{#1}}\end{english}}
\newcommand{\cerclé}[1]{\raisebox{0pt}{\textcircled{\raisebox{-0.5pt} {\footnotesize{\pjya{#1}}}}}}
\newcommand{\caractère}[1]{\phantomsection\addcontentsline{toc}{section}{#1}{\begin{center}\textbf{\Large\pjya{#1}}\end{center}}}
\newenvironment{entrée}[3]{\hypertarget{#3}{}\phantomsection\addcontentsline{toc}{subsection}{#1\homonyme{#2}}\hspace*{-0.5cm}\textbf{\Large\pjya{#1 \homonyme{#2}}}\markright{#1 \homonyme{#2}}}{\stepcounter{compteur}}
\newenvironment{sous-entrée}[2]{\hypertarget{#2}{}\phantomsection\addcontentsline{toc}{subsubsection}{#1}\hspace*{-0.3cm}\pprin{■} \textbf{\Large\pjya{#1}}}{}\newcommand{\homonyme}[1]{#1}
\newcommand{\variante}[1]{\small \pjya{#1}}
\newcommand{\classe}[1]{\textcolor{PineGreen}{#1} }
\newcommand{\paradigme}[2]{#1 : \pjya{#2} }
\newcommand{\relationsémantique}[2]{\pcmn{【#1】}\pjya{#2} }
\newcommand{\forme}[2]{#1 : \pjya{#2} }
\newcommand{\sens}[1]{ \cerclé{#1} }
\newenvironment{définition}{}{\hspace{5pt}}
\newenvironment{déclaration}{}{}
\newenvironment{exemple}{\pprin{¶} }{\hspace{5pt}}
\newenvironment{informationencyclopédique}{}{\hspace{5pt}}
\newcommand{\étiquette}[1]{\pcmn{~【同义词】~\pjya{#1}} }
\newcommand{\synonyme}[1]{\pcmn{~【同义词】~\pjya{#1}} }
\newcommand{\antonyme}[1]{\pcmn{~【反义词】~\pjya{#1}} }
\newcommand{\confer}[1]{\pcmn{~【参考】~\pjya{#1}} }
\newcommand{\emprunt}[1]{\pcmn{~【借词】~#1} }
\newcommand{\étymologie}[1]{\pcmn{~【词源】~\pjya{#1}} }
\newcommand{\utilisation}[1]{\pcmn{~【用法】#1} }
\newcommand{\grammaire}[1]{\textsc{#1} }
\newcommand{\lien}[2]{\hyperlink{#1}{\pjya{#2}}}
\newcommand{\stylefv}[1]{\pjya{#1}}
\newcommand{\stylefn}[1]{\pcmn{#1}}
\newcommand{\stylefi}[1]{\textit{#1}}
\newcommand{\stylefg}[1]{\textsc{#1}}
\newcommand{\caps}[1]{\pprin{\textsc{#1}}}
\newcommand{\ital}[1]{\pprin{\textit{#1}}}
\newenvironment{bottompar}{\par\vspace*{\fill}}{\clearpage}
\newcommand{\écouterélevée}[1]{\includemedia[activate=onclick,addresource=#1.mp3,flashvars={source=#1.mp3&autoPlay=true&autoRewind=true&loop=false&hideBar=true&volume=1.0&balance=0.0}]{ \includegraphics[height=8pt]{../images/volume.png}}{APlayer.swf}}
\newcommand{\écouterfaible}[1]{\includemedia[activate=onclick,addresource=8_#1.mp3,flashvars={source=#1.mp3&autoPlay=true&autoRewind=true&loop=false&hideBar=true&volume=1.0&balance=0.0}]{ \includegraphics[height=8pt]{../images/volume.png}}{APlayer.swf}}
\addmediapath{/run/media/benjamin/848A-5DEB/Lacito/AudioJaphug}
\XeTeXlinebreaklocale "zh"
\XeTeXlinebreakskip = 0pt plus 1pt
\ExplSyntaxOn
% Code spécial pour la gestion générique des césures applicable aux formes de surface
\RenewDocumentCommand{\formedesurface}{m}
{
% nouvelle variable « expression »
\tl_set:Nn \expression { #1 }
% remplace ˩˧˥ par ˩˧˥\-
\regex_replace_all:nnN { (\B[˩˧˥]) } { \1\c{-} } \expression
% renvoie la séquence totale
{\tl_use: {\hspace{0.5cm}/\pjya{\expression}/\hspace{0.5cm}}}
}
\ExplSyntaxOff

\begin{document}
introduction.tex
\pagenumbering{arabic}
\setcounter{page}{1}
\setlength{\parindent}{0pt}
\begin{multicols}{2}
\lhead{\firstmark}
\rhead{\botmark}
\newpage\caractère{a}

\begin{entrée}{aboʁboʁ}{}{ⓔaboʁboʁ} 
\classe{vs} \paradigme{dir}{kɤ-}
\begin{définition}\pfra{se blottir ensemble}\end{définition}
\begin{définition}\pcmn{集中在一起;缩成一团}\end{définition}
\begin{exemple}\pjya{khɯzɤpɯ ra nɯ-mu ɯ-ɕki ko-k-ɤboʁboʁ-nɯ-ci ma ɲɯ-nɤndʐo-nɯ}\hspace{5pt}\pcmn{因为冷,狗崽子们集拢在妈妈身边}\end{exemple}\relationsémantique{参考}{\lien{ⓔnɤboʁboʁ}{nɤboʁboʁ}}\relationsémantique{参考}{\lien{ⓔboʁ}{boʁ}}\end{entrée}

\begin{entrée}{abrɤlbrɤl/\variante{abrabrɤl}}{}{ⓔabrɤlbrɤl} 
\classe{vs} \paradigme{dir}{nɯ-}
\begin{définition}\pfra{espacé, clairsemé}\end{définition}
\begin{définition}\pcmn{稀疏不密(树)}\end{définition}
\begin{définition}\pfra{rendre espacé}\end{définition}
\begin{définition}\pcmn{使稀疏}\end{définition}
\begin{exemple}\pjya{tɯ-rɣi kɤ-lɤt tɕe, ɲɯ́-wɣ-sɤbrabrɤl ra}\hspace{5pt}\pcmn{播种的时候,要播得稀疏一点}\end{exemple}
\begin{exemple}\pjya{laχtɕha kɤ-ta tɕe, ɲɯ́-wɣ-sɤbrabrɤl ra ma nɯ maʁ nɤ andɯndo}\hspace{5pt}\pcmn{放东西的时候,要放开一点,不然就会粘在一起}\end{exemple}
\begin{sous-entrée}{sɤbrabrɤl}{ⓔabrɤlbrɤlⓝsɤbrabrɤl} 
\classe{vt} \end{sous-entrée}

\end{entrée}

\begin{entrée}{aβdɤβde}{}{ⓔaβdɤβde} 
\classe{vi}  
\grammaire{caus} \paradigme{dir}{thɯ-}\paradigme{dir}{thɯ-}
\begin{définition}\pfra{être retardé, être reporté à plus tard}\end{définition}
\begin{définition}\pcmn{拖延}\end{définition}
\begin{exemple}\pjya{kɤ-nɤma nɯ thɯ-aβdɤβde pɯ-ra}\hspace{5pt}\pcmn{工作拖延了}\end{exemple}\relationsémantique{参考}{\lien{ⓔβde}{βde}}
\begin{sous-entrée}{sɤβdɤβde}{ⓔaβdɤβdeⓝsɤβdɤβde} 
\classe{vt} \end{sous-entrée}

\begin{définition}\pfra{retarder}\end{définition}
\begin{définition}\pcmn{拖延}\end{définition}
\begin{exemple}\pjya{kɤ-nɤma thɯ-sɤβdɤβde-t-a}\hspace{5pt}\pcmn{我拖延了工作}\end{exemple}
\begin{exemple}\pjya{ɯ-ma chɤ-sɤβdɤβde}\hspace{5pt}\pcmn{他的工作拖延了时间}\end{exemple}
\begin{exemple}\pjya{smɤn kɤ-ndza nɯ ma-thɯ-tɯ-sɤβdɤβde kɯ ɯ-khrɤt ʑo tɤ-ndze ɲɯ-ra ma nɯ mɤɕtʂa mɤ-phɤn}\hspace{5pt}\pcmn{你要按规定的时间和数量吃药,不然就没有效果。}\end{exemple}\end{entrée}

\begin{entrée}{aβdoʁβdi}{}{ⓔaβdoʁβdi} 
\classe{vs} 
\begin{définition}\pfra{aller bien}\end{définition}
\begin{définition}\pcmn{平安;健康}\end{définition}
\begin{exemple}\pjya{iʑora ɕɤxɕo ku-oβdoʁβdi-j}\hspace{5pt}\pcmn{我们这段时间都日子过得很平安}\end{exemple}\relationsémantique{参考}{\lien{ⓔβdi}{βdi}}\relationsémantique{参考}{\lien{ⓔɣɤβdi}{ɣɤβdi}}\end{entrée}

\begin{entrée}{aβdoʁdi}{}{ⓔaβdoʁdi}\relationsémantique{参考}{\lien{ⓔɣɤβdi}{ɣɤβdi}}\end{entrée}

\begin{entrée}{aβraʁ}{}{ⓔaβraʁ}\relationsémantique{参考}{\lien{ⓔβraʁ}{βraʁ}}\end{entrée}

\begin{entrée}{aβrdaβrdoŋ}{}{ⓔaβrdaβrdoŋ} 
\classe{vs} \paradigme{dir}{thɯ-}
\begin{définition}\pfra{vigoureux, robuste}\end{définition}
\begin{définition}\pcmn{粗壮}\end{définition}
\begin{exemple}\pjya{iɕqha tɯrme nɯ kɯ-ɤβrdaβrdoŋ ci ɲɯ-ŋu}\hspace{5pt}\pcmn{这个人很粗壮}\end{exemple}\relationsémantique{同义词}{\lien{ⓔjpumqa}{jpumqa}}\end{entrée}

\begin{entrée}{aβʁum}{}{ⓔaβʁum}\relationsémantique{参考}{\lien{ⓔβʁum}{βʁum}}\end{entrée}

\begin{entrée}{aβzu}{}{ⓔaβzu} 
\classe{vi}  
\grammaire{pass} \sens{1}\paradigme{dir}{tɤ-}\paradigme{dir}{nɯ-}
\begin{définition}\pfra{devenir}\end{définition}
\begin{définition}\pcmn{变成}\end{définition}
\begin{exemple}\pjya{@yangyu ɲo-k-ɤβzu-ci}\hspace{5pt}\pcmn{土豆长大了}\end{exemple}
\begin{exemple}\pjya{tɤɕi kɤ-tɣa to-k-ɤβzu-ci}\hspace{5pt}\pcmn{青稞可以收割了}\end{exemple}\sens{2}\paradigme{dir}{thɯ-}
\begin{définition}\pfra{grandir}\end{définition}
\begin{définition}\pcmn{长成,成熟}\end{définition}
\begin{exemple}\pjya{kɯki tɤ-pɤtso ʁʑɯnɯ chɤ-k-ɤβzu-ci}\hspace{5pt}\pcmn{这个孩子已经长成青年了}\end{exemple}
\begin{exemple}\pjya{@yangyu chɤ-k-ɤβzu-ci}\hspace{5pt}\pcmn{土豆成熟了}\end{exemple}\relationsémantique{参考}{\lien{ⓔβzuⓗ1}{βzu₁}}\end{entrée}

\begin{entrée}{aβzdoʁβzdɯ}{}{ⓔaβzdoʁβzdɯ}\relationsémantique{参考}{\lien{ⓔβzdɯ}{βzdɯ}}\end{entrée}

\begin{entrée}{aβzɯrχsɯm}{}{ⓔaβzɯrχsɯm} 
\classe{vs} 
\begin{définition}\pfra{triangulaire}\end{définition}
\begin{définition}\pcmn{三角形}\end{définition}\étymologie{bzur.gsum}\end{entrée}

\begin{entrée}{aβʑɯrdu}{}{ⓔaβʑɯrdu} 
\classe{vs} \paradigme{dir}{tɤ-}
\begin{définition}\pfra{carré}\end{définition}
\begin{définition}\pcmn{四方形}\end{définition}
\begin{exemple}\pjya{aβzɯrdu, ɲɯ-ɤβʑɯrdu}\hspace{5pt}\pcmn{是四方形的}\end{exemple}
\begin{exemple}\pjya{kɯki rdɤstaʁ ki kɯ-ɤβʑɯrdu ci ɲɯ-ŋu}\hspace{5pt}\pcmn{这块石头是方形的}\end{exemple}\étymologie{bʑi.rdo}\end{entrée}

\begin{entrée}{acu}{}{ⓔacu}\relationsémantique{参考}{\lien{ⓔcu}{cu}}\end{entrée}

\begin{entrée}{acɤrlu}{}{ⓔacɤrlu} 
\classe{vi} \paradigme{dir}{tɤ-}
\begin{définition}\pfra{être mélangé}\end{définition}
\begin{définition}\pcmn{混合}\end{définition}
\begin{exemple}\pjya{ɲɯ-ɤcɤrlu, to-k-ɤcɤrlu-ci}\hspace{5pt}\pcmn{混在一起了}\end{exemple}
\begin{exemple}\pjya{ki tɤjmɤɣ ki ɲɯ-ɤcɤrlu}\hspace{5pt}\pcmn{这些蘑菇混在一起}\end{exemple}
\begin{exemple}\pjya{kɯrɯ kupa tɯrme ra acɤrlu-j ʑo ɕti}\hspace{5pt}\pcmn{我们藏族跟汉族混合在一起}\end{exemple}
\begin{sous-entrée}{sɤcɤrlu}{ⓔacɤrluⓝsɤcɤrlu}
\begin{définition}\pfra{mélanger}\end{définition}
\begin{définition}\pcmn{把……混在一起}\end{définition}
\begin{exemple}\pjya{kɯrɯ skɤt cho kupa skɤt tu-sɤcɤrle-a ʑo tɕe tu-ti-a ɕti}\hspace{5pt}\pcmn{我把藏语和汉语混在一起说}\end{exemple}\relationsémantique{同义词}{\lien{ⓔatʂoʁloʁ}{atʂoʁloʁ}}\end{sous-entrée}

\end{entrée}

\begin{entrée}{achala}{}{ⓔachala} 
\classe{vi} 
\begin{définition}\pfra{capable}\end{définition}
\begin{définition}\pcmn{能干}\end{définition}
\begin{exemple}\pjya{aʑo achala-a tɕe, aj tu-spe-a jɤɣ}\hspace{5pt}\pcmn{我很能干,我会做}\end{exemple}
\begin{sous-entrée}{achɤle}{ⓔachalaⓝachɤle}
\begin{exemple}\pjya{ɯʑo achɤle, nɤʑo tɯ-achɤle}\hspace{5pt}\pcmn{他很能干,你很能干}\end{exemple}
\begin{exemple}\pjya{aʑo achɤle-a tɕe, aj tu-spe-a jɤɣ}\hspace{5pt}\pcmn{我很能干,我会做}\end{exemple}\end{sous-entrée}

\end{entrée}

\begin{entrée}{achale}{}{ⓔachale}\relationsémantique{参考}{\lien{ⓔachala}{achala}}\end{entrée}

\begin{entrée}{achɤt}{}{ⓔachɤt} 
\classe{vs} \paradigme{dir}{tɤ-}
\begin{définition}\pfra{être séparé par}\end{définition}
\begin{définition}\pcmn{相差;相隔}\end{définition}
\begin{exemple}\pjya{tɕiʑo tɕi-pɤrthɤβ kɯmŋu-pɤrme achɤt}\end{exemple}
\begin{exemple}\pjya{tɕiʑo kɯmŋu-pɤrme achɤt-tɕi}\hspace{5pt}\pcmn{我们之间相差五岁}\end{exemple}
\begin{exemple}\pjya{jiʑo ji-kha pɤrthɤβ nɯ tɯ-sŋi tʂu jamar achɤt}\hspace{5pt}\pcmn{我们之间相隔一天的路}\end{exemple}\end{entrée}

\begin{entrée}{achɯcha}{}{ⓔachɯcha} 
\classe{vs} \paradigme{dir}{tɤ-}
\begin{définition}\pfra{qui a du talent}\end{définition}
\begin{définition}\pcmn{有才能}\end{définition}
\begin{exemple}\pjya{to-k-ɤchɯcha-ci}\hspace{5pt}\pcmn{他变得有才能了}\end{exemple}\end{entrée}

\begin{entrée}{achɯrʁu}{}{ⓔachɯrʁu} 
\classe{vs} \paradigme{dir}{thɯ-}\paradigme{dir}{tɤ-}\paradigme{dir}{tɤ-}
\begin{définition}\pfra{être froissé}\end{définition}
\begin{définition}\pcmn{皱着}\end{définition}
\begin{définition}\pfra{froisser}\end{définition}
\begin{définition}\pcmn{皱起}\end{définition}
\begin{exemple}\pjya{nɤ-ŋga to-k-ɤchɯrʁu-ci}\hspace{5pt}\pcmn{你的衣服皱了}\end{exemple}
\begin{exemple}\pjya{ɯ-rŋa to-sɤchɯrʁu}\hspace{5pt}\pcmn{他皱了眉毛(脸)}\end{exemple}
\begin{sous-entrée}{sɤchɯrʁu}{ⓔachɯrʁuⓝsɤchɯrʁu} 
\classe{vt} \end{sous-entrée}

\end{entrée}

\begin{entrée}{aci}{}{ⓔaci} 
\classe{vi} \paradigme{dir}{nɯ-}\paradigme{dir}{nɯ-}
\begin{définition}\pfra{être mouillé}\end{définition}
\begin{définition}\pcmn{湿}\end{définition}
\begin{définition}\pfra{mouiller}\end{définition}
\begin{définition}\pcmn{弄湿}\end{définition}
\begin{exemple}\pjya{ɲo-k-ɤci-ci}\hspace{5pt}\pcmn{湿了}\end{exemple}
\begin{exemple}\pjya{ki tɯ-mɯ kɯ pjɤ-χtɕi tɕe, ɲo-k-ɤci-ci}\hspace{5pt}\pcmn{被雨淋湿了}\end{exemple}
\begin{exemple}\pjya{ɯ-ŋga ɲɤ-sɤci}\hspace{5pt}\pcmn{他把衣服弄湿了}\end{exemple}
\begin{exemple}\pjya{tɯ-kɤrme mɤ-kɯ-sɤci}\hspace{5pt}\pcmn{浴帽}\end{exemple}
\begin{sous-entrée}{sɤci}{ⓔaciⓝsɤci} 
\classe{vt}  
\grammaire{caus} \end{sous-entrée}

\end{entrée}

\begin{entrée}{acilaj}{}{ⓔacilaj} 
\classe{vi} \paradigme{dir}{nɯ-}
\begin{définition}\pfra{être humide}\end{définition}
\begin{définition}\pcmn{湿漉漉}\end{définition}
\begin{exemple}\pjya{kɯki kɤ-ɕkho ɲɯ-ra ma ɲɯ-ɤcilaj}\hspace{5pt}\pcmn{这个东西很湿,要晒一下}\end{exemple}
\begin{exemple}\pjya{kɯki tɯ-ŋga ki ɲɯ-ɤcilaj tɕe, chɯ́-wɣ-ɕkho ɲɯ-ntshi}\hspace{5pt}\pcmn{这件衣服很湿,只好晒一下}\end{exemple}
\begin{exemple}\pjya{a-βri acilaj}\hspace{5pt}\pcmn{我身上很湿}\end{exemple}
\begin{exemple}\pjya{a-jaʁ ɲɯ-ɤcilaj}\hspace{5pt}\pcmn{我的手很湿}\end{exemple}\end{entrée}

\begin{entrée}{acɯfkri}{}{ⓔacɯfkri} 
\classe{vs} 
\begin{définition}\pfra{être sale, humide et en désordre}\end{définition}
\begin{définition}\pcmn{又脏又湿又乱}\end{définition}\end{entrée}

\begin{entrée}{aɕɤl}{}{ⓔaɕɤl} 
\classe{vi} \paradigme{dir}{nɯ-}
\begin{définition}\pfra{souffrir de la cataracte}\end{définition}
\begin{définition}\pcmn{患有白内障【白眼病】}\end{définition}
\begin{exemple}\pjya{ɯ-mɲaʁ ɲo-k-ɤɕɤl-ci}\hspace{5pt}\pcmn{他得了白内障}\end{exemple}
\begin{exemple}\pjya{ɲo-k-ɤɕɤl-ci tɕe mɯ́j-mtɤm}\hspace{5pt}\pcmn{得了白内障,他看不见了}\end{exemple}\end{entrée}

\begin{entrée}{aɕɤrɣi}{}{ⓔaɕɤrɣi} 
\classe{vs} \paradigme{dir}{tɤ-}
\begin{définition}\pfra{être rapide}\end{définition}
\begin{définition}\pcmn{快,迅速}\end{définition}
\begin{définition}\pfra{faire rapidement}\end{définition}
\begin{définition}\pcmn{做得快}\end{définition}
\begin{exemple}\pjya{ɯ-rju ɲɯ-ɤɕɤrɣi}\hspace{5pt}\pcmn{他话说得快}\end{exemple}
\begin{exemple}\pjya{ɯ-mɢla ɲɯ-ɤɕɤrɣi}\hspace{5pt}\pcmn{他步子走得快}\end{exemple}
\begin{exemple}\pjya{ɯʑo kɯ tɯ-rju kɤ-ti ɲɯ-sɤɕɤrɣi}\hspace{5pt}\pcmn{他说话说得快}\end{exemple}
\begin{sous-entrée}{sɤɕɤrɣi}{ⓔaɕɤrɣiⓝsɤɕɤrɣi} 
\classe{vt} \end{sous-entrée}

\end{entrée}

\begin{entrée}{aɕɣa}{}{ⓔaɕɣa} 
\classe{vi} 
\begin{définition}\pfra{être du même âge}\end{définition}
\begin{définition}\pcmn{同年}\end{définition}
\begin{exemple}\pjya{tɕiʑo kɯ-ɤɕɣa ŋu-tɕi}\hspace{5pt}\pcmn{我们俩同岁}\end{exemple}
\begin{exemple}\pjya{ɲɯ-tɯ-ɤɕɣa-ndʑi}\hspace{5pt}\pcmn{你们俩同岁}\end{exemple}
\begin{exemple}\pjya{nɤj nɤ-mu cho aʑo ni aɕɣa-tɕi}\hspace{5pt}\pcmn{我跟你母亲同岁}\end{exemple}\relationsémantique{参考}{\lien{ⓔtɯ-ɕɣa}{tɯ-ɕɣa}}\end{entrée}

\begin{entrée}{aɕi}{₁}{ⓔaɕiⓗ1} 
\classe{intj} 
\begin{définition}\pfra{interjection qui exprime que l'on revient sur ce que l'on vient de dire}\end{définition}
\begin{définition}\pcmn{对刚才说的话表示反悔}\end{définition}
\begin{exemple}\pjya{aɕi, nɯ ma-tɤ-tɯ-ste}\hspace{5pt}\pcmn{算了,你不要那么做}\end{exemple}
\begin{exemple}\pjya{aɕi ma-jɤ-tɯ-ɕe tɕe, kɤ-nɯ-rɤʑi!}\hspace{5pt}\pcmn{算了,不要走了,留在这里吧}\end{exemple}
\begin{exemple}\pjya{aɕi, a-mɤ-tɤ-tɯ-qhe je}\hspace{5pt}\pcmn{(我收回刚才说的话),你不要放在心上}\end{exemple}\end{entrée}

\begin{entrée}{aɕi}{₂}{ⓔaɕiⓗ2} 
\classe{vs}  
\grammaire{caus} \paradigme{dir}{pɯ-}
\begin{définition}\pfra{être mélangé dans}\end{définition}
\begin{définition}\pcmn{掺在一起;和在一起}\end{définition}
\begin{définition}\pfra{ajouter dans}\end{définition}
\begin{définition}\pcmn{加进}\end{définition}
\begin{exemple}\pjya{ma-pɯ-tɯ-nɯci ma nɤ-pa ɯ-se aɕi}\hspace{5pt}\pcmn{你别喝(湖里的水),里面掺着你父亲的血}\end{exemple}\relationsémantique{同义词}{\lien{ⓔacu}{acu}}
\begin{sous-entrée}{sɤɕi}{ⓔaɕiⓗ2ⓝsɤɕi} 
\classe{vt} \end{sous-entrée}

\relationsémantique{同义词}{\lien{ⓔcu}{cu}}\end{entrée}

\begin{entrée}{aɕkala}{}{ⓔaɕkala} 
\classe{vs}  
\grammaire{denom} \paradigme{dir}{tɤ-}
\begin{définition}\pfra{être boiteux}\end{définition}
\begin{définition}\pcmn{跛脚}\end{définition}
\begin{exemple}\pjya{ɯʑo ɲɯ-ɤɕkala}\hspace{5pt}\pcmn{他跛脚}\end{exemple}
\begin{exemple}\pjya{nɯ ɕɯŋgɯ mɯ-pɯ-aɕkala ri, tham tɕa to-k-ɤɕkala-ci}\hspace{5pt}\pcmn{他以前不跛脚,现在就跛脚了}\end{exemple}\relationsémantique{同义词}{\lien{ⓔaʑɤwu}{aʑɤwu}}\relationsémantique{参考}{\lien{ⓔɕkala}{ɕkala}}\end{entrée}

\begin{entrée}{aɕoʁri}{}{ⓔaɕoʁri} 
\classe{vi} \paradigme{dir}{tɤ-}
\begin{définition}\pfra{aller et venir}\end{définition}
\begin{définition}\pcmn{来回走动穿梭}\end{définition}\end{entrée}

\begin{entrée}{aɕoχɕi}{}{ⓔaɕoχɕi} 
\classe{vi} \paradigme{dir}{tɤ-}
\begin{définition}\pfra{inspirer et expirer}\end{définition}
\begin{définition}\pcmn{吸气和呼气}\end{définition}
\begin{définition}\pfra{inspirer et expirer}\end{définition}
\begin{définition}\pcmn{吸气和呼气}\end{définition}
\begin{exemple}\pjya{ɯ-sŋɯro tɤ-sɤɕoχɕi}\hspace{5pt}\pcmn{他吸了气又呼了气}\end{exemple}
\begin{exemple}\pjya{ɯ-sŋɯro ɲɯ-ɤsɯ-sɤɕoχɕi}\hspace{5pt}\pcmn{他在呼吸}\end{exemple}
\begin{sous-entrée}{sɤɕoχɕi}{ⓔaɕoχɕiⓝsɤɕoχɕi} 
\classe{vt} \end{sous-entrée}

\end{entrée}

\begin{entrée}{aɕpala}{}{ⓔaɕpala} 
\classe{vi} 
\begin{définition}\pfra{ayant des mouvement alertes}\end{définition}
\begin{définition}\pcmn{动作灵活(小伙子)}\end{définition}
\begin{exemple}\pjya{kɯ-ɤɕpala ci ɲɯ-ŋu}\hspace{5pt}\pcmn{他动作很灵活}\end{exemple}
\begin{exemple}\pjya{jiɕqha tɯrme nɯ mɤ-kɯ-ɤɕpala ci rʁoʁrʁoʁ ɲɯ-ɕti}\hspace{5pt}\pcmn{那个人动作不灵活}\end{exemple}\relationsémantique{同义词}{\lien{ⓔaphala}{aphala}}\end{entrée}

\begin{entrée}{aɕprɯm}{}{ⓔaɕprɯm} 
\classe{vi} \paradigme{dir}{kɤ-}\paradigme{dir}{kɤ-}\paradigme{dir}{kɤ-}
\begin{définition}\pfra{être mal raccommodé (habits, chaussure)}\end{définition}
\begin{définition}\pcmn{补得不平整,皱着的,不舒展(衣服、鞋子)}\end{définition}
\begin{définition}\pfra{être mal raccommodé (habits, chaussure)}\end{définition}
\begin{définition}\pcmn{补得不好;皱着的(衣服、鞋子)}\end{définition}
\begin{définition}\pfra{coudre à l'à peu près}\end{définition}
\begin{définition}\pcmn{将就缝}\end{définition}
\begin{exemple}\pjya{ko-k-ɤɕprɯm-ci}\hspace{5pt}\pcmn{补得不好}\end{exemple}
\begin{exemple}\pjya{ɯ-ɕphɤt mɯ-ko-ɣɤβdi tɕe, ɲɯ-ɤɕprɯm}\hspace{5pt}\pcmn{补丁没有补好,所以不平整}\end{exemple}
\begin{exemple}\pjya{kɤ-ɣɤβdi mɯ́j-khɯ tɕe, aɕprɯm}\hspace{5pt}\pcmn{(衣服)没有能修补,所以不平整}\end{exemple}
\begin{exemple}\pjya{ko-k-ɤɕprɯmtsɯ-ci}\hspace{5pt}\pcmn{补得不好}\end{exemple}
\begin{exemple}\pjya{tɤ-ɕphɤt mɯ-ko-βdi tɕe ɲɯ-ɤɕprɯmtsɯ}\hspace{5pt}\pcmn{补丁没有补好,所以不平整}\end{exemple}
\begin{exemple}\pjya{pɤjkhu mɯ́j-mna tɕe ɲɯ-ɤɕprɯmtsɯ}\hspace{5pt}\pcmn{(伤口)还没好,没有愈合}\end{exemple}
\begin{exemple}\pjya{ɯ-ŋga ka-sɤɕprɯm}\hspace{5pt}\pcmn{他把衣服将就缝了一下}\end{exemple}
\begin{sous-entrée}{aɕprɯmtsɯ}{ⓔaɕprɯmⓝaɕprɯmtsɯ} 
\classe{vi} \end{sous-entrée}

\begin{sous-entrée}{sɤɕprɯm}{ⓔaɕprɯmⓝsɤɕprɯm} 
\classe{vt} \end{sous-entrée}

\end{entrée}

\begin{entrée}{aɕprɯmtsɯ}{}{ⓔaɕprɯmtsɯ}\relationsémantique{参考}{\lien{ⓔaɕprɯm}{aɕprɯm}}\end{entrée}

\begin{entrée}{aɕpɯɕpa}{}{ⓔaɕpɯɕpa} 
\classe{vs} \paradigme{dir}{nɯ-}\paradigme{dir}{kɤ-}\paradigme{dir}{nɯ-}
\begin{définition}\pfra{être plat}\end{définition}
\begin{définition}\pcmn{瘪,扁}\end{définition}
\begin{définition}\pfra{aplatir}\end{définition}
\begin{définition}\pcmn{弄扁}\end{définition}
\begin{exemple}\pjya{ɲo-k-ɤɕpɯɕpa-ci, kɤ-aɕpɯɕpa}\hspace{5pt}\pcmn{已经压扁了}\end{exemple}
\begin{exemple}\pjya{ɯ-xtu ɲɤ-k-ɤɕpɯɕpa-ci}\hspace{5pt}\pcmn{他肚子扁了}\end{exemple}
\begin{exemple}\pjya{nɯ-sɤɕpɯɕpa-t-a}\hspace{5pt}\pcmn{我把它弄扁了}\end{exemple}
\begin{sous-entrée}{sɤɕpɯɕpa}{ⓔaɕpɯɕpaⓝsɤɕpɯɕpa} 
\classe{vt}  
\grammaire{caus} \end{sous-entrée}

\end{entrée}

\begin{entrée}{aɕqhe}{}{ⓔaɕqhe} 
\classe{vi} \paradigme{dir}{tɤ-}
\begin{définition}\pfra{tousser}\end{définition}
\begin{définition}\pcmn{咳嗽}\end{définition}
\begin{exemple}\pjya{ɲɯ-tɯ-ɤɕqhe, ɯʑo ɲɯ-ɤɕqhe}\hspace{5pt}\pcmn{你咳嗽,他咳嗽}\end{exemple}
\begin{exemple}\pjya{to-k-ɤɕqhe-ci}\hspace{5pt}\pcmn{他咳嗽了}\end{exemple}
\begin{exemple}\pjya{aʑo ku-oɕqhe-a}\hspace{5pt}\pcmn{我正在咳嗽}\end{exemple}
\begin{exemple}\pjya{aʑo sɲikuku ʑo tu-oɕqhe-a ŋu}\hspace{5pt}\pcmn{我天天咳嗽}\end{exemple}
\begin{exemple}\pjya{ɯ-ɲɯ-tɯ-ɤɕqhe ?}\hspace{5pt}\pcmn{你咳嗽吗?}\end{exemple}
\begin{exemple}\pjya{a-mɤ-tu-sɤ-ɕqhɯ-ɕqhe, smɤn ci tɤ-ndza-t-a}\hspace{5pt}\pcmn{为了不咳嗽,我吃了药}\end{exemple}\relationsémantique{参考}{\lien{ⓔtɤ-ɕqhe}{tɤ-ɕqhe}}
\begin{sous-entrée}{sɤɕqhe}{ⓔaɕqheⓝsɤɕqhe} 
\classe{vt}  
\grammaire{caus} 
\begin{définition}\pfra{faire tousser}\end{définition}
\begin{définition}\pcmn{令人咳嗽}\end{définition}
\begin{exemple}\pjya{aʑo a-rtshɤz ɯ-ŋgɯ thɯ-ari tɕe tɤ́-wɣ-sɤɕqhe-a}\hspace{5pt}\pcmn{(茶)进到肺里了,让我咳了起来}\end{exemple}
\begin{exemple}\pjya{tɤ-khɯ ɯ-di pjɯ-tɯ-mtshɤm tɕe, tú-wɣ-sɤɕqhe-a ŋu}\hspace{5pt}\pcmn{我一闻到烟味就会咳嗽}\end{exemple}\end{sous-entrée}

\end{entrée}

\begin{entrée}{aɕqhlu}{}{ⓔaɕqhlu} 
\classe{vs} 
\begin{définition}\pfra{concave}\end{définition}
\begin{définition}\pcmn{凹}\end{définition}
\begin{exemple}\pjya{pjɤ-k-ɤɕqhlu-ci}\hspace{5pt}\pcmn{凹进去了}\end{exemple}\relationsémantique{同义词}{\lien{ⓔaχchowolu}{aχchowolu}}\relationsémantique{同义词}{\lien{ⓔaʁloʁlu}{aʁloʁlu}}\relationsémantique{同义词}{\lien{ⓔaqhowolu}{aqhowolu}}\relationsémantique{同义词}{\lien{ⓔasqhlu}{asqhlu}}\relationsémantique{同义词}{\lien{ⓔarɴɢlɯm}{arɴɢlɯm}}\end{entrée}

\begin{entrée}{aɕquwa}{}{ⓔaɕquwa} 
\classe{vs}  
\grammaire{denom} \paradigme{dir}{tɤ-}
\begin{définition}\pfra{être aveugle}\end{définition}
\begin{définition}\pcmn{瞎}\end{définition}
\begin{exemple}\pjya{ɯ-mɲaʁ ntsi aɕquwa (ɯ-mɲaʁ ntsi mɤ-pe)}\hspace{5pt}\pcmn{他一只眼睛瞎了}\end{exemple}\relationsémantique{参考}{\lien{ⓔɕquwa}{ɕquwa}}\end{entrée}

\begin{entrée}{aɕtɯɕte}{}{ⓔaɕtɯɕte}\relationsémantique{参考}{\lien{ⓔɕte}{ɕte}}\end{entrée}

\begin{entrée}{adaʁlu}{}{ⓔadaʁlu} 
\classe{vs} \paradigme{dir}{thɯ-}
\begin{définition}\pfra{avec du noir et du blanc (mélangés)}\end{définition}
\begin{définition}\pcmn{黑色和白色混在一起}\end{définition}
\begin{exemple}\pjya{chɤ-k-ɤdaʁlu-ci}\end{exemple}\end{entrée}

\begin{entrée}{adrɤt}{}{ⓔadrɤt} 
\classe{vs} \paradigme{dir}{pɯ-}\paradigme{dir}{thɯ-}\paradigme{dir}{lɤ-}
\begin{définition}\pfra{en désordre}\end{définition}
\begin{définition}\pcmn{凌乱}\end{définition}
\begin{définition}\pfra{mettre le désordre}\end{définition}
\begin{définition}\pcmn{乱放}\end{définition}
\begin{exemple}\pjya{a-mɤ-pɯ-ɤdrɤt tɕe tɤ-rɤwum}\hspace{5pt}\pcmn{(东西)不要这么乱,收拾一下}\end{exemple}
\begin{exemple}\pjya{tɤ-fkɯm ɯ-ŋgɯ a-pɯ-ɤrku ma ma-lɤ-tɯ-sɤdrɤt}\hspace{5pt}\pcmn{东西在袋子里放着,不要拿出来到处乱放}\end{exemple}\relationsémantique{同义词}{\lien{ⓔaphɤlɤjɤt}{aphɤlɤjɤt}}\relationsémantique{参考}{\lien{ⓔnɤqadrɤt}{nɤqadrɤt}}
\begin{sous-entrée}{sɤdrɤt}{ⓔadrɤtⓝsɤdrɤt} 
\classe{vt} \end{sous-entrée}

\end{entrée}

\begin{entrée}{adʑɯgli}{}{ⓔadʑɯgli} 
\classe{vi} \paradigme{dir}{nɯ-}
\begin{définition}\pfra{craquer les uns les autres (os)}\end{définition}
\begin{définition}\pcmn{(骨头)互相摩擦发出声音}\end{définition}
\begin{exemple}\pjya{a-mke ɯ-ɕɤrɯ ɲɯ-ɤndʑɯgli ɲɯ-ŋu tɕe glinɤgli ʑo ɲɯ-ti}\hspace{5pt}\pcmn{我脖子的骨头互相摩擦发出咯咯声}\end{exemple}\relationsémantique{参考}{\lien{ⓔglinɤgli}{glinɤgli}}\end{entrée}

\begin{entrée}{afɕu}{}{ⓔafɕu} 
\classe{vi} \sens{1}\paradigme{dir}{nɯ-}
\begin{définition}\pfra{se refroidir (objet)}\end{définition}
\begin{définition}\pcmn{冷却(东西)}\end{définition}
\begin{exemple}\pjya{tʂha nɯ-afɕu kóʁmɯz kú-wɣ-tshi ra}\hspace{5pt}\pcmn{茶要凉一点喝}\end{exemple}
\begin{exemple}\pjya{ki ɲɯ-sɤɕke tɕe, a-nɯ-ɤfɕu ɲɯ-ntshi}\hspace{5pt}\pcmn{很烫,要先冷却一下}\end{exemple}
\begin{exemple}\pjya{ki kɤ-tshi ɲɯ-jɤɣ ma ɲɤ-k-ɤfɕu-ci}\hspace{5pt}\pcmn{这个已经凉了,可以喝}\end{exemple}
\begin{exemple}\pjya{mɯ-ɲɤ-sɤɕke, ɲo-k-ɤfɕu-ci}\hspace{5pt}\pcmn{已经不烫了,冷却了}\end{exemple}
\begin{exemple}\pjya{a-nɯ-ɤfɕu ku-nɤjam-a}\hspace{5pt}\pcmn{我在等它冷却一下}\end{exemple}
\begin{exemple}\pjya{nɯkɤcu jiɕqha tɤ-ala pɯ-ŋu ri, tham ɲɤ-k-ɤfɕu-ci}\hspace{5pt}\pcmn{刚才那个在沸腾,现在冷了}\end{exemple}\sens{2}\paradigme{dir}{tɤ-}
\begin{définition}\pfra{se reposer}\end{définition}
\begin{définition}\pcmn{休息,放松,轻松}\end{définition}
\begin{exemple}\pjya{a-tɤɣɲat tɤ-afɕu}\hspace{5pt}\pcmn{我轻松了(休息好了)}\end{exemple}
\begin{exemple}\pjya{tɤ-afɕu-a}\hspace{5pt}\pcmn{我轻松了(休息好了)}\end{exemple}
\begin{exemple}\pjya{nɯ kɯnɤ ʑo ɯ-ro mɯ-pjɤ-k-ɤfɕu-ci}\hspace{5pt}\pcmn{这样他都不解恨}\end{exemple}
\begin{sous-entrée}{sɤfɕu}{ⓔafɕuⓢ2ⓝsɤfɕu} 
\classe{vt}  
\grammaire{caus} \sens{1}\paradigme{dir}{nɯ-}
\begin{définition}\pfra{faire refroidir}\end{définition}
\begin{définition}\pcmn{使冷却}\end{définition}
\begin{exemple}\pjya{tʂha nɯ-sɤfɕe}\hspace{5pt}\pcmn{让茶冷却一下}\end{exemple}\end{sous-entrée}

\sens{2}\paradigme{dir}{tɤ-}\paradigme{dir}{tɤ-}
\begin{définition}\pfra{se reposer}\end{définition}
\begin{définition}\pcmn{休息,放松,轻松}\end{définition}
\begin{définition}\pfra{se reposer un peu}\end{définition}
\begin{définition}\pcmn{歇一会}\end{définition}
\begin{exemple}\pjya{nɤ-tɤɣɲat tɤ-sɤfɕe}\hspace{5pt}\pcmn{你放松一下}\end{exemple}
\begin{exemple}\pjya{nɤ-mgɯr ci tɤ-sɤfɕe}\hspace{5pt}\pcmn{你把背部放松一下(靠一下)}\end{exemple}
\begin{exemple}\pjya{pjɤ-tɯ-ɴqa tɕe, tɤ-ʑɣɤsɤfɕu}\hspace{5pt}\pcmn{你累了,休息一下}\end{exemple}
\begin{sous-entrée}{ʑɣɤsɤfɕu}{ⓔafɕuⓢ2ⓝʑɣɤsɤfɕu} 
\classe{vi}  
\grammaire{caus}
\grammaire{refl} \end{sous-entrée}

\end{entrée}

\begin{entrée}{afɕɤra}{}{ⓔafɕɤra} 
\classe{vs} \paradigme{dir}{nɯ-}
\begin{définition}\pfra{être connu publiquement}\end{définition}
\begin{définition}\pcmn{公开,传开}\end{définition}
\begin{exemple}\pjya{ki tɯ-tɕha ki ɲɤ-k-ɤfɕɤra}\hspace{5pt}\pcmn{这个消息已经传开了}\end{exemple}
\begin{exemple}\pjya{mɤ-kɯ-ɤfɕɤra nɯ ɕɯ tso?}\hspace{5pt}\pcmn{如果不公开的话,谁会知道?}\end{exemple}\relationsémantique{参考}{\lien{ⓔsɤfɕɤra}{sɤfɕɤra}}\end{entrée}

\begin{entrée}{afɕɤt}{}{ⓔafɕɤt}\relationsémantique{参考}{\lien{ⓔfɕɤtⓗ1}{fɕɤt₁}}\end{entrée}

\begin{entrée}{afkrɤχsɤl}{}{ⓔafkrɤχsɤl} 
\classe{vi} \paradigme{dir}{tɤ-}
\begin{définition}\pfra{être clair (voir)}\end{définition}
\begin{définition}\pcmn{清晰;明显}\end{définition}
\begin{exemple}\pjya{kɯ-ɲaʁ kɯ-wɣrum ɲɯ-ɤfkrɤχsɤl}\hspace{5pt}\pcmn{黑色和红色的区别很分明}\end{exemple}
\begin{exemple}\pjya{to-k-ɤfkrɤχsɤl-ci}\hspace{5pt}\pcmn{变清晰了}\end{exemple}\étymologie{bkra.gsal}\end{entrée}

\begin{entrée}{afsuja}{}{ⓔafsuja} 
\classe{vs} 
\begin{définition}\pfra{être de même taille}\end{définition}
\begin{définition}\pcmn{一样大;一样长;平等}\end{définition}
\begin{exemple}\pjya{tɕiʑo afsuja-tɕi / tɕi-tɯ-mbro afsuja}\hspace{5pt}\pcmn{我们长得一样高}\end{exemple}\relationsémantique{参考}{\lien{ⓔsɤfsuja}{sɤfsuja}}\end{entrée}

\begin{entrée}{afsujɯja}{}{ⓔafsujɯja} 
\classe{vi} 
\begin{définition}\pfra{être de même longueur}\end{définition}
\begin{définition}\pcmn{长短一致}\end{définition}\relationsémantique{参考}{\lien{ⓔsɤfsuja}{sɤfsuja}}\relationsémantique{参考}{\lien{ⓔafsɯfsu}{afsɯfsu}}\end{entrée}

\begin{entrée}{afsoʁŋgi}{}{ⓔafsoʁŋgi} 
\classe{vs} \paradigme{dir}{tɤ-}
\begin{définition}\pfra{être clair}\end{définition}
\begin{définition}\pcmn{颜色浅;色彩亮丽;亮堂}\end{définition}
\begin{exemple}\pjya{afsoʁŋgi}\hspace{5pt}\pcmn{很亮}\end{exemple}
\begin{exemple}\pjya{to-k-ɤfsoʁŋgi-ci}\hspace{5pt}\pcmn{变亮了}\end{exemple}
\begin{exemple}\pjya{kɯki ɯ-mdoʁ kɯ-ɤfsoʁŋgi ci ŋu}\hspace{5pt}\pcmn{颜色很浅}\end{exemple}\end{entrée}

\begin{entrée}{afsɯfsu}{}{ⓔafsɯfsu} 
\classe{vs} 
\begin{définition}\pfra{être égal}\end{définition}
\begin{définition}\pcmn{长短一样}\end{définition}
\begin{exemple}\pjya{nɤ-ndʑu ni ɲɯ-ɤfsɯfsu-ndʑi}\hspace{5pt}\pcmn{你的筷子的长短一样}\end{exemple}\relationsémantique{参考}{\lien{ⓔɯ-fsu}{ɯ-fsu}}\relationsémantique{参考}{\lien{ⓔafsujɯja}{afsujɯja}}\end{entrée}

\begin{entrée}{aftɕaka}{}{ⓔaftɕaka} 
\classe{vs} 
\begin{définition}\pfra{avoir les ornements et vêtements au complet}\end{définition}
\begin{définition}\pcmn{服饰齐全}\end{définition}\relationsémantique{参考}{\lien{ⓔftɕaka}{ftɕaka}}\end{entrée}

\begin{entrée}{aɣɤmphɯmphri/\variante{amphɯmphri}}{}{ⓔaɣɤmphɯmphri} 
\classe{vi} \paradigme{dir}{pɯ-}
\begin{définition}\pfra{être l'un après l'autre}\end{définition}
\begin{définition}\pcmn{一个接着一个}\end{définition}
\begin{exemple}\pjya{kɤ-nɤma ɲɯ-dɤn tɕe ɲɯ-ɤmphɯmphri ɕti}\hspace{5pt}\pcmn{工作很多,连续不断}\end{exemple}
\begin{exemple}\pjya{kɤtsa ra nɯ-kɯ-mŋɤm pjɤ-dɤn tɕe pjɤ-k-ɤɣɤmphɯmphri-ci}\hspace{5pt}\pcmn{他们这一家人过去很多生病的,一个接着一个}\end{exemple}
\begin{exemple}\pjya{iʑɤɣ ji-ma ra ɲɯ-ɤɣɤmphɯmphri ɕti}\hspace{5pt}\pcmn{我们的工作连续不断,一个接着一个}\end{exemple}\end{entrée}

\begin{entrée}{aɣɤŋɯŋoʁ}{}{ⓔaɣɤŋɯŋoʁ}\relationsémantique{参考}{\lien{ⓔɣɤŋoʁ}{ɣɤŋoʁ}}\end{entrée}

\begin{entrée}{aɣɤrlɯrlaʁ}{}{ⓔaɣɤrlɯrlaʁ}\relationsémantique{参考}{\lien{ⓔɣɤrlaʁ}{ɣɤrlaʁ}}\end{entrée}

\begin{entrée}{aɣɤtɯɣ}{}{ⓔaɣɤtɯɣ}\relationsémantique{参考}{\lien{ⓔɣɤtɯɣ}{ɣɤtɯɣ}}\end{entrée}

\begin{entrée}{aɣɤzdɯzda}{}{ⓔaɣɤzdɯzda}\relationsémantique{参考}{\lien{ⓔɣɤzda}{ɣɤzda}}\end{entrée}

\begin{entrée}{aɣndɯɣnda}{}{ⓔaɣndɯɣnda} 
\classe{vs} 
\begin{définition}\pfra{devenu ferme après avoir été piétinée (terre)}\end{définition}
\begin{définition}\pcmn{踩紧的}\end{définition}
\begin{exemple}\pjya{tɯ-ji lo-ɕlu-nɯ ri, ɯ-taʁ to-ŋke-nɯ tɕe pjɤ-k-ɤɣndɯɣnda-ci}\hspace{5pt}\pcmn{他们虽然种了地,但是因为踩在上面把地踩紧了}\end{exemple}\end{entrée}

\begin{entrée}{aɣrɤɣrum}{}{ⓔaɣrɤɣrum} 
\classe{vs} 
\begin{définition}\pfra{être blanchâtre}\end{définition}
\begin{définition}\pcmn{微白的}\end{définition}\relationsémantique{参考}{\lien{ⓔwɣrum}{wɣrum}}\end{entrée}

\begin{entrée}{aɣro}{}{ⓔaɣro} 
\classe{vi}  
\grammaire{appl} \paradigme{dir}{nɯ-}\paradigme{dir}{tɤ-}\paradigme{dir}{pɯ-}\paradigme{dir}{nɯ-}\paradigme{dir}{tɤ-}\paradigme{dir}{pɯ-}
\begin{définition}\pfra{jouer}\end{définition}
\begin{définition}\pcmn{玩}\end{définition}
\begin{exemple}\pjya{pɯ-aɣro-tɕi}\hspace{5pt}\pcmn{(过去)我们在玩}\end{exemple}
\begin{exemple}\pjya{a-xtɤɣ nɯ cho tɤ-anɯɣro-tɕi}\hspace{5pt}\pcmn{我跟哥哥玩了一下}\end{exemple}
\begin{exemple}\pjya{@lanqiu pɯ-asɯ-lɤt-tɕi pɯ-anɯɣro-tɕi}\hspace{5pt}\pcmn{(过去)我们打篮球}\end{exemple}
\begin{sous-entrée}{nɤɣro}{ⓔaɣroⓝnɤɣro} 
\classe{vt} \end{sous-entrée}

\sens{1}
\begin{définition}\pfra{jouer à}\end{définition}
\begin{définition}\pcmn{玩(某种游戏)}\end{définition}
\begin{exemple}\pjya{@lanqiu nɯ-nɤɣro-t-a}\hspace{5pt}\pcmn{我打了篮球}\end{exemple}
\begin{exemple}\pjya{ta-ma tɤ-nɤme @lanqiu ma-nɯ-tɯ-nɤɣrɤm}\hspace{5pt}\pcmn{你工作啦,不要打篮球}\end{exemple}
\begin{exemple}\pjya{aki aʑo @lanqiu pɯ-az-nɤɣro-a}\hspace{5pt}\pcmn{(过去时)我打篮球}\end{exemple}\sens{2}
\begin{définition}\pfra{taquiner}\end{définition}
\begin{définition}\pcmn{逗弄}\end{définition}\end{entrée}

\begin{entrée}{aɣɯβlu}{}{ⓔaɣɯβlu} 
\classe{vs} 
\begin{définition}\pfra{rusé}\end{définition}
\begin{définition}\pcmn{足智多谋,狡猾}\end{définition}\relationsémantique{参考}{\lien{ⓔɯ-βlu}{ɯ-βlu}}\end{entrée}

\begin{entrée}{aɣɯci}{}{ⓔaɣɯci} 
\classe{vs}  
\grammaire{denom} 
\begin{définition}\pfra{qui a du jus}\end{définition}
\begin{définition}\pcmn{有汁}\end{définition}\relationsémantique{参考}{\lien{ⓔtɯ-ci}{tɯ-ci}}\end{entrée}

\begin{entrée}{aɣɯɕa}{}{ⓔaɣɯɕa} 
\classe{vs}  
\grammaire{denom} \paradigme{dir}{thɯ-}
\begin{définition}\pfra{qui a beaucoup de chair, charnu}\end{définition}
\begin{définition}\pcmn{肉多}\end{définition}
\begin{exemple}\pjya{ki paʁ ki ɲɯ-ɤɣɯɕa}\hspace{5pt}\pcmn{这只猪肉很多}\end{exemple}
\begin{exemple}\pjya{nɤʑo nɤ-tɯ-ɤɣɯɕa nɯ!}\hspace{5pt}\pcmn{你长得很胖啊}\end{exemple}\end{entrée}

\begin{entrée}{aɣɯɕɤrɯ}{}{ⓔaɣɯɕɤrɯ} 
\classe{vs}  
\grammaire{denom} 
\begin{définition}\pfra{qui a la peau sur les os}\end{définition}
\begin{définition}\pcmn{瘦}\end{définition}\relationsémantique{同义词}{\lien{ⓔnɯɲɤmkhe}{nɯɲɤmkhe}}\relationsémantique{参考}{\lien{ⓔɕɤrɯ}{ɕɤrɯ}}\end{entrée}

\begin{entrée}{aɣɯɕkat}{}{ⓔaɣɯɕkat}\relationsémantique{参考}{\lien{ⓔɣɯɕkat}{ɣɯɕkat}}\end{entrée}

\begin{entrée}{aɣɯɕnɯɕna}{}{ⓔaɣɯɕnɯɕna} 
\classe{vs}  
\grammaire{denom} 
\begin{définition}\pfra{qui a le sens de l'odorat}\end{définition}
\begin{définition}\pcmn{有嗅觉}\end{définition}
\begin{exemple}\pjya{kɯ-ɤɣɯɕnɯɕna ʁɟa ɕti tɕe, tɤ-di mtshɤm-nɯ}\hspace{5pt}\pcmn{他们都有嗅觉,也会闻到臭味}\end{exemple}\relationsémantique{参考}{\lien{ⓔtɯ-ɕna}{tɯ-ɕna}}\end{entrée}

\begin{entrée}{aɣɯɕnɯɕnaβ}{}{ⓔaɣɯɕnɯɕnaβ} 
\classe{vs}  
\grammaire{denom} 
\begin{définition}\pfra{être visqueux}\end{définition}
\begin{définition}\pcmn{黏稠,像胶水}\end{définition}\relationsémantique{参考}{\lien{ⓔtɯ-ɕnaβ}{tɯ-ɕnaβ}}\end{entrée}

\begin{entrée}{aɣɯdɯχɯn}{}{ⓔaɣɯdɯχɯn} 
\classe{vi}  
\grammaire{denom} \paradigme{dir}{tɤ-}
\begin{définition}\pfra{avoir une bonne odeur}\end{définition}
\begin{définition}\pcmn{香(气味)}\end{définition}
\begin{exemple}\pjya{jiɕqha tɯsqar nɯ ɲɯ-ɤɣɯdɯχɯn}\hspace{5pt}\pcmn{这个糌粑气味很香}\end{exemple}
\begin{exemple}\pjya{to-k-ɤɣɯdɯχɯn-ci}\hspace{5pt}\pcmn{气味比以前香}\end{exemple}
\begin{exemple}\pjya{@pingguo nɯ kɯ-ɤɣɯdɯχɯn ci ɲɯ-ŋu}\hspace{5pt}\pcmn{那个苹果有香味}\end{exemple}\relationsémantique{参考}{\lien{ⓔɯ-dɯχɯn}{ɯ-dɯχɯn}}\end{entrée}

\begin{entrée}{aɣɯɣu}{}{ⓔaɣɯɣu} 
\classe{vi} \paradigme{dir}{tɤ-}\paradigme{construction}{participe sujet}
\begin{définition}\pfra{se préparer}\end{définition}
\begin{définition}\pcmn{准备}\end{définition}
\begin{exemple}\pjya{ku-oɣɯɣu-a}\hspace{5pt}\pcmn{我在准备}\end{exemple}
\begin{exemple}\pjya{ku-oɣɯɣu}\hspace{5pt}\pcmn{他在准备}\end{exemple}
\begin{exemple}\pjya{to-k-ɤɣɯɣu-ci}\hspace{5pt}\pcmn{已经准备好了}\end{exemple}
\begin{exemple}\pjya{rpɣo kɯ-ɕe to-k-ɤɣɯɣu-ci}\hspace{5pt}\pcmn{他准备到高山上去}\end{exemple}
\begin{exemple}\pjya{cɤmi kɯ-ɕe to-k-ɤɣɯɣu-ci}\hspace{5pt}\pcmn{他准备去河边}\end{exemple}
\begin{exemple}\pjya{aj @Chengdu kɯ-ɕe tɤ-aɣɯɣu-a}\hspace{5pt}\pcmn{我准备去成都}\end{exemple}
\begin{exemple}\pjya{rqaco kɯ-ɕe to-k-ɤɣɯɣu-ci}\hspace{5pt}\pcmn{他准备去呷脚}\end{exemple}\end{entrée}

\begin{entrée}{aɣɯɣli}{}{ⓔaɣɯɣli} 
\classe{vs}  
\grammaire{denom} 
\begin{définition}\pfra{qui produit beaucoup de purin}\end{définition}
\begin{définition}\pcmn{粪很多的}\end{définition}\relationsémantique{参考}{\lien{ⓔtɯ-ɣli}{tɯ-ɣli}}\end{entrée}

\begin{entrée}{aɣɯjaʁ/\variante{aɣɯjɯjaʁ}}{₂}{ⓔaɣɯjaʁⓗ2} 
\classe{vs} \sens{1}
\begin{définition}\pfra{toucher les objets des autres}\end{définition}
\begin{définition}\pcmn{乱摸别人的东西(偷东西)}\end{définition}
\begin{exemple}\pjya{jiɕqha tɯrme kɯ-ɤɣɯjaʁ ci ŋu}\hspace{5pt}\pcmn{这个人会偷东西}\end{exemple}\sens{2}
\begin{définition}\pfra{toucher les objets des autres}\end{définition}
\begin{définition}\pcmn{乱摸别人的东西(偷东西),做事很快}\end{définition}
\begin{exemple}\pjya{tɤ-pɤtso nɯ kɯ-ɤɣɯjɯjaʁ ci ɲɯ-ŋu}\hspace{5pt}\pcmn{那个孩子习惯乱摸别人的东西}\end{exemple}
\begin{exemple}\pjya{ɕɯŋgɯ mɯ́j-fse to-k-ɤɣɯjɯjaʁ-ci}\hspace{5pt}\pcmn{跟以前不一样,现在总是乱碰别人的东西}\end{exemple}\sens{3}
\begin{définition}\pfra{rapide}\end{définition}
\begin{définition}\pcmn{做事很快}\end{définition}
\begin{exemple}\pjya{jiɕqha nɯ ɲɯ-ɤɣɯjɯjaʁ}\hspace{5pt}\pcmn{这个人做事很快}\end{exemple}\relationsémantique{参考}{\lien{ⓔtɯ-jaʁ}{tɯ-jaʁ}}\end{entrée}

\begin{entrée}{aɣɯjɯjaʁ}{₁}{ⓔaɣɯjɯjaʁⓗ1} 
\classe{vi} \paradigme{dir}{tɤ-}
\begin{définition}\pfra{avoir beaucoup de pattes}\end{définition}
\begin{définition}\pcmn{有很多只脚(蜈蚣等虫子)}\end{définition}
\begin{exemple}\pjya{qaprɤftsa ɲɯ-ɤɣɯjɯjaʁ}\hspace{5pt}\pcmn{蜈蚣有很多只脚}\end{exemple}
\begin{exemple}\pjya{ɴɢoɕna ɲɯ-ɤɣɯjɯjaʁ ɲɯ-ɤɣɯmɯmi, ɯ-mɤlɤjaʁ ɲɯ-dɤn}\hspace{5pt}\pcmn{蜘蛛有很多脚}\end{exemple}\relationsémantique{参考}{\lien{ⓔtɯ-jaʁ}{tɯ-jaʁ}}\end{entrée}

\begin{entrée}{aɣɯjwaʁ}{}{ⓔaɣɯjwaʁ} 
\classe{vi}  
\grammaire{denom} \paradigme{dir}{tɤ-}
\begin{définition}\pfra{touffu, ayant beaucoup de feuilles}\end{définition}
\begin{définition}\pcmn{叶子茂盛}\end{définition}
\begin{exemple}\pjya{jiɕqha si nɯ ɲɯ-ɤɣɯjwaʁ}\hspace{5pt}\pcmn{这棵树长出很多叶子}\end{exemple}
\begin{exemple}\pjya{jiɕqha ɟu nɯ ɲɯ-ɤɣɯjwaʁ}\hspace{5pt}\pcmn{这棵竹子长出很多叶子}\end{exemple}
\begin{exemple}\pjya{@wosun to-k-ɤɣɯjwaʁ-ci}\hspace{5pt}\pcmn{莴笋跟以前不一样,现在长出很多叶子}\end{exemple}\end{entrée}

\begin{entrée}{aɣɯlu}{}{ⓔaɣɯlu} 
\classe{vi}  
\grammaire{denom} 
\begin{définition}\pfra{qui a beaucoup de lait}\end{définition}
\begin{définition}\pcmn{产奶多的(母牛)}\end{définition}
\begin{exemple}\pjya{ki nɯŋa ki ɲɯ-ɤɣɯlu}\hspace{5pt}\pcmn{这个母牛产很多奶}\end{exemple}\relationsémantique{参考}{\lien{ⓔtɤ-lu}{tɤ-lu}}\end{entrée}

\begin{entrée}{aɣɯli}{}{ⓔaɣɯli} 
\classe{vs} \paradigme{dir}{tɤ-}
\begin{définition}\pfra{patient}\end{définition}
\begin{définition}\pcmn{有耐性;不怕失败;不怕吃苦}\end{définition}
\begin{exemple}\pjya{kɤ-rɤβzjoz nɯ, pjɯ-kɯ-ɤɣɯli ʑo ra, nɯ mɤɕtʂa kú-wɣ-spa mɤ-kɯ-cha}\hspace{5pt}\pcmn{学习要有耐性,不然不是会成功的}\end{exemple}
\begin{exemple}\pjya{nɤʑo ɲɯ-tɯ-ɤɣɯli tɕe ɲɯ-pe}\hspace{5pt}\pcmn{你很有耐性,这样很好}\end{exemple}\end{entrée}

\begin{entrée}{aɣɯlɯtshɤt}{}{ⓔaɣɯlɯtshɤt} 
\classe{vs} 
\begin{définition}\pfra{à peu près du même âge}\end{définition}
\begin{définition}\pcmn{跟自己年龄差不多的}\end{définition}
\begin{exemple}\pjya{ndʑiʑo ɲɯ-tɯ-ɤɣɯlɯtshɤt-ndʑi}\hspace{5pt}\pcmn{你们俩年龄差不多}\end{exemple}\relationsémantique{参考}{\lien{ⓔɯ-lɯtshɤt}{ɯ-lɯtshɤt}}\end{entrée}

\begin{entrée}{aɣɯmar}{}{ⓔaɣɯmar} 
\classe{vi}  
\grammaire{denom} 
\begin{définition}\pfra{qui peut produire beaucoup de beurre}\end{définition}
\begin{définition}\pcmn{可以打出很多酥油}\end{définition}
\begin{exemple}\pjya{ki nɯŋa ɯ-lu ɲɯ-ɤɣɯmar}\hspace{5pt}\pcmn{这个母牛的奶可以打出很多酥油}\end{exemple}\relationsémantique{参考}{\lien{ⓔta-mar}{ta-mar}}\end{entrée}

\begin{entrée}{aɣɯmat}{}{ⓔaɣɯmat} 
\classe{vi}  
\grammaire{denom} \paradigme{dir}{thɯ-}
\begin{définition}\pfra{faire beaucoup de fruits}\end{définition}
\begin{définition}\pcmn{结很多果子}\end{définition}
\begin{exemple}\pjya{@pingguo ɲɯ-ɤɣɯmat}\hspace{5pt}\pcmn{苹果树结很多果子}\end{exemple}
\begin{exemple}\pjya{ʑɴɢɯloʁ ɲɯ-ɣɯmat}\hspace{5pt}\pcmn{核桃结很多果子}\end{exemple}
\begin{exemple}\pjya{ɣɯjpa ji-tɕɣom chɤ-k-ɤɣɯmat-ci}\hspace{5pt}\pcmn{今年我们家的花椒结的果子比以前多}\end{exemple}\end{entrée}

\begin{entrée}{aɣɯmdoʁ}{}{ⓔaɣɯmdoʁ} 
\classe{vi}  
\grammaire{denom} \paradigme{dir}{tɤ-}
\begin{définition}\pfra{avoir la même couleur}\end{définition}
\begin{définition}\pcmn{颜色相同}\end{définition}
\begin{exemple}\pjya{to-k-ɤɣɯmdoʁ-ci}\hspace{5pt}\pcmn{颜色以前不一样,现在一样了}\end{exemple}
\begin{exemple}\pjya{tɕiʑo ɣɯ tɕi-ŋga ɲɯ-ɤɣɯmdoʁ}\hspace{5pt}\pcmn{我们的衣服是同一个颜色的}\end{exemple}\relationsémantique{参考}{\lien{ⓔɯ-mdoʁ}{ɯ-mdoʁ}}\end{entrée}

\begin{entrée}{aɣɯmɲaʁ}{}{ⓔaɣɯmɲaʁ} 
\classe{vs}  
\grammaire{denom} \sens{1}
\begin{définition}\pfra{qui a beaucoup de trous}\end{définition}
\begin{définition}\pcmn{有很多小洞}\end{définition}
\begin{exemple}\pjya{nɤki tshaʁ wuma ʑo ɲɯ-ɤɣɯmɲaʁ}\hspace{5pt}\pcmn{那个筛子有很多眼儿}\end{exemple}\sens{2}
\begin{définition}\pfra{qui a des yeux}\end{définition}
\begin{définition}\pcmn{有眼睛}\end{définition}\relationsémantique{参考}{\lien{ⓔtɯ-mɲaʁ}{tɯ-mɲaʁ}}\end{entrée}

\begin{entrée}{aɣɯmphɯmphru}{}{ⓔaɣɯmphɯmphru} 
\classe{vi} \paradigme{dir}{tɤ-}
\begin{définition}\pfra{cohérent, suivi}\end{définition}
\begin{définition}\pcmn{连贯(话)}\end{définition}
\begin{exemple}\pjya{li to-k-ɤ-ɣɯmphɯmphru-ci}\hspace{5pt}\pcmn{又变得连贯(连续不断)了}\end{exemple}
\begin{exemple}\pjya{nɯɕɯŋgɯ ɯ-rju pɯ-aɣɯmphɯmphru ri, tham tɕe mɯ-ɲɤ-cha}\hspace{5pt}\pcmn{他以前讲话比较连贯,现在不行了}\end{exemple}
\begin{sous-entrée}{zɣɯmphɯmphru}{ⓔaɣɯmphɯmphruⓝzɣɯmphɯmphru} 
\classe{vt} 
\begin{définition}\pfra{faire de façon complète, cohérente}\end{définition}
\begin{définition}\pcmn{做得连贯}\end{définition}
\begin{exemple}\pjya{χpi kɤ-fɕɤt kɤ-zɣɯmphɯmphru ɲɯ-tɯ-cha}\hspace{5pt}\pcmn{你能够把故事连贯地讲出来}\end{exemple}\relationsémantique{参考}{\lien{ⓔɯ-mphru}{ɯ-mphru}}\end{sous-entrée}

\end{entrée}

\begin{entrée}{aɣɯmtɕhi}{}{ⓔaɣɯmtɕhi} 
\classe{vs}  
\grammaire{denom} 
\begin{définition}\pfra{bavard}\end{définition}
\begin{définition}\pcmn{说话很多}\end{définition}
\begin{exemple}\pjya{ki tɯrme ki kɯ-ɤɣɯmtɕhi ci ŋu}\hspace{5pt}\pcmn{这个人说的话多}\end{exemple}\relationsémantique{参考}{\lien{ⓔtɯ-mtɕhi}{tɯ-mtɕhi}}\end{entrée}

\begin{entrée}{aɣɯmɯmi}{}{ⓔaɣɯmɯmi} 
\classe{vs}  
\grammaire{denom} 
\begin{définition}\pfra{qui a beaucoup de pattes}\end{définition}
\begin{définition}\pcmn{脚很多}\end{définition}\relationsémantique{参考}{\lien{}{aɣɯjɯjaʁ}}\relationsémantique{参考}{\lien{ⓔtɯ-mi}{tɯ-mi}}\end{entrée}

\begin{entrée}{aɣɯndzɣi}{}{ⓔaɣɯndzɣi} 
\classe{vs}  
\grammaire{denom} 
\begin{définition}\pfra{qui a des crocs}\end{définition}
\begin{définition}\pcmn{有獠牙}\end{définition}
\begin{exemple}\pjya{srɯnmɯ ɲɯ-ɤɣɯndzɯndzrɯ ɲɯ-ɤɣɯndzɯndzɣi}\hspace{5pt}\pcmn{妖精有爪子又有獠牙}\end{exemple}\relationsémantique{参考}{\lien{ⓔtɯ-ndzɣi}{tɯ-ndzɣi}}\end{entrée}

\begin{entrée}{aɣɯndzrɯ}{}{ⓔaɣɯndzrɯ} 
\classe{vs}  
\grammaire{denom} 
\begin{définition}\pfra{qui a des griffes}\end{définition}
\begin{définition}\pcmn{有爪子}\end{définition}
\begin{exemple}\pjya{srɯnmɯ ɲɯ-ɤɣɯndzɯndzrɯ ɲɯ-ɤɣɯndzɯndzɣi}\hspace{5pt}\pcmn{妖精有爪子又有獠牙}\end{exemple}\relationsémantique{参考}{\lien{ⓔtɯ-ndzrɯ}{tɯ-ndzrɯ}}\end{entrée}

\begin{entrée}{aɣɯndʑɯɣ}{}{ⓔaɣɯndʑɯɣ} 
\classe{vs}  
\grammaire{denom} \paradigme{dir}{nɯ-}\sens{1}
\begin{définition}\pfra{qui a beaucoup de résine}\end{définition}
\begin{définition}\pcmn{树脂多}\end{définition}
\begin{exemple}\pjya{tɯrgi ɲɯ-ɤɣɯndʑɯɣ}\hspace{5pt}\pcmn{杉树有很多树脂}\end{exemple}
\begin{exemple}\pjya{tɤtho ɲɯ-ɤɣɯndʑɯɣ}\hspace{5pt}\pcmn{松树有很多树脂}\end{exemple}\sens{2}
\begin{définition}\pfra{collant}\end{définition}
\begin{définition}\pcmn{有粘性}\end{définition}
\begin{exemple}\pjya{ɯ-ɕtʂi pjɤ-ɬoʁ tɕe, ɯ-ɕtʂi ra ɲɯ-ɤɣɯndʑɯɣ ʑo}\hspace{5pt}\pcmn{他流汗,一身黏糊糊的}\end{exemple}\relationsémantique{参考}{\lien{ⓔtɤ-ndʑɯɣ}{tɤ-ndʑɯɣ}}\end{entrée}

\begin{entrée}{aɣɯntɤβ}{}{ⓔaɣɯntɤβ} 
\classe{vs} 
\begin{définition}\pfra{qui a des bulles, moussant}\end{définition}
\begin{définition}\pcmn{泡沫多的}\end{définition}\relationsémantique{参考}{\lien{ⓔtɤntɤβ}{tɤntɤβ}}\end{entrée}

\begin{entrée}{aɣɯɲchaz}{}{ⓔaɣɯɲchaz} 
\classe{vi} 
\begin{définition}\pfra{identique}\end{définition}
\begin{définition}\pcmn{(话、动作都)一致;整齐}\end{définition}
\begin{exemple}\pjya{ʁmaʁmi ra wuma ʑo ɲɯ-ɤɣɯɲchaz-nɯ, nɯsthɯci ʑo kɯ-dɤn nɤ, tɯrme tɯ-rdoʁ kɯ-ŋu ɲɯ-fse}\hspace{5pt}\pcmn{士兵操练发出整齐,动作一致好像只有一个人一样}\end{exemple}\end{entrée}

\begin{entrée}{aɣɯŋgɤr}{}{ⓔaɣɯŋgɤr} 
\classe{vs}  
\grammaire{denom} 
\begin{définition}\pfra{qui a beaucoup de lard}\end{définition}
\begin{définition}\pcmn{猪膘多}\end{définition}
\begin{exemple}\pjya{ki paʁ ki ɲɯ-ɤɣɯŋgɤr}\hspace{5pt}\pcmn{这只猪膘很多}\end{exemple}\end{entrée}

\begin{entrée}{aɣɯŋgɯŋgɯ}{}{ⓔaɣɯŋgɯŋgɯ} 
\classe{vi}  
\grammaire{denom} \paradigme{dir}{thɯ-}\paradigme{dir}{thɯ-}\paradigme{dir}{pɯ-}
\begin{définition}\pfra{être l'un dans l'autre (sac etc)}\end{définition}
\begin{définition}\pcmn{一层一层套在一起(袋子)、由很多层组成的}\end{définition}
\begin{définition}\pfra{mettre l'un dans l'autre (sac etc)}\end{définition}
\begin{définition}\pcmn{连套几层}\end{définition}
\begin{exemple}\pjya{cho-k-ɤɣɯŋgɯŋgɯ-ci}\hspace{5pt}\pcmn{本来不是一层一层套在一起的,现在是了}\end{exemple}
\begin{exemple}\pjya{paʁɕi nɯ kɤ-ɣɯŋgɯŋgɯ ʑo kɤ-ndza sna}\hspace{5pt}\pcmn{苹果可以连皮子一起吃。}\end{exemple}
\begin{exemple}\pjya{tɯ-rju kɯ-ɤmɯtso tɤ-βze, kɯ-ɤɣɯŋgɯŋgɯ nɯ ma-tɤ-tɯ-ti}\hspace{5pt}\pcmn{你话讲得清楚一点,不要好像话中有话}\end{exemple}
\begin{exemple}\pjya{tɤ-fkɯm ʁnɯz chɯ́-wɣ-rku chɯ́-wɣ-zɣɯŋgɯŋgɯ}\hspace{5pt}\pcmn{把两个袋子套起来}\end{exemple}
\begin{exemple}\pjya{khɯtsa tɯ-rdoʁ ɲɯ́-wɣ-ta, nɯ ɯ-taʁ li ci tɯ-rdoʁ pjɯ́-wɣ-ta pjɯ́-wɣ-zɣɯŋgɯŋgɯ}\hspace{5pt}\pcmn{在那里放了一个碗,上面又放了另外一个碗,把两个碗套起来放。}\end{exemple}
\begin{sous-entrée}{zɣɯŋgɯŋgɯ}{ⓔaɣɯŋgɯŋgɯⓝzɣɯŋgɯŋgɯ} 
\classe{vt} \end{sous-entrée}

\end{entrée}

\begin{entrée}{aɣɯŋkɯ}{}{ⓔaɣɯŋkɯ} 
\classe{vs}  
\grammaire{denom} 
\begin{définition}\pfra{qui a la couenne épaisse}\end{définition}
\begin{définition}\pcmn{皮子很厚(猪)}\end{définition}
\begin{exemple}\pjya{ki paʁ ki ɲɯ-ɤɣɯŋkɯ}\hspace{5pt}\pcmn{这只猪的皮子很厚}\end{exemple}\relationsémantique{参考}{\lien{ⓔtɤ-ŋkɯ}{tɤ-ŋkɯ}}\end{entrée}

\begin{entrée}{aɣɯpɤrtsaβ}{}{ⓔaɣɯpɤrtsaβ} 
\classe{vi} \paradigme{dir}{tɤ-}
\begin{définition}\pfra{zélé, assidu}\end{définition}
\begin{définition}\pcmn{勤快}\end{définition}
\begin{exemple}\pjya{iɕqha ta-ma nɯ wuma ɲɯ-ɤɣɯpɤrtsaβ}\hspace{5pt}\pcmn{他工作很勤快}\end{exemple}
\begin{exemple}\pjya{ki ɯ-ʁɤri staʁ ta-ma wuma ʑo to-k-ɤɣɯpɤrtsaβ-ci}\hspace{5pt}\pcmn{他工作比以前勤快很多}\end{exemple}\end{entrée}

\begin{entrée}{aɣɯpharɤβ}{}{ⓔaɣɯpharɤβ} 
\classe{vi} \paradigme{dir}{tɤ-}
\begin{définition}\pfra{généreux}\end{définition}
\begin{définition}\pcmn{大方}\end{définition}
\begin{exemple}\pjya{ɕɯŋgɯ staʁ to-k-ɤɣɯpharɤβ-ci}\hspace{5pt}\pcmn{他比以前大方}\end{exemple}
\begin{exemple}\pjya{ɯ-ndzɤtshi cho wuma kɯ-ɤɣɯpharɤβ ɲɯ-ŋu}\hspace{5pt}\pcmn{他对吃的东西舍得花钱}\end{exemple}\end{entrée}

\begin{entrée}{aɣɯpɯpɯ}{}{ⓔaɣɯpɯpɯ} 
\classe{vs}  
\grammaire{denom} 
\begin{définition}\pfra{qui a beaucoup de petits}\end{définition}
\begin{définition}\pcmn{生很多崽子}\end{définition}\relationsémantique{参考}{\lien{ⓔtɤ-pɯ}{tɤ-pɯ}}\end{entrée}

\begin{entrée}{aɣɯqe}{}{ⓔaɣɯqe} 
\classe{vs}  
\grammaire{denom} 
\begin{définition}\pfra{qui produit beaucoup d'excréments}\end{définition}
\begin{définition}\pcmn{屙很多屎}\end{définition}
\begin{exemple}\pjya{ki paʁ ki ɲɯ-ɤɣɯqe}\hspace{5pt}\pcmn{这头猪产很多肥料}\end{exemple}\relationsémantique{参考}{\lien{ⓔtɯ-qe}{tɯ-qe}}\end{entrée}

\begin{entrée}{aɣɯrɟit}{}{ⓔaɣɯrɟit} 
\classe{vs}  
\grammaire{denom} 
\begin{définition}\pfra{qui a beaucoup d'enfants}\end{définition}
\begin{définition}\pcmn{生很多孩子}\end{définition}\relationsémantique{参考}{\lien{ⓔtɤ-rɟit}{tɤ-rɟit}}\end{entrée}

\begin{entrée}{aɣɯrkɯrkɯ}{}{ⓔaɣɯrkɯrkɯ} 
\classe{vi} \paradigme{dir}{kɤ-}\paradigme{dir}{tɤ-}\paradigme{dir}{tɤ-}
\begin{définition}\pfra{s'enrouler}\end{définition}
\begin{définition}\pcmn{盘起来}\end{définition}
\begin{définition}\pfra{enrouler}\end{définition}
\begin{définition}\pcmn{卷起来}\end{définition}
\begin{exemple}\pjya{qapri ko-k-ɤɣɯrkɯrkɯ-ci}\hspace{5pt}\pcmn{蛇盘起来了}\end{exemple}
\begin{exemple}\pjya{tɤ-ri ɲɯ-ɤɣɯrkɯrkɯ}\hspace{5pt}\pcmn{线是盘起来的}\end{exemple}
\begin{exemple}\pjya{tɤ-ri tɤ-zɣɯrkɯrkɯ-t-a}\hspace{5pt}\pcmn{我把线卷起来了}\end{exemple}\relationsémantique{同义词}{\lien{ⓔrɤlkɯɣ}{rɤlkɯɣ}}
\begin{sous-entrée}{zɣɯrkɯrkɯ}{ⓔaɣɯrkɯrkɯⓝzɣɯrkɯrkɯ} 
\classe{vt} \end{sous-entrée}

\end{entrée}

\begin{entrée}{aɣɯrmbi}{}{ⓔaɣɯrmbi} 
\classe{vs}  
\grammaire{denom} 
\begin{définition}\pfra{qui urine souvent}\end{définition}
\begin{définition}\pcmn{经常撒尿}\end{définition}
\begin{exemple}\pjya{nɤki tɤ-pɤtso nɯ ɲɯ-ɤɣɯrmbi}\hspace{5pt}\pcmn{那个小孩子经常撒尿}\end{exemple}\relationsémantique{参考}{\lien{ⓔtɯ-rmbi}{tɯ-rmbi}}\end{entrée}

\begin{entrée}{aɣɯrme}{}{ⓔaɣɯrme} 
\classe{vs}  
\grammaire{denom} \paradigme{dir}{thɯ-}
\begin{définition}\pfra{poilu}\end{définition}
\begin{définition}\pcmn{毛多}\end{définition}
\begin{exemple}\pjya{cho-k-ɤɣɯrme-ci}\hspace{5pt}\pcmn{他长出了很多毛}\end{exemple}
\begin{exemple}\pjya{ɣzɯ ɯ-rŋa nɯ tɯrme mɤ-fse kɯ aɣɯrmɯrme}\hspace{5pt}\pcmn{猴子的脸跟人脸不同的地方就是长有很多毛}\end{exemple}\relationsémantique{参考}{\lien{ⓔtɤ-rme}{tɤ-rme}}\end{entrée}

\begin{entrée}{aɣɯrna}{}{ⓔaɣɯrna} 
\classe{vs}  
\grammaire{denom} 
\begin{définition}\pfra{qui a des oreilles}\end{définition}
\begin{définition}\pcmn{长有耳朵}\end{définition}
\begin{exemple}\pjya{nɤki tɯrme ra kɯ-ɤɣɯrna ʁɟa ɕti tɕe mtshɤm-nɯ}\hspace{5pt}\pcmn{那些人是长了耳朵的,会听见的}\end{exemple}\relationsémantique{参考}{\lien{ⓔtɯ-rna}{tɯ-rna}}\end{entrée}

\begin{entrée}{aɣɯrnɯɕɯr}{}{ⓔaɣɯrnɯɕɯr} 
\classe{vi} \paradigme{dir}{nɯ-}
\begin{définition}\pfra{rougeâtre}\end{définition}
\begin{définition}\pcmn{淡红色}\end{définition}
\begin{exemple}\pjya{ki ɯ-mdoʁ ɲɯ-ɤɣɯrnɯɕɯr}\hspace{5pt}\pcmn{这个东西变成淡红色}\end{exemple}
\begin{exemple}\pjya{kɯki ɯ-mdoʁ ɲɤ-k-ɤɣɯrnɯɕɯr-ci}\hspace{5pt}\pcmn{这个东西的颜色变淡红色}\end{exemple}\relationsémantique{参考}{\lien{ⓔɣɯrni}{ɣɯrni}}\end{entrée}

\begin{entrée}{aɣɯrŋa}{}{ⓔaɣɯrŋa} 
\classe{vs}  
\grammaire{denom} 
\begin{définition}\pfra{se ressembler}\end{définition}
\begin{définition}\pcmn{脸很像}\end{définition}
\begin{exemple}\pjya{a-pi cho ɯ-ɲɯ-ɤ́ɣɯrŋa-tɕi?}\hspace{5pt}\pcmn{我跟哥哥像不像?}\end{exemple}\relationsémantique{参考}{\lien{ⓔɣɤrŋa}{ɣɤrŋa}}\relationsémantique{参考}{\lien{ⓔtɯ-rŋa}{tɯ-rŋa}}\end{entrée}

\begin{entrée}{aɣɯrŋɯl}{}{ⓔaɣɯrŋɯl} 
\classe{vs}  
\grammaire{denom} 
\begin{définition}\pfra{qui a beaucoup d'argent}\end{définition}
\begin{définition}\pcmn{有很多银子,很多钱}\end{définition}\relationsémantique{参考}{\lien{ⓔrŋɯl}{rŋɯl}}\end{entrée}

\begin{entrée}{aɣɯrpaʁ}{}{ⓔaɣɯrpaʁ} 
\classe{vi}  
\grammaire{denom} 
\begin{définition}\pfra{s'entendre bien}\end{définition}
\begin{définition}\pcmn{合得来}\end{définition}
\begin{exemple}\pjya{nɯni ɲɯ-ɤɣɯrpaʁ-ndʑi}\hspace{5pt}\pcmn{那两个人很合得来}\end{exemple}\relationsémantique{参考}{\lien{ⓔanɤrpɯrpaʁ}{anɤrpɯrpaʁ}}\relationsémantique{参考}{\lien{ⓔtɯ-rpaʁ}{tɯ-rpaʁ}}\end{entrée}

\begin{entrée}{aɣɯrqhu}{}{ⓔaɣɯrqhu} 
\classe{vs}  
\grammaire{denom} 
\begin{définition}\pfra{qui a une écorce épaisse}\end{définition}
\begin{définition}\pcmn{树皮很厚、很多}\end{définition}
\begin{exemple}\pjya{sɤjku cho mbraj ni aɣɯrqhu-ndʑi}\hspace{5pt}\pcmn{白桦树和红桦树树皮很多}\end{exemple}\relationsémantique{参考}{\lien{ⓔtɤ-rqhu}{tɤ-rqhu}}\end{entrée}

\begin{entrée}{aɣɯrtsi}{}{ⓔaɣɯrtsi} 
\classe{vs}  
\grammaire{denom} \paradigme{dir}{thɯ-}
\begin{définition}\pfra{qui produit beaucoup de graisse}\end{définition}
\begin{définition}\pcmn{产油多}\end{définition}
\begin{exemple}\pjya{ki paʁ ki ɲɯ-ɤɣɯrtsi}\hspace{5pt}\pcmn{这只猪的油很多}\end{exemple}\relationsémantique{参考}{\lien{ⓔtɤ-rtsi}{tɤ-rtsi}}\end{entrée}

\begin{entrée}{aɣɯrtsɯrtsɤɣ}{}{ⓔaɣɯrtsɯrtsɤɣ} 
\classe{vs} \paradigme{dir}{tɤ-}
\begin{définition}\pfra{composé de sections}\end{définition}
\begin{définition}\pcmn{一节一节组成的}\end{définition}
\begin{exemple}\pjya{jima ɲɯ-ɤɣɯrtsɯrtsɤɣ}\hspace{5pt}\pcmn{玉米是一节一节的}\end{exemple}
\begin{exemple}\pjya{ɟu ɲɯ-ɤɣɯrtsɯrtsɤɣ}\hspace{5pt}\pcmn{竹子是一节一节的}\end{exemple}\relationsémantique{参考}{\lien{ⓔarɤrtsɯrtsɤɣ}{arɤrtsɯrtsɤɣ}}\relationsémantique{参考}{\lien{ⓔtɯ-rtsɤɣ}{tɯ-rtsɤɣ}}\end{entrée}

\begin{entrée}{aɣɯrtɯrtaʁ}{}{ⓔaɣɯrtɯrtaʁ} 
\classe{vi}  
\grammaire{denom} \paradigme{dir}{tɤ-}
\begin{définition}\pfra{avoir beaucoup de branches}\end{définition}
\begin{définition}\pcmn{枝桠多}\end{définition}
\begin{exemple}\pjya{kɯki si ɲɯ-ɤɣɯrtɯrtaʁ}\hspace{5pt}\pcmn{这棵树有很多枝桠}\end{exemple}
\begin{exemple}\pjya{kɯki @huatong ɯ-jɯ ki ɲɯ-ɤɣɯrtɯrtaʁ}\hspace{5pt}\pcmn{这个话筒的支架有架脚(三脚架)}\end{exemple}
\begin{exemple}\pjya{to-k-ɤɣɯrtɯrtaʁ-ci}\hspace{5pt}\pcmn{枝桠比以前多}\end{exemple}\relationsémantique{参考}{\lien{ⓔtɤ-rtaʁ}{tɤ-rtaʁ}}\end{entrée}

\begin{entrée}{aɣɯrɯru}{₁}{ⓔaɣɯrɯruⓗ1} 
\classe{vs} 
\begin{définition}\pfra{que l'on peut faire en même temps que quelque chose d'autre}\end{définition}
\begin{définition}\pcmn{可以顺便进行(不需要专门去)}\end{définition}
\begin{exemple}\pjya{aʑo nɯ-anɯri-a tɕe, @cai tu-χti-a ɲɯ-ɤɣɯrɯru tɕe ɲɯ-tsɯm-a ŋu}\hspace{5pt}\pcmn{我回家的时候,可以顺便买菜带回家}\end{exemple}
\begin{sous-entrée}{zɣɯrɯru}{ⓔaɣɯrɯruⓗ1ⓝzɣɯrɯru} 
\classe{vi} 
\begin{définition}\pfra{en profiter pour}\end{définition}
\begin{définition}\pcmn{顺便进行}\end{définition}
\begin{exemple}\pjya{kɤ-nɤma ra tú-wɣ-zɣɯrɯru tɕe mbat}\hspace{5pt}\pcmn{可以顺道做的事情,不要专门耽误时间就会更方便}\end{exemple}\end{sous-entrée}

\end{entrée}

\begin{entrée}{aɣɯrɯru}{₂}{ⓔaɣɯrɯruⓗ2} 
\classe{vs}  
\grammaire{denom} 
\begin{définition}\pfra{qui a plusieurs troncs, et peu de feuilles et de branches}\end{définition}
\begin{définition}\pcmn{树干很多,叶子和树枝不多}\end{définition}
\begin{exemple}\pjya{ki si ki kɯ-ɤɣɯrɯru ci ɲɯ-ŋu ma ɯ-jwaʁ mɯ́j-dɤn}\hspace{5pt}\pcmn{这棵树树干很多,树叶不多}\end{exemple}\relationsémantique{参考}{\lien{ⓔɯ-ru}{ɯ-ru}}\end{entrée}

\begin{entrée}{aɣɯrɯz}{}{ⓔaɣɯrɯz} 
\classe{vs}  
\grammaire{denom} 
\begin{définition}\pfra{hériter (d'un trait)}\end{définition}
\begin{définition}\pcmn{继承}\end{définition}
\begin{exemple}\pjya{ɯʑo ɯ-skɤt kɯ-sna nɯ, ɯ-wa ɲɤ-k-ɤɣɯrɯz-ci}\hspace{5pt}\pcmn{他的声音很好听,是继承了他父亲}\end{exemple}
\begin{exemple}\pjya{aʑo a-wa fse-a ma ɲɤ-k-ɤɣɯrɯz-a-ci}\hspace{5pt}\pcmn{我像我父亲因为遗传他}\end{exemple}
\begin{exemple}\pjya{nɤʑo ɲɤ-k-ɤɣɯrɯz-a-ci tɕe a-tɕhaʁa ɣɤʑu}\hspace{5pt}\pcmn{我遗传你,我有双眼皮}\end{exemple}\end{entrée}

\begin{entrée}{aɣɯrʑɯrʑɯɣ}{}{ⓔaɣɯrʑɯrʑɯɣ} 
\classe{vi}  
\grammaire{denom} \paradigme{dir}{tɤ-}
\begin{définition}\pfra{ridé}\end{définition}
\begin{définition}\pcmn{皱(脸)}\end{définition}
\begin{exemple}\pjya{tɕheme ra nɯ-rti nɯ ɲɯ-ɤɣɯrʑɯrʑɯɣ}\hspace{5pt}\pcmn{女孩们的裙子是皱着的}\end{exemple}
\begin{exemple}\pjya{to-k-ɤɣɯrʑɯrʑɯɣ-ci}\hspace{5pt}\pcmn{变得有皱褶了}\end{exemple}
\begin{exemple}\pjya{nɤki nɯ to-rgɤz ma ɯ-rŋa ra to-ɣɯrʑɯrʑɯɣ}\hspace{5pt}\pcmn{他已经老了,满脸都是皱纹}\end{exemple}\relationsémantique{参考}{\lien{ⓔtɤ-rʑɯɣ}{tɤ-rʑɯɣ}}\end{entrée}

\begin{entrée}{aɣɯsmɤɣ}{}{ⓔaɣɯsmɤɣ} 
\classe{vs}  
\grammaire{denom} \paradigme{dir}{thɯ-}
\begin{définition}\pfra{qui a beaucoup de laine}\end{définition}
\begin{définition}\pcmn{毛长得多}\end{définition}
\begin{exemple}\pjya{ki qaʑo ki ɲɯ-ɤɣɯsmɤɣ}\hspace{5pt}\pcmn{这只羊长了很多毛}\end{exemple}\relationsémantique{参考}{\lien{ⓔsmɤɣ}{smɤɣ}}\end{entrée}

\begin{entrée}{aɣɯsmɤn}{}{ⓔaɣɯsmɤn} 
\classe{vi}  
\grammaire{denom} \paradigme{dir}{tɤ-}
\begin{définition}\pfra{avoir un effet médical}\end{définition}
\begin{définition}\pcmn{有药性}\end{définition}
\begin{exemple}\pjya{ɕɯŋgɯ staʁnɤ ɯ-jaʁ to-oɣɯsmɤn}\hspace{5pt}\pcmn{(药)医他的手效果比以前好}\end{exemple}\relationsémantique{参考}{\lien{ⓔsmɤn}{smɤn}}\end{entrée}

\begin{entrée}{aɣɯsɯm}{}{ⓔaɣɯsɯm} 
\classe{vi}  
\grammaire{denom} \paradigme{dir}{nɯ-}
\begin{définition}\pfra{s’entendre}\end{définition}
\begin{définition}\pcmn{一条心;想得一致}\end{définition}
\begin{exemple}\pjya{ɲɯ-ɤɣɯsɯm-tɕi}\hspace{5pt}\pcmn{我们俩想得一致}\end{exemple}
\begin{exemple}\pjya{ʑara ɲɯ-ɤɣɯsɯm-nɯ}\hspace{5pt}\pcmn{他们想得一致}\end{exemple}
\begin{exemple}\pjya{jisŋi tɕi-tɯkrɤz ɲɯ-ɣi ma ɲɯ-ɤɣɯsɯm-tɕi}\hspace{5pt}\pcmn{今天我们谈得很融洽}\end{exemple}
\begin{exemple}\pjya{to-k-ɤɣɯsɯm-ndʑi-ci}\hspace{5pt}\pcmn{他们俩变得齐心了}\end{exemple}\relationsémantique{参考}{\lien{ⓔtɯ-sɯm}{tɯ-sɯm}}\end{entrée}

\begin{entrée}{aɣɯʂwaŋ}{}{ⓔaɣɯʂwaŋ} 
\classe{vs} \paradigme{dir}{tɤ-}
\begin{définition}\pfra{se correspondre}\end{définition}
\begin{définition}\pcmn{相称}\end{définition}
\begin{exemple}\pjya{nɤ-ŋga cho nɤ-xtsa ra a-pɯ-ɤɣɯmdoʁ tɕe ɲɯ-ɤɣɯʂwaŋ}\hspace{5pt}\pcmn{你的衣服和鞋子如果颜色一样的话就很搭}\end{exemple}
\begin{exemple}\pjya{tɯ-rju nɯ ɯ-qhu ɯ-ʁɤri kɯ-ɤɣɯʂwaŋ tu-kɯ-ti ra}\hspace{5pt}\pcmn{话要说得前后一致}\end{exemple}\end{entrée}

\begin{entrée}{aɣɯtɕha}{}{ⓔaɣɯtɕha} 
\classe{vi}  
\grammaire{denom} \paradigme{dir}{tɤ-}
\begin{définition}\pfra{être par paire}\end{définition}
\begin{définition}\pcmn{成双}\end{définition}
\begin{exemple}\pjya{ki tɯ-xtsa ki ɲɯ-ɤɣɯtɕha}\hspace{5pt}\pcmn{这些鞋子是成双的(码是一样的)}\end{exemple}
\begin{exemple}\pjya{ki tɯ-xtsa mɯ-ɲɯ-ɤɣɯtɕha}\hspace{5pt}\pcmn{这些鞋子不是成双的(码不一样)}\end{exemple}
\begin{exemple}\pjya{kɯki tɯ-xtsa ɯ-χtɯ to-k-ɤɣɯtɕha-ci}\hspace{5pt}\pcmn{这些鞋子买了成双的}\end{exemple}
\begin{sous-entrée}{aɣɯtɕhɯtɕha}{ⓔaɣɯtɕhaⓝaɣɯtɕhɯtɕha}
\begin{exemple}\pjya{nɯ-laχtɕha kɯ-ɤɣɯtɕhɯtɕha ʁɟa ɲɯ-ŋu}\hspace{5pt}\pcmn{他们的东西全是成双的}\end{exemple}\relationsémantique{参考}{\lien{ⓔtɯ-tɕhaⓗ1}{tɯ-tɕha₁}}\end{sous-entrée}

\end{entrée}

\begin{entrée}{aɣɯtʂɤm}{}{ⓔaɣɯtʂɤm} 
\classe{vs}  
\grammaire{denom} \paradigme{dir}{thɯ-}
\begin{définition}\pfra{qui produit beaucoup de graisse}\end{définition}
\begin{définition}\pcmn{产的油脂多}\end{définition}
\begin{exemple}\pjya{ki paʁ ki ɲɯ-ɤɣɯtʂɤm}\hspace{5pt}\pcmn{这只猪的油很多}\end{exemple}\end{entrée}

\begin{entrée}{aɣɯtʂɯn}{}{ⓔaɣɯtʂɯn} 
\classe{vi}  
\grammaire{denom} \paradigme{dir}{pɯ-}
\begin{définition}\pfra{être le bienfaiteur}\end{définition}
\begin{définition}\pcmn{有恩}\end{définition}
\begin{exemple}\pjya{jɯfɕɯndʐi wuma pjɤ-k-ɤɣɯtʂɯn-ci}\hspace{5pt}\pcmn{他前几天(对别人)有恩}\end{exemple}
\begin{exemple}\pjya{a-taʁ tɯ-aɣɯtʂɯn}\hspace{5pt}\pcmn{你对我有恩}\end{exemple}\relationsémantique{参考}{\lien{ⓔtɯ-tʂɯn}{tɯ-tʂɯn}}\étymologie{drin}\end{entrée}

\begin{entrée}{aɣɯtɯɣ}{}{ⓔaɣɯtɯɣ} 
\classe{vs}  
\grammaire{denom}
\grammaire{denom} \paradigme{dir}{tɤ-}
\begin{définition}\pfra{vénéneux}\end{définition}
\begin{définition}\pcmn{有毒性}\end{définition}
\begin{exemple}\pjya{to-k-ɤɣɯtɯɣ-ci}\hspace{5pt}\pcmn{毒性比以前大}\end{exemple}
\begin{exemple}\pjya{jiɕqha ɣʑo kɯ kó-wɣ-mtsɯɣ-a tɕe, wuma ɲɯ-ɤɣɯtɯɣ}\hspace{5pt}\pcmn{我被蜜蜂叮了,毒性很大}\end{exemple}
\begin{exemple}\pjya{βɣɤrtshi kɯ kó-wɣ-mtsɯɣ-a tɕe ɲɯ-ɤɣɯtɯɣ}\hspace{5pt}\pcmn{我被蚊子叮了,有毒}\end{exemple}
\begin{exemple}\pjya{jɯfɕɯndʐi smɤn mɯ-pɯ-aɣɯtɯɣ, jɤxtshi pa-ɣɤjɯ tɕe ɲɯ-ɤɣɯtɯɣ}\hspace{5pt}\pcmn{前几天农药毒性不够大,这一次他多加了一些,现在就有毒性了}\end{exemple}\étymologie{dug}\end{entrée}

\begin{entrée}{aɣɯxɕɤt}{}{ⓔaɣɯxɕɤt} 
\classe{vs}  
\grammaire{denom} 
\begin{définition}\pfra{efficace}\end{définition}
\begin{définition}\pcmn{效果好}\end{définition}
\begin{exemple}\pjya{ki smɤn ki ɲɯ-ɤɣɯxɕɤt}\hspace{5pt}\pcmn{这种药很有效}\end{exemple}\end{entrée}

\begin{entrée}{aɣɯzɤrŋɤn}{}{ⓔaɣɯzɤrŋɤn} 
\classe{vi} 
\begin{définition}\pfra{manger tout seul}\end{définition}
\begin{définition}\pcmn{自己一个人吃}\end{définition}
\begin{exemple}\pjya{nɯ-tɯ-ɤɣɯzɤrŋɤn}\hspace{5pt}\pcmn{你自己一个人吃了}\end{exemple}
\begin{exemple}\pjya{nɤ-zda mɯ-kɤ-tɯ-nɤjo, nɯ-tɯ-ɤɣɯzɤrŋɤn}\hspace{5pt}\pcmn{你没有等你的伙伴,你自己一个人吃了}\end{exemple}\relationsémantique{同义词}{\lien{ⓔrɯndzɤqhɤjɯ}{rɯndzɤqhɤjɯ}}\end{entrée}

\begin{entrée}{aɣɯzda}{}{ⓔaɣɯzda} 
\classe{vi}  
\grammaire{denom} \paradigme{dir}{kɤ-}
\begin{définition}\pfra{être ensemble}\end{définition}
\begin{définition}\pcmn{互相陪伴}\end{définition}
\begin{exemple}\pjya{ko-k-ɤɣɯzda-ndʑi-ci}\hspace{5pt}\pcmn{他们俩互相陪伴了}\end{exemple}\end{entrée}

\begin{entrée}{aɣɯzrɯɣ}{}{ⓔaɣɯzrɯɣ} 
\classe{vs}  
\grammaire{denom} 
\begin{définition}\pfra{qui a beaucoup de poux}\end{définition}
\begin{définition}\pcmn{身上虱子多}\end{définition}\relationsémantique{参考}{\lien{ⓔzrɯɣ}{zrɯɣ}}\end{entrée}

\begin{entrée}{aɣɯʑɤzdaŋ}{}{ⓔaɣɯʑɤzdaŋ} 
\classe{vi}  
\grammaire{denom} \paradigme{dir}{tɤ-}
\begin{définition}\pfra{envieux}\end{définition}
\begin{définition}\pcmn{爱嫉妒别人}\end{définition}
\begin{exemple}\pjya{ki ɯ-ʁɤri mɯ-pɯ-aɣɯʑɤzdaŋ ri tham wuma to-k-ɤɣɯʑɤzdaŋ-ci}\hspace{5pt}\pcmn{他以前不妒忌别人,现在就很妒忌}\end{exemple}\étymologie{ʑe.sdaŋ}\end{entrée}

\begin{entrée}{aɣɯʑɯʑat}{}{ⓔaɣɯʑɯʑat} 
\classe{vs} \paradigme{dir}{tɤ-}
\begin{définition}\pfra{espiègle, coquin}\end{définition}
\begin{définition}\pcmn{调皮}\end{définition}
\begin{exemple}\pjya{a-ʁi aɣɯʑɯʑat}\hspace{5pt}\pcmn{我的弟弟很调皮}\end{exemple}\end{entrée}

\begin{entrée}{aj}{}{ⓔaj} 
\classe{pro} 
\begin{définition}\pfra{moi}\end{définition}
\begin{définition}\pcmn{我}\end{définition}\relationsémantique{参考}{\lien{ⓔaʑo}{aʑo}}\end{entrée}

\begin{entrée}{aja}{}{ⓔaja}\relationsémantique{参考}{\lien{ⓔjaⓗ1}{ja₁}}\end{entrée}

\begin{entrée}{ajɤr}{}{ⓔajɤr} 
\classe{vi} \paradigme{dir}{nɯ-}\paradigme{dir}{nɯ-}
\begin{définition}\pfra{en biais, de travers}\end{définition}
\begin{définition}\pcmn{歪,偏}\end{définition}
\begin{exemple}\pjya{ɲɤ-k-ɤjɤr-ci}\hspace{5pt}\pcmn{以前不歪,现在歪了}\end{exemple}
\begin{exemple}\pjya{@luyinji nɯ ɲɯ-ɤjɤr}\hspace{5pt}\pcmn{录音机放歪了}\end{exemple}
\begin{exemple}\pjya{nɤkɤcu @luyinji nɯ kɤ-ta ɲɤ-tɯ-sɤjɤr}\hspace{5pt}\pcmn{你把录音机放歪了}\end{exemple}
\begin{sous-entrée}{sɤjɤr}{ⓔajɤrⓝsɤjɤr} 
\classe{vt} \end{sous-entrée}

\begin{sous-entrée}{ʑɣɤsɤjɤr}{ⓔajɤrⓝʑɣɤsɤjɤr} 
\classe{vt}  
\grammaire{refl}
\grammaire{caus} 
\begin{définition}\pfra{se mettre de travers}\end{définition}
\begin{définition}\pcmn{斜着身子}\end{définition}
\begin{exemple}\pjya{ɯ-stu kɤ-ɤmdzɯ, ma-nɯ-tɯ-ʑɣɤsɤjɤr ʑo}\hspace{5pt}\pcmn{你坐直,不要斜着坐}\end{exemple}
\begin{exemple}\pjya{ɯʑo tɤ-ŋke tɕe, ɲɯ-ʑɣɤsɤjɤr ʑo ɲɯ-ŋu}\hspace{5pt}\pcmn{他走路的时候是斜着身子的}\end{exemple}\end{sous-entrée}

\end{entrée}

\begin{entrée}{ajɤzjɯ}{}{ⓔajɤzjɯ} 
\classe{vi} \paradigme{dir}{tɤ-}\paradigme{dir}{tɤ-}
\begin{définition}\pfra{être rajouté}\end{définition}
\begin{définition}\pcmn{增加一点}\end{définition}
\begin{exemple}\pjya{kɯki mɤʑɯ tɤ-ndɤm tɕe a-tɤ-ɤjɤzjɯ}\hspace{5pt}\pcmn{多拿一点}\end{exemple}
\begin{sous-entrée}{sɤjɤzjɯ}{ⓔajɤzjɯⓝsɤjɤzjɯ} 
\classe{vt} \end{sous-entrée}

\sens{1}
\begin{définition}\pfra{rajouter}\end{définition}
\begin{définition}\pcmn{加一点;填充一点}\end{définition}
\begin{exemple}\pjya{to-sɤjɤzjɯ}\hspace{5pt}\pcmn{他多加了一点}\end{exemple}\sens{2}
\begin{définition}\pfra{rajouter (des ennuis)}\end{définition}
\begin{définition}\pcmn{添麻烦;火上加油}\end{définition}\end{entrée}

\begin{entrée}{ajŋgɯɣ}{}{ⓔajŋgɯɣ} 
\classe{vs} 
\begin{définition}\pfra{courbé}\end{définition}
\begin{définition}\pcmn{弯着}\end{définition}
\begin{exemple}\pjya{ɯ-phoŋbu ɲɯ-ɤjŋgɯɣ}\hspace{5pt}\pcmn{他的身子是弯着的}\end{exemple}\relationsémantique{参考}{\lien{ⓔazgrɯ}{azgrɯ}}\relationsémantique{参考}{\lien{ⓔajʁu}{ajʁu}}\end{entrée}

\begin{entrée}{ajpomxtshɯm}{}{ⓔajpomxtshɯm} 
\classe{vs} \paradigme{dir}{thɯ-}\paradigme{dir}{thɯ-}
\begin{définition}\pfra{d'une grosseur inégale}\end{définition}
\begin{définition}\pcmn{粗细不一}\end{définition}
\begin{exemple}\pjya{tɤ-ri ɲɯ-ɤjpomxtshɯm}\hspace{5pt}\pcmn{线有一头粗另一头细}\end{exemple}
\begin{exemple}\pjya{thɯ-ajpomxtshɯm}\hspace{5pt}\pcmn{变得粗细不一了}\end{exemple}
\begin{exemple}\pjya{chɤ-sɤjpomxtshɯm-a}\hspace{5pt}\pcmn{我弄得不粗细不一了}\end{exemple}\relationsémantique{参考}{\lien{ⓔjpum}{jpum}}\relationsémantique{参考}{\lien{ⓔxtshɯm}{xtshɯm}}\relationsémantique{参考}{\lien{ⓔjpumxtshɯm}{jpumxtshɯm}}
\begin{sous-entrée}{sɤjpomxtshɯm}{ⓔajpomxtshɯmⓝsɤjpomxtshɯm} 
\classe{vt} \end{sous-entrée}

\end{entrée}

\begin{entrée}{ajʁu}{}{ⓔajʁu} 
\classe{vi} \paradigme{dir}{tɤ-}\paradigme{dir}{nɯ-}\paradigme{dir}{tɤ-}
\begin{définition}\pfra{courbé}\end{définition}
\begin{définition}\pcmn{弯(路、植物、人的四肢等)}\end{définition}
\begin{définition}\pfra{courber}\end{définition}
\begin{définition}\pcmn{弄弯}\end{définition}
\begin{exemple}\pjya{ki si ki ɲɯ-ɤjʁu}\hspace{5pt}\pcmn{这棵树是弯的}\end{exemple}
\begin{exemple}\pjya{kɯki @dianxian ki ɲɯ-ɤjʁu}\hspace{5pt}\pcmn{这根电线是弯的}\end{exemple}
\begin{exemple}\pjya{kɯki laʁjɯɣ ki kutɕu pɯ-ata tɕe to-k-ɤjʁu-ci}\hspace{5pt}\pcmn{这根棍子放在这里就变弯了}\end{exemple}
\begin{exemple}\pjya{romɲa kutɕu pɯ-ata ri ɲɤ-k-ɤjʁu-ci}\hspace{5pt}\pcmn{梁放在这里就变弯了}\end{exemple}
\begin{exemple}\pjya{ɯ-mi ki kɯ-fse to-sɤjʁu}\hspace{5pt}\pcmn{他这样弓了腿}\end{exemple}
\begin{exemple}\pjya{si ta-sɤjʁu}\hspace{5pt}\pcmn{他把木头弄弯了}\end{exemple}
\begin{exemple}\pjya{ɕom ta-sɤjʁu}\hspace{5pt}\pcmn{他把铁弄弯了}\end{exemple}\relationsémantique{参考}{\lien{ⓔamɤʁu}{amɤʁu}}\relationsémantique{同义词}{\lien{ⓔkɤɣ}{kɤɣ}}
\begin{sous-entrée}{sɤjʁu}{ⓔajʁuⓝsɤjʁu} 
\classe{vt} \end{sous-entrée}

\end{entrée}

\begin{entrée}{ajtshi}{}{ⓔajtshi}\relationsémantique{参考}{\lien{ⓔjtshi}{jtshi}}\end{entrée}

\begin{entrée}{ajtɯ}{}{ⓔajtɯ} 
\classe{vi} \paradigme{dir}{tɤ-}\paradigme{dir}{tɤ-}
\begin{définition}\pfra{s'accumuler}\end{définition}
\begin{définition}\pcmn{累积}\end{définition}
\begin{définition}\pfra{accumuler}\end{définition}
\begin{définition}\pcmn{积攒}\end{définition}
\begin{exemple}\pjya{tɯ-ci nɯ tɤ-ajtɯ}\hspace{5pt}\pcmn{水积起来了}\end{exemple}
\begin{exemple}\pjya{tɤ-scoz nɯ tɤ-rʑaʁ kɯ-rɲɟi tu-kɯ-stu pjɯ́-wɣ-βzjoz tɕe, ku-ojtɯ kɯ-ra ɕti}\hspace{5pt}\pcmn{文化知识是要长期积累起来的}\end{exemple}
\begin{exemple}\pjya{laχtɕha khro to-k-ɤjtɯ-ci}\hspace{5pt}\pcmn{积累了很多东西}\end{exemple}
\begin{exemple}\pjya{kɤ-ndza mɯma ajtɯ}\hspace{5pt}\pcmn{除了食物,什么都可以积累}\end{exemple}
\begin{exemple}\pjya{staχpɯ tɤ-nɯrdoʁ-a, tɤ-sɤjtɯ-t-a}\hspace{5pt}\pcmn{我把豌豆捡了,积攒起来了}\end{exemple}
\begin{exemple}\pjya{a-zda ra kɯ nɯ-kɤ-mbi ra tɤ-nɯ-ndza-nɯ, aʑo nɯ-nɯ-sɤjtɯ-t-a}\hspace{5pt}\pcmn{我的伙计们自己吃别人给的东西,我就全部积攒起来了}\end{exemple}
\begin{exemple}\pjya{stoʁ tɤ-sɤjtɯ-t-a}\hspace{5pt}\pcmn{我攒了胡豆}\end{exemple}
\begin{exemple}\pjya{tɯ-ci tɤ-sɤjtɯ-t-a}\hspace{5pt}\pcmn{我积累了水}\end{exemple}
\begin{exemple}\pjya{kɤ-ndza tɤ-sɤjtɯ-t-a}\hspace{5pt}\pcmn{我积累了食物}\end{exemple}
\begin{sous-entrée}{sɤjtɯ}{ⓔajtɯⓝsɤjtɯ} 
\classe{vt} \end{sous-entrée}

\begin{sous-entrée}{ɣɤjtɯ}{ⓔajtɯⓝɣɤjtɯ} 
\classe{vi} 
\begin{définition}\pfra{s'accumuler vite, facilement}\end{définition}
\begin{définition}\pcmn{容易积累}\end{définition}
\begin{exemple}\pjya{laχtɕha ra ɯ-grɤl kɯ-me ma-tɤ́-wɣ-ntɕhoz tɕe, tɕe ɣɤjtɯ}\hspace{5pt}\pcmn{东西不要乱用,就积得起来}\end{exemple}\relationsémantique{参考}{\lien{ⓔndɯⓗ2}{ndɯ₂}}\end{sous-entrée}

\end{entrée}

\begin{entrée}{ajχoʁ}{}{ⓔajχoʁ} 
\classe{vi} \paradigme{dir}{kɤ-}
\begin{définition}\pfra{avoir le ventre plat}\end{définition}
\begin{définition}\pcmn{肚子瘪}\end{définition}
\begin{exemple}\pjya{ɯ-xtu ko-k-ɤjχoʁ-ci}\hspace{5pt}\pcmn{他肚子瘪了}\end{exemple}
\begin{exemple}\pjya{tɤ-fkɯm nɯ ɲɯ-ɤjχoʁ}\hspace{5pt}\pcmn{口袋是瘪的}\end{exemple}\relationsémantique{参考}{\lien{ⓔɲchoʁ}{ɲchoʁ}}\end{entrée}

\begin{entrée}{akarɯ}{}{ⓔakarɯ} 
\classe{n} 
\begin{définition}\pfra{origan}\end{définition}
\begin{définition}\pcmn{牛至}\end{définition}
\begin{exemple}\pjya{akarɯ nɯ sɯjno kɯ-xtɕi ci ŋu, ɯ-ru kɯ-xtshɯ-xtshɯm kɯ-ɣɯrni ci ŋu, ʁnɯ-tɣa jamar ma mɤ-mbro, ɯ-jwaʁ kɯ-ɤrtɯm, kɯ-rɲɟi tsa ci ŋu, ɯ-di mnɤm, ɯ-mɯntoʁ kɯ-ɣɯrni ɯ-ŋgɯ kɯ-wɣrum tsa ci ŋu, ɯ-zrɤm kɯ-xtɕɯ-xtɕi ma me, ɯʑo smɤn ɯ-ŋgɯ kɤ-lɤt ɲɯ-sna.}\hspace{5pt}\pcmn{牛至是一种小型植物,茎很细,呈红色,只有约两拃高,有椭圆形小叶,花红里透白,有香味,根很小。可入药。}\end{exemple}\end{entrée}

\begin{entrée}{akɤlɤt}{}{ⓔakɤlɤt} 
\classe{vi} \paradigme{dir}{thɯ-}\paradigme{dir}{nɯ-}\paradigme{dir}{nɯ-}\paradigme{dir}{pɯ-}
\begin{définition}\pfra{se détacher}\end{définition}
\begin{définition}\pcmn{分裂;脱掉}\end{définition}
\begin{définition}\pfra{séparer}\end{définition}
\begin{définition}\pcmn{分开;断裂(分成两截)}\end{définition}
\begin{exemple}\pjya{ɯ-rpaʁ chɤ-k-ɤkɤlɤt-ci}\hspace{5pt}\pcmn{他肩膀脱臼了}\end{exemple}
\begin{exemple}\pjya{tɯ-jaʁ ɯ-βzɯr nɯ tɕu ``hu" tu-kɯ-ti qhe tɕe sporɟɤlɯla ɯ-taʁ tú-wɣ-lɤt qhe tɕe a-pɯ-tɯɣ ʑo qhe tɕe ɯʑo ɲɯ-ɤkɤlɤt ɲɯ-ŋu}\hspace{5pt}\pcmn{在手边“呒呒”地吹一下的时候,呼出的气如果吹到四脚蛇的尾巴上,尾巴就会脱节}\end{exemple}
\begin{exemple}\pjya{qaʁ nɯ ɯ-jɯ cho-χɕoʁ qhe tɕe ɲɯ-ɤkɤlɤt ɕti}\hspace{5pt}\pcmn{他拿锄头的把子抽了一下就脱落了}\end{exemple}
\begin{exemple}\pjya{nɤ-ŋga kɤ-tʂɯβ ma akɤlɤt ɲɯ-ŋu}\hspace{5pt}\pcmn{你把衣服补一下,不然就会脱成两半}\end{exemple}
\begin{exemple}\pjya{nɤki laχtɕha nɯ ma-pɯ-tɯ-sɤkɤlɤt}\hspace{5pt}\pcmn{你不要把那个东西弄断}\end{exemple}
\begin{exemple}\pjya{rtɤltɕaʁ ci pjɤ-rtɤβ tɕe ɲɤ-sɤkɤlɤt}\hspace{5pt}\pcmn{他打了马鞭子把他们俩分开了}\end{exemple}
\begin{exemple}\pjya{sporɟɤlɯla nɯ-sɤkɤlat-a}\hspace{5pt}\pcmn{我把四脚蛇弄脱节了}\end{exemple}
\begin{sous-entrée}{sɤkɤlɤt}{ⓔakɤlɤtⓝsɤkɤlɤt} 
\classe{vt} \end{sous-entrée}

\end{entrée}

\begin{entrée}{akɤmtɕoʁ}{}{ⓔakɤmtɕoʁ} 
\classe{vs}  
\grammaire{incorp} \paradigme{dir}{lɤ-}
\begin{définition}\pfra{pointu (sur le sommet)}\end{définition}
\begin{définition}\pcmn{(顶部、尖头)很尖}\end{définition}
\begin{exemple}\pjya{ɯ-ku ɲɯ-ɤkɤmtɕoʁ}\hspace{5pt}\pcmn{尖头很尖}\end{exemple}
\begin{exemple}\pjya{lo-k-ɤkɤmtɕoʁ-ci}\hspace{5pt}\pcmn{(用了很久),顶部就变得很尖}\end{exemple}
\begin{exemple}\pjya{mbrɯtɕɯ ɲɤ-sa tɕe lo-k-ɤkɤmtɕoʁ-ci}\hspace{5pt}\pcmn{刀子磨损后,顶部变得很尖}\end{exemple}\relationsémantique{参考}{\lien{ⓔtɯ-ku}{tɯ-ku}}\relationsémantique{参考}{\lien{ⓔamtɕoʁ}{amtɕoʁ}}\end{entrée}

\begin{entrée}{akɤtɕɤβ}{}{ⓔakɤtɕɤβ} 
\classe{vs} \paradigme{dir}{nɯ-}\paradigme{dir}{nɯ-}
\begin{définition}\pfra{être croisé}\end{définition}
\begin{définition}\pcmn{交叉}\end{définition}
\begin{définition}\pfra{croiser}\end{définition}
\begin{exemple}\pjya{jiʑo kɯrɯ ɣɯ ji-ŋga nɯ ɯ-naŋma cho ɯ-pɕima ɲɯ́-wɣ-sɤkɤtɕɤβ ra}\hspace{5pt}\pcmn{我们藏装穿的时候要两边交错着}\end{exemple}\relationsémantique{参考}{\lien{ⓔaqɤtʂha}{aqɤtʂha}}
\begin{sous-entrée}{sɤkɤtɕɤβ}{ⓔakɤtɕɤβⓝsɤkɤtɕɤβ} 
\classe{vt} \end{sous-entrée}

\end{entrée}

\begin{entrée}{akhu}{}{ⓔakhu} 
\classe{vi} \paradigme{dir}{jɤ-}
\begin{définition}\pfra{appeler}\end{définition}
\begin{définition}\pcmn{叫(某人)}\end{définition}
\begin{exemple}\pjya{atu tɯrme ci ɲɯ-ɤkhu}\hspace{5pt}\pcmn{上头有个人在叫(你)}\end{exemple}
\begin{exemple}\pjya{aʑo a-ɕki ɲɯ-ɤkhu}\hspace{5pt}\pcmn{他在叫我}\end{exemple}
\begin{exemple}\pjya{chɤ-k-ɤkhu-ci}\hspace{5pt}\pcmn{他从里面(往外面)叫了}\end{exemple}\relationsémantique{参考}{\lien{ⓔakhɤzŋga}{akhɤzŋga}}\relationsémantique{参考}{\lien{ⓔnɤkhɤzŋga}{nɤkhɤzŋga}}\end{entrée}

\begin{entrée}{akhar}{}{ⓔakhar} 
\classe{vi} \paradigme{dir}{lɤ-}\paradigme{dir}{pɯ-}
\begin{définition}\pfra{se mettre autour}\end{définition}
\begin{définition}\pcmn{围着(坐、站)}\end{définition}
\begin{exemple}\pjya{ɲɯ-rɯndzɤtshi-nɯ pjɤ-k-ɤkhar-nɯ-ci}\hspace{5pt}\pcmn{他们围着吃饭了}\end{exemple}\relationsémantique{参考}{\lien{ⓔsɤkhar}{sɤkhar}}\relationsémantique{参考}{\lien{ⓔnɤkhar}{nɤkhar}}\end{entrée}

\begin{entrée}{akhɤɟor}{}{ⓔakhɤɟor} 
\classe{vs} 
\begin{définition}\pfra{ni rond ni carré}\end{définition}
\begin{définition}\pcmn{不圆不方}\end{définition}\relationsémantique{同义词}{\lien{ⓔamkhɤrju}{amkhɤrju}}\end{entrée}

\begin{entrée}{akhɤzŋga}{}{ⓔakhɤzŋga} 
\classe{vi} \paradigme{dir}{nɯ-}\paradigme{dir}{jɤ-}
\begin{définition}\pfra{crier}\end{définition}
\begin{définition}\pcmn{喊}\end{définition}
\begin{exemple}\pjya{atu tɯrme ci ɣɤʑu ɲɯ-ɤkhɤzŋga}\hspace{5pt}\pcmn{上头有个人在喊}\end{exemple}
\begin{exemple}\pjya{staʁthɤr ɲɤ-k-ɤkhɤzŋga-ci}\hspace{5pt}\pcmn{斯达塔尔喊了}\end{exemple}\relationsémantique{参考}{\lien{ⓔakhu}{akhu}}\relationsémantique{参考}{\lien{ⓔnɤkhɤzŋga}{nɤkhɤzŋga}}\end{entrée}

\begin{entrée}{akhi}{}{ⓔakhi} 
\classe{intj} 
\begin{définition}\pfra{exprime que le locuteur estime avoir de la chance}\end{définition}
\begin{définition}\pcmn{表示自己很幸运}\end{définition}
\begin{exemple}\pjya{akhi ma pɯ-ŋgrɯ!}\hspace{5pt}\pcmn{很幸运,成功了}\end{exemple}\end{entrée}

\begin{entrée}{akhra}{}{ⓔakhra} 
\classe{vi}  
\grammaire{denom} \paradigme{dir}{tɤ-}
\begin{définition}\pfra{bariolé}\end{définition}
\begin{définition}\pcmn{花的(颜色);有花纹的}\end{définition}
\begin{exemple}\pjya{jiɕqha tɯ-ŋga nɯ ɲɯ-ɤkhra}\hspace{5pt}\pcmn{这件衣服是花的}\end{exemple}
\begin{exemple}\pjya{jiɕqha nɯŋa nɯ kɯ-ɤkhra ɲɯ-ŋu}\hspace{5pt}\pcmn{这头牛是花的}\end{exemple}
\begin{exemple}\pjya{ɯ-rŋa to-k-ɤkhra-ci}\hspace{5pt}\pcmn{他脸花了}\end{exemple}
\begin{exemple}\pjya{ɯ-skhrɯ mɯ́j-βdi tɕe ɯ-rŋa to-k-ɤkhra-ci}\hspace{5pt}\pcmn{她怀孕,脸花了(有斑纹)}\end{exemple}\relationsémantique{参考}{\lien{}{akhrala}}\étymologie{kʰra}\end{entrée}

\begin{entrée}{akhrɤla}{}{ⓔakhrɤla} 
\classe{vs} 
\begin{définition}\pfra{bariolé}\end{définition}
\begin{définition}\pcmn{花的(颜色)}\end{définition}\end{entrée}

\begin{entrée}{akhrɤlɯla/\variante{akhrɤlɯlu}}{}{ⓔakhrɤlɯla} 
\classe{vs} 
\begin{définition}\pfra{bariolé, aux couleurs bigarrées et voyantes}\end{définition}
\begin{définition}\pcmn{花花绿绿}\end{définition}
\begin{exemple}\pjya{ɯ-ŋga ɲɯ-ɤkhrɤlɯla ʑo tɕe mɯ́j-mpɕɤr}\hspace{5pt}\pcmn{他的衣服花花绿绿,不好看}\end{exemple}\relationsémantique{参考}{\lien{ⓔakhra}{akhra}}\end{entrée}

\begin{entrée}{akundi}{}{ⓔakundi} 
\classe{vi} \paradigme{dir}{kɤ-}\paradigme{dir}{kɤ-}
\begin{définition}\pfra{être aligné de gauche à droite}\end{définition}
\begin{définition}\pcmn{左右排列的}\end{définition}
\begin{définition}\pfra{aligner de gauche à droite}\end{définition}
\begin{définition}\pcmn{排成一左一右}\end{définition}
\begin{exemple}\pjya{tɕiʑo tɕi-kha ku-okundi ŋu}\hspace{5pt}\pcmn{我们的家,一个在左边一个在右边}\end{exemple}
\begin{exemple}\pjya{laχtɕha kɤ-ta kɤ-sɤkundi-t-a}\hspace{5pt}\pcmn{我把东西排成一左一右}\end{exemple}
\begin{sous-entrée}{sɤkundi}{ⓔakundiⓝsɤkundi} 
\classe{vt} \end{sous-entrée}

\end{entrée}

\begin{entrée}{akɯ}{}{ⓔakɯ} 
\classe{adv} 
\begin{définition}\pfra{à l'est}\end{définition}
\begin{définition}\pcmn{在东边}\end{définition}\relationsémantique{参考}{\lien{ⓔtɕɤkɯ}{tɕɤkɯ}}\end{entrée}

\begin{entrée}{akɯchoʁle}{}{ⓔakɯchoʁle} 
\classe{n} 
\begin{définition}\pfra{vent du nord}\end{définition}
\begin{définition}\pcmn{北风}\end{définition}\relationsémantique{参考}{\lien{ⓔqale}{qale}}\relationsémantique{参考}{\lien{ⓔandichoʁle}{andichoʁle}}\end{entrée}

\begin{entrée}{akɯzgumba}{}{ⓔakɯzgumba} 
\classe{n} 
\begin{définition}\pfra{ver blanc}\end{définition}
\begin{définition}\pcmn{蛴螬}\end{définition}\end{entrée}

\begin{entrée}{ala}{}{ⓔala} 
\classe{vi} \paradigme{dir}{tɤ-}\paradigme{dir}{nɯ-}\paradigme{dir}{kɤ-}\paradigme{dir}{tɤ-}
\begin{définition}\pfra{bouillir}\end{définition}
\begin{définition}\pcmn{开(水)、沸腾}\end{définition}
\begin{définition}\pfra{faire bouillir}\end{définition}
\begin{définition}\pcmn{烧开}\end{définition}
\begin{exemple}\pjya{tɯ-ci to-k-ɤla-ci}\hspace{5pt}\pcmn{水开了}\end{exemple}
\begin{exemple}\pjya{tɯtshi ka-sɤla}\hspace{5pt}\pcmn{他煲了粥}\end{exemple}
\begin{exemple}\pjya{tʂha ta-sɤla}\hspace{5pt}\pcmn{他熬了茶}\end{exemple}
\begin{sous-entrée}{sɤla}{ⓔalaⓝsɤla} 
\classe{vt}  
\grammaire{caus} \end{sous-entrée}

\end{entrée}

\begin{entrée}{alala}{}{ⓔalala} 
\classe{adv} 
\begin{définition}\pfra{bien sûr}\end{définition}
\begin{définition}\pcmn{理所当然}\end{définition}
\begin{exemple}\pjya{alala, ʑatsa ɣi-a ɕti}\hspace{5pt}\pcmn{当然,我可以早点来}\end{exemple}
\begin{sous-entrée}{ʁo alala ri}{ⓔalalaⓝʁo alala ri} 
\classe{cnj} 
\begin{définition}\pfra{non seulement ... mais}\end{définition}
\begin{définition}\pcmn{不光是……而且}\end{définition}
\begin{exemple}\pjya{nɤʑo nɤ-ŋga ʁo alala ri kɯmaʁ kɯ-tu nɯra kɯnɤ tɤ-ndɤm}\hspace{5pt}\pcmn{不光是你的衣服,其他所有的衣服都要带上}\end{exemple}\end{sous-entrée}

\end{entrée}

\begin{entrée}{alɤɣɯ}{}{ⓔalɤɣɯ} 
\classe{vi} \paradigme{dir}{nɯ-}
\begin{définition}\pfra{être connecté}\end{définition}
\begin{définition}\pcmn{连在一起}\end{définition}
\begin{exemple}\pjya{jiɕqha laχtɕha nɯ ɲɯ-ɤlɤɣɯ-ndʑi}\hspace{5pt}\pcmn{这两个东西是连在一起的}\end{exemple}
\begin{exemple}\pjya{tɤ-ri nɯ ɲɤ-k-ɤlɤɣɯ-ndʑi-ci tɕe nɯ-rla-t-a}\hspace{5pt}\pcmn{两根线缠在一起了,我把它们解开了}\end{exemple}\relationsémantique{参考}{\lien{ⓔsɤlɤɣɯ}{sɤlɤɣɯ}}\end{entrée}

\begin{entrée}{alɤt}{}{ⓔalɤt}\relationsémantique{参考}{\lien{ⓔlɤtⓗ1}{lɤt₁}}\end{entrée}

\begin{entrée}{aluj}{}{ⓔaluj}\relationsémantique{参考}{\lien{ⓔluj}{luj}}\end{entrée}

\begin{entrée}{alɟɣi}{}{ⓔalɟɣi} 
\classe{vi} 
\begin{définition}\pfra{bouger (dent)}\end{définition}
\begin{définition}\pcmn{摇动(牙齿)}\end{définition}
\begin{exemple}\pjya{a-ɕɣa ɲɤ-k-ɤlɟɣi-ci tɕe ɲɯ-mŋɤm}\hspace{5pt}\pcmn{我的牙齿松动了,很痛}\end{exemple}\end{entrée}

\begin{entrée}{alo}{}{ⓔalo} 
\classe{adv} 
\begin{définition}\pfra{en amont}\end{définition}
\begin{définition}\pcmn{上游}\end{définition}\relationsémantique{参考}{\lien{ⓔtɕɤlo}{tɕɤlo}}\end{entrée}

\begin{entrée}{alothi}{}{ⓔalothi} 
\classe{vi} \paradigme{dir}{lɤ-}\paradigme{dir}{lɤ-}
\begin{définition}\pfra{être aligné d'amont en aval}\end{définition}
\begin{définition}\pcmn{一个在上游一个在下游}\end{définition}
\begin{définition}\pfra{aligner de l'amont vers l'aval}\end{définition}
\begin{définition}\pcmn{排成一个在上游一个在下游}\end{définition}
\begin{exemple}\pjya{tɕi-kha lu-olothi ŋu}\hspace{5pt}\pcmn{我们的家一个在上游一个在下游}\end{exemple}
\begin{exemple}\pjya{ndʑiʑo lɤ-ɤlothi-ndʑi tɕe tɤ-ndzur-ndʑi}\hspace{5pt}\pcmn{你们俩上下站着}\end{exemple}
\begin{exemple}\pjya{laχtɕha kɤ-ta lɤ-sɤlothi-t-a pɯ-ra}\hspace{5pt}\pcmn{我只好把东西排得一上一下}\end{exemple}
\begin{sous-entrée}{sɤlothi}{ⓔalothiⓝsɤlothi} 
\classe{vt} \end{sous-entrée}

\end{entrée}

\begin{entrée}{alpɯm}{}{ⓔalpɯm} 
\classe{vs} 
\begin{définition}\pfra{en commun}\end{définition}
\begin{définition}\pcmn{共同的}\end{définition}
\begin{exemple}\pjya{kɯki laχtɕha ki alpɯm ɕti}\hspace{5pt}\pcmn{这个东西是(我们)共同拥有的}\end{exemple}\relationsémantique{同义词}{\lien{ⓔangɯt}{angɯt}}
\begin{sous-entrée}{sɤlpɯm}{ⓔalpɯmⓝsɤlpɯm} 
\classe{vt} 
\begin{définition}\pfra{mettre ensemble}\end{définition}
\begin{définition}\pcmn{装在一起}\end{définition}
\begin{exemple}\pjya{ji-laχtɕha ra thɯ-sɤlpɯm}\hspace{5pt}\pcmn{把我们的东西都装在一起}\end{exemple}\end{sous-entrée}

\end{entrée}

\begin{entrée}{alɯlɤt}{}{ⓔalɯlɤt} 
\classe{vi}  
\grammaire{recip} \paradigme{dir}{tɤ-}
\begin{définition}\pfra{se battre}\end{définition}
\begin{définition}\pcmn{打架}\end{définition}
\begin{exemple}\pjya{ʑɤni alɯlɤt-ndʑi ɲɯ-ŋu tɕe tɤ-βri-t-a}\hspace{5pt}\pcmn{他们俩差一点打架了,我保护了他(把他们俩劝开了)}\end{exemple}
\begin{exemple}\pjya{to-k-ɤnɯmqaj-ndʑi-ci tɕe tu-olɯlɤt-ndʑi}\hspace{5pt}\pcmn{他们俩吵架了,还打了起来}\end{exemple}
\begin{exemple}\pjya{to-k-ɤlɯlɤt-ndʑi-ci}\hspace{5pt}\pcmn{他们俩打架了}\end{exemple}
\begin{exemple}\pjya{nɯ-ɕki tɤ-alɯlat-a}\hspace{5pt}\pcmn{我跟他们打架了}\end{exemple}\relationsémantique{参考}{\lien{ⓔlɤtⓗ1}{lɤt₁}}\end{entrée}

\begin{entrée}{alɯlju}{}{ⓔalɯlju} 
\classe{vs} \paradigme{dir}{thɯ-}
\begin{définition}\pfra{cylindrique}\end{définition}
\begin{définition}\pcmn{圆柱形}\end{définition}
\begin{exemple}\pjya{kɯ-ɤlɯlju ɲɯ-ŋu}\hspace{5pt}\pcmn{(这只笔)是圆柱形的}\end{exemple}\end{entrée}

\begin{entrée}{alxaj}{}{ⓔalxaj} 
\classe{vi} \paradigme{dir}{nɯ-}
\begin{définition}\pfra{négligé (habits)}\end{définition}
\begin{définition}\pcmn{衣冠不整}\end{définition}
\begin{exemple}\pjya{ɯ-ro ɯ-ŋga ɲɯ-ɤlxɤj}\hspace{5pt}\pcmn{他的上衣是敞开着的,没有扣好}\end{exemple}
\begin{exemple}\pjya{jiɕqha ɯ-ŋga nɯ ɲɤ-k-ɤlxɤj-ci}\hspace{5pt}\pcmn{他的衣服乱了(他没有发觉)}\end{exemple}\end{entrée}

\begin{entrée}{aɬɯt}{}{ⓔaɬɯt} 
\classe{vi} \paradigme{dir}{nɯ-}\paradigme{dir}{nɯ-}\paradigme{dir}{thɯ-}
\begin{définition}\pfra{dans le désordre (fil, pelote de laine)}\end{définition}
\begin{définition}\pcmn{乱(线)}\end{définition}
\begin{définition}\pfra{mettre dans le désordre}\end{définition}
\begin{définition}\pcmn{弄乱(线)}\end{définition}
\begin{exemple}\pjya{kɤtɯm pjɤ-ɴɢia tɕe ɲɯ-ɤɬɯt}\hspace{5pt}\pcmn{线团散了就乱了}\end{exemple}
\begin{exemple}\pjya{kɤtɯm ɲɤ-k-ɤɬɯt-ci}\hspace{5pt}\pcmn{线团散了}\end{exemple}
\begin{exemple}\pjya{tɤ-ri ɲɤ-sɤɬɯt}\hspace{5pt}\pcmn{他把线弄乱了}\end{exemple}\relationsémantique{同义词}{\lien{ⓔatʂoʁloʁ}{atʂoʁloʁ}}\relationsémantique{同义词}{\lien{ⓔatʂoʁloʁⓝsɤtʂoʁloʁ}{sɤtʂoʁloʁ}}
\begin{sous-entrée}{sɤɬɯt}{ⓔaɬɯtⓝsɤɬɯt} 
\classe{vt}  
\grammaire{caus} \end{sous-entrée}

\end{entrée}

\begin{entrée}{ama}{}{ⓔama} 
\classe{intj} 
\begin{définition}\pfra{exprime la surprise}\end{définition}
\begin{définition}\pcmn{表示惊奇}\end{définition}
\begin{exemple}\pjya{ama, ɯ-tɯ-mpɕɤr nɯ!}\hspace{5pt}\pcmn{哎呀,多么漂亮!}\end{exemple}\end{entrée}

\begin{entrée}{amar}{}{ⓔamar}\relationsémantique{参考}{\lien{ⓔmar}{mar}}\end{entrée}

\begin{entrée}{amɤʁu}{}{ⓔamɤʁu} 
\classe{vs}  
\grammaire{incorp} 
\begin{définition}\pfra{souffrir de rachitisme, avoir les jambes courbées}\end{définition}
\begin{définition}\pcmn{患佝偻病(弯着脚)}\end{définition}\relationsémantique{参考}{\lien{ⓔtɯ-mi}{tɯ-mi}}\relationsémantique{参考}{\lien{}{jʁu}}\end{entrée}

\begin{entrée}{ambɤldʑɤm}{}{ⓔambɤldʑɤm} 
\classe{vs} \paradigme{dir}{nɯ-}\paradigme{dir}{nɯ-}
\begin{définition}\pfra{au caractère doux et calme}\end{définition}
\begin{définition}\pcmn{性格文静}\end{définition}
\begin{définition}\pfra{régler les contentieux entre personnes}\end{définition}
\begin{définition}\pcmn{解决人之间的矛盾}\end{définition}
\begin{exemple}\pjya{ɯʑo ʁo ɲɯ-ɤmbɤldʑɤm ɕti}\hspace{5pt}\pcmn{他倒是性格平静的人}\end{exemple}
\begin{exemple}\pjya{jiɕqha nɯ ɯ-kɤ-qha ra rkɯn tɕe, kɯ-ɤmbɤldʑɤm ci ɕti}\hspace{5pt}\pcmn{那个人很少生气,性格文静}\end{exemple}
\begin{exemple}\pjya{ʑɤni ɲɯ-ɤlɯlɤt-ndʑi ri, nɯ-nɯkhɤda-t-a-ndʑi tɕe, nɯ-sɤmbɤldʑam-a}\hspace{5pt}\pcmn{他们在打架,我劝了他们,解决了他们的矛盾}\end{exemple}
\begin{sous-entrée}{sɤmbɤldʑɤm}{ⓔambɤldʑɤmⓝsɤmbɤldʑɤm} 
\classe{vt} \end{sous-entrée}

\end{entrée}

\begin{entrée}{ambi}{}{ⓔambi}\relationsémantique{参考}{\lien{ⓔmbi}{mbi}}\end{entrée}

\begin{entrée}{amboʁ}{}{ⓔamboʁ} 
\classe{vi} \paradigme{dir}{nɯ-}
\begin{définition}\pfra{exploser}\end{définition}
\begin{définition}\pcmn{爆炸;爆裂}\end{définition}
\begin{exemple}\pjya{ɕɤmɯɣdɯ nɯ-amboʁ}\hspace{5pt}\pcmn{枪爆炸了}\end{exemple}
\begin{exemple}\pjya{qandʑi ɲɯ-ɤmboʁ ŋu}\hspace{5pt}\pcmn{子弹可能会爆炸}\end{exemple}
\begin{exemple}\pjya{@dahuoji ɲɯ-ɤmboʁ}\hspace{5pt}\pcmn{打火机爆炸了}\end{exemple}
\begin{exemple}\pjya{mɯzi ɲɤ-k-ɤmboʁ-ci}\hspace{5pt}\pcmn{火药爆炸了}\end{exemple}\relationsémantique{参考}{\lien{ⓔstoʁmboʁ}{stoʁmboʁ}}\end{entrée}

\begin{entrée}{ambrɤqɤt}{}{ⓔambrɤqɤt} 
\classe{vs} \paradigme{dir}{nɯ-}
\begin{définition}\pfra{être différent}\end{définition}
\begin{définition}\pcmn{有区别}\end{définition}
\begin{exemple}\pjya{kɯ-wɣrum kɯ-ɲaʁ ɲɯ-ɤmbrɤqɤt}\hspace{5pt}\pcmn{白色和黑色容易分得开}\end{exemple}
\begin{exemple}\pjya{ɲo-k-ɤmbrɤqɤt-ci}\hspace{5pt}\pcmn{变得有区别}\end{exemple}
\begin{exemple}\pjya{tsɯʁot cho qajdo ndʑi-skɤt ɲɯ-ɤmbrɤqɤt}\hspace{5pt}\pcmn{野鸡和乌鸦的叫声是有区别的}\end{exemple}\relationsémantique{参考}{\lien{ⓔsɤmbrɤqɤt}{sɤmbrɤqɤt}}\end{entrée}

\begin{entrée}{ambɯmbi}{}{ⓔambɯmbi}\relationsémantique{参考}{\lien{ⓔmbi}{mbi}}\end{entrée}

\begin{entrée}{amdzɯ/\variante{amdzɯt}}{}{ⓔamdzɯ} 
\classe{vi} \paradigme{dir}{kɤ-}\paradigme{dir}{thɯ-}\paradigme{dir}{kɤ-}
\begin{définition}\pfra{s'asseoir}\end{définition}
\begin{définition}\pcmn{坐}\end{définition}
\begin{définition}\pfra{faire asseoir}\end{définition}
\begin{définition}\pcmn{使坐}\end{définition}
\begin{exemple}\pjya{kɤ-amdzɯt}\hspace{5pt}\pcmn{他坐了}\end{exemple}
\begin{exemple}\pjya{ko-k-ɤmdzɯ-ci}\hspace{5pt}\pcmn{他坐了(我来的时候他已经坐下来了)}\end{exemple}
\begin{exemple}\pjya{ma-tɤ-tɯ-ndzur kɯ thɯ-amdzɯ}\hspace{5pt}\pcmn{你不要站在那里,坐下(对小孩子,严厉的口气)}\end{exemple}
\begin{exemple}\pjya{rŋɯl khri ɯ-taʁ kɤ́-wɣ-sɤmdzɯ ɲɯ-ŋu}\hspace{5pt}\pcmn{他们让她坐在银座位上}\end{exemple}\relationsémantique{参考}{\lien{ⓔnɤmdzɯ}{nɤmdzɯ}}
\begin{sous-entrée}{sɤmdzɯ}{ⓔamdzɯⓝsɤmdzɯ} 
\classe{vt}  
\grammaire{caus} \end{sous-entrée}

\end{entrée}

\begin{entrée}{amdʑɯβ}{}{ⓔamdʑɯβ} 
\classe{vi} \paradigme{dir}{pɯ-}
\begin{définition}\pfra{serré (avec une pince ou avec ses dents)}\end{définition}
\begin{définition}\pcmn{紧;密封(用钳子、牙齿夹得很紧 )}\end{définition}
\begin{exemple}\pjya{ki ɯ-srɯβ ɲɯ-ɤmdʑɯβ (=ɲɯ-mphrɤt)}\hspace{5pt}\pcmn{这个缝隙很紧}\end{exemple}
\begin{exemple}\pjya{tamɢom nɯnɯ ɲɯ-ɤmdʑɯβ}\hspace{5pt}\pcmn{夹子夹得很紧}\end{exemple}
\begin{exemple}\pjya{kɯki ɯ-srɯβ pjɤ-tu ri pjɤ-sɯɲcɤr tɕe pjɤ-k-ɤmdʑɯβ-ci}\hspace{5pt}\pcmn{这个东西原来有缝隙,把它按住了就变得很紧,缝隙都看不出来}\end{exemple}
\begin{sous-entrée}{sɤmdʑɯβ}{ⓔamdʑɯβⓝsɤmdʑɯβ} 
\classe{vt} 
\begin{définition}\pfra{fermer de façon hermétique}\end{définition}
\begin{définition}\pcmn{夹得很紧;使密封}\end{définition}\end{sous-entrée}

\end{entrée}

\begin{entrée}{amgri}{}{ⓔamgri} 
\classe{vi} \paradigme{dir}{nɯ-}
\begin{définition}\pfra{claire (eau)}\end{définition}
\begin{définition}\pcmn{清(水)}\end{définition}
\begin{exemple}\pjya{ki tɯ-ci ɲɯ-ɤmgri}\hspace{5pt}\pcmn{水很清}\end{exemple}
\begin{exemple}\pjya{kɯki tɯ-ci ɲɤ-k-ɤmgri-ci}\hspace{5pt}\pcmn{水变清了}\end{exemple}
\begin{exemple}\pjya{ki cha mɯm ma ɲɯ-ɤmgri}\hspace{5pt}\pcmn{这个酒很清,好喝}\end{exemple}\relationsémantique{反义词}{\lien{ⓔqarndɯm}{qarndɯm}}\relationsémantique{参考}{\lien{ⓔarɤmgrɯndɯr}{arɤmgrɯndɯr}}\end{entrée}

\begin{entrée}{amɟɤkho}{}{ⓔamɟɤkho} 
\classe{vi}  
\grammaire{comp} \paradigme{dir}{}\paradigme{}{nɯ-}
\begin{définition}\pfra{remettre, confier}\end{définition}
\begin{définition}\pcmn{交接}\end{définition}
\begin{exemple}\pjya{tɤ-rɟit nɯ-amɟɤkho-tɕi}\hspace{5pt}\pcmn{我把孩子交到你手里了;你从我手中把孩子接过来了}\end{exemple}\relationsémantique{参考}{\lien{ⓔmɟaⓢ2}{mɟa₂}}\relationsémantique{参考}{\lien{ⓔkhoⓗ2}{kho₂}}\end{entrée}

\begin{entrée}{amkhɤrju}{}{ⓔamkhɤrju} 
\classe{vs} \paradigme{dir}{nɯ-}
\begin{définition}\pfra{ni rond ni carré}\end{définition}
\begin{définition}\pcmn{不圆不方}\end{définition}
\begin{exemple}\pjya{tɤ-fkɯm ki kɯ-ɤmkhɤrju ɲɯ-ŋu}\hspace{5pt}\pcmn{这个袋子又不圆又不方}\end{exemple}
\begin{exemple}\pjya{kɯki ɲɯ-ɤmkhɤrju}\hspace{5pt}\pcmn{这个东西又不圆又不方}\end{exemple}\relationsémantique{同义词}{\lien{ⓔakhɤɟor}{akhɤɟor}}\end{entrée}

\begin{entrée}{amnoʁ}{}{ⓔamnoʁ} 
\classe{vs} \paradigme{dir}{tɤ-}
\begin{définition}\pfra{très large}\end{définition}
\begin{définition}\pcmn{容量大}\end{définition}
\begin{exemple}\pjya{ki tɤ-fkɯm ki ɲɯ-ɤmnoʁ tɕe khro ɲɯ-xtɕhɯt}\hspace{5pt}\pcmn{这个袋子容量很大,可以装很多东西}\end{exemple}
\begin{exemple}\pjya{ki kha ki ɲɯ-ɤmnoʁ}\hspace{5pt}\pcmn{这个房子容量很大}\end{exemple}\end{entrée}

\begin{entrée}{amɲaχtshɯm}{}{ⓔamɲaχtshɯm} 
\classe{vs}  
\grammaire{incorp} \paradigme{dir}{tɤ-}
\begin{définition}\pfra{être mesquin}\end{définition}
\begin{définition}\pcmn{小心眼;斤斤计较}\end{définition}
\begin{exemple}\pjya{ɲɯ-tɯ-amɲaχtshɯm}\hspace{5pt}\pcmn{你斤斤计较}\end{exemple}\relationsémantique{参考}{\lien{ⓔtɯ-mɲaʁ}{tɯ-mɲaʁ}}\relationsémantique{参考}{\lien{ⓔxtshɯm}{xtshɯm}}\end{entrée}

\begin{entrée}{amɲɤm}{}{ⓔamɲɤm} 
\classe{vs} \paradigme{dir}{thɯ-}\paradigme{dir}{thɯ-}\paradigme{construction}{infinitive raising}
\begin{définition}\pfra{homogène}\end{définition}
\begin{définition}\pcmn{均匀}\end{définition}
\begin{définition}\pfra{rendre homogène}\end{définition}
\begin{définition}\pcmn{使均匀}\end{définition}
\begin{exemple}\pjya{jaʁmba ɲɯ-ɤmɲɤm}\hspace{5pt}\pcmn{厚薄均匀}\end{exemple}
\begin{exemple}\pjya{jpumxtshɯm ɲɯ-ɤmɲɤm}\hspace{5pt}\pcmn{粗细均匀}\end{exemple}
\begin{exemple}\pjya{stoʁ ndɯβjndʐɤz ɲɯ-ɤmɲɤm}\hspace{5pt}\pcmn{胡豆大小均匀}\end{exemple}
\begin{exemple}\pjya{ɯ-skɤt ɲɯ-ɤmɲɤm}\hspace{5pt}\pcmn{他的声音(高低)很均匀}\end{exemple}
\begin{exemple}\pjya{ɯ-mɲɯtɕhɤz ɲɯ-ɤmɲɤm}\hspace{5pt}\pcmn{他的性格比较稳定}\end{exemple}
\begin{exemple}\pjya{cho-k-ɤmɲɤm-ci}\hspace{5pt}\pcmn{原来不均匀,现在变得很均匀}\end{exemple}
\begin{exemple}\pjya{kɯ-ɤmɲɤm ɲɯ-ŋke (mɤ-kɯ-nɯna ʑo ɲɯ-ŋke)}\hspace{5pt}\pcmn{(汽车)不停地开}\end{exemple}
\begin{exemple}\pjya{ɯ-kɤrme kɤ-rɤpjɤz chɤ-sɤmɲɤm}\hspace{5pt}\pcmn{她把头发编得很均匀}\end{exemple}
\begin{exemple}\pjya{kɤ-ɣndʑɯr chɤ-sɤmɲɤm}\hspace{5pt}\pcmn{他磨得均匀}\end{exemple}\relationsémantique{同义词}{\lien{ⓔamɯzɣɯt}{amɯzɣɯt}}
\begin{sous-entrée}{sɤmɲɤm}{ⓔamɲɤmⓝsɤmɲɤm}\end{sous-entrée}

\end{entrée}

\begin{entrée}{amɲɯmɲo}{}{ⓔamɲɯmɲo}\relationsémantique{参考}{\lien{ⓔmɲoⓗ2}{mɲo₂}}\end{entrée}

\begin{entrée}{amŋaʁ}{}{ⓔamŋaʁ} 
\classe{vi} \paradigme{dir}{nɯ-}\paradigme{dir}{nɯ-}
\begin{définition}\pfra{moelleux (tissus)}\end{définition}
\begin{définition}\pcmn{松软【泡】(绒布)}\end{définition}
\begin{définition}\pfra{rendre moelleux}\end{définition}
\begin{définition}\pcmn{弄松软}\end{définition}
\begin{exemple}\pjya{smɤɣ ɲɤ-saʁ tɕe ɲɯ-ɤmŋaʁ}\hspace{5pt}\pcmn{羊毛梳了以后就很软}\end{exemple}
\begin{exemple}\pjya{ɲɤ-k-ɤmŋaʁ-ci}\hspace{5pt}\pcmn{变软了}\end{exemple}
\begin{exemple}\pjya{smɤɣ nɯ-sɤmŋaʁ-a}\hspace{5pt}\pcmn{我把羊毛弄松软了}\end{exemple}
\begin{sous-entrée}{sɤmŋaʁ}{ⓔamŋaʁⓝsɤmŋaʁ} 
\classe{vt} \end{sous-entrée}

\end{entrée}

\begin{entrée}{amqaj}{}{ⓔamqaj} 
\classe{vi} 
\begin{définition}\pfra{se disputer}\end{définition}
\begin{définition}\pcmn{争吵}\end{définition}
\begin{exemple}\pjya{ɯʑo ɲɯ-ɤmqaj}\hspace{5pt}\pcmn{他在争吵}\end{exemple}\relationsémantique{参考}{\lien{ⓔanɯmqaj}{anɯmqaj}}\relationsémantique{参考}{\lien{ⓔtɯ-mqaj}{tɯ-mqaj}}\end{entrée}

\begin{entrée}{amtɕhoʁ}{}{ⓔamtɕhoʁ} 
\classe{vi} \paradigme{dir}{thɯ-}
\begin{définition}\pfra{être en ordre}\end{définition}
\begin{définition}\pcmn{整齐}\end{définition}
\begin{exemple}\pjya{ndʑu tɯ-spra ɲɯ-ɤmtɕhoʁ}\hspace{5pt}\pcmn{那把筷子很整齐}\end{exemple}
\begin{exemple}\pjya{ɯ-rɟit ra cho-wxti-nɯ tɕe cho-k-ɤmtɕhoʁ-nɯ-ci}\hspace{5pt}\pcmn{他的孩子们长大,都能做事了}\end{exemple}\relationsémantique{参考}{\lien{ⓔsɤmtɕhoʁ}{sɤmtɕhoʁ}}\end{entrée}

\begin{entrée}{amtɕoʁ}{}{ⓔamtɕoʁ} 
\classe{vi} \paradigme{dir}{nɯ-}
\begin{définition}\pfra{pointu}\end{définition}
\begin{définition}\pcmn{尖}\end{définition}
\begin{exemple}\pjya{jiɕqha mbrɯtɕɯ nɯ ɯ-ku ɲɯ-ɤmtɕoʁ}\hspace{5pt}\pcmn{这把刀很尖}\end{exemple}
\begin{exemple}\pjya{ndʑu ɯ-ku nɯ ɲɯ-ɤmtɕoʁ}\hspace{5pt}\pcmn{筷子很尖}\end{exemple}
\begin{exemple}\pjya{qaʁ ɯ-ku ɲɯ-ɤmtɕoʁ}\hspace{5pt}\pcmn{锄头很尖}\end{exemple}
\begin{exemple}\pjya{ɕɤmtshoʁ ɯ-ku kɯ-ɤmtɕoʁ ŋu}\hspace{5pt}\pcmn{钉子很尖}\end{exemple}\relationsémantique{参考}{\lien{ⓔakɤmtɕoʁ}{akɤmtɕoʁ}}\end{entrée}

\begin{entrée}{amthoʁmthɯt}{}{ⓔamthoʁmthɯt} 
\classe{vi} \paradigme{dir}{thɯ-}\paradigme{dir}{thɯ-}
\begin{définition}\pfra{suffisant}\end{définition}
\begin{définition}\pcmn{刚刚够用}\end{définition}
\begin{définition}\pfra{rendre suffisant}\end{définition}
\begin{définition}\pcmn{使够用;补充;填充}\end{définition}
\begin{exemple}\pjya{kɤ-ntɕhoz amthoʁmthɯt}\hspace{5pt}\pcmn{刚刚够用}\end{exemple}
\begin{exemple}\pjya{japa qhu ʁo tɕe, ji-ŋga ji-ndza ra thɯ-amthoʁmthɯt}\hspace{5pt}\pcmn{从去年开始,我们吃的、穿的就足够了}\end{exemple}
\begin{exemple}\pjya{kɯki laχtɕha ki thɯ-sɤmthoʁmthɯt-a}\hspace{5pt}\pcmn{我补充了这个东西}\end{exemple}
\begin{exemple}\pjya{tɯmbri kɤ-sɤlɤɣɯ-t-a tɕe thɯ-sɤmthoʁmthɯt-a}\hspace{5pt}\pcmn{我把绳子接起来了,使它够长}\end{exemple}\relationsémantique{参考}{\lien{ⓔmthɯt}{mthɯt}}
\begin{sous-entrée}{sɤmthoʁmthɯt}{ⓔamthoʁmthɯtⓝsɤmthoʁmthɯt} 
\classe{vt}  
\grammaire{caus} \end{sous-entrée}

\end{entrée}

\begin{entrée}{amthɯn}{}{ⓔamthɯn} 
\classe{vi} \paradigme{dir}{nɯ-}
\begin{définition}\pfra{s'apprécier réciproquement}\end{définition}
\begin{définition}\pcmn{谈恋爱}\end{définition}
\begin{exemple}\pjya{ʑɤni ɲɯ-ɤmthɯn-ndʑi}\hspace{5pt}\pcmn{他们在谈恋爱}\end{exemple}\étymologie{mtʰun}\end{entrée}

\begin{entrée}{amɯβde}{}{ⓔamɯβde}\relationsémantique{参考}{\lien{ⓔβde}{βde}}\end{entrée}

\begin{entrée}{amɯβɟɤt}{}{ⓔamɯβɟɤt} 
\classe{vi} \paradigme{dir}{nɯ-}
\begin{définition}\pfra{recevoir à parts égales}\end{définition}
\begin{définition}\pcmn{平均分到}\end{définition}
\begin{exemple}\pjya{jiʑora mɯ-ɲɤ-k-ɤmɯβɟɤt-i ma kɤ-nɯkro ɯ-spa pjɤ-rkɯn}\hspace{5pt}\pcmn{我们没有分到,因为要分的东西太少了}\end{exemple}\relationsémantique{参考}{\lien{ⓔβɟɤt}{βɟɤt}}\end{entrée}

\begin{entrée}{amɯfse}{}{ⓔamɯfse} 
\classe{vi} \paradigme{dir}{kɤ-}
\begin{définition}\pfra{se connaître}\end{définition}
\begin{définition}\pcmn{互相认识}\end{définition}
\begin{exemple}\pjya{tɕiʑo amɯfse-tɕi}\hspace{5pt}\pcmn{我们俩互相认识}\end{exemple}
\begin{exemple}\pjya{jiʑo kɤ-amɯfse-j kɯmŋuxpa tɤ-tsu}\hspace{5pt}\pcmn{我们几个认识已经五年了}\end{exemple}
\begin{exemple}\pjya{tɕiʑo kɤ-amɯfse-tɕi nɯ kɯβdɤxpa tɤ-tsu}\hspace{5pt}\pcmn{我们俩认识已经四年了}\end{exemple}\end{entrée}

\begin{entrée}{amɯmi}{₂}{ⓔamɯmiⓗ2} 
\classe{adv} 
\begin{définition}\pfra{certainement}\end{définition}
\begin{définition}\pcmn{肯定;必然;不出意料}\end{définition}
\begin{exemple}\pjya{nɯ ɲɯ-ti ri amɯmi ɕti ma nɯ fse ŋgrɤl}\hspace{5pt}\pcmn{他那样说就肯定是那样的了}\end{exemple}\end{entrée}

\begin{entrée}{amɯmi}{₁}{ⓔamɯmiⓗ1} 
\classe{vi}  
\grammaire{recip} \paradigme{dir}{tɤ-}
\begin{définition}\pfra{s’entendre}\end{définition}
\begin{définition}\pcmn{合得来}\end{définition}
\begin{exemple}\pjya{ʑɤni ɲɯ-ɤmɯmi-ndʑi}\hspace{5pt}\pcmn{他们俩合得来}\end{exemple}
\begin{exemple}\pjya{ndʑiʑo ni kɯ-ɤmɯmi ɲɯ-tɯ-ŋu-ndʑi}\hspace{5pt}\pcmn{你们俩合得来}\end{exemple}\relationsémantique{同义词}{\lien{ⓔɲɟɯɣ}{ɲɟɯɣ}}\end{entrée}

\begin{entrée}{amɯmto}{}{ⓔamɯmto} 
\classe{vi}  
\grammaire{recip} \paradigme{dir}{nɯ-}
\begin{définition}\pfra{se voir}\end{définition}
\begin{définition}\pcmn{互相看见}\end{définition}
\begin{exemple}\pjya{kutɕu cho @xiaoshuigou zgo nɯra ɲɯ-ɤmɯmto}\hspace{5pt}\pcmn{这里和小水沟上的山互相看得见}\end{exemple}
\begin{exemple}\pjya{tɕiʑo ni kɤ-ɤmɯmto mɯ-pɯ-rɲo-tɕi}\hspace{5pt}\pcmn{我们俩从来都没有见过面}\end{exemple}
\begin{exemple}\pjya{ŋotɕu tɯ-rɤʑi ma mɯ-ɲɯ-ɤmɯmto-tɕi}\hspace{5pt}\pcmn{你在哪里,我们俩见不到对方}\end{exemple}
\begin{sous-entrée}{sɤmɯmto}{ⓔamɯmtoⓝsɤmɯmto} 
\classe{vt} 
\begin{définition}\pfra{faire se rencontrer}\end{définition}
\begin{définition}\pcmn{使几个人互相碰见}\end{définition}\relationsémantique{参考}{\lien{ⓔmtoⓝmto}{mto}}\end{sous-entrée}

\end{entrée}

\begin{entrée}{amɯmtshɤm}{}{ⓔamɯmtshɤm} 
\classe{vi}  
\grammaire{recip}
\grammaire{caus} \paradigme{dir}{tɤ-}\paradigme{dir}{nɯ-}\paradigme{dir}{pɯ-}
\begin{définition}\pfra{s'entendre les uns les autres}\end{définition}
\begin{définition}\pcmn{互相听得到对方的声音}\end{définition}
\begin{exemple}\pjya{@dianhua kɤ-lɤt mɯ́j-khɯ tɕe mɯ-ɲɯ-ɤmɯmtshɤm-tɕi}\hspace{5pt}\pcmn{电话没有打成,所以我们俩听不到对方的声音}\end{exemple}\relationsémantique{参考}{\lien{ⓔmtshɤm}{mtshɤm}}
\begin{sous-entrée}{sɤmɯmtshɤm}{ⓔamɯmtshɤmⓝsɤmɯmtshɤm} 
\classe{vt} \end{sous-entrée}

\paradigme{dir}{nɯ-}
\begin{définition}\pfra{permettre à ... de s'entendre les uns les autres}\end{définition}
\begin{définition}\pcmn{让……互相听到对方的声音}\end{définition}
\begin{définition}\pfra{se transmettre les nouvelles les uns aux autres}\end{définition}
\begin{définition}\pcmn{互相传递消息;互相报信}\end{définition}
\begin{exemple}\pjya{nɯ-sɤmɯmtsham-a-nɯ}\hspace{5pt}\pcmn{我让他们互相听到对方的消息}\end{exemple}
\begin{exemple}\pjya{ɲɤ-k-ɤsɤmɯmtshɯmtshɤm-nɯ-ci}\hspace{5pt}\pcmn{他们互相传递了消息}\end{exemple}
\begin{sous-entrée}{asɤmɯmtshɯmtshɤm}{ⓔamɯmtshɤmⓝasɤmɯmtshɯmtshɤm} 
\classe{vi}  
\grammaire{refl} \end{sous-entrée}

\end{entrée}

\begin{entrée}{amɯrga}{}{ⓔamɯrga} 
\classe{n} 
\begin{définition}\pfra{albanais}\end{définition}
\begin{définition}\pcmn{阿尔巴尼亚人}\end{définition}\end{entrée}

\begin{entrée}{amɯrmbat}{}{ⓔamɯrmbat} 
\classe{vi}  
\grammaire{recip} \paradigme{dir}{tɤ-}
\begin{définition}\pfra{être proche}\end{définition}
\begin{définition}\pcmn{挨近;接近}\end{définition}
\begin{exemple}\pjya{tɕiʑo ni mɯ-ɲɯ-ɤmɯrmbat-tɕi}\hspace{5pt}\pcmn{我们俩很远}\end{exemple}
\begin{exemple}\pjya{nɤj nɤ-kha cho aʑo a-kha nɯ mɯ-ɲɯ-ɤmɯrmbat-ndʑi}\hspace{5pt}\pcmn{你的家离我的家很远}\end{exemple}
\begin{exemple}\pjya{jiʑora amɯrmbat-i}\hspace{5pt}\pcmn{我们很接近}\end{exemple}\relationsémantique{参考}{\lien{ⓔarmbat}{armbat}}\end{entrée}

\begin{entrée}{amɯrmbɯ}{}{ⓔamɯrmbɯ} 
\classe{vi}  
\grammaire{caus} \paradigme{dir}{tɤ-}\paradigme{dir}{tɤ-}
\begin{définition}\pfra{être rempli}\end{définition}
\begin{définition}\pcmn{被堆满;被装满}\end{définition}
\begin{exemple}\pjya{ɯ-tɯ-dɤn kɯ tɤ-k-ɤmɯrmbɯ-ci zjaŋzjaŋ}\hspace{5pt}\pcmn{装满了}\end{exemple}
\begin{sous-entrée}{zmɯrmbɯ}{ⓔamɯrmbɯⓝzmɯrmbɯ} 
\classe{vt} \end{sous-entrée}

\begin{définition}\pfra{remplir}\end{définition}
\begin{définition}\pcmn{装满;盛满;堆满}\end{définition}
\begin{exemple}\pjya{tɯsqar tɤ-zmɯrmbɯ-t-a}\hspace{5pt}\pcmn{我盛满了糌粑}\end{exemple}
\begin{exemple}\pjya{tɤjlu tɤ-zmɯrmbɯ-t-a}\hspace{5pt}\pcmn{我盛满了面粉}\end{exemple}
\begin{exemple}\pjya{khɯtsa ɯ-ŋgɯ tɯsqar tɤ-zmɯrmbɯ-t-a}\hspace{5pt}\pcmn{我在碗里盛满了糌粑}\end{exemple}\relationsémantique{参考}{\lien{ⓔrmbɯ}{rmbɯ}}\end{entrée}

\begin{entrée}{amɯrpu}{}{ⓔamɯrpu} 
\classe{vi}  
\grammaire{recip} \paradigme{dir}{kɤ-}\paradigme{dir}{tɤ-}\paradigme{dir}{tɤ-}
\begin{définition}\pfra{se heurter}\end{définition}
\begin{définition}\pcmn{互相碰撞}\end{définition}
\begin{définition}\pfra{faire se heurter}\end{définition}
\begin{définition}\pcmn{使互相碰撞}\end{définition}
\begin{exemple}\pjya{ko-k-ɤmɯrpu-ndʑi-ci}\hspace{5pt}\pcmn{他们俩相撞了}\end{exemple}
\begin{exemple}\pjya{ɯ-tɯ-ɤŋgɤrŋgɤr kɯ ɲɯ-ɤmɯrpu-tɕi}\hspace{5pt}\pcmn{很狭窄,所以我们俩就相撞了}\end{exemple}
\begin{exemple}\pjya{kɯɕɯŋgɯ tɕe kɯmaʁ smi pjɤ-me tɕe, qapi tú-wɣ-sɤmɯrpu tɕe smi ɲɯ́-wɣ-sɤβzu pjɤ-ra}\hspace{5pt}\pcmn{过去,没有其它火源,只能用白石头取火。}\end{exemple}\relationsémantique{参考}{\lien{ⓔrpu}{rpu}}
\begin{sous-entrée}{sɤmɯrpu}{ⓔamɯrpuⓝsɤmɯrpu} 
\classe{vt} \end{sous-entrée}

\end{entrée}

\begin{entrée}{amɯrqhi}{}{ⓔamɯrqhi} 
\classe{vi}  
\grammaire{recip} \paradigme{dir}{tɤ-}
\begin{définition}\pfra{loin l'un de l'autre}\end{définition}
\begin{définition}\pcmn{相距很远}\end{définition}
\begin{exemple}\pjya{a-kha cho a-@bangongshi ɲɯ-ɤmɯrqhi-ndʑi tɕe aʑo ʑa kɤ-zɣɯt mɯ-to-khɯ}\hspace{5pt}\pcmn{我家离办公室很远,所以就迟到了}\end{exemple}\relationsémantique{参考}{\lien{ⓔarqhi}{arqhi}}\end{entrée}

\begin{entrée}{amɯsthaβ}{}{ⓔamɯsthaβ} 
\classe{vi} \paradigme{dir}{kɤ-}\paradigme{dir}{kɤ-}
\begin{définition}\pfra{être l'un à côté de l'autre (deux morceaux)}\end{définition}
\begin{définition}\pcmn{互相接触}\end{définition}
\begin{définition}\pfra{mettre ensemble}\end{définition}
\begin{définition}\pcmn{拼凑}\end{définition}
\begin{exemple}\pjya{tɕiʑo kɤ-amɯsthaβ-tɕi}\hspace{5pt}\pcmn{我们俩挨在一起}\end{exemple}
\begin{exemple}\pjya{tɤjmɤɣ to-ɬoʁ-nɯ tɕe ɲɯ-dɤn tɕe ɲɯ-ɤmɯsthaβ ʑo}\hspace{5pt}\pcmn{长出了很多菌子,挨在一起}\end{exemple}
\begin{exemple}\pjya{jiʑora kɤ-amɯrmbat-i tɕe kɤ-amɯsthaβ-i ʑo ɕti}\hspace{5pt}\pcmn{我们挨得很近,挨在一起}\end{exemple}
\begin{exemple}\pjya{si ka-sɤmɯsthaβ}\hspace{5pt}\pcmn{他把木头拼凑在一起了}\end{exemple}
\begin{exemple}\pjya{rdɤstaʁ ka-sɤmɯsthaβ}\hspace{5pt}\pcmn{他把石头拼凑在一起了}\end{exemple}
\begin{sous-entrée}{sɤmɯsthaβ}{ⓔamɯsthaβⓝsɤmɯsthaβ} 
\classe{vt}  
\grammaire{caus} \end{sous-entrée}

\end{entrée}

\begin{entrée}{amɯsɯz}{}{ⓔamɯsɯz} 
\classe{vi} \paradigme{dir}{nɯ-}\paradigme{dir}{nɯ-}
\begin{définition}\pfra{s'ébruiter, être connu de tous}\end{définition}
\begin{définition}\pcmn{传开}\end{définition}
\begin{définition}\pfra{transmettre une information}\end{définition}
\begin{définition}\pcmn{传递消息}\end{définition}
\begin{exemple}\pjya{nɤʑo jɯfɕɯndʐi kɯre jɤ-tɯ-ɣe tɕe a-rkɯ tɯrme ra nɯ-ɕki ɲɤ-k-ɤmɯsɯz-ci}\hspace{5pt}\pcmn{你前天到这里的消息已经传开了,我周围的人都知道}\end{exemple}
\begin{exemple}\pjya{kɤ-nɤtsɯ mɤ-ra tɕe a-nɯ-ɤmɯsɯz jɤɣ}\hspace{5pt}\pcmn{这件事不用保密,可以公开}\end{exemple}
\begin{exemple}\pjya{kɤ-ɕe kɯ-ra nɯ nɯ-sɤmɯsɯz-a-nɯ}\hspace{5pt}\pcmn{我让他们都知道要去的那个消息}\end{exemple}\relationsémantique{参考}{\lien{ⓔsɯz}{sɯz}}
\begin{sous-entrée}{sɤmɯsɯz}{ⓔamɯsɯzⓝsɤmɯsɯz} 
\classe{vt} \end{sous-entrée}

\end{entrée}

\begin{entrée}{amɯti}{}{ⓔamɯti} 
\classe{vi} \paradigme{dir}{tɤ-}
\begin{définition}\pfra{se parler}\end{définition}
\begin{définition}\pcmn{彼此说}\end{définition}
\begin{exemple}\pjya{tɕhi kɯ-fse nɯ kɤ-ɤmɯti ra ma nɯ mɤɕtʂa mɤ-kɤ-ɤmɯtso}\hspace{5pt}\pcmn{有什么事情要互相说,才能互相理解}\end{exemple}
\begin{exemple}\pjya{jiʑora tɕhi kɯ-fse nɯra tɤ-amɯti-j}\hspace{5pt}\pcmn{我们互相说明了情况}\end{exemple}\end{entrée}

\begin{entrée}{amɯtso}{}{ⓔamɯtso} 
\classe{vi}  
\grammaire{refl} \paradigme{dir}{tɤ-}\paradigme{dir}{tɤ-}
\begin{définition}\pfra{clair (parole)}\end{définition}
\begin{définition}\pcmn{清楚(话、事情)}\end{définition}
\begin{définition}\pfra{dire clairement}\end{définition}
\begin{définition}\pcmn{说清楚}\end{définition}
\begin{exemple}\pjya{tɯ-rju mɯ-ɲɯ-ɤmɯtso}\hspace{5pt}\pcmn{话不清楚}\end{exemple}
\begin{exemple}\pjya{tɤ-amɯtso-tɕi}\hspace{5pt}\pcmn{我们俩说通了}\end{exemple}
\begin{exemple}\pjya{kɯɕɯŋgɯ mɯ-pjɤ-k-ɤmɯtso-ndʑi tɕe, tham tɕe tɤ-amɯti-ndʑi tɕe to-k-ɤmɯtso-ndʑi}\hspace{5pt}\pcmn{他们俩以前说不通,现在互相说话就说通了}\end{exemple}
\begin{exemple}\pjya{tɯ-rju ta-sɤmɯtso (tɤ-sɤmɯtso-t-a)}\hspace{5pt}\pcmn{他(我)把话说清楚了}\end{exemple}
\begin{exemple}\pjya{tɯ-rju mɯ-to-sɤmɯtso}\hspace{5pt}\pcmn{他没有把话说清楚}\end{exemple}
\begin{exemple}\pjya{mɯ-to-tɯ-sɤmɯtso-t}\hspace{5pt}\pcmn{你没有说清楚}\end{exemple}
\begin{exemple}\pjya{pɯ-kɯ-fse nɯ kɤ-sɤmɯtso ra}\hspace{5pt}\pcmn{要把发生的事情说清楚}\end{exemple}
\begin{exemple}\pjya{tɯrme kɯ tɤ-kɤ-tɯt nɯnɯ kɤ-sɤmɯtso}\hspace{5pt}\pcmn{要把人家讲过的话说清楚}\end{exemple}
\begin{sous-entrée}{sɤmɯtso}{ⓔamɯtsoⓝsɤmɯtso} 
\classe{vt}  
\grammaire{caus} \end{sous-entrée}

\begin{sous-entrée}{amɯtsɯtso}{ⓔamɯtsoⓝamɯtsɯtso} 
\classe{vi} \sens{1}
\begin{définition}\pfra{se comprendre}\end{définition}
\begin{définition}\pcmn{互相理解}\end{définition}
\begin{exemple}\pjya{tɕiʑo ni kɯ-ɤmɯtsɯtso ɕti-tɕi}\hspace{5pt}\pcmn{我们俩很了解对方}\end{exemple}
\begin{exemple}\pjya{kɤ-ɤmɯtsɯtso a-tɤ-tɯ-βze je}\hspace{5pt}\pcmn{你要讲得很清楚}\end{exemple}\end{sous-entrée}

\sens{2}
\begin{définition}\pfra{très clair}\end{définition}
\begin{définition}\pcmn{非常清楚}\end{définition}
\begin{exemple}\pjya{ɯʑo kɯ kɤ-ɤmɯtsɯtso to-βzu}\hspace{5pt}\pcmn{他讲得很清楚}\end{exemple}\relationsémantique{参考}{\lien{ⓔtso}{tso}}\end{entrée}

\begin{entrée}{amɯtsɯtso}{}{ⓔamɯtsɯtso}\relationsémantique{参考}{\lien{ⓔamɯtso}{amɯtso}}\end{entrée}

\begin{entrée}{amɯtɯɣ}{}{ⓔamɯtɯɣ} 
\classe{vi}  
\grammaire{refl} \paradigme{dir}{nɯ-}\paradigme{dir}{nɯ-}
\begin{définition}\pfra{se rencontrer}\end{définition}
\begin{définition}\pcmn{相逢}\end{définition}
\begin{définition}\pfra{faire se rencontrer}\end{définition}
\begin{définition}\pcmn{使合拢}\end{définition}
\begin{exemple}\pjya{ɲo-k-ɤmɯtɯɣ-ndʑi-ci}\hspace{5pt}\pcmn{他们俩相逢了}\end{exemple}
\begin{exemple}\pjya{tʂu kɤ-βzu nɯ-sɤmɯtɯɣ-i}\hspace{5pt}\pcmn{(我们把路从两边开始修起)最后在中间合拢}\end{exemple}\relationsémantique{参考}{\lien{ⓔatɯɣ}{atɯɣ}}
\begin{sous-entrée}{sɤmɯtɯɣ}{ⓔamɯtɯɣⓝsɤmɯtɯɣ} 
\classe{vt} \end{sous-entrée}

\end{entrée}

\begin{entrée}{amɯxtɕhɯxtɕhɯt}{}{ⓔamɯxtɕhɯxtɕhɯt} 
\classe{vs} 
\begin{définition}\pfra{être serré les uns contre les autres}\end{définition}
\begin{définition}\pcmn{挤挤挨挨,非常拥挤}\end{définition}
\begin{exemple}\pjya{ji-tɯ-dɤn kɯ mɯ-ɲɯ-ɤmɯxtɕhɯxtɕhɯt-i ʑo}\hspace{5pt}\pcmn{我们(人)多到非常拥挤}\end{exemple}
\begin{exemple}\pjya{tʂu ɯ-taʁ @qiche ɯ-tɯ-dɤn kɯ mɯ-ɲɯ-ɤmɯxtɕhɯxtɕhɯt-nɯ ʑo}\hspace{5pt}\pcmn{路上汽车多到挤也挤不动}\end{exemple}\end{entrée}

\begin{entrée}{amɯzɣɯt}{}{ⓔamɯzɣɯt} 
\classe{vi} \paradigme{dir}{nɯ-}\paradigme{dir}{nɯ-}
\begin{définition}\pfra{égal, homogène}\end{définition}
\begin{définition}\pcmn{均匀}\end{définition}
\begin{définition}\pfra{rendre homogène, faire de façon homogène}\end{définition}
\begin{définition}\pcmn{使均匀;做得均匀}\end{définition}
\begin{exemple}\pjya{ɯ-mdoʁ ɲɯ-ɤmɯzɣɯt}\hspace{5pt}\pcmn{颜色均匀}\end{exemple}
\begin{exemple}\pjya{mɯ-ɲɯ-ɤmɯzɣɯt}\hspace{5pt}\pcmn{不均匀}\end{exemple}
\begin{exemple}\pjya{ki tɯ-tɕha ki aʁɤndɯndɤt ɲɤ-k-ɤmɯzɣɯt-ci}\hspace{5pt}\pcmn{这个消息到处传遍了}\end{exemple}
\begin{exemple}\pjya{tɯ-ntʂu ɲɤ-sɤmɯzɣɯt}\hspace{5pt}\pcmn{他薅均匀了}\end{exemple}
\begin{exemple}\pjya{kɤ-kro nɯ-sɤmɯzɣɯt}\hspace{5pt}\pcmn{你平均分吧}\end{exemple}
\begin{exemple}\pjya{kɤ-rɤmbi nɯ-sɤmɯzɣɯt}\hspace{5pt}\pcmn{你平均给吧}\end{exemple}
\begin{exemple}\pjya{nɤ-rŋa kɤ-mar nɯ-sɤmɯzɣɯt}\hspace{5pt}\pcmn{你要在你脸上涂均匀}\end{exemple}
\begin{exemple}\pjya{kɤ-nɤma nɯ-sɤmɯzɣɯt}\hspace{5pt}\pcmn{工作要做得全面一点}\end{exemple}
\begin{exemple}\pjya{@Chengdu kɤ-sɤmɯzɣɯt mɯ́j-sɤcha}\hspace{5pt}\pcmn{不可能走遍整个成都}\end{exemple}
\begin{exemple}\pjya{tɯ-ŋga kɤ-ŋga nɯ-sɤmɯzɣɯt-a}\hspace{5pt}\pcmn{我所有的衣服都穿过}\end{exemple}\relationsémantique{参考}{\lien{ⓔamɲɤm}{amɲɤm}}
\begin{sous-entrée}{sɤmɯzɣɯt}{ⓔamɯzɣɯtⓝsɤmɯzɣɯt} 
\classe{vt} \end{sous-entrée}

\end{entrée}

\begin{entrée}{amɯzwɤr}{}{ⓔamɯzwɤr} 
\classe{vi} \paradigme{dir}{nɯ-}\sens{1}
\begin{définition}\pfra{se répandre (feu)}\end{définition}
\begin{définition}\pcmn{蔓延(火)}\end{définition}
\begin{exemple}\pjya{taŋi tɤ-ɣe tɕe ɣndʑɤβ a-mɤ-tɤ-lɯɣ ra ma ɲɯ-ɤmɯzwɤr mbat}\hspace{5pt}\pcmn{天旱的时候不要有火灾,不然活容易蔓延}\end{exemple}\sens{2}
\begin{définition}\pfra{se répandre}\end{définition}
\begin{définition}\pcmn{传开}\end{définition}
\begin{exemple}\pjya{ki tɯ-tɕha ki ɲɤ-k-ɤmɯzwɤr-ci tɕe ɲɤ-k-ɤmɯsɯz-ci}\hspace{5pt}\pcmn{这个消息传开了}\end{exemple}\relationsémantique{参考}{\lien{ⓔzwɤrⓗ1}{zwɤr₁}}\end{entrée}

\begin{entrée}{anamana}{}{ⓔanamana} 
\classe{adv} 
\begin{définition}\pfra{identique}\end{définition}
\begin{définition}\pcmn{一模一样}\end{définition}
\begin{exemple}\pjya{kɯki tɯ-ŋga ki ni anamana ɲɯ-ŋu-ndʑi}\hspace{5pt}\pcmn{这两件衣服是一模一样的}\end{exemple}\relationsémantique{同义词}{\lien{ⓔnaχtɕɯɣ}{naχtɕɯɣ}}\étymologie{a.na.ma.na}\end{entrée}

\begin{entrée}{anaχtɯχto}{}{ⓔanaχtɯχto}\relationsémantique{参考}{\lien{ⓔnaχto}{naχto}}\end{entrée}

\begin{entrée}{anɤjɯjo}{}{ⓔanɤjɯjo}\relationsémantique{参考}{\lien{ⓔnɤjo}{nɤjo}}\end{entrée}

\begin{entrée}{anɤkhɤzŋgɯzŋga}{}{ⓔanɤkhɤzŋgɯzŋga}\relationsémantique{参考}{\lien{ⓔnɤkhɤzŋga}{nɤkhɤzŋga}}\end{entrée}

\begin{entrée}{anɤmtsɯmtsioʁ}{}{ⓔanɤmtsɯmtsioʁ}\relationsémantique{参考}{\lien{ⓔnɤmtsioʁ}{nɤmtsioʁ}}\end{entrée}

\begin{entrée}{anɤntshɯntshi}{}{ⓔanɤntshɯntshi}\relationsémantique{参考}{\lien{ⓔnɤntshi}{nɤntshi}}\end{entrée}

\begin{entrée}{anɤrkɯrko}{}{ⓔanɤrkɯrko} 
\classe{vi} \paradigme{dir}{tɤ-}\paradigme{dir}{pɯ-}
\begin{définition}\pfra{se forcer les un les autres}\end{définition}
\begin{définition}\pcmn{互相逼迫}\end{définition}
\begin{exemple}\pjya{jiɕqha tɯrme nɯ kɯ ``tʂha kɤ-tshi ra" ɲɯ-ti ri, aʑo ``mɤ-tshi-a" tɤ-tɯt-a ri, mɯ́j-khɯ tɕe tɤ-anɤrkɯrko-tɕi}\hspace{5pt}\pcmn{那个人叫我喝茶,我说不喝,我们争了一段时间}\end{exemple}
\begin{exemple}\pjya{ɯʑo kɯ ``nɤʑo kɤ-rɤʑi, aj ju-ɕe-a" ɲɯ-ti ri, aʑo kɯ ``nɤʑo kɤ-rɤʑi tɕe aj ju-ɕe-a" tɤ-tɯt-a tɕe, tɤ-anɤrkɯrko-tɕi}\hspace{5pt}\pcmn{那个人叫我留下,说他自己会去,我叫他留下,说我自己去,互相争了一下}\end{exemple}\relationsémantique{参考}{\lien{ⓔnɤrkoⓗ1}{nɤrko₁}}\end{entrée}

\begin{entrée}{anɤrpɯrpaʁ}{}{ⓔanɤrpɯrpaʁ} 
\classe{vi} \paradigme{dir}{nɯ-}
\begin{définition}\pfra{s'entendre bien ensemble}\end{définition}
\begin{définition}\pcmn{合得来}\end{définition}
\begin{exemple}\pjya{tɕiʑo tɯtsɣe kɤβzu ɲɯ-ɤnɤrpɯrpaʁ-tɕi}\hspace{5pt}\pcmn{我们做生意很合得来}\end{exemple}
\begin{exemple}\pjya{tɕiʑo ki tshoŋ βzu-tɕi tɕe, anɤrpɯrpaʁ-tɕi wo}\hspace{5pt}\pcmn{我们做生意,非常投合}\end{exemple}
\begin{exemple}\pjya{ɯʑo cho ɲɯ-ɤnɤrpɯrpaʁ-tɕi}\hspace{5pt}\pcmn{我们俩关系融洽}\end{exemple}\relationsémantique{参考}{\lien{ⓔaɣɯrpaʁ}{aɣɯrpaʁ}}\relationsémantique{参考}{\lien{ⓔnɤrpaʁ}{nɤrpaʁ}}\end{entrée}

\begin{entrée}{anɤrɯre}{}{ⓔanɤrɯre}\relationsémantique{参考}{\lien{ⓔnɤreⓗ1ⓢ2ⓝnɤre}{nɤre}}\end{entrée}

\begin{entrée}{anɤsɯso}{}{ⓔanɤsɯso} 
\classe{vi}  
\grammaire{recip} \paradigme{dir}{tɤ-}
\begin{définition}\pfra{se manquer l'un à l'autre}\end{définition}
\begin{définition}\pcmn{互相想念}\end{définition}
\begin{exemple}\pjya{jiɕqha nɯ cho ɲɯ-ɤnɤsɯso-ndʑi}\hspace{5pt}\pcmn{他跟这个人互相想念着对方}\end{exemple}
\begin{exemple}\pjya{anɤsɯso-ndʑi}\hspace{5pt}\pcmn{他们俩互相想念着对方}\end{exemple}\relationsémantique{参考}{\lien{ⓔsɯso}{sɯso}}\end{entrée}

\begin{entrée}{anɤtsɯtsɯ}{}{ⓔanɤtsɯtsɯ} 
\classe{vs} 
\begin{définition}\pfra{être caché}\end{définition}
\begin{définition}\pcmn{隐藏着}\end{définition}\relationsémantique{参考}{\lien{ⓔnɤtsɯ}{nɤtsɯ}}\end{entrée}

\begin{entrée}{anɤtɯtɯɣ}{}{ⓔanɤtɯtɯɣ}\relationsémantique{参考}{\lien{ⓔatɯɣ}{atɯɣ}}\end{entrée}

\begin{entrée}{anɤʑɤmŋɯmŋɤn}{}{ⓔanɤʑɤmŋɯmŋɤn}\relationsémantique{参考}{\lien{ⓔnɤʑɤmŋɤn}{nɤʑɤmŋɤn}}\end{entrée}

\begin{entrée}{anbaʁ}{}{ⓔanbaʁ} 
\classe{vi}  
\grammaire{appl} \paradigme{dir}{kɤ-}\paradigme{dir}{kɤ-}\paradigme{dir}{kɤ-}\paradigme{dir}{kɤ-}
\begin{définition}\pfra{se cacher}\end{définition}
\begin{définition}\pcmn{躲藏}\end{définition}
\begin{définition}\pfra{cacher}\end{définition}
\begin{définition}\pcmn{藏起来}\end{définition}
\begin{définition}\pfra{se cacher}\end{définition}
\begin{définition}\pcmn{躲起来}\end{définition}
\begin{exemple}\pjya{pjɯ-kɯ-mto mɯ-tɤ-pe tɕe, ku-kɯ-ɤnbaʁ ra}\hspace{5pt}\pcmn{不适合被人发现的时候,就应该躲起来}\end{exemple}
\begin{exemple}\pjya{nɤʑo cischiz ɕ-kɤ-ɤnbaʁ tɕe a-mɤ-pɯ-tɯ́-wɣ-mto}\hspace{5pt}\pcmn{你在某个地方躲起来,免得别人看见你}\end{exemple}
\begin{exemple}\pjya{ku-tɯ-ɤnbaʁ mɤ-ra ma tɤɣa jɤ-ɣi}\hspace{5pt}\pcmn{你别躲起来了,出来}\end{exemple}
\begin{exemple}\pjya{nɤki nɯ kɤ-sɤnbaʁ tɕe, a-mɤ-pɯ-mto-nɯ}\hspace{5pt}\pcmn{你把这个藏起来,免得他们看见}\end{exemple}
\begin{exemple}\pjya{ko-tɯ-ʑɣɤsɤnbaʁ ɕti tɕe mɯ-pɯ-ta-mto}\hspace{5pt}\pcmn{你躲起来了,我没有看到你}\end{exemple}
\begin{sous-entrée}{nɤnbɯnbaʁ}{ⓔanbaʁⓝnɤnbɯnbaʁ} 
\classe{vi} 
\begin{définition}\pfra{se cacher partout}\end{définition}
\begin{définition}\pcmn{躲来躲去}\end{définition}\end{sous-entrée}

\begin{sous-entrée}{sɤnbaʁ}{ⓔanbaʁⓝsɤnbaʁ} 
\classe{vt} \end{sous-entrée}

\begin{sous-entrée}{ʑɣɤsɤnbaʁ}{ⓔanbaʁⓝʑɣɤsɤnbaʁ} 
\classe{vi} \end{sous-entrée}

\begin{sous-entrée}{nɤnbaʁ}{ⓔanbaʁⓝnɤnbaʁ} 
\classe{vt} \end{sous-entrée}

\begin{définition}\pfra{éviter, se cacher de}\end{définition}
\begin{définition}\pcmn{躲避}\end{définition}\end{entrée}

\begin{entrée}{andu}{}{ⓔandu} 
\classe{vi} \paradigme{dir}{nɯ-}
\begin{définition}\pfra{s'échanger}\end{définition}
\begin{définition}\pcmn{兑换}\end{définition}
\begin{exemple}\pjya{ɕku pɯ-rkɯn ɕti ri sqɯ-mpɕar nɯ-andu}\hspace{5pt}\pcmn{大蒜很少,要十块钱}\end{exemple}
\begin{exemple}\pjya{tɯ-tɯrpa kɯ sqɯ-mpɕar andu}\hspace{5pt}\pcmn{一斤可以卖十块钱}\end{exemple}\relationsémantique{参考}{\lien{ⓔsɤndu}{sɤndu}}\end{entrée}

\begin{entrée}{andɤko}{}{ⓔandɤko} 
\classe{vi} \paradigme{dir}{nɯ-}
\begin{définition}\pfra{se tendre (allongé, horizontalement)}\end{définition}
\begin{définition}\pcmn{横着伸直}\end{définition}
\begin{exemple}\pjya{qapri tʂɤχcɤl ʑo ɲɤ-k-ɤndɤko-ci}\hspace{5pt}\pcmn{蛇在路中间}\end{exemple}
\begin{exemple}\pjya{tɤ-ri nɯ pɯ-ajʁu ɕti ri, ɲɤ-k-ɤndɤko-ci}\hspace{5pt}\pcmn{线原来是弯的,现在变直了}\end{exemple}
\begin{sous-entrée}{sɤndɤko}{ⓔandɤkoⓝsɤndɤko}
\begin{exemple}\pjya{kɯki ɲɯ-ɤjʁu tɕe, ɲɯ-sɤndɤkam-a}\hspace{5pt}\pcmn{这个东西是弯着的,我把它拉直了}\end{exemple}\relationsémantique{同义词}{\lien{ⓔastɤko}{astɤko}}\end{sous-entrée}

\end{entrée}

\begin{entrée}{andɤr}{}{ⓔandɤr} 
\classe{vi} \paradigme{dir}{nɯ-}
\begin{définition}\pfra{être touché (blessure)}\end{définition}
\begin{définition}\pcmn{被碰(疮、伤口)}\end{définition}
\begin{exemple}\pjya{a-ʑmbɤr nɯ-andɤr}\hspace{5pt}\pcmn{碰了一下我的疮}\end{exemple}
\begin{exemple}\pjya{nɤ-jaʁ pjɤ-ɴɢraʁ ri, ko-tɯ-nɯ-sɯrput tɕe ɲɤ-k-ɤndɤr-ci}\hspace{5pt}\pcmn{你的手受伤了,你不小心就碰了一下}\end{exemple}\relationsémantique{参考}{\lien{ⓔsɤndɤrⓗ1}{sɤndɤr₁}}\end{entrée}

\begin{entrée}{andi}{}{ⓔandi} 
\classe{adv} 
\begin{définition}\pfra{à l'ouest}\end{définition}
\begin{définition}\pcmn{在西边}\end{définition}\end{entrée}

\begin{entrée}{andichoʁle}{}{ⓔandichoʁle} 
\classe{n} 
\begin{définition}\pfra{vent du sud}\end{définition}
\begin{définition}\pcmn{南风}\end{définition}\relationsémantique{参考}{\lien{ⓔqale}{qale}}\relationsémantique{参考}{\lien{ⓔakɯchoʁle}{akɯchoʁle}}\end{entrée}

\begin{entrée}{ando}{}{ⓔando} 
\classe{vi}  
\grammaire{pass} \paradigme{dir}{tɤ-}
\begin{définition}\pfra{être pris}\end{définition}
\begin{définition}\pcmn{带上}\end{définition}
\begin{exemple}\pjya{ando ma tɤ-fkɯm ɯ-ŋgɯ arku}\hspace{5pt}\pcmn{带上了,在口袋里}\end{exemple}\relationsémantique{参考}{\lien{ⓔndo}{ndo}}\end{entrée}

\begin{entrée}{andɯja}{}{ⓔandɯja} 
\classe{vi} \paradigme{dir}{tɤ-}\paradigme{dir}{tɤ-}
\begin{définition}\pfra{se rassembler}\end{définition}
\begin{définition}\pcmn{聚集}\end{définition}
\begin{définition}\pfra{rassembler}\end{définition}
\begin{définition}\pcmn{召集;聚集}\end{définition}
\begin{exemple}\pjya{to-k-ɤndɯja-nɯ-ci}\hspace{5pt}\pcmn{他们聚在一起了}\end{exemple}
\begin{exemple}\pjya{jiʑora jisŋi ta-ndɯja-j}\hspace{5pt}\pcmn{我们今天聚会了}\end{exemple}
\begin{exemple}\pjya{tɤ-pɤtso ra tɤ-sɤndɯjat-a-nɯ}\hspace{5pt}\pcmn{我把孩子们聚集在一起了}\end{exemple}
\begin{sous-entrée}{sɤndɯja}{ⓔandɯjaⓝsɤndɯja} 
\classe{vt}  
\grammaire{caus} \end{sous-entrée}

\end{entrée}

\begin{entrée}{andɯndo}{}{ⓔandɯndo}\relationsémantique{参考}{\lien{ⓔndo}{ndo}}\end{entrée}

\begin{entrée}{andzoʁjoʁ}{}{ⓔandzoʁjoʁ} 
\classe{vi} \paradigme{dir}{kɤ-}\paradigme{dir}{kɤ-}
\begin{définition}\pfra{coller ensemble}\end{définition}
\begin{définition}\pcmn{粘在一起}\end{définition}
\begin{définition}\pfra{faire coller ensemble}\end{définition}
\begin{définition}\pcmn{连接在一起}\end{définition}
\begin{exemple}\pjya{laχtɕha ɲɯ-ɤndzoʁjoʁ}\hspace{5pt}\pcmn{东西粘在一起}\end{exemple}
\begin{sous-entrée}{sɤndzoʁjoʁ}{ⓔandzoʁjoʁⓝsɤndzoʁjoʁ} 
\classe{vt} \end{sous-entrée}

\end{entrée}

\begin{entrée}{andzɯndza}{}{ⓔandzɯndza} 
\classe{vi}  
\grammaire{recip} \paradigme{dir}{tɤ-}\sens{1}
\begin{définition}\pfra{se manger les uns les autres}\end{définition}
\begin{définition}\pcmn{互相吃}\end{définition}\sens{2}
\begin{définition}\pfra{se faire du mal les uns aux autres}\end{définition}
\begin{définition}\pcmn{互相伤害}\end{définition}
\begin{exemple}\pjya{jiɕqha nɯ to-k-ɤnɯmqaj-ndʑi tɕe tu-ondzɯndza-ndʑi ɲɯ-ŋu}\hspace{5pt}\pcmn{他们俩吵架了,互相伤害对方}\end{exemple}\relationsémantique{参考}{\lien{ⓔndza}{ndza}}\end{entrée}

\begin{entrée}{andzɯndzri}{}{ⓔandzɯndzri}\relationsémantique{参考}{\lien{ⓔndzri}{ndzri}}\end{entrée}

\begin{entrée}{andzɯqoʁ}{}{ⓔandzɯqoʁ} 
\classe{vi} \paradigme{dir}{tɤ-}
\begin{définition}\pfra{hoqueter}\end{définition}
\begin{définition}\pcmn{打嗝儿}\end{définition}
\begin{exemple}\pjya{to-k-ɤndzɯqoʁ-ci}\hspace{5pt}\pcmn{他打嗝了}\end{exemple}
\begin{exemple}\pjya{ma-tɤ-tɯ-ɤndzɯqoʁ ntsɯ ma ɲɯ-sɤɣdɯɣ}\hspace{5pt}\pcmn{你不要总是打嗝,很讨厌}\end{exemple}\end{entrée}

\begin{entrée}{andzɯt}{}{ⓔandzɯt} 
\classe{vi}  
\grammaire{appl} \paradigme{dir}{tɤ-}\paradigme{dir}{tɤ-}
\begin{définition}\pfra{aboyer}\end{définition}
\begin{définition}\pcmn{狗吠}\end{définition}
\begin{exemple}\pjya{khɯna ɲɯ-ɤndzɯt}\hspace{5pt}\pcmn{狗在叫}\end{exemple}
\begin{exemple}\pjya{khɯna to-k-ɤndzɯt-ci}\hspace{5pt}\pcmn{狗叫了一声}\end{exemple}
\begin{exemple}\pjya{qapar ɲɯ-ɤndzɯt}\hspace{5pt}\pcmn{豺狗在叫}\end{exemple}
\begin{sous-entrée}{nɤndzɯt}{ⓔandzɯtⓝnɤndzɯt} 
\classe{vt} \end{sous-entrée}

\begin{définition}\pfra{aboyer sur qqn}\end{définition}
\begin{définition}\pcmn{对(某人)吠}\end{définition}\end{entrée}

\begin{entrée}{andʑɤmstu}{}{ⓔandʑɤmstu} 
\classe{vs}  
\grammaire{comp} 
\begin{définition}\pfra{lisse (tissu)}\end{définition}
\begin{définition}\pcmn{平整(衣服)}\end{définition}
\begin{sous-entrée}{sɤndʑɤmstu}{ⓔandʑɤmstuⓝsɤndʑɤmstu} 
\classe{vt} \paradigme{dir}{nɯ-}
\begin{définition}\pfra{repasser (vêtement)}\end{définition}
\begin{définition}\pcmn{熨平}\end{définition}
\begin{exemple}\pjya{ma-tɤ-tɯ-mphɯr, nɯ-sɤndʑɤmste}\hspace{5pt}\pcmn{不要裹起来,把它摆平}\end{exemple}
\begin{exemple}\pjya{tɯ-ŋga nɯ-sɤndʑɤmstu-t-a}\hspace{5pt}\pcmn{我(用熨斗)把衣服熨平了}\end{exemple}\relationsémantique{参考}{\lien{ⓔndʑɤm}{ndʑɤm}}\relationsémantique{参考}{\lien{ⓔastu}{astu}}\end{sous-entrée}

\end{entrée}

\begin{entrée}{andʑɯɣndʑɯɣ}{}{ⓔandʑɯɣndʑɯɣ} 
\classe{vs} 
\begin{définition}\pfra{l'un à côté de l'autre}\end{définition}
\begin{définition}\pcmn{一个挨着一个}\end{définition}
\begin{exemple}\pjya{ɯ-ɕɣa wuma ɲɯ-ɤndʑɯɣndʑɯɣ, mɯ-ɲɯ-ɤχa}\hspace{5pt}\pcmn{他的牙齿很密,没有洞}\end{exemple}\end{entrée}

\begin{entrée}{andʑɯndʑu}{}{ⓔandʑɯndʑu} 
\classe{vs} \paradigme{dir}{nɯ-}
\begin{définition}\pfra{raide (cadavre)}\end{définition}
\begin{définition}\pcmn{僵直}\end{définition}
\begin{exemple}\pjya{pjɤ-si tɕe ɲɤ-k-ɤndʑɯndʑu-ci}\hspace{5pt}\pcmn{他死了就僵直了}\end{exemple}\relationsémantique{参考}{\lien{ⓔndʑu}{ndʑu}}\end{entrée}

\begin{entrée}{andʑɯrɣa}{}{ⓔandʑɯrɣa} 
\classe{vs} \paradigme{dir}{tɤ-}
\begin{définition}\pfra{être voisins}\end{définition}
\begin{définition}\pcmn{住在一起;挨着的}\end{définition}\relationsémantique{参考}{\lien{ⓔtɤ-rɣa}{tɤ-rɣa}}\end{entrée}

\begin{entrée}{angɯt}{}{ⓔangɯt} 
\classe{vs} 
\begin{définition}\pfra{commun, que l'on possède ensemble}\end{définition}
\begin{définition}\pcmn{共同的}\end{définition}
\begin{exemple}\pjya{angɯt-i}\hspace{5pt}\pcmn{我们都有一份}\end{exemple}\relationsémantique{同义词}{\lien{ⓔalpɯm}{alpɯm}}\relationsémantique{参考}{\lien{ⓔnɤngɯt}{nɤngɯt}}\end{entrée}

\begin{entrée}{antɤm}{}{ⓔantɤm} 
\classe{vi} \paradigme{dir}{nɯ-}
\begin{définition}\pfra{plat}\end{définition}
\begin{définition}\pcmn{平(地)}\end{définition}
\begin{exemple}\pjya{stɤmku ɲɯ-ɤntɤm}\hspace{5pt}\pcmn{草地很平}\end{exemple}
\begin{exemple}\pjya{khɤxtu antɤm}\hspace{5pt}\pcmn{房背很平}\end{exemple}
\begin{exemple}\pjya{ɯ-thoʁ antɤm}\hspace{5pt}\pcmn{地很平}\end{exemple}\relationsémantique{参考}{\lien{ⓔapjɤntɤm}{apjɤntɤm}}\end{entrée}

\begin{entrée}{antɕhoʁjɤr}{}{ⓔantɕhoʁjɤr} 
\classe{vi} \paradigme{dir}{nɯ-}
\begin{définition}\pfra{être incomplet}\end{définition}
\begin{définition}\pcmn{疏漏}\end{définition}
\begin{exemple}\pjya{nɤ-laχtɕha koŋla tɤ-rɤwum tɕe a-mɤ-nɯ-ɤntɕhoʁjɤr}\hspace{5pt}\pcmn{你把东西收拾好,不要漏掉}\end{exemple}
\begin{exemple}\pjya{ɯβrɤ-ɲɤ-k-ɤntɕhoʁjɤr-ci kɯ}\hspace{5pt}\pcmn{应该没有疏漏了吧}\end{exemple}\relationsémantique{同义词}{\lien{ⓔatɕɯtɕit}{atɕɯtɕit}}
\begin{sous-entrée}{sɤntɕhoʁjɤr}{ⓔantɕhoʁjɤrⓝsɤntɕhoʁjɤr} 
\classe{vt} \end{sous-entrée}

\end{entrée}

\begin{entrée}{antɕhɯ}{}{ⓔantɕhɯ} 
\classe{vs} \paradigme{dir}{nɯ-}
\begin{définition}\pfra{nombreux}\end{définition}
\begin{définition}\pcmn{数量多}\end{définition}
\begin{exemple}\pjya{jisŋi tɯrme ɲɯ-ɤntɕhɯ-nɯ tɕe ndzɤtshi khro tsa kɤ-βzu ra}\hspace{5pt}\pcmn{今天人很多,要多做一点饭菜}\end{exemple}
\begin{exemple}\pjya{aʑo ji-rɣa ra nɯ nɯ-kha kɤntɕhɯ-ɣjɤn jɤ-ari-a}\hspace{5pt}\pcmn{我去了我们邻居的家很多次}\end{exemple}
\begin{exemple}\pjya{aj @chengdu kɤntɕhɯ-ɣjɤn thɯ-ɣe-a}\hspace{5pt}\pcmn{我来过成都很多次}\end{exemple}
\begin{sous-entrée}{sɤntɕhɯ}{ⓔantɕhɯⓝsɤntɕhɯ} 
\classe{vt} 
\begin{exemple}\pjya{a-ŋga tɤ-sɤntɕhɯ-t-a / a-ŋga kɯ-ɤntɕhɯ tɤ-ŋga-t-a}\hspace{5pt}\pcmn{我穿了好几件衣服}\end{exemple}
\begin{exemple}\pjya{rɟɤɣi tɤ-sɤntɕhɯ-t-a}\hspace{5pt}\pcmn{我吃了好几碗糌粑}\end{exemple}\end{sous-entrée}

\end{entrée}

\begin{entrée}{anthɣar}{}{ⓔanthɣar} 
\classe{vi} \sens{1}\paradigme{dir}{tɤ-}
\begin{définition}\pfra{rebondir}\end{définition}
\begin{définition}\pcmn{弹起来}\end{définition}
\begin{exemple}\pjya{tɤ-k-ɤnthɣar-ci}\hspace{5pt}\pcmn{弹起来了}\end{exemple}\sens{2}\paradigme{dir}{nɯ-}\paradigme{dir}{nɯ-}
\begin{définition}\pfra{être perdu}\end{définition}
\begin{définition}\pcmn{丢失了}\end{définition}
\begin{définition}\pfra{perdre}\end{définition}
\begin{définition}\pcmn{丢失}\end{définition}
\begin{exemple}\pjya{a-taqaβ ɲɤ-k-ɤnthɣar-ci tɕe maŋe}\hspace{5pt}\pcmn{我的针不见了}\end{exemple}
\begin{exemple}\pjya{a-tɤ-fkɯm ɲɤ-nɯ-sɤnthɣar-a tɕe maŋe}\hspace{5pt}\pcmn{我把袋子丢失了,看不见了}\end{exemple}
\begin{sous-entrée}{sɤnthɣar}{ⓔanthɣarⓢ2ⓝsɤnthɣar} 
\classe{vt} \end{sous-entrée}

\end{entrée}

\begin{entrée}{antsɤndu}{}{ⓔantsɤndu} 
\classe{vi} \paradigme{dir}{nɯ-}\paradigme{dir}{tɤ-}
\begin{définition}\pfra{être échangé par erreur}\end{définition}
\begin{définition}\pcmn{无意中被交换}\end{définition}
\begin{exemple}\pjya{laχtɕha ɲɯ-ɤntsɤndu}\hspace{5pt}\pcmn{东西不小心调换了}\end{exemple}
\begin{exemple}\pjya{tɕiʑo tɕi-ŋga nɯ ɲɤ-k-ɤntsɤndu-ci}\hspace{5pt}\pcmn{我们俩的衣服不小心弄错了}\end{exemple}
\begin{exemple}\pjya{tɯ-rju ɯ-qhu ɯ-ʁɤri ɲɤ-k-ɤntsɤndu-ci}\hspace{5pt}\pcmn{颠倒说话了}\end{exemple}\relationsémantique{参考}{\lien{ⓔsɤntsɤndu}{sɤntsɤndu}}\relationsémantique{参考}{\lien{ⓔsɤndu}{sɤndu}}\end{entrée}

\begin{entrée}{antɯ}{}{ⓔantɯ} 
\classe{vi} \paradigme{dir}{tɤ-}
\begin{définition}\pfra{avoir l'ouverture tournée vers le haut}\end{définition}
\begin{définition}\pcmn{放平,口朝上}\end{définition}
\begin{exemple}\pjya{sɤlaŋphɤn antɯ tɕe pe ma nɯ maʁ tɯ-ci jit ɕti}\hspace{5pt}\pcmn{盆子要放平,不然水流出来}\end{exemple}
\begin{exemple}\pjya{ki tú-wɣ-sɤntɯ ra ma nɯ maʁ nɤ jit}\hspace{5pt}\pcmn{要把口朝上,不然水会流出来}\end{exemple}\relationsémantique{反义词}{\lien{ⓔβʁum}{βʁum}}
\begin{sous-entrée}{sɤntɯ}{ⓔantɯⓝsɤntɯ} 
\classe{vt} \end{sous-entrée}

\end{entrée}

\begin{entrée}{anɯɕqhɯɕqhu}{}{ⓔanɯɕqhɯɕqhu}\relationsémantique{参考}{\lien{ⓔnɯɕqhu}{nɯɕqhu}}\end{entrée}

\begin{entrée}{anɯɣbɯɣbɯɣ}{}{ⓔanɯɣbɯɣbɯɣ}\relationsémantique{参考}{\lien{ⓔnɯɣbɯɣ}{nɯɣbɯɣ}}\end{entrée}

\begin{entrée}{anɯkɯjŋɤŋgɯ}{}{ⓔanɯkɯjŋɤŋgɯ} 
\classe{vi}  
\grammaire{recip} \paradigme{dir}{nɯ-}
\begin{définition}\pfra{se jurer l'un à l'autre}\end{définition}
\begin{définition}\pcmn{互相立下誓言}\end{définition}
\begin{exemple}\pjya{ɲɤ-k-ɤnɯkɯjŋɤŋgɯ-ndʑi}\hspace{5pt}\pcmn{他们俩互相立下了誓言}\end{exemple}\relationsémantique{同义词}{\lien{ⓔanɯkɯjŋɯjŋu}{anɯkɯjŋɯjŋu}}\relationsémantique{同义词}{\lien{ⓔnɯkɯjŋu}{nɯkɯjŋu}}\end{entrée}

\begin{entrée}{anɯkɯjŋɯjŋu}{}{ⓔanɯkɯjŋɯjŋu} 
\classe{vi}  
\grammaire{recip} \paradigme{dir}{tɤ-}\paradigme{dir}{nɯ-}
\begin{définition}\pfra{se jurer l'un à l'autre}\end{définition}
\begin{définition}\pcmn{互相立下誓言}\end{définition}
\begin{exemple}\pjya{to-k-ɤnɯkɯjŋɯjŋu-ndʑi}\hspace{5pt}\pcmn{他们俩互相立下了誓言}\end{exemple}\relationsémantique{同义词}{\lien{}{anɯkhɯjŋɤŋgɯ}}\relationsémantique{参考}{\lien{ⓔnɯkɯjŋu}{nɯkɯjŋu}}\end{entrée}

\begin{entrée}{anɯmbɯmbɣom}{}{ⓔanɯmbɯmbɣom}\relationsémantique{参考}{\lien{ⓔnɯmbɣom}{nɯmbɣom}}\end{entrée}

\begin{entrée}{anɯmqaj}{}{ⓔanɯmqaj} 
\classe{vi} \paradigme{dir}{tɤ-}
\begin{définition}\pfra{se disputer}\end{définition}
\begin{définition}\pcmn{吵架}\end{définition}
\begin{exemple}\pjya{jiɕqha kɤtsa ra ɲɯ-ɤnɯmqaj-nɯ}\hspace{5pt}\pcmn{他们一家在吵架}\end{exemple}
\begin{exemple}\pjya{to-k-ɤnɯmqaj-ndʑi}\hspace{5pt}\pcmn{他们吵架了}\end{exemple}\relationsémantique{参考}{\lien{ⓔtɯ-mqaj}{tɯ-mqaj}}\relationsémantique{参考}{\lien{ⓔamqaj}{amqaj}}\end{entrée}

\begin{entrée}{anɯndzɤqɯqɤr}{}{ⓔanɯndzɤqɯqɤr}\relationsémantique{参考}{\lien{ⓔnɯndzɤqɤr}{nɯndzɤqɤr}}\end{entrée}

\begin{entrée}{anɯpɕɯpɕoʁ}{}{ⓔanɯpɕɯpɕoʁ} 
\classe{vi}  
\grammaire{denom} \paradigme{dir}{tɤ-}
\begin{définition}\pfra{tourné dans la même direction}\end{définition}
\begin{définition}\pcmn{朝同一个方向;方向准确}\end{définition}
\begin{exemple}\pjya{jiɕqha nɯ kɯ-ɤnɯpɕɯpɕoʁ ci ɲɯ-ŋu}\hspace{5pt}\pcmn{他朝同一个方向}\end{exemple}
\begin{exemple}\pjya{ɲɯ-ɤnɯpɕɯpɕoʁ-ndʑi}\hspace{5pt}\pcmn{他们俩朝同一个方向}\end{exemple}
\begin{exemple}\pjya{ki tɯ-rju ʁnɯ-ŋka ki mɯ-ɲɯ-ɤnɯpɕɯpɕoʁ kɯ ɲɯ-ɤnɯɕqhɯɕqhu ɕti}\hspace{5pt}\pcmn{这两句话不相符合,是矛盾的}\end{exemple}\relationsémantique{参考}{\lien{ⓔtɯ-pɕoʁ}{tɯ-pɕoʁ}}\end{entrée}

\begin{entrée}{anɯphɤrɯri}{}{ⓔanɯphɤrɯri} 
\classe{vi}  
\grammaire{denom} \paradigme{dir}{nɯ-}
\begin{définition}\pfra{être l'un en face de l'autre}\end{définition}
\begin{définition}\pcmn{一个对着一个(隔着河、山对着山)}\end{définition}
\begin{exemple}\pjya{kɯki jiʑora ɲɯ-ɤnɯphɤrɯri-j}\hspace{5pt}\pcmn{我们隔着(河)相对}\end{exemple}
\begin{exemple}\pjya{tɕɤndi phɤri ra cho ɲɯ-ɤnɯphɤrɯri-j}\hspace{5pt}\pcmn{我们跟对面的(那一家人)隔着(河)相对}\end{exemple}
\begin{exemple}\pjya{tɯji cho praʁ ɲɯ-ɤnɯphɤrɯri-ndʑi}\hspace{5pt}\pcmn{这面山是田地,那一面是岩石(相对着)}\end{exemple}\relationsémantique{参考}{\lien{ⓔphɤri}{phɤri}}\end{entrée}

\begin{entrée}{anɯphotɯtɯɣ}{}{ⓔanɯphotɯtɯɣ} 
\classe{vi} 
\begin{définition}\pfra{moyen}\end{définition}
\begin{définition}\pcmn{恰当的;中等的;不多不少}\end{définition}
\begin{exemple}\pjya{kɯ-ɤnɯphotɯtɯɣ ci ra}\hspace{5pt}\pcmn{需要一个中等的}\end{exemple}
\begin{exemple}\pjya{tɯ-rju nɯ ɣɯ ɯ-tsa tsa tɤ-βze, kɯ-ɤnɯphotɯtɯɣ tsa tɤ-βze a-mɤ-tɤ-tɕhom}\hspace{5pt}\pcmn{话要讲得适当一些,不要太多}\end{exemple}\end{entrée}

\begin{entrée}{anɯrgɯrga}{}{ⓔanɯrgɯrga}\relationsémantique{参考}{\lien{ⓔnɯrga}{nɯrga}}\end{entrée}

\begin{entrée}{anɯrŋɤrɯru}{}{ⓔanɯrŋɤrɯru} 
\classe{vi}  
\grammaire{refl}
\grammaire{incorp} \paradigme{dir}{kɤ-}
\begin{définition}\pfra{se regarder les uns les autres}\end{définition}
\begin{définition}\pcmn{互相望着}\end{définition}
\begin{exemple}\pjya{tɕiʑo kɤ-anɯrŋɤrɯru-tɕi}\hspace{5pt}\pcmn{我们俩互相望了一眼}\end{exemple}\relationsémantique{参考}{\lien{ⓔtɯ-rŋa}{tɯ-rŋa}}\relationsémantique{参考}{\lien{ⓔruⓗ1}{ru₁}}\end{entrée}

\begin{entrée}{anɯrqhɯrqhu}{}{ⓔanɯrqhɯrqhu} 
\classe{vi} \paradigme{dir}{tɤ-}
\begin{définition}\pfra{l'un dos à l'autre}\end{définition}
\begin{définition}\pcmn{背对背}\end{définition}
\begin{exemple}\pjya{tɕiʑo ni tɤ-anɯrqhɯrqhu-tɕi}\hspace{5pt}\pcmn{我们俩背对着背}\end{exemple}
\begin{exemple}\pjya{kɤ-nɯ-rŋgɯ-tɕi tɕe tɤ-anɯrqhɯrqhu-tɕi}\hspace{5pt}\pcmn{我们俩睡觉的时候背对着背}\end{exemple}\relationsémantique{参考}{\lien{ⓔɯ-qhu}{ɯ-qhu}}\end{entrée}

\begin{entrée}{anɯrɯɕmɯɕmi}{}{ⓔanɯrɯɕmɯɕmi}\relationsémantique{参考}{\lien{ⓔrɯɕmi}{rɯɕmi}}\end{entrée}

\begin{entrée}{anɯrɯtʂɯtʂa}{}{ⓔanɯrɯtʂɯtʂa}\relationsémantique{参考}{\lien{ⓔnɯrɯtʂa}{nɯrɯtʂa}}\end{entrée}

\begin{entrée}{anɯʁɤndɯndu}{}{ⓔanɯʁɤndɯndu} 
\classe{vi} \paradigme{dir}{tɤ-}
\begin{définition}\pfra{s'échanger des travaux les uns les autres}\end{définition}
\begin{définition}\pcmn{互相换工}\end{définition}
\begin{exemple}\pjya{tɕiʑo anɯʁɤndɯndu-tɕi}\hspace{5pt}\pcmn{我们俩互相换工}\end{exemple}\relationsémantique{参考}{\lien{ⓔtaʁɤndu}{taʁɤndu}}\end{entrée}

\begin{entrée}{anɯʁɤrɯri}{}{ⓔanɯʁɤrɯri} 
\classe{vi}  
\grammaire{denom} \paradigme{dir}{tɤ-}
\begin{définition}\pfra{l'un en face de l'autre}\end{définition}
\begin{définition}\pcmn{面对面}\end{définition}
\begin{exemple}\pjya{jiɕqha nɯni ɲɯ-ɤnɯʁɤrɯri-ndʑi}\hspace{5pt}\pcmn{他们俩面对面}\end{exemple}
\begin{exemple}\pjya{tɕiʑo ni tu-onɯʁɤrɯri-tɕi ŋu}\hspace{5pt}\pcmn{我们俩面对面}\end{exemple}
\begin{exemple}\pjya{to-k-ɤnɯʁɤrɯri-ndʑi-ci}\hspace{5pt}\pcmn{他俩面对面}\end{exemple}
\begin{exemple}\pjya{tu-onɯʁɤrɯri-ndʑi tɕe ku-omdzɯ-ndʑi}\hspace{5pt}\pcmn{他们俩对着坐}\end{exemple}\relationsémantique{参考}{\lien{ⓔɯ-ʁɤri}{ɯ-ʁɤri}}\end{entrée}

\begin{entrée}{anɯʁgrɯʁgra}{}{ⓔanɯʁgrɯʁgra}\relationsémantique{参考}{\lien{ⓔnɯʁgra}{nɯʁgra}}\end{entrée}

\begin{entrée}{anɯstɯstu}{}{ⓔanɯstɯstu} 
\classe{vi} \paradigme{dir}{tɤ-}
\begin{définition}\pfra{être sur une même ligne}\end{définition}
\begin{définition}\pcmn{呈一条直线}\end{définition}
\begin{exemple}\pjya{khɯtsa cho ɕoʁɕoʁ ɲɯ-ɤnɯstɯstu-ndʑi}\hspace{5pt}\pcmn{碗和纸在同一水平线上}\end{exemple}\relationsémantique{同义词}{\lien{ⓔanɯpɕɯpɕoʁ}{anɯpɕɯpɕoʁ}}\relationsémantique{参考}{\lien{ⓔastu}{astu}}\end{entrée}

\begin{entrée}{anɯsɯkhɯkho}{}{ⓔanɯsɯkhɯkho} 
\classe{vi}  
\grammaire{recip} \paradigme{dir}{nɯ-}
\begin{définition}\pfra{se battre pour}\end{définition}
\begin{définition}\pcmn{争夺;互相抢}\end{définition}
\begin{exemple}\pjya{paʁtsa ni ndʑi-ndza ɲɯ-ɤnɯsɯkhɯkho-ndʑi}\hspace{5pt}\pcmn{两个小猪互相抢食物}\end{exemple}
\begin{exemple}\pjya{tɤ-pɤtso ni ndʑi-kɯmtɕhɯ ɲɯ-ɤnɯsɯkhɯkho-ndʑi}\hspace{5pt}\pcmn{两个小孩子互相抢玩具}\end{exemple}\relationsémantique{参考}{\lien{ⓔnɯsɯkho}{nɯsɯkho}}\end{entrée}

\begin{entrée}{anɯsɯslaʁ}{}{ⓔanɯsɯslaʁ} 
\classe{vi} \paradigme{dir}{tɤ-}
\begin{définition}\pfra{assidu, qui travaille rapidement}\end{définition}
\begin{définition}\pcmn{勤快}\end{définition}
\begin{exemple}\pjya{ta-ma ɲɯ-ɤnɯsɯslaʁ}\hspace{5pt}\pcmn{他工作做得很快(很勤快)}\end{exemple}
\begin{exemple}\pjya{nɯ ta-ma kɤ-nɤma tu-kɯ-ɤnɯsɯslaʁ ʑo jɤɣ wo}\hspace{5pt}\pcmn{这个工作可以做得快一些(长辈对晚辈说的话)}\end{exemple}
\begin{exemple}\pjya{kɯ-rɤma jɤ-tɯ-ari tɕe, a-tɤ-tɯ-ɤnɯsɯslaʁ ʑo je!}\hspace{5pt}\pcmn{你去工作的时候要勤快一些!}\end{exemple}
\begin{exemple}\pjya{ɯʑo ta-ma pe tɕe, tu-onɯsɯslaʁ ʑo ɕti}\hspace{5pt}\pcmn{他工作得很好,很勤快}\end{exemple}
\begin{exemple}\pjya{to-k-ɤnɯsɯslaʁ-ci tɕe, ʑaʑa ɲɤ-sthɯt}\hspace{5pt}\pcmn{因为他很勤快所以早就完工了}\end{exemple}\end{entrée}

\begin{entrée}{aɲaj}{}{ⓔaɲaj} 
\classe{vi} \paradigme{dir}{tɤ-}
\begin{définition}\pfra{rapide (travail)}\end{définition}
\begin{définition}\pcmn{迅速;快(工作)}\end{définition}
\begin{exemple}\pjya{jiɕqha nɯ ɯ-ma ɲɯ-ɤɲaj}\hspace{5pt}\pcmn{他工作得很快}\end{exemple}
\begin{exemple}\pjya{aʑo a-ma aɲaj}\hspace{5pt}\pcmn{我工作得很快}\end{exemple}
\begin{exemple}\pjya{to-k-ɤɲaj-ci}\hspace{5pt}\pcmn{比以前快}\end{exemple}\relationsémantique{参考}{\lien{ⓔsɤɲaj}{sɤɲaj}}\end{entrée}

\begin{entrée}{aɲaʁndzɯm}{}{ⓔaɲaʁndzɯm} 
\classe{vi} \paradigme{dir}{tɤ-}
\begin{définition}\pfra{marron foncé}\end{définition}
\begin{définition}\pcmn{黑褐色}\end{définition}
\begin{exemple}\pjya{tʂha kɤ-ta-t-a ri, nɯ-ala, ɲɤ-ɬoʁ tɕe ɲɤ-k-ɤɲaʁdzɯm-ci}\hspace{5pt}\pcmn{我把茶烧开了,把茶叶熬出来了就变成黑褐色}\end{exemple}\end{entrée}

\begin{entrée}{aɲɟoʁ}{}{ⓔaɲɟoʁ}\relationsémantique{参考}{\lien{ⓔɲɟoʁ}{ɲɟoʁ}}\end{entrée}

\begin{entrée}{aŋgɤjom}{}{ⓔaŋgɤjom} 
\classe{vs} \paradigme{dir}{nɯ-}
\begin{définition}\pfra{large, ouvert}\end{définition}
\begin{définition}\pcmn{宽敞}\end{définition}
\begin{exemple}\pjya{tɯ-ŋga ɲɯ-ɤŋgɤjom}\hspace{5pt}\pcmn{衣服很宽}\end{exemple}
\begin{exemple}\pjya{kha ɲɯ-ɤŋgɤjom}\hspace{5pt}\pcmn{房子很宽}\end{exemple}
\begin{exemple}\pjya{ɲɤ-k-ɤŋgɤjom-ci}\hspace{5pt}\pcmn{变宽了}\end{exemple}\relationsémantique{反义词}{\lien{ⓔaŋgɤrŋgɤr}{aŋgɤrŋgɤr}}\relationsémantique{参考}{\lien{ⓔjom}{jom}}\end{entrée}

\begin{entrée}{aŋgɤrŋgɤr}{}{ⓔaŋgɤrŋgɤr} 
\classe{vs} \paradigme{dir}{kɤ-}
\begin{définition}\pfra{à l'étroit}\end{définition}
\begin{définition}\pcmn{拥挤(房子)}\end{définition}
\begin{exemple}\pjya{ko-k-ɤŋgɤrŋgɤr-ci}\hspace{5pt}\pcmn{变得很拥挤}\end{exemple}
\begin{exemple}\pjya{tɕi-sta ɲɯ-ɤŋgɤrŋgɤr}\hspace{5pt}\pcmn{我们俩的房间很窄}\end{exemple}
\begin{exemple}\pjya{ki tɕi-sta ko-k-ɤŋgɤrŋgɤr-ci}\hspace{5pt}\pcmn{我们俩的房间比以前拥挤了}\end{exemple}\relationsémantique{参考}{\lien{ⓔŋgɤr}{ŋgɤr}}\relationsémantique{反义词}{\lien{ⓔaŋgɤjom}{aŋgɤjom}}\end{entrée}

\begin{entrée}{aŋgorji}{}{ⓔaŋgorji} 
\classe{vs} 
\begin{définition}\pfra{calme, silencieux}\end{définition}
\begin{définition}\pcmn{安静}\end{définition}
\begin{exemple}\pjya{ɲɯ-ɤŋgorji}\hspace{5pt}\pcmn{很安静}\end{exemple}
\begin{exemple}\pjya{kɯ-rɤma ra tɤ-nɯna-nɯ tɕe ɲɯ-ɤŋgorji}\hspace{5pt}\pcmn{打工的人休息了,现在很安静}\end{exemple}\end{entrée}

\begin{entrée}{apa}{}{ⓔapa} 
\classe{vi} \paradigme{dir}{nɯ-}\sens{1}
\begin{définition}\pfra{devenir}\end{définition}
\begin{définition}\pcmn{变成(自然形成的)}\end{définition}\sens{2}
\begin{définition}\pfra{être fermé}\end{définition}
\begin{définition}\pcmn{关着}\end{définition}
\begin{exemple}\pjya{kɯm ɲɯ-ɤpa}\hspace{5pt}\pcmn{门是关着的}\end{exemple}\sens{3}
\begin{définition}\pfra{félicitation pour...}\end{définition}
\begin{définition}\pcmn{恭喜……}\end{définition}
\begin{exemple}\pjya{ɲɯ-tɯ-mna tɕe pɯ-apa}\hspace{5pt}\pcmn{你痊愈了,恭喜你了}\end{exemple}\sens{4}
\begin{définition}\pfra{être en tort (au négatif)}\end{définition}
\begin{définition}\pcmn{有错(否定式)}\end{définition}
\begin{exemple}\pjya{tɯʑo mɯ-pɯ-kɯ-ɤpa tɕe tɯ-zda ɯ-ɕki kɤ-ti mɤ-mbat}\hspace{5pt}\pcmn{自己有错的时候不好说别人}\end{exemple}
\begin{exemple}\pjya{nɯ mɯ-pɯ-apa-a ko!}\hspace{5pt}\pcmn{很对不起!(我没有对)}\end{exemple}
\begin{sous-entrée}{tɕhi napapa}{ⓔapaⓢ4ⓝtɕhi napapa} 
\classe{adv} 
\begin{définition}\pfra{quoi qu'il arrive}\end{définition}
\begin{définition}\pcmn{不顾一切,不管发生什么}\end{définition}
\begin{exemple}\pjya{tɕhi napapa ʑo kɤ-nɤɕqa ɬoʁ}\hspace{5pt}\pcmn{不管发生什么都要坚持}\end{exemple}\relationsémantique{参考}{\lien{ⓔpaⓗ3}{pa}}\relationsémantique{参考}{\lien{ⓔsɤpa}{sɤpa}}\end{sous-entrée}

\end{entrée}

\begin{entrée}{apɤmbat}{}{ⓔapɤmbat} 
\classe{vi}  
\grammaire{comp} \paradigme{dir}{tɤ-}\paradigme{dir}{tɤ-}
\begin{définition}\pfra{facile à faire}\end{définition}
\begin{définition}\pcmn{容易做}\end{définition}
\begin{définition}\pfra{simplifier}\end{définition}
\begin{définition}\pcmn{简化}\end{définition}
\begin{exemple}\pjya{jiɕqha nɯ ʁo kɯ-ɤpɤmbat ci ɲɯ-ŋu}\hspace{5pt}\pcmn{这倒是一件容易的事情}\end{exemple}
\begin{exemple}\pjya{jiɕqha kɤ-nɤma ʁo ɲɯ-ɤpɤmbat}\hspace{5pt}\pcmn{这个工作倒很容易做}\end{exemple}
\begin{exemple}\pjya{ɯ-ɲɯ-ɤpɤmbat}\hspace{5pt}\pcmn{容不容易?}\end{exemple}
\begin{exemple}\pjya{mɯ-ɲɯ-ɤpɤmbat}\hspace{5pt}\pcmn{不容易}\end{exemple}\relationsémantique{参考}{\lien{ⓔmbat}{mbat}}\relationsémantique{参考}{\lien{ⓔpaⓗ3}{pa}}
\begin{sous-entrée}{sɤpɤmbat}{ⓔapɤmbatⓝsɤpɤmbat} 
\classe{vt} \end{sous-entrée}

\end{entrée}

\begin{entrée}{apɤɴqa}{}{ⓔapɤɴqa} 
\classe{vi} \paradigme{dir}{tɤ-}
\begin{définition}\pfra{dur à faire}\end{définition}
\begin{définition}\pcmn{难做}\end{définition}
\begin{exemple}\pjya{jiɕqha kɤ-nɤma ɲɯ-ɤpɤɴqa}\hspace{5pt}\pcmn{这个工作很难做}\end{exemple}\relationsémantique{反义词}{\lien{ⓔapɤmbat}{apɤmbat}}\relationsémantique{参考}{\lien{ⓔɴqa}{ɴqa}}\relationsémantique{参考}{\lien{ⓔpaⓗ3}{pa}}\end{entrée}

\begin{entrée}{apɤpɣi}{}{ⓔapɤpɣi} 
\classe{vs} 
\begin{définition}\pfra{gris}\end{définition}
\begin{définition}\pcmn{带有灰色}\end{définition}\relationsémantique{参考}{\lien{ⓔpɣi}{pɣi}}\end{entrée}

\begin{entrée}{apɕɯβjɤl}{}{ⓔapɕɯβjɤl} 
\classe{vi} \paradigme{dir}{tɤ-}\paradigme{dir}{tɤ-}
\begin{définition}\pfra{en biais, penché}\end{définition}
\begin{définition}\pcmn{斜着}\end{définition}
\begin{définition}\pfra{placer en biais, penché}\end{définition}
\begin{définition}\pcmn{斜着放}\end{définition}
\begin{exemple}\pjya{ndzom ɲɯ-ɤpɕɯβjɤl tɕe mɯ́j-nɯɣɯŋke}\hspace{5pt}\pcmn{桥是斜着的,不好走}\end{exemple}
\begin{exemple}\pjya{tɕhi kɤ-thɯ tɕe tú-wɣ-sɤpɕɯβjɤl ra}\hspace{5pt}\pcmn{搭梯子的时候,要斜着搭}\end{exemple}
\begin{sous-entrée}{sɤpɕɯβjɤl}{ⓔapɕɯβjɤlⓝsɤpɕɯβjɤl} 
\classe{vt} \end{sous-entrée}

\end{entrée}

\begin{entrée}{apɣaʁsci}{}{ⓔapɣaʁsci} 
\classe{vi}  
\grammaire{caus} \paradigme{dir}{tɤ-}
\begin{définition}\pfra{se retourner}\end{définition}
\begin{définition}\pcmn{翻过去}\end{définition}
\begin{sous-entrée}{sɤpɣaʁsci}{ⓔapɣaʁsciⓝsɤpɣaʁsci} 
\classe{vt} 
\begin{définition}\pfra{retourner}\end{définition}
\begin{définition}\pcmn{翻过去}\end{définition}
\begin{exemple}\pjya{qajɣi ta-sɤpɣaʁsci}\hspace{5pt}\pcmn{他翻了馍馍}\end{exemple}\end{sous-entrée}

\begin{sous-entrée}{ʑɣɤsɤpɣaʁsci}{ⓔapɣaʁsciⓝʑɣɤsɤpɣaʁsci} 
\classe{vi}  
\grammaire{refl}
\grammaire{caus} 
\begin{définition}\pfra{se retourner}\end{définition}
\begin{définition}\pcmn{翻身}\end{définition}
\begin{exemple}\pjya{tɤ-ʑɣɤsɤpɣaʁsci-a}\hspace{5pt}\pcmn{我翻了身(例如,睡觉的时候)}\end{exemple}
\begin{exemple}\pjya{tɤ-ʑɣɤsɤpɣaʁsci}\hspace{5pt}\pcmn{他翻了身}\end{exemple}
\begin{exemple}\pjya{tu-tɯ-ʑɣɤsɤpɣaʁsci ntsɯ ɲɯ-ŋu, ɲɯ-sɤɣdɯɣ}\hspace{5pt}\pcmn{你不停地翻身,很烦}\end{exemple}\end{sous-entrée}

\end{entrée}

\begin{entrée}{apɣɤpɣi}{}{ⓔapɣɤpɣi} 
\classe{vi} \paradigme{dir}{thɯ-}
\begin{définition}\pfra{grisâtre}\end{définition}
\begin{définition}\pcmn{淡灰色}\end{définition}
\begin{exemple}\pjya{tɯ-ŋga ɯ-mdoʁ kɯ-ɤpɣɤpɣi ci ɲɯ-ŋu}\hspace{5pt}\pcmn{衣服的颜色是淡灰色}\end{exemple}
\begin{exemple}\pjya{rɯdaʁ kɯ-ɤpɣɤpɣi ci ɲɯ-ŋu}\hspace{5pt}\pcmn{是一头淡灰色的野兽}\end{exemple}\relationsémantique{同义词}{\lien{ⓔapɣɯlu}{apɣɯlu}}\relationsémantique{参考}{\lien{ⓔpɣi}{pɣi}}\end{entrée}

\begin{entrée}{apɣɯlu}{}{ⓔapɣɯlu} 
\classe{vi} \paradigme{dir}{thɯ-}
\begin{définition}\pfra{grisâtre}\end{définition}
\begin{définition}\pcmn{灰扑扑}\end{définition}
\begin{exemple}\pjya{jiɕqha nɯ kɯ-ɤpɣɯlu ci ɲɯ-ŋu}\hspace{5pt}\pcmn{刚才那个是灰扑扑的}\end{exemple}\relationsémantique{同义词}{\lien{ⓔapɤpɣi}{apɤpɣi}}\relationsémantique{参考}{\lien{ⓔpɣi}{pɣi}}\end{entrée}

\begin{entrée}{aphala}{}{ⓔaphala} 
\classe{vi} 
\begin{définition}\pfra{ayant des mouvement alertes}\end{définition}
\begin{définition}\pcmn{动作灵活(小伙子)}\end{définition}
\begin{exemple}\pjya{kɯ-ɤphala ci ɲɯ-ŋu}\hspace{5pt}\pcmn{他动作很灵活}\end{exemple}\relationsémantique{同义词}{\lien{ⓔaɕpala}{aɕpala}}\end{entrée}

\begin{entrée}{aphɤlɤjɤt}{}{ⓔaphɤlɤjɤt} 
\classe{vs} \paradigme{dir}{pɯ-}
\begin{définition}\pfra{en désordre}\end{définition}
\begin{définition}\pcmn{凌乱}\end{définition}
\begin{exemple}\pjya{a-mɤ-pɯ-ɤphɤlɤjɤt kɯ tɤ-rɤwum}\hspace{5pt}\pcmn{(东西)不要这么乱,收拾一下}\end{exemple}\relationsémantique{同义词}{\lien{ⓔadrɤt}{adrɤt}}
\begin{sous-entrée}{sɤphɤlɤjɤt}{ⓔaphɤlɤjɤtⓝsɤphɤlɤjɤt} 
\classe{vt} 
\begin{définition}\pfra{mettre en désordre}\end{définition}
\begin{définition}\pcmn{乱放}\end{définition}\end{sous-entrée}

\end{entrée}

\begin{entrée}{apjɤntɤm}{}{ⓔapjɤntɤm} 
\classe{vs} \paradigme{dir}{nɯ-}\paradigme{dir}{nɯ-}\paradigme{dir}{thɯ-}
\begin{définition}\pfra{plat}\end{définition}
\begin{définition}\pcmn{平整}\end{définition}
\begin{définition}\pfra{aplanir}\end{définition}
\begin{définition}\pcmn{弄平}\end{définition}
\begin{exemple}\pjya{khɤxtu ɲɯ-ɤpjɤntɤm}\hspace{5pt}\pcmn{房背是平的}\end{exemple}
\begin{exemple}\pjya{sɤtɕha nɯ-sɤpjɤntam-a}\hspace{5pt}\pcmn{我把地弄平了}\end{exemple}
\begin{exemple}\pjya{kha ɯ-sta ɯ-spa nɯ kɤ-sɤpjɤntɤm ra}\hspace{5pt}\pcmn{要把房基(修房子的地方)弄平}\end{exemple}\relationsémantique{参考}{\lien{ⓔantɤm}{antɤm}}
\begin{sous-entrée}{sɤpjɤntɤm}{ⓔapjɤntɤmⓝsɤpjɤntɤm} 
\classe{vt}  
\grammaire{caus} \end{sous-entrée}

\end{entrée}

\begin{entrée}{apupu}{}{ⓔapupu} 
\classe{vi} \paradigme{dir}{thɯ-}\paradigme{dir}{tɤ-}
\begin{définition}\pfra{prospère}\end{définition}
\begin{définition}\pcmn{美满}\end{définition}
\begin{exemple}\pjya{nɯnɯ tɯrme nɯ nɯ ɕɯŋgɯ tɕe wuma ʑo pɯ-ŋgɯ ri, tham tɕe chɤ-mɤɕi tɕe cho-k-ɤpupu-ci}\hspace{5pt}\pcmn{那个人以前很穷,现在变得很富有了}\end{exemple}\end{entrée}

\begin{entrée}{apɯpa}{}{ⓔapɯpa} 
\classe{vi} \paradigme{dir}{tɤ-}\sens{1}
\begin{définition}\pfra{s'accumuler}\end{définition}
\begin{définition}\pcmn{积累}\end{définition}
\begin{exemple}\pjya{nɤ-smɤn khro to-k-ɤpɯpa-ci}\hspace{5pt}\pcmn{你的药积累了很多}\end{exemple}\relationsémantique{同义词}{\lien{ⓔajtɯ}{ajtɯ}}\sens{2}
\begin{définition}\pfra{être riche}\end{définition}
\begin{définition}\pcmn{富裕}\end{définition}\relationsémantique{同义词}{\lien{ⓔɯ-ŋgu,thon}{ɯ-ŋgu,thon}}\end{entrée}

\begin{entrée}{apɯpri}{}{ⓔapɯpri} 
\classe{vs} 
\begin{définition}\pfra{en continu}\end{définition}
\begin{définition}\pcmn{连续不断}\end{définition}
\begin{exemple}\pjya{kɯ-ɤpɯpri laʁnɯ-sŋi ʑo pjɤ-rɤʑi}\hspace{5pt}\pcmn{他连续待了几天}\end{exemple}
\begin{exemple}\pjya{tɯ-xpa ɯ-ŋgɯ kɯ-ɤpɯpri ʑo tɤ-ngo-a}\hspace{5pt}\pcmn{我在一年里生病了很多次}\end{exemple}
\begin{sous-entrée}{sɤpɯpri}{ⓔapɯpriⓝsɤpɯpri} 
\classe{vt} 
\begin{définition}\pfra{faire en continu}\end{définition}
\begin{définition}\pcmn{连续做……}\end{définition}\end{sous-entrée}

\end{entrée}

\begin{entrée}{aqandʐɯlu}{}{ⓔaqandʐɯlu} 
\classe{vi} \paradigme{dir}{nɯ-}
\begin{définition}\pfra{noirâtre, sombre, violacé}\end{définition}
\begin{définition}\pcmn{乌黑;紫}\end{définition}
\begin{exemple}\pjya{tɯ-mɯ nɤmkha zdɯm kɯ-ɤqandʐɯlu ʑo ko-ɣi}\hspace{5pt}\pcmn{天上来了乌云}\end{exemple}\relationsémantique{同义词}{\lien{ⓔapɣɯlu}{apɣɯlu}}\end{entrée}

\begin{entrée}{aqarŋɤmbru}{}{ⓔaqarŋɤmbru} 
\classe{vi} \paradigme{dir}{thɯ-}
\begin{définition}\pfra{jaune pâle}\end{définition}
\begin{définition}\pcmn{淡黄色}\end{définition}
\begin{exemple}\pjya{nɤ-ŋga kɯ-ɤqarŋɤmbru ɲɯ-ŋu}\hspace{5pt}\pcmn{你的衣服是淡黄色的}\end{exemple}
\begin{exemple}\pjya{ɲɯ-ɤqarŋɤmbru}\hspace{5pt}\pcmn{是淡黄色的}\end{exemple}\end{entrée}

\begin{entrée}{aqarŋɯrŋe}{}{ⓔaqarŋɯrŋe} 
\classe{vs} 
\begin{définition}\pfra{jaune clair}\end{définition}
\begin{définition}\pcmn{淡黄}\end{définition}
\begin{exemple}\pjya{ɲɯ-ɤqarŋɯrŋe}\hspace{5pt}\pcmn{是淡黄色的}\end{exemple}\relationsémantique{参考}{\lien{ⓔqarŋe}{qarŋe}}\relationsémantique{反义词}{\lien{ⓔqarŋɯrŋe}{qarŋɯrŋe}}\end{entrée}

\begin{entrée}{aqɤrle}{}{ⓔaqɤrle} 
\classe{vi} \paradigme{dir}{nɯ-}
\begin{définition}\pfra{être séparé}\end{définition}
\begin{définition}\pcmn{分开的}\end{définition}
\begin{exemple}\pjya{stoʁ cho staχpɯ aqɤrle}\hspace{5pt}\pcmn{胡豆和豌豆是分开的}\end{exemple}\relationsémantique{参考}{\lien{ⓔsɤqɤrle}{sɤqɤrle}}\end{entrée}

\begin{entrée}{aqɤtsa}{}{ⓔaqɤtsa} 
\classe{vi} \paradigme{dir}{tɤ-}
\begin{définition}\pfra{préparé}\end{définition}
\begin{définition}\pcmn{组装起来(机器、器具)}\end{définition}
\begin{sous-entrée}{sɤqɤtsa}{ⓔaqɤtsaⓝsɤqɤtsa} 
\classe{vt} 
\begin{définition}\pfra{préparer}\end{définition}
\begin{définition}\pcmn{装备好}\end{définition}
\begin{exemple}\pjya{@luyinji tɤ-sɤqɤtse}\hspace{5pt}\pcmn{你把录音机准备好}\end{exemple}
\begin{exemple}\pjya{mbɣo tɤ-sɤqɤtse}\hspace{5pt}\pcmn{你把犁组装起来吧}\end{exemple}
\begin{exemple}\pjya{βɣa to-sɤqɤtsa}\hspace{5pt}\pcmn{他把磨坊准备好了}\end{exemple}\relationsémantique{参考}{\lien{ⓔnɤqɤtsa}{nɤqɤtsa}}\end{sous-entrée}

\end{entrée}

\begin{entrée}{aqɤtʂha}{}{ⓔaqɤtʂha} 
\classe{vi} \paradigme{dir}{nɯ-}\paradigme{dir}{pɯ-}
\begin{définition}\pfra{être croisé}\end{définition}
\begin{définition}\pcmn{交叉}\end{définition}
\begin{définition}\pfra{croiser}\end{définition}
\begin{définition}\pcmn{使交叉}\end{définition}
\begin{exemple}\pjya{tɤ-ri ɲɯ-ɤqɤtʂha}\hspace{5pt}\pcmn{线互相交叉}\end{exemple}
\begin{exemple}\pjya{si ra pjɤ-k-ɤqɤtʂha-ci}\hspace{5pt}\pcmn{树交叉在一起}\end{exemple}
\begin{exemple}\pjya{ɯ-mɤlɤjaʁ pjɤ-sɤqɤtʂha}\hspace{5pt}\pcmn{它把四肢交叉在一起}\end{exemple}
\begin{exemple}\pjya{tɯmbri ra ɲɤ-sɤqɤtʂha-nɯ}\hspace{5pt}\pcmn{他们把绳子交叉了起来}\end{exemple}
\begin{exemple}\pjya{laʁjɯɣ ɲɤ-sɤqɤtʂha}\hspace{5pt}\pcmn{他把棍子交叉了起来}\end{exemple}
\begin{exemple}\pjya{ɯ-mi pjɤ-sɤqɤtʂha}\hspace{5pt}\pcmn{他把脚交叉了起来}\end{exemple}
\begin{exemple}\pjya{aʑo tɤ-ri pɯ-sɤqɤtʂha-t-a}\hspace{5pt}\pcmn{我把线交叉在一起}\end{exemple}
\begin{sous-entrée}{sɤqɤtʂha}{ⓔaqɤtʂhaⓝsɤqɤtʂha} 
\classe{vt}  
\grammaire{caus} \end{sous-entrée}

\end{entrée}

\begin{entrée}{aqhe}{}{ⓔaqhe} 
\classe{vs} \paradigme{dir}{nɯ-}\paradigme{dir}{nɯ-}
\begin{définition}\pfra{guérir}\end{définition}
\begin{définition}\pcmn{痊愈}\end{définition}
\begin{définition}\pfra{guérir}\end{définition}
\begin{définition}\pcmn{治疗}\end{définition}
\begin{exemple}\pjya{ɯ-ku ɯ-kɯ-mŋɤm ɲɤ-k-ɤqhe-ci}\hspace{5pt}\pcmn{他头痛痊愈了}\end{exemple}
\begin{exemple}\pjya{a-kɯ-mŋɤm na-sɤqhe}\hspace{5pt}\pcmn{他治好了我的病}\end{exemple}\relationsémantique{同义词}{\lien{ⓔɣɤmna}{ɣɤmna}}
\begin{sous-entrée}{sɤqhe}{ⓔaqheⓝsɤqhe} 
\classe{vt} \end{sous-entrée}

\end{entrée}

\begin{entrée}{aqhoβlu}{}{ⓔaqhoβlu} 
\classe{vs} 
\begin{définition}\pfra{concave}\end{définition}
\begin{définition}\pcmn{凹进去}\end{définition}\relationsémantique{参考}{\lien{}{aχchowɤlu}}\relationsémantique{参考}{\lien{ⓔʁlɯβʁlɯβ}{ʁlɯβʁlɯβ}}\end{entrée}

\begin{entrée}{aqhowolu}{}{ⓔaqhowolu} 
\classe{vs} 
\begin{définition}\pfra{concave}\end{définition}
\begin{définition}\pcmn{凹}\end{définition}\relationsémantique{同义词}{\lien{ⓔaχchowolu}{aχchowolu}}\relationsémantique{同义词}{\lien{ⓔaʁloʁlu}{aʁloʁlu}}\relationsémantique{同义词}{\lien{ⓔaɕqhlu}{aɕqhlu}}\relationsémantique{同义词}{\lien{ⓔasqhlu}{asqhlu}}\relationsémantique{同义词}{\lien{ⓔarɴɢlɯm}{arɴɢlɯm}}\end{entrée}

\begin{entrée}{aqurle}{}{ⓔaqurle} 
\classe{vi} \paradigme{dir}{tɤ-}\paradigme{construction}{infinitive}
\begin{définition}\pfra{collaborer}\end{définition}
\begin{définition}\pcmn{协作共事;互相帮忙}\end{définition}
\begin{exemple}\pjya{tɤ-scoz kɤ-rɤt pɯ-aqurle-tɕi}\hspace{5pt}\pcmn{我们一起写了信}\end{exemple}\relationsémantique{参考}{\lien{ⓔqur}{qur}}\end{entrée}

\begin{entrée}{araʁ}{}{ⓔaraʁ} 
\classe{n} 
\begin{définition}\pfra{alcool distillé}\end{définition}
\begin{définition}\pcmn{白酒}\end{définition}\étymologie{a.rag}\end{entrée}

\begin{entrée}{araχtɯ}{}{ⓔaraχtɯ} 
\classe{vs} 
\begin{définition}\pfra{endroit bien caché}\end{définition}
\begin{définition}\pcmn{僻静,不容易被别人发现的地方}\end{définition}
\begin{exemple}\pjya{ki sɤtɕha ɲɯ-ɤraχtɯ tɕe tɯrme kɯ-ɕe rkɯn}\hspace{5pt}\pcmn{这个地方很僻静,去的人少}\end{exemple}\end{entrée}

\begin{entrée}{arɤɕɯɕrɤz}{}{ⓔarɤɕɯɕrɤz} 
\classe{vi} \paradigme{dir}{nɯ-}\paradigme{dir}{thɯ-}
\begin{définition}\pfra{bariolé en bandes}\end{définition}
\begin{définition}\pcmn{几种不同颜色的成条形的纹路(一般指某个物体的切面,例如:猪肉的切面一层肥肉、一层瘦肉)}\end{définition}
\begin{exemple}\pjya{kɯki ɯ-mdoʁ kɯ-ɤrɤɕɯɕrɤz ɲɯ-ŋu}\hspace{5pt}\pcmn{这个东西有不同颜色的纹路}\end{exemple}\end{entrée}

\begin{entrée}{arɤkhɯmkhɤl/\variante{arɤkhɯkhɤl}}{}{ⓔarɤkhɯmkhɤl} 
\classe{vi} \paradigme{dir}{tɤ-}
\begin{définition}\pfra{hétérogène}\end{définition}
\begin{définition}\pcmn{不均匀;有些地方有,有些地方没有}\end{définition}
\begin{exemple}\pjya{a-βri kɯ-mŋɤm ɲɯ-ɤrɤkhɯmkhɤl}\hspace{5pt}\pcmn{身体有些地方疼,一些地方不疼}\end{exemple}
\begin{exemple}\pjya{tɤ-rɤku ɲɯ-ɤrɤkhɯmkhɤl}\hspace{5pt}\pcmn{有些地方有庄稼,有些地方没有}\end{exemple}
\begin{exemple}\pjya{tɯji ɯ-ŋgɯ tɤ-rɤku wuma ʑo ɲɯ-pe ri, ɲɯ-ɤrɤkhɯmkhɤl}\hspace{5pt}\pcmn{田里的庄稼长得很好,但是很不均匀}\end{exemple}\relationsémantique{参考}{\lien{ⓔtɯ-khɤl}{tɯ-khɤl}}\end{entrée}

\begin{entrée}{arɤmboʁɲɟi}{}{ⓔarɤmboʁɲɟi} 
\classe{vi}  
\grammaire{denom} \paradigme{dir}{nɯ-}
\begin{définition}\pfra{tomber en milles morceaux}\end{définition}
\begin{définition}\pcmn{粉粹}\end{définition}\relationsémantique{参考}{\lien{ⓔmboʁɲɟi}{mboʁɲɟi}}\end{entrée}

\begin{entrée}{arɤmbɯmbri}{}{ⓔarɤmbɯmbri} 
\classe{vi} 
\begin{définition}\pfra{pas rassemblé en un endroit, dispersé sur un chemin}\end{définition}
\begin{définition}\pcmn{不密集(一些地方多,一些地方少),一路上撒下的}\end{définition}\relationsémantique{参考}{\lien{ⓔrɤmbɯmbri}{rɤmbɯmbri}}\end{entrée}

\begin{entrée}{arɤmgrɯndɯr}{}{ⓔarɤmgrɯndɯr} 
\classe{vi.s} \paradigme{dir}{nɯ-}
\begin{définition}\pfra{dont la lie et la partie liquide ne se mélangent pas}\end{définition}
\begin{définition}\pcmn{渣滓和液体部分间隔分明}\end{définition}\relationsémantique{参考}{\lien{ⓔamgri}{amgri}}\end{entrée}

\begin{entrée}{arɤmgɯmgo}{}{ⓔarɤmgɯmgo} 
\classe{vs} 
\begin{définition}\pfra{ayant des grumeaux}\end{définition}
\begin{définition}\pcmn{不均匀,一坨一坨的(粥)}\end{définition}
\begin{exemple}\pjya{tɯtshi lɤ́-wɣ-sɤla tɕe khro ɲɯ́-wɣ-ɕmi tɕe mɤ-arɤmgɯmgo}\hspace{5pt}\pcmn{煲粥的时候要搅拌很多次才会均匀}\end{exemple}\relationsémantique{同义词}{\lien{ⓔarɤtshi}{arɤtshi}}\end{entrée}

\begin{entrée}{arɤmtɕɯmtɕoʁ}{}{ⓔarɤmtɕɯmtɕoʁ} 
\classe{vs} 
\begin{définition}\pfra{nombreux et rassemblés}\end{définition}
\begin{définition}\pcmn{很多;聚集在一起}\end{définition}
\begin{exemple}\pjya{ɯ-zrɯɣ ɯ-tɯ-dɤn kɯ ɲɯ-ɤrɤmtɕɯmtɕoʁ ʑo}\hspace{5pt}\pcmn{他身上长满了虱子}\end{exemple}\end{entrée}

\begin{entrée}{arɤmtʂɯmtʂaj}{}{ⓔarɤmtʂɯmtʂaj} 
\classe{vi} \paradigme{dir}{tɤ-}
\begin{définition}\pfra{collant}\end{définition}
\begin{définition}\pcmn{黏糊}\end{définition}
\begin{exemple}\pjya{cha ɲɯ-mɯm ɲɯ-ɤrɤmtʂɯmtʂaj}\hspace{5pt}\pcmn{酒好喝,很稠厚}\end{exemple}
\begin{exemple}\pjya{mbrɤz ɲɤ-mɲɤt tɕe ɲɯ-ɤrɤmtʂɯmtʂaj}\hspace{5pt}\pcmn{饭变味了,是黏糊糊的}\end{exemple}\end{entrée}

\begin{entrée}{arɤmɯzda}{}{ⓔarɤmɯzda} 
\classe{vi} \paradigme{dir}{nɯ-}
\begin{définition}\pfra{se transmettre les informations les uns aux autres}\end{définition}
\begin{définition}\pcmn{互相传递(信息、消息、话),通知每一个人}\end{définition}
\begin{exemple}\pjya{jiʑora nɯ-arɤmɯzda-j tɕe tɤ-rɤŋgat-i}\hspace{5pt}\pcmn{我们互相通知了,准备出发}\end{exemple}\relationsémantique{同义词}{\lien{ⓔarɤzdɯzda}{arɤzdɯzda}}\relationsémantique{同义词}{\lien{ⓔasɤmɯsɯz}{asɤmɯsɯz}}\relationsémantique{同义词}{\lien{ⓔasɤmɯmtshɯmtshɤm}{asɤmɯmtshɯmtshɤm}}\end{entrée}

\begin{entrée}{arɤntɕhɯntɕhɯr}{}{ⓔarɤntɕhɯntɕhɯr} 
\classe{vs} 
\begin{définition}\pfra{en mille morceaux}\end{définition}
\begin{définition}\pcmn{有很多碎片}\end{définition}\relationsémantique{参考}{\lien{ⓔtɯ-ntɕhɯr}{tɯ-ntɕhɯr}}\end{entrée}

\begin{entrée}{arɤɲɟiɲɟi}{}{ⓔarɤɲɟiɲɟi} 
\classe{vi} \paradigme{dir}{nɯ-}\paradigme{dir}{pɯ-}\paradigme{dir}{nɯ-}
\begin{définition}\pfra{tomber en lambeaux, tomber en morceau}\end{définition}
\begin{définition}\pcmn{成碎片}\end{définition}
\begin{définition}\pfra{briser en mille morceaux}\end{définition}
\begin{définition}\pcmn{打得粉碎}\end{définition}
\begin{exemple}\pjya{@guamian ɲɤ-k-ɤrɤɲɟiɲɟi-ci}\hspace{5pt}\pcmn{挂面成了碎片}\end{exemple}
\begin{exemple}\pjya{a-@chabei pjɤ-ɴɢrɯ tɕe ɲɤ-k-ɤrɤɲɟiɲɟi-ci}\hspace{5pt}\pcmn{我的茶杯破了,成了碎片}\end{exemple}
\begin{exemple}\pjya{χɕɤl nɯra pɯ-zrɤɲɟiɲɟi-t-a ʑo}\hspace{5pt}\pcmn{我打碎了玻璃}\end{exemple}
\begin{sous-entrée}{zrɤɲɟiɲɟi}{ⓔarɤɲɟiɲɟiⓝzrɤɲɟiɲɟi} 
\classe{vt} \end{sous-entrée}

\end{entrée}

\begin{entrée}{arɤɲɯɣ}{}{ⓔarɤɲɯɣ} 
\classe{vi} \paradigme{dir}{tɤ-}
\begin{définition}\pfra{en grande quantité}\end{définition}
\begin{définition}\pcmn{很多}\end{définition}
\begin{exemple}\pjya{ɯʑo ɯ-ŋga arɤɲɯɣ ʑo ɕti}\hspace{5pt}\pcmn{他有很多衣服}\end{exemple}
\begin{exemple}\pjya{tɤ-scoz a-kɤ-rɤt ɯ-spa arɤɲɯɣ ʑo}\hspace{5pt}\pcmn{我要写的字很多}\end{exemple}\end{entrée}

\begin{entrée}{arɤphɤjqa}{}{ⓔarɤphɤjqa} 
\classe{vi} \paradigme{dir}{nɯ-}
\begin{définition}\pfra{pousser (plusieurs pousses à partir d'un grain)}\end{définition}
\begin{définition}\pcmn{一颗种子可以长成很多根苗}\end{définition}
\begin{exemple}\pjya{ki @cai ɲɯ-ɤrɤphɤjqa}\hspace{5pt}\pcmn{这种菜长很多根苗}\end{exemple}
\begin{exemple}\pjya{tɤɕi ɲɯ-ɤrɤphɤjqa}\hspace{5pt}\pcmn{青稞长很多根苗}\end{exemple}
\begin{exemple}\pjya{tɤ-rɤku wuma ʑo ɲɯ-ɤrɤphɤjqa}\hspace{5pt}\pcmn{庄稼长很多根苗}\end{exemple}
\begin{exemple}\pjya{tɤɕi lo-ji tɕe wuma ʑo ɲɤ-k-ɤrɤphɤjqa-ci}\hspace{5pt}\pcmn{种了青稞以后,长了很多根苗}\end{exemple}\relationsémantique{参考}{\lien{ⓔtɯ-qa}{tɯ-qa}}\end{entrée}

\begin{entrée}{arɤrɤɣ}{}{ⓔarɤrɤɣ} 
\classe{vi} \paradigme{dir}{nɯ-}
\begin{définition}\pfra{se produire au moment prévu}\end{définition}
\begin{définition}\pcmn{在预定的时间发生}\end{définition}
\begin{exemple}\pjya{aʑo a-ʑɯβ ɲɤ-k-ɤrɤɣ-ci}\hspace{5pt}\pcmn{我在预定的时间就想睡觉了}\end{exemple}
\begin{exemple}\pjya{jisŋi saχsɯ ɯ-qhu tɕe ku-kɯ-rŋgɯ tɕe a-pɯ-kɯ-nɯʑɯβ tɕe ɯ-fso saχsɯ ɯ-mphru, li tɯ-ʑɯβ pjɯ-ɣi ŋu, tɕe nɯ ɯ-sta nɯ-kɤ-βzu nɯ ɲɯ-kɯ-ɤrɤrɤɣ tu-kɯ-ti ŋu}\hspace{5pt}\pcmn{今天午饭后睡觉睡着了,第二天午饭后又想瞌睡了,这种习惯性的现象叫做\lien{}{ɲɯ-kɤrɤrɤɣ}}\end{exemple}\relationsémantique{参考}{\lien{ⓔɯ-rɤɣ}{ɯ-rɤɣ}}\end{entrée}

\begin{entrée}{arɤrkhɯrkhe}{}{ⓔarɤrkhɯrkhe} 
\classe{vs} \paradigme{dir}{lɤ-}\paradigme{dir}{thɯ-}
\begin{définition}\pfra{pas cohérent}\end{définition}
\begin{définition}\pcmn{不平整;不连贯;吞吞吐吐}\end{définition}
\begin{exemple}\pjya{tʂu ɲɯ-ɤrɤrkhɯrkhe}\hspace{5pt}\pcmn{路不平整}\end{exemple}
\begin{exemple}\pjya{si ɲɯ-ɤrɤrkhɯrkhe}\hspace{5pt}\pcmn{木料上刻得不均匀}\end{exemple}
\begin{exemple}\pjya{jiɕqha ɯ-rju ɲɯ-ɤrɤrkhɯrkhe}\hspace{5pt}\pcmn{他说话吞吞吐吐}\end{exemple}
\begin{exemple}\pjya{tɤrɤm ɲɯ-ɤrɤrkhɯrkhe}\hspace{5pt}\pcmn{木板不平整}\end{exemple}\relationsémantique{参考}{\lien{ⓔrkhe}{rkhe}}\end{entrée}

\begin{entrée}{arɤrɴɢioʁ}{}{ⓔarɤrɴɢioʁ} 
\classe{vi} 
\begin{définition}\pfra{avoir une encoche}\end{définition}
\begin{définition}\pcmn{有一条槽}\end{définition}\relationsémantique{参考}{\lien{ⓔtɤ-rɴɢioʁ}{tɤ-rɴɢioʁ}}\end{entrée}

\begin{entrée}{arɤrqhɯrqhioʁ}{}{ⓔarɤrqhɯrqhioʁ} 
\classe{vi} 
\begin{définition}\pfra{qui a des entailles}\end{définition}
\begin{définition}\pcmn{有一条一条的槽口;纹路}\end{définition}\relationsémantique{参考}{\lien{ⓔtɤ-rqhioʁ}{tɤ-rqhioʁ}}\end{entrée}

\begin{entrée}{arɤrtsɯrtsɤɣ}{}{ⓔarɤrtsɯrtsɤɣ} 
\classe{vs} 
\begin{définition}\pfra{composé de sections}\end{définition}
\begin{définition}\pcmn{一节一节组成的}\end{définition}\relationsémantique{参考}{\lien{ⓔaɣɯrtsɯrtsɤɣ}{aɣɯrtsɯrtsɤɣ}}\relationsémantique{参考}{\lien{ⓔtɯ-rtsɤɣ}{tɯ-rtsɤɣ}}\end{entrée}

\begin{entrée}{arɤstoʁsta}{}{ⓔarɤstoʁsta} 
\classe{vs} \sens{1}
\begin{définition}\pfra{fiable}\end{définition}
\begin{définition}\pcmn{可靠;说话算数}\end{définition}
\begin{exemple}\pjya{jiɕqha nɯ ɯ-rju mɤ-kɯ-ɤrɤstoʁsta ci ŋu}\hspace{5pt}\pcmn{这是说话不算数的一个人}\end{exemple}\sens{2}
\begin{définition}\pfra{stable}\end{définition}
\begin{définition}\pcmn{稳定}\end{définition}\relationsémantique{反义词}{\lien{ⓔɲɟɯrmbloʁ}{ɲɟɯrmbloʁ}}\end{entrée}

\begin{entrée}{arɤt}{}{ⓔarɤt}\relationsémantique{参考}{\lien{ⓔrɤt}{rɤt}}\end{entrée}

\begin{entrée}{arɤtɕha}{}{ⓔarɤtɕha}\relationsémantique{参考}{\lien{ⓔzrɤtɕha}{zrɤtɕha}}\end{entrée}

\begin{entrée}{arɤtshi}{}{ⓔarɤtshi} 
\classe{vi} \paradigme{dir}{nɯ-}\paradigme{dir}{nɯ-}
\begin{définition}\pfra{trop cuit}\end{définition}
\begin{définition}\pcmn{煮得很烂,变得像粥一样}\end{définition}
\begin{définition}\pfra{trop cuire}\end{définition}
\begin{définition}\pcmn{煮得很烂}\end{définition}
\begin{exemple}\pjya{kɤ́-wɣ-sqa tɕe a-mɤ-nɯ-ɤrɤtshi ra ma mɤ-mɯm}\hspace{5pt}\pcmn{煮饭的时候,不要煮得太烂,不然不好吃}\end{exemple}
\begin{exemple}\pjya{ko-ɣɤsmi-t-a tɕe ɲɤ-zrɤtshi-t-a}\hspace{5pt}\pcmn{我煮得太烂了(不小心)}\end{exemple}
\begin{sous-entrée}{zrɤtshi}{ⓔarɤtshiⓝzrɤtshi} 
\classe{vt} \end{sous-entrée}

\end{entrée}

\begin{entrée}{arɤzdɯzda}{}{ⓔarɤzdɯzda} 
\classe{vi} \paradigme{dir}{nɯ-}
\begin{définition}\pfra{se transmettre les informations les uns aux autres}\end{définition}
\begin{définition}\pcmn{互相传递(信息、消息、话),通知每一个人}\end{définition}
\begin{exemple}\pjya{ɕe-j tɤ-mda tɕe, kɤ-ɤrɤzdɯzda ra}\hspace{5pt}\pcmn{我们要走的时候,要互相通知}\end{exemple}\relationsémantique{同义词}{\lien{ⓔarɤmɯzda}{arɤmɯzda}}\end{entrée}

\begin{entrée}{arɤʑɯʑrɤz}{}{ⓔarɤʑɯʑrɤz} 
\classe{vi} \paradigme{dir}{nɯ-}
\begin{définition}\pfra{ayant des bandes de couleurs différentes}\end{définition}
\begin{définition}\pcmn{有几种不同颜色的成条形的纹路(物体表面的颜色)}\end{définition}
\begin{exemple}\pjya{kɯki kɯ-ɤrɤʑɯʑrɤz ci ɲɯ-ŋu}\hspace{5pt}\pcmn{这个东西有几种不同颜色的纹路}\end{exemple}\relationsémantique{参考}{\lien{ⓔarɤɕɯɕrɤz}{arɤɕɯɕrɤz}}\relationsémantique{参考}{\lien{ⓔtɯ-ʑrɤz}{tɯ-ʑrɤz}}\end{entrée}

\begin{entrée}{arɕɤt}{}{ⓔarɕɤt} 
\classe{vi} 
\begin{définition}\pfra{avoir une relation de parenté}\end{définition}
\begin{définition}\pcmn{有血缘关系}\end{définition}
\begin{exemple}\pjya{jiʑo kɯmdza mɤ-arɕɤt-i}\hspace{5pt}\pcmn{我们没有一点亲戚关系}\end{exemple}
\begin{exemple}\pjya{jiɕqha nɯ nɤ-kɯmdza mɤ-arɕɤt}\hspace{5pt}\pcmn{那个不是你的亲戚}\end{exemple}\end{entrée}

\begin{entrée}{arɕo}{}{ⓔarɕo} 
\classe{vi} \paradigme{dir}{thɯ-}\paradigme{dir}{thɯ-}
\begin{définition}\pfra{finir}\end{définition}
\begin{définition}\pcmn{完;用光了}\end{définition}
\begin{définition}\pfra{utiliser complètement}\end{définition}
\begin{définition}\pcmn{用完}\end{définition}
\begin{exemple}\pjya{tʂha thɯ-arɕo}\hspace{5pt}\pcmn{没有茶了}\end{exemple}
\begin{exemple}\pjya{kɤndza chɯ-ɤrɕo}\hspace{5pt}\pcmn{没有吃的}\end{exemple}
\begin{exemple}\pjya{kɤ-ndza a-mɤ-thɯ-ɤrɕo, kɤ-ŋga a-mɤ-thɯ-ɤrɕo}\hspace{5pt}\pcmn{希望不会缺吃的,也不会缺穿的}\end{exemple}
\begin{exemple}\pjya{a-mɤ-thɯ-ɤrɕo ma ɯ-qhu tɕe me}\hspace{5pt}\pcmn{不要用完,不然以后就没有了}\end{exemple}
\begin{exemple}\pjya{ɯ-ro ɯ-ɲɤ-ri nɤ a-pɯ-ɤnɯta jɤɣ ma kɤ-sɤrɕo mɤ-ra}\hspace{5pt}\pcmn{如果有剩的就留在那里,不一定要用完}\end{exemple}
\begin{sous-entrée}{sɤrɕo}{ⓔarɕoⓝsɤrɕo} 
\classe{vt} \end{sous-entrée}

\begin{sous-entrée}{ɣɤrɕo}{ⓔarɕoⓝɣɤrɕo} 
\classe{vs} 
\begin{définition}\pfra{qui se fini rapidement}\end{définition}
\begin{définition}\pcmn{很快用完}\end{définition}\relationsémantique{同义词}{\lien{ⓔɣɤsa}{ɣɤsa}}\end{sous-entrée}

\end{entrée}

\begin{entrée}{arɣi}{}{ⓔarɣi} 
\classe{vs}  
\grammaire{caus} \paradigme{dir}{tɤ-}\paradigme{dir}{tɤ-}
\begin{définition}\pfra{s'accumuler}\end{définition}
\begin{définition}\pcmn{积累}\end{définition}
\begin{exemple}\pjya{tɯ-ci to-k-ɤrɣi-ci}\hspace{5pt}\pcmn{水积起来了}\end{exemple}
\begin{sous-entrée}{sɤrɣi}{ⓔarɣiⓝsɤrɣi} 
\classe{vt} \end{sous-entrée}

\begin{définition}\pfra{accumuler}\end{définition}
\begin{définition}\pcmn{积累}\end{définition}
\begin{exemple}\pjya{tɯftsaʁ ɲɯ-ɣi tɕe, zɯm tɤ-ɕthɯz-a tɕe, nɯ ɯ-ŋgɯ tɤ-sɤrɣi-t-a tɕe, ɯ-thoʁ mɤ-kɯ-ɕe tɤ-sɤβzu-t-a}\hspace{5pt}\pcmn{屋顶在漏水,我用水桶把水积下来了,免得水流到地板上}\end{exemple}\relationsémantique{同义词}{\lien{ⓔajtɯ}{ajtɯ}}\end{entrée}

\begin{entrée}{ari}{}{ⓔari} 
\classe{vi} \paradigme{dir}{nɯ-}\paradigme{dir}{pɯ-}\sens{1}
\begin{définition}\pfra{couler}\end{définition}
\begin{définition}\pcmn{漏}\end{définition}
\begin{exemple}\pjya{pjɤ-spoʁ tɕe pjɤ-k-ɤri-ci}\hspace{5pt}\pcmn{有了洞就漏了水}\end{exemple}
\begin{exemple}\pjya{tɯ-ci pjɤ-k-ɤri-ci}\hspace{5pt}\pcmn{水漏了}\end{exemple}
\begin{exemple}\pjya{khɯtsa ɲɯ-ɤri}\hspace{5pt}\pcmn{碗在漏水}\end{exemple}\sens{2}
\begin{définition}\pfra{appartenir à (groupe)}\end{définition}
\begin{définition}\pcmn{属于}\end{définition}
\begin{exemple}\pjya{aʑo tɤrca tɤ-ari-a}\hspace{5pt}\pcmn{我加入了}\end{exemple}
\begin{exemple}\pjya{ɯʑo ɯ-zdɤrca mɤ-kɯ-ɤri ci ɲɯ-ŋu}\hspace{5pt}\pcmn{他是个不合群的人}\end{exemple}\relationsémantique{同义词}{\lien{ⓔsɤriⓝʑɣɤsɤri}{ʑɣɤsɤri}}\relationsémantique{参考}{\lien{ⓔsɤri}{sɤri}}\end{entrée}

\begin{entrée}{arju}{}{ⓔarju} 
\classe{vi} \paradigme{dir}{tɤ-}\paradigme{dir}{tɤ-}
\begin{définition}\pfra{parler}\end{définition}
\begin{définition}\pcmn{说话}\end{définition}
\begin{définition}\pfra{laisser, faire parler}\end{définition}
\begin{définition}\pcmn{让人说话}\end{définition}
\begin{exemple}\pjya{ɲɯ-tɯ-ɤrju tɕe pɯ-kɯ-sɯmtsham-a}\hspace{5pt}\pcmn{你说话,让我听见了}\end{exemple}
\begin{exemple}\pjya{nɯ ɲɯ-ɤrju}\hspace{5pt}\pcmn{他在说那个}\end{exemple}
\begin{exemple}\pjya{a-mɤ-tɯ-ɤrju}\hspace{5pt}\pcmn{不要说话}\end{exemple}
\begin{exemple}\pjya{tɤ-arju-a}\hspace{5pt}\pcmn{我说了}\end{exemple}
\begin{exemple}\pjya{ma-tɯ-ɤrju}\hspace{5pt}\pcmn{你别说话}\end{exemple}
\begin{exemple}\pjya{mɯ-tɤ-sɤrju-t-a}\hspace{5pt}\pcmn{我没有让他说话}\end{exemple}\relationsémantique{参考}{\lien{ⓔtɯ-rju}{tɯ-rju}}\relationsémantique{参考}{\lien{ⓔmasɤrɯrju}{masɤrɯrju}}
\begin{sous-entrée}{sɤrju}{ⓔarjuⓝsɤrju} 
\classe{vt}  
\grammaire{caus} \end{sous-entrée}

\end{entrée}

\begin{entrée}{arɟambrɯɣ}{}{ⓔarɟambrɯɣ} 
\classe{vs} 
\begin{définition}\pfra{aux yeux entourés de rouge (chien)}\end{définition}
\begin{définition}\pcmn{四眼狗}\end{définition}\relationsémantique{参考}{\lien{ⓔrɟambrɯɣ}{rɟambrɯɣ}}\end{entrée}

\begin{entrée}{arɟumtɕɤr}{}{ⓔarɟumtɕɤr} 
\classe{vi}  
\grammaire{comp} \paradigme{dir}{tɤ-}\paradigme{dir}{rɟum}
\begin{définition}\pfra{ayant des largeurs différentes}\end{définition}
\begin{définition}\pcmn{宽窄不一}\end{définition}
\begin{exemple}\pjya{to-k-ɤrɟumtɕɤr-ci}\hspace{5pt}\pcmn{宽窄不一了}\end{exemple}
\begin{exemple}\pjya{tɯ-ŋga ɯ-spa ɲɯ-ɤrɟumtɕɤr tɕe, ɣɯ́-nɯβʑit ɲɯ-ra}\hspace{5pt}\pcmn{衣服的材料宽窄不一,要剪一下}\end{exemple}\relationsémantique{参考}{\lien{ⓔtɕɤr}{tɕɤr}}\end{entrée}

\begin{entrée}{arku}{}{ⓔarku} 
\classe{vi} \paradigme{dir}{tɤ-}
\begin{définition}\pfra{être dans}\end{définition}
\begin{définition}\pcmn{装在}\end{définition}
\begin{exemple}\pjya{a-tɯ-ci tu, arku}\hspace{5pt}\pcmn{杯子里装着水}\end{exemple}\relationsémantique{参考}{\lien{ⓔrku}{rku}}\end{entrée}

\begin{entrée}{arla}{}{ⓔarla}\relationsémantique{参考}{\lien{ⓔrla}{rla}}\end{entrée}

\begin{entrée}{arlɯrla}{}{ⓔarlɯrla} 
\classe{vi} \paradigme{dir}{nɯ-}
\begin{définition}\pfra{s'étendre}\end{définition}
\begin{définition}\pcmn{舒展}\end{définition}
\begin{exemple}\pjya{nɤ-phoŋbu a-nɯ-ɤrlɯrla!}\hspace{5pt}\pcmn{舒展一下筋骨!}\end{exemple}
\begin{sous-entrée}{sɤrlɯrla}{ⓔarlɯrlaⓝsɤrlɯrla} 
\classe{vt} 
\begin{définition}\pfra{étendre}\end{définition}
\begin{définition}\pcmn{舒展,伸展}\end{définition}
\begin{exemple}\pjya{nɤ-phoŋbu nɯ-sɤrlɯrle!}\hspace{5pt}\pcmn{舒展一下筋骨!}\end{exemple}
\begin{exemple}\pjya{sɯjno tɤ-kɤ-rmbɯ nɯra ɲɯ́-wɣ-sɤrlɯrla tɕe a-nɯ-rom}\hspace{5pt}\pcmn{把堆在一起的草展开(晒干)就会干}\end{exemple}
\begin{exemple}\pjya{ɯ-rŋa ɲɤ-sɤrlɯrla}\hspace{5pt}\pcmn{他皱着的眉头舒展了}\end{exemple}\end{sous-entrée}

\end{entrée}

\begin{entrée}{arlɯt}{}{ⓔarlɯt} 
\classe{vs} 
\begin{définition}\pfra{nombreux}\end{définition}
\begin{définition}\pcmn{多}\end{définition}
\begin{exemple}\pjya{tɯrme ɲɯ-ɤrlɯt}\hspace{5pt}\pcmn{人很多}\end{exemple}
\begin{exemple}\pjya{tɯ-ci ɲɯ-ɤrlɯt}\hspace{5pt}\pcmn{水很多}\end{exemple}\relationsémantique{同义词}{\lien{ⓔdɤn}{dɤn}}\relationsémantique{同义词}{\lien{ⓔxcat}{xcat}}\end{entrée}

\begin{entrée}{armɤjɯβ}{}{ⓔarmɤjɯβ} 
\classe{vi} 
\begin{définition}\pfra{être le crépuscule}\end{définition}
\begin{définition}\pcmn{黄昏;天黑}\end{définition}
\begin{exemple}\pjya{ko-k-ɤrmɤjɯβ-ci}\hspace{5pt}\pcmn{天黑了}\end{exemple}\end{entrée}

\begin{entrée}{armbat}{}{ⓔarmbat} 
\classe{vs}  
\grammaire{caus}
\grammaire{refl}
\grammaire{caus} \paradigme{dir}{\_}\paradigme{dir}{\_}
\begin{définition}\pfra{proche}\end{définition}
\begin{définition}\pcmn{近}\end{définition}
\begin{exemple}\pjya{kɤ-ɤrmbat nɯ-ɣi}\hspace{5pt}\pcmn{你来近一点的地方}\end{exemple}
\begin{exemple}\pjya{ki nɤj nɤ-ɕki ɯ-j-armbat?}\hspace{5pt}\pcmn{离你那里远不远?}\end{exemple}
\begin{sous-entrée}{sɤrmbat}{ⓔarmbatⓝsɤrmbat} 
\classe{vt} \end{sous-entrée}

\paradigme{dir}{\_}
\begin{définition}\pfra{rapprocher}\end{définition}
\begin{définition}\pcmn{拿过来;使靠近}\end{définition}
\begin{exemple}\pjya{ko-sɤrmbat}\hspace{5pt}\pcmn{他拿过来了}\end{exemple}
\begin{exemple}\pjya{na-sɤrmbat}\hspace{5pt}\pcmn{他拿过来了}\end{exemple}
\begin{sous-entrée}{ʑɣɤsɤrmbat}{ⓔarmbatⓝʑɣɤsɤrmbat} 
\classe{vi} \end{sous-entrée}

\begin{définition}\pfra{s'approcher}\end{définition}
\begin{définition}\pcmn{靠近}\end{définition}
\begin{exemple}\pjya{tɤ-ʑɣɤsɤrmbat-a}\hspace{5pt}\pcmn{我靠近了}\end{exemple}\relationsémantique{反义词}{\lien{ⓔarqhi}{arqhi}}\relationsémantique{参考}{\lien{ⓔmbarqhi}{mbarqhi}}\relationsémantique{参考}{\lien{ⓔamɯrmbat}{amɯrmbat}}\end{entrée}

\begin{entrée}{armbɯrmbɯ}{}{ⓔarmbɯrmbɯ} 
\classe{vi} \paradigme{dir}{tɤ-}
\begin{définition}\pfra{en tas}\end{définition}
\begin{définition}\pcmn{堆起来的;积在一起}\end{définition}\relationsémantique{参考}{\lien{ⓔrmbɯ}{rmbɯ}}\end{entrée}

\begin{entrée}{arndɤtsa}{}{ⓔarndɤtsa} 
\classe{vs} \paradigme{dir}{nɯ-}
\begin{définition}\pfra{dressé et imposant}\end{définition}
\begin{définition}\pcmn{矗立}\end{définition}
\begin{exemple}\pjya{jinde kha ra rcanɯ tɯ-mɯ ɯ-pa kɯ-ɤrndɤtsa ʑo kɯ-mbro tu-βzu-nɯ ɲɯ-ŋu}\hspace{5pt}\pcmn{现在修的房子非常巨大}\end{exemple}\end{entrée}

\begin{entrée}{arɲɟɤle}{}{ⓔarɲɟɤle} 
\classe{vi} \paradigme{dir}{thɯ-}
\begin{définition}\pfra{se tendre}\end{définition}
\begin{définition}\pcmn{伸展}\end{définition}
\begin{exemple}\pjya{cho-k-ɤrɲɟɤle-ci}\hspace{5pt}\pcmn{他伸出来了}\end{exemple}
\begin{exemple}\pjya{kɯ-ɤrɲɟɤle ci ɲɯ-ŋu kɯ-zri tsa ɲɯ-ŋu ɯ-skɤt ɲɯ-ŋu}\hspace{5pt}\pcmn{“伸着”表示“长一点”的意思}\end{exemple}
\begin{sous-entrée}{sɤrɲɟɤle}{ⓔarɲɟɤleⓝsɤrɲɟɤle}
\begin{définition}\pfra{étendre}\end{définition}
\begin{définition}\pcmn{伸}\end{définition}
\begin{exemple}\pjya{ɯ-mɤlɤjaʁ chɤ-sɤrɲɟɤle}\hspace{5pt}\pcmn{他伸了四肢、他伸了懒腰}\end{exemple}
\begin{exemple}\pjya{kɤ-nɯmdzɯ, nɤ-mi thɯ-nɯsɤrɲɟɤle}\hspace{5pt}\pcmn{你坐下,把脚伸一下(请人休息)}\end{exemple}\end{sous-entrée}

\end{entrée}

\begin{entrée}{arŋɤrtɯm}{}{ⓔarŋɤrtɯm} 
\classe{vs} 
\begin{définition}\pfra{qui a le visage rond}\end{définition}
\begin{définition}\pcmn{脸很圆}\end{définition}\relationsémantique{参考}{\lien{ⓔtɯ-rŋa}{tɯ-rŋa}}\relationsémantique{参考}{\lien{ⓔartɯm}{artɯm}}\end{entrée}

\begin{entrée}{arŋi}{}{ⓔarŋi} 
\classe{vi} \paradigme{dir}{nɯ-}
\begin{définition}\pfra{bleu, vert}\end{définition}
\begin{définition}\pcmn{蓝色;绿色}\end{définition}
\begin{exemple}\pjya{stomku ɲɤ-k-ɤrŋi-ci}\hspace{5pt}\pcmn{草坪变绿了}\end{exemple}
\begin{exemple}\pjya{xɕaj ɲɤ-k-ɤrŋi-ci}\hspace{5pt}\pcmn{草变绿了}\end{exemple}
\begin{exemple}\pjya{tɯ-ŋga kɯ-ɤrŋi ɲɯ-ŋu}\hspace{5pt}\pcmn{衣服是绿色的}\end{exemple}
\begin{exemple}\pjya{tɯ-mɯ kɯ-ɤrŋi}\hspace{5pt}\pcmn{青天、天宫}\end{exemple}
\begin{exemple}\pjya{tɯ-ci ɯ-tɯ-rnaʁ kɯ ɲɯ-ɤrŋi ʑo ndɯrndɯr}\hspace{5pt}\pcmn{水很深,显得很蓝}\end{exemple}
\begin{exemple}\pjya{nɤ-tɯ-ɤrŋi kɯ kupa ʑo ɲɯ-tɯ-fse}\hspace{5pt}\pcmn{你穿那么蓝的衣服,就像汉人一样}\end{exemple}
\begin{sous-entrée}{sɤrŋi}{ⓔarŋiⓝsɤrŋi} 
\classe{vt}  
\grammaire{caus} 
\begin{définition}\pfra{rendre bleu}\end{définition}
\begin{définition}\pcmn{使变蓝}\end{définition}
\begin{exemple}\pjya{tɯ-ŋga ɲɤ-sɤrŋi-t-a}\hspace{5pt}\pcmn{我(不小心)把衣服弄蓝了}\end{exemple}\end{sous-entrée}

\end{entrée}

\begin{entrée}{arŋɯlɯz/\variante{arŋilɯz}}{}{ⓔarŋɯlɯz} 
\classe{vi} \paradigme{dir}{nɯ-}
\begin{définition}\pfra{bleuâtre}\end{définition}
\begin{définition}\pcmn{淡蓝色}\end{définition}
\begin{exemple}\pjya{laχtɕha ɯ-mdoʁ ɲɯ-ɤrŋɯlɯz}\hspace{5pt}\pcmn{那个东西的颜色是淡蓝色的}\end{exemple}
\begin{exemple}\pjya{jiɕqha nɯ ɯ-mdoʁ kɯ-ɤrŋɯlɯz ci ɲɯ-ŋu}\hspace{5pt}\pcmn{那个东西的颜色是淡蓝色的}\end{exemple}\relationsémantique{参考}{\lien{ⓔarŋi}{arŋi}}\end{entrée}

\begin{entrée}{arɴɢlɯm}{}{ⓔarɴɢlɯm} 
\classe{vi} \paradigme{dir}{kɤ-}
\begin{définition}\pfra{concave}\end{définition}
\begin{définition}\pcmn{凹(地面)}\end{définition}
\begin{exemple}\pjya{ko-k-ɤrɴɢlɯm-ci}\hspace{5pt}\pcmn{凹进去了}\end{exemple}
\begin{exemple}\pjya{ɯ-thoʁ ɲɯ-ɤrɴɢlɯm}\hspace{5pt}\pcmn{地面是凹进去的}\end{exemple}\relationsémantique{参考}{\lien{ⓔsqlɯm}{sqlɯm}}
\begin{sous-entrée}{sɤrɴɢlɯm}{ⓔarɴɢlɯmⓝsɤrɴɢlɯm} 
\classe{vt} 
\begin{définition}\pfra{rendre concave}\end{définition}
\begin{définition}\pcmn{让……凹进去}\end{définition}
\begin{exemple}\pjya{rdɤstaʁ pjɤ-ɣi tɕe, tʂu pjɤ-sɤrɴɢlɯm}\hspace{5pt}\pcmn{大石头从上面滚下来了,让路凹进去了}\end{exemple}\end{sous-entrée}

\end{entrée}

\begin{entrée}{aro}{}{ⓔaro} 
\classe{vi-t} \paradigme{dir}{tɤ-}
\begin{définition}\pfra{posséder}\end{définition}
\begin{définition}\pcmn{拥有}\end{définition}
\begin{exemple}\pjya{ɕɯŋgɯ pɯ-me, tham to-k-ɤro-ci}\hspace{5pt}\pcmn{以前没有,现在有了}\end{exemple}
\begin{exemple}\pjya{aʑo nɯ aro-a}\hspace{5pt}\pcmn{我拥有那个东西}\end{exemple}
\begin{exemple}\pjya{nɯ pɯ-aro-nɯ}\hspace{5pt}\pcmn{他们以前拥有那个东西(现在没有了)}\end{exemple}
\begin{exemple}\pjya{nɯ tɤ-aro-nɯ tu-kɯ-ti tɕe, ɕɯŋgɯ pɯ-me, tham to-tu}\hspace{5pt}\pcmn{“他们拥有了”的意思就是以前没有,现在就有了}\end{exemple}
\begin{exemple}\pjya{nɤʑo nɤ-kɯ-ra nɯ aʑo aro-a}\hspace{5pt}\pcmn{我有你需要的那个东西}\end{exemple}
\begin{exemple}\pjya{aʑo tɤ-rte aro-a}\hspace{5pt}\pcmn{我有帽子}\end{exemple}
\begin{exemple}\pjya{tɯ-rɟɯ aro-a}\hspace{5pt}\pcmn{我有财产}\end{exemple}
\begin{exemple}\pjya{aʑo a-kɯ-ɤro nɯ lonba nɯ-kho-t-a}\hspace{5pt}\pcmn{我把拥有的东西全部给了(他)}\end{exemple}
\begin{exemple}\pjya{aʑo qaʑo aro-a nɯra kɯki ŋu}\hspace{5pt}\pcmn{这是我所拥有的绵羊}\end{exemple}
\begin{exemple}\pjya{aʑo tɤ-rɟit χsɯm aro-a (= a-rɟit χsɯm tu)}\hspace{5pt}\pcmn{我有三个孩子}\end{exemple}\end{entrée}

\begin{entrée}{arqhi}{}{ⓔarqhi} 
\classe{vs} \paradigme{dir}{\_}\paradigme{dir}{\_}\paradigme{dir}{\_}
\begin{définition}\pfra{lointain}\end{définition}
\begin{définition}\pcmn{远}\end{définition}
\begin{définition}\pfra{éloigner}\end{définition}
\begin{définition}\pcmn{把……离远}\end{définition}
\begin{définition}\pfra{s'éloigner}\end{définition}
\begin{définition}\pcmn{离得远一点}\end{définition}
\begin{exemple}\pjya{jiɕqha sɤtɕha ɲɯ-ɤrqhi}\hspace{5pt}\pcmn{这个地方很远}\end{exemple}
\begin{exemple}\pjya{wuma arqhi}\hspace{5pt}\pcmn{非常远}\end{exemple}
\begin{exemple}\pjya{nɤ-mɲaʁ ɯ-ɕki ma-jɤ-tɯ-sɤrqhi ma mɤ-tɯ-sɯχsɤl}\hspace{5pt}\pcmn{你的视线不要离他太远了,不然看不清楚}\end{exemple}\relationsémantique{反义词}{\lien{ⓔarmbat}{armbat}}\relationsémantique{参考}{\lien{ⓔmbarqhi}{mbarqhi}}\relationsémantique{参考}{\lien{ⓔamɯrqhi}{amɯrqhi}}
\begin{sous-entrée}{sɤrqhi}{ⓔarqhiⓝsɤrqhi} 
\classe{vt} \end{sous-entrée}

\begin{sous-entrée}{ʑɣɤsɤrqhi}{ⓔarqhiⓝʑɣɤsɤrqhi} 
\classe{vi} \end{sous-entrée}

\end{entrée}

\begin{entrée}{arqɯrqoʁ}{}{ⓔarqɯrqoʁ} 
\classe{vi}  
\grammaire{recip} \paradigme{dir}{kɤ-}
\begin{définition}\pfra{se prendre dans les bras}\end{définition}
\begin{définition}\pcmn{互相拥抱}\end{définition}
\begin{exemple}\pjya{nɤ-rʑaβ cho kɤ-tɯ-ɤrqɯrqoʁ-ndʑi}\hspace{5pt}\pcmn{你跟你妻子互相拥抱了}\end{exemple}\relationsémantique{参考}{\lien{ⓔrqoʁ}{rqoʁ}}\end{entrée}

\begin{entrée}{arʁɯrʁu}{}{ⓔarʁɯrʁu} 
\classe{vs}  
\grammaire{caus}
\grammaire{refl} \paradigme{dir}{tɤ-}\paradigme{dir}{lɤ-}\paradigme{dir}{tɤ-}
\begin{définition}\pfra{froissé}\end{définition}
\begin{définition}\pcmn{皱(衣服)}\end{définition}
\begin{définition}\pfra{ramasser (ses jambes)}\end{définition}
\begin{définition}\pcmn{收回来(脚)}\end{définition}
\begin{exemple}\pjya{nɤ-ku ɲɯ-ɤrʁɯrʁu}\hspace{5pt}\pcmn{你的头是皱着的}\end{exemple}
\begin{exemple}\pjya{nɤ-ŋga ɲɯ-ɤrʁɯrʁu}\hspace{5pt}\pcmn{你的衣服是皱着的}\end{exemple}
\begin{exemple}\pjya{to-k-ɤrʁɯrʁu-ci (=to-k-ɤɣɯrʑɯrʑɯɣ-ci)}\hspace{5pt}\pcmn{变皱了}\end{exemple}
\begin{exemple}\pjya{a-mi thɯ-sɤstɤko-t-a, a-mi lɤ-sɤrʁɯrʁu-t-a}\hspace{5pt}\pcmn{我把脚伸出来了,我把脚收回来了}\end{exemple}\relationsémantique{参考}{\lien{ⓔachɯrʁu}{achɯrʁu}}
\begin{sous-entrée}{sɤrʁɯrʁu}{ⓔarʁɯrʁuⓝsɤrʁɯrʁu} 
\classe{vt}  
\grammaire{caus} \end{sous-entrée}

\begin{sous-entrée}{ʑɣɤsɤrʁɯrʁu}{ⓔarʁɯrʁuⓝʑɣɤsɤrʁɯrʁu} 
\classe{vt} \end{sous-entrée}

\paradigme{dir}{tɤ-}
\begin{définition}\pfra{se ramasser, se blottir}\end{définition}
\begin{définition}\pcmn{缩成一团}\end{définition}\end{entrée}

\begin{entrée}{artaʁ}{}{ⓔartaʁ} 
\classe{vi} \paradigme{dir}{nɯ-}
\begin{définition}\pfra{fourchu}\end{définition}
\begin{définition}\pcmn{分叉;成双}\end{définition}
\begin{exemple}\pjya{ki si nɯ tɯ-ldʑa ma pɯ-me ri ɲɤ-k-ɤrtaʁ-ci}\hspace{5pt}\pcmn{这棵树原来只有一根,现在分叉了}\end{exemple}\relationsémantique{参考}{\lien{ⓔtɤ-rtaʁ}{tɤ-rtaʁ}}\relationsémantique{参考}{\lien{ⓔjmɤrtaʁ}{jmɤrtaʁ}}\end{entrée}

\begin{entrée}{artaʁlaʁ}{}{ⓔartaʁlaʁ}\relationsémantique{参考}{\lien{ⓔrtaʁ}{rtaʁ}}\end{entrée}

\begin{entrée}{artɕhoʁ}{}{ⓔartɕhoʁ} 
\classe{vs} \paradigme{dir}{\_}\paradigme{dir}{\_}
\begin{définition}\pfra{rassemblé à un endroit}\end{définition}
\begin{définition}\pcmn{堆在一边,集中在一个地方}\end{définition}
\begin{définition}\pfra{rassembler dans un coin}\end{définition}
\begin{définition}\pcmn{把分散的物体堆在一边}\end{définition}
\begin{exemple}\pjya{tɯrme ra lo-rɟɯɣ-nɯ tɕe tɕelo lo-k-ɤrtɕhoʁ-nɯ-ci}\hspace{5pt}\pcmn{那些人跑上去集中在上面了}\end{exemple}
\begin{sous-entrée}{sɤrtɕhoʁ}{ⓔartɕhoʁⓝsɤrtɕhoʁ} 
\classe{vt} \end{sous-entrée}

\end{entrée}

\begin{entrée}{artɕi}{}{ⓔartɕi} 
\classe{vs} \sens{1}\paradigme{dir}{nɯ-}
\begin{définition}\pfra{être apaisée (soif)}\end{définition}
\begin{définition}\pcmn{解(渴)}\end{définition}
\begin{exemple}\pjya{a-ɕpaʁ nɯ-artɕi}\hspace{5pt}\pcmn{我不渴了}\end{exemple}\sens{2}\paradigme{dir}{kɤ-}
\begin{définition}\pfra{avoir assez (dormi)}\end{définition}
\begin{définition}\pcmn{(睡)够}\end{définition}
\begin{exemple}\pjya{a-ʑɯβ kɤ-artɕi}\hspace{5pt}\pcmn{我睡够了}\end{exemple}
\begin{sous-entrée}{sɤrtɕi}{ⓔartɕiⓢ2ⓝsɤrtɕi} 
\classe{vt} \sens{1}\paradigme{dir}{nɯ-}
\begin{définition}\pfra{apaiser (la soif)}\end{définition}
\begin{définition}\pcmn{让……解渴}\end{définition}
\begin{exemple}\pjya{ɯ-ɕpaʁ ɲɯ-sɤrtɕi-a ra ma mɤ-pe}\hspace{5pt}\pcmn{我要让他解渴}\end{exemple}\end{sous-entrée}

\sens{2}\paradigme{dir}{kɤ-}
\begin{définition}\pfra{permettre de dormir assez}\end{définition}
\begin{définition}\pcmn{让……睡够}\end{définition}
\begin{exemple}\pjya{ɯ-ʑɯβ ci kú-wɣ-sɤrtɕi ɲɯ-ra}\hspace{5pt}\pcmn{要让他睡够}\end{exemple}\end{entrée}

\begin{entrée}{artɕɯwa}{}{ⓔartɕɯwa} 
\classe{n} 
\begin{définition}\pfra{mendiant}\end{définition}
\begin{définition}\pcmn{乞丐}\end{définition}\end{entrée}

\begin{entrée}{artsɯɣdu/\variante{\_artsɯrdu}}{}{ⓔartsɯɣdu} 
\classe{vi} \paradigme{dir}{pɯ-}
\begin{définition}\pfra{occupé}\end{définition}
\begin{définition}\pcmn{事务繁忙}\end{définition}
\begin{exemple}\pjya{a-ma pɯ-artsɯɣdu}\hspace{5pt}\pcmn{我工作变得很紧张}\end{exemple}
\begin{exemple}\pjya{stonka pɯ-ɣe tɕe ɯ-kɤtɣa cho kɤ-nɤma ra artsɯɣdu}\hspace{5pt}\pcmn{秋天来了,收割等农活非常忙碌}\end{exemple}
\begin{exemple}\pjya{ɯ-ŋgɯ jɤznɤ tu-kɯ-znɤʁɤmɟa tɕe, ɯ-ndo tɕe kɤ-nɤma mɤ-kɤ-ɤrtsɯɣdu phɤn}\hspace{5pt}\pcmn{一开始把时间抓紧的话,到最后就不会那么紧张}\end{exemple}\end{entrée}

\begin{entrée}{artsɯrtso}{}{ⓔartsɯrtso} 
\classe{vi} \paradigme{dir}{tɤ-}\paradigme{dir}{tɤ-}
\begin{définition}\pfra{s'empiler}\end{définition}
\begin{définition}\pcmn{重叠}\end{définition}
\begin{définition}\pfra{empiler}\end{définition}
\begin{définition}\pcmn{堆起来;垒起来}\end{définition}
\begin{exemple}\pjya{iʑo ji-khɯtsa dɤn tɕe artsɯrtso ʑo ɕti}\hspace{5pt}\pcmn{我们有很多碗,都是叠在一起的}\end{exemple}\relationsémantique{同义词}{\lien{ⓔrtsɯɣ}{rtsɯɣ}}
\begin{sous-entrée}{sɤrtsɯrtso}{ⓔartsɯrtsoⓝsɤrtsɯrtso} 
\classe{vt} \end{sous-entrée}

\end{entrée}

\begin{entrée}{artɯm}{}{ⓔartɯm} 
\classe{vs} \paradigme{dir}{tɤ-}\paradigme{dir}{tɤ-}
\begin{définition}\pfra{rond}\end{définition}
\begin{définition}\pcmn{圆形}\end{définition}
\begin{définition}\pfra{rendre rond}\end{définition}
\begin{définition}\pcmn{弄圆}\end{définition}
\begin{exemple}\pjya{kɯki ɯ-mŋu ɲɯ-ɤrtɯm}\hspace{5pt}\pcmn{它的口是圆形的}\end{exemple}
\begin{exemple}\pjya{qajɣi ɲɯ-ɤrtɯm}\hspace{5pt}\pcmn{馍馍是圆的}\end{exemple}
\begin{exemple}\pjya{@lanqiu ɲɯ-ɤrtɯm}\hspace{5pt}\pcmn{篮球是圆的}\end{exemple}
\begin{exemple}\pjya{@baigua ɲɯ-ɤrtɯm}\hspace{5pt}\pcmn{白瓜是圆的}\end{exemple}
\begin{exemple}\pjya{si tɤ-kaɣ-a tɕe tɤ-sɤrtɯm-a}\hspace{5pt}\pcmn{我把木条弯成了圆形}\end{exemple}\relationsémantique{参考}{\lien{ⓔarŋɤrtɯm}{arŋɤrtɯm}}
\begin{sous-entrée}{sɤrtɯm}{ⓔartɯmⓝsɤrtɯm} 
\classe{vt}  
\grammaire{caus} \end{sous-entrée}

\end{entrée}

\begin{entrée}{artɯmloʁ}{}{ⓔartɯmloʁ} 
\classe{vs} \paradigme{dir}{tɤ-}
\begin{définition}\pfra{en forme de boule}\end{définition}
\begin{définition}\pcmn{球形}\end{définition}
\begin{définition}\pfra{faire une boule (de tsampa)}\end{définition}
\begin{définition}\pcmn{(把糌粑)挼成一坨}\end{définition}
\begin{exemple}\pjya{ki tɯrme ki ɲɯ-tshu ɲɯ-ɤrtɯmloʁ ʑo}\hspace{5pt}\pcmn{这个人很胖,身体像球形一样}\end{exemple}
\begin{exemple}\pjya{ɲɯ-ɤrtɯmloʁ ʑo tslɯɣtslɯɣ}\hspace{5pt}\pcmn{又圆又硬}\end{exemple}
\begin{exemple}\pjya{rɟɤɣi tɤ-sɤrtɯmloʁ-a tɕe tɤ-βzu-t-a}\hspace{5pt}\pcmn{我把糌粑捏成一团了}\end{exemple}
\begin{exemple}\pjya{tɯ-ŋga to-sɤrtɯmloʁ ʑo tslɯɣtslɯɣ}\hspace{5pt}\pcmn{他把衣服乱裹成一团}\end{exemple}
\begin{sous-entrée}{sɤrtɯmloʁ}{ⓔartɯmloʁⓝsɤrtɯmloʁ} 
\classe{vt} \end{sous-entrée}

\end{entrée}

\begin{entrée}{artɯrtɤβ}{}{ⓔartɯrtɤβ} 
\classe{vi} \paradigme{dir}{tɤ-}
\begin{définition}\pfra{s'emmêler}\end{définition}
\begin{définition}\pcmn{缠在一起}\end{définition}
\begin{exemple}\pjya{tɤ-ri ɲɯ-ɤrtɯrtɤβ}\hspace{5pt}\pcmn{线缠在一起}\end{exemple}
\begin{exemple}\pjya{tɯmbri tu-ortɯrtɤβ}\hspace{5pt}\pcmn{绳子缠在一起}\end{exemple}
\begin{exemple}\pjya{tɤ-ri to-k-ɤrtɯrtɤβ-ci tɕe nɯ-sɤqɤrle-t-a}\hspace{5pt}\pcmn{线缠在一起了,我就把它分开了}\end{exemple}\relationsémantique{参考}{\lien{ⓔrtɤβ}{rtɤβ}}\relationsémantique{同义词}{\lien{}{aɬɤt}}
\begin{sous-entrée}{sɤrtɯrtɤβ}{ⓔartɯrtɤβⓝsɤrtɯrtɤβ} 
\classe{vt}  
\grammaire{caus} \sens{1}
\begin{définition}\pfra{emmêler}\end{définition}
\begin{définition}\pcmn{缠线}\end{définition}\end{sous-entrée}

\sens{2}
\begin{définition}\pfra{coller à quelqu'un}\end{définition}
\begin{définition}\pcmn{缠着别人不放}\end{définition}
\begin{exemple}\pjya{nɯ ma ma-kɯ-sɤrtɯrtaβ-a ma ɲɯ-sɤɣdɯɣ}\hspace{5pt}\pcmn{你不要再缠着我了,很烦}\end{exemple}\end{entrée}

\begin{entrée}{artɯrtoʁ}{}{ⓔartɯrtoʁ}\relationsémantique{参考}{\lien{ⓔrtoʁ}{rtoʁ}}\end{entrée}

\begin{entrée}{arɯcɤrna}{}{ⓔarɯcɤrna} 
\classe{vs}  
\grammaire{denom} \paradigme{dir}{tɤ-}\sens{1}
\begin{définition}\pfra{rond}\end{définition}
\begin{définition}\pcmn{圆形的}\end{définition}\sens{2}
\begin{définition}\pfra{courbées l'une vers l'autre (cornes)}\end{définition}
\begin{définition}\pcmn{往里面弯(角)}\end{définition}
\begin{exemple}\pjya{jla ɯ-ʁrɯ ɲɯ-ɤrɯcɤrna}\hspace{5pt}\pcmn{犏牛的角往里面弯,几乎形成了圆形}\end{exemple}
\begin{exemple}\pjya{ɯ-mŋu ɲɯ-ɤrtɯm tɕe ɲɯ-ɤrɯcɤrna}\hspace{5pt}\pcmn{(碗的)口是圆的}\end{exemple}\relationsémantique{同义词}{\lien{ⓔartɯm}{artɯm}}\relationsémantique{参考}{\lien{ⓔcɤrna}{cɤrna}}\end{entrée}

\begin{entrée}{arɯɕoʁɕoʁ}{}{ⓔarɯɕoʁɕoʁ} 
\classe{vs}  
\grammaire{denom} 
\begin{définition}\pfra{comme du papier}\end{définition}
\begin{définition}\pcmn{像纸一样}\end{définition}
\begin{exemple}\pjya{ɯ-tɯ-wɣrum kɯ ɲɯ-ɤrɯɕoʁɕoʁ}\hspace{5pt}\pcmn{像纸一样白}\end{exemple}\relationsémantique{参考}{\lien{ⓔɕoʁɕoʁ}{ɕoʁɕoʁ}}\end{entrée}

\begin{entrée}{arɯfsapaʁ}{}{ⓔarɯfsapaʁ} 
\classe{vs}  
\grammaire{denom} 
\begin{définition}\pfra{comme un animal}\end{définition}
\begin{définition}\pcmn{像牲畜一样}\end{définition}
\begin{exemple}\pjya{ɯ-tɯ-khe kɯ ɲɯ-ɤrɯfsapaʁ ʑo}\hspace{5pt}\pcmn{他像牲畜一样笨}\end{exemple}\relationsémantique{参考}{\lien{ⓔfsapaʁ}{fsapaʁ}}\end{entrée}

\begin{entrée}{arɯfsɤlko}{}{ⓔarɯfsɤlko} 
\classe{vs} 
\begin{définition}\pfra{reculé, à l'écart (endroit)}\end{définition}
\begin{définition}\pcmn{偏僻}\end{définition}\end{entrée}

\begin{entrée}{arɯjɤrɤβ/\variante{arɯjɤrɯrɤβ}}{}{ⓔarɯjɤrɤβ} 
\classe{vs} 
\begin{définition}\pfra{poli}\end{définition}
\begin{définition}\pcmn{文明;有礼貌}\end{définition}\étymologie{ja.rabs}\end{entrée}

\begin{entrée}{arɯkhɯtsa}{}{ⓔarɯkhɯtsa} 
\classe{vs} 
\begin{définition}\pfra{comme un bol}\end{définition}
\begin{définition}\pcmn{像碗一样}\end{définition}
\begin{exemple}\pjya{li sɤlaŋphɤn ɯ-tɯ-xtɕi kɯ ɲɯ-ɤrɯkhɯtsa ɕti}\hspace{5pt}\pcmn{这个盆子像碗一样小}\end{exemple}\relationsémantique{参考}{\lien{ⓔkhɯtsa}{khɯtsa}}\end{entrée}

\begin{entrée}{arɯlaba}{}{ⓔarɯlaba} 
\classe{vi}  
\grammaire{denom} 
\begin{définition}\pfra{ressemblant à une trompette}\end{définition}
\begin{définition}\pcmn{像喇叭的形状}\end{définition}
\begin{exemple}\pjya{kɯ-ɤrɯlaba ci ɲɯ-ŋu}\hspace{5pt}\pcmn{有喇叭的形状}\end{exemple}\relationsémantique{同义词}{\lien{ⓔajpomxtshɯm}{ajpomxtshɯm}}\end{entrée}

\begin{entrée}{arɯldʑaŋkɯ}{}{ⓔarɯldʑaŋkɯ} 
\classe{vs}  
\grammaire{denom} \paradigme{dir}{nɯ-}
\begin{définition}\pfra{verdâtre}\end{définition}
\begin{définition}\pcmn{淡绿色}\end{définition}
\begin{exemple}\pjya{ɲɯ-ɤrɯldʑaŋkɯ}\hspace{5pt}\pcmn{是淡绿色的}\end{exemple}\relationsémantique{参考}{\lien{ⓔldʑaŋkɯ}{ldʑaŋkɯ}}\end{entrée}

\begin{entrée}{arɯlɯŋkɤr}{}{ⓔarɯlɯŋkɤr} 
\classe{vs} \paradigme{dir}{nɯ-}
\begin{définition}\pfra{bleu ciel}\end{définition}
\begin{définition}\pcmn{天蓝色}\end{définition}
\begin{exemple}\pjya{kɯ-ɤrɯlɯŋkɤr ci ɲɯ-ŋu}\hspace{5pt}\pcmn{是天蓝色的东西}\end{exemple}\end{entrée}

\begin{entrée}{arɯmɯntoʁ}{}{ⓔarɯmɯntoʁ} 
\classe{vs}  
\grammaire{denom} 
\begin{définition}\pfra{comme une fleur}\end{définition}
\begin{définition}\pcmn{像花一样}\end{définition}
\begin{exemple}\pjya{ɯ-tɯ-ɤrɯmɯntoʁ nɯ!}\hspace{5pt}\pcmn{他像一朵花一样没有价值}\end{exemple}
\begin{exemple}\pjya{ɯ-tɯ-mpɕɤr kɯ ɲɯ-ɤrɯmɯntoʁ}\hspace{5pt}\pcmn{漂亮得像花一样}\end{exemple}\relationsémantique{参考}{\lien{ⓔmɯntoʁ}{mɯntoʁ}}\end{entrée}

\begin{entrée}{arɯqromkemdoʁ}{}{ⓔarɯqromkemdoʁ} 
\classe{vs}  
\grammaire{denom} 
\begin{définition}\pfra{violet}\end{définition}
\begin{définition}\pcmn{带有紫色}\end{définition}\relationsémantique{参考}{\lien{ⓔqromkemdoʁ}{qromkemdoʁ}}\end{entrée}

\begin{entrée}{arɯʁmɤrsmɯɣ}{}{ⓔarɯʁmɤrsmɯɣ} 
\classe{vs}  
\grammaire{denom} 
\begin{définition}\pfra{bordeau}\end{définition}
\begin{définition}\pcmn{带有紫红色}\end{définition}\relationsémantique{参考}{\lien{ⓔʁmɤrsmɯɣ}{ʁmɤrsmɯɣ}}\end{entrée}

\begin{entrée}{arɯsrɯn}{}{ⓔarɯsrɯn} 
\classe{vs} 
\begin{définition}\pfra{comme du coton}\end{définition}
\begin{définition}\pcmn{像棉花一样}\end{définition}
\begin{exemple}\pjya{ɯ-tɯ-mnu kɯ ɲɯ-ɤrɯsrɯn}\hspace{5pt}\pcmn{像棉花一样柔软}\end{exemple}\end{entrée}

\begin{entrée}{arɯsɯjno}{}{ⓔarɯsɯjno} 
\classe{vs}  
\grammaire{denom} 
\begin{définition}\pfra{comme une herbe}\end{définition}
\begin{définition}\pcmn{像……草一样……}\end{définition}
\begin{exemple}\pjya{kɤ-phɯt ɯ-tɯ-mbat kɯ ɲɯ-ɤrɯsɯjno ʑo}\hspace{5pt}\pcmn{像拔一根草一样容易拔}\end{exemple}\relationsémantique{参考}{\lien{ⓔsɯjnoⓗ1}{sɯjno₁}}\end{entrée}

\begin{entrée}{arɯtaqaβ}{}{ⓔarɯtaqaβ} 
\classe{vs}  
\grammaire{denom} 
\begin{définition}\pfra{comme une aiguille}\end{définition}
\begin{définition}\pcmn{像……针一样……}\end{définition}\relationsémantique{参考}{\lien{ⓔtaqaβ}{taqaβ}}\end{entrée}

\begin{entrée}{arɯtɤjpa}{}{ⓔarɯtɤjpa} 
\classe{vs}  
\grammaire{denom} 
\begin{définition}\pfra{comme de la neige}\end{définition}
\begin{définition}\pcmn{像雪一样}\end{définition}
\begin{exemple}\pjya{ɯ-tɯ-wɣrum kɯ ɲɯ-ɤrɯtɤjpa ʑo}\hspace{5pt}\pcmn{像雪一样白}\end{exemple}\relationsémantique{参考}{\lien{ⓔtɤjpa}{tɤjpa}}\end{entrée}

\begin{entrée}{arɯtɤpɤtso}{}{ⓔarɯtɤpɤtso} 
\classe{vs} 
\begin{définition}\pfra{comme un enfant}\end{définition}
\begin{définition}\pcmn{像小孩子一样}\end{définition}
\begin{exemple}\pjya{nɤki tɯrme nɯ ɲɯ-ɤrɯtɤpɤtso}\hspace{5pt}\pcmn{那个人像小孩子一样}\end{exemple}\relationsémantique{参考}{\lien{ⓔtɤ-pɤtso}{tɤ-pɤtso}}\end{entrée}

\begin{entrée}{arɯtɤtɕɯ}{}{ⓔarɯtɤtɕɯ} 
\classe{vs} 
\begin{définition}\pfra{comme un garçon}\end{définition}
\begin{définition}\pcmn{像男孩子一样}\end{définition}
\begin{exemple}\pjya{nɤki tɕheme nɯ ɯ-tɯ-rkaŋ kɯ ɲɯ-ɤrɯtɤtɕɯ ʑo ɕti}\hspace{5pt}\pcmn{那个女子像男生一样壮}\end{exemple}\relationsémantique{参考}{\lien{ⓔtɤ-tɕɯ}{tɤ-tɕɯ}}\end{entrée}

\begin{entrée}{arɯtɕheme}{}{ⓔarɯtɕheme} 
\classe{vs} 
\begin{définition}\pfra{comme une fille}\end{définition}
\begin{définition}\pcmn{像女子一样}\end{définition}
\begin{exemple}\pjya{nɤki tɤ-tɕɯ nɯ ɯ-tɯ-ndʑɤm kɯ ɲɯ-ɤrɯtɕheme ʑo ɕti}\hspace{5pt}\pcmn{那个男生像女子一样温柔}\end{exemple}\relationsémantique{参考}{\lien{ⓔtɕheme}{tɕheme}}\end{entrée}

\begin{entrée}{arɯzdɯxthɯɣ}{}{ⓔarɯzdɯxthɯɣ} 
\classe{vs} 
\begin{définition}\pfra{qui arrive à peine à survivre}\end{définition}
\begin{définition}\pcmn{勉强过日子}\end{définition}
\begin{exemple}\pjya{zdɯxthɯɣ}\end{exemple}\end{entrée}

\begin{entrée}{aʁɤndɯndɤt}{}{ⓔaʁɤndɯndɤt} 
\classe{adv} 
\begin{définition}\pfra{partout}\end{définition}
\begin{définition}\pcmn{到处}\end{définition}\relationsémantique{参考}{\lien{ⓔŋotɕuŋondɤt}{ŋotɕuŋondɤt}}\relationsémantique{参考}{\lien{ⓔnɤndɯndɤt}{nɤndɯndɤt}}\end{entrée}

\begin{entrée}{aʁdɤt}{}{ⓔaʁdɤt} 
\classe{vi} \paradigme{dir}{pɯ-}\paradigme{dir}{thɯ-}
\begin{définition}\pfra{glisser}\end{définition}
\begin{définition}\pcmn{跌倒}\end{définition}
\begin{exemple}\pjya{a-mi pɯ-aʁdɤt}\hspace{5pt}\pcmn{我的脚(被)绊了一下}\end{exemple}
\begin{exemple}\pjya{ɲɯ-ɤci tɕe pɯ-aʁdat-a}\hspace{5pt}\pcmn{地很湿,我就跌倒了}\end{exemple}
\begin{exemple}\pjya{ɲɯ-ɣɤrɤβ tɕe pɯ-aʁdat-a}\hspace{5pt}\pcmn{坡很陡,我就跌倒了}\end{exemple}
\begin{exemple}\pjya{pjɤ-k-ɤʁdɤt-ci}\hspace{5pt}\pcmn{他跌倒了}\end{exemple}
\begin{exemple}\pjya{rdɤstaʁ ɯ-taʁ pɯ-aʁdat-a}\hspace{5pt}\pcmn{我给石头绊倒了}\end{exemple}\end{entrée}

\begin{entrée}{aʁe}{}{ⓔaʁe} 
\classe{vi-t} \paradigme{dir}{pɯ-}\paradigme{dir}{pɯ-}
\begin{définition}\pfra{avoir à boire, avoir à manger}\end{définition}
\begin{définition}\pcmn{喝到,吃到}\end{définition}
\begin{définition}\pfra{donner à manger}\end{définition}
\begin{définition}\pcmn{给别人吃}\end{définition}
\begin{exemple}\pjya{tɤ-ndza-t-a tɕe pɯ-aʁe-a}\hspace{5pt}\pcmn{我吃到了}\end{exemple}
\begin{exemple}\pjya{paχɕi to-χtɯ ri mɯ-pɯ-aʁe-a}\hspace{5pt}\pcmn{他买了苹果,但是我没有吃到}\end{exemple}
\begin{exemple}\pjya{cha to-χtɯ tɕe pɯ́-wɣ-sɤʁe-a}\hspace{5pt}\pcmn{他买了酒就给我喝了}\end{exemple}
\begin{exemple}\pjya{mɯ-pɯ-kɯ-sɤʁe-a}\hspace{5pt}\pcmn{你没有给我吃}\end{exemple}
\begin{sous-entrée}{sɤʁe}{ⓔaʁeⓝsɤʁe} 
\classe{vt} \end{sous-entrée}

\end{entrée}

\begin{entrée}{aʁjɤr}{}{ⓔaʁjɤr} 
\classe{vi} \paradigme{dir}{nɯ-}\paradigme{dir}{pɯ-}
\begin{définition}\pfra{être en retard}\end{définition}
\begin{définition}\pcmn{耽误时间;迟到}\end{définition}
\begin{exemple}\pjya{toʁde nɯ-aʁjar-a}\hspace{5pt}\pcmn{我耽误了一下}\end{exemple}
\begin{exemple}\pjya{tɯ-rju ɲɯ-ti tɕe, nɯ-aʁjar-a}\hspace{5pt}\pcmn{他说话,我就耽误了时间}\end{exemple}
\begin{exemple}\pjya{fso tɕe aʁjar-a ra ɲɯ-ŋu}\hspace{5pt}\pcmn{我明天要耽误一下时间(没有时间上课)}\end{exemple}\relationsémantique{参考}{\lien{}{caus}}
\begin{sous-entrée}{saʁjɤr}{ⓔaʁjɤrⓝsaʁjɤr} 
\classe{vt} \end{sous-entrée}

\sens{1}
\begin{définition}\pfra{retarder, déranger}\end{définition}
\begin{définition}\pcmn{耽误;打扰}\end{définition}\relationsémantique{同义词}{\lien{ⓔβzgɤr}{βzgɤr}}\sens{2}
\begin{définition}\pfra{déranger}\end{définition}
\begin{définition}\pcmn{打扰}\end{définition}
\begin{exemple}\pjya{ma-pɯ-kɯ-saʁjar-a}\hspace{5pt}\pcmn{你不要耽误我的时间}\end{exemple}
\begin{exemple}\pjya{pɯ-kɯ-saʁjar-a}\hspace{5pt}\pcmn{你耽误了我的时间}\end{exemple}
\begin{exemple}\pjya{ɲɯ-ta-saʁjɤr}\hspace{5pt}\pcmn{我在耽误你的时间}\end{exemple}
\begin{exemple}\pjya{ɯ-zgra kɤ-ʑmbri mɤ-pe ma tɯrme saʁjar-a}\hspace{5pt}\pcmn{我不能发出声音,不然会打扰别人}\end{exemple}
\begin{exemple}\pjya{tɯ-mɯ ɲɯ-lɤt tɕe, ji-ma pɯ́-wɣ-saʁjɤr-i}\hspace{5pt}\pcmn{下雨就耽误了我们的工作}\end{exemple}
\begin{exemple}\pjya{khɤjhwi ɲɯ-ra tɕe, kɤ-ɣi mɯ-pɯ-ŋgrɯ, pɯ-ta-saʁjɤr}\hspace{5pt}\pcmn{因为需要开会没能来到,耽误了你的时间}\end{exemple}
\begin{sous-entrée}{sɤsaʁjɤr}{ⓔaʁjɤrⓢ2ⓝsɤsaʁjɤr} 
\classe{vi}  
\grammaire{apass} \end{sous-entrée}

\begin{sous-entrée}{ʑɣɤsaʁjɤr}{ⓔaʁjɤrⓢ2ⓝʑɣɤsaʁjɤr} 
\classe{vi}  
\grammaire{refl}
\grammaire{caus} 
\begin{définition}\pfra{prendre du retard}\end{définition}
\begin{définition}\pcmn{耽误时间}\end{définition}\end{sous-entrée}

\end{entrée}

\begin{entrée}{aʁɟa}{}{ⓔaʁɟa} 
\classe{vi} \paradigme{dir}{thɯ-}\sens{1}
\begin{définition}\pfra{chauve}\end{définition}
\begin{définition}\pcmn{秃}\end{définition}\sens{2}
\begin{exemple}\pjya{ɯ-kɤrme chɤ-k-ɤʁɟa-ci}\hspace{5pt}\pcmn{他变成光头了}\end{exemple}\sens{2}
\begin{définition}\pfra{dénudé (terrain)}\end{définition}
\begin{définition}\pcmn{光秃秃}\end{définition}
\begin{exemple}\pjya{jiɕqha stɤmku ɲɯ-ɤʁɟa}\hspace{5pt}\pcmn{草坪光秃秃的}\end{exemple}
\begin{exemple}\pjya{@Wenchuan zgo ra ɲɯ-ɤʁɟa-nɯ}\hspace{5pt}\pcmn{汶川的山是光秃秃的}\end{exemple}
\begin{sous-entrée}{saʁɟa}{ⓔaʁɟaⓝsaʁɟa} 
\classe{vt}  
\grammaire{caus} \paradigme{dir}{thɯ-}\paradigme{dir}{nɯ-}
\begin{définition}\pfra{raser entièrement}\end{définition}
\begin{définition}\pcmn{剃光;砍光}\end{définition}
\begin{exemple}\pjya{jiɕqha nɯ ɯ-stu ɲɤ-saʁɟa}\hspace{5pt}\pcmn{他把那个地方砍光了(拔光了草)}\end{exemple}\end{sous-entrée}

\étymologie{gja.ma |fv{“石板”}}\end{entrée}

\begin{entrée}{aʁɟɤle}{}{ⓔaʁɟɤle} 
\classe{vi} \paradigme{dir}{nɯ-}
\begin{définition}\pfra{nu, vide (lieu)}\end{définition}
\begin{définition}\pcmn{光秃秃(不均匀)}\end{définition}
\begin{exemple}\pjya{tɯ-mɯ nɯ-aʁɟɤle}\hspace{5pt}\pcmn{天晴了,云散了}\end{exemple}
\begin{exemple}\pjya{ɯ-ku ɲɯ-ɤʁɟɤle}\hspace{5pt}\pcmn{他头发稀少了}\end{exemple}\relationsémantique{参考}{\lien{ⓔaʁɟa}{aʁɟa}}\end{entrée}

\begin{entrée}{aʁloʁlu}{}{ⓔaʁloʁlu} 
\classe{vs} 
\begin{définition}\pfra{concave}\end{définition}
\begin{définition}\pcmn{凹陷}\end{définition}
\begin{exemple}\pjya{ko-k-ɤʁloʁlu-ci}\hspace{5pt}\pcmn{变成凹进去了}\end{exemple}
\begin{exemple}\pjya{jiɕqha ɲɯ-ɤʁloʁlu tɕe, nɯ ɯ-stu nɯ to-k-ɤjtɯ-ci tɯ-ci}\hspace{5pt}\pcmn{那个东西是凹进去的,在那里就积了水}\end{exemple}
\begin{exemple}\pjya{ki sɤtɕha ɲɯ-ɤʁloʁlu tɕe, tɯ-ci tu-orɣi mbat}\hspace{5pt}\pcmn{这个地方是凹进去的,水会在里面积起来流不走}\end{exemple}\relationsémantique{同义词}{\lien{}{aχchowɤlu}}\relationsémantique{参考}{\lien{ⓔɯ-ʁlu}{ɯ-ʁlu}}\end{entrée}

\begin{entrée}{aʁmbɯm}{}{ⓔaʁmbɯm} 
\classe{vi} \paradigme{dir}{kɤ-}
\begin{définition}\pfra{concave}\end{définition}
\begin{définition}\pcmn{凹(物体)}\end{définition}
\begin{exemple}\pjya{tɯthɯ ko-oʁmbɯm}\hspace{5pt}\pcmn{锅子凹进去了}\end{exemple}
\begin{exemple}\pjya{ko-k-ɤʁmbɯm-ci}\hspace{5pt}\pcmn{凹进去了}\end{exemple}
\begin{exemple}\pjya{ɯ-laz ko-nɯrpu tɕe ko-oʁmbɯm}\hspace{5pt}\pcmn{他不小心撞了额头,额头就凹进去了}\end{exemple}
\begin{exemple}\pjya{ma-kɤ-tɯ-sɯrpe ma tha aʁmbɯm}\hspace{5pt}\pcmn{你不要撞他,不然就会凹进去}\end{exemple}\end{entrée}

\begin{entrée}{aʁzrɤwolu}{}{ⓔaʁzrɤwolu} 
\classe{vs} \paradigme{dir}{nɯ-}
\begin{définition}\pfra{ébouriffé}\end{définition}
\begin{définition}\pcmn{乱蓬蓬}\end{définition}
\begin{exemple}\pjya{ɯ-ku ɲɯ-ɤʁzrɤwolu}\hspace{5pt}\pcmn{他头发乱蓬蓬}\end{exemple}
\begin{exemple}\pjya{nɤ-ku ɲɤ-k-ɤʁzrɤwolu-ci tɕe, kɤ-sɤɕɤt ɲɯ-ra}\hspace{5pt}\pcmn{你的头发变得乱蓬蓬的,要梳头}\end{exemple}
\begin{exemple}\pjya{nɯ ɕɯŋgɯ kɯ-pɯ-pe pɯ-ɕti ri, tham tɕe ɲɤ-k-ɤʁzrɤwolu-ci}\hspace{5pt}\pcmn{以前好好的,现在变得乱蓬蓬的}\end{exemple}\end{entrée}

\begin{entrée}{aʁzɯɣβzaŋ}{}{ⓔaʁzɯɣβzaŋ} 
\classe{vs}  
\grammaire{denom} \paradigme{dir}{tɤ-}
\begin{définition}\pfra{d'allure avenante, impressionnant}\end{définition}
\begin{définition}\pcmn{相貌堂堂}\end{définition}
\begin{exemple}\pjya{ɲɯ-ɤʁzɯɣβzaŋ}\hspace{5pt}\pcmn{他相貌堂堂}\end{exemple}
\begin{exemple}\pjya{nɤki nɯ kɯ-ɤʁzɯɣβzaŋ ci ŋu}\hspace{5pt}\pcmn{那是一个相貌堂堂的人}\end{exemple}
\begin{exemple}\pjya{nɯ ɕɯŋgɯ ɯ-ʁzɯɣ mɯ-pɯ-pe ri, tham to-k-ɤʁzɯɣβzaŋ-ci}\hspace{5pt}\pcmn{以前相貌不好,现在相貌变得很好}\end{exemple}\relationsémantique{反义词}{\lien{ⓔaʁzɯŋɤn}{aʁzɯŋɤn}}\étymologie{gzugs.bzaŋ}\end{entrée}

\begin{entrée}{aʁzɯŋɤn}{}{ⓔaʁzɯŋɤn} 
\classe{vs}  
\grammaire{denom} 
\begin{définition}\pfra{d'une apparence peu avenante}\end{définition}
\begin{définition}\pcmn{相貌不好}\end{définition}\relationsémantique{反义词}{\lien{ⓔaʁzɯɣβzaŋ}{aʁzɯɣβzaŋ}}\étymologie{gzugs.ŋan}\end{entrée}

\begin{entrée}{asaχpɯχpaʁ}{}{ⓔasaχpɯχpaʁ}\relationsémantique{参考}{\lien{ⓔsaχpaʁ}{saχpaʁ}}\end{entrée}

\begin{entrée}{asɤmɯβde}{}{ⓔasɤmɯβde}\relationsémantique{参考}{\lien{ⓔβde}{βde}}\end{entrée}

\begin{entrée}{asɤmɯmtshɯmtshɤm}{}{ⓔasɤmɯmtshɯmtshɤm}\relationsémantique{参考}{\lien{ⓔamɯmtshɤm}{amɯmtshɤm}}\end{entrée}

\begin{entrée}{asɤmɯsɯz}{}{ⓔasɤmɯsɯz} 
\classe{vi}  
\grammaire{recip} \paradigme{dir}{nɯ-}
\begin{définition}\pfra{se transmettre les informations les uns aux autres}\end{définition}
\begin{définition}\pcmn{互相传递消息}\end{définition}\relationsémantique{同义词}{\lien{ⓔarɤmɯzda}{arɤmɯzda}}\relationsémantique{同义词}{\lien{ⓔarɤzdɯzda}{arɤzdɯzda}}\relationsémantique{同义词}{\lien{ⓔasɤmɯmtshɯmtshɤm}{asɤmɯmtshɯmtshɤm}}\relationsémantique{参考}{\lien{ⓔsɯz}{sɯz}}\end{entrée}

\begin{entrée}{ascaʁa}{}{ⓔascaʁa} 
\classe{vs}  
\grammaire{denom} 
\begin{définition}\pfra{blanc et noir, comme une pie}\end{définition}
\begin{définition}\pcmn{黑白相间,像喜鹊的羽毛一样}\end{définition}\relationsémantique{参考}{\lien{ⓔscaʁa}{scaʁa}}\end{entrée}

\begin{entrée}{ascɯsco}{}{ⓔascɯsco}\relationsémantique{参考}{\lien{ⓔsco}{sco}}\end{entrée}

\begin{entrée}{askɯ}{}{ⓔaskɯ} 
\classe{vs} 
\begin{définition}\pfra{qui est blanche sur le dos (vache)}\end{définition}
\begin{définition}\pcmn{脊梁上有花纹(牛类)}\end{définition}
\begin{exemple}\pjya{nɯ nɯŋa nɯ ɲɯ-ɤskɯ}\hspace{5pt}\pcmn{这头牛脊梁上有白色}\end{exemple}\end{entrée}

\begin{entrée}{asqhlu}{}{ⓔasqhlu} 
\classe{vi} \paradigme{dir}{pɯ-}
\begin{définition}\pfra{concave}\end{définition}
\begin{définition}\pcmn{凹}\end{définition}
\begin{exemple}\pjya{pjɤ-k-ɤsqhlu-ci}\hspace{5pt}\pcmn{凹进去了}\end{exemple}\relationsémantique{同义词}{\lien{ⓔaχchowolu}{aχchowolu}}\relationsémantique{同义词}{\lien{ⓔaʁloʁlu}{aʁloʁlu}}\relationsémantique{同义词}{\lien{ⓔaqhowolu}{aqhowolu}}\relationsémantique{同义词}{\lien{ⓔaɕqhlu}{aɕqhlu}}\relationsémantique{同义词}{\lien{ⓔarɴɢlɯm}{arɴɢlɯm}}\end{entrée}

\begin{entrée}{asqɯsqɤr}{}{ⓔasqɯsqɤr}\relationsémantique{参考}{\lien{ⓔsqɤr}{sqɤr}}\end{entrée}

\begin{entrée}{astu}{}{ⓔastu} 
\classe{vs} \paradigme{dir}{tɤ-}\paradigme{dir}{thɯ-}
\begin{définition}\pfra{droit}\end{définition}
\begin{définition}\pcmn{直}\end{définition}
\begin{exemple}\pjya{si ɲɯ-ɤstu}\hspace{5pt}\pcmn{树是直的}\end{exemple}
\begin{exemple}\pjya{tʂu ɲɯ-ɤstu}\hspace{5pt}\pcmn{路是直的}\end{exemple}
\begin{exemple}\pjya{ɕoŋtɕa ɲɯ-ɤstu}\hspace{5pt}\pcmn{木材是直的}\end{exemple}\relationsémantique{参考}{\lien{ⓔastɤko}{astɤko}}\relationsémantique{参考}{\lien{ⓔsɤstuⓗ2}{sɤstu₂}}\relationsémantique{参考}{\lien{ⓔɯ-stuⓗ1}{ɯ-stu₁}}
\begin{sous-entrée}{ɯ-ʁɤri,astu}{ⓔastuⓝɯ-ʁɤri,astu}
\begin{définition}\pfra{avoir du succès, réussir}\end{définition}
\begin{définition}\pcmn{顺利}\end{définition}
\begin{exemple}\pjya{ɯ-ʁɤri ɲɯ-ɤstu}\hspace{5pt}\pcmn{他很顺利}\end{exemple}
\begin{exemple}\pjya{a-ʁɤri pɯ-astu}\hspace{5pt}\pcmn{我运气好,很顺利}\end{exemple}
\begin{exemple}\pjya{tɕhi tɤ-nɤmɯma-t-a a-ʁɤri astu}\hspace{5pt}\pcmn{我无论做什么都很顺利}\end{exemple}
\begin{exemple}\pjya{ɯ-ʁɤri ɯ-mɤ-tɯ-ɤstu kɯ}\hspace{5pt}\pcmn{他多么地不幸运}\end{exemple}\end{sous-entrée}

\end{entrée}

\begin{entrée}{astɤko}{}{ⓔastɤko} 
\classe{vi} \paradigme{dir}{thɯ-}\paradigme{dir}{tɤ-}
\begin{définition}\pfra{droit}\end{définition}
\begin{définition}\pcmn{直;站直;伸直}\end{définition}
\begin{exemple}\pjya{cho-k-ɤstoko-ci}\hspace{5pt}\pcmn{变直了}\end{exemple}
\begin{exemple}\pjya{kɯki tɯrme ɲɯ-ɤstɤko}\hspace{5pt}\pcmn{这个人身材端正,站得很直}\end{exemple}
\begin{exemple}\pjya{ɲɯ-tɯ-ɤstɤko}\hspace{5pt}\pcmn{你身材端正}\end{exemple}\relationsémantique{参考}{\lien{ⓔastu}{astu}}\relationsémantique{参考}{\lien{ⓔsɤstɤko}{sɤstɤko}}\end{entrée}

\begin{entrée}{asti}{}{ⓔasti} 
\classe{vi} \paradigme{dir}{nɯ-}
\begin{définition}\pfra{être bouché}\end{définition}
\begin{définition}\pcmn{堵塞着}\end{définition}
\begin{exemple}\pjya{kɯ-spoʁ ɯ-j-ásti?}\hspace{5pt}\pcmn{堵了没有}\end{exemple}
\begin{exemple}\pjya{kɯ-spoʁ ɲɤ-k-ɤstɯsti-ci tɕe, kɤ-ɣɤmɯt mɯ́j-khɯ}\hspace{5pt}\pcmn{洞堵上了,不能吹了}\end{exemple}\relationsémantique{参考}{\lien{ⓔstiⓗ1}{sti₁}}\end{entrée}

\begin{entrée}{astɯsti}{}{ⓔastɯsti} 
\classe{vt} \paradigme{dir}{nɯ-}\paradigme{dir}{thɯ-}
\begin{définition}\pfra{être bouché}\end{définition}
\begin{définition}\pcmn{堵塞}\end{définition}
\begin{exemple}\pjya{ɯ-ɕna ɲɯ-ɤstɯsti}\hspace{5pt}\pcmn{他鼻子塞了}\end{exemple}
\begin{exemple}\pjya{a-sŋɯro nɯ-astɯsti}\hspace{5pt}\pcmn{我呼吸有点困难}\end{exemple}\relationsémantique{参考}{\lien{ⓔstiⓗ1}{sti₁}}\end{entrée}

\begin{entrée}{asɯɣ}{}{ⓔasɯɣ} 
\classe{vi} \paradigme{dir}{kɤ-}
\begin{définition}\pfra{serré, tendu}\end{définition}
\begin{définition}\pcmn{紧}\end{définition}
\begin{exemple}\pjya{ɯ-xtɕɤr ɲɯ-ɤsɯɣ}\hspace{5pt}\pcmn{绑得很紧}\end{exemple}
\begin{exemple}\pjya{ɯ-ma ɲɯ-ɤsɯɣ}\hspace{5pt}\pcmn{他工作很紧张}\end{exemple}\relationsémantique{参考}{\lien{ⓔsɤsɯɣ}{sɤsɯɣ}}\end{entrée}

\begin{entrée}{asɯɣɲɯɣɲaʁ}{}{ⓔasɯɣɲɯɣɲaʁ}\relationsémantique{参考}{\lien{ⓔsɯɣɲaʁ}{sɯɣɲaʁ}}\end{entrée}

\begin{entrée}{asɯjaʁndzɯʁndzu}{}{ⓔasɯjaʁndzɯʁndzu}\relationsémantique{参考}{\lien{ⓔsɯjaʁndzu}{sɯjaʁndzu}}\end{entrée}

\begin{entrée}{asɯmɲɯmɲo}{}{ⓔasɯmɲɯmɲo} 
\classe{vi} \paradigme{dir}{nɯ-}
\begin{définition}\pfra{se pardonner}\end{définition}
\begin{définition}\pcmn{互相谅解}\end{définition}
\begin{exemple}\pjya{ndʑiʑo ʁo tɯ-asɯmɲɯmɲo-ndʑi ɕti nɤ}\hspace{5pt}\pcmn{你们俩倒挺能互相谅解的}\end{exemple}\end{entrée}

\begin{entrée}{asɯsu}{}{ⓔasɯsu} 
\classe{vi} \paradigme{dir}{tɤ-}
\begin{définition}\pfra{copuler}\end{définition}
\begin{définition}\pcmn{交配}\end{définition}
\begin{exemple}\pjya{jiɕqha qajɯ ni ɲɯ-ɤsɯsu-ndʑi}\hspace{5pt}\pcmn{虫子在交配}\end{exemple}\end{entrée}

\begin{entrée}{asɯsat}{}{ⓔasɯsat}\relationsémantique{参考}{\lien{ⓔsat}{sat}}\end{entrée}

\begin{entrée}{asɯxɕɯxɕɤt}{}{ⓔasɯxɕɯxɕɤt}\relationsémantique{参考}{\lien{ⓔsɯxɕɤt}{sɯxɕɤt}}\end{entrée}

\begin{entrée}{aʂŋaʁ}{}{ⓔaʂŋaʁ} 
\classe{vi} \paradigme{dir}{nɯ-}
\begin{définition}\pfra{se faire une entorse, se fouler le pied}\end{définition}
\begin{définition}\pcmn{扭伤;崴}\end{définition}
\begin{exemple}\pjya{ɯ-mi ɲɯ-k-ɤʂŋaʁ-ci}\hspace{5pt}\pcmn{他崴了脚}\end{exemple}
\begin{exemple}\pjya{ɯ-mke ɲɯ-k-ɤʂŋaʁ-ci}\hspace{5pt}\pcmn{他崴了脖子}\end{exemple}
\begin{exemple}\pjya{ɯ-mthɤɣ ɲɯ-k-ɤʂŋaʁ-ci}\hspace{5pt}\pcmn{他崴了腰}\end{exemple}
\begin{exemple}\pjya{a-mi ``rŋaʁ" ʑo ɲɯ-ti tɕe nɯ-aʂŋaʁ}\hspace{5pt}\pcmn{我的脚“喀嚓”一声就扭伤了}\end{exemple}\end{entrée}

\begin{entrée}{atu}{}{ⓔatu}\relationsémantique{参考}{\lien{ⓔnɤtu}{nɤtu}}\end{entrée}

\begin{entrée}{ata}{}{ⓔata} 
\classe{vi}  
\grammaire{pass} \paradigme{dir}{\_}
\begin{définition}\pfra{être posé}\end{définition}
\begin{définition}\pcmn{放着}\end{définition}\relationsémantique{参考}{\lien{ⓔta}{ta}}\end{entrée}

\begin{entrée}{ataʁki}{}{ⓔataʁki} 
\classe{vi} \paradigme{dir}{tɤ-}
\begin{définition}\pfra{l'un au dessus et l'autre en dessous}\end{définition}
\begin{définition}\pcmn{一个在上面,一个在下面}\end{définition}
\begin{définition}\pfra{mettre l'un au dessus et l'autre en dessous}\end{définition}
\begin{définition}\pcmn{一个放上面,一个放下面}\end{définition}
\begin{exemple}\pjya{a-mtɕhi cho a-ɕna tu-otaʁki ŋu}\hspace{5pt}\pcmn{我的嘴巴和鼻子一个在上面,一个在下面}\end{exemple}
\begin{exemple}\pjya{laχtɕha kɤ-ɕɯɴqoʁ tɤ-sɤtaʁki-t-a}\hspace{5pt}\pcmn{我把那些东西一个挂在上面,一个挂在下面}\end{exemple}\relationsémantique{参考}{\lien{ⓔtaʁⓗ3}{taʁ₃}}
\begin{sous-entrée}{sɤtaʁki}{ⓔataʁkiⓝsɤtaʁki} 
\classe{vt} \end{sous-entrée}

\end{entrée}

\begin{entrée}{atɤr}{}{ⓔatɤr} 
\classe{vi} \paradigme{dir}{tɤ-}
\begin{définition}\pfra{tomber}\end{définition}
\begin{définition}\pcmn{掉下}\end{définition}
\begin{exemple}\pjya{pjɤ-k-ɤtɤr-ci}\hspace{5pt}\pcmn{掉下去了}\end{exemple}
\begin{exemple}\pjya{pɯ-atɤr}\hspace{5pt}\pcmn{掉下去了}\end{exemple}
\begin{exemple}\pjya{nɤ-tʂoŋtʂoŋ rca j-atɤr j-atɤr ʑo ɲɯ-ŋu}\hspace{5pt}\pcmn{你的杯子很快就要掉下去}\end{exemple}\end{entrée}

\begin{entrée}{atɕaʁ}{}{ⓔatɕaʁ} 
\classe{vi}  
\grammaire{caus} \paradigme{dir}{nɯ-}\paradigme{dir}{nɯ-}\paradigme{dir}{nɯ-}\paradigme{dir}{tɤ-}
\begin{définition}\pfra{être sali}\end{définition}
\begin{définition}\pcmn{沾上(脏东西、水)}\end{définition}
\begin{définition}\pfra{salir}\end{définition}
\begin{définition}\pcmn{沾到}\end{définition}
\begin{définition}\pfra{salir partout}\end{définition}
\begin{définition}\pcmn{到处沾}\end{définition}
\begin{exemple}\pjya{ɯ-taʁ ɲɤ-k-ɤtɕaʁ-ci}\hspace{5pt}\pcmn{沾在上面了}\end{exemple}
\begin{exemple}\pjya{a-ŋga ɯ-taʁ tɤndʑɯɣ ɲɤ-k-ɤtɕaʁ-ci}\hspace{5pt}\pcmn{衣服上沾上了树脂}\end{exemple}
\begin{exemple}\pjya{znde ɯ-taʁ tɤrcoʁ nɯ-sɤtɕaʁ-a}\hspace{5pt}\pcmn{我把泥巴沾到墙壁上了}\end{exemple}
\begin{exemple}\pjya{snaχtsa nɯ́-wɣ-sɤtɕaʁ-a}\hspace{5pt}\pcmn{我身上沾了墨}\end{exemple}
\begin{sous-entrée}{sɤtɕaʁ}{ⓔatɕaʁⓝsɤtɕaʁ} 
\classe{vt} \end{sous-entrée}

\begin{sous-entrée}{atɕaʁlaʁ}{ⓔatɕaʁⓝatɕaʁlaʁ} 
\classe{vs} \end{sous-entrée}

\begin{sous-entrée}{sɤtɕaʁlaʁ}{ⓔatɕaʁⓝsɤtɕaʁlaʁ} 
\classe{vt} \end{sous-entrée}

\begin{exemple}\pjya{tɯtshi kú-wɣ-sɤla tɕe mɯ́j-pe ma tɯthɯ cho scoʁ khɯtsa ɲɯ-sɤtɕaʁlaʁ ʑo ɲɯ-ŋu}\hspace{5pt}\pcmn{煲粥不好,因为锅子、瓢和碗上都会沾得到处都是}\end{exemple}\end{entrée}

\begin{entrée}{atɕaʁlaʁ}{}{ⓔatɕaʁlaʁ}\relationsémantique{参考}{\lien{ⓔatɕaʁ}{atɕaʁ}}\end{entrée}

\begin{entrée}{atɕɤβ}{}{ⓔatɕɤβ} 
\classe{vi} \sens{1}\paradigme{dir}{tɤ-}
\begin{définition}\pfra{arriver à complétion}\end{définition}
\begin{définition}\pcmn{(时间)到了}\end{définition}
\begin{exemple}\pjya{tɯ-sla tɤ-atɕɤβ ʑo pɯ-rɤʑi-a}\hspace{5pt}\pcmn{我在这里待了整整一个月}\end{exemple}\sens{2}\paradigme{dir}{nɯ-}\paradigme{dir}{nɯ-}
\begin{définition}\pfra{croisés (mains, vêtements)}\end{définition}
\begin{définition}\pcmn{交错着(衣服、手)}\end{définition}
\begin{définition}\pfra{croiser}\end{définition}
\begin{définition}\pcmn{交错(衣服、手)}\end{définition}
\begin{exemple}\pjya{ɯ-jaʁ ɲɤ-sɤtɕɤβ}\hspace{5pt}\pcmn{他把手交叉放着}\end{exemple}
\begin{sous-entrée}{sɤtɕɤβ}{ⓔatɕɤβⓢ2ⓝsɤtɕɤβ} 
\classe{vt} \end{sous-entrée}

\end{entrée}

\begin{entrée}{atɕhɯtɕhɯ}{}{ⓔatɕhɯtɕhɯ}\relationsémantique{参考}{\lien{ⓔtɕhɯ}{tɕhɯ}}\end{entrée}

\begin{entrée}{atɕhɯz}{}{ⓔatɕhɯz} 
\classe{vi} \paradigme{dir}{tɤ-}\paradigme{dir}{tɤ-}
\begin{définition}\pfra{éternuer}\end{définition}
\begin{définition}\pcmn{打喷嚏}\end{définition}
\begin{définition}\pfra{faire éternuer}\end{définition}
\begin{définition}\pcmn{令人打喷嚏}\end{définition}
\begin{exemple}\pjya{nɯ-tɕhomba-a ɲɯ-ŋu ma ɲɯ-ɤtɕhɯz-a}\hspace{5pt}\pcmn{我在打喷嚏,感冒了}\end{exemple}
\begin{exemple}\pjya{jiʑora tɕe, tu-kɯ-ɤtɕhɯz tɕe, tɯrme ra kɯ, nɤ-kɯ-nɤkhu tu ɲɯ-ŋu tu-ti-nɯ, nɯ maʁ nɤ, tɯ-ci kɯ a-pɯ-kɯ-sɯɣro tɕe, tɯ-ci nɤ-kɯ-mbi tu ŋu tu-ti-nɯ}\hspace{5pt}\pcmn{人打喷嚏的时候,大家就说“有人要请你”;人被水呛到的时候,别人就说“有人要给你水”}\end{exemple}
\begin{exemple}\pjya{ɕnɤto tɤ-nɯ-lat-a tó-wɣ-sɤtɕhɯz-a}\hspace{5pt}\pcmn{我吸了鼻烟,被呛得打喷嚏}\end{exemple}
\begin{exemple}\pjya{hajtsu chɯ́-wɣ-pu tɕe, kɯ-sɤtɕhɯz ŋgrɤl}\hspace{5pt}\pcmn{烧黑椒的时候会让人打喷嚏}\end{exemple}
\begin{sous-entrée}{sɤtɕhɯz}{ⓔatɕhɯzⓝsɤtɕhɯz} 
\classe{vt}  
\grammaire{caus} \end{sous-entrée}

\end{entrée}

\begin{entrée}{atɕɯmthɯt}{}{ⓔatɕɯmthɯt} 
\classe{vi} \paradigme{dir}{tɤ-}
\begin{définition}\pfra{être formé de nombreux morceaux de tissu}\end{définition}
\begin{définition}\pcmn{由很多布块组成}\end{définition}
\begin{exemple}\pjya{ki raz kɯ ra kɤ-tɕɯmthɯt ɲɯ-khɯ}\hspace{5pt}\pcmn{可以把这几块布拼在一起}\end{exemple}
\begin{exemple}\pjya{ki raz kɯ ra a-pɯ-tu tɕe, ɲɯ-ɤtɕɯmthɯt ɕti}\hspace{5pt}\pcmn{有了这些布块就可以拼成有用的东西}\end{exemple}\end{entrée}

\begin{entrée}{atɕɯtɕit}{}{ⓔatɕɯtɕit} 
\classe{vi} \paradigme{dir}{nɯ-}\paradigme{dir}{nɯ-}\paradigme{construction}{infinitive}
\begin{définition}\pfra{être incomplet}\end{définition}
\begin{définition}\pcmn{遗漏}\end{définition}
\begin{définition}\pfra{perdre}\end{définition}
\begin{définition}\pcmn{漏掉(一点)}\end{définition}
\begin{exemple}\pjya{a-@cai pɯ-atɕɯtɕit}\hspace{5pt}\pcmn{我的菜掉了一些}\end{exemple}
\begin{exemple}\pjya{laχtɕha koŋla tɤ-rɤwum tɕe a-mɤ-nɯ-ɤtɕɯtɕit}\hspace{5pt}\pcmn{你要把东西带全,不要带漏}\end{exemple}
\begin{exemple}\pjya{ɲɤ-sɤtɕɯtɕit-a}\hspace{5pt}\pcmn{我漏掉了}\end{exemple}
\begin{exemple}\pjya{kɤ-nɯkɯɕnom ma-nɯ-tɯ-sɤtɕɯtɕit}\hspace{5pt}\pcmn{你捡青稞穗的时候不要漏掉一些}\end{exemple}
\begin{exemple}\pjya{laχtɕha koŋla tɤ-rɤwum ma tɯ-sɤtɕɯtɕit}\hspace{5pt}\pcmn{你要把东西带全,不要带漏}\end{exemple}\relationsémantique{同义词}{\lien{ⓔantɕhoʁjɤr}{antɕhoʁjɤr}}
\begin{sous-entrée}{sɤtɕɯtɕit}{ⓔatɕɯtɕitⓝsɤtɕɯtɕit} 
\classe{vt} \end{sous-entrée}

\end{entrée}

\begin{entrée}{atɕɯtʂi}{}{ⓔatɕɯtʂi} 
\classe{vi} \paradigme{dir}{\_}
\begin{définition}\pfra{se produire en même temps}\end{définition}
\begin{définition}\pcmn{几件事同时进行;状况没有改变}\end{définition}
\begin{exemple}\pjya{nɤ-kɯ-mŋɤm kɯ-mna ɯ-ɲɯ-ɤtɕɯtʂi}\hspace{5pt}\pcmn{你的病继续好转吗?}\end{exemple}
\begin{exemple}\pjya{ɲɯ-ɤtɕɯtʂi ɕti}\hspace{5pt}\pcmn{没有变化}\end{exemple}\relationsémantique{参考}{\lien{ⓔsɤtɕɯtʂi}{sɤtɕɯtʂi}}\end{entrée}

\begin{entrée}{atɕɯxtaʁ}{}{ⓔatɕɯxtaʁ}\relationsémantique{参考}{\lien{ⓔtaʁⓗ2}{taʁ₂}}\end{entrée}

\begin{entrée}{atɕɯxtʂot}{}{ⓔatɕɯxtʂot} 
\classe{vi} \paradigme{dir}{tɤ-}\paradigme{dir}{nɯ-}
\begin{définition}\pfra{prospère}\end{définition}
\begin{définition}\pcmn{兴旺}\end{définition}
\begin{définition}\pfra{rendre prospère, attiser (feu)}\end{définition}
\begin{définition}\pcmn{使兴旺;拨旺}\end{définition}
\begin{exemple}\pjya{ɯ-@gongsi ɲɯ-ɤtɕɯxtʂot}\hspace{5pt}\pcmn{他的公司很兴旺}\end{exemple}
\begin{exemple}\pjya{kɤntɕhaʁ wuma ʑo ɲɯ-ɤtɕɯxtʂot, tɕhi kɯ-ra ʑo ɣɤʑu}\hspace{5pt}\pcmn{城市非常兴旺,要什么就有什么}\end{exemple}
\begin{exemple}\pjya{smi nɯ-sɤtɕɯxtʂo-t-a}\hspace{5pt}\pcmn{我把火拨旺了一些}\end{exemple}
\begin{sous-entrée}{sɤtɕɯxtʂot}{ⓔatɕɯxtʂotⓝsɤtɕɯxtʂot} 
\classe{vt}  
\grammaire{caus} \end{sous-entrée}

\end{entrée}

\begin{entrée}{athɤri/\variante{athɤrɯri}}{}{ⓔathɤri} 
\classe{vi} 
\begin{définition}\pfra{se connecter}\end{définition}
\begin{définition}\pcmn{自然地连接起来}\end{définition}\relationsémantique{参考}{\lien{ⓔsɤthɤri}{sɤthɤri}}\end{entrée}

\begin{entrée}{athi}{}{ⓔathi} 
\classe{adv} 
\begin{définition}\pfra{en aval}\end{définition}
\begin{définition}\pcmn{下游}\end{définition}\relationsémantique{参考}{\lien{ⓔtɕɤthi}{tɕɤthi}}\end{entrée}

\begin{entrée}{athoʁmphrɤt}{}{ⓔathoʁmphrɤt} 
\classe{vt} \paradigme{dir}{tɤ-}
\begin{définition}\pfra{adéquate (pièces)}\end{définition}
\begin{définition}\pcmn{吻合(零件、盖子)}\end{définition}
\begin{exemple}\pjya{ɕoŋβzu ɯ-kɯ-βzu nɯ ɣɯ ɯ-tɯ-sprɤt nɯra wuma ɲɯ-ɤthoʁmphrɤt}\hspace{5pt}\pcmn{他做木匠的时候,把接头接得非常好}\end{exemple}\relationsémantique{参考}{\lien{ⓔsɤthoʁmphrɤt}{sɤthoʁmphrɤt}}\end{entrée}

\begin{entrée}{athoχɕaβ}{}{ⓔathoχɕaβ} 
\classe{vi} 
\begin{définition}\pfra{qui peut être relié}\end{définition}
\begin{définition}\pcmn{(可以)连在一起}\end{définition}
\begin{exemple}\pjya{tɤ-ri ɯ-ɕnɤz ni mɯ-ɲɯ-ɤthoχɕaβ / tɤ-ri ɯ-ɕnɤz mɯ-ɲɯ-ɤthoχɕaβ-ndʑi}\hspace{5pt}\pcmn{绳子的的两头不能接在一起}\end{exemple}\relationsémantique{同义词}{\lien{ⓔalɤɣɯ}{alɤɣɯ}}\end{entrée}

\begin{entrée}{atsa}{}{ⓔatsa} 
\classe{vi} \paradigme{dir}{pɯ-}\paradigme{dir}{kɤ-}\paradigme{dir}{tɤ-}\paradigme{dir}{pɯ-}
\begin{définition}\pfra{être planté}\end{définition}
\begin{définition}\pcmn{插着}\end{définition}
\begin{définition}\pfra{insérer, planter}\end{définition}
\begin{définition}\pcmn{插;戳;刺痛}\end{définition}
\begin{exemple}\pjya{tɤtshoʁ to-k-ɤtsa-ci}\hspace{5pt}\pcmn{钉子是插着的}\end{exemple}
\begin{sous-entrée}{sɤtsa}{ⓔatsaⓝsɤtsa} 
\classe{vt}  
\grammaire{caus} \end{sous-entrée}

\sens{1}
\begin{définition}\pfra{planter, enfoncer}\end{définition}
\begin{définition}\pcmn{插;戳}\end{définition}
\begin{exemple}\pjya{tɤtshoʁ pɯ-sɤtsa-t-a (=pɯ-no-t-a)}\hspace{5pt}\pcmn{我钉了钉子}\end{exemple}
\begin{exemple}\pjya{ɕɤmtshoʁ kɤ-sɤtsa-t-a (=kɤ-lat-a, kɤ-no-t-a)}\hspace{5pt}\pcmn{我钉了铁钉}\end{exemple}
\begin{exemple}\pjya{kɯm pɯ-sɤtsa-ta}\hspace{5pt}\pcmn{我锁了门}\end{exemple}
\begin{exemple}\pjya{tɤtshoʁ ko-sɤtsa}\hspace{5pt}\pcmn{他钉了钉子}\end{exemple}\sens{2}\paradigme{dir}{thɯ-}
\begin{définition}\pfra{poignarder}\end{définition}
\begin{définition}\pcmn{捅(一刀);刺痛}\end{définition}
\begin{définition}\pfra{se verrouiller automatiquement}\end{définition}
\begin{définition}\pcmn{自动地锁起来(门)}\end{définition}
\begin{exemple}\pjya{kɯm chɤ-nɯ-ʑɣɤsɤtsa}\hspace{5pt}\pcmn{门自动地关起来了}\end{exemple}
\begin{sous-entrée}{ʑɣɤsɤtsa}{ⓔatsaⓢ2ⓝʑɣɤsɤtsa} 
\classe{vi}  
\grammaire{refl}
\grammaire{caus} \end{sous-entrée}

\end{entrée}

\begin{entrée}{atsatsa}{}{ⓔatsatsa} 
\classe{intj} 
\begin{définition}\pfra{exprime la douleur (brûlure)}\end{définition}
\begin{définition}\pcmn{表示很痛,很烫}\end{définition}\end{entrée}

\begin{entrée}{atshɤxtshɯ}{}{ⓔatshɤxtshɯ} 
\classe{vi} \paradigme{dir}{tɤ-}\paradigme{construction}{infinitive}\paradigme{construction}{degree}
\begin{définition}\pfra{pressé}\end{définition}
\begin{définition}\pcmn{急促;紧张}\end{définition}
\begin{exemple}\pjya{ɯ-tɯ-ɕe ɲɯ-ɤtshɤxtshɯ}\hspace{5pt}\pcmn{他去得很仓促}\end{exemple}
\begin{exemple}\pjya{kɤ-nɤma ɲɯ-ɤtshɤxtshɯ}\hspace{5pt}\pcmn{他工作很紧张}\end{exemple}
\begin{exemple}\pjya{a-sŋɯro ɲɯ-ɤtshɤxtshɯ}\hspace{5pt}\pcmn{我呼吸很困难}\end{exemple}\end{entrée}

\begin{entrée}{atshoʁ}{}{ⓔatshoʁ}\relationsémantique{参考}{\lien{ⓔtshoʁ}{tshoʁ}}\end{entrée}

\begin{entrée}{atsɯtsu}{}{ⓔatsɯtsu} 
\classe{vi} \paradigme{dir}{pɯ-}
\begin{définition}\pfra{avoir le temps}\end{définition}
\begin{définition}\pcmn{来得及}\end{définition}
\begin{exemple}\pjya{nɤ-kɯ-mŋɤm a-tɤ-mna tɕe, kɤ-nɤma atsɯtsu ɕti wo}\hspace{5pt}\pcmn{你的病康复了,工作还来得及做。}\end{exemple}
\begin{exemple}\pjya{kɤ-nɯna mɤ-atsɯtsu}\hspace{5pt}\pcmn{来不及休息}\end{exemple}\relationsémantique{参考}{\lien{ⓔtsu}{tsu}}\end{entrée}

\begin{entrée}{atʂoʁloʁ}{}{ⓔatʂoʁloʁ} 
\classe{vi} \paradigme{dir}{tɤ-}\paradigme{dir}{tɤ-}\paradigme{dir}{pɯ-}
\begin{définition}\pfra{être mélangé}\end{définition}
\begin{définition}\pcmn{混合}\end{définition}
\begin{définition}\pfra{mélanger}\end{définition}
\begin{définition}\pcmn{混在一起}\end{définition}
\begin{exemple}\pjya{tɤɕi qaj atʂoʁloʁ}\hspace{5pt}\pcmn{青稞和小麦混在一起了}\end{exemple}
\begin{exemple}\pjya{tɯ-rdoʁ tɯ-rdoʁ kɯ fɕɤt-i ma lonba kɯ tɯrca tu-ti-j ri ji-rju atʂoʁloʁ}\hspace{5pt}\pcmn{我们一个一个地讲故事,要是我们一起讲的话就会乱}\end{exemple}
\begin{exemple}\pjya{tɕi-ŋga pɯ-sɤtʂoʁloʁ-a}\hspace{5pt}\pcmn{我们俩的衣服装在一起}\end{exemple}
\begin{sous-entrée}{sɤtʂoʁloʁ}{ⓔatʂoʁloʁⓝsɤtʂoʁloʁ} 
\classe{vt} \end{sous-entrée}

\end{entrée}

\begin{entrée}{atɯɣ}{}{ⓔatɯɣ} 
\classe{vi-t}  
\grammaire{refl} \paradigme{dir}{nɯ-}\sens{1}
\begin{définition}\pfra{rencontrer}\end{définition}
\begin{définition}\pcmn{遇见}\end{définition}\sens{2}\paradigme{dir}{\_}
\begin{définition}\pfra{toucher}\end{définition}
\begin{définition}\pcmn{触碰}\end{définition}
\begin{exemple}\pjya{nɯ-atɯɣ-ndʑi}\hspace{5pt}\pcmn{他们俩遇到了(他)}\end{exemple}
\begin{exemple}\pjya{tɕhomba ɲɤ-k-ɤtɯɣ-a-ci}\hspace{5pt}\pcmn{我感冒了}\end{exemple}
\begin{sous-entrée}{nɤtɯɣ}{ⓔatɯɣⓢ2ⓝnɤtɯɣ} 
\classe{vi}  
\grammaire{autoben} 
\begin{définition}\pfra{se retrouver dans/avec (par hasard)}\end{définition}
\begin{définition}\pcmn{(无意中)遇到}\end{définition}
\begin{exemple}\pjya{jisŋi kɯ-ɣɤndʐo tɕe ɲɤ-nɤtɯɣ-a}\hspace{5pt}\pcmn{今天我碰上个大冷天}\end{exemple}
\begin{exemple}\pjya{a-ʁɤri pɯ-astu ma kɯ-pe ɲɤ-nɤtɯɣ-a}\hspace{5pt}\pcmn{我运气好,拿到好东西了}\end{exemple}\end{sous-entrée}

\begin{sous-entrée}{anɤtɯtɯɣ}{ⓔatɯɣⓢ2ⓝanɤtɯtɯɣ} 
\classe{vi}  
\grammaire{autoben}
\grammaire{recip} 
\begin{définition}\pfra{se rencontrer par hasard}\end{définition}
\begin{définition}\pcmn{正巧相遇}\end{définition}
\begin{exemple}\pjya{kɯ-rɤχtɯ jo-ɕe-ndʑi tɕe ɲɤ-k-ɤnɤtɯtɯɣ-ndʑi-ci}\hspace{5pt}\pcmn{他们俩去卖东西,偶然地碰见了}\end{exemple}\end{sous-entrée}

\begin{sous-entrée}{ʑɣɤnɤtɯɣ}{ⓔatɯɣⓢ2ⓝʑɣɤnɤtɯɣ} 
\classe{vi} \end{sous-entrée}

\paradigme{dir}{\_}
\begin{définition}\pfra{faire en sorte de rencontrer, d'obtenir}\end{définition}
\begin{définition}\pcmn{想办法拿到/遇到}\end{définition}
\begin{exemple}\pjya{laχtɕha kɯ-pe tɤ-tu tɕe, ɲɯ-kɯ-ʑɣɤ-nɤtɯɣ ra}\hspace{5pt}\pcmn{有好东西的时候,要想办法拿到手}\end{exemple}
\begin{sous-entrée}{znɤtɯɣ/\variante{znɤtɯtɯɣ}}{ⓔatɯɣⓝznɤtɯɣ} 
\classe{vt} \end{sous-entrée}

\sens{1}
\begin{définition}\pfra{mettre bien en place}\end{définition}
\begin{définition}\pcmn{对端;对着放}\end{définition}\sens{2}
\begin{définition}\pfra{profiter de l'occasion}\end{définition}
\begin{définition}\pcmn{趁……的机会、恰恰在那个时候……}\end{définition}
\begin{exemple}\pjya{kɤ-qanɯ ʑo a-jɤ-tɯ-znɤtɯɣ tɕe a-jɤ-tɯ-ɣi nɯ!}\hspace{5pt}\pcmn{你趁天黑的时候来吧}\end{exemple}
\begin{exemple}\pjya{kɤ-nɯʑɯβ ʑo a-nɯ-tɯ-znɤtɯɣ tɕe mɤ-mtshɤm}\hspace{5pt}\pcmn{你趁他睡着了的机会,他就听不到}\end{exemple}
\begin{exemple}\pjya{lu-fsoʁ ɕɯŋgɯ ʑo a-nɯ-tɯ-znɤtɯɣ}\hspace{5pt}\pcmn{你趁天还没有亮}\end{exemple}\sens{3}
\begin{définition}\pfra{laisser à}\end{définition}
\begin{définition}\pcmn{让给}\end{définition}
\begin{exemple}\pjya{kɯki laχtɕha kɯ-pe tɤ-tu tɕe, nɤʑo ɲɯ-ta-znɤtɯɣ jɤɣ}\hspace{5pt}\pcmn{有好东西,我可以让给你}\end{exemple}
\begin{sous-entrée}{ʑɣɤsɤtɯɣ}{ⓔatɯɣⓢ3ⓝʑɣɤsɤtɯɣ} 
\classe{vi}  
\grammaire{refl}
\grammaire{caus} 
\begin{définition}\pfra{aller à la rencontrer de}\end{définition}
\begin{définition}\pcmn{去找(某人)}\end{définition}\relationsémantique{参考}{\lien{ⓔamɯtɯɣ}{amɯtɯɣ}}\end{sous-entrée}

\end{entrée}

\begin{entrée}{atɯta}{}{ⓔatɯta} 
\classe{vi}  
\grammaire{recip} \paradigme{dir}{nɯ-}\paradigme{dir}{nɯ-}
\begin{définition}\pfra{se relâcher les uns les autres}\end{définition}
\begin{définition}\pcmn{放开对方,放手}\end{définition}
\begin{définition}\pfra{séparer (des gens qui se battent)}\end{définition}
\begin{définition}\pcmn{劝开}\end{définition}
\begin{exemple}\pjya{ɲɤ-k-ɤtɯta-nɯ-ci}\hspace{5pt}\pcmn{他们放开对方了}\end{exemple}\relationsémantique{参考}{\lien{ⓔta}{ta}}
\begin{sous-entrée}{sɤtɯta}{ⓔatɯtaⓝsɤtɯta} 
\classe{vt} \end{sous-entrée}

\end{entrée}

\begin{entrée}{awi}{}{ⓔawi} 
\classe{vi} \paradigme{dir}{kɤ-}
\begin{définition}\pfra{se fermer (yeux)}\end{définition}
\begin{définition}\pcmn{闭着眼睛}\end{définition}
\begin{exemple}\pjya{tɯ-ʑɯβ pɯ-ɣe tɕe, tɯ-mɲaʁ ɲɯ-ɤwi}\hspace{5pt}\pcmn{打瞌睡的时候,眼睛是闭着的}\end{exemple}\relationsémantique{参考}{\lien{ⓔsɤwi}{sɤwi}}\end{entrée}

\begin{entrée}{awɯwum}{}{ⓔawɯwum} 
\classe{vi} \paradigme{dir}{tɤ-}\paradigme{dir}{thɯ-}
\begin{définition}\pfra{se rassembler}\end{définition}
\begin{définition}\pcmn{聚集在一起}\end{définition}
\begin{exemple}\pjya{kɤ-nɤʁaʁ tɤ-mda tɕe chɯ-ɤwɯwum-i ŋgrɤl}\hspace{5pt}\pcmn{要玩的时候,我们就聚在一起}\end{exemple}\relationsémantique{参考}{\lien{ⓔwum}{wum}}\end{entrée}

\begin{entrée}{axtɕɯxte}{}{ⓔaxtɕɯxte} 
\classe{vs} 
\begin{définition}\pfra{de taille différente}\end{définition}
\begin{définition}\pcmn{大小不一}\end{définition}
\begin{exemple}\pjya{kɤndʑɯʁi ni wuma ɲɯ-ɤxtɕɯxte-ndʑi}\hspace{5pt}\pcmn{两兄弟一个大一个小}\end{exemple}
\begin{exemple}\pjya{kɯ-ɤxtɕɯxte ci ɣɤʑu-ndʑi}\hspace{5pt}\pcmn{有一个大有一个小}\end{exemple}\relationsémantique{参考}{\lien{ⓔxtɕi}{xtɕi}}\relationsémantique{参考}{\lien{ⓔmɯxte}{mɯxte}}\end{entrée}

\begin{entrée}{aχa}{}{ⓔaχa} 
\classe{vi} \paradigme{dir}{nɯ-}\paradigme{dir}{pɯ-}\paradigme{dir}{nɯ-}
\begin{définition}\pfra{manquer un morceau}\end{définition}
\begin{définition}\pcmn{缺(口)}\end{définition}
\begin{définition}\pfra{faire perdre un morceau}\end{définition}
\begin{définition}\pcmn{使缺口}\end{définition}
\begin{exemple}\pjya{tʂu ɲɯ-ɤχa}\hspace{5pt}\pcmn{路面破了坑}\end{exemple}
\begin{exemple}\pjya{ɯ-ɕɣa ɲɯ-ɤχa}\hspace{5pt}\pcmn{他牙齿缺了个口子}\end{exemple}
\begin{exemple}\pjya{pjɤ-k-ɤχa-ci}\hspace{5pt}\pcmn{少了个口子}\end{exemple}
\begin{exemple}\pjya{khɯtsa ɲɯ-ɤχa}\hspace{5pt}\pcmn{碗缺了个口子}\end{exemple}
\begin{exemple}\pjya{ɯ-sta ɲɯ-ɤχa}\hspace{5pt}\pcmn{他不在(缺席了)}\end{exemple}
\begin{exemple}\pjya{ɲɤ-sɤχa}\hspace{5pt}\pcmn{他(把这个东西)弄缺了个口子}\end{exemple}\relationsémantique{参考}{\lien{ⓔɕɣɤχa}{ɕɣɤχa}}
\begin{sous-entrée}{sɤχa}{ⓔaχaⓝsɤχa} 
\classe{vt}  
\grammaire{caus} \end{sous-entrée}

\end{entrée}

\begin{entrée}{aχchowolu}{}{ⓔaχchowolu} 
\classe{vi} \paradigme{dir}{kɤ-}
\begin{définition}\pfra{creux, concave}\end{définition}
\begin{définition}\pcmn{凹}\end{définition}
\begin{exemple}\pjya{tʂu kɯ-ɤχchowolu ɯ-stu nɯ tɕu, @qiche tu-ɣɤrkhoŋloŋ ʑo ɲɯ-ŋu tɕe ɲɯ-sɤɣdɯɣ}\hspace{5pt}\pcmn{路凹下去的地方,汽车就会颠簸,坐起来很不舒服}\end{exemple}\relationsémantique{同义词}{\lien{ⓔaʁloʁlu}{aʁloʁlu}}\relationsémantique{参考}{\lien{ⓔʁlɯβʁlɯβ}{ʁlɯβʁlɯβ}}\end{entrée}

\begin{entrée}{aχɕɯβ}{}{ⓔaχɕɯβ} 
\classe{vi} \paradigme{dir}{tɤ-}
\begin{définition}\pfra{être debout ou allongé l'un à côté de l'autre}\end{définition}
\begin{définition}\pcmn{一起躺着;一起站着}\end{définition}
\begin{exemple}\pjya{stukɤr ɲɯ-ɤχɕɯβ}\hspace{5pt}\pcmn{梁是双根的}\end{exemple}
\begin{exemple}\pjya{tɯmbri ɲɯ-ɤχɕɯβ}\hspace{5pt}\pcmn{绳子是两股的}\end{exemple}
\begin{exemple}\pjya{tɕelo ɲɯ-ɤχɕɯβ-ndʑi tɕe ɲɯ-ndzur-ndʑi}\hspace{5pt}\pcmn{他们俩一起在那里站着}\end{exemple}\relationsémantique{参考}{\lien{ⓔsaχɕɯβ}{saχɕɯβ}}\étymologie{gɕib}\end{entrée}

\begin{entrée}{aχɕɯldɤn}{}{ⓔaχɕɯldɤn} 
\classe{vs} 
\begin{définition}\pfra{en sécurité}\end{définition}
\begin{définition}\pcmn{安全;安康}\end{définition}
\begin{exemple}\pjya{ku-oχɕɯldɤn-i}\hspace{5pt}\pcmn{我们都平安无事}\end{exemple}\relationsémantique{参考}{\lien{ⓔχɕɯldɤn}{χɕɯldɤn}}\end{entrée}

\begin{entrée}{aχom}{}{ⓔaχom} 
\classe{vi} \paradigme{dir}{tɤ-}
\begin{définition}\pfra{bailler}\end{définition}
\begin{définition}\pcmn{打哈欠}\end{définition}
\begin{exemple}\pjya{a-ʑɯβ ɲɯ-ɣi tɕe, tɤ-aχom-a}\hspace{5pt}\pcmn{我困了,所以打了哈欠}\end{exemple}
\begin{exemple}\pjya{khɯna ɲɯ-ɤχom}\hspace{5pt}\pcmn{狗在打哈欠}\end{exemple}
\begin{exemple}\pjya{a, ɲɯ-ɤχom-a}\hspace{5pt}\pcmn{啊,我在打哈欠}\end{exemple}\end{entrée}

\begin{entrée}{aχpɯχpjɤt}{}{ⓔaχpɯχpjɤt}\relationsémantique{参考}{\lien{ⓔχpjɤt}{χpjɤt}}\end{entrée}

\begin{entrée}{aχsi}{}{ⓔaχsi} 
\classe{vs} \paradigme{dir}{tɤ-}
\begin{définition}\pfra{propre}\end{définition}
\begin{définition}\pcmn{干净;没有剩余}\end{définition}
\begin{exemple}\pjya{tɕhaʁla ra to-raʁrɯz-nɯ tɕe to-k-ɤχsi-ci}\hspace{5pt}\pcmn{他们扫了院子,现在院子显得很干净了}\end{exemple}\relationsémantique{参考}{\lien{ⓔsaχsi}{saχsi}}\end{entrée}

\begin{entrée}{aχsom}{}{ⓔaχsom} 
\classe{vi} \paradigme{dir}{nɯ-}\paradigme{dir}{nɯ-}
\begin{définition}\pfra{être réveillé}\end{définition}
\begin{définition}\pcmn{清醒}\end{définition}
\begin{définition}\pfra{réveiller}\end{définition}
\begin{définition}\pcmn{弄醒}\end{définition}
\begin{exemple}\pjya{nɯ-aχsom-a}\hspace{5pt}\pcmn{我醒了}\end{exemple}
\begin{exemple}\pjya{ɲɤ-k-ɤχsom-ci}\hspace{5pt}\pcmn{他醒了}\end{exemple}
\begin{exemple}\pjya{jiɕqha pɯ-nɯʑɯβa ri, tham tɕe nɯ-aχsom-a}\hspace{5pt}\pcmn{我刚才睡着了,现在就醒了}\end{exemple}
\begin{exemple}\pjya{nɤki tɤ-pɤtso nɯ kɯ-ɤχsom ci ɲɯ-ŋu}\hspace{5pt}\pcmn{这个小孩子看起来很聪明}\end{exemple}
\begin{exemple}\pjya{aʑo nɯ-saχsom-a}\hspace{5pt}\pcmn{我把他弄醒}\end{exemple}
\begin{sous-entrée}{saχsom}{ⓔaχsomⓝsaχsom} 
\classe{vt}  
\grammaire{caus} \end{sous-entrée}

\end{entrée}

\begin{entrée}{aχsɯko}{}{ⓔaχsɯko} 
\classe{vs} \paradigme{dir}{tɤ-}\paradigme{dir}{tɤ-}
\begin{définition}\pfra{propre}\end{définition}
\begin{définition}\pcmn{干净}\end{définition}
\begin{définition}\pfra{rendre propre}\end{définition}
\begin{définition}\pcmn{弄干净}\end{définition}
\begin{exemple}\pjya{jiɕqha ra nɯ-kha ra wuma ɲɯ-ɤχsɯko}\hspace{5pt}\pcmn{他们的家很整洁}\end{exemple}
\begin{exemple}\pjya{kha ra tú-wɣ-raʁrɯz tɕe tú-wɣ-saχsɯko ɲɯ-ra}\hspace{5pt}\pcmn{要把家里扫干净}\end{exemple}\relationsémantique{参考}{\lien{ⓔaχsi}{aχsi}}\relationsémantique{参考}{\lien{ⓔsaχsi}{saχsi}}
\begin{sous-entrée}{saχsɯko}{ⓔaχsɯkoⓝsaχsɯko} 
\classe{vt} \end{sous-entrée}

\end{entrée}

\begin{entrée}{aχtɕɤz}{}{ⓔaχtɕɤz} 
\classe{n} 
\begin{définition}\pfra{terme affectueux pour les enfants}\end{définition}
\begin{définition}\pcmn{对下一代的爱称}\end{définition}\étymologie{gtɕes}\end{entrée}

\begin{entrée}{azbraʁ}{}{ⓔazbraʁ}\relationsémantique{参考}{\lien{ⓔzbraʁ}{zbraʁ}}\end{entrée}

\begin{entrée}{azdaʁ}{}{ⓔazdaʁ} 
\classe{vi} \paradigme{dir}{tɤ-}\paradigme{dir}{tɤ-}
\begin{définition}\pfra{avoir deux couches}\end{définition}
\begin{définition}\pcmn{有双层}\end{définition}
\begin{définition}\pfra{mettre deux couches}\end{définition}
\begin{définition}\pcmn{做成双层的}\end{définition}
\begin{exemple}\pjya{jɤlwa ɲɯ-ɤzdaʁ}\hspace{5pt}\pcmn{有双层窗帘}\end{exemple}
\begin{exemple}\pjya{khɯɣɲɟɯ kɯ-ɤzdaʁ ɣɤʑu}\hspace{5pt}\pcmn{有双层窗子}\end{exemple}
\begin{exemple}\pjya{a-ŋga tɤ-sɤzdaʁ-a tɕe mɤ-nɤndʐo-a}\hspace{5pt}\pcmn{我穿了两层衣服,不觉得冷}\end{exemple}
\begin{sous-entrée}{sɤzdaʁ}{ⓔazdaʁⓝsɤzdaʁ} 
\classe{vt} \end{sous-entrée}

\end{entrée}

\begin{entrée}{azgroʁ}{}{ⓔazgroʁ}\relationsémantique{参考}{\lien{ⓔzgroʁⓗ1}{zgroʁ₁}}\end{entrée}

\begin{entrée}{azgrɯ}{}{ⓔazgrɯ} 
\classe{vi} \paradigme{dir}{tɤ-}
\begin{définition}\pfra{courber le dos}\end{définition}
\begin{définition}\pcmn{弯下腰;鞠躬}\end{définition}
\begin{exemple}\pjya{jiɕqha tɯrme nɯ ɲɯ-ɤzgrɯ}\hspace{5pt}\pcmn{那个人弯着腰}\end{exemple}
\begin{exemple}\pjya{to-k-ɤzgrɯ-ci}\hspace{5pt}\pcmn{他弯下了腰}\end{exemple}\end{entrée}

\begin{entrée}{azgɯr}{}{ⓔazgɯr} 
\classe{vi} \paradigme{dir}{tɤ-}
\begin{définition}\pfra{être recroquevillé}\end{définition}
\begin{définition}\pcmn{驼背;全身蜷缩}\end{définition}
\begin{exemple}\pjya{ɯʑo to-ngo tɕe, ɲɯ-ɤzgɯr ʑo}\hspace{5pt}\pcmn{因为他病了,所以全身蜷缩}\end{exemple}
\begin{exemple}\pjya{a-mgɯr tɤ-mŋɤm tɕe tu-ostu-a ŋu, mɯ-tɤ-mŋɤm tɕe pjɯ-ɤzgɯr-a ŋu}\hspace{5pt}\pcmn{我背痛的时候就坐直,不痛的时候就弯腰}\end{exemple}
\begin{exemple}\pjya{to-k-ɤzgɯr-ci}\hspace{5pt}\pcmn{他背驼了}\end{exemple}\relationsémantique{参考}{\lien{ⓔazgrɯ}{azgrɯ}}\relationsémantique{参考}{\lien{ⓔazgɯrloʁ}{azgɯrloʁ}}\relationsémantique{参考}{\lien{ⓔɯ-zgɯr}{ɯ-zgɯr}}\end{entrée}

\begin{entrée}{azgɯrloʁ}{}{ⓔazgɯrloʁ} 
\classe{vi} \paradigme{dir}{pɯ-}
\begin{définition}\pfra{se recroqueviller}\end{définition}
\begin{définition}\pcmn{蜷缩}\end{définition}
\begin{exemple}\pjya{pjɤ-k-ɤzgɯrloʁ-ci}\hspace{5pt}\pcmn{他缩成了一团}\end{exemple}\relationsémantique{参考}{\lien{ⓔazgɯr}{azgɯr}}\étymologie{sgur}\end{entrée}

\begin{entrée}{azɣɤʁrɯʁre}{}{ⓔazɣɤʁrɯʁre}\relationsémantique{参考}{\lien{ⓔɣɤʁre}{ɣɤʁre}}\end{entrée}

\begin{entrée}{azmɯjqhɯjqha}{}{ⓔazmɯjqhɯjqha} 
\classe{vi} 
\begin{définition}\pfra{se faire du mal les uns aux autres}\end{définition}
\begin{définition}\pcmn{互相伤害}\end{définition}
\begin{exemple}\pjya{ʑɤni to-k-ɤzmɯjqhɯjqha-ndʑi-ci}\hspace{5pt}\pcmn{他们俩互相伤害了对方}\end{exemple}\relationsémantique{参考}{\lien{ⓔqha}{qha}}\end{entrée}

\begin{entrée}{aʑaʁ}{₁}{ⓔaʑaʁⓗ1} 
\classe{vi} \paradigme{dir}{pɯ-}
\begin{définition}\pfra{couler}\end{définition}
\begin{définition}\pcmn{漏水}\end{définition}
\begin{exemple}\pjya{tɯthɯ ɲɯ-ɤʑaʁ}\hspace{5pt}\pcmn{锅子在漏水}\end{exemple}
\begin{exemple}\pjya{pjɤ-k-ɤʑaʁ-ci}\hspace{5pt}\pcmn{漏水了}\end{exemple}\relationsémantique{参考}{\lien{ⓔari}{ari}}\end{entrée}

\begin{entrée}{aʑɤwu}{}{ⓔaʑɤwu} 
\classe{vs}  
\grammaire{denom} \paradigme{dir}{nɯ-}
\begin{définition}\pfra{être boiteux}\end{définition}
\begin{définition}\pcmn{跛脚}\end{définition}
\begin{exemple}\pjya{ɯ-mi ɲɯ-ɤʑɤwu}\hspace{5pt}\pcmn{他跛脚}\end{exemple}
\begin{exemple}\pjya{kɯ-ɤʑɤwu ʑo jo-nɯɕe}\hspace{5pt}\pcmn{他跛着脚地回家了}\end{exemple}\relationsémantique{同义词}{\lien{ⓔaɕkala}{aɕkala}}\relationsémantique{参考}{\lien{ⓔʑɤwu}{ʑɤwu}}\end{entrée}

\begin{entrée}{aʑɴɢɯʑɴɢoʁ}{}{ⓔaʑɴɢɯʑɴɢoʁ}\relationsémantique{参考}{\lien{ⓔʑɴɢoʁ}{ʑɴɢoʁ}}\end{entrée}

\begin{entrée}{aʑo}{}{ⓔaʑo} 
\classe{pro} 
\begin{définition}\pfra{moi}\end{définition}
\begin{définition}\pcmn{我}\end{définition}
\begin{exemple}\pjya{aʑo maʁ-a!}\hspace{5pt}\pcmn{不是我!}\end{exemple}\relationsémantique{参考}{\lien{ⓔaj}{aj}}\end{entrée}

\begin{entrée}{aʑɯrja}{}{ⓔaʑɯrja} 
\classe{vi}  
\grammaire{caus} \paradigme{dir}{nɯ-}\paradigme{dir}{thɯ-}\paradigme{dir}{\_}\paradigme{dir}{nɯ-}
\begin{définition}\pfra{faire la queue}\end{définition}
\begin{définition}\pcmn{排队}\end{définition}
\begin{exemple}\pjya{tɯrme ɲɯ-dɤn tɕe ɲɯ-ɤʑɯrja-nɯ ʑo}\hspace{5pt}\pcmn{人很多,在排队}\end{exemple}
\begin{exemple}\pjya{tɤjmɤɣ ɲɯ-ɤʑɯrja}\hspace{5pt}\pcmn{菌子长得很整齐}\end{exemple}
\begin{exemple}\pjya{thɯ-aʑɯrja-j}\hspace{5pt}\pcmn{我们排队了}\end{exemple}
\begin{exemple}\pjya{znde ɯ-taʁ zɯ tɯrme ra ɲɯ-ɤkɤʑɯrja-nɯ-ci}\hspace{5pt}\pcmn{人们背靠着墙排起了队}\end{exemple}
\begin{sous-entrée}{sɤʑɯrja}{ⓔaʑɯrjaⓝsɤʑɯrja} 
\classe{vt} \end{sous-entrée}

\paradigme{dir}{\_}
\begin{définition}\pfra{aligner}\end{définition}
\begin{définition}\pcmn{排列整齐}\end{définition}
\begin{exemple}\pjya{kha ɲɤ-sɤʑɯrja-nɯ ʑo}\hspace{5pt}\pcmn{他们把房子排列整齐了}\end{exemple}\end{entrée}

\begin{entrée}{aʑɯχtso}{}{ⓔaʑɯχtso} 
\classe{vs} \paradigme{dir}{tɤ-}\paradigme{dir}{tɤ-}
\begin{définition}\pfra{propre, bien entretenu}\end{définition}
\begin{définition}\pcmn{干净,整洁(有人弄干净)}\end{définition}
\begin{définition}\pfra{rendre propre}\end{définition}
\begin{définition}\pcmn{弄干净}\end{définition}
\begin{exemple}\pjya{tɯ-ci ɲɯ-ɤʑɯχtso}\hspace{5pt}\pcmn{水很干净}\end{exemple}
\begin{exemple}\pjya{kɤndza ɲɯ-ɤʑɯχtso}\hspace{5pt}\pcmn{食物很干净}\end{exemple}
\begin{exemple}\pjya{mɯ́j-tɯ-ɤʑɯχtso}\hspace{5pt}\pcmn{你不干净}\end{exemple}
\begin{exemple}\pjya{tɤ-sɤʑɯχtso-t-a}\hspace{5pt}\pcmn{我弄干净了}\end{exemple}\relationsémantique{参考}{\lien{ⓔχtso}{χtso}}
\begin{sous-entrée}{sɤʑɯχtso}{ⓔaʑɯχtsoⓝsɤʑɯχtso} 
\classe{vt}  
\grammaire{caus} \end{sous-entrée}

\end{entrée}

\begin{entrée}{aʑɯʑu}{}{ⓔaʑɯʑu} 
\classe{vi} \paradigme{dir}{tɤ-}\paradigme{dir}{tɤ-}
\begin{définition}\pfra{faire de la lutte}\end{définition}
\begin{définition}\pcmn{摔跤;角力}\end{définition}
\begin{définition}\pfra{lutter avec}\end{définition}
\begin{définition}\pcmn{跟……一起角力}\end{définition}
\begin{exemple}\pjya{tɤ-aʑɯʑu-ndʑi}\hspace{5pt}\pcmn{他们俩一起角力了}\end{exemple}
\begin{exemple}\pjya{tɤ́-wɣ-nɤʑɯʑu-a}\hspace{5pt}\pcmn{他跟我角力了}\end{exemple}
\begin{sous-entrée}{nɤʑɯʑu}{ⓔaʑɯʑuⓝnɤʑɯʑu} 
\classe{vt}  
\grammaire{appl} \end{sous-entrée}

\end{entrée}

\newpage\caractère{b}

\begin{entrée}{babɯ}{}{ⓔbabɯ} 
\classe{n} 
\begin{définition}\pfra{cassis}\end{définition}
\begin{définition}\pcmn{黑茶藨子}\end{définition}
\begin{exemple}\pjya{babɯ nɯ si kɯ-mbɤr tsa ci ŋu. tɯrme ɯ-taʁ kɯ-xtɕɯ-xtɕi ma tu-mbro mɤ-cha. ɯ-ru nɯ kɯ-pɣi tɕe, ɯ-rqhu pjɯ-kɯ-ɴɢaʁ kɯ-fse kɯ-tu ŋu, tɯ-xpa ɲɯ-rɯmɯntoʁ tɕe, nɯɕɯmɯma ɯ-mat chɯ-βze ŋu, ɯ-mat nɯ thɯ-tɯt tɕe ɲaʁ, kɤ-ndza mɯm, chi. ɯ-mat ɯ-ŋgɯ ɯ-rɣi kɯ-ndɯβ kɯ-dɤn tsa tɕe kɯ-mpɯ ŋu. ɯ-jwaʁ ndɯβ cho dɤn. zgoku ɯ-taʁ ɯ-pa ʑo tu-ɬoʁ cha.}\hspace{5pt}\pcmn{\lien{ⓔbabɯ}{babɯ} 是矮小的树种,比人高不出多少。树干是灰色的,有好像快要脱落的树皮。当年开花,马上结果,果实成熟后是黑色的,好吃,很甜。果实里有又小又多的种子,是很嫩的,叶子小而多。山上山下都能生长。}\end{exemple}\end{entrée}

\begin{entrée}{bɤbɤβ}{}{ⓔbɤbɤβ} 
\classe{idph.2} \sens{1}
\begin{définition}\pfra{obstiné}\end{définition}
\begin{définition}\pcmn{形容固执的样子}\end{définition}
\begin{exemple}\pjya{jiɕqha tɯrme nɯ, ɯ-phe ti mɤ-ti maŋe, bɤbɤβ ʑo ku-nɯ-rɤʑi ɕti}\hspace{5pt}\pcmn{那个人,给他说是没有用的,他什么都听不进去}\end{exemple}\sens{2}\paradigme{dir}{pɯ-}
\begin{définition}\pfra{épais, lourd et peu pratique, poussant en touffe (champignons)}\end{définition}
\begin{définition}\pcmn{形容物体笨重或者菌子长在一块的样子}\end{définition}
\begin{définition}\pfra{jeter de toutes ses forces}\end{définition}
\begin{définition}\pcmn{不顾一切地往下扔(重的东西)}\end{définition}
\begin{exemple}\pjya{tɤjmɤɣ ɲɯ-xcat ʑo bɤbɤβ ʑo ɲɯ-pa}\hspace{5pt}\pcmn{菌子很多,都长在一块}\end{exemple}
\begin{exemple}\pjya{sɯmat ɲɯ-ɲaʁ ʑo bɤbɤβ ʑo ɲɯ-pa}\hspace{5pt}\pcmn{黑果子都长在一块}\end{exemple}
\begin{exemple}\pjya{tɤjpa khoxtu ɕ-tha-βde tɕe, pa-nɯbɤβ ʑo pa-βde}\hspace{5pt}\pcmn{他把雪从房背上不顾一切地扔下去了}\end{exemple}
\begin{exemple}\pjya{rdɤstaʁ pa-nɯbɤβ ʑo pa-βde}\hspace{5pt}\pcmn{他把石头不顾一切地扔下去了}\end{exemple}
\begin{sous-entrée}{bɤβ}{ⓔbɤbɤβⓢ2ⓝbɤβ} 
\classe{idph.1} 
\begin{définition}\pfra{bruit d'un objet lourd qui tombe de haut}\end{définition}
\begin{définition}\pcmn{重物从高处掉下来的声音}\end{définition}\end{sous-entrée}

\begin{sous-entrée}{bɤβnɤbɤβ}{ⓔbɤbɤβⓢ2ⓝbɤβnɤbɤβ} 
\classe{idph.3} 
\begin{exemple}\pjya{tɤjpa bɤβnɤbɤβ ʑo pa-βde}\hspace{5pt}\pcmn{他把雪(从房背)一块一块扔下来了}\end{exemple}
\begin{exemple}\pjya{bɤβnɤbɤβ ʑo ɲo-nɯjʁo}\hspace{5pt}\pcmn{他很没有分寸地骂了他}\end{exemple}\end{sous-entrée}

\begin{sous-entrée}{nɯbɤβ}{ⓔbɤbɤβⓢ2ⓝnɯbɤβ} 
\classe{vt}  
\grammaire{deidph} \end{sous-entrée}

\begin{sous-entrée}{ɣɤbɤbɤβ}{ⓔbɤbɤβⓢ2ⓝɣɤbɤbɤβ} 
\classe{vs} 
\begin{définition}\pfra{être bruyant}\end{définition}
\begin{définition}\pcmn{发出很响的声音}\end{définition}
\begin{exemple}\pjya{qale ɯ-tɯ-wxti kɯ ɲɯ-ɣɤbɤbɤβ ʑo}\hspace{5pt}\pcmn{风发出很响的声音}\end{exemple}
\begin{exemple}\pjya{tɯrme ʁnɯz ɲɯ-rɯɕmi-ndʑi tɕe, ɲɯ-ɣɤbɤbɤβ-ndʑi}\hspace{5pt}\pcmn{两个人在说话,很吵,听不清楚他们在讲什么}\end{exemple}\end{sous-entrée}

\end{entrée}

\begin{entrée}{bjɯbjɯɣ}{}{ⓔbjɯbjɯɣ} 
\classe{idph.2} 
\begin{définition}\pfra{qui pend en grand nombre, mou}\end{définition}
\begin{définition}\pcmn{形容多而柔软,向下垂吊的样子}\end{définition}
\begin{exemple}\pjya{jiʑo ji-kha ɯ-ʁɤri ʑmbri tɯ-phɯ tu tɕe, ftɕar tɕe ɯ-jwaʁ ɲɯ-dɤn, ɯ-rtaʁ ɲɯ-mpɯ ɲɯ-dɤn tɕe, bjɯbjɯɣ ʑo pjɯ-ɴqoʁ ɲɯ-ŋu}\hspace{5pt}\pcmn{他们家前面有一棵柳树,一到春天,树叶茂密,树枝又软又多地吊在那儿。}\end{exemple}\relationsémantique{同义词}{\lien{ⓔlbjɯlbjɯɣ}{lbjɯlbjɯɣ}}\end{entrée}

\begin{entrée}{boŋboŋ}{}{ⓔboŋboŋ} 
\classe{idph.2} 
\begin{définition}\pfra{qui a la forme d'un œuf}\end{définition}
\begin{définition}\pcmn{形容鸡蛋的形状;椭圆形}\end{définition}\end{entrée}

\begin{entrée}{boʁ}{}{ⓔboʁ} 
\classe{idph.1} 
\begin{définition}\pfra{d'un seul coup tous ensemble}\end{définition}
\begin{définition}\pcmn{一下子全部}\end{définition}
\begin{exemple}\pjya{smi ɯ-taʁ tɯ-ci pjɯ́-wɣ-lɤt tɕe boʁ pjɯ-mi ɕti}\hspace{5pt}\pcmn{火上加了水就会一下子灭掉}\end{exemple}
\begin{sous-entrée}{boʁboʁ}{ⓔboʁⓝboʁboʁ} 
\classe{idph.2} 
\begin{définition}\pfra{en ordre}\end{définition}
\begin{définition}\pcmn{形容(收捡得)很整齐的样子}\end{définition}
\begin{exemple}\pjya{nɤ-ŋga boʁboʁ tɤ-ste}\hspace{5pt}\pcmn{你把衣服收捡得整齐一点}\end{exemple}\end{sous-entrée}

\begin{sous-entrée}{boʁnɤboʁ}{ⓔboʁⓝboʁnɤboʁ} 
\classe{idph.3} 
\begin{exemple}\pjya{tɤ-pɤtso ra boʁboʁ nɤ boʁboʁ jɤ-ari-nɯ}\hspace{5pt}\pcmn{小孩子一下子一起去了}\end{exemple}\relationsémantique{参考}{\lien{ⓔnɤboʁboʁ}{nɤboʁboʁ}}\relationsémantique{参考}{\lien{ⓔaboʁboʁ}{aboʁboʁ}}\end{sous-entrée}

\end{entrée}

\begin{entrée}{buqa}{}{ⓔbuqa} 
\classe{n} 
\begin{définition}\pfra{mycose du pied}\end{définition}
\begin{définition}\pcmn{脚癣}\end{définition}\end{entrée}

\begin{entrée}{brɤβbrɤβ/\variante{brɤbrɤβ}}{}{ⓔbrɤβbrɤβ} 
\classe{idph.2} 
\begin{définition}\pfra{rugueux et irrégulier (pierre)}\end{définition}
\begin{définition}\pcmn{形容石子等物粗糙的样子}\end{définition}
\begin{exemple}\pjya{rdɤstaʁ ɲɯ-dɤn, kɤntɕhaʁ brɤbrɤβ ʑo ɲɯ-pa}\hspace{5pt}\pcmn{街上小石子多,坎坷不平}\end{exemple}
\begin{exemple}\pjya{stoʁ ɯ-tɯ-jndʐɤz kɯ brɤbrɤβ ʑo ɲɯ-pa}\hspace{5pt}\pcmn{胡豆的颗粒又大又多}\end{exemple}
\begin{sous-entrée}{brɤβnɤbrɤβ}{ⓔbrɤβbrɤβⓝbrɤβnɤbrɤβ} 
\classe{idph.3} 
\begin{exemple}\pjya{ʁmaʁmi ra brɤβnɤbrɤβ ʑo ɲɯ-nɤŋkɯŋke-nɯ}\hspace{5pt}\pcmn{士兵们身材高大,雄赳赳地路过了这里}\end{exemple}\end{sous-entrée}

\begin{sous-entrée}{brɤβnɤlɤβ}{ⓔbrɤβbrɤβⓝbrɤβnɤlɤβ} 
\classe{idph.4} 
\begin{exemple}\pjya{brɤβnɤlɤβ ʑo ɲɯ-ʑɣɤstu-nɯ}\hspace{5pt}\pcmn{动作和声音都很大,没有规律地运动着}\end{exemple}\relationsémantique{参考}{\lien{ⓔbrɯzbrɯz}{brɯzbrɯz}}\relationsémantique{参考}{\lien{ⓔbrɯɣbrɯɣ}{brɯɣbrɯɣ}}\end{sous-entrée}

\end{entrée}

\begin{entrée}{brɯɣbrɯɣ}{}{ⓔbrɯɣbrɯɣ} 
\classe{idph.2} 
\begin{définition}\pfra{couvert de petits boutons}\end{définition}
\begin{définition}\pcmn{粗糙,长满了小点点}\end{définition}\end{entrée}

\begin{entrée}{brɯzbrɯz}{}{ⓔbrɯzbrɯz} 
\classe{idph.2} 
\begin{définition}\pfra{couvert de petits boutons}\end{définition}
\begin{définition}\pcmn{粗糙,长满了小点点}\end{définition}
\begin{exemple}\pjya{a-βri tɤ-ndɤr brɯzbrɯz ʑo ɲɤ-ɬoʁ}\hspace{5pt}\pcmn{我身上长满了痘痘}\end{exemple}\relationsémantique{参考}{\lien{ⓔbrɯɣbrɯɣ}{brɯɣbrɯɣ}}\relationsémantique{参考}{\lien{ⓔbrɤβbrɤβ}{brɤβbrɤβ}}\end{entrée}

\begin{entrée}{bɯɣ}{}{ⓔbɯɣ} 
\classe{vi} \paradigme{dir}{tɤ-}\paradigme{dir}{pɯ-}
\begin{définition}\pfra{avoir le mal du pays}\end{définition}
\begin{définition}\pcmn{思念家乡}\end{définition}
\begin{exemple}\pjya{ɲɯ-bɯɣ-a}\hspace{5pt}\pcmn{我想家}\end{exemple}
\begin{exemple}\pjya{pjɤ-bɯɣ}\hspace{5pt}\pcmn{他以前很想家}\end{exemple}
\begin{exemple}\pjya{tɯrme sɤtɕha jɤ-kɯ-ɤri tɕe, tu-kɯ-bɯɣ ɕti}\hspace{5pt}\pcmn{去了其他地方,肯定会想家}\end{exemple}\relationsémantique{参考}{\lien{ⓔnɯɣbɯɣ}{nɯɣbɯɣ}}\end{entrée}

\begin{entrée}{bɯɣbɯɣ}{}{ⓔbɯɣbɯɣ} 
\classe{idph.2} 
\begin{définition}\pfra{concentré}\end{définition}
\begin{définition}\pcmn{形容草木等集中,茂盛的样子}\end{définition}
\begin{exemple}\pjya{tɯrme kɯ-nɤmɲo jo-dɤn tɕe bɯɣbɯɣ ʑo ɲɯ-pa}\end{exemple}
\begin{exemple}\pjya{bɯɣbɯɣ ʑo jo-ɣi-nɯ tɕe ɲɯ-nɤmɲo-nɯ}\hspace{5pt}\pcmn{人从四面八方赶来观看}\end{exemple}
\begin{sous-entrée}{bɯɣnɤbɯɣ}{ⓔbɯɣbɯɣⓝbɯɣnɤbɯɣ} 
\classe{idph.3} \end{sous-entrée}

\end{entrée}

\begin{entrée}{bɯlɯbali}{}{ⓔbɯlɯbali} 
\classe{n} 
\begin{définition}\pfra{personne qui n'en fait qu'à sa tête}\end{définition}
\begin{définition}\pcmn{爱一意孤行的人}\end{définition}
\begin{exemple}\pjya{ɯʑo na-nɯ-ʁjit nɯ tu-nɯ-ste ɲɯ-ɕti, tɯrme bɯlɯbali ci ɲɯ-ŋu}\hspace{5pt}\pcmn{他想做什么就做什么,不管别人的意见}\end{exemple}\end{entrée}

\begin{entrée}{bɯwa}{}{ⓔbɯwa} 
\classe{vt} \paradigme{dir}{tɤ-}
\begin{définition}\pfra{porter un enfant sur le dos}\end{définition}
\begin{définition}\pcmn{背孩子}\end{définition}
\begin{exemple}\pjya{tɤ-bɯwa-t-a, tɤ-tɯ-bɯwa-t, ta-bɯwa}\hspace{5pt}\pcmn{我背了他,你背了他,他背了他}\end{exemple}
\begin{exemple}\pjya{kɯki kɤ-ŋke mɯ́j-cha tɕɤn, tɤ-bɯwe}\hspace{5pt}\pcmn{他不能走,你背他吧}\end{exemple}
\begin{exemple}\pjya{tɤ-bɯwe ɲɯ-ntshi}\hspace{5pt}\pcmn{只好背了他}\end{exemple}\relationsémantique{参考}{\lien{ⓔzbɯwa}{zbɯwa}}\end{entrée}

\newpage\caractère{β}

\begin{entrée}{βdaʁ,βzu}{}{ⓔβdaʁ,βzu} 
\classe{n}
\classe{vt} 
\begin{définition}\pfra{s'occuper de ses tâches}\end{définition}
\begin{définition}\pcmn{管(任务、职责),一般前加否定前缀}\end{définition}
\begin{exemple}\pjya{βdaʁ maka tɤ-βzu-t-a me}\hspace{5pt}\pcmn{我没有管好、没有理他}\end{exemple}\relationsémantique{Component 1}{\lien{}{βdaʁ}}\relationsémantique{Component 2}{\lien{}{βzu}}\relationsémantique{参考}{\lien{ⓔnɯβdaʁ}{nɯβdaʁ}}\relationsémantique{参考}{\lien{ⓔβzuⓗ1}{βzu₁}}\end{entrée}

\begin{entrée}{βdaʁmu}{}{ⓔβdaʁmu} 
\classe{n} 
\begin{définition}\pfra{maîtresse de la maison}\end{définition}
\begin{définition}\pcmn{女主人}\end{définition}\étymologie{bdag.mo}\end{entrée}

\begin{entrée}{βdaχpu}{₁}{ⓔβdaχpuⓗ1} 
\classe{n} 
\begin{définition}\pfra{hôte, maître de maison}\end{définition}
\begin{définition}\pcmn{主人}\end{définition}\relationsémantique{参考}{\lien{ⓔnɯβdaχpu}{nɯβdaχpu}}\étymologie{bdag.po}\end{entrée}

\begin{entrée}{βdaχpu}{₂}{ⓔβdaχpuⓗ2} 
\classe{adv} 
\begin{définition}\pfra{en ce qui concerne ...}\end{définition}
\begin{définition}\pcmn{至于……}\end{définition}
\begin{exemple}\pjya{ɯʑo βdaχpu nɯ, ɯʑo a-tɤ-naχpjɤt}\hspace{5pt}\pcmn{至于他,(来不来)由他来决定}\end{exemple}\end{entrée}

\begin{entrée}{βdaχtɕɤl}{}{ⓔβdaχtɕɤl} 
\classe{n} 
\begin{définition}\pfra{marche en pierre}\end{définition}
\begin{définition}\pcmn{石制台阶}\end{définition}\end{entrée}

\begin{entrée}{βdɤmŋaʁ}{}{ⓔβdɤmŋaʁ} 
\classe{n} 
\begin{définition}\pfra{méthode (pour tenir tête à qqn)}\end{définition}
\begin{définition}\pcmn{(对付别人的)办法、计策}\end{définition}\étymologie{bdams.mŋags}\end{entrée}

\begin{entrée}{βde}{}{ⓔβde} 
\classe{vt}  
\grammaire{habil} \sens{1}\paradigme{dir}{thɯ-}\paradigme{dir}{\_}
\begin{définition}\pfra{jeter}\end{définition}
\begin{définition}\pcmn{扔}\end{définition}
\begin{exemple}\pjya{cho-βde, tha-βde}\hspace{5pt}\pcmn{他扔了}\end{exemple}
\begin{exemple}\pjya{laχtɕha mɤ-kɯ-ra nɯra cho-βde}\hspace{5pt}\pcmn{他扔了不必要的东西}\end{exemple}
\begin{exemple}\pjya{ki laχtɕha ki kɤ-βde ɯ-spa ɕti}\hspace{5pt}\pcmn{这个东西可以扔}\end{exemple}
\begin{exemple}\pjya{qapri ɯ-χsiu cho-βde}\hspace{5pt}\pcmn{蛇脱皮了}\end{exemple}
\begin{exemple}\pjya{tɯ-ɴɢar jo-βde}\hspace{5pt}\pcmn{他吐了痰}\end{exemple}\sens{2}\paradigme{dir}{nɯ-}\paradigme{dir}{nɯ-}\paradigme{dir}{pɯ-}
\begin{définition}\pfra{abandonner}\end{définition}
\begin{définition}\pcmn{放弃}\end{définition}
\begin{définition}\pfra{perdre}\end{définition}
\begin{définition}\pcmn{遗失;弄丢}\end{définition}
\begin{exemple}\pjya{ɲo-βde}\hspace{5pt}\pcmn{他放弃了}\end{exemple}
\begin{exemple}\pjya{ɯ-ma z-ɲɤ-βde}\hspace{5pt}\pcmn{他辞职去了}\end{exemple}
\begin{exemple}\pjya{nɤ-kɤ-nɤma nɯ mɤ-kɯ-ftɯɣ a-mɤ-nɯ-tɯ-βde}\hspace{5pt}\pcmn{你不要放弃没有完成的工作}\end{exemple}
\begin{exemple}\pjya{qartsɯ ɲɤ-βde tɕe χɕitka ko-ndzoʁ}\hspace{5pt}\pcmn{冬天结束了,春天到来了}\end{exemple}
\begin{exemple}\pjya{laχtɕha ɲɤ-nɯβde-t-a}\hspace{5pt}\pcmn{我把东西弄丢了}\end{exemple}
\begin{exemple}\pjya{ɲɤ-nɯβde}\hspace{5pt}\pcmn{他弄丢了}\end{exemple}
\begin{exemple}\pjya{nɤ-rte ma-nɯ-tɯ-nɯβde}\hspace{5pt}\pcmn{你不要把帽子弄丢}\end{exemple}
\begin{sous-entrée}{nɯβde}{ⓔβdeⓢ2ⓝnɯβde} 
\classe{vt}  
\grammaire{autoben} \end{sous-entrée}

\begin{sous-entrée}{sɯβde}{ⓔβdeⓢ2ⓝsɯβde} 
\classe{vt} \end{sous-entrée}

\paradigme{dir}{nɯ-}\paradigme{dir}{nɯ-}
\begin{définition}\pfra{pouvoir se résigner à abandonner}\end{définition}
\begin{définition}\pcmn{舍得离开}\end{définition}
\begin{définition}\pfra{se quitter, se séparer}\end{définition}
\begin{définition}\pcmn{互相离别,互相分开}\end{définition}
\begin{définition}\pfra{séparer (des gens)}\end{définition}
\begin{définition}\pcmn{使……分开}\end{définition}
\begin{exemple}\pjya{ɯ-pi ra mɯ-pjɤ-sɯβde}\hspace{5pt}\pcmn{她舍不得离开她的姐姐们}\end{exemple}
\begin{exemple}\pjya{ɲɤ-k-ɤmɯβde-ndʑi-ci}\hspace{5pt}\pcmn{他们互相离别了}\end{exemple}
\begin{sous-entrée}{amɯβde}{ⓔβdeⓝamɯβde} 
\classe{vi} \end{sous-entrée}

\begin{sous-entrée}{sɤmɯβde}{ⓔβdeⓝsɤmɯβde} 
\classe{vt} \end{sous-entrée}

\begin{sous-entrée}{asɤmɯβde}{ⓔβdeⓝasɤmɯβde} 
\classe{vi} 
\begin{définition}\pfra{pouvoir se résigner à se quitter}\end{définition}
\begin{définition}\pcmn{舍得离别}\end{définition}
\begin{exemple}\pjya{mɯ-pjɤ-k-ɤsɤmɯβde-ndʑi-ci}\hspace{5pt}\pcmn{他们俩不舍得离别}\end{exemple}\end{sous-entrée}

\begin{sous-entrée}{nɤβdɤle}{ⓔβdeⓝnɤβdɤle} 
\classe{vt} \sens{1}
\begin{définition}\pfra{jeter dans tous les sens}\end{définition}
\begin{définition}\pcmn{甩来甩去,扔来扔去}\end{définition}
\begin{exemple}\pjya{tɤ-pɤtso kɯ ɯ-kɯmtɕhɯ nɯ ɲɯ-ɤz-nɤβdɤle tɕe ɲɯ-ɤnɯɣro}\hspace{5pt}\pcmn{小孩子把玩具甩来甩去}\end{exemple}\end{sous-entrée}

\sens{2}\paradigme{dir}{nɯ-}
\begin{définition}\pfra{négliger}\end{définition}
\begin{définition}\pcmn{忽略,不管}\end{définition}
\begin{exemple}\pjya{tɤ-pɤtso ma-nɯ-tɯ-nɤβdɤle ma mɤ-pe}\hspace{5pt}\pcmn{你不忽略小孩子}\end{exemple}\end{entrée}

\begin{entrée}{βdeti}{}{ⓔβdeti} 
\classe{n} 
\begin{définition}\pfra{fête du septième mois}\end{définition}
\begin{définition}\pcmn{七月份的看花节}\end{définition}\end{entrée}

\begin{entrée}{βdi}{}{ⓔβdi} 
\classe{vs} \paradigme{dir}{tɤ-}\paradigme{dir}{nɯ-}\paradigme{dir}{thɯ-}\paradigme{construction}{bare infinitive}
\begin{définition}\pfra{beau}\end{définition}
\begin{définition}\pcmn{美观}\end{définition}
\begin{exemple}\pjya{ɯ-βzu ɲɯ-βdi}\hspace{5pt}\pcmn{他做得很好}\end{exemple}
\begin{exemple}\pjya{ɯ-sɯm βdi}\hspace{5pt}\pcmn{他放心了}\end{exemple}
\begin{exemple}\pjya{ɯ-ta mɯ-ɲɤ-βdi}\hspace{5pt}\pcmn{放得不平}\end{exemple}\relationsémantique{参考}{\lien{ⓔaβdoʁβdi}{aβdoʁβdi}}
\begin{sous-entrée}{tɯ-skhrɯ mɯ-ɲɤ-βdi}{ⓔβdiⓝtɯ-skhrɯ mɯ-ɲɤ-βdi}
\begin{définition}\pfra{tomber enceinte}\end{définition}
\begin{définition}\pcmn{怀孕}\end{définition}
\begin{exemple}\pjya{ɯ-skhrɯ mɯ-ɲɤ-βdi}\hspace{5pt}\pcmn{她怀孕了}\end{exemple}\end{sous-entrée}

\begin{sous-entrée}{mɤβdi ma}{ⓔβdiⓝmɤβdi ma}
\begin{définition}\pfra{au cas où}\end{définition}
\begin{définition}\pcmn{万一}\end{définition}
\begin{exemple}\pjya{tɤ-pɤtso pjɯ-ndʐaβ mɤβdi ma mɤ-nɯɣɯfɕɤt}\hspace{5pt}\pcmn{万一小孩子摔倒了,就不好交代}\end{exemple}
\begin{exemple}\pjya{ɯʑo tu-ngo mɤβdi ma nɤj nɤ-taʁ ŋu}\hspace{5pt}\pcmn{万一他生病了的话,就是你的责任了}\end{exemple}\end{sous-entrée}

\étymologie{bde}\end{entrée}

\begin{entrée}{βdiwa}{}{ⓔβdiwa} 
\classe{n} \sens{1}
\begin{définition}\pfra{calme}\end{définition}
\begin{définition}\pcmn{平安}\end{définition}\sens{2}
\begin{définition}\pfra{bonne action}\end{définition}
\begin{définition}\pcmn{善事}\end{définition}
\begin{exemple}\pjya{βdiwa ɲɤ-fkot}\hspace{5pt}\pcmn{他做了善事}\end{exemple}\étymologie{bde.ba}\end{entrée}

\begin{entrée}{βdɯnba}{}{ⓔβdɯnba} 
\classe{n} 
\begin{définition}\pfra{septième mois}\end{définition}
\begin{définition}\pcmn{七月}\end{définition}\étymologie{bdun.pa}\end{entrée}

\begin{entrée}{βdɯscit}{}{ⓔβdɯscit} 
\classe{n} 
\begin{définition}\pfra{bonheur}\end{définition}
\begin{définition}\pcmn{幸福}\end{définition}
\begin{exemple}\pjya{ɯ-βdɯscit ɲɤ-rɤru (=kɤ-scit to-khɯ)}\hspace{5pt}\pcmn{他开始过幸福的生活了}\end{exemple}\étymologie{bde.skʲit}\end{entrée}

\begin{entrée}{βdɯt}{₁}{ⓔβdɯtⓗ1} 
\classe{np} \sens{1}
\begin{définition}\pfra{gaspillage}\end{définition}
\begin{définition}\pcmn{浪费}\end{définition}
\begin{exemple}\pjya{khɯna kɤ-χsu ʁo tɯ-χsu βdɯt ɕti}\hspace{5pt}\pcmn{喂狗简直是白喂}\end{exemple}
\begin{exemple}\pjya{ɯ-βdɯt ma-pɯ-tɯ-sɯβze}\hspace{5pt}\pcmn{不要浪费}\end{exemple}\sens{2}
\begin{définition}\pfra{dépense}\end{définition}
\begin{définition}\pcmn{花费}\end{définition}
\begin{exemple}\pjya{nɤ-βdɯt nɯ-tɕat-a}\hspace{5pt}\pcmn{让你花费了很多(谢你请我吃饭)}\end{exemple}
\begin{exemple}\pjya{nɤ-βdɯt tu ma jɤ-ɣe-j ɕti, dɤn-i ko!}\hspace{5pt}\pcmn{你花费要多,我们来了很多人}\end{exemple}\relationsémantique{参考}{\lien{ⓔkɤnɯβdɯt}{kɤnɯβdɯt}}\end{entrée}

\begin{entrée}{βdɯt}{₂}{ⓔβdɯtⓗ2} 
\classe{n} 
\begin{définition}\pfra{une sorte de monstre}\end{définition}
\begin{définition}\pcmn{魔鬼}\end{définition}\étymologie{bdud}\end{entrée}

\begin{entrée}{βgoz/\variante{fkoz}}{}{ⓔβgoz} 
\classe{vt} \paradigme{dir}{tɤ-}\paradigme{dir}{pɯ-}
\begin{définition}\pfra{préparer, planifier}\end{définition}
\begin{définition}\pcmn{计划;设计;准备需要的材料}\end{définition}
\begin{exemple}\pjya{kɯki kha ci tɤ-βgoz-a}\hspace{5pt}\pcmn{我设计了这个房子}\end{exemple}
\begin{exemple}\pjya{tɯ-ŋga ɯ-spa ci tɤ-βgoz-a}\hspace{5pt}\pcmn{我准备了缝衣服的材料}\end{exemple}
\begin{exemple}\pjya{tɯ-xtsa kɤ-βzu to-βgoz}\hspace{5pt}\pcmn{他设计了鞋子的款式}\end{exemple}
\begin{exemple}\pjya{kɯki tɤ-pɤtso ki kɯ-rɤβzjoz tɤ-βgoz-i}\hspace{5pt}\pcmn{我们准备让这个孩子去读书}\end{exemple}
\begin{sous-entrée}{βgoz}{ⓔβgozⓝβgoz} 
\classe{vi} 
\begin{définition}\pfra{arranger (marriage)}\end{définition}
\begin{définition}\pcmn{包办(婚姻)}\end{définition}
\begin{exemple}\pjya{phama pɯ-βgoz pɯ-ŋu ma ʑɤni nɯ-anɯtɯɣ-ndʑi pɯ-maʁ}\hspace{5pt}\pcmn{他们的婚姻是父母包办的,不是他们自己主张的}\end{exemple}\end{sous-entrée}

\étymologie{bgod}\end{entrée}

\begin{entrée}{βɣu}{}{ⓔβɣu} 
\classe{n} 
\begin{définition}\pfra{louche en bois}\end{définition}
\begin{définition}\pcmn{木头刻成的小瓢}\end{définition}\end{entrée}

\begin{entrée}{βɣa}{}{ⓔβɣa} 
\classe{n} 
\begin{définition}\pfra{moulin à eau}\end{définition}
\begin{définition}\pcmn{水磨}\end{définition}
\begin{exemple}\pjya{βɣa ɲɯ-pe tɕe khro tha-ɣndʑɯr}\hspace{5pt}\pcmn{磨子很好用(水量多),磨了很多粮食}\end{exemple}\end{entrée}

\begin{entrée}{βɣɤɣur}{}{ⓔβɣɤɣur} 
\classe{n} 
\begin{définition}\pfra{farine qui sort de la meule}\end{définition}
\begin{définition}\pcmn{从磨撒出来的面粉}\end{définition}\relationsémantique{参考}{\lien{ⓔβɣa}{βɣa}}\relationsémantique{参考}{\lien{ⓔtɤ-ɣur}{tɤ-ɣur}}\end{entrée}

\begin{entrée}{βɣɤmu}{}{ⓔβɣɤmu} 
\classe{n} 
\begin{définition}\pfra{entonnoir en peau pour verser l'orge dans la meule}\end{définition}
\begin{définition}\pcmn{用来把青稞灌到磨口中的三角形牛皮,中间有洞}\end{définition}
\begin{exemple}\pjya{βɣɤmu nɯ kɤ-ɣndʑɯr ɯ-spa tɯjpu nɯ ɯ-sɤ-rku tɯ-ndʐi thɯ-kɤ-rɤɣdɯt tɕe βɣa ɯ-ŋgɯ ɯ-pjɯ-kɯ-lɤt ɯ-spa ŋu}\hspace{5pt}\pcmn{\lien{ⓔβɣɤmu}{βɣɤmu}是用来装待磨的粮食的(保持整头牛原形的)牛皮,也是用来把粮食倒入磨坊的工具。}\end{exemple}\end{entrée}

\begin{entrée}{βɣɤno}{}{ⓔβɣɤno} 
\classe{n} 
\begin{définition}\pfra{partie inférieure de la meule}\end{définition}
\begin{définition}\pcmn{下磨盘}\end{définition}
\begin{exemple}\pjya{βɣɤno nɯ rdɤstaʁ kɯ-ɤrtɯm nɯ kɤ-βzu ŋu tɕe pjɯ-maŋpa tɕe ku-rɤsta ŋu}\hspace{5pt}\pcmn{\lien{ⓔβɣɤno}{βɣɤno}是由一块圆形石块凿成的,处于磨子的下面,是固定的。}\end{exemple}\end{entrée}

\begin{entrée}{βɣɤŋgɯ}{}{ⓔβɣɤŋgɯ} 
\classe{n} 
\begin{définition}\pfra{partie supérieure du moulin}\end{définition}
\begin{définition}\pcmn{磨坊的上部}\end{définition}\relationsémantique{参考}{\lien{ⓔβɣa}{βɣa}}\relationsémantique{参考}{\lien{ⓔɯ-ŋgɯ}{ɯ-ŋgɯ}}\end{entrée}

\begin{entrée}{βɣɤpa}{}{ⓔβɣɤpa} 
\classe{n} 
\begin{définition}\pfra{partie inférieure du moulin (dans l'eau)}\end{définition}
\begin{définition}\pcmn{磨坊的下部(河水流动的地方)}\end{définition}\relationsémantique{参考}{\lien{ⓔβɣa}{βɣa}}\relationsémantique{参考}{\lien{ⓔpaⓗ3ⓝɯ-pa}{ɯ-pa}}\end{entrée}

\begin{entrée}{βɣɤru}{}{ⓔβɣɤru} 
\classe{n} 
\begin{définition}\pfra{meunier}\end{définition}
\begin{définition}\pcmn{守磨坊的佣人}\end{définition}\relationsémantique{参考}{\lien{ⓔrɯru}{rɯru}}\relationsémantique{参考}{\lien{ⓔβɣa}{βɣa}}\end{entrée}

\begin{entrée}{βɣɤrkhɤrkhɤt}{}{ⓔβɣɤrkhɤrkhɤt} 
\classe{n} 
\begin{définition}\pfra{bâton qui sert à frapper l'entonnoir de peau}\end{définition}
\begin{définition}\pcmn{用来震动牛皮漏斗的木棒}\end{définition}
\begin{exemple}\pjya{βɣɤrkhɤrkhɤt nɯ laʁjɯɣ ci ŋu tɕe tɯ-pɕoʁ chiz nɯ βɣɤmu ɯ-taʁ kú-wɣ-βraʁ tɕe mɤpɕoʁ chiz nɯ βɣɤtu ɯ-taʁ pjɯ-tɯɣ tɕe βɣa kɤ-mtɕɯr tɕe βɣɤrkhɤrkhɤt tu-znɤndɤr tɕe βɣɤmu ɯ-ŋgɯ kɤ-ɣndʑɯr ɯ-spa nɯ pjɯ-nɯɬoʁ ɲɯ-ŋu tɕe βɣa ɯ-ŋgɯ pjɯ-ɕe ŋu}\hspace{5pt}\pcmn{\lien{ⓔβɣɤrkhɤrkhɤt}{βɣɤrkhɤrkhɤt}是一根木棒,一头拴在牛皮漏斗上,另一头靠着石磨。石磨转动时,就使木棒震动,使漏斗里的粮食漏进磨子里去}\end{exemple}\relationsémantique{参考}{\lien{ⓔβɣa}{βɣa}}\relationsémantique{参考}{\lien{ⓔrkhɤrkhɤt}{rkhɤrkhɤt}}\end{entrée}

\begin{entrée}{βɣɤrnɤjwaʁ}{}{ⓔβɣɤrnɤjwaʁ} 
\classe{n} 
\begin{définition}\pfra{pales du moulin}\end{définition}
\begin{définition}\pcmn{磨坊下面的木板}\end{définition}
\begin{exemple}\pjya{βɣɤrnɤjwaʁ nɯ tɕhɯŋkhɤr ɯ-taʁ tɤrɤm kɤ-kɤ-tshoʁ nɯ ŋu, tɯ-ci βɣɤrnɤjwaʁ ɯ-taʁ chɯ-lɤt tɕe tɕhɯŋkhɤr ku-sɯ-mtɕɯr tɕe tɕhɯŋkhɤr ɯ-taʁ βɣɤsni ndzoʁ tɕe βɣɤtu ku-sɯ-mtɕɯr ŋu}\hspace{5pt}\pcmn{\lien{ⓔβɣɤrnɤjwaʁ}{βɣɤrnɤjwaʁ}是装在磨房下的水车上的木板,水冲在木板上时,木板使水车转动,上面装有磨心,使上磨盘转动起来。}\end{exemple}\relationsémantique{参考}{\lien{ⓔβɣa}{βɣa}}\relationsémantique{参考}{\lien{ⓔtɯ-rna}{tɯ-rna}}\relationsémantique{参考}{\lien{ⓔtɤ-jwaʁ}{tɤ-jwaʁ}}\end{entrée}

\begin{entrée}{βɣɤrqo}{}{ⓔβɣɤrqo} 
\classe{n} 
\begin{définition}\pfra{joint entre le sac de graines et la meule}\end{définition}
\begin{définition}\pcmn{粮食口袋和磨斗之间的接头部分}\end{définition}\relationsémantique{参考}{\lien{ⓔβɣa}{βɣa}}\relationsémantique{参考}{\lien{ⓔtɯ-rqo}{tɯ-rqo}}\end{entrée}

\begin{entrée}{βɣɤrtshi}{}{ⓔβɣɤrtshi} 
\classe{n} 
\begin{définition}\pfra{moustique}\end{définition}
\begin{définition}\pcmn{蚊子}\end{définition}\end{entrée}

\begin{entrée}{βɣɤsɤprɤt}{}{ⓔβɣɤsɤprɤt} 
\classe{n} 
\begin{définition}\pfra{écluses du moulin}\end{définition}
\begin{définition}\pcmn{开关磨坊水道用的闸门}\end{définition}
\begin{exemple}\pjya{βɣɤsɤprɤt nɯ kɤ-ɣndʑɯr pɯ-jɤɣ tɕe tɯ-ci ɯ-sɤ-prɤt tɤrɤm ci tu tɕe nɯnɯ ŋu}\hspace{5pt}\pcmn{\lien{ⓔβɣɤsɤprɤt}{βɣɤsɤprɤt}是一块木板,磨好面以后,用来把水闸住。}\end{exemple}\relationsémantique{参考}{\lien{ⓔβɣa}{βɣa}}\relationsémantique{参考}{\lien{ⓔprɤt}{prɤt}}\end{entrée}

\begin{entrée}{βɣɤsni}{}{ⓔβɣɤsni} 
\classe{n} 
\begin{définition}\pfra{axe du moulin}\end{définition}
\begin{définition}\pcmn{磨心}\end{définition}
\begin{exemple}\pjya{βɣɤsni nɯ βɣɤno ɯ-ku kɯ-sɯ-mtɕɯr ɕom thɯ-kɤ-βzu ŋu, ɯ-tshɯɣa nɯ thɤtɕɯ cho naχtɕɯɣ ri ɯ-jɯ nɯ ɕom ɕti}\hspace{5pt}\pcmn{磨心用来转动上磨盘。用生铁打成,形状和二锤一样,但是柄也是铁打的。}\end{exemple}\relationsémantique{参考}{\lien{ⓔβɣa}{βɣa}}\relationsémantique{参考}{\lien{ⓔtɯ-sni}{tɯ-sni}}\end{entrée}

\begin{entrée}{βɣɤsroʁ}{}{ⓔβɣɤsroʁ} 
\classe{n} 
\begin{définition}\pfra{levier utiliser pour contrôler la vitesse de la meule}\end{définition}
\begin{définition}\pcmn{用来控制磨面粗细的木棒}\end{définition}
\begin{exemple}\pjya{βɣɤsroʁ nɯ si nɯ-kɤ-βzu ci ŋu tɕe kɤ-ɣndʑɯr thɯ́-wɣ-lɤt tɕe pjɯ-jndʐɤz cho βɣa pjɯ-rɟɯɣ tɤ-ra tɕe kɤ-joʁ spa, kɤ-ɣndʑɯr pjɯ-ndɯβ cho pjɯ-ɣɤdal tɤ-ra tɕe kɤ-phɤβ spa ŋu}\hspace{5pt}\pcmn{\lien{ⓔβɣɤsroʁ}{βɣɤsroʁ}是一根木棒。需要磨得粗、转速快的时候,把它抬高;需要磨得细、转速慢的时候,把它放低。}\end{exemple}\relationsémantique{参考}{\lien{ⓔβɣa}{βɣa}}\relationsémantique{参考}{\lien{ⓔmbɣɤsroʁ}{mbɣɤsroʁ}}\end{entrée}

\begin{entrée}{βɣɤtu}{}{ⓔβɣɤtu} 
\classe{n} 
\begin{définition}\pfra{partie supérieure de la meule}\end{définition}
\begin{définition}\pcmn{上磨盘}\end{définition}
\begin{exemple}\pjya{βɣɤtu nɯ rdɤstaʁ kɯ-ɤrtɯm nɯ kɤ-βzu tɕe βɣɤno ɯ-taʁ pjɯ́-wɣ-ta tɕe ɯ-ŋgɯ βɣɤsni tú-wɣ-sthoʁ tɕe ku-mtɕɯr ŋu}\hspace{5pt}\pcmn{\lien{ⓔβɣɤtu}{βɣɤtu}是由一块圆形石块作成的,处于下磨盘的上面,中间装上磨心。可以转动。}\end{exemple}\end{entrée}

\begin{entrée}{βɣɤtɤtɤɣ}{}{ⓔβɣɤtɤtɤɣ} 
\classe{n} 
\begin{définition}\pfra{bord de la meule}\end{définition}
\begin{définition}\pcmn{磨盘边缘}\end{définition}
\begin{exemple}\pjya{βɣɤtɤtɤɣ nɯ kɯ-ɣndʑɯr pɯ́-wɣ-lɤt tɕe tɯ-ɣndʑɤr ɣɯ ɯ-ɣur spa ŋu, rgɤm pɯ-kɤ-sprɤt kɯ-fse ŋu}\hspace{5pt}\pcmn{\lien{ⓔβɣɤtɤtɤɣ}{βɣɤtɤtɤɣ}围在磨盘周围,挡着磨出来的面粉,安装得像木箱一样。}\end{exemple}\relationsémantique{参考}{\lien{ⓔβɣa}{βɣa}}\relationsémantique{参考}{\lien{ⓔtɤtɤɣ}{tɤtɤɣ}}\end{entrée}

\begin{entrée}{βɣɤχtsiɯ}{}{ⓔβɣɤχtsiɯ} 
\classe{n} 
\begin{définition}\pfra{bouche de la meule}\end{définition}
\begin{définition}\pcmn{磨斗}\end{définition}
\begin{exemple}\pjya{βɣɤχtsiɯ nɯ βɣɤtu ɯ-taʁ ku-ndzoʁ tɕe kɤ-ɣndʑɯr ɯ-spa pjɯ-kɯ-ɣi nɯ ɯ-kɯ-nɤjo ɯ-spa ŋu}\hspace{5pt}\pcmn{\lien{ⓔβɣɤχtsiɯ}{βɣɤχtsiɯ}是装在上磨盘的中间,用来接住漏斗里出来的粮食。}\end{exemple}\end{entrée}

\begin{entrée}{βɣɤza}{}{ⓔβɣɤza} 
\classe{n} 
\begin{définition}\pfra{mouche}\end{définition}
\begin{définition}\pcmn{苍蝇}\end{définition}\end{entrée}

\begin{entrée}{βɣɯt}{}{ⓔβɣɯt} 
\classe{vi} \paradigme{dir}{kɤ-}\paradigme{dir}{kɤ-}\paradigme{dir}{thɯ-}
\begin{définition}\pfra{brûler (poils, plumes)}\end{définition}
\begin{définition}\pcmn{被烧(毛,羽毛)}\end{définition}
\begin{définition}\pfra{brûler (poils, plumes)}\end{définition}
\begin{définition}\pcmn{烧(毛,羽毛)}\end{définition}
\begin{exemple}\pjya{ɯ-kɤrme ko-βɣɯt}\hspace{5pt}\pcmn{他头发被烧到}\end{exemple}
\begin{exemple}\pjya{tɤ-ŋkɯ thɯ-sɯβɣɯt-a}\hspace{5pt}\pcmn{我把猪毛烧掉了}\end{exemple}
\begin{sous-entrée}{sɯβɣɯt}{ⓔβɣɯtⓝsɯβɣɯt} 
\classe{vt} \end{sous-entrée}

\end{entrée}

\begin{entrée}{βɣɯz}{}{ⓔβɣɯz} 
\classe{n} 
\begin{définition}\pfra{blaireau}\end{définition}
\begin{définition}\pcmn{獾【臭猪子】}\end{définition}
\begin{exemple}\pjya{βɣɯz nɯ sɤtɕha kɯ-spoʁ ɯ-ŋgɯ ku-rɤʑi ɲɯ-ŋu, praʁ pa tɕu ku-rɤʑi ɲɯ-ŋu, ftɕar tɕe ma qartsɯ tɕe kɤ-mto me, βɣɯz nɯ ɯ-mtɕhi nɯ ra kɯ-ɤmtɕoʁ, ɯ-ku ɯ-tshɯɣa nɯ ra βʑɯ tsa fse, ɯ-rme nɯ pɣi ri paʁ ɯ-rme ʑo fse tɕe nɤrko. ɯ-xtɤpa ɯ-rme ra aqarŋɯrŋe, qandʐe kɤ-ndza wuma ʑo rga, tɕe ɕɤr tɕe, kha ɯ-rkɯ tɯ-ɣli kɯ-dɤn ɣɯ sɤtɕha ɲɯ-sloʁ tɕe, tɕe tɤ-rɤku ra ɯ-qa chɯ-tɕɤt tɕe chɯ-tʂaβ tɕe pjɯ-sɯ-lni ŋu. ju-phɣo tɕe, kɯ-spoʁ ɯ-ŋgɯ lu-nɯɕe tɕe, tɕe kɤ-sat mɤ-khɯ tɕe, ɯ-kɯm smi pjɯ-βlɯ-nɯ tɕe, lu-sɤkhɯ-nɯ tɕe, tɕe mɯ-tɤ-tɕhaʁ tɕe chɯ-nɯɬoʁ tɕe pjɯ-sat-nɯ ŋgrɤl. βɣɯz nɯ ɯ-di wuma ʑo kɯ-χɕɤβ ci tu, tɕe ɯ-ndʐi nɯ pjɯ́-wɣ-nɤβɟu nɯ maʁ nɤ tɯ-mthɤɣ ɲɯ́-wɣ-rtɤβ tɕe χtɕoŋ kɯ-phɤn ɲɯ-ŋu}\hspace{5pt}\pcmn{獾住在土洞和岩洞里,只在夏天出现,冬天看不到。獾的嘴很尖,头部的形状像老鼠,毛是灰色的,像猪的毛一样,但比较硬。肚子下面的毛是淡黄色。它很喜欢吃蚯蚓,所以晚上它去拱房子旁边的肥地,会把庄稼拱倒,使庄稼枯萎。逃跑的时候就会钻进洞里,没法打死它,就在洞口烧火让烟子熏它,它受不了从洞里跑出来的时候才能把它打死。獾有很浓的臭味,它的皮子要么当作坐垫,要么围在腰部,可以治风湿病。}\end{exemple}\relationsémantique{参考}{\lien{ⓔnɯβɣɯz}{nɯβɣɯz}}\end{entrée}

\begin{entrée}{βɟɤt}{}{ⓔβɟɤt} 
\classe{vi} \paradigme{dir}{pɯ-}
\begin{définition}\pfra{obtenir quelque chose, ne pas partir les mains vides}\end{définition}
\begin{définition}\pcmn{得到(某个大家在抢的东西);中奖}\end{définition}
\begin{exemple}\pjya{pɯ-βɟat-a, pɯ-βɟɤt}\hspace{5pt}\pcmn{我得到了,他得到了}\end{exemple}
\begin{exemple}\pjya{tɤ-mbɣom tɕe a-pɯ-tɯ-βɟɤt}\hspace{5pt}\pcmn{你快点就会得到}\end{exemple}
\begin{exemple}\pjya{kɤ-βɟɤt khɯ}\hspace{5pt}\pcmn{可以得到}\end{exemple}
\begin{exemple}\pjya{laχtɕha ɲɯ-rkɯn ri, aʑo pɯ-βɟat-a}\hspace{5pt}\pcmn{东西虽然很少,但是我还是得到了}\end{exemple}
\begin{exemple}\pjya{laχtɕha kɤ-χtɯ pjɤ-βɟɤt}\hspace{5pt}\pcmn{买到了东西}\end{exemple}
\begin{sous-entrée}{sɯβɟɤt}{ⓔβɟɤtⓝsɯβɟɤt} 
\classe{vt} 
\begin{définition}\pfra{faire obtenir}\end{définition}
\begin{définition}\pcmn{令……得到}\end{définition}
\begin{exemple}\pjya{tɤ-ndze ma tha mɤ-ta-sɯβɟɤt}\hspace{5pt}\pcmn{你吃,不然我会吃完,令你吃不到}\end{exemple}\end{sous-entrée}

\end{entrée}

\begin{entrée}{βɟi}{₁}{ⓔβɟiⓗ1} 
\classe{vt} \paradigme{dir}{\_}\paradigme{dir}{\_}
\begin{définition}\pfra{suivre, poursuivre}\end{définition}
\begin{définition}\pcmn{追赶}\end{définition}
\begin{définition}\pfra{faire poursuivre}\end{définition}
\begin{définition}\pcmn{令……追赶}\end{définition}
\begin{exemple}\pjya{tɤ-βɟi-t-a}\hspace{5pt}\pcmn{我追了他}\end{exemple}
\begin{exemple}\pjya{ja-βɟi}\end{exemple}
\begin{exemple}\pjya{tha-βɟi}\hspace{5pt}\pcmn{他追了他}\end{exemple}
\begin{exemple}\pjya{ɯʑo jɤ-anɯri tɕe, z-ja-βɟi-t-a}\hspace{5pt}\pcmn{他回去了,我就追了他}\end{exemple}\relationsémantique{参考}{\lien{ⓔnɤβɟɯβɟi}{nɤβɟɯβɟi}}
\begin{sous-entrée}{sɯβɟi}{ⓔβɟiⓗ1ⓝsɯβɟi} 
\classe{vt} \end{sous-entrée}

\end{entrée}

\begin{entrée}{βɟi}{₂}{ⓔβɟiⓗ2} 
\classe{vs} \paradigme{dir}{nɯ-}
\begin{définition}\pfra{ancien}\end{définition}
\begin{définition}\pcmn{陈旧}\end{définition}
\begin{exemple}\pjya{tɤ-mthɯm ɲɤ-βɟi}\hspace{5pt}\pcmn{肉放陈了}\end{exemple}
\begin{exemple}\pjya{laχtɕha ɲɤ-βɟi}\hspace{5pt}\pcmn{东西旧了}\end{exemple}\end{entrée}

\begin{entrée}{βlama}{}{ⓔβlama} 
\classe{n} 
\begin{définition}\pfra{lama}\end{définition}
\begin{définition}\pcmn{喇嘛}\end{définition}\étymologie{bla.ma}\end{entrée}

\begin{entrée}{βlɤmɤjmɤɣ}{}{ⓔβlɤmɤjmɤɣ} 
\classe{n} 
\begin{définition}\pfra{une espèce de champignon}\end{définition}
\begin{définition}\pcmn{鹅蛋菌}\end{définition}
\begin{exemple}\pjya{βlɤmɤjmɤɣ nɯ ɕkrɤz ɯ-ŋgɯ tu-ɬoʁ ŋu, ɯ-mgɯrqhu nɯ ʁmɤrsɤr ŋu, ɯ-rʑɯɣ cho ɯ-ru nɯ kɯ-qarŋe ŋu, tɤ-ɬoʁ ɕɯmɯma tɕe kɯ-wɣrum ci kɯ aluj tɕe tɤ-ŋgɯm ʑo fse, tɕe tɤ-wxti tɕe ɯ-luj pjɤ-ɴɢaʁ ŋu.}\hspace{5pt}\pcmn{鹅蛋菌长在青冈树林里,背面是金黄色,菌褶和主干是黄色的。刚长出来的时候,有一种白色的外层遮盖,像鸡蛋一样,长大了以后外层就会破。}\end{exemple}\end{entrée}

\begin{entrée}{βlɤmtɕhɤt}{}{ⓔβlɤmtɕhɤt} 
\classe{n} 
\begin{définition}\pfra{récitation de soutras}\end{définition}
\begin{définition}\pcmn{(为别人)念经}\end{définition}\relationsémantique{参考}{\lien{ⓔnɯβlɤmtɕhɤt}{nɯβlɤmtɕhɤt}}\end{entrée}

\begin{entrée}{βluβra}{}{ⓔβluβra} 
\classe{n} 
\begin{définition}\pfra{suggestion, idée, conseil}\end{définition}
\begin{définition}\pcmn{主意;指挥}\end{définition}
\begin{exemple}\pjya{a-βluβra ci tɤ-tɕɤt}\hspace{5pt}\pcmn{给我出主意吧}\end{exemple}\relationsémantique{同义词}{\lien{}{ɯ-ftɕɤfkɤt}}\relationsémantique{同义词}{\lien{ⓔɯ-βlaβlu}{ɯ-βlaβlu}}\relationsémantique{参考}{\lien{ⓔrɯβluβra}{rɯβluβra}}\end{entrée}

\begin{entrée}{βli}{}{ⓔβli} 
\classe{vt} \paradigme{dir}{pɯ-}\paradigme{dir}{lɤ-}
\begin{définition}\pfra{planter}\end{définition}
\begin{définition}\pcmn{栽}\end{définition}
\begin{exemple}\pjya{pɯ-βli-t-a, pa-βli}\hspace{5pt}\pcmn{我栽种了,他栽种了}\end{exemple}
\begin{exemple}\pjya{ɕaŋβli pɯ-βli-t-a}\hspace{5pt}\pcmn{我栽了树苗}\end{exemple}\end{entrée}

\begin{entrée}{βlunbu}{}{ⓔβlunbu} 
\classe{n} 
\begin{définition}\pfra{ministre}\end{définition}
\begin{définition}\pcmn{大臣}\end{définition}\étymologie{blon.po}\end{entrée}

\begin{entrée}{βlɯ}{}{ⓔβlɯ} 
\classe{vl} \sens{1}\paradigme{dir}{tɤ-}\paradigme{dir}{thɯ-}
\begin{définition}\pfra{allumer un feu}\end{définition}
\begin{définition}\pcmn{烧火}\end{définition}
\begin{exemple}\pjya{smi thɯ-βlɯ-t-a, smi tha-βlɯ}\hspace{5pt}\pcmn{我烧了火,他烧了火}\end{exemple}
\begin{exemple}\pjya{aʑo tɤ-βlɯ-a tɕe ndzɤtshi βze-a ŋu}\hspace{5pt}\pcmn{我烧了火就做饭}\end{exemple}\sens{2}\paradigme{dir}{lɤ-}
\begin{définition}\pfra{brûler (du bois)}\end{définition}
\begin{définition}\pcmn{烧(木柴)}\end{définition}
\begin{exemple}\pjya{si lɤ-βlɯ-t-a}\hspace{5pt}\pcmn{我烧了柴火}\end{exemple}\relationsémantique{参考}{\lien{ⓔnɯrmɤβlɯ}{nɯrmɤβlɯ}}\end{entrée}

\begin{entrée}{βlɯɣnɤβlɯɣ}{}{ⓔβlɯɣnɤβlɯɣ} 
\classe{idph.3} 
\begin{définition}\pfra{source de lumière irisée qui scintille}\end{définition}
\begin{définition}\pcmn{形容彩色的光闪光的样子}\end{définition}
\begin{exemple}\pjya{@jingdeng βlɯɣnɤβlɯɣ ɲɯ-ɤsɯ-stu}\hspace{5pt}\pcmn{警灯在闪光}\end{exemple}\end{entrée}

\begin{entrée}{βraʁ}{}{ⓔβraʁ} 
\classe{vt} \paradigme{dir}{kɤ-}\paradigme{dir}{tɤ-}\paradigme{dir}{pɯ-}
\begin{définition}\pfra{attacher}\end{définition}
\begin{définition}\pcmn{拴}\end{définition}
\begin{définition}\pfra{porter (collier)}\end{définition}
\begin{définition}\pcmn{戴(项链)}\end{définition}
\begin{exemple}\pjya{kɤ-βraʁ-a, kɤ-tɯ-βraʁ, ka-βraʁ}\hspace{5pt}\pcmn{我拴了,你拴了,他拴了}\end{exemple}
\begin{exemple}\pjya{khɯna ɲɤ-lɯɣ tɕe kɤ-βraʁ-a}\hspace{5pt}\pcmn{狗摆脱了绳子,我又把它拴起来了}\end{exemple}
\begin{exemple}\pjya{tɯmbri kɤ-βraʁ-a}\hspace{5pt}\pcmn{我拴了绳子(在某个地方打了个结)}\end{exemple}\relationsémantique{参考}{\lien{ⓔzbraʁ}{zbraʁ}}\relationsémantique{参考}{\lien{ⓔɯ-βraʁ}{ɯ-βraʁ}}
\begin{sous-entrée}{aβraʁ}{ⓔβraʁⓝaβraʁ} 
\classe{vi} 
\begin{définition}\pfra{attaché}\end{définition}
\begin{définition}\pcmn{拴着}\end{définition}
\begin{exemple}\pjya{nɯtɕu khɯna pjɤ-k-ɤβraʁ-ci}\hspace{5pt}\pcmn{那里以前拴过狗}\end{exemple}
\begin{exemple}\pjya{ɯ-sɯmpa nɯtɕu pjɤ-k-ɤβraʁ-ci}\hspace{5pt}\pcmn{他心里一直牵挂着}\end{exemple}\end{sous-entrée}

\begin{sous-entrée}{nɤβrɯβraʁ}{ⓔβraʁⓝnɤβrɯβraʁ} 
\classe{vt} 
\begin{définition}\pfra{attacher dans tous les sens}\end{définition}
\begin{définition}\pcmn{拴来拴去}\end{définition}\end{sous-entrée}

\begin{sous-entrée}{nɯβraʁ}{ⓔβraʁⓝnɯβraʁ} 
\classe{vt} \end{sous-entrée}

\end{entrée}

\begin{entrée}{βri}{}{ⓔβri} 
\classe{vt} \paradigme{dir}{kɤ-}\paradigme{dir}{tɤ-}\paradigme{dir}{tɤ-}
\begin{définition}\pfra{défendre}\end{définition}
\begin{définition}\pcmn{掩护;维护}\end{définition}
\begin{définition}\pfra{se défendre}\end{définition}
\begin{définition}\pcmn{保护自己}\end{définition}
\begin{exemple}\pjya{kɤ-βrit-a, tɤ-βrit-a, ka-βri}\hspace{5pt}\pcmn{我维护了,他维护了}\end{exemple}
\begin{exemple}\pjya{ɲɯ-ɤlɯlɤt-ndʑi tɕe, tɤ-βri-t-a}\hspace{5pt}\pcmn{他们打架,我维护了他}\end{exemple}
\begin{exemple}\pjya{khɯna kɯ ɣɯ-mtsɯɣ ɲɯ-ŋu tɕe, kɤ-βri-t-a}\hspace{5pt}\pcmn{狗快要咬他的时候,我保护了他}\end{exemple}
\begin{sous-entrée}{nɯʑɣɤβri}{ⓔβriⓝnɯʑɣɤβri} 
\classe{vi} \end{sous-entrée}

\begin{sous-entrée}{sɤβri}{ⓔβriⓝsɤβri} 
\classe{vi} 
\begin{définition}\pfra{protéger des gens}\end{définition}
\begin{définition}\pcmn{维护别人}\end{définition}\end{sous-entrée}

\end{entrée}

\begin{entrée}{βrɟaŋ}{}{ⓔβrɟaŋ} 
\classe{vt} \paradigme{dir}{nɯ-}
\begin{définition}\pfra{tendre (peau)}\end{définition}
\begin{définition}\pcmn{绷紧}\end{définition}
\begin{exemple}\pjya{aʑo tɯ-ndʐi nɯ-βrɟaŋ-a}\hspace{5pt}\pcmn{我把皮子绷紧了}\end{exemple}\étymologie{brgʲaŋ}\end{entrée}

\begin{entrée}{βʁa}{}{ⓔβʁa} 
\classe{vi} \paradigme{dir}{pɯ-}\paradigme{dir}{pɯ-}\paradigme{dir}{pɯ-}
\begin{définition}\pfra{gagner}\end{définition}
\begin{définition}\pcmn{赢}\end{définition}
\begin{définition}\pfra{faire gagner}\end{définition}
\begin{définition}\pcmn{使……赢}\end{définition}
\begin{définition}\pfra{faire en sorte de gagner}\end{définition}
\begin{définition}\pcmn{使自己赢}\end{définition}
\begin{exemple}\pjya{tɤ-alɯlɤt-ndʑi tɕe, ɯʑo pɯ-βʁa}\hspace{5pt}\pcmn{他们打架了,他赢了}\end{exemple}
\begin{exemple}\pjya{nɤʑo kɤ-ɤnɯɣro tɕe pɯ-tɯ-βʁa}\hspace{5pt}\pcmn{你赢了游戏}\end{exemple}
\begin{exemple}\pjya{pɯ́-wɣ-sɯβʁa}\hspace{5pt}\pcmn{他令我赢了}\end{exemple}\relationsémantique{参考}{\lien{ⓔkɯβʁa}{kɯβʁa}}\relationsémantique{参考}{\lien{ⓔtɤβʁa}{tɤβʁa}}
\begin{sous-entrée}{sɯβʁa}{ⓔβʁaⓝsɯβʁa} 
\classe{vt} \end{sous-entrée}

\begin{sous-entrée}{ʑɣɤsɯβʁa}{ⓔβʁaⓝʑɣɤsɯβʁa} 
\classe{vi} \end{sous-entrée}

\begin{sous-entrée}{znɤβʁaβʁa}{ⓔβʁaⓝznɤβʁaβʁa} 
\classe{vi} 
\begin{définition}\pfra{arrogant}\end{définition}
\begin{définition}\pcmn{对人粗暴,霸道}\end{définition}
\begin{exemple}\pjya{ma-tɯ-znɤβʁaβʁa}\hspace{5pt}\pcmn{你不要对人你们粗暴}\end{exemple}\end{sous-entrée}

\end{entrée}

\begin{entrée}{βʁuβʁu}{}{ⓔβʁuβʁu} 
\classe{idph.2} \sens{1}
\begin{définition}\pfra{comme une demi-sphère creuse}\end{définition}
\begin{définition}\pcmn{形容空心的半球形(老鼠的耳朵)}\end{définition}\sens{2}
\begin{définition}\pfra{tout petit}\end{définition}
\begin{définition}\pcmn{很小的样子}\end{définition}\end{entrée}

\begin{entrée}{βʁum}{}{ⓔβʁum} 
\classe{vt} \paradigme{dir}{pɯ-}\paradigme{dir}{\_}
\begin{définition}\pfra{renverser}\end{définition}
\begin{définition}\pcmn{口朝下盖}\end{définition}
\begin{exemple}\pjya{khɯtsa pa-βʁum}\hspace{5pt}\pcmn{他把碗口朝下}\end{exemple}
\begin{exemple}\pjya{khɯtsa kɤ-βʁum mɤ-pe}\hspace{5pt}\pcmn{把碗的口朝下是不好的}\end{exemple}
\begin{exemple}\pjya{tɯthɯ kɤ-βʁum mɤ-pe}\end{exemple}\relationsémantique{反义词}{\lien{ⓔantɯⓝsɤntɯ}{sɤntɯ}}
\begin{sous-entrée}{aβʁum}{ⓔβʁumⓝaβʁum} 
\grammaire{pass} 
\begin{définition}\pfra{être renversé}\end{définition}
\begin{définition}\pcmn{口朝下}\end{définition}
\begin{exemple}\pjya{ki khɯtsa ki aβʁum}\hspace{5pt}\pcmn{这个碗口朝下}\end{exemple}\end{sous-entrée}

\begin{sous-entrée}{ʑɣɤβʁum}{ⓔβʁumⓝʑɣɤβʁum} 
\classe{vi}  
\grammaire{refl} 
\begin{définition}\pfra{s'allonger sur le ventre}\end{définition}
\begin{définition}\pcmn{趴着(躺着)}\end{définition}
\begin{exemple}\pjya{pjɯ-ʑɣɤβʁum ku-nɯ-rŋgɯ ɲɯ-ŋu}\hspace{5pt}\pcmn{他趴着睡}\end{exemple}\end{sous-entrée}

\end{entrée}

\begin{entrée}{βʁɯz}{}{ⓔβʁɯz} 
\classe{n} 
\begin{définition}\pfra{amadou}\end{définition}
\begin{définition}\pcmn{火绒}\end{définition}\end{entrée}

\begin{entrée}{βzu}{₁}{ⓔβzuⓗ1} 
\classe{vt} \sens{1}\paradigme{dir}{tɤ-}
\begin{définition}\pfra{faire}\end{définition}
\begin{définition}\pcmn{做}\end{définition}
\begin{exemple}\pjya{tɤ-βzu-t-a, ta-βzu, tu-βze-a, tɤ-βze}\hspace{5pt}\pcmn{我做了,他做了,我做,你做吧}\end{exemple}
\begin{exemple}\pjya{tʂu nɯ-βzu-t-a}\hspace{5pt}\pcmn{我让了路}\end{exemple}
\begin{exemple}\pjya{pɕaʁ tɤ-βzu-t-a}\hspace{5pt}\pcmn{我磕了头}\end{exemple}
\begin{exemple}\pjya{rnajɯ pɯ-βzu-t-a}\hspace{5pt}\pcmn{我做了耳环}\end{exemple}
\begin{exemple}\pjya{qajɣi lɤ-βzu-t-a}\hspace{5pt}\pcmn{我做了馍馍}\end{exemple}
\begin{exemple}\pjya{sɤfkur thɯ-βzu-t-a}\hspace{5pt}\pcmn{我捆了柴}\end{exemple}
\begin{exemple}\pjya{ɕkɤbɯ kɤ-βzu-t-a}\hspace{5pt}\pcmn{我包了韭菜包子}\end{exemple}
\begin{exemple}\pjya{ki @jie a-tɤ-βze ra}\hspace{5pt}\pcmn{你来接电话!}\end{exemple}
\begin{exemple}\pjya{tɕhi kɯ-fse chɯ-βze-a}\hspace{5pt}\pcmn{我要做什么样的东西?(例如,铁匠)}\end{exemple}\sens{2}\paradigme{dir}{kɤ-}
\begin{définition}\pfra{devenir}\end{définition}
\begin{définition}\pcmn{当}\end{définition}
\begin{exemple}\pjya{ɯ-slama kɤ-βzu-t-a}\hspace{5pt}\pcmn{我当了他的徒弟}\end{exemple}\sens{3}\paradigme{dir}{nɯ-}
\begin{définition}\pfra{devenir}\end{définition}
\begin{définition}\pcmn{变成}\end{définition}
\begin{exemple}\pjya{ki kha ki aʑɯɣ ɲɯ-βze ɕti}\hspace{5pt}\pcmn{这个房子会变成我的了}\end{exemple}
\begin{exemple}\pjya{ki kha ki aʑɯɣ a-nɯ-βze ra}\hspace{5pt}\pcmn{这个房子要变成我的}\end{exemple}
\begin{exemple}\pjya{tɯjpu a-mɤ-nɯ-βɟi ra ma qajɯ ɲɯ-βze ŋgrɤl (=qajɯ aβzu)}\hspace{5pt}\pcmn{粮食不要放太久,不然会生虫}\end{exemple}\sens{4}
\begin{définition}\pfra{être possible (impersonnel)}\end{définition}
\begin{définition}\pcmn{可能(无人称)}\end{définition}
\begin{exemple}\pjya{ʑɯmkhɤm ʑo tu-ndza-nɯ mɤɕtʂa kɤ-mqlaʁ mɯ́j-βze.}\hspace{5pt}\pcmn{(这种草)牛要咀嚼很久才能吞下}\end{exemple}\relationsémantique{参考}{\lien{ⓔtɯ-ndzɯⓝtɯ-ndzɯ,βzu}{tɯ-ndzɯ,βzu}}\relationsémantique{参考}{\lien{ⓔβdaʁ,βzu}{βdaʁ,βzu}}\relationsémantique{参考}{\lien{ⓔtɯ-skɤtⓝtɯ-skɤt,βzu}{tɯ-skɤt,βzu}}\relationsémantique{参考}{\lien{ⓔɯ-qhuⓝɯ-qhu,βzu}{ɯ-qhu,βzu}}\relationsémantique{参考}{\lien{ⓔɯ-sciⓝɯ-sci,βzu}{ɯ-sci,βzu}}\relationsémantique{参考}{\lien{ⓔaβzu}{aβzu}}
\begin{sous-entrée}{nɯβzu}{ⓔβzuⓗ1ⓢ4ⓝnɯβzu} 
\classe{vt}  
\grammaire{autoben} 
\begin{exemple}\pjya{nɤʑo tɤ-tɯ-nɯ-βzu-t ɕti}\hspace{5pt}\pcmn{这是你自己造成的(是你一个人的错)}\end{exemple}\end{sous-entrée}

\begin{sous-entrée}{sɯβzu}{ⓔβzuⓗ1ⓢ4ⓝsɯβzu} 
\classe{vt} 
\begin{définition}\pfra{faire faire}\end{définition}
\begin{définition}\pcmn{使别人做}\end{définition}
\begin{exemple}\pjya{@zuoye tɯ-sɯβze ra ?}\hspace{5pt}\pcmn{你是不是让他们做作业?}\end{exemple}\end{sous-entrée}

\begin{sous-entrée}{nɤβzɯβzu}{ⓔβzuⓗ1ⓢ4ⓝnɤβzɯβzu} 
\classe{vt} 
\begin{définition}\pfra{aller faire à des endroits différents}\end{définition}
\begin{définition}\pcmn{到处去做}\end{définition}
\begin{exemple}\pjya{ɯʑo kɯ tɯtsɣe ɲɯ-ɤz-nɤβzɯβzu}\hspace{5pt}\pcmn{他到处去做生意}\end{exemple}\end{sous-entrée}

\étymologie{bzo}\end{entrée}

\begin{entrée}{βzaŋɤnŋu}{}{ⓔβzaŋɤnŋu} 
\classe{n} 
\begin{définition}\pfra{bien et mal}\end{définition}
\begin{définition}\pcmn{好坏}\end{définition}
\begin{exemple}\pjya{ɯ-βzaŋɤnŋu kɯ-me}\hspace{5pt}\pcmn{忘恩负义的人}\end{exemple}\étymologie{bzaŋ.ŋan}\end{entrée}

\begin{entrée}{βzaŋlɤn}{}{ⓔβzaŋlɤn} 
\classe{n} 
\begin{définition}\pfra{récompense}\end{définition}
\begin{définition}\pcmn{奖赏}\end{définition}
\begin{exemple}\pjya{nɤ-βzaŋlɤn βze-a ra ma a-tɕɯ ɯ-sroʁ ko-tɯ-ri}\hspace{5pt}\pcmn{我要报你的恩,因为你救了我儿子的命}\end{exemple}\étymologie{bzaŋ.len}\end{entrée}

\begin{entrée}{βzaŋsa}{}{ⓔβzaŋsa} 
\classe{n} 
\begin{définition}\pfra{ami}\end{définition}
\begin{définition}\pcmn{朋友}\end{définition}
\begin{exemple}\pjya{ɯʑo kɯ ɯ-βzaŋsa ɯ-tshɤt tú-wɣ-sɯβzu-a ŋu}\hspace{5pt}\pcmn{他把我当朋友}\end{exemple}\étymologie{bzaŋ.sa}\end{entrée}

\begin{entrée}{βzaʁlu}{}{ⓔβzaʁlu} 
\classe{n} 
\begin{définition}\pfra{personne qui agit ou parle de façon frivole}\end{définition}
\begin{définition}\pcmn{说话不严谨的人}\end{définition}\relationsémantique{参考}{\lien{ⓔɣɤβzaʁlaʁ}{ɣɤβzaʁlaʁ}}\end{entrée}

\begin{entrée}{βzdɤr}{}{ⓔβzdɤr} 
\classe{vt} \paradigme{dir}{nɯ-}
\begin{définition}\pfra{ajouter de l'huile ou du beurre}\end{définition}
\begin{définition}\pcmn{加油,加酥油}\end{définition}
\begin{exemple}\pjya{tʂha ci nɯ-βzdar-a}\hspace{5pt}\pcmn{我在茶里加了酥油}\end{exemple}
\begin{exemple}\pjya{tʂha pɯ-kɤ-βzdɤr}\hspace{5pt}\pcmn{酥油茶}\end{exemple}\relationsémantique{参考}{\lien{ⓔtɤ-βzdɤr}{tɤ-βzdɤr}}\étymologie{sdor}\end{entrée}

\begin{entrée}{βzdɯ}{}{ⓔβzdɯ} 
\classe{vt} \paradigme{dir}{tɤ-}
\begin{définition}\pfra{ramasser}\end{définition}
\begin{définition}\pcmn{捡起来}\end{définition}\paradigme{dir}{tɤ-}\paradigme{dir}{thɯ-}\paradigme{dir}{tɤ-}
\begin{définition}\pfra{être en ordre}\end{définition}
\begin{définition}\pcmn{摆得整齐}\end{définition}
\begin{définition}\pfra{mettre en ordre}\end{définition}
\begin{définition}\pcmn{整理,收拾}\end{définition}
\begin{exemple}\pjya{tɤ-βzdɯ-t-a}\hspace{5pt}\pcmn{我捡了}\end{exemple}
\begin{exemple}\pjya{stoʁ tɤ-βzdi}\hspace{5pt}\pcmn{把胡豆捡了起来}\end{exemple}
\begin{exemple}\pjya{stoʁ pjɤ-ʁndɤr tɕe tɤ-βzdɯ-t-a}\hspace{5pt}\pcmn{胡豆撒了一地,我就捡起来了}\end{exemple}
\begin{exemple}\pjya{laχtɕha ra aβzdoʁβzdɯ ɕti}\hspace{5pt}\pcmn{东西放得很整齐}\end{exemple}
\begin{exemple}\pjya{ɯʑo kɯ laχtɕha ra ta-sɤβzdoʁβzdɯ}\hspace{5pt}\pcmn{他把东西收拾好了}\end{exemple}\relationsémantique{同义词}{\lien{ⓔrɤwum}{rɤwum}}
\begin{sous-entrée}{aβzdoʁβzdɯ}{ⓔβzdɯⓝaβzdoʁβzdɯ} 
\classe{vi}  
\grammaire{pass} \end{sous-entrée}

\begin{sous-entrée}{sɤβzdoʁβzdɯ}{ⓔβzdɯⓝsɤβzdoʁβzdɯ}\end{sous-entrée}

\étymologie{bsdu}\end{entrée}

\begin{entrée}{βzgɤr}{}{ⓔβzgɤr} 
\classe{vt} \paradigme{dir}{pɯ-}
\begin{définition}\pfra{retarder le temps}\end{définition}
\begin{définition}\pcmn{耽误时间}\end{définition}
\begin{exemple}\pjya{pjɯ-ta-βzgɤr mɯ́j-pe}\hspace{5pt}\pcmn{我耽误你的时间很不好}\end{exemple}
\begin{exemple}\pjya{tɯrme nɯ ma-pɯ-tɯ-βzgɤr ma ɯ-ʁa me}\hspace{5pt}\pcmn{你不要耽误他,他没有时间}\end{exemple}\relationsémantique{参考}{\lien{ⓔaʁjɤrⓝsaʁjɤr}{saʁjɤr}}\end{entrée}

\begin{entrée}{βzi}{}{ⓔβzi} 
\classe{vi}  
\grammaire{caus} \paradigme{dir}{lɤ-}
\begin{définition}\pfra{devenir saoul}\end{définition}
\begin{définition}\pcmn{醉}\end{définition}\paradigme{dir}{lɤ-}
\begin{exemple}\pjya{ɯʑo lɤ-βzi, ɯʑo lo-βzi}\hspace{5pt}\pcmn{他喝醉了}\end{exemple}
\begin{exemple}\pjya{aʑo a-ku lɤ-kɯ-βzi ʑo ɲɯ-fse}\hspace{5pt}\pcmn{我头有点晕}\end{exemple}
\begin{exemple}\pjya{cha ɯ-tshɤt kɤ-tshi ma tɯ-βzi}\hspace{5pt}\pcmn{酒少喝一些,不然你会喝醉}\end{exemple}
\begin{exemple}\pjya{nɤʑo ɯ-mɤ-lɤ-tɯ-βzi-ci}\hspace{5pt}\pcmn{你喝醉了吧}\end{exemple}
\begin{sous-entrée}{ɣɤβzi}{ⓔβziⓝɣɤβzi} 
\classe{vs}  
\grammaire{facil} 
\begin{définition}\pfra{être facilement saoul}\end{définition}
\begin{définition}\pcmn{容易醉}\end{définition}\end{sous-entrée}

\begin{sous-entrée}{sɯβzi}{ⓔβziⓝsɯβzi} 
\classe{vt} \end{sous-entrée}

\begin{définition}\pfra{saouler}\end{définition}
\begin{définition}\pcmn{灌醉}\end{définition}
\begin{sous-entrée}{ʑɣɤsɯβzi}{ⓔβziⓝʑɣɤsɯβzi} 
\classe{vi}  
\grammaire{refl}
\grammaire{caus} 
\begin{définition}\pfra{se rendre saoul}\end{définition}
\begin{définition}\pcmn{喝醉}\end{définition}
\begin{exemple}\pjya{kɤ-ʑɣɤsɯβzi mɤ-ra}\hspace{5pt}\pcmn{别喝醉}\end{exemple}\end{sous-entrée}

\begin{sous-entrée}{sɤβzi/\variante{sɤsɯβzi}}{ⓔβziⓝsɤβzi} 
\classe{vs}  
\grammaire{deexp} 
\begin{définition}\pfra{qui monte facilement à la tête}\end{définition}
\begin{définition}\pcmn{容易上头的}\end{définition}
\begin{exemple}\pjya{cha kɯ-sɤβzi}\hspace{5pt}\pcmn{容易上头的酒}\end{exemple}\end{sous-entrée}

\étymologie{bzi}\end{entrée}

\begin{entrée}{βzjoz}{}{ⓔβzjoz} 
\classe{vt} \paradigme{dir}{pɯ-}\paradigme{dir}{kɤ-}
\begin{définition}\pfra{étudier}\end{définition}
\begin{définition}\pcmn{学}\end{définition}
\begin{exemple}\pjya{tɤ-scoz kɤ-βzjoz ɴqa}\hspace{5pt}\pcmn{文字很难学}\end{exemple}
\begin{exemple}\pjya{aʑo nɤ-ɕki a-kɤ-βzjoz dɤn}\hspace{5pt}\pcmn{我跟你学了很多}\end{exemple}
\begin{exemple}\pjya{kɤ-βzjoz kɤ-sthɯt mɯ́j-khɯ}\hspace{5pt}\pcmn{学无止境}\end{exemple}\relationsémantique{参考}{\lien{ⓔrɤβzjoz}{rɤβzjoz}}
\begin{sous-entrée}{nɯɣɯβzjoz}{ⓔβzjozⓝnɯɣɯβzjoz} 
\classe{vs} 
\begin{définition}\pfra{facile à apprendre}\end{définition}
\begin{définition}\pcmn{好学,容易学}\end{définition}\end{sous-entrée}

\étymologie{sbʲaŋs}\end{entrée}

\begin{entrée}{βzɟɯr}{}{ⓔβzɟɯr} 
\classe{vt}  
\grammaire{refl} \paradigme{dir}{nɯ-}\sens{1}
\begin{définition}\pfra{transformer}\end{définition}
\begin{définition}\pcmn{改变}\end{définition}
\begin{exemple}\pjya{nɯ-βzɟɯr-a, na-βzɟɯr}\hspace{5pt}\pcmn{我改变了,他改变了}\end{exemple}
\begin{exemple}\pjya{mɤ-kɯ-pe tɤ-kɤ-nɤma nɯra ɯ-qhu tɕe ɲɯ́-wɣ-βzɟɯr ra}\hspace{5pt}\pcmn{没有做好的那些以后要修正}\end{exemple}
\begin{exemple}\pjya{ɯ-ɲɯ-nɯkɯmaʁ-a nɤ ɲɯ-kɯ-sɯ-βzɟɯr-a}\hspace{5pt}\pcmn{我说错了的话,请帮我纠正一下}\end{exemple}
\begin{exemple}\pjya{ʑara kɯ kɯm nɯ na-βzɟɯr-nɯ}\hspace{5pt}\pcmn{他们换了门(工人修房子)}\end{exemple}\sens{2}\paradigme{dir}{nɯ-}
\begin{définition}\pfra{corriger}\end{définition}
\begin{définition}\pcmn{纠正}\end{définition}
\begin{exemple}\pjya{kɤ-βzɟɯr mɯ́j-ra, ɲɯ-pe}\hspace{5pt}\pcmn{不用修改,很好}\end{exemple}
\begin{exemple}\pjya{tɤ-scoz ɲɤ-βzɟɯr}\hspace{5pt}\pcmn{他把信改了改}\end{exemple}
\begin{sous-entrée}{ʑɣɤβzɟɯr}{ⓔβzɟɯrⓢ2ⓝʑɣɤβzɟɯr} 
\classe{vi} \end{sous-entrée}

\begin{définition}\pfra{se transformer}\end{définition}
\begin{définition}\pcmn{转变}\end{définition}\étymologie{bsgʲur}\end{entrée}

\begin{entrée}{βzɯr}{}{ⓔβzɯr} 
\classe{vt} \paradigme{dir}{tɤ-}
\begin{définition}\pfra{déplacer}\end{définition}
\begin{définition}\pcmn{拿过去;搬走}\end{définition}
\begin{exemple}\pjya{tɤ-βzɯr-a, ta-βzɯr}\hspace{5pt}\pcmn{我拿走了,他拿走了}\end{exemple}
\begin{exemple}\pjya{ɲɯ-saʁdɯɣ tɕe, tɤ-βzɯr-a}\hspace{5pt}\pcmn{(那个东西)很碍事,所以我把它拿走了}\end{exemple}
\begin{exemple}\pjya{kɯki kutɕu a-mɤ-pɯ-ɤta, tɤ-βzɯr}\hspace{5pt}\pcmn{这个东西不要放在这里,你把它拿走}\end{exemple}\étymologie{bzur}\end{entrée}

\begin{entrée}{βzɯrtɕoʁ}{}{ⓔβzɯrtɕoʁ} 
\classe{n} 
\begin{définition}\pfra{excroissance sur les angles du toit}\end{définition}
\begin{définition}\pcmn{房背左右两角上的顶端}\end{définition}
\begin{exemple}\pjya{kha ɣɯ znde kɤ-βzu tɤ-jɤɣ tɕe ɯ-ʁɤri pɕoʁ χchoʁe tɯ-βzɯr ɯ-taʁ tu-kɤ-sɤmtɕoʁ nɯ βzɯrtɕoʁ rmi}\hspace{5pt}\pcmn{房子的墙壁修完了的时候,在房子正面左右两角顶上修的尖角部分叫\lien{ⓔβzɯrtɕoʁ}{βzɯrtɕoʁ}}\end{exemple}\étymologie{bzur.ltɕog}\end{entrée}

\begin{entrée}{βzuwa}{}{ⓔβzuwa} 
\classe{n} 
\begin{définition}\pfra{artisan}\end{définition}
\begin{définition}\pcmn{工匠}\end{définition}\étymologie{bzo.ba}\end{entrée}

\begin{entrée}{βʑar}{}{ⓔβʑar} 
\classe{n} 
\begin{définition}\pfra{busard}\end{définition}
\begin{définition}\pcmn{鵟,鹞}\end{définition}\end{entrée}

\begin{entrée}{βʑaʁβʑɯɣ}{}{ⓔβʑaʁβʑɯɣ} 
\classe{n} 
\begin{définition}\pfra{résidence}\end{définition}
\begin{définition}\pcmn{住宿}\end{définition}
\begin{exemple}\pjya{βlama ra kɯ kutɕu βʑaʁβʑɯɣ a-kɤ-nɯ-βzu-nɯ jɤɣ}\hspace{5pt}\pcmn{喇嘛可以在这里就寝}\end{exemple}\étymologie{bʑag.bʑugs}\end{entrée}

\begin{entrée}{βʑɤzu}{}{ⓔβʑɤzu} 
\classe{n} 
\begin{définition}\pfra{seau à lait}\end{définition}
\begin{définition}\pcmn{挤奶桶}\end{définition}\étymologie{bʑo.zo}\end{entrée}

\begin{entrée}{βʑoʁ}{}{ⓔβʑoʁ} 
\classe{vt} \paradigme{dir}{thɯ-}
\begin{définition}\pfra{tailler, éplucher}\end{définition}
\begin{définition}\pcmn{削}\end{définition}
\begin{exemple}\pjya{chɯ-βʑoʁ-a, tha-βʑoʁ}\hspace{5pt}\pcmn{我削,他削了}\end{exemple}
\begin{exemple}\pjya{kɯki laʁdɯn mɯ́j-sna tɕe, chɯ́-wɣ-βʑoʁ ɲɯ-ra}\hspace{5pt}\pcmn{这个工具不好,要削一下}\end{exemple}
\begin{exemple}\pjya{ɲɯ-jpum tɕe chɯ́-wɣ-βʑoʁ ɲɯ-ra}\hspace{5pt}\pcmn{太粗,要削一下}\end{exemple}
\begin{exemple}\pjya{sɲɯɣjɯ thɯ-βʑoʁ-a}\hspace{5pt}\pcmn{我削了笔}\end{exemple}
\begin{sous-entrée}{sɯβʑoʁ}{ⓔβʑoʁⓝsɯβʑoʁ}\sens{1}
\begin{définition}\pfra{tailler avec}\end{définition}
\begin{définition}\pcmn{用……削}\end{définition}\end{sous-entrée}

\sens{2}
\begin{définition}\pfra{pouvoir tailler}\end{définition}
\begin{définition}\pcmn{削得了}\end{définition}
\begin{exemple}\pjya{mbrɯtɕɯ ɯʑo kɯ ɯʑo ɯ-jɯ mɤ-sɯβʑoʁ}\hspace{5pt}\pcmn{刀子削不了自己的把子}\end{exemple}\étymologie{bʑogs}\end{entrée}

\begin{entrée}{βʑɯ}{}{ⓔβʑɯ} 
\classe{n} 
\begin{définition}\pfra{souris}\end{définition}
\begin{définition}\pcmn{老鼠}\end{définition}\end{entrée}

\begin{entrée}{βʑɯjme}{}{ⓔβʑɯjme} 
\classe{n} 
\begin{définition}\pfra{encoche pour farine (meule)}\end{définition}
\begin{définition}\pcmn{面粉槽(磨)}\end{définition}\relationsémantique{参考}{\lien{ⓔtɤ-jme}{tɤ-jme}}\relationsémantique{参考}{\lien{ⓔβʑɯ}{βʑɯ}}\end{entrée}

\begin{entrée}{βʑɯlu}{}{ⓔβʑɯlu} 
\classe{n} 
\begin{définition}\pfra{année du rat}\end{définition}
\begin{définition}\pcmn{鼠年}\end{définition}\end{entrée}

\begin{entrée}{βʑɯndɤpa}{}{ⓔβʑɯndɤpa} 
\classe{n} 
\begin{définition}\pfra{dans cinq ans}\end{définition}
\begin{définition}\pcmn{五年以后}\end{définition}\relationsémantique{参考}{\lien{ⓔβʑɯndiⓗ1}{βʑɯndi₁}}\relationsémantique{参考}{\lien{ⓔtɯ-xpa}{tɯ-xpa}}\end{entrée}

\begin{entrée}{βʑɯndi}{₁}{ⓔβʑɯndiⓗ1} 
\classe{n} 
\begin{définition}\pfra{dans cinq jours}\end{définition}
\begin{définition}\pcmn{五天以后}\end{définition}\étymologie{bʑi}\end{entrée}

\begin{entrée}{βʑɯndi}{₂}{ⓔβʑɯndiⓗ2} 
\classe{n} 
\begin{définition}\pfra{bande molletière}\end{définition}
\begin{définition}\pcmn{裹腿}\end{définition}
\begin{exemple}\pjya{βʑɯndi nɯ-nɯrtaβ-a}\hspace{5pt}\pcmn{我缠了裹腿}\end{exemple}\end{entrée}

\begin{entrée}{βʑɯpa}{}{ⓔβʑɯpa} 
\classe{n} 
\begin{définition}\pfra{quatrième mois}\end{définition}
\begin{définition}\pcmn{四月}\end{définition}\étymologie{bʑi.pa}\end{entrée}

\begin{entrée}{βʑɯrna}{}{ⓔβʑɯrna} 
\classe{n} 
\begin{définition}\pfra{oreilles de souris}\end{définition}
\begin{définition}\pcmn{老鼠的耳朵}\end{définition}
\begin{exemple}\pjya{βʑɯrna ɲɤ-sprɤt}\hspace{5pt}\pcmn{植物脱落了子叶,长出了真正的叶子}\end{exemple}\end{entrée}

\newpage\caractère{c}

\begin{entrée}{cu}{}{ⓔcu} 
\classe{vt} \paradigme{dir}{kɤ-}\paradigme{dir}{nɯ-}
\begin{définition}\pfra{ajouter des ingrédients}\end{définition}
\begin{définition}\pcmn{另加配菜}\end{définition}
\begin{exemple}\pjya{ka-cu}\hspace{5pt}\pcmn{他加了}\end{exemple}
\begin{exemple}\pjya{@cai ɯ-ŋgɯ tɕe ɕku kú-wɣ-cu tɕe mɯm}\hspace{5pt}\pcmn{在菜里放点葱就好吃}\end{exemple}
\begin{exemple}\pjya{@yangyu ɯ-ŋgɯ tɤjko ci kɤ-ce}\hspace{5pt}\pcmn{你在土豆里面加一点酸菜}\end{exemple}
\begin{sous-entrée}{acu}{ⓔcuⓝacu} 
\classe{vs}  
\grammaire{pass} 
\begin{définition}\pfra{être mélangé dans}\end{définition}
\begin{définition}\pcmn{掺在一起;和在一起}\end{définition}
\begin{exemple}\pjya{rɟɤɣi ɯ-ŋgɯ ta-mar a-pɯ-ɤcu tɕe mɯm}\hspace{5pt}\pcmn{在糌粑里加一点酥油就好吃}\end{exemple}\relationsémantique{同义词}{\lien{ⓔaɕiⓗ1}{aɕi}}\end{sous-entrée}

\begin{sous-entrée}{nɤcu}{ⓔcuⓝnɤcu} 
\classe{vs} 
\begin{définition}\pfra{bien s'entendre avec}\end{définition}
\begin{définition}\pcmn{合得来}\end{définition}
\begin{exemple}\pjya{tɯrme ra nɯ-rca ɲɯ-tɯ-nɤcu}\hspace{5pt}\pcmn{你跟那些人合得来}\end{exemple}\relationsémantique{参考}{\lien{ⓔnɯrɯcu}{nɯrɯcu}}\end{sous-entrée}

\end{entrée}

\begin{entrée}{ca}{}{ⓔca} 
\classe{n} 
\begin{définition}\pfra{chevrotin}\end{définition}
\begin{définition}\pcmn{麝香鹿}\end{définition}
\begin{exemple}\pjya{ca nɯ sɯŋgɯ kɯ-rnaʁ tsa cho stɤmku nɯ ra ku-rɤʑi ɲɯ-ŋu. sɯjno ma mɯ́j-ndze. ɯ-ku nɯ tshɤt ɯ-ku tsa ɲɯ-fse ri ɯ-ʁrɯ maŋe, ɯ-rna nɯ ra tshɤt wuma ɲɯ-fse, ɯ-phoŋbu ɯ-tshɯɣa nɯ ra li tshɤt ɲɯ-fse, ɯ-mɤlɤjaʁ nɯ ra tshɤt ɣɯ sɤznɤ ɲɯ-rɲɟi, ɯ-phoŋbu nɯ ɲɯ-xtshɯm, ɯ-mdoʁ nɯ ɲɯ-pɣi. ɯ-qa nɯ ta-ʁrɯ ɲɯ-ŋu, ɯ-jme kɯ-xtɯ-xtɯt ŋu. ɯ-jme ɯ-pa cho ɯ-xtɤpa lu-kɯ-ɕe nɯ ra ɲɯ-wɣrum. ɲɯ-ɣɤwu tɕe, tshɯtho ɯ-skɤt tu-βze ɲɯ-ŋu. ɯ-ɕa ɯ-ŋgɯ ɯ-tʂɤm me, ɯ-ɕa ʁɟa ŋu, ɯ-ndʐi nɯ ɯ-χtsɤβ a-pɯ-βdi tɕe mba ri wuma ʑo ngɯt, tɕe kɯɕɯŋgɯ tɕe tɯ-xtsa ɯ-ku spa stu kɯ-pe pjɤ-ŋu. ca phu mu tu tɕe, cɤmu nɯ kɯ-rɤpɯ ɯ-spa ɲɯ-ŋu, cɤmtsho nɯ ɯ-mtsho tu tɕe, nɯ wuma ʑo smɤn kɯ-ʑru ɲɯ-ŋu, tɕe ɯ-phɯ wuma ʑo wxti.}\hspace{5pt}\pcmn{麝香鹿住在大森林和草地上。只吃草。头有点像山羊,但是没有角,耳朵很像山羊的耳朵,四肢比山羊的长一些,身子细,颜色是灰的。有蹄子,尾巴很短。尾巴下面和肚皮都是白色的。叫的时候发出小山羊的叫声。肉里没有脂肪,只有瘦肉。皮子搓揉好了以后,虽然薄但是很结实。在过去,是藏式皮鞋筒最好的材料。麝香鹿有公母,母鹿下崽子,公鹿有麝香,是一种很名贵的药材,价格很高。}\end{exemple}\end{entrée}

\begin{entrée}{caŋ}{}{ⓔcaŋ} 
\classe{n} 
\begin{définition}\pfra{mur en terre}\end{définition}
\begin{définition}\pcmn{土墙}\end{définition}\étymologie{gʲaŋ}\end{entrée}

\begin{entrée}{capɣi}{}{ⓔcapɣi} 
\classe{n} 
\begin{définition}\pfra{cerf (moschus sifanicus)}\end{définition}
\begin{définition}\pcmn{马麂}\end{définition}\end{entrée}

\begin{entrée}{caʁ}{}{ⓔcaʁ} 
\classe{vs} 
\begin{définition}\pfra{célèbre}\end{définition}
\begin{définition}\pcmn{出名}\end{définition}
\begin{exemple}\pjya{tɤ-caʁ-a}\hspace{5pt}\pcmn{我出名了}\end{exemple}
\begin{exemple}\pjya{wuma ʑo pjɤ-cha tɕe, ɯ-rmi to-caʁ}\hspace{5pt}\pcmn{他很能干,所以就变得很出名}\end{exemple}\end{entrée}

\begin{entrée}{caʁɕɣɤz}{}{ⓔcaʁɕɣɤz} 
\classe{n} 
\begin{définition}\pfra{laine épaisse et fragile}\end{définition}
\begin{définition}\pcmn{又粗又脆的劣质羊毛}\end{définition}\end{entrée}

\begin{entrée}{cɤɕna}{}{ⓔcɤɕna} 
\classe{n} 
\begin{définition}\pfra{Rumex japonicus}\end{définition}
\begin{définition}\pcmn{山菠菜;羊蹄}\end{définition}
\begin{exemple}\pjya{cɤɕna nɯ sɯjno ci ŋu, sɯŋgɯ cho stɤmku ra tɯ-ji ɯ-rkɯ ra tu-ɬoʁ ŋgrɤl. ɯ-jwaʁ nɯ aɲaʁndzɯm, ɯ-ru aɣɯrnɯɕɯr, kɯ-ɤrŋi tɕe tu, ɯ-mat tu-βze ɕɯŋgɯ nɤ, kɤ-ndza sna. ɯ-mɯntoʁ mɤ-mpɕɤr, ndɯβ, kɯ-ɣɯrni tsa ŋu. ɯ-mat thɯ-aβzu tɕe, pjɯ-ŋgra tɕe, tu-ɬoʁ mɤ-cha. ɯʑo ɯ-kɯ-sɯ-mphɤl nɯ ɯ-zrɤm ɲɯ-ɕti. tɯ-ɟom jamar ma tɯ-zri mɤ-cha. ɯ-jwaʁ ɯ-qhu nɯ mpɕu, ɯ-βzɯr nɯ ra rʁom.}\hspace{5pt}\pcmn{山波菜是一种植物,一般生长在森林、草地和地边。叶子深绿色,茎淡红色,也有的是绿色的,结果之前可以吃。花不美,小,有点红。结了果以后,种子掉下而不能生长:使山波菜繁殖的是它的根。只能长一米来高。叶子背面是光滑的,边角有点粗糙。}\end{exemple}\end{entrée}

\begin{entrée}{cɤjmɤɣ}{}{ⓔcɤjmɤɣ} 
\classe{n} 
\begin{définition}\pfra{une espèce de champignon}\end{définition}
\begin{définition}\pcmn{【獐子菌】}\end{définition}
\begin{exemple}\pjya{cɤjmɤɣ nɯ tɯrgi ɯ-ŋgɯ tu-ɬoʁ ɲɯ-ŋu, ɯ-mdoʁ nɯ kɯ-ɤpɣɯlu ʁɟa ʑo nɯ ŋu, ɯ-rʑɯɣ nɯ ca ɯ-rme ɲɯ-fse, kɯ-wxti tsa ci ɲɯ-ŋu, ɯ-mdoʁ cho ɯ-rʑɯɣ nɯ ra ndʐa cɤjmɤɣ ɲɯ-rmi}\hspace{5pt}\pcmn{獐子菌生长在杉木林里,通体灰色,菌褶像麝香鹿(獐子)的毛,有点大,因为它的颜色和菌褶像獐子,所以称它为“獐子菌”。}\end{exemple}\relationsémantique{参考}{\lien{ⓔca}{ca}}\relationsémantique{参考}{\lien{ⓔtɤjmɤɣ}{tɤjmɤɣ}}\end{entrée}

\begin{entrée}{cɤmu}{}{ⓔcɤmu} 
\classe{n} 
\begin{définition}\pfra{chevrotain femelle}\end{définition}
\begin{définition}\pcmn{母麝香鹿}\end{définition}\relationsémantique{参考}{\lien{ⓔca}{ca}}\end{entrée}

\begin{entrée}{cɤmi}{}{ⓔcɤmi} 
\classe{n} 
\begin{définition}\pfra{près du fleuve}\end{définition}
\begin{définition}\pcmn{沿着河流的地方}\end{définition}\end{entrée}

\begin{entrée}{cɤmirɤku}{}{ⓔcɤmirɤku} 
\classe{n} 
\begin{définition}\pfra{cultures de vallée}\end{définition}
\begin{définition}\pcmn{河坝农物(玉米)}\end{définition}\relationsémantique{参考}{\lien{ⓔcɤmi}{cɤmi}}\relationsémantique{参考}{\lien{ⓔtɤ-rɤku}{tɤ-rɤku}}\end{entrée}

\begin{entrée}{cɤmtsaʁ}{}{ⓔcɤmtsaʁ} 
\classe{n} 
\begin{définition}\pfra{ruade}\end{définition}
\begin{définition}\pcmn{尥蹶子}\end{définition}
\begin{exemple}\pjya{mbro kɯ cɤmtsaʁ ja-lɤt tɕe pɯ́-wɣ-βde-a pɯ-ŋgrɤl}\hspace{5pt}\pcmn{我尥蹶子把我扔下来了}\end{exemple}\relationsémantique{同义词}{\lien{ⓔnɯsɲɤtqha}{nɯsɲɤtqha}}\relationsémantique{参考}{\lien{ⓔmtsaʁ}{mtsaʁ}}\end{entrée}

\begin{entrée}{cɤmtsho}{}{ⓔcɤmtsho} 
\classe{n} 
\begin{définition}\pfra{chevrotain porte-musc mâle}\end{définition}
\begin{définition}\pcmn{公麝}\end{définition}\relationsémantique{参考}{\lien{ⓔca}{ca}}\end{entrée}

\begin{entrée}{cɤndʐi}{}{ⓔcɤndʐi} 
\classe{n} 
\begin{définition}\pfra{peau de chevrotain}\end{définition}
\begin{définition}\pcmn{麝香鹿皮}\end{définition}\relationsémantique{参考}{\lien{ⓔca}{ca}}\relationsémantique{参考}{\lien{ⓔtɯ-ndʐi}{tɯ-ndʐi}}\end{entrée}

\begin{entrée}{cɤpɤcrɤle}{}{ⓔcɤpɤcrɤle} 
\classe{n} 
\begin{définition}\pfra{nourriture de mauvaise qualité}\end{définition}
\begin{définition}\pcmn{低等(的食物),素食}\end{définition}
\begin{exemple}\pjya{cɤpɤcrɤle ɲɯ-ta-mbi}\hspace{5pt}\pcmn{我给你吃得很素}\end{exemple}\relationsémantique{参考}{\lien{ⓔldʐɤpɤldʐɤle}{ldʐɤpɤldʐɤle}}\end{entrée}

\begin{entrée}{cɤrme}{}{ⓔcɤrme} 
\classe{n} 
\begin{définition}\pfra{poil de chevrotain}\end{définition}
\begin{définition}\pcmn{麝香鹿毛}\end{définition}\relationsémantique{参考}{\lien{ⓔca}{ca}}\relationsémantique{参考}{\lien{ⓔtɤ-rme}{tɤ-rme}}\end{entrée}

\begin{entrée}{cɤrna}{}{ⓔcɤrna} 
\classe{n} 
\begin{définition}\pfra{petit pain rond}\end{définition}
\begin{définition}\pcmn{圆形的小馍馍}\end{définition}\relationsémantique{参考}{\lien{ⓔarɯcɤrna}{arɯcɤrna}}\end{entrée}

\begin{entrée}{cɤtʂha}{}{ⓔcɤtʂha} 
\classe{n} 
\begin{définition}\pfra{une espèce d'arbrisseau}\end{définition}
\begin{définition}\pcmn{灌木的一种}\end{définition}
\begin{exemple}\pjya{cɤtʂha nɯ si kɯ-mbɤr ci ŋu, ɯ-ru kɯ-pɣi ci ŋu, ɯ-si wuma ʑo ngɯt, tɕe khɯzi kɤ-βzu sna, ɯ-jwaʁ kɯ-ndɯβ ci ŋu, ɯ-jwaʁ kɯnɤ pɣi, ɯ-mat cho ɯ-mɯntoʁ ra me, cɤmi tsa tu-ɬoʁ ŋu, rpɣo pɕoʁ me.}\hspace{5pt}\pcmn{\lien{ⓔcɤtʂha}{cɤtʂha}是矮小的树种,树干灰色,木质结实,可以用来制造连枷的接头。叶子很小,也是灰色的,既没有花也没有果实。生长在河坝上,高山上不能生长。}\end{exemple}\end{entrée}

\begin{entrée}{chu}{}{ⓔchu} 
\classe{postp} 
\begin{définition}\pfra{direction}\end{définition}
\begin{définition}\pcmn{方向}\end{définition}
\begin{exemple}\pjya{a-thi kupa chu}\hspace{5pt}\pcmn{下游的汉区}\end{exemple}\end{entrée}

\begin{entrée}{cha}{₂}{ⓔchaⓗ2} 
\classe{n} 
\begin{définition}\pfra{alcool fermenté, tchang}\end{définition}
\begin{définition}\pcmn{酒}\end{définition}\relationsémantique{参考}{\lien{ⓔchɤtshi}{chɤtshi}}\relationsémantique{参考}{\lien{ⓔɣɯchɤtshi}{ɣɯchɤtshi}}\relationsémantique{参考}{\lien{ⓔnɯchɤrga}{nɯchɤrga}}\end{entrée}

\begin{entrée}{cha}{₁}{ⓔchaⓗ1} 
\classe{vi} \paradigme{dir}{tɤ-}\sens{1}
\begin{définition}\pfra{pouvoir}\end{définition}
\begin{définition}\pcmn{能够}\end{définition}
\begin{exemple}\pjya{mɯ-pɯ-cha tɕe, tɕe tham tɕe ɯ-kɯ-mŋɤm to-mna tɕe, kɤ-rɯndzɤtshi to-cha}\hspace{5pt}\pcmn{她原来吃饭不行,现在病痊愈了就可以吃饭了}\end{exemple}
\begin{exemple}\pjya{mɯ-ɕɯ-tɯ-cha nɯ-sɯso-t-a}\hspace{5pt}\pcmn{我怕你不行(忧虑式的例句)}\end{exemple}\sens{2}
\begin{définition}\pfra{être capable}\end{définition}
\begin{définition}\pcmn{能干}\end{définition}
\begin{exemple}\pjya{aʑo a-kɤ-cha me}\hspace{5pt}\pcmn{什么都不会}\end{exemple}
\begin{exemple}\pjya{nɤʑo kɯ ɲɯ-tɯ-cha tɕe nɤʑo jɤ-ɕe}\hspace{5pt}\pcmn{你比较能干,你去吧}\end{exemple}\relationsémantique{参考}{\lien{ⓔsɯxcha}{sɯxcha}}\relationsémantique{参考}{\lien{ⓔsɤcha}{sɤcha}}\end{entrée}

\begin{entrée}{chaŋskɯ}{}{ⓔchaŋskɯ} 
\classe{n} 
\begin{définition}\pfra{bovidé au dos et au ventre blanc}\end{définition}
\begin{définition}\pcmn{背部和肚子都是白色的牛}\end{définition}\étymologie{kʰʲuŋ}\end{entrée}

\begin{entrée}{chɤβ}{}{ⓔchɤβ} 
\classe{vt} \paradigme{dir}{tɤ-}\paradigme{dir}{pɯ-}\sens{1}
\begin{définition}\pfra{aplatir}\end{définition}
\begin{définition}\pcmn{弄扁}\end{définition}
\begin{exemple}\pjya{tɤ-chaβ-a, ɯʑo kɯ ta-chɤβ}\hspace{5pt}\pcmn{我弄扁了,他弄扁了}\end{exemple}
\begin{exemple}\pjya{kɯki laχtɕha ki kɤ-chɤβ mɤ-pe}\hspace{5pt}\pcmn{把这个东西压扁了不好}\end{exemple}\sens{2}
\begin{définition}\pfra{plier (papier)}\end{définition}
\begin{définition}\pcmn{折在一起}\end{définition}
\begin{exemple}\pjya{ɕoʁɕoʁ kɤ-chaβ-a}\hspace{5pt}\pcmn{我把纸折在一起了}\end{exemple}\sens{3}
\begin{définition}\pfra{courber (complètement)}\end{définition}
\begin{définition}\pcmn{(完全)弄弯}\end{définition}
\begin{exemple}\pjya{a-mthɤβ pɯ-chaβ-a}\hspace{5pt}\pcmn{我弯下腰了}\end{exemple}
\begin{exemple}\pjya{tɯ-sloχpɯn ɯ-ɕki tɯ-mthɤɣ pjɯ́-wɣ-chɤβ ra}\hspace{5pt}\pcmn{(学生)都得在老师面前鞠躬}\end{exemple}\relationsémantique{参考}{\lien{ⓔɲɟɤβ}{ɲɟɤβ}}\end{entrée}

\begin{entrée}{chɤci}{}{ⓔchɤci} 
\classe{n} 
\begin{définition}\pfra{tchang (en jarre)}\end{définition}
\begin{définition}\pcmn{青稞酒放在坛子里以后挤出来的水}\end{définition}\relationsémantique{参考}{\lien{ⓔchaⓗ2}{cha₂}}\relationsémantique{参考}{\lien{ⓔtɯ-ci}{tɯ-ci}}\end{entrée}

\begin{entrée}{chɤdi}{}{ⓔchɤdi} 
\classe{n} 
\begin{définition}\pfra{odeur d'alcool}\end{définition}
\begin{définition}\pcmn{酒味}\end{définition}\relationsémantique{参考}{\lien{ⓔtɤ-di}{tɤ-di}}\relationsémantique{参考}{\lien{ⓔchaⓗ2}{cha₂}}\end{entrée}

\begin{entrée}{chɤlɤnnɤ}{}{ⓔchɤlɤnnɤ} 
\classe{adv} 
\begin{définition}\pfra{peut-être}\end{définition}
\begin{définition}\pcmn{也许}\end{définition}
\begin{exemple}\pjya{chɤlɤnnɤ ɕe-a thaŋ, chɤlɤnnɤ mɤ-ɕe-a}\hspace{5pt}\pcmn{我可能去,可能不去}\end{exemple}
\begin{exemple}\pjya{jisŋi chɤlɤnnɤ tɯ-mɯ lɤt}\hspace{5pt}\pcmn{今天也许会下雨}\end{exemple}
\begin{exemple}\pjya{a-@dian chɤlɤnnɤ rtaʁ thaŋ nɯ-sɯso-t-a tɕe, nɯ ma @chong mɯ-kɤ-βzu-t-a}\hspace{5pt}\pcmn{我想可能电足够,所以没有充}\end{exemple}\relationsémantique{参考}{\lien{ⓔlɤtⓗ1}{lɤt₁}}\end{entrée}

\begin{entrée}{chɤle}{}{ⓔchɤle}\relationsémantique{参考}{\lien{ⓔachala}{achala}}\end{entrée}

\begin{entrée}{chɤmda}{}{ⓔchɤmda} 
\classe{n} 
\begin{définition}\pfra{tchang en jarre que l'on boit à la paille}\end{définition}
\begin{définition}\pcmn{坛坛酒}\end{définition}
\begin{exemple}\pjya{chɤmda pɯ-tshi-t-a}\hspace{5pt}\pcmn{我喝了坛坛酒}\end{exemple}\relationsémantique{参考}{\lien{ⓔnɯchɤmda}{nɯchɤmda}}\end{entrée}

\begin{entrée}{chɤmdɤru}{}{ⓔchɤmdɤru} 
\classe{n} 
\begin{définition}\pfra{paille en bambou pour boire le tchang}\end{définition}
\begin{définition}\pcmn{用竹子制成的管子,用来喝坛坛酒}\end{définition}\end{entrée}

\begin{entrée}{chɤmthɯm}{}{ⓔchɤmthɯm} 
\classe{n} 
\begin{définition}\pfra{mets et boissons}\end{définition}
\begin{définition}\pcmn{酒菜}\end{définition}\relationsémantique{参考}{\lien{ⓔtɤ-mthɯm}{tɤ-mthɯm}}\relationsémantique{参考}{\lien{ⓔchaⓗ2}{cha₂}}\end{entrée}

\begin{entrée}{chɤt}{₁}{ⓔchɤtⓗ1} 
\classe{vs} 
\begin{définition}\pfra{avoir pour différence}\end{définition}
\begin{définition}\pcmn{区别在于}\end{définition}
\begin{exemple}\pjya{tɕiʑo ni kɯ-mɤku kɯ-maqhu ci chɤt ma, ʁnaʁna jɤ-ɣe-tɕi ɕti}\hspace{5pt}\pcmn{我们的区别在于一个来得早,另一个来得晚,但是两个都来了}\end{exemple}\relationsémantique{参考}{\lien{ⓔachɤt}{achɤt}}\end{entrée}

\begin{entrée}{chɤt}{₂}{ⓔchɤtⓗ2} 
\classe{vi} \paradigme{dir}{pɯ-}
\begin{définition}\pfra{réussir}\end{définition}
\begin{définition}\pcmn{成功;得逞}\end{définition}
\begin{exemple}\pjya{tɯ-tɤ-kɤnɤma a-pɯ-chɤt}\hspace{5pt}\pcmn{希望所做的一切都能成功}\end{exemple}\relationsémantique{同义词}{\lien{ⓔŋgrɯ}{ŋgrɯ}}\end{entrée}

\begin{entrée}{chɤtpa}{}{ⓔchɤtpa} 
\classe{n} 
\begin{définition}\pfra{différence}\end{définition}
\begin{définition}\pcmn{区别;差别}\end{définition}
\begin{exemple}\pjya{ɕɤxɕo ndɤre ɯ-tɯ-mɯɕtaʁ kɯ, qartsɯ cho chɤtpa maŋe}\hspace{5pt}\pcmn{最近很难冷,跟冬天没有什么区别}\end{exemple}\end{entrée}

\begin{entrée}{chɤtshi}{}{ⓔchɤtshi} 
\classe{n} 
\begin{définition}\pfra{fait de boire beaucoup}\end{définition}
\begin{définition}\pcmn{喝很多酒}\end{définition}\relationsémantique{参考}{\lien{ⓔchaⓗ2}{cha₂}}\relationsémantique{参考}{\lien{ⓔtshiⓗ1}{tshi₁}}\relationsémantique{参考}{\lien{ⓔɣɯchɤtshi}{ɣɯchɤtshi}}\end{entrée}

\begin{entrée}{chɤzwa}{}{ⓔchɤzwa} 
\classe{n} 
\begin{définition}\pfra{lie du vin}\end{définition}
\begin{définition}\pcmn{酒渣滓}\end{définition}\relationsémantique{参考}{\lien{ⓔtʂhazwa}{tʂhazwa}}\end{entrée}

\begin{entrée}{che}{}{ⓔche} 
\classe{intj} 
\begin{définition}\pfra{attend un peu}\end{définition}
\begin{définition}\pcmn{等一下}\end{définition}\end{entrée}

\begin{entrée}{chɣaʁchɣaʁ}{}{ⓔchɣaʁchɣaʁ} 
\classe{idph.2} 
\begin{définition}\pfra{propre et bien rangé}\end{définition}
\begin{définition}\pcmn{形容又干净又整齐的样子}\end{définition}\end{entrée}

\begin{entrée}{chi}{}{ⓔchi} 
\classe{vi} \paradigme{dir}{nɯ-}\paradigme{dir}{pɯ-}\paradigme{dir}{nɯ-}\paradigme{dir}{pɯ-}
\begin{définition}\pfra{sucré}\end{définition}
\begin{définition}\pcmn{甜}\end{définition}
\begin{définition}\pfra{sucrer}\end{définition}
\begin{définition}\pcmn{放糖;使……变得更甜}\end{définition}
\begin{définition}\pfra{trouver sucré}\end{définition}
\begin{définition}\pcmn{觉得甜}\end{définition}
\begin{exemple}\pjya{ɲɯ-chi}\hspace{5pt}\pcmn{是甜的}\end{exemple}
\begin{exemple}\pjya{pɯ-ɣɤchi-t-a}\hspace{5pt}\pcmn{我放了糖}\end{exemple}\relationsémantique{反义词}{\lien{ⓔqiaβ}{qiaβ}}
\begin{sous-entrée}{ɣɤchi}{ⓔchiⓝɣɤchi} 
\classe{vt}  
\grammaire{caus} \end{sous-entrée}

\begin{sous-entrée}{sɯxchi}{ⓔchiⓝsɯxchi} 
\classe{vt}  
\grammaire{caus} 
\begin{définition}\pfra{sucrer}\end{définition}
\begin{définition}\pcmn{使……变得更甜}\end{définition}\end{sous-entrée}

\begin{sous-entrée}{nɤxchi}{ⓔchiⓝnɤxchi} 
\classe{vt}  
\grammaire{trop} \end{sous-entrée}

\end{entrée}

\begin{entrée}{choŋtɕɯn}{}{ⓔchoŋtɕɯn} 
\classe{n} 
\begin{définition}\pfra{grand oiseau mythologique}\end{définition}
\begin{définition}\pcmn{大鹏}\end{définition}\étymologie{kʰʲuŋ.tɕʰen}\end{entrée}

\begin{entrée}{chrɤβchrɤβ}{}{ⓔchrɤβchrɤβ} 
\classe{idph.2} 
\begin{définition}\pfra{sale et en désordre}\end{définition}
\begin{définition}\pcmn{没有打扫干净}\end{définition}
\begin{exemple}\pjya{nɤ-kha ra chrɤβchrɤβ ʑo to-tɯ-stu-t}\hspace{5pt}\pcmn{你的家没有打扫干净}\end{exemple}
\begin{exemple}\pjya{kha mɯ-thɯ́-wɣ-raʁrɯz tɕe, chrɤβchrɤβ ʑo pa}\hspace{5pt}\pcmn{家里萌芽扫干净,很脏}\end{exemple}
\begin{sous-entrée}{chrɤβnɤchrɤβ}{ⓔchrɤβchrɤβⓝchrɤβnɤchrɤβ} 
\classe{idph.3} 
\begin{exemple}\pjya{paʁtshi ɲɯ-ɤsɯ-ɕmi chrɤβnɤchrɤβ ɲɯ-ɤsɯ-stu}\hspace{5pt}\pcmn{他在搅拌猪食,很脏}\end{exemple}\relationsémantique{参考}{\lien{ⓔɣɤchrɤβchrɤβ}{ɣɤchrɤβchrɤβ}}\end{sous-entrée}

\end{entrée}

\begin{entrée}{chɯ}{}{ⓔchɯ} 
\classe{n}  
\grammaire{n.lieu} 
\begin{définition}\pfra{l'un des hameaux de Gyutshapa}\end{définition}
\begin{définition}\pcmn{二茶村的大队之一}\end{définition}\end{entrée}

\begin{entrée}{chɯβchɯβ/\variante{chɯpchɯp}}{}{ⓔchɯβchɯβ} 
\classe{idph.2} 
\begin{définition}\pfra{crasseux}\end{définition}
\begin{définition}\pcmn{很脏(地)}\end{définition}
\begin{exemple}\pjya{ɯ-thoʁ chɯpchɯp ʑo ɲɯ-pa}\hspace{5pt}\pcmn{地很脏(到处都有赃物,没有扫干净)}\end{exemple}
\begin{exemple}\pjya{kha nɯ chɯpchɯp ma-kɤ-tɯ-ste}\hspace{5pt}\pcmn{你不要把家里弄得那么乱}\end{exemple}
\begin{exemple}\pjya{tɕhaχɯ ɲɯ-ɲaʁ chɯβchɯβ ʑo ɲɯ-pa}\hspace{5pt}\pcmn{茶壶又脏又黑的样子}\end{exemple}
\begin{sous-entrée}{chɯβ}{ⓔchɯβchɯβⓝchɯβ} 
\classe{idph.1} 
\begin{définition}\pfra{bruit d'objet qui se casse}\end{définition}
\begin{définition}\pcmn{东西折断的声音(咔嚓)}\end{définition}
\begin{exemple}\pjya{chɯβ ʑo pa-qlɯt}\hspace{5pt}\pcmn{他咔嚓一声就弄断了}\end{exemple}
\begin{exemple}\pjya{chɯβ ʑo pɯ-ɴɢlɯt}\hspace{5pt}\pcmn{咔嚓一声就断了}\end{exemple}\end{sous-entrée}

\begin{sous-entrée}{chɯβnɤchɯβ}{ⓔchɯβchɯβⓝchɯβnɤchɯβ}
\begin{définition}\pfra{manger à grande bouchées}\end{définition}
\begin{définition}\pcmn{大口大口地吃}\end{définition}
\begin{exemple}\pjya{paʁ chɯβnɤchɯβ ʑo ɲɯ-rɯndzɤtshi}\hspace{5pt}\pcmn{猪大口大口地吃}\end{exemple}\end{sous-entrée}

\end{entrée}

\begin{entrée}{chɯchɯβ}{}{ⓔchɯchɯβ} 
\classe{idph.2} 
\begin{définition}\pfra{découpé n'importe comment en petits morceaux}\end{définition}
\begin{définition}\pcmn{形容(砍)得稀烂的样子}\end{définition}
\begin{exemple}\pjya{si ra chɯchɯβ ʑo to-ta (=to-lɤt)}\hspace{5pt}\pcmn{他把柴砍得稀烂}\end{exemple}\end{entrée}

\begin{entrée}{chɯmu}{}{ⓔchɯmu} 
\classe{n} 
\begin{définition}\pfra{chienne}\end{définition}
\begin{définition}\pcmn{母狗}\end{définition}\étymologie{kʰʲi.mo}\end{entrée}

\begin{entrée}{chɯrdom}{}{ⓔchɯrdom} 
\classe{n} 
\begin{définition}\pfra{chien vagabond}\end{définition}
\begin{définition}\pcmn{流浪狗}\end{définition}\end{entrée}

\begin{entrée}{chɯsɲu}{}{ⓔchɯsɲu} 
\classe{n} 
\begin{définition}\pfra{rage}\end{définition}
\begin{définition}\pcmn{狂犬病}\end{définition}
\begin{exemple}\pjya{chɯsɲu ɲɤ-k-ɤtɯɣ-ci}\hspace{5pt}\pcmn{它得了狂犬病}\end{exemple}\étymologie{kʰʲi.smʲo}\end{entrée}

\begin{entrée}{ci}{₂}{ⓔciⓗ2} 
\classe{num} \sens{1}
\begin{définition}\pfra{un}\end{définition}
\begin{définition}\pcmn{一}\end{définition}\relationsémantique{同义词}{\lien{ⓔtɤɣ}{tɤɣ}}\sens{2}
\begin{définition}\pfra{indéfini}\end{définition}
\begin{définition}\pcmn{不定}\end{définition}
\begin{exemple}\pjya{kɯɕɯŋgɯ tɕe ʁzɤmi ci pjɤ-tu-ndʑi}\hspace{5pt}\pcmn{从前,有一对夫妻}\end{exemple}
\begin{exemple}\pjya{ci ɯ-mphru ci ʑo}\hspace{5pt}\pcmn{一个接着一个}\end{exemple}\sens{3}
\begin{définition}\pfra{un peu}\end{définition}
\begin{définition}\pcmn{一点}\end{définition}\sens{4}
\begin{définition}\pfra{une fois}\end{définition}
\begin{définition}\pcmn{一下}\end{définition}
\begin{exemple}\pjya{li ci tɤ-ti}\end{exemple}
\begin{sous-entrée}{ci nɯ}{ⓔciⓗ2ⓢ4ⓝci nɯ} 
\classe{pro} 
\begin{définition}\pfra{l'autre (dont on a déjà parlé)}\end{définition}
\begin{définition}\pcmn{另外那个}\end{définition}
\begin{exemple}\pjya{ci nɯ kɯ nɯra to-nɯ-ndo qhe nɯ jo-nɯ-ɕe}\hspace{5pt}\pcmn{另外那个(人)把那些(东西)拿走了}\end{exemple}\end{sous-entrée}

\begin{sous-entrée}{ci nɤ}{ⓔciⓗ2ⓢ4ⓝci nɤ}
\begin{définition}\pfra{pas (même) un}\end{définition}
\begin{définition}\pcmn{连……也}\end{définition}
\begin{exemple}\pjya{ɯ-kɤrme tɯ-ldʑa ci nɤ a-mɤ-jɤ-tɯ-ɕɣɤz ma mɤ-jɤɣ}\hspace{5pt}\pcmn{连一根头发都不要带回来}\end{exemple}\end{sous-entrée}

\begin{sous-entrée}{ci ntsɯ}{ⓔciⓗ2ⓢ4ⓝci ntsɯ}\sens{1}
\begin{définition}\pfra{chacun (sans exception)}\end{définition}
\begin{définition}\pcmn{一个也不漏}\end{définition}
\begin{exemple}\pjya{khɯtsa tɯ-rdoʁ tɕe ci ntsɯ pjɤ-ta, maka mɯ-ɲɤ-sɤtɕɯtɕit}\hspace{5pt}\pcmn{每一碗都放了一块,一个也没有漏掉}\end{exemple}\end{sous-entrée}

\sens{2}
\begin{définition}\pfra{chaque fois}\end{définition}
\begin{définition}\pcmn{每一次}\end{définition}
\begin{exemple}\pjya{ci a-mɤ-ɕ-tɤ-ru tɕe, ci ntsɯ tu-dɤn pjɤ-ŋgrɤl}\hspace{5pt}\pcmn{只要少看一次,就会变得多一点}\end{exemple}
\begin{sous-entrée}{ci ci}{ⓔciⓗ2ⓢ2ⓝci ci}\sens{1}
\begin{définition}\pfra{certains}\end{définition}
\begin{définition}\pcmn{有些}\end{définition}\end{sous-entrée}

\sens{2}
\begin{définition}\pfra{parfois}\end{définition}
\begin{définition}\pcmn{有时候}\end{définition}
\begin{exemple}\pjya{tɯ-mɯ ci ci ku-lɤt, ci ci mɯ́j-lɤt}\hspace{5pt}\pcmn{偶尔下雨,偶尔不下}\end{exemple}\end{entrée}

\begin{entrée}{ci}{₁}{ⓔciⓗ1} 
\classe{vt} \paradigme{dir}{pɯ-}
\begin{définition}\pfra{verser complètement d'un récipient à l'autre}\end{définition}
\begin{définition}\pcmn{完全倒出来}\end{définition}
\begin{exemple}\pjya{ki tɯ-rdoʁ ra pɯ-ci-t-a}\hspace{5pt}\pcmn{我把这些粮食装进了(那个口袋)}\end{exemple}
\begin{exemple}\pjya{a-tʂha ɲɯ-sɤɕke tɕe, tsuku pɯ-ci}\hspace{5pt}\pcmn{我的茶很烫,倒一点出来}\end{exemple}\relationsémantique{同义词}{\lien{ⓔphoʁ}{phoʁ}}\end{entrée}

\begin{entrée}{cischiz}{}{ⓔcischiz} 
\classe{pro} 
\begin{définition}\pfra{n'importe où, à un certain endroit}\end{définition}
\begin{définition}\pcmn{在某个地方,随便在哪个地方}\end{définition}\end{entrée}

\begin{entrée}{cit}{}{ⓔcit} 
\classe{vi} \paradigme{dir}{jɤ-}
\begin{définition}\pfra{bouger}\end{définition}
\begin{définition}\pcmn{移动;搬迁}\end{définition}
\begin{exemple}\pjya{jɤ-cit-a, ɯʑo jɤ-cit}\hspace{5pt}\pcmn{我搬了,他搬了}\end{exemple}
\begin{exemple}\pjya{kha ta-βzu-nɯ tɕe, ju-cit-i ra}\hspace{5pt}\pcmn{他们修了房子,我们要搬}\end{exemple}\relationsémantique{同义词}{\lien{ⓔspɤr}{spɤr}}\relationsémantique{参考}{\lien{ⓔnɯqhɤcit}{nɯqhɤcit}}\end{entrée}

\begin{entrée}{ciz tɕe}{}{ⓔciz tɕe} 
\classe{adv} 
\begin{définition}\pfra{à un moment futur}\end{définition}
\begin{définition}\pcmn{总有一天(未来)}\end{définition}
\begin{exemple}\pjya{ɯʑo ciz tɕe ɣi ɕti}\hspace{5pt}\pcmn{他总会要来的}\end{exemple}
\begin{exemple}\pjya{aʑo ciz tɕe ɕɯ-rtoʁ-a ra}\hspace{5pt}\pcmn{我总有一天会去看他}\end{exemple}\end{entrée}

\begin{entrée}{claŋclaŋ}{}{ⓔclaŋclaŋ} 
\classe{idph.2} 
\begin{définition}\pfra{rond et lisse}\end{définition}
\begin{définition}\pcmn{形容又圆又光滑的样子(油光发亮)}\end{définition}
\begin{exemple}\pjya{ɯ-ku pjɤ-nɯ-sɯ-qrɤz tɕe, claŋclaŋ ʑo ɲɯ-pa}\hspace{5pt}\pcmn{他理了发,头光滑得发亮}\end{exemple}
\begin{sous-entrée}{claŋnɤclaŋ}{ⓔclaŋclaŋⓝclaŋnɤclaŋ} 
\classe{idph.3} 
\begin{exemple}\pjya{claŋnɤclaŋ kɤ-ari}\hspace{5pt}\pcmn{他(头光溜得发亮)地去了}\end{exemple}\relationsémantique{参考}{\lien{ⓔɕlaŋɕlaŋ}{ɕlaŋɕlaŋ}}\relationsémantique{参考}{\lien{ⓔslaŋslaŋ}{slaŋslaŋ}}\relationsémantique{参考}{\lien{ⓔɕliɕli}{ɕliɕli}}\end{sous-entrée}

\end{entrée}

\begin{entrée}{claʁclaʁ}{}{ⓔclaʁclaʁ} 
\classe{idph.2} 
\begin{définition}\pfra{brillant}\end{définition}
\begin{définition}\pcmn{形容闪亮耀眼的样子}\end{définition}
\begin{exemple}\pjya{rgali wuma ʑo ɯ-ɲɤm ɲɯ-sna tɕe, ɲɯ-nɤmbju ʑo claʁclaʁ}\hspace{5pt}\pcmn{小奶牛膘满肉肥,皮毛闪闪发亮}\end{exemple}\end{entrée}

\begin{entrée}{clicli}{}{ⓔclicli} 
\classe{idph.2} 
\begin{définition}\pfra{noir, lisse et brillant}\end{définition}
\begin{définition}\pcmn{形容脑袋光溜溜又黑的模样}\end{définition}\relationsémantique{参考}{\lien{ⓔclɯclɯ}{clɯclɯ}}\end{entrée}

\begin{entrée}{cloʁcloʁ}{}{ⓔcloʁcloʁ} 
\classe{idph.2} 
\begin{définition}\pfra{noir brillant}\end{définition}
\begin{définition}\pcmn{形容耀眼的黑色}\end{définition}
\begin{exemple}\pjya{tɯrme ra tɤŋe ɯ-ŋgɯ ntsɯ ɲɯ-rɤma-nɯ tɕe ɲɯ-ɲaʁ-nɯ cloʁcloʁ ʑo ɲɯ-ŋu}\hspace{5pt}\pcmn{人们长期在太阳底下劳动,皮肤就变得黑油油的}\end{exemple}
\begin{exemple}\pjya{ko-naʁri cloʁcloʁ ʑo}\hspace{5pt}\pcmn{这个东西上面有油迹,变得黑油油的}\end{exemple}\end{entrée}

\begin{entrée}{clɯclɯ}{}{ⓔclɯclɯ} 
\classe{idph.2} 
\begin{définition}\pfra{noir et chauve}\end{définition}
\begin{définition}\pcmn{形容脑袋光溜溜又黑的模样}\end{définition}
\begin{exemple}\pjya{jiɕqha tɤ-rɟit nɯ clɯclɯ ɲɯ-ɲaʁ}\hspace{5pt}\pcmn{小孩子,脸是黑的,又是光头}\end{exemple}
\begin{sous-entrée}{clɯnɤclɯ}{ⓔclɯclɯⓝclɯnɤclɯ} 
\classe{idph.3} 
\begin{exemple}\pjya{clɯnɤclɯ ɲɯ-ŋke}\hspace{5pt}\pcmn{脸黑、光头(的小孩子)在走}\end{exemple}\relationsémantique{参考}{\lien{ⓔclicli}{clicli}}\end{sous-entrée}

\end{entrée}

\begin{entrée}{co}{}{ⓔco} 
\classe{n} 
\begin{définition}\pfra{vallée}\end{définition}
\begin{définition}\pcmn{沟}\end{définition}
\begin{sous-entrée}{tɯ-co}{ⓔcoⓝtɯ-co} 
\classe{clf} 
\begin{définition}\pfra{toute la vallée}\end{définition}
\begin{définition}\pcmn{整条沟}\end{définition}
\begin{exemple}\pjya{mbarkhom rcanɯ, kha tɯ-co ʑo ɣɤʑu}\hspace{5pt}\pcmn{马尔康整条沟都是房子}\end{exemple}\end{sous-entrée}

\end{entrée}

\begin{entrée}{coɕit}{}{ⓔcoɕit} 
\classe{n} 
\begin{définition}\pfra{vallée}\end{définition}
\begin{définition}\pcmn{山谷}\end{définition}\end{entrée}

\begin{entrée}{comuco}{}{ⓔcomuco} 
\classe{n} 
\begin{définition}\pfra{Kyomkyo}\end{définition}
\begin{définition}\pcmn{脚木足}\end{définition}\end{entrée}

\begin{entrée}{coŋ}{₁}{ⓔcoŋⓗ1} 
\classe{n} 
\begin{définition}\pfra{perte, dommage}\end{définition}
\begin{définition}\pcmn{损失}\end{définition}
\begin{exemple}\pjya{a-coŋ pɯ-ɣe}\hspace{5pt}\pcmn{我受损了}\end{exemple}
\begin{exemple}\pjya{coŋ wuma ʑo ɲɯ-wxti}\hspace{5pt}\pcmn{损失很惨重}\end{exemple}\relationsémantique{参考}{\lien{ⓔcoŋⓗ2}{coŋ₂}}\end{entrée}

\begin{entrée}{coŋ}{₂}{ⓔcoŋⓗ2} 
\classe{vi} \paradigme{dir}{pɯ-}
\begin{définition}\pfra{subir des dommages}\end{définition}
\begin{définition}\pcmn{受损}\end{définition}
\begin{exemple}\pjya{pɯ-coŋ-a}\hspace{5pt}\pcmn{我受损了}\end{exemple}\end{entrée}

\begin{entrée}{cotcot}{}{ⓔcotcot} 
\classe{idph.2} 
\begin{définition}\pfra{petit et mignon}\end{définition}
\begin{définition}\pcmn{形容又小又可爱的样子}\end{définition}
\begin{exemple}\pjya{wɯɟa cotcot ɲɯ-pa}\hspace{5pt}\pcmn{调羹很小}\end{exemple}
\begin{exemple}\pjya{tɤmɤru cotcot ʑo ɲɯ-pa}\hspace{5pt}\pcmn{脚印很小}\end{exemple}
\begin{exemple}\pjya{karkɯm cotcot ʑo to-ɬoʁ}\hspace{5pt}\pcmn{圆根的子叶很小}\end{exemple}
\begin{exemple}\pjya{tɤ-pɤtso ɯ-jaʁ cotcot ʑo ko-ɕthɯz}\hspace{5pt}\pcmn{小孩子伸了手,很可爱}\end{exemple}\end{entrée}

\begin{entrée}{cupa}{}{ⓔcupa} 
\classe{n} 
\begin{définition}\pfra{ardoise}\end{définition}
\begin{définition}\pcmn{石板}\end{définition}\end{entrée}

\begin{entrée}{craŋ}{}{ⓔcraŋ} 
\classe{idph.1} 
\begin{définition}\pfra{bruit du verre qui se brise}\end{définition}
\begin{définition}\pcmn{玻璃被砸碎的声音}\end{définition}
\begin{sous-entrée}{craŋnɤlaŋ}{ⓔcraŋⓝcraŋnɤlaŋ} 
\classe{idph.4} 
\begin{exemple}\pjya{craŋnɤlaŋ ɲɯ-ɤkhɤzŋga}\hspace{5pt}\pcmn{他大声地喊}\end{exemple}\relationsémantique{参考}{\lien{ⓔɣɤcraŋlaŋ}{ɣɤcraŋlaŋ}}\end{sous-entrée}

\end{entrée}

\begin{entrée}{crɯβcrɯβ}{}{ⓔcrɯβcrɯβ} 
\classe{idph.2}
\classe{idph.2} 
\begin{définition}\pfra{brisé en mille morceaux}\end{définition}
\begin{définition}\pcmn{玻璃被砸碎}\end{définition}
\begin{exemple}\pjya{χɕɤl crɯβcrɯβ ʑo pjɤ-ɴɢrɯ}\hspace{5pt}\pcmn{玻璃砸得粉碎}\end{exemple}\relationsémantique{参考}{\lien{ⓔcrɯmcrɯm}{crɯmcrɯm}}\relationsémantique{参考}{\lien{ⓔcɯmcɯm}{cɯmcɯm}}\end{entrée}

\begin{entrée}{crɯcrɯ}{}{ⓔcrɯcrɯ} 
\classe{idph.2} 
\begin{définition}\pfra{sale, crasseux et plein de morve}\end{définition}
\begin{définition}\pcmn{形容又脏又湿,吊着鼻涕的样子}\end{définition}
\begin{exemple}\pjya{ki tɤ-pɤtso pjɯ́-wɣ-zraχtɕi ɲɯ-ra ma crɯcrɯ ʑo ɲɯ-pa}\hspace{5pt}\pcmn{要给这个小孩子洗澡,他很脏的样子}\end{exemple}\end{entrée}

\begin{entrée}{crɯɣcrɯɣ}{}{ⓔcrɯɣcrɯɣ} 
\classe{idph.2} 
\begin{définition}\pfra{en désordre}\end{définition}
\begin{définition}\pcmn{形容乱七八糟的样子(很多东西)}\end{définition}
\begin{exemple}\pjya{laχtɕha crɯɣcrɯɣ ʑo ko-rmbɯ}\hspace{5pt}\pcmn{他把东西得乱七八糟}\end{exemple}\relationsémantique{参考}{\lien{ⓔɟrɯɣɟrɯɣ}{ɟrɯɣɟrɯɣ}}
\begin{sous-entrée}{ɣɤcrɯɣlɯɣ}{ⓔcrɯɣcrɯɣⓝɣɤcrɯɣlɯɣ} 
\classe{vi} 
\begin{définition}\pfra{beaucoup d'objets en désordre qui s'entrechoquent}\end{définition}
\begin{définition}\pcmn{很多硬的乱七八糟的东西相碰发出声音}\end{définition}\end{sous-entrée}

\end{entrée}

\begin{entrée}{crɯmcrɯm}{}{ⓔcrɯmcrɯm} 
\classe{idph.2}
\classe{idph.2} 
\begin{définition}\pfra{brisé en mille morceaux}\end{définition}
\begin{définition}\pcmn{玻璃被砸碎}\end{définition}
\begin{exemple}\pjya{χɕɤl crɯmcrɯm ʑo pjɤ-ɴɢrɯ}\hspace{5pt}\pcmn{玻璃砸得粉碎}\end{exemple}\relationsémantique{参考}{\lien{ⓔcrɯβcrɯβ}{crɯβcrɯβ}}\relationsémantique{参考}{\lien{ⓔcɯmcɯm}{cɯmcɯm}}\end{entrée}

\begin{entrée}{cɯ}{₃}{ⓔcɯⓗ3} 
\classe{n} 
\begin{définition}\pfra{pierre}\end{définition}
\begin{définition}\pcmn{石头}\end{définition}\end{entrée}

\begin{entrée}{cɯ}{₁}{ⓔcɯⓗ1} 
\classe{vt} \sens{1}\paradigme{dir}{tɤ-}
\begin{définition}\pfra{ouvrir}\end{définition}
\begin{définition}\pcmn{打开}\end{définition}
\begin{exemple}\pjya{kɯm tɤ-cɯ-t-a}\hspace{5pt}\pcmn{我开了门}\end{exemple}
\begin{exemple}\pjya{kɯm tɤ-ci}\hspace{5pt}\pcmn{你开门吧}\end{exemple}\sens{2}
\begin{définition}\pfra{laisser le passage}\end{définition}
\begin{définition}\pcmn{让(路)}\end{définition}
\begin{exemple}\pjya{tʂu tɤ-cɯ-t-a}\hspace{5pt}\pcmn{我让了路}\end{exemple}\sens{3}
\begin{définition}\pfra{ouvrir les yeux}\end{définition}
\begin{définition}\pcmn{睁开}\end{définition}
\begin{exemple}\pjya{ɯ-mɲaʁ ɲɤ-cɯ}\hspace{5pt}\pcmn{他睁开了眼睛}\end{exemple}\relationsémantique{参考}{\lien{ⓔɲɟɯ}{ɲɟɯ}}\relationsémantique{参考}{\lien{ⓔɯ-lu,cɯ}{ɯ-lu,cɯ}}
\begin{sous-entrée}{tɯ-rnoʁ,cɯ}{ⓔcɯⓗ1ⓢ3ⓝtɯ-rnoʁ,cɯ}
\begin{définition}\pfra{assourdir}\end{définition}
\begin{définition}\pcmn{震耳欲聋}\end{définition}
\begin{exemple}\pjya{a-rnoʁ ʑo ku-ci ɲɯ-ɕti, tɕe kɤ-sɤŋo mɯ́j-khɯ}\hspace{5pt}\pcmn{很吵,无法听清楚}\end{exemple}
\begin{exemple}\pjya{ma-tɯ-ɤrju ma a-rnoʁ ma-tɯ-ci ma ɲɯ-sɤɣdɯɣ}\hspace{5pt}\pcmn{你不要说了,不要吵我}\end{exemple}\end{sous-entrée}

\end{entrée}

\begin{entrée}{cɯ}{₂}{ⓔcɯⓗ2} 
\classe{vi} \paradigme{dir}{kɤ-}
\begin{définition}\pfra{hiberner}\end{définition}
\begin{définition}\pcmn{冬眠}\end{définition}
\begin{exemple}\pjya{rɯdaʁ ra qartsɯ tɕe ku-cɯ-nɯ ŋu}\hspace{5pt}\pcmn{动物冬天的时候冬眠}\end{exemple}
\begin{exemple}\pjya{qajɯ kɯnɤ ku-cɯ-nɯ ŋu ŋgrɤl}\hspace{5pt}\pcmn{虫子也冬眠}\end{exemple}
\begin{exemple}\pjya{χɕitka jɤ-ɣe tɕe chɯ-ɬoʁ-nɯ ŋu, qartsɯ tɕe ku-cɯ-nɯ ŋu}\hspace{5pt}\pcmn{春天的时候出来,冬天的时候冬眠}\end{exemple}\end{entrée}

\begin{entrée}{cɯβjiz}{}{ⓔcɯβjiz} 
\classe{n} 
\begin{définition}\pfra{plaque de pierre}\end{définition}
\begin{définition}\pcmn{石板}\end{définition}\end{entrée}

\begin{entrée}{cɯβloʁ}{}{ⓔcɯβloʁ} 
\classe{n} 
\begin{définition}\pfra{flaque d'eau}\end{définition}
\begin{définition}\pcmn{水洼;水坑}\end{définition}
\begin{exemple}\pjya{cɯβloʁ ɯ-ŋgɯ qaɕpa ɣɤʑu}\hspace{5pt}\pcmn{在水坑里有个青蛙}\end{exemple}\end{entrée}

\begin{entrée}{cɯɣɬaj}{}{ⓔcɯɣɬaj} 
\classe{n} 
\begin{définition}\pfra{symptôme dans lequel la muqueuse buccale devient blanche}\end{définition}
\begin{définition}\pcmn{口腔内膜变白(症状)}\end{définition}
\begin{exemple}\pjya{ɯ-kɯr cɯɣɬaj ɲɤ-xtsu}\hspace{5pt}\pcmn{嘴巴里发白了}\end{exemple}\end{entrée}

\begin{entrée}{cɯm}{}{ⓔcɯm} 
\classe{vt} \paradigme{dir}{kɤ-}
\begin{définition}\pfra{retourner vers l'intérieur (les bords d'un tissu découpé avec une paire de ciseaux)}\end{définition}
\begin{définition}\pcmn{把(用剪刀剪过的)布料的边缘往里面折进去}\end{définition}
\begin{exemple}\pjya{tɯ-ŋga ɯ-tɕhɤz nɯ nɯ-ʁndzar-a tɕe, ɯ-rtsho nɯ kɤ-cɯm-a}\hspace{5pt}\pcmn{我把衣服的吊边布剪出来,然后把边缘折进去缝起来}\end{exemple}
\begin{sous-entrée}{nɯɣɯcɯm}{ⓔcɯmⓝnɯɣɯcɯm} 
\classe{vs} 
\begin{définition}\pfra{facile à retourner vers l'intérieur (d'un tissu)}\end{définition}
\begin{définition}\pcmn{容易折进去}\end{définition}
\begin{exemple}\pjya{tɯ-ŋga kɯ-mba nɯra nɯɣɯcɯm, kɯ-jaʁ nɯra mɤ-nɯɣɯcɯm}\hspace{5pt}\pcmn{布料薄的衣服容易折进去,布料厚的衣服就不容易折进去}\end{exemple}\end{sous-entrée}

\end{entrée}

\begin{entrée}{cɯmbɤrom}{}{ⓔcɯmbɤrom} 
\classe{n} 
\begin{définition}\pfra{ampoule}\end{définition}
\begin{définition}\pcmn{水疱}\end{définition}
\begin{exemple}\pjya{aʑo a-jaʁ pɯ-ɕke tɕe, cɯmbɤrom to-rku}\hspace{5pt}\pcmn{我烫伤了手就出现了水泡}\end{exemple}\relationsémantique{参考}{\lien{}{tɕhɯwur}}\end{entrée}

\begin{entrée}{cɯmcɯm}{}{ⓔcɯmcɯm} 
\classe{idph.2}
\classe{idph.2} 
\begin{définition}\pfra{brisé en mille morceaux}\end{définition}
\begin{définition}\pcmn{玻璃被砸碎}\end{définition}
\begin{exemple}\pjya{χɕɤl pjɤ-xtsɯ cɯmcɯm ʑo}\hspace{5pt}\pcmn{玻璃砸得粉碎}\end{exemple}\relationsémantique{参考}{\lien{ⓔcrɯmcrɯm}{crɯmcrɯm}}\relationsémantique{参考}{\lien{ⓔcrɯβcrɯβ}{crɯβcrɯβ}}\end{entrée}

\begin{entrée}{cɯnkhɤβ}{}{ⓔcɯnkhɤβ} 
\classe{n} 
\begin{définition}\pfra{satin}\end{définition}
\begin{définition}\pcmn{缎子}\end{définition}\relationsémantique{参考}{\lien{ⓔcɯrzɯn}{cɯrzɯn}}\end{entrée}

\begin{entrée}{cɯnmu}{}{ⓔcɯnmu} 
\classe{n} 
\begin{définition}\pfra{rumeurs, zizanie}\end{définition}
\begin{définition}\pcmn{谣言(挑拨离间)}\end{définition}
\begin{exemple}\pjya{cɯnmu a-mɤ-tɤ-tɯ-fkri (=a-mɤ-tɤ-tɯ-βze)}\hspace{5pt}\pcmn{你不要挑拨离间}\end{exemple}\relationsémantique{参考}{\lien{ⓔrɯcɯnmu}{rɯcɯnmu}}\end{entrée}

\begin{entrée}{cɯŋcɯŋ}{}{ⓔcɯŋcɯŋ} 
\classe{idph.2} 
\begin{définition}\pfra{radieux, ensoleillé}\end{définition}
\begin{définition}\pcmn{阳光强烈,普照大地}\end{définition}
\begin{exemple}\pjya{tɤŋe cɯŋcɯŋ ʑo pjɤ-zɣɯt tɕe ɲɯ-sɤscit}\hspace{5pt}\pcmn{阳光很强烈,很舒服}\end{exemple}\end{entrée}

\begin{entrée}{cɯŋglɯɣ}{}{ⓔcɯŋglɯɣ} 
\classe{n} 
\begin{définition}\pfra{pilon}\end{définition}
\begin{définition}\pcmn{杵【捶捶】}\end{définition}\end{entrée}

\begin{entrée}{cɯpɤspoʁspoʁ}{}{ⓔcɯpɤspoʁspoʁ} 
\classe{n} 
\begin{définition}\pfra{jouet en forme de meule}\end{définition}
\begin{définition}\pcmn{水磨模型的玩具}\end{définition}\end{entrée}

\begin{entrée}{cɯphɯt}{}{ⓔcɯphɯt} 
\classe{n} 
\begin{définition}\pfra{ramassage de pierres}\end{définition}
\begin{définition}\pcmn{捡石头(庄稼地)}\end{définition}
\begin{exemple}\pjya{cɯphɯt nɯ-βzu-t-a (=cɯ nɯ-phɯt-a, nɯ-ɣɯcɯphɯt-a)}\hspace{5pt}\pcmn{我捡了石头}\end{exemple}\relationsémantique{参考}{\lien{ⓔcɯⓗ3}{cɯ₃}}\relationsémantique{参考}{\lien{ⓔphɯt}{phɯt}}\relationsémantique{参考}{\lien{ⓔɣɯcɯphɯt}{ɣɯcɯphɯt}}\end{entrée}

\begin{entrée}{cɯrmbɯ}{}{ⓔcɯrmbɯ} 
\classe{n} 
\begin{définition}\pfra{tas de pierre}\end{définition}
\begin{définition}\pcmn{石头堆}\end{définition}\relationsémantique{参考}{\lien{ⓔcɯⓗ3}{cɯ₃}}\relationsémantique{参考}{\lien{ⓔtɯ-tɤrmbɯ}{tɯ-tɤrmbɯ}}\relationsémantique{参考}{\lien{ⓔrmbɯ}{rmbɯ}}\end{entrée}

\begin{entrée}{cɯrndʑi}{}{ⓔcɯrndʑi} 
\classe{n} 
\begin{définition}\pfra{sable}\end{définition}
\begin{définition}\pcmn{沙子}\end{définition}\end{entrée}

\begin{entrée}{cɯrzɯn}{}{ⓔcɯrzɯn} 
\classe{n} 
\begin{définition}\pfra{satin}\end{définition}
\begin{définition}\pcmn{缎子的一种}\end{définition}\relationsémantique{参考}{\lien{ⓔcɯnkhɤβ}{cɯnkhɤβ}}\end{entrée}

\begin{entrée}{cɯʁrɤt}{}{ⓔcɯʁrɤt} 
\classe{n} 
\begin{définition}\pfra{charbon; houille}\end{définition}
\begin{définition}\pcmn{煤}\end{définition}\end{entrée}

\begin{entrée}{cɯχɕiz}{}{ⓔcɯχɕiz} 
\classe{n} 
\begin{définition}\pfra{terre pierreuse}\end{définition}
\begin{définition}\pcmn{沙地}\end{définition}\relationsémantique{参考}{\lien{ⓔcɯⓗ3}{cɯ}}\end{entrée}

\newpage\caractère{ɕ}

\begin{entrée}{ɕu}{}{ⓔɕu} 
\classe{n} 
\begin{définition}\pfra{carte}\end{définition}
\begin{définition}\pcmn{牌戏}\end{définition}
\begin{exemple}\pjya{ɕu pɯ-lɤt-tɕi tɕe pɯ-ta-ɕɯnŋo}\hspace{5pt}\pcmn{我们俩打牌的时候,我把你打败了}\end{exemple}\relationsémantique{参考}{\lien{ⓔnɯɕu}{nɯɕu}}\étymologie{ɕo}\end{entrée}

\begin{entrée}{ɕa}{}{ⓔɕa} 
\classe{n} 
\begin{définition}\pfra{viande crue}\end{définition}
\begin{définition}\pcmn{生肉}\end{définition}\étymologie{ɕa}\end{entrée}

\begin{entrée}{ɕaβ}{₁}{ⓔɕaβⓗ1} 
\classe{vt} \paradigme{dir}{jɤ-}
\begin{définition}\pfra{rattraper}\end{définition}
\begin{définition}\pcmn{赶上;追上}\end{définition}
\begin{exemple}\pjya{jɤ-ɕaβ-a, jɤ-tɯ-ɕaβ, ja-ɕaβ}\hspace{5pt}\pcmn{我追上了他,你追上了他,他追上了他}\end{exemple}
\begin{exemple}\pjya{a-wɯ a-wi ra mɯ-tɤ-ɕaβ-a}\hspace{5pt}\pcmn{我们出生的时候我的爷爷奶奶已经去世了}\end{exemple}
\begin{sous-entrée}{sɤɕaβ}{ⓔɕaβⓗ1ⓝsɤɕaβ} 
\classe{vi}  
\grammaire{apass} 
\begin{exemple}\pjya{nɯnɯ laχtɕha nɯ kɤ-mɟa mɯ́j-sɤɕaβ mɤ ɲɯ-ɤrqhi, ɲɯ-mbro}\hspace{5pt}\pcmn{这个东西拿不着,因为太远了(太高了)}\end{exemple}
\begin{exemple}\pjya{mɯ́j-sɤɕaβ}\hspace{5pt}\pcmn{赶不上}\end{exemple}\end{sous-entrée}

\end{entrée}

\begin{entrée}{ɕaβ}{₂}{ⓔɕaβⓗ2} 
\classe{vs} 
\begin{définition}\pfra{être assez long}\end{définition}
\begin{définition}\pcmn{够长(可以连接起来)}\end{définition}
\begin{exemple}\pjya{tɤ-ri ɲɯ-ɕaβ}\hspace{5pt}\pcmn{线够长}\end{exemple}
\begin{exemple}\pjya{tɯmbri ɲɯ-ɕaβ}\hspace{5pt}\pcmn{绳子够长}\end{exemple}
\begin{exemple}\pjya{ɯ-tɯ-rɲɟi ɲɯ-ɕaβ}\hspace{5pt}\pcmn{够长}\end{exemple}\end{entrée}

\begin{entrée}{ɕamɤr}{}{ⓔɕamɤr} 
\classe{n} 
\begin{définition}\pfra{une espèce de cerf}\end{définition}
\begin{définition}\pcmn{一种鹿}\end{définition}\end{entrée}

\begin{entrée}{ɕaɲaʁ}{}{ⓔɕaɲaʁ} 
\classe{n} 
\begin{définition}\pfra{une espèce de cerf}\end{définition}
\begin{définition}\pcmn{一种黑色的鹿}\end{définition}\end{entrée}

\begin{entrée}{ɕaŋ}{}{ⓔɕaŋ} 
\classe{vs} \paradigme{dir}{tɤ-}
\begin{définition}\pfra{vieillir (corne d'un cerf)}\end{définition}
\begin{définition}\pcmn{老(鹿角)}\end{définition}
\begin{exemple}\pjya{ɯ-ʁrɯ to-ɕaŋ}\hspace{5pt}\pcmn{它的角老了}\end{exemple}\end{entrée}

\begin{entrée}{ɕaŋβli}{}{ⓔɕaŋβli} 
\classe{n} 
\begin{définition}\pfra{pousse d'arbre}\end{définition}
\begin{définition}\pcmn{树苗}\end{définition}\relationsémantique{参考}{\lien{}{tɯβli}}\étymologie{ɕiŋ}\end{entrée}

\begin{entrée}{ɕaŋdi}{}{ⓔɕaŋdi} 
\classe{adv} 
\begin{définition}\pfra{d'ici jusqu'à là-bas}\end{définition}
\begin{définition}\pcmn{从这边直到那边(西边)}\end{définition}\end{entrée}

\begin{entrée}{ɕaŋkɯ}{}{ⓔɕaŋkɯ} 
\classe{adv} 
\begin{définition}\pfra{d'ici jusqu'à là-bas}\end{définition}
\begin{définition}\pcmn{从这边直到那边(东边)}\end{définition}
\begin{exemple}\pjya{ki ɕaŋdi mbarkhom sɤtɕha ŋu, ki ɕaŋkɯ tɕɯχtsi sɤtɕha ŋu}\hspace{5pt}\pcmn{从这里到边就是马尔康的地区,这里到那边就是卓克基的地区}\end{exemple}\end{entrée}

\begin{entrée}{ɕaŋlo}{₁₂}{ⓔɕaŋloⓗ1ⓗ2} 
\classe{adv}
\classe{n} 
\begin{définition}\pfra{d'ici vers l'amont}\end{définition}
\begin{définition}\pcmn{从这里一直往上游}\end{définition}
\begin{sous-entrée}{ɕaŋlo}{ⓔɕaŋloⓗ1ⓝɕaŋlo}\end{sous-entrée}

\begin{définition}\pfra{place des anciens, au sud}\end{définition}
\begin{définition}\pcmn{老年人坐的地方;往南方}\end{définition}\end{entrée}

\begin{entrée}{ɕaŋpa}{}{ⓔɕaŋpa} 
\classe{adv} 
\begin{définition}\pfra{d'ici jusqu'en bas}\end{définition}
\begin{définition}\pcmn{从这里一直往下 ;以下}\end{définition}\end{entrée}

\begin{entrée}{ɕaŋpɕi}{}{ⓔɕaŋpɕi} 
\classe{postp} 
\begin{définition}\pfra{à partir de ce moment}\end{définition}
\begin{définition}\pcmn{从那以后}\end{définition}\étymologie{pʰʲis}\end{entrée}

\begin{entrée}{ɕaŋpin}{}{ⓔɕaŋpin} 
\classe{n} 
\begin{définition}\pfra{bord des habits tibétains}\end{définition}
\begin{définition}\pcmn{(女式藏装)缝在外面的吊边}\end{définition}\end{entrée}

\begin{entrée}{ɕaŋtaʁ}{}{ⓔɕaŋtaʁ} 
\classe{adv} 
\begin{définition}\pfra{au plus}\end{définition}
\begin{définition}\pcmn{最多;以上}\end{définition}
\begin{exemple}\pjya{tɯ-sŋi tɯ-ɣjɤn ɕaŋtaʁ smɤn kɤ-ndza mɯ́j-ra}\hspace{5pt}\pcmn{一天不要吃药超过一次}\end{exemple}\end{entrée}

\begin{entrée}{ɕaŋthi}{}{ⓔɕaŋthi} 
\classe{adv} 
\begin{définition}\pfra{d'ici vers l'aval}\end{définition}
\begin{définition}\pcmn{从这里一直往上游}\end{définition}\end{entrée}

\begin{entrée}{ɕaŋtoʁ}{}{ⓔɕaŋtoʁ} 
\classe{n} \sens{1}
\begin{définition}\pfra{fruit}\end{définition}
\begin{définition}\pcmn{果子}\end{définition}\relationsémantique{同义词}{\lien{ⓔsɯmat}{sɯmat}}\sens{2}
\begin{définition}\pfra{arbre fruitier}\end{définition}
\begin{définition}\pcmn{果树}\end{définition}\étymologie{ɕiŋ.tog}\end{entrée}

\begin{entrée}{ɕaŋɯ}{}{ⓔɕaŋɯ} 
\classe{n} 
\begin{définition}\pfra{période de chaleur des cerfs}\end{définition}
\begin{définition}\pcmn{鹿的发情期}\end{définition}\étymologie{ɕʷa.ŋu}\end{entrée}

\begin{entrée}{ɕaqajɯ}{}{ⓔɕaqajɯ} 
\classe{n} 
\begin{définition}\pfra{asticot}\end{définition}
\begin{définition}\pcmn{蛆}\end{définition}\relationsémantique{参考}{\lien{ⓔɕa}{ɕa}}\relationsémantique{参考}{\lien{ⓔqajɯ}{qajɯ}}\end{entrée}

\begin{entrée}{ɕar}{}{ⓔɕar} 
\classe{vt} \paradigme{dir}{nɯ-}\paradigme{dir}{pɯ-}
\begin{définition}\pfra{chercher}\end{définition}
\begin{définition}\pcmn{寻找}\end{définition}
\begin{exemple}\pjya{na-ɕar}\hspace{5pt}\pcmn{他找了}\end{exemple}
\begin{exemple}\pjya{kɯki kɯ-fse kɯ-pe kɤ-ɕar me}\hspace{5pt}\pcmn{找不到这么好的}\end{exemple}
\begin{exemple}\pjya{a-laχtɕha nɯ-nɯ-βde-t-a nɯ kɤ-ɕar mtam-a}\hspace{5pt}\pcmn{我会找到我丢失的东西}\end{exemple}
\begin{exemple}\pjya{nɯ-ɕar-a tɕe pɯ-mto-t-a}\hspace{5pt}\pcmn{我找到了}\end{exemple}
\begin{sous-entrée}{rɤɕar}{ⓔɕarⓝrɤɕar} 
\classe{vi}  
\grammaire{apass} 
\begin{définition}\pfra{chercher quelque chose}\end{définition}
\begin{définition}\pcmn{找东西}\end{définition}\end{sous-entrée}

\begin{sous-entrée}{sɤɕar}{ⓔɕarⓝsɤɕar} 
\classe{vi}  
\grammaire{apass} 
\begin{définition}\pfra{chercher quelqu'un}\end{définition}
\begin{définition}\pcmn{找人}\end{définition}\relationsémantique{参考}{\lien{ⓔnɤɕɯɕar}{nɤɕɯɕar}}\relationsémantique{参考}{\lien{ⓔnɤɕarlar}{nɤɕarlar}}\end{sous-entrée}

\end{entrée}

\begin{entrée}{ɕaʁja}{}{ⓔɕaʁja} 
\classe{intj} 
\begin{définition}\pfra{bien fait (pour lui)}\end{définition}
\begin{définition}\pcmn{活该}\end{définition}\end{entrée}

\begin{entrée}{ɕaʁwɯ}{}{ⓔɕaʁwɯ} 
\classe{n} 
\begin{définition}\pfra{navet séché}\end{définition}
\begin{définition}\pcmn{干了的芜菁根}\end{définition}\end{entrée}

\begin{entrée}{ɕaχpu}{}{ⓔɕaχpu} 
\classe{n} 
\begin{définition}\pfra{ami}\end{définition}
\begin{définition}\pcmn{朋友}\end{définition}
\begin{exemple}\pjya{a-ɕaχpu ŋu}\hspace{5pt}\pcmn{是我的朋友}\end{exemple}\étymologie{ɕag.po}\end{entrée}

\begin{entrée}{ɕɤci}{}{ⓔɕɤci} 
\classe{n} 
\begin{définition}\pfra{soupe}\end{définition}
\begin{définition}\pcmn{汤}\end{définition}\relationsémantique{参考}{\lien{ⓔɕa}{ɕa}}\relationsémantique{参考}{\lien{ⓔtɯ-ci}{tɯ-ci}}\end{entrée}

\begin{entrée}{ɕɤfɕo/\variante{ɕɤxɕo}}{}{ⓔɕɤfɕo} 
\classe{adv} 
\begin{définition}\pfra{ces derniers jours}\end{définition}
\begin{définition}\pcmn{这几天}\end{définition}\end{entrée}

\begin{entrée}{ɕɤɣ}{₂}{ⓔɕɤɣⓗ2} 
\classe{n} 
\begin{définition}\pfra{genévrier}\end{définition}
\begin{définition}\pcmn{柏树}\end{définition}
\begin{exemple}\pjya{ɕɤɣ nɯ si kɯ-jpum kɯ-mbro ci ŋu, ɯ-ru nɯ tɯrgi sthɯci mɤ-jpum, mɤ-mbro, ɯ-rtaʁ nɯ tɯrgi ɯ-rtaʁ sɤznɤ jpum cho rɲɟi, taʁ tɤ-ari ɯ-jija tu-xtɯt ŋu, ɕɤɣ tɯrgi sɤznɤ ɯ-rtaʁ dɤn, ɯ-jwaʁ nɯ alɯlju, aɣɯrtɯrtaʁ, ɯ-mdoʁ nɯ ldʑaŋsɤr ŋu. ɕɤɣ ɣɯ ɯ-ru nɯ li tɤrɤm wuma ʑo kɯ-ʑru kɤ-sɯ-pa ŋu ma ɯ-rɯmu tu, ɯ-mdoʁ mpɕɤr, ɕɤɣ tɯ-phɯ tɕu ɕoŋtɕa lɤβdɤlɤŋu-rzɯɣ ma kɤ-ʁndzɤr mɤ-rtaʁ. ɯ-jwaʁ nɯ tɤ-fsaŋ spa stu kɯ-ʑru ŋu, pjɯ́-wɣ-tɕɤβ tɕe, ɯ-dɯχɯn wuma mɯm. ɯ-rqhu rɕɯrɕɯβ ʑo pa, aɣɯrnɯɕɯr. qartsɯmɤftɕar ɯ-mdoʁ ɲɯ-nɤsci mɤ-cha.}\hspace{5pt}\pcmn{柏树是长得又粗又高的树种,树干没有杉树那么粗和高,枝桠比杉树的要粗和长,枝桠越是长在上面,就越短。柏树的枝桠比杉树的要多,叶子是圆柱形的,长很多枝桠,是淡绿色的,柏树也是制造木板的好材料,有条纹,颜色很美观。一棵柏树只能锯成四五节木料。叶子是烧香的最好的材料,烧了味道很香,树皮很粗,带有一点红色。颜色一年四季都不会变。}\end{exemple}\relationsémantique{参考}{\lien{ⓔnɯɕɤɣ}{nɯɕɤɣ}}\étymologie{ɕug}\end{entrée}

\begin{entrée}{ɕɤɣ}{₁}{ⓔɕɤɣⓗ1} 
\classe{vs} \paradigme{dir}{tɤ-}
\begin{définition}\pfra{nouveau}\end{définition}
\begin{définition}\pcmn{新}\end{définition}
\begin{exemple}\pjya{jiɕqha laχtɕha ɲɯ-ɕɤɣ}\hspace{5pt}\pcmn{那个东西是新的}\end{exemple}\relationsémantique{反义词}{\lien{ⓔmbe}{mbe}}\end{entrée}

\begin{entrée}{ɕɤɣpɣa}{}{ⓔɕɤɣpɣa} 
\classe{n} 
\begin{définition}\pfra{Turdus sp. (rubrocanus, kessleri)}\end{définition}
\begin{définition}\pcmn{鸫}\end{définition}\relationsémantique{参考}{\lien{ⓔɕɤɣⓗ2}{ɕɤɣ₂}}\relationsémantique{参考}{\lien{ⓔpɣa}{pɣa}}\end{entrée}

\begin{entrée}{ɕɤɣrum/\variante{ɕoʁɣrum}}{}{ⓔɕɤɣrum} 
\classe{n} 
\begin{définition}\pfra{espèce de sarrasin}\end{définition}
\begin{définition}\pcmn{甜荞}\end{définition}\relationsémantique{参考}{\lien{ⓔɕoʁ}{ɕoʁ}}\relationsémantique{参考}{\lien{ⓔwɣrum}{wɣrum}}\end{entrée}

\begin{entrée}{ɕɤjaʁ}{}{ⓔɕɤjaʁ} 
\classe{n} 
\begin{définition}\pfra{viande des membres antérieurs}\end{définition}
\begin{définition}\pcmn{牲畜前腿的肉}\end{définition}\relationsémantique{参考}{\lien{ⓔɕa}{ɕa}}\relationsémantique{参考}{\lien{ⓔtɯ-jaʁ}{tɯ-jaʁ}}\end{entrée}

\begin{entrée}{ɕɤku}{}{ⓔɕɤku} 
\classe{n} 
\begin{définition}\pfra{tête coupée}\end{définition}
\begin{définition}\pcmn{被砍掉的头}\end{définition}
\begin{exemple}\pjya{paʁ ɯ-ɕɤku}\hspace{5pt}\pcmn{(被砍掉的)猪头}\end{exemple}
\begin{exemple}\pjya{ɕɤku ɯ-ɕki ɕɤrna}\hspace{5pt}\pcmn{对着头说耳朵的坏话(打草惊蛇)}\end{exemple}\relationsémantique{参考}{\lien{ⓔtɯ-ku}{tɯ-ku}}\relationsémantique{参考}{\lien{ⓔɕɤjaʁ}{ɕɤjaʁ}}\relationsémantique{参考}{\lien{ⓔɕɤmi}{ɕɤmi}}\relationsémantique{参考}{\lien{}{ɕɤrn}}\end{entrée}

\begin{entrée}{ɕɤkhe}{}{ⓔɕɤkhe} 
\classe{n} 
\begin{définition}\pfra{viande maigre}\end{définition}
\begin{définition}\pcmn{瘦肉}\end{définition}\end{entrée}

\begin{entrée}{ɕɤkhoz}{}{ⓔɕɤkhoz} 
\classe{n} 
\begin{définition}\pfra{sac que l'on porte en bandoulière}\end{définition}
\begin{définition}\pcmn{挎包}\end{définition}\relationsémantique{反义词}{\lien{ⓔphɯrkhɯɣ}{phɯrkhɯɣ}}\end{entrée}

\begin{entrée}{ɕɤkhrɯ}{}{ⓔɕɤkhrɯ} 
\classe{n} 
\begin{définition}\pfra{viande séchée}\end{définition}
\begin{définition}\pcmn{牛肉干}\end{définition}\relationsémantique{参考}{\lien{ⓔɕa}{ɕa}}\relationsémantique{参考}{\lien{ⓔkhrɯ}{khrɯ}}\end{entrée}

\begin{entrée}{ɕɤldʐa}{}{ⓔɕɤldʐa} 
\classe{n} 
\begin{définition}\pfra{viande sanglante}\end{définition}
\begin{définition}\pcmn{血淋淋的肉块}\end{définition}
\begin{exemple}\pjya{ɯ-mɲaʁ ɯ-ŋgɯ ɕɤldʐa kɯ-fse ko-ɣi}\hspace{5pt}\pcmn{他眼睛发红(几乎看不到眼珠)}\end{exemple}\end{entrée}

\begin{entrée}{ɕɤltɕhɯz}{}{ⓔɕɤltɕhɯz} 
\classe{n} 
\begin{définition}\pfra{peler à la base des ongles}\end{définition}
\begin{définition}\pcmn{指甲的底部脱皮}\end{définition}\end{entrée}

\begin{entrée}{ɕɤltsaʁ}{}{ⓔɕɤltsaʁ} 
\classe{n} 
\begin{définition}\pfra{habit d'homme en cuir}\end{définition}
\begin{définition}\pcmn{皮袄(男生穿)}\end{définition}
\begin{exemple}\pjya{tɯ-rcu ɕɤltsaʁ}\hspace{5pt}\pcmn{皮袄}\end{exemple}\end{entrée}

\begin{entrée}{ɕɤmcɤthoʁ}{}{ⓔɕɤmcɤthoʁ} 
\classe{n}  
\grammaire{n.lieu} 
\begin{définition}\pfra{nom de lieu}\end{définition}
\begin{définition}\pcmn{龙头滩}\end{définition}\end{entrée}

\begin{entrée}{ɕɤmi}{}{ⓔɕɤmi} 
\classe{n} 
\begin{définition}\pfra{viande des membres postérieurs}\end{définition}
\begin{définition}\pcmn{牲畜后腿的肉}\end{définition}\relationsémantique{参考}{\lien{ⓔɕa}{ɕa}}\relationsémantique{参考}{\lien{ⓔtɯ-mi}{tɯ-mi}}\end{entrée}

\begin{entrée}{ɕɤmiɕtʂɤt}{}{ⓔɕɤmiɕtʂɤt} 
\classe{n} 
\begin{définition}\pfra{crochet utilisé pour accrocher la viande cuite}\end{définition}
\begin{définition}\pcmn{专门用来钩住熟肉的铁钩}\end{définition}\relationsémantique{参考}{\lien{ⓔɕomⓗ1}{ɕom₁}}\end{entrée}

\begin{entrée}{ɕɤmiŋoʁ}{}{ⓔɕɤmiŋoʁ} 
\classe{n} 
\begin{définition}\pfra{crochet de cheminée}\end{définition}
\begin{définition}\pcmn{火钩}\end{définition}\relationsémantique{同义词}{\lien{ⓔsmɯʁjoʁ}{smɯʁjoʁ}}\relationsémantique{参考}{\lien{ⓔɕomⓗ1}{ɕom₁}}\relationsémantique{参考}{\lien{}{tɤjŋoʁ}}\end{entrée}

\begin{entrée}{ɕɤmloʁ}{}{ⓔɕɤmloʁ} 
\classe{n} 
\begin{définition}\pfra{heurtoir}\end{définition}
\begin{définition}\pcmn{门环}\end{définition}
\begin{exemple}\pjya{ɕɤmloʁ nɯ ko-ndo tɕe kɯm lo-nɤrkhɯrkhɯβ}\hspace{5pt}\pcmn{他用门环敲门}\end{exemple}\relationsémantique{参考}{\lien{ⓔɕomⓗ1}{ɕom}}\end{entrée}

\begin{entrée}{ɕɤmscoʁ}{}{ⓔɕɤmscoʁ} 
\classe{n} 
\begin{définition}\pfra{mors}\end{définition}
\begin{définition}\pcmn{马嚼子}\end{définition}\relationsémantique{参考}{\lien{ⓔɕomⓗ1}{ɕom₁}}\relationsémantique{参考}{\lien{ⓔscoʁ}{scoʁ}}\end{entrée}

\begin{entrée}{ɕɤmtshoʁ}{}{ⓔɕɤmtshoʁ} 
\classe{n} 
\begin{définition}\pfra{clou en fer}\end{définition}
\begin{définition}\pcmn{铁钉}\end{définition}\relationsémantique{参考}{\lien{ⓔɕomⓗ1}{ɕom}}\relationsémantique{参考}{\lien{ⓔtɤtshoʁ}{tɤtshoʁ}}\end{entrée}

\begin{entrée}{ɕɤmɯɣdɯ}{}{ⓔɕɤmɯɣdɯ} 
\classe{n} 
\begin{définition}\pfra{arme à feu}\end{définition}
\begin{définition}\pcmn{枪}\end{définition}\relationsémantique{参考}{\lien{ⓔɕomⓗ1}{ɕom₁}}\end{entrée}

\begin{entrée}{ɕɤnthɤβ}{}{ⓔɕɤnthɤβ} 
\classe{n} 
\begin{définition}\pfra{robe de moine}\end{définition}
\begin{définition}\pcmn{和尚服装的一种,穿在腰上}\end{définition}\end{entrée}

\begin{entrée}{ɕɤntsɯt}{}{ⓔɕɤntsɯt} 
\classe{n} 
\begin{définition}\pfra{habit de femme en tissu}\end{définition}
\begin{définition}\pcmn{布;呢子制成的女装}\end{définition}\end{entrée}

\begin{entrée}{ɕɤŋi}{}{ⓔɕɤŋi} 
\classe{n} 
\begin{définition}\pfra{viande crue}\end{définition}
\begin{définition}\pcmn{生肉}\end{définition}\end{entrée}

\begin{entrée}{ɕɤr}{}{ⓔɕɤr} 
\classe{n} 
\begin{définition}\pfra{soir}\end{définition}
\begin{définition}\pcmn{晚上}\end{définition}\end{entrée}

\begin{entrée}{ɕɤrɤɕa}{}{ⓔɕɤrɤɕa} 
\classe{n} 
\begin{définition}\pfra{viande du ventre du cochon}\end{définition}
\begin{définition}\pcmn{猪的肚皮的内层肉}\end{définition}\end{entrée}

\begin{entrée}{ɕɤrkha}{}{ⓔɕɤrkha} 
\classe{n} 
\begin{définition}\pfra{aube}\end{définition}
\begin{définition}\pcmn{黎明}\end{définition}
\begin{exemple}\pjya{ɕɤrkha ɲɤ-ɴɢraʁ}\hspace{5pt}\pcmn{已经破晓了}\end{exemple}\relationsémantique{参考}{\lien{ⓔɴɢraʁⓝɕɤrkha,ɴɢraʁ}{ɕɤrkha,ɴɢraʁ}}\end{entrée}

\begin{entrée}{ɕɤrma}{}{ⓔɕɤrma} 
\classe{n} 
\begin{définition}\pfra{brigand}\end{définition}
\begin{définition}\pcmn{土匪}\end{définition}\end{entrée}

\begin{entrée}{ɕɤrmasŋi}{}{ⓔɕɤrmasŋi} 
\classe{adv} 
\begin{définition}\pfra{le jour et la nuit}\end{définition}
\begin{définition}\pcmn{日夜}\end{définition}\relationsémantique{参考}{\lien{ⓔɕɤr}{ɕɤr}}\relationsémantique{参考}{\lien{ⓔtɯ-sŋi}{tɯ-sŋi}}\end{entrée}

\begin{entrée}{ɕɤrna}{}{ⓔɕɤrna} 
\classe{n} 
\begin{définition}\pfra{oreille coupée}\end{définition}
\begin{définition}\pcmn{被砍掉的耳朵}\end{définition}
\begin{exemple}\pjya{paʁ ɯ-ɕɤrna}\hspace{5pt}\pcmn{被砍掉的猪耳朵}\end{exemple}\relationsémantique{参考}{\lien{ⓔɕɤku}{ɕɤku}}\relationsémantique{参考}{\lien{ⓔɕɤjaʁ}{ɕɤjaʁ}}\relationsémantique{参考}{\lien{ⓔɕɤmi}{ɕɤmi}}\relationsémantique{参考}{\lien{ⓔtɯ-rna}{tɯ-rna}}\end{entrée}

\begin{entrée}{ɕɤrɯ}{}{ⓔɕɤrɯ} 
\classe{n} 
\begin{définition}\pfra{os}\end{définition}
\begin{définition}\pcmn{骨头}\end{définition}\end{entrée}

\begin{entrée}{ɕɤrwa}{}{ⓔɕɤrwa} 
\classe{n} 
\begin{définition}\pfra{musulman}\end{définition}
\begin{définition}\pcmn{回族}\end{définition}\étymologie{ɕar.pa}\end{entrée}

\begin{entrée}{ɕɤrzaŋ}{}{ⓔɕɤrzaŋ} 
\classe{n} 
\begin{définition}\pfra{casserole en cuivre}\end{définition}
\begin{définition}\pcmn{铜锅}\end{définition}\end{entrée}

\begin{entrée}{ɕɤsca}{}{ⓔɕɤsca} 
\classe{n} 
\begin{définition}\pfra{cerf (mâle), bovidé de couleur noire dont le milieu du corps est blanc}\end{définition}
\begin{définition}\pcmn{公鹿;全身黑色、腰白色的牛}\end{définition}\end{entrée}

\begin{entrée}{ɕɤt}{}{ⓔɕɤt} 
\classe{vi} \paradigme{dir}{nɯ-}
\begin{définition}\pfra{s'habituer}\end{définition}
\begin{définition}\pcmn{习惯}\end{définition}
\begin{exemple}\pjya{kɯki aʑo ɕat-a, tɯ-ɕɤt, ɯʑo ɕɤt}\hspace{5pt}\pcmn{我习惯这个,你习惯,他习惯}\end{exemple}
\begin{exemple}\pjya{ɯʑo ɲɤ-ɕɤt, ɯʑo nɯ-ɕɤt}\hspace{5pt}\pcmn{他习惯了}\end{exemple}
\begin{exemple}\pjya{tɯ-ŋga kɤ-ŋga nɯ kɤ-ɕɤt ci tu wo}\hspace{5pt}\pcmn{(人)穿衣服是有(自己的)习惯的}\end{exemple}
\begin{exemple}\pjya{ɯʑo tɯ-ŋga kɯ-jaʁ kɤ-ŋga ɲɤ-ɕɤt}\hspace{5pt}\pcmn{他习惯穿很厚的衣服}\end{exemple}
\begin{exemple}\pjya{aʑo tɯcɯrqɯ kɤ-tshi nɯ-ɕat-a}\hspace{5pt}\pcmn{我习惯喝冷水}\end{exemple}\end{entrée}

\begin{entrée}{ɕɤwɤr}{}{ⓔɕɤwɤr} 
\classe{n} 
\begin{définition}\pfra{conjonctivite}\end{définition}
\begin{définition}\pcmn{结膜炎}\end{définition}\end{entrée}

\begin{entrée}{ɕɤχcɤl}{}{ⓔɕɤχcɤl} 
\classe{n} 
\begin{définition}\pfra{minuit}\end{définition}
\begin{définition}\pcmn{半夜}\end{définition}\relationsémantique{参考}{\lien{ⓔɕɤr}{ɕɤr}}\relationsémantique{参考}{\lien{ⓔɯ-χcɤl}{ɯ-χcɤl}}\end{entrée}

\begin{entrée}{ɕe}{}{ⓔɕe} 
\classe{vi} \paradigme{dir}{\_}\paradigme{past stem}{ari}\paradigme{construction}{participe sujet}\sens{1}
\begin{définition}\pfra{aller}\end{définition}
\begin{définition}\pcmn{去}\end{définition}
\begin{exemple}\pjya{jɤ-ari-a, jɤ-ari}\hspace{5pt}\pcmn{我走了,他走了}\end{exemple}
\begin{exemple}\pjya{kha tu-ɕe-a ŋu}\hspace{5pt}\pcmn{我在回家的路上}\end{exemple}\sens{2}\paradigme{dir}{lɤ-}
\begin{définition}\pfra{entrer}\end{définition}
\begin{définition}\pcmn{进去}\end{définition}
\begin{exemple}\pjya{kɯm tu-ci-a tɕe kha ɯ-ŋgɯ lɤ-ari-a tɕe a-pɯ-ŋu}\hspace{5pt}\pcmn{我开门,进了门再说}\end{exemple}
\begin{exemple}\pjya{a-mɲaʁ ɯ-ŋgɯ thɯci ko-ɕe}\hspace{5pt}\pcmn{我眼睛里进了什么东西}\end{exemple}\sens{3}\paradigme{dir}{thɯ-}
\begin{définition}\pfra{sortir}\end{définition}
\begin{définition}\pcmn{出去}\end{définition}
\begin{exemple}\pjya{tɯ-ɕna ɯ-ŋgɯ ri tɯ-skɤt chɯ-ɕe ɲɯ-ra}\hspace{5pt}\pcmn{这个音要从鼻子发出来}\end{exemple}\sens{4}\paradigme{dir}{thɯ-}
\begin{définition}\pfra{couler}\end{définition}
\begin{définition}\pcmn{流(水)}\end{définition}
\begin{exemple}\pjya{tɯ-ci chɯ-ɕe ɲɯ-ŋu}\hspace{5pt}\pcmn{水在流淌}\end{exemple}\sens{5}\paradigme{dir}{\_}
\begin{définition}\pfra{être (aligné, étendu) le long de}\end{définition}
\begin{définition}\pcmn{排过去}\end{définition}
\begin{exemple}\pjya{tɤrɲɟo nɯ jɯɣi ʁɟa ʑo ku-ɕe ɲɯ-ŋu}\hspace{5pt}\pcmn{架子上排过去的全都是书}\end{exemple}\sens{6}\paradigme{dir}{nɯ-}
\begin{définition}\pfra{se passer (temps)}\end{définition}
\begin{définition}\pcmn{过(时间)}\end{définition}
\begin{exemple}\pjya{kɯmŋu-xpa ɲɤ-ɕe}\hspace{5pt}\pcmn{过了五年}\end{exemple}\relationsémantique{参考}{\lien{ⓔari}{ari}}\relationsémantique{参考}{\lien{ⓔsɯxɕe}{sɯxɕe}}\relationsémantique{参考}{\lien{ⓔtɯ-sɯmⓝtɯ-sɯm,ɕe}{tɯ-sɯm,ɕe}}\relationsémantique{参考}{\lien{ⓔtɯ-sroʁⓝtɯ-sroʁ,ɕe}{tɯ-sroʁ,ɕe}}\relationsémantique{参考}{\lien{ⓔtɯ-βriⓝtɯ-βri,ɕe}{tɯ-βri,ɕe}}
\begin{sous-entrée}{ɯ-taʁ ɕe}{ⓔɕeⓢ6ⓝɯ-taʁ ɕe}
\begin{définition}\pfra{faire comme si}\end{définition}
\begin{définition}\pcmn{当做是}\end{définition}
\begin{exemple}\pjya{aʑo jisŋi tɤrca pɯ-me-a tɕe, tɤ-kɯ-nɯna ɯ-taʁ nɯ-ari}\hspace{5pt}\pcmn{我今天没有跟他们一起去,就(大家就)当做我休息(其实不是)}\end{exemple}\end{sous-entrée}

\end{entrée}

\begin{entrée}{ɕɣaʁɕɣaʁ}{}{ⓔɕɣaʁɕɣaʁ} 
\classe{idph.2} 
\begin{définition}\pfra{pointus et brillants (crocs)}\end{définition}
\begin{définition}\pcmn{形容獠牙等尖而光滑的样子}\end{définition}
\begin{exemple}\pjya{ɯ-ndzɣi ɕɣaʁɕɣaʁ ʑo pa}\hspace{5pt}\pcmn{它的獠牙又尖又光滑}\end{exemple}\end{entrée}

\begin{entrée}{ɕɣɤχa}{}{ⓔɕɣɤχa} 
\classe{n} 
\begin{définition}\pfra{auquel il manque une dent}\end{définition}
\begin{définition}\pcmn{少了一颗牙齿的(人)}\end{définition}\relationsémantique{参考}{\lien{ⓔtɯ-ɕɣa}{tɯ-ɕɣa}}\relationsémantique{参考}{\lien{ⓔaχa}{aχa}}\end{entrée}

\begin{entrée}{ɕɣɤz}{}{ⓔɕɣɤz} 
\classe{vt} \paradigme{dir}{jɤ-}
\begin{définition}\pfra{rendre}\end{définition}
\begin{définition}\pcmn{还东西;退回}\end{définition}
\begin{exemple}\pjya{ɯʑo kɯ ja-ɕɣɤz}\hspace{5pt}\pcmn{他还了}\end{exemple}
\begin{exemple}\pjya{kɯki mɯ́j-ra tɕe, kɤ-ɕɣɤz ɲɯ-ra}\hspace{5pt}\pcmn{不再需要的话请还(给我)}\end{exemple}
\begin{exemple}\pjya{nɯ-kɯ-ɕɣaz-a}\hspace{5pt}\pcmn{你还给我了}\end{exemple}\relationsémantique{同义词}{\lien{ⓔfsɯɣ}{fsɯɣ}}\end{entrée}

\begin{entrée}{ɕico}{}{ⓔɕico} 
\classe{n} 
\begin{définition}\pfra{plastique}\end{définition}
\begin{définition}\pcmn{塑料}\end{définition}\end{entrée}

\begin{entrée}{ɕintɕhi}{}{ⓔɕintɕhi} 
\classe{n} 
\begin{définition}\pfra{jour de repos}\end{définition}
\begin{définition}\pcmn{(放的)假}\end{définition}
\begin{exemple}\pjya{jisŋi a-ɕintɕhi ŋu}\hspace{5pt}\pcmn{我今天放假}\end{exemple}\étymologie{fn:星期}\end{entrée}

\begin{entrée}{ɕirʁaʁ}{}{ⓔɕirʁaʁ} 
\classe{n} 
\begin{définition}\pfra{une espèce de champignon}\end{définition}
\begin{définition}\pcmn{青冈菌}\end{définition}
\begin{exemple}\pjya{ɕirʁaʁ nɯ ɕkrɤz kɯ-xtɕi tsa ɯ-ŋgɯ tu-ɬoʁ ŋu, ɯ-mgɯrqhu ɣɯrni ɯ-pa cho ɯ-ru nɯ ra wɣrum, kɤ-ndza mɯm, kɯ-xtɕɯ-xtɕi qiaβ}\hspace{5pt}\pcmn{青冈菌生长在比较矮小的青冈树林里,背面粉红色,下部和菌柄白色。好吃,略苦。}\end{exemple}\end{entrée}

\begin{entrée}{ɕku}{}{ⓔɕku} 
\classe{n} 
\begin{définition}\pfra{oignon}\end{définition}
\begin{définition}\pcmn{葱}\end{définition}\end{entrée}

\begin{entrée}{ɕkala}{}{ⓔɕkala} 
\classe{n} 
\begin{définition}\pfra{boiteux}\end{définition}
\begin{définition}\pcmn{跛子}\end{définition}\relationsémantique{同义词}{\lien{ⓔʑɤwu}{ʑɤwu}}\relationsémantique{参考}{\lien{ⓔaɕkala}{aɕkala}}\end{entrée}

\begin{entrée}{ɕkatmbri}{}{ⓔɕkatmbri} 
\classe{n} 
\begin{définition}\pfra{corde pour attacher les marchandises}\end{définition}
\begin{définition}\pcmn{绑驮子用的绳子}\end{définition}\relationsémantique{参考}{\lien{ⓔtɯ-ɕkat}{tɯ-ɕkat}}\relationsémantique{参考}{\lien{ⓔtɯmbri}{tɯmbri}}\end{entrée}

\begin{entrée}{ɕkɤbɯ}{}{ⓔɕkɤbɯ} 
\classe{n} 
\begin{définition}\pfra{pain aux poireaux}\end{définition}
\begin{définition}\pcmn{韭菜的包子}\end{définition}\relationsémantique{参考}{\lien{ⓔɕku}{ɕku}}\end{entrée}

\begin{entrée}{ɕkɤfkri}{}{ⓔɕkɤfkri} 
\classe{n} 
\begin{définition}\pfra{sauce à l'air}\end{définition}
\begin{définition}\pcmn{大蒜沾水}\end{définition}\relationsémantique{参考}{\lien{ⓔɕku}{ɕku}}\end{entrée}

\begin{entrée}{ɕkɤɣɕkɤɣ}{}{ⓔɕkɤɣɕkɤɣ} 
\classe{idph.2} 
\begin{définition}\pfra{grand, maigre et bossu}\end{définition}
\begin{définition}\pcmn{形容又高又瘦又驼背的样子}\end{définition}\relationsémantique{同义词}{\lien{ⓔʑgɤβʑgɤβ}{ʑgɤβʑgɤβ}}\end{entrée}

\begin{entrée}{ɕkɤɣnɤɕkɤɣ}{}{ⓔɕkɤɣnɤɕkɤɣ} 
\classe{idph.3} 
\begin{définition}\pfra{en saccade}\end{définition}
\begin{définition}\pcmn{一震一震}\end{définition}
\begin{exemple}\pjya{ɕkɤɣnɤɕkɤɣ pa-xtsɯ}\hspace{5pt}\pcmn{他一震一震地砸了}\end{exemple}
\begin{exemple}\pjya{ɯ-mi ɲɯ-mŋɤm tɕe, ɕkɤɣnɤɕkɤɣ kɤ-anɯri}\hspace{5pt}\pcmn{他脚很痛,蹒跚地回去了}\end{exemple}\relationsémantique{参考}{\lien{ⓔɣɤɕkɤɣɕkɤɣ}{ɣɤɕkɤɣɕkɤɣ}}\end{entrée}

\begin{entrée}{ɕkɤkhe}{}{ⓔɕkɤkhe} 
\classe{n} 
\begin{définition}\pfra{oignon}\end{définition}
\begin{définition}\pcmn{葱}\end{définition}
\begin{exemple}\pjya{ɕkɤkhe nɯ praʁ kɯ-fse kɯ-tu tsa ɣɯ sɤtɕha tu-ɬoʁ ŋu, ɯ-qa jpum tsa ɲɯ-βze cha, ɯ-rqhu kɤntɕhɯ ʑo tu, ɯ-jwaʁ nɯ kɯ-ɤlɯlju tɕe qhoʁsjɯβ ŋu ri nɤrko, ɯ-tho tu, ɯ-mɯntoʁ kɯ-ɤrŋi ɲɯ-lɤt ŋu. kɤ-ndza mɯm.}\hspace{5pt}\pcmn{\lien{ⓔɕkɤkhe}{ɕkɤkhe}生长在岩石一样石头比较多的山上,根粗,有多层皮,叶子也是圆柱形的、空心的,但很结实。有花梗,开蓝色的花。好吃。}\end{exemple}\relationsémantique{参考}{\lien{ⓔɕku}{ɕku}}\end{entrée}

\begin{entrée}{ɕkɤkɯm}{}{ⓔɕkɤkɯm} 
\classe{n} 
\begin{définition}\pfra{jardin}\end{définition}
\begin{définition}\pcmn{菜园}\end{définition}\relationsémantique{参考}{\lien{ⓔɕku}{ɕku}}\end{entrée}

\begin{entrée}{ɕkɤɲcɣa}{}{ⓔɕkɤɲcɣa} 
\classe{n} 
\begin{définition}\pfra{poireau sauvage}\end{définition}
\begin{définition}\pcmn{野韭菜}\end{définition}
\begin{exemple}\pjya{ɕkɤɲcɣa nɯ ɯ-mdoʁ kɯ-pɣi tsa ŋu, ɯ-qa ɯ-zrɤm sɤɣ-ndzoʁ nɯ ɯ-rqhu tu jpum, ɯ-jwaʁ kɯ-jaʁ tɕe, tu-ŋgɤɣ tɕe, tɯ-ɲcɣa ɯ-tshɯɣa fse. ɯ-χcɤl ɯ-spjɯŋ tɤ-ɣe tɕe, ɯ-kɤχcɤl ɲɯ-rɯmɯntoʁ ŋu. ɯ-mɯntoʁ nɯ ɕkɤzoŋzoŋ rmi, li ɕkɤtho rmi. kɤ-ndza sna, ɯ-di mɯm.}\hspace{5pt}\pcmn{{ɕkɤɲcɣa}呈灰色,长根的部位有硬皮。根略粗。叶子厚而弯,形似镰刀(\lien{ⓔtɯɲcɣa}{tɯɲcɣa})。中间长主心干,顶端开花。花叫做\lien{ⓔɕkɤzoŋzoŋ}{ɕkɤzoŋzoŋ},也叫做\lien{ⓔɕkɤtho}{ɕkɤtho}。可食用,好吃。}\end{exemple}\relationsémantique{参考}{\lien{ⓔɕku}{ɕku}}\relationsémantique{参考}{\lien{ⓔtɯɲcɣa}{tɯɲcɣa}}\end{entrée}

\begin{entrée}{ɕkɤphɤr}{}{ⓔɕkɤphɤr} 
\classe{n} 
\begin{définition}\pfra{poireau sauvage}\end{définition}
\begin{définition}\pcmn{鹿耳韭}\end{définition}
\begin{exemple}\pjya{ɕkɤphɤr nɯ mbraj, sɤjku, tɯrgi kɯ-mbro ɯ-ŋgɯ tu-ɬoʁ ŋu. ɯ-qa ɣɯ ɯ-rqhu kɤntɕhɯ-tɤlɤβ ʑo tu, ɯ-ru me, ɯ-jwaʁ ʁnɯ-mpɕar ma me, ɯ-χcɤl ri ɯ-tho tu-ɬoʁ tɕe, lonba χsɯ-ldʑa ma tu-kɯ-ɬoʁ me. ɯ-jwaʁ nɯ qartshaz ɣɯ ɯ-rna ɯ-tshɯɣa fse, kɯ-ɤrtɯm kɯ-rɲɟi tɕe lu-kɯ-ɤkɤmtɕoʁ ŋu, ɯ-jwaʁ nɯ mpɕu, mpɯ, tsuku kɯ ɕkɤphɤr tu-ti-nɯ ŋu, tsuku kɯ ɕkɤjwaʁ tu-ti-nɯ ŋu. kɤ-ndza mɯm.}\hspace{5pt}\pcmn{\lien{ⓔɕkɤphɤr}{ɕkɤphɤr}生长在树高较高的红桦、白桦或杉树林里。根长有多层皮,没有茎,只有两片叶子。中间长花梗,总共只有三根。叶子形状像鹿的耳朵,椭圆形,顶部尖。叶子光滑、嫩。有的人叫\lien{ⓔɕkɤphɤr}{ɕkɤphɤr},有的人叫\lien{}{ɕkɤjwaʁ}。好吃。}\end{exemple}\end{entrée}

\begin{entrée}{ɕkɤpja}{}{ⓔɕkɤpja} 
\classe{n} 
\begin{définition}\pfra{poireau sauvage}\end{définition}
\begin{définition}\pcmn{野韭菜}\end{définition}
\begin{exemple}\pjya{ɕkɤpja nɯ sɯŋgɯ arɤkhɯmkhɤl ma sɯŋgɯ thamtɕɤt tu maʁ, ɕkɤpja nɯ kɯ-ndɯβ ci ŋu, ɯ-jwaʁ arŋi, ɯ-ru nɯ ɯ-jwaʁ lu-ɬoʁ ɕɯŋgɯ nɯ kɯ-ɤrɤʑɯʑrɤz ŋu, kɯ-wɣrum tɯ-ʑrɤz, kɯ-ɣɯrni tɯ-ʑrɤz ŋu. ɯ-jwaʁ χsɯm ɕaŋtaʁ me. ɯ-χcɤl ɯ-mɯntoʁ tu-ɬoʁ tɕe, chɯ-do ɕti. mɯntoʁ kɯ-ɣɯrni ɲɯ-lɤt ŋu. tú-wɣ-ndza tɕe wuma ʑo mɤrtsaβ tɕe ɯ-dɯχɯn tu.}\hspace{5pt}\pcmn{\lien{ⓔɕkɤpja}{ɕkɤpja}生长在某些森林中,并不是所有地方都有。\lien{ⓔɕkɤpja}{ɕkɤpja}长得小,叶呈绿色。叶子长出来之前,茎上有条纹,白一条,红一条。叶子最多只有三片。中间长花梗时,就老了。开红色花。吃起来辣,有香味。}\end{exemple}\end{entrée}

\begin{entrée}{ɕkɤrnɤɕkɤr}{}{ⓔɕkɤrnɤɕkɤr} 
\classe{idph.3} 
\begin{définition}\pfra{en boitant}\end{définition}
\begin{définition}\pcmn{形容人跛脚的样子}\end{définition}
\begin{exemple}\pjya{ɕkɤrnɤɕkɤr ʑo jo-nɯɕe}\hspace{5pt}\pcmn{他跛着脚地回家了}\end{exemple}\end{entrée}

\begin{entrée}{ɕkɤrɯ}{}{ⓔɕkɤrɯ} 
\classe{n} 
\begin{définition}\pfra{capricornus sumatraensis}\end{définition}
\begin{définition}\pcmn{鬣羚}\end{définition}
\begin{exemple}\pjya{ɕkɤrɯ nɯ sɯŋgɯ praʁ ɯ-rchɤβ aʁɤndɯndɤt ku-rɤʑi ɕti, rpɣo, co, sɯŋgɯ ɯ-ŋgɯ kɤsɯfse kɯ-tu ci ɕti, ɯʑo nɯ zɯmi nɯŋa fse, ɯ-qa ta-ʁrɯ nɯŋa ɯ-qa ta-ʁrɯ fse, pjɯ́-wɣ-sat tɕe, ɯ-mɤlɤjaʁ ɯ-ndʐi nɯ sɲɤt sna, ɯ-ʁrɯ tɯ-tɕha kɯ-ɤmtɕɯ-mtɕoʁ ŋu, tɯ-tɣa jamar zri, ɯ-rna nɯnɯ tɤrka ɯ-rna fse, ɯ-mtɕhi nɯ nɯŋa ɯ-mtɕhi fse, kɯ-ɤɲaʁndzɯm kɯ-ɤɣɯrnɯɕɯr tsa ɯ-mdoʁ ŋu, ɯ-rpaʁ nɯ tɕu kɯ-wɣrum tɯ-snaʁ tu.}\hspace{5pt}\pcmn{鬣羚栖息在森林里,也生活在岩石上,在高山、河坝到处都有,它有点像奶牛,蹄子像奶牛的蹄子。打死了以后,四肢的皮子可以用来作后鞧。有一对很尖的角,有一拃长,耳朵像驴子的耳朵,嘴像牛的嘴,颜色是黑里透红,肩部有一块白点。}\end{exemple}\end{entrée}

\begin{entrée}{ɕkɤtho}{}{ⓔɕkɤtho} 
\classe{n} 
\begin{définition}\pfra{pédoncule d'ail}\end{définition}
\begin{définition}\pcmn{蒜梗}\end{définition}\relationsémantique{参考}{\lien{ⓔɕku}{ɕku}}\relationsémantique{参考}{\lien{ⓔɯ-tho}{ɯ-tho}}\end{entrée}

\begin{entrée}{ɕkɤtshoŋ}{}{ⓔɕkɤtshoŋ} 
\classe{n} 
\begin{définition}\pfra{oignon}\end{définition}
\begin{définition}\pcmn{葱}\end{définition}
\begin{exemple}\pjya{ɕkɤtshoŋ nɯ tɯ-ji ɯ-ŋgɯ lu-kɤ-nɯ-ji ci ŋu, ɯ-qa nɯ ɯ-zrɤm kɯ-wɣrum ŋu, ɯ-ru me, ɯ-jwaʁ kɯ-ɤlɯlju tɕe qhoʁsjɯβ ŋu, ɯ-ku tu-omtɕoʁ ŋu, ɯ-dɯχɯn mɯm, tɕeri kɤ-smi tɕe ɯ-di ɲɯ-me ɕti.}\hspace{5pt}\pcmn{\lien{ⓔɕkɤtshoŋ}{ɕkɤtshoŋ}是自己种在地里的(农作物),须根白色,没有茎,叶子圆柱形、空心。顶部尖,有香味,但煮熟后就没有香味了。}\end{exemple}\étymologie{fn:葱}\end{entrée}

\begin{entrée}{ɕkɤtɯm}{}{ⓔɕkɤtɯm} 
\classe{n} 
\begin{définition}\pfra{racine de l'ail}\end{définition}
\begin{définition}\pcmn{大蒜的根}\end{définition}\relationsémantique{参考}{\lien{ⓔɕku}{ɕku}}\end{entrée}

\begin{entrée}{ɕkɤzoŋzoŋ}{}{ⓔɕkɤzoŋzoŋ} 
\classe{n} 
\begin{définition}\pfra{espèce d'oignon}\end{définition}
\begin{définition}\pcmn{葱的一种}\end{définition}\end{entrée}

\begin{entrée}{ɕke}{}{ⓔɕke} 
\classe{vi} \paradigme{dir}{pɯ-}\sens{1}
\begin{définition}\pfra{brûler}\end{définition}
\begin{définition}\pcmn{烫;被烫到}\end{définition}
\begin{exemple}\pjya{pɯ-ɕke-a, pɯ-ɕke}\hspace{5pt}\pcmn{我(被)烫到了,他(被)烫到了}\end{exemple}
\begin{exemple}\pjya{tɯcila pjɤ-lwoʁ tɕe pjɤ-ɕke}\hspace{5pt}\pcmn{把滚烫的水洒了一地,烫到了}\end{exemple}
\begin{exemple}\pjya{a-jaʁ pɯ-ɕke}\hspace{5pt}\pcmn{我烫伤了手}\end{exemple}\sens{2}
\begin{définition}\pfra{importante (affaire)}\end{définition}
\begin{définition}\pcmn{重要(事情)}\end{définition}
\begin{exemple}\pjya{jiɕqha nɯ kɯ-ɕke ci ɲɯ-ŋu (=kɯ-ʁzi)}\hspace{5pt}\pcmn{这件事情很重要}\end{exemple}\sens{3}
\begin{définition}\pfra{se déteriorer après avoir été exposé aux éléments (tendons, cuir etc)}\end{définition}
\begin{définition}\pcmn{因为受风而变得不结实(牛筋、皮子等)}\end{définition}\relationsémantique{参考}{\lien{ⓔsɤɕkeⓗ1}{sɤɕke₁}}\relationsémantique{参考}{\lien{ⓔsɤɕkeⓗ2}{sɤɕke₂}}\relationsémantique{参考}{\lien{ⓔnɤsɤɕke}{nɤsɤɕke}}\relationsémantique{参考}{\lien{ⓔʑɣɤsɤɕke}{ʑɣɤsɤɕke}}\relationsémantique{参考}{\lien{ⓔsɯɕke}{sɯɕke}}\end{entrée}

\begin{entrée}{ɕkho}{}{ⓔɕkho} 
\classe{vt} \paradigme{dir}{tɤ-}\paradigme{dir}{lɤ-}\paradigme{dir}{nɯ-}\paradigme{dir}{thɯ-}\paradigme{dir}{tɤ-}\paradigme{dir}{nɯ-}
\begin{définition}\pfra{faire sécher, étendre}\end{définition}
\begin{définition}\pcmn{晒;铺}\end{définition}
\begin{définition}\pfra{sécher des choses}\end{définition}
\begin{définition}\pcmn{晒东西}\end{définition}
\begin{exemple}\pjya{tɤ-ɕkho-t-a, ta-ɕkho, tɤ-ɕkhɤm}\hspace{5pt}\pcmn{我晒了,他晒了,你晒吧}\end{exemple}
\begin{exemple}\pjya{tɯ-mɯ tɤ-pe tɕe, tɤŋe nɯ-ɬoʁ tɕe, tɕe tɯ-ŋga ra kɤ-ɕkho ra}\hspace{5pt}\pcmn{太阳出来了,我们要晒衣服了}\end{exemple}
\begin{exemple}\pjya{a-ŋga tɤ-ɕkho-t-a}\hspace{5pt}\pcmn{晒衣服}\end{exemple}
\begin{exemple}\pjya{tɤ-βɟu lɤ-ɕkho-t-a}\hspace{5pt}\pcmn{铺褥子}\end{exemple}
\begin{exemple}\pjya{tɯjpu nɯ-ɕkho-t-a}\hspace{5pt}\pcmn{我把粮食晒干了}\end{exemple}
\begin{sous-entrée}{rɤɕkho}{ⓔɕkhoⓝrɤɕkho} 
\classe{vi}  
\grammaire{apass} \end{sous-entrée}

\begin{sous-entrée}{aɕkho}{ⓔɕkhoⓝaɕkho} 
\classe{vs} 
\begin{définition}\pfra{qui a une ouverture large}\end{définition}
\begin{définition}\pcmn{底部小,口很大}\end{définition}
\begin{exemple}\pjya{ki tɯthɯ ki ɲɯ-ɤɕkho}\hspace{5pt}\pcmn{这个锅子底小口大}\end{exemple}\end{sous-entrée}

\end{entrée}

\begin{entrée}{ɕkliɕkli}{}{ⓔɕkliɕkli} 
\classe{idph.2} 
\begin{définition}\pfra{rond et dur}\end{définition}
\begin{définition}\pcmn{形容长条形的东西圆而硬的样子}\end{définition}
\begin{exemple}\pjya{ʁʑɯnɯ ra nɯ-ʁla kú-wɣ-ndo tɕe ɕkliɕkli ʑo ɲɯ-rko}\hspace{5pt}\pcmn{小伙子的手臂拿起来圆圆的硬硬的}\end{exemple}
\begin{exemple}\pjya{tɤ-ri pɯ́-wɣ-rɤtɕaʁ tɕe tɯ-mɤpa ɲɯ-rko ɕkliɕkli ʑo}\hspace{5pt}\pcmn{踩在绳子的上面,感觉脚下很硬,站不稳}\end{exemple}\relationsémantique{同义词}{\lien{ⓔɕklɯɣɕklɯɣ}{ɕklɯɣɕklɯɣ}}\end{entrée}

\begin{entrée}{ɕklɯɣ}{}{ⓔɕklɯɣ} 
\classe{vt} \paradigme{dir}{nɯ-}
\begin{définition}\pfra{gêner dans le dos (d'un objet dur sur lequel on s'allonge ou que l'on porte sur le dos)}\end{définition}
\begin{définition}\pcmn{硌着背}\end{définition}
\begin{exemple}\pjya{kɯ-rko tú-wɣ-fkur tɕe tɯ-mgɯr ku-ɕklɯɣ ŋu tɕe ɕɯmŋɤm}\hspace{5pt}\pcmn{背硬的东西的时候,会硌着背,感觉很痛}\end{exemple}\end{entrée}

\begin{entrée}{ɕklɯɣɕklɯɣ}{}{ⓔɕklɯɣɕklɯɣ} 
\classe{idph.2} 
\begin{définition}\pfra{ferme et rond}\end{définition}
\begin{définition}\pcmn{形容长条形的东西圆而硬的样子}\end{définition}\relationsémantique{同义词}{\lien{ⓔɕkliɕkli}{ɕkliɕkli}}\end{entrée}

\begin{entrée}{ɕkom}{}{ⓔɕkom} 
\classe{n} 
\begin{définition}\pfra{muntjac}\end{définition}
\begin{définition}\pcmn{麂子}\end{définition}
\begin{exemple}\pjya{ɕkom nɯ sɯŋgɯnaχtɕin ɯ-ŋgɯ zɯ ku-rɤʑi ɲɯ-ŋu, zgoku χcɤl tsa zɯ ku-rɤʑi, ɯ-qa taʁrɯ nɯ qaʑo ɯ-qa ta-ʁrɯ fse, ɯʑo kɯ-pɣi ŋu, ɯ-jme tɯ-tɯɣa jamar maŋe, kɯ-wɣrum ŋu, ɯ-ku nɯ qaʑo ɯ-ku tsa fse, ɯ-ʁrɯ me.}\hspace{5pt}\pcmn{麂子生活在深山老林中的半山地带。蹄子类似绵羊蹄,身灰色,尾巴只有一拃长,白色。头部类似绵羊,没有角。}\end{exemple}\end{entrée}

\begin{entrée}{ɕkrɤɣɕkrɤɣ}{}{ⓔɕkrɤɣɕkrɤɣ} 
\classe{idph.2} 
\begin{définition}\pfra{dur et froid (sensation lorsqu'on s'allonge sur le sol)}\end{définition}
\begin{définition}\pcmn{又硬又冷,没有衣服盖(躺在地上的感觉)}\end{définition}
\begin{exemple}\pjya{ɯ-thoʁ ɕkrɤɣɕkrɤɣ pɯ-nɯ-rŋgɯ-a}\hspace{5pt}\pcmn{我在地上睡觉,感觉又硬又冷}\end{exemple}\relationsémantique{参考}{\lien{ⓔnɯɕkrɤɣ}{nɯɕkrɤɣ}}\end{entrée}

\begin{entrée}{ɕkrɤz}{}{ⓔɕkrɤz} 
\classe{n} 
\begin{définition}\pfra{chêne}\end{définition}
\begin{définition}\pcmn{青冈树;槲栎}\end{définition}
\begin{exemple}\pjya{ɕkrɤz nɯ zgoku aʁɤndɯndɤt tu-ɬoʁ cha, tu-wxti cha, aɣɯrtɯrtaʁ tɕe, ɯ-ru tɯ-ldʑa kɯ-jpum ɲɯ-βze mɤ-cha, ɯ-rtaʁ jpum, sɯŋgɯ kɯ-wxti nɯ ra ɯ-ru kɯ-zri tu-βze tɕe, ɯ-kɤχcɤl tɕe ɲɯ-ɤɣɯrtɯrtaʁ ŋu. ɯ-jwaʁ nɯ kɯ-ɤrtɯm tsa tɕe ɯ-βzɯr nɯ tɕu ɯ-mdzu kɯ-mtɕoʁ ʑo ku-nɯgrɤl ŋu. ɯ-jwaʁ ɯ-qhu chu kɯ-qarŋe tɯ-ɣndʑɤr kɯ-fse tu, ɯ-jwaʁ ɯ-ʁɤri arŋi mpɕu, ɯ-jwaʁ nɯ qamphoʁ rmi. ɕkrɤz ɯ-mat chɯ-βze ŋu tɕe, ɯ-mat nɯ thɣe rmi, paʁndza wuma ʑo pe, ɯ-ku ri ɯ-mat kɯ-maʁ kɯ-rko ci ku-ndzoʁ ŋu tɕe nɯ tɕamu sna. ɕkrɤz kɯ-do ɣɯ ɯ-ci tɤ-se kɯ-fse pjɯ-kɯ-nɯ-ɬoʁ ci ɣɤʑu tɕe, nɯ ku-jkrɯt tɕe, nɯ tʂha stu kɯ-ʑru ɲɯ-ŋu, smɤn ɲɯ-ŋu. ɕkrɤz ʁnɯ-tɯphu ɣɤʑu tɕe, tɯ-tɯphu nɯ ɯ-ru kɯ-xtshɯm kɯ-zri ɲɯ-ŋu, kɯ-ɤstɤko ɲɯ-ŋu, ɯnɯnɯ qaprɤsi ɲɯ-rmi, li ci tɯ-tɯphu nɯ praʁ kɯ-fse mɤ-kɯ-sɤɣa tu-ɬoʁ ɲɯ-ŋu tɕe, ɯ-ru cho ɯ-rtaʁ ra mɯ-ɲɯ-ɤstɤko, ɲɯ-ɤjʁu tɕe nɯ praʁkɤsi ɲɯ-rmi.}\hspace{5pt}\pcmn{青冈树在山上到处都可以生长,长得很大,因为枝桠发达所以没有一棵比较粗的主干,枝桠很粗,在大森林里树干长得高,在顶端上才长枝桠。叶子有点圆,在边缘排列着锋利的刺。叶子的背面有黄色粉状的东西,叶子正面是绿色的,光滑的。叶子叫\lien{ⓔqamphoʁ}{qamphoʁ}。 青冈树也结果实,这种果实叫\lien{ⓔthɣe}{thɣe},可以是喂猪的好饲料。在树梢上长出一块不是果实的硬东西,可以熬茶。老青冈树上有一种像血一样的液体流出来,凝结后是最优质的茶,是一种药。有两种青冈树,一种树干细而长,很直,叫\lien{ⓔqaprɤsi}{qaprɤsi}(蛇树),另一种长在岩石和陡峭的地方,树干和树枝都长得不直,弯曲的,这种叫\lien{ⓔpraʁkɤsi}{praʁkɤsi}(岩石上的树)。}\end{exemple}\end{entrée}

\begin{entrée}{ɕkrɯɣɕkrɯɣ}{}{ⓔɕkrɯɣɕkrɯɣ} 
\classe{idph.2} 
\begin{définition}\pfra{dur et rugueux}\end{définition}
\begin{définition}\pcmn{形容搓紧的线绳硬而粗糙的样子}\end{définition}
\begin{exemple}\pjya{tɤ-fsɤri lú-wɣ-sɤsɯɣ tɕe ɕkrɯɣɕkrɯɣ ʑo pa}\hspace{5pt}\pcmn{把绳子搓紧的时候,就又硬又粗}\end{exemple}\end{entrée}

\begin{entrée}{ɕkɯβɕkɯβ}{}{ⓔɕkɯβɕkɯβ} 
\classe{idph.2} 
\begin{définition}\pfra{dos courbé}\end{définition}
\begin{définition}\pcmn{形容驼着背的样子(个子高)}\end{définition}
\begin{exemple}\pjya{ɕkɯβɕkɯβ ʑo ɲɯ-ŋu}\hspace{5pt}\pcmn{他驼着背站在那里}\end{exemple}
\begin{sous-entrée}{ɕkɯβnɤɕkɯβ}{ⓔɕkɯβɕkɯβⓝɕkɯβnɤɕkɯβ} 
\classe{idph.3} 
\begin{exemple}\pjya{ɕkɯβnɤɕkɯβ kɤ-ari}\hspace{5pt}\pcmn{他驼着背去了}\end{exemple}\end{sous-entrée}

\end{entrée}

\begin{entrée}{ɕkɯt}{}{ⓔɕkɯt} 
\classe{vt} \paradigme{dir}{thɯ-}\sens{1}
\begin{définition}\pfra{finir de manger}\end{définition}
\begin{définition}\pcmn{吃完}\end{définition}
\begin{exemple}\pjya{tɤ-ɕkɯta, ta-ɕkɯt}\hspace{5pt}\pcmn{我吃完了,他吃完了}\end{exemple}
\begin{exemple}\pjya{nɤ-khɯtsa ɯ-ŋgɯ nɯ tɤ-ɕkɯt}\hspace{5pt}\pcmn{你把碗里的吃完}\end{exemple}
\begin{exemple}\pjya{nɤ-@beimu ɯ-thɯ-tɯ-nɯ-ɕkɯt}\hspace{5pt}\pcmn{你把贝母吃完了吗?}\end{exemple}\sens{2}
\begin{définition}\pfra{finir de boire}\end{définition}
\begin{définition}\pcmn{喝完}\end{définition}
\begin{exemple}\pjya{ɕkɯt-tɕi}\hspace{5pt}\pcmn{干杯}\end{exemple}\sens{3}
\begin{définition}\pfra{utiliser complètement}\end{définition}
\begin{définition}\pcmn{用完}\end{définition}
\begin{exemple}\pjya{qarma pjɤ-ɕkɯt-nɯ}\hspace{5pt}\pcmn{马鸡没有了}\end{exemple}\end{entrée}

\begin{entrée}{ɕlu}{}{ⓔɕlu} 
\classe{vl} \paradigme{dir}{tɤ-}\paradigme{dir}{lɤ-}
\begin{définition}\pfra{labourer}\end{définition}
\begin{définition}\pcmn{耕地}\end{définition}
\begin{exemple}\pjya{tɤ-ɕlu-a, tɤ-ɕlu, ku-ɕlu-a}\hspace{5pt}\pcmn{我耕了地,他耕了地,我正在耕地}\end{exemple}
\begin{exemple}\pjya{ki tɯji ki tɤ-ɕlu-t-a}\hspace{5pt}\pcmn{我耕了这块地}\end{exemple}\end{entrée}

\begin{entrée}{ɕlaŋɕlaŋ}{}{ⓔɕlaŋɕlaŋ} 
\classe{idph.2} 
\begin{définition}\pfra{rond et lisse}\end{définition}
\begin{définition}\pcmn{又圆又光滑(发光)}\end{définition}
\begin{exemple}\pjya{ɯ-laz ɕlaŋɕlaŋ to-tɕɤt}\hspace{5pt}\pcmn{他把头探出来了,是光头}\end{exemple}
\begin{exemple}\pjya{ɯ-ku ɕlaŋɕlaŋ to-stu}\hspace{5pt}\pcmn{他的头光滑得发亮}\end{exemple}
\begin{exemple}\pjya{ɯ-ku pjɤ-nɯ-sɯ-qrɤz tɕe, ɕlaŋɕlaŋ ʑo ɲɯ-pa}\hspace{5pt}\pcmn{他理了发(把头发剃光了就)光滑得发亮}\end{exemple}\relationsémantique{参考}{\lien{ⓔslaŋslaŋ}{slaŋslaŋ}}\relationsémantique{参考}{\lien{ⓔclaŋclaŋ}{claŋclaŋ}}\relationsémantique{参考}{\lien{ⓔrlaŋrlaŋ}{rlaŋrlaŋ}}\end{entrée}

\begin{entrée}{ɕlaʁ}{}{ⓔɕlaʁ} 
\classe{idph.1} 
\begin{définition}\pfra{d'un seul coup}\end{définition}
\begin{définition}\pcmn{一下子}\end{définition}\relationsémantique{参考}{\lien{ⓔslaʁ}{slaʁ}}
\begin{sous-entrée}{ɕlaʁnɤɕlaʁ}{ⓔɕlaʁⓝɕlaʁnɤɕlaʁ} 
\classe{idph.3} 
\begin{définition}\pfra{en plusieurs coups}\end{définition}
\begin{définition}\pcmn{几下}\end{définition}
\begin{exemple}\pjya{kɤ-nɤma ɕlaʁnɤɕlaʁ ʑo na-sthɯt}\hspace{5pt}\pcmn{几下就把工作做完了}\end{exemple}\end{sous-entrée}

\begin{sous-entrée}{phɯɕlaʁ}{ⓔɕlaʁⓝphɯɕlaʁ} 
\classe{idph.6} \end{sous-entrée}

\end{entrée}

\begin{entrée}{ɕliɕli}{}{ⓔɕliɕli}
\begin{définition}\pfra{rond et lisse}\end{définition}
\begin{définition}\pcmn{又圆又光滑(发光)}\end{définition}
\begin{exemple}\pjya{ɕlinɤɕli kɤ-ari}\hspace{5pt}\pcmn{他(光头发光地)去了}\end{exemple}\relationsémantique{参考}{\lien{ⓔclaŋclaŋ}{claŋclaŋ}}\relationsémantique{参考}{\lien{ⓔɕlaŋɕlaŋ}{ɕlaŋɕlaŋ}}\end{entrée}

\begin{entrée}{ɕlɯɣ}{}{ⓔɕlɯɣ} 
\classe{vt} \paradigme{dir}{pɯ-}\paradigme{dir}{nɯ-}
\begin{définition}\pfra{lâcher sans faire attention}\end{définition}
\begin{définition}\pcmn{失手使物品掉落}\end{définition}
\begin{définition}\pfra{détacher}\end{définition}
\begin{définition}\pcmn{解开}\end{définition}
\begin{exemple}\pjya{nɯ-ɕlɯɣ-a, na-ɕlɯɣ}\hspace{5pt}\pcmn{我失手了,他失手了}\end{exemple}
\begin{exemple}\pjya{ki kɤ-ɕlɯɣ mɤ-pe (ma ɴɢrɯ ɕti)}\hspace{5pt}\pcmn{不能让它掉下来,不然会破掉}\end{exemple}
\begin{exemple}\pjya{tɤ-ri nɯ-kɯ-raʁ nɯ-sɯ-ɕlɯɣ-a tɕe, kɤ-rɯkɤtɯm jɤɣ}\hspace{5pt}\pcmn{我把卡住了的线解开了,可以牵线了}\end{exemple}\relationsémantique{参考}{\lien{ⓔlɯɣ}{lɯɣ}}\relationsémantique{参考}{\lien{ⓔsɯta}{sɯta}}
\begin{sous-entrée}{sɯɕlɯɣ}{ⓔɕlɯɣⓝsɯɕlɯɣ} 
\classe{vt} \end{sous-entrée}

\end{entrée}

\begin{entrée}{ɕmi}{}{ⓔɕmi} 
\classe{vt} \paradigme{dir}{tɤ-}\paradigme{dir}{pɯ-}
\begin{définition}\pfra{mélanger un liquide}\end{définition}
\begin{définition}\pcmn{搅拌}\end{définition}
\begin{exemple}\pjya{tɤ-ɕmi-t-a, ta-ɕmi}\hspace{5pt}\pcmn{我搅拌了,他搅拌了}\end{exemple}
\begin{exemple}\pjya{nɤ@cai tɤ-ɕmi ma tsha ra mɤ-amɯzɣɯt}\hspace{5pt}\pcmn{拌一下你的菜,盐放得不均匀}\end{exemple}\end{entrée}

\begin{entrée}{ɕmɯɣ/\variante{mɯɣ}}{}{ⓔɕmɯɣ} 
\classe{n} 
\begin{définition}\pfra{léger et soudain (d'un rire)}\end{définition}
\begin{définition}\pcmn{突然轻轻地笑}\end{définition}
\begin{exemple}\pjya{ɯʑo kɯ pa-mtshɤm tɕe, ɕmɯɣ ɲɤ-nɤre}\hspace{5pt}\pcmn{听他这样一说,突然笑了一下}\end{exemple}\end{entrée}

\begin{entrée}{ɕnaβndʑɣi}{}{ⓔɕnaβndʑɣi} 
\classe{n} 
\begin{définition}\pfra{morveux}\end{définition}
\begin{définition}\pcmn{总是流鼻涕的孩子}\end{définition}
\begin{exemple}\pjya{nɤʑo ɕnaβndʑɣi ki}\hspace{5pt}\pcmn{你这个爱流鼻涕的家伙}\end{exemple}\end{entrée}

\begin{entrée}{ɕnat}{}{ⓔɕnat} 
\classe{n} 
\begin{définition}\pfra{élément du métier à tisser (lice)}\end{définition}
\begin{définition}\pcmn{用来提经线的织具(综)}\end{définition}
\begin{exemple}\pjya{ɕnat nɯ ɯ-sqar ɯ-tu-kɯ-sɯ-βzu ɣɯ tɤ-ri ŋu, ɯ-tu-kɯ-rɤɕi ndʑu nɯ ɕnat-ndʑu rmi}\hspace{5pt}\pcmn{综是用来钩住经线,使经线上下交叉的线,提着这种线的木棒叫提综杆}\end{exemple}\étymologie{snas}\end{entrée}

\begin{entrée}{ɕnɤcat}{}{ⓔɕnɤcat} 
\classe{num} 
\begin{définition}\pfra{sept ou huit}\end{définition}
\begin{définition}\pcmn{七八个}\end{définition}
\begin{exemple}\pjya{ɕnɤcɤ-sŋi}\hspace{5pt}\pcmn{七八天}\end{exemple}\relationsémantique{参考}{\lien{ⓔkɯɕnɯz}{kɯɕnɯz}}\relationsémantique{参考}{\lien{ⓔkɯrcat}{kɯrcat}}\end{entrée}

\begin{entrée}{ɕnɤku}{}{ⓔɕnɤku} 
\classe{n} 
\begin{définition}\pfra{bout du nez}\end{définition}
\begin{définition}\pcmn{鼻尖}\end{définition}\relationsémantique{参考}{\lien{ⓔtɯ-ɕna}{tɯ-ɕna}}\relationsémantique{参考}{\lien{ⓔtɯ-ku}{tɯ-ku}}\end{entrée}

\begin{entrée}{ɕnɤloʁ}{}{ⓔɕnɤloʁ} 
\classe{n} 
\begin{définition}\pfra{anneau nasal}\end{définition}
\begin{définition}\pcmn{牛鼻圈}\end{définition}\end{entrée}

\begin{entrée}{ɕnɤri}{}{ⓔɕnɤri} 
\classe{n} 
\begin{définition}\pfra{corde accrochée à l'anneau nasal}\end{définition}
\begin{définition}\pcmn{牛鼻绳}\end{définition}\relationsémantique{参考}{\lien{ⓔtɯ-ɕna}{tɯ-ɕna}}\relationsémantique{参考}{\lien{ⓔtɤ-ri}{tɤ-ri}}\end{entrée}

\begin{entrée}{ɕnɤsti}{}{ⓔɕnɤsti} 
\classe{n} 
\begin{définition}\pfra{personne qui a le nez bouché}\end{définition}
\begin{définition}\pcmn{鼻子塞了的人}\end{définition}\relationsémantique{参考}{\lien{ⓔtɯ-ɕna}{tɯ-ɕna}}\relationsémantique{参考}{\lien{ⓔstiⓗ1}{sti₁}}\end{entrée}

\begin{entrée}{ɕnɤto}{}{ⓔɕnɤto} 
\classe{n} 
\begin{définition}\pfra{tabac à priser}\end{définition}
\begin{définition}\pcmn{鼻烟}\end{définition}
\begin{exemple}\pjya{ɕnɤto tɤ-nɯ-lat-a}\hspace{5pt}\pcmn{我吸了鼻烟。}\end{exemple}\étymologie{du.ba}\end{entrée}

\begin{entrée}{ɕnɤtoʁrɯ}{}{ⓔɕnɤtoʁrɯ} 
\classe{n} 
\begin{définition}\pfra{tabatière}\end{définition}
\begin{définition}\pcmn{鼻烟壶}\end{définition}\étymologie{sna.du.ba.ru}\end{entrée}

\begin{entrée}{ɕnɤxsɯr}{}{ⓔɕnɤxsɯr} 
\classe{n} 
\begin{exemple}\pjya{ɕnɤxsɯr kɯ-tu ma khɤxsɯr kɯ-me}\hspace{5pt}\pcmn{只能闻到而吃不到}\end{exemple}\relationsémantique{参考}{\lien{ⓔtɯ-ɕna}{tɯ-ɕna}}\relationsémantique{参考}{\lien{ⓔxsɯr}{xsɯr}}\end{entrée}

\begin{entrée}{ɕnoʁɕnoʁ}{}{ⓔɕnoʁɕnoʁ} 
\classe{idph.2} 
\begin{définition}\pfra{pointu}\end{définition}
\begin{définition}\pcmn{尖}\end{définition}
\begin{exemple}\pjya{ɯ-mtɕhi ɲɯ-ɤmtɕoʁ ʑo ɕnoʁɕnoʁ}\hspace{5pt}\pcmn{他嘴巴很尖(他在怄气)}\end{exemple}
\begin{exemple}\pjya{pɣɤtɕɯ ɯ-mtsioʁ ɕnoʁɕnoʁ ɲɯ-pa}\hspace{5pt}\pcmn{鸟的嘴是尖的}\end{exemple}
\begin{sous-entrée}{ɕnoʁnɤɕnoʁ}{ⓔɕnoʁɕnoʁⓝɕnoʁnɤɕnoʁ} 
\classe{idph.3} 
\begin{exemple}\pjya{kumpɣa kɯ ɕnoʁnɤɕnoʁ ɲɯ-ɤz-nɯrdoʁ}\hspace{5pt}\pcmn{母鸡一啄一啄地在啄食}\end{exemple}
\begin{exemple}\pjya{tɤ-pɤtso kɯ tɤ-rɣe ɕnoʁnɤɕnoʁ ʑo ta-nɯrdoʁ}\hspace{5pt}\pcmn{小孩子用小巧的手很快地把珠子捡起来了}\end{exemple}\end{sous-entrée}

\end{entrée}

\begin{entrée}{ɕnɯɣnɤlɯɣ}{}{ⓔɕnɯɣnɤlɯɣ} 
\classe{idph.4} 
\begin{définition}\pfra{irrespectueux}\end{définition}
\begin{définition}\pcmn{形容不稳重,说话没有礼貌,不尊重人的样子}\end{définition}
\begin{exemple}\pjya{jiɕqha nɯ ɯ-zda ra nɯ-phe ɕnɯɣnɤlɯɣ ɲɯ-ʑɣɤstu}\hspace{5pt}\pcmn{那个人在伙计面前不稳重}\end{exemple}
\begin{sous-entrée}{ɣɤɕnɯɣlɯɣ}{ⓔɕnɯɣnɤlɯɣⓝɣɤɕnɯɣlɯɣ} 
\classe{vi} 
\begin{exemple}\pjya{nɯ ɯ-zda ra nɯ-phe ɲɯ-ɣɤɕnɯɣlɯɣ ntsɯ}\hspace{5pt}\pcmn{他在伙计面前不稳重}\end{exemple}
\begin{exemple}\pjya{ɲɯ-rɯpjɤβlaʁ tɕe ɲɯ-ɣɤɕnɯɣlɯɣ}\hspace{5pt}\pcmn{他很狡猾,动作也不雅观}\end{exemple}\end{sous-entrée}

\end{entrée}

\begin{entrée}{ɕɲɯɣ}{}{ⓔɕɲɯɣ} 
\classe{idph.1} 
\begin{définition}\pfra{douleur ressentie lorsqu'on se foule la cheville}\end{définition}
\begin{définition}\pcmn{崴脚的痛感}\end{définition}
\begin{exemple}\pjya{a-mi ɕɲɯɣ ʑo (ɲɯ-ti) na-ʂŋaʁ}\hspace{5pt}\pcmn{我崴了脚,一下子就感到剧痛}\end{exemple}\end{entrée}

\begin{entrée}{ɕŋaʁɕŋaʁ}{₁}{ⓔɕŋaʁɕŋaʁⓗ1} 
\classe{idph.2} 
\begin{définition}\pfra{jaune vif}\end{définition}
\begin{définition}\pcmn{深黄色}\end{définition}
\begin{exemple}\pjya{raz kɯ-qarŋe ɕŋaʁɕŋaʁ ʑo ɲɯ-ŋu}\hspace{5pt}\pcmn{布是深黄色的}\end{exemple}\end{entrée}

\begin{entrée}{ɕŋaʁɕŋaʁ}{₂}{ⓔɕŋaʁɕŋaʁⓗ2} 
\classe{idph.2} \paradigme{dir}{pɯ-}
\begin{définition}\pfra{malin et agaçant (enfant)}\end{définition}
\begin{définition}\pcmn{形容小孩子又机灵又瘦小又讨厌的样子}\end{définition}
\begin{définition}\pfra{parler sans arrêt à tort et à travers}\end{définition}
\begin{définition}\pcmn{不停地乱说话}\end{définition}
\begin{exemple}\pjya{ki tɤ-rɟit ki ɕŋaʁɕŋaʁ ʑo ɲɯ-pa}\hspace{5pt}\pcmn{这个孩子摆出一副机灵的样子,真令人讨厌}\end{exemple}
\begin{exemple}\pjya{aʁɤndɯndɤt pɯ-tɯ-ɣɤɕŋaʁɕŋaʁ ntsɯ}\hspace{5pt}\pcmn{你四处制造谎言}\end{exemple}
\begin{sous-entrée}{ɣɤɕŋaʁɕŋaʁ}{ⓔɕŋaʁɕŋaʁⓗ2ⓝɣɤɕŋaʁɕŋaʁ} 
\classe{vi} \end{sous-entrée}

\end{entrée}

\begin{entrée}{ɕŋɤr}{}{ⓔɕŋɤr} 
\classe{n} 
\begin{définition}\pfra{givre}\end{définition}
\begin{définition}\pcmn{霜}\end{définition}
\begin{exemple}\pjya{ɕŋɤr pjɤ-ta}\hspace{5pt}\pcmn{下了霜}\end{exemple}
\begin{exemple}\pjya{ɕŋɤr pjɤ-ɣi}\hspace{5pt}\pcmn{下了霜}\end{exemple}\relationsémantique{同义词}{\lien{ⓔtɯɣur}{tɯɣur}}\end{entrée}

\begin{entrée}{ɕŋiɕŋi}{}{ⓔɕŋiɕŋi} 
\classe{idph.2} 
\begin{définition}\pfra{qui se retient de rire}\end{définition}
\begin{définition}\pcmn{形容笑嘻嘻的样子}\end{définition}
\begin{exemple}\pjya{ɕŋiɕŋi ʑo to-ʑɣɤstu}\hspace{5pt}\pcmn{他做出一副笑嘻嘻的样子}\end{exemple}\end{entrée}

\begin{entrée}{ɕŋoʁɕŋoʁ}{}{ⓔɕŋoʁɕŋoʁ} 
\classe{idph.2} 
\begin{définition}\pfra{maigre et laid}\end{définition}
\begin{définition}\pcmn{形容瘦而不好看的样子}\end{définition}
\begin{exemple}\pjya{kɯ-sɤmbrɯŋgɯ ci ɕŋoʁɕŋoʁ ɲɯ-ŋu}\hspace{5pt}\pcmn{他又瘦又不好看,让人讨厌}\end{exemple}\end{entrée}

\begin{entrée}{ɕo}{}{ⓔɕo} 
\classe{vs} \paradigme{dir}{nɯ-}\paradigme{dir}{tɤ-}
\begin{définition}\pfra{propre}\end{définition}
\begin{définition}\pcmn{洗得很干净}\end{définition}
\begin{exemple}\pjya{ɯ-ŋga ɲɯ-ɕo}\hspace{5pt}\pcmn{他的衣服洗得很干净}\end{exemple}
\begin{sous-entrée}{ɣɤɕo}{ⓔɕoⓝɣɤɕo} 
\classe{vt} \paradigme{dir}{nɯ-}
\begin{définition}\pfra{laver}\end{définition}
\begin{définition}\pcmn{洗干净}\end{définition}
\begin{exemple}\pjya{tɯ-ŋga ɲɤ-ɣɤɕo}\hspace{5pt}\pcmn{他把衣服洗干净了}\end{exemple}\end{sous-entrée}

\end{entrée}

\begin{entrée}{ɕom}{₁}{ⓔɕomⓗ1} 
\classe{n} \sens{1}
\begin{définition}\pfra{fer}\end{définition}
\begin{définition}\pcmn{铁}\end{définition}\sens{2}
\begin{définition}\pfra{métal}\end{définition}
\begin{définition}\pcmn{金属}\end{définition}\sens{3}
\begin{définition}\pfra{clou}\end{définition}
\begin{définition}\pcmn{铁钉}\end{définition}\relationsémantique{同义词}{\lien{ⓔtɤtshoʁ}{tɤtshoʁ}}\relationsémantique{参考}{\lien{ⓔɕɤmɯɣdɯ}{ɕɤmɯɣdɯ}}\relationsémantique{参考}{\lien{ⓔɕɤmiŋoʁ}{ɕɤmiŋoʁ}}\relationsémantique{参考}{\lien{ⓔɕɤmloʁ}{ɕɤmloʁ}}\end{entrée}

\begin{entrée}{ɕom}{₂}{ⓔɕomⓗ2} 
\classe{n} 
\begin{définition}\pfra{peau du lait}\end{définition}
\begin{définition}\pcmn{奶皮}\end{définition}\relationsémantique{参考}{\lien{ⓔrɤɕom}{rɤɕom}}\relationsémantique{参考}{\lien{ⓔtɤlɤɕom}{tɤlɤɕom}}\end{entrée}

\begin{entrée}{ɕommbri}{}{ⓔɕommbri} 
\classe{n} 
\begin{définition}\pfra{chaîne}\end{définition}
\begin{définition}\pcmn{铁链子}\end{définition}\relationsémantique{参考}{\lien{ⓔɕomⓗ1}{ɕom}}\relationsémantique{参考}{\lien{ⓔtɯmbri}{tɯmbri}}\end{entrée}

\begin{entrée}{ɕomnaʁ}{}{ⓔɕomnaʁ} 
\classe{n} 
\begin{définition}\pfra{fer noir}\end{définition}
\begin{définition}\pcmn{黑铁}\end{définition}\relationsémantique{参考}{\lien{ⓔɕomⓗ1}{ɕom₁}}\étymologie{nag}\end{entrée}

\begin{entrée}{ɕomskrɯt}{}{ⓔɕomskrɯt} 
\classe{n} 
\begin{définition}\pfra{fil de fer}\end{définition}
\begin{définition}\pcmn{铁丝}\end{définition}\relationsémantique{参考}{\lien{ⓔskrɯt}{skrɯt}}\relationsémantique{参考}{\lien{ⓔɕomⓗ1}{ɕom₁}}\end{entrée}

\begin{entrée}{ɕomthɯ}{}{ⓔɕomthɯ} 
\classe{n} 
\begin{définition}\pfra{casserole en fer}\end{définition}
\begin{définition}\pcmn{铁锅}\end{définition}\relationsémantique{参考}{\lien{ⓔɕomⓗ1}{ɕom₁}}\relationsémantique{参考}{\lien{}{tɯthɯ}}\end{entrée}

\begin{entrée}{ɕoŋβzu}{}{ⓔɕoŋβzu} 
\classe{n} 
\begin{définition}\pfra{menuisier}\end{définition}
\begin{définition}\pcmn{木匠}\end{définition}
\begin{exemple}\pjya{ɕoŋβzu ko-spa}\hspace{5pt}\pcmn{他学会木工了}\end{exemple}
\begin{exemple}\pjya{a-βɣo nɯ ɕoŋβzu ŋu}\hspace{5pt}\pcmn{我的伯父是个木匠}\end{exemple}
\begin{exemple}\pjya{nɯ-ɕoŋβzu tu-βze-a}\hspace{5pt}\pcmn{我给他们做木工}\end{exemple}\relationsémantique{参考}{\lien{ⓔrɯɕoŋβzu}{rɯɕoŋβzu}}\end{entrée}

\begin{entrée}{ɕoŋphu}{}{ⓔɕoŋphu} 
\classe{n} 
\begin{définition}\pfra{arbre fruitier}\end{définition}
\begin{définition}\pcmn{果树}\end{définition}\relationsémantique{同义词}{\lien{ⓔsɯphɯ}{sɯphɯ}}\end{entrée}

\begin{entrée}{ɕoŋtɕa}{}{ⓔɕoŋtɕa} 
\classe{n} 
\begin{définition}\pfra{bois}\end{définition}
\begin{définition}\pcmn{木料}\end{définition}\étymologie{ɕiŋ.tɕʰa}\end{entrée}

\begin{entrée}{ɕoʁ}{}{ⓔɕoʁ} 
\classe{n} 
\begin{définition}\pfra{sarrasin}\end{définition}
\begin{définition}\pcmn{荞麦}\end{définition}\end{entrée}

\begin{entrée}{ɕoʁɕɣa}{}{ⓔɕoʁɕɣa} 
\classe{n} 
\begin{définition}\pfra{Artemisia suboligata}\end{définition}
\begin{définition}\pcmn{茶绒蒿}\end{définition}\end{entrée}

\begin{entrée}{ɕoʁɕoʁ}{}{ⓔɕoʁɕoʁ} 
\classe{n} 
\begin{définition}\pfra{papier}\end{définition}
\begin{définition}\pcmn{纸}\end{définition}
\begin{exemple}\pjya{ɕoʁɕoʁ kɯ smi mɤ-mphɯr}\hspace{5pt}\pcmn{纸包不住火(你的坏事总有一天会暴露出来)}\end{exemple}\relationsémantique{参考}{\lien{ⓔarɯɕoʁɕoʁ}{arɯɕoʁɕoʁ}}\étymologie{ɕog}\end{entrée}

\begin{entrée}{ɕoʁmboʁ}{}{ⓔɕoʁmboʁ} 
\classe{n} 
\begin{définition}\pfra{sarrasin grillé}\end{définition}
\begin{définition}\pcmn{荞麦爆花}\end{définition}\relationsémantique{参考}{\lien{ⓔrŋɤmboʁ}{rŋɤmboʁ}}\relationsémantique{参考}{\lien{ⓔɕoʁ}{ɕoʁ}}\end{entrée}

\begin{entrée}{ɕoʁɲaʁ}{}{ⓔɕoʁɲaʁ} 
\classe{n} 
\begin{définition}\pfra{espèce de sarrasin}\end{définition}
\begin{définition}\pcmn{苦荞}\end{définition}\relationsémantique{参考}{\lien{ⓔɕoʁ}{ɕoʁ}}\end{entrée}

\begin{entrée}{ɕoχpi}{}{ⓔɕoχpi} 
\classe{n} 
\begin{définition}\pfra{pain au sarrasin}\end{définition}
\begin{définition}\pcmn{荞面馍馍}\end{définition}\relationsémantique{参考}{\lien{ⓔɕoʁ}{ɕoʁ}}\end{entrée}

\begin{entrée}{ɕpaʁ}{}{ⓔɕpaʁ} 
\classe{vi} \paradigme{dir}{nɯ-}
\begin{définition}\pfra{avoir soif}\end{définition}
\begin{définition}\pcmn{渴}\end{définition}
\begin{exemple}\pjya{ɯ-pɯ ɲɤ-ɕpaʁ}\hspace{5pt}\pcmn{他的儿子口渴了}\end{exemple}
\begin{exemple}\pjya{tɯ-ci mɯ-pjɤ-k-ɤʁe-ci tɕe ɲɯ-ɕpaʁ}\hspace{5pt}\pcmn{他没有喝到水,所以很渴}\end{exemple}
\begin{exemple}\pjya{nɤ-tɯ-ɕpaʁ thaŋnɤ}\hspace{5pt}\pcmn{你渴了吧}\end{exemple}
\begin{exemple}\pjya{a-mɤ-ɲɯ-sɤ-ɕpɯ-ɕpaʁ, tʂha ku-tshi-a}\hspace{5pt}\pcmn{我喝茶,免得渴着自己}\end{exemple}\end{entrée}

\begin{entrée}{ɕpɤr}{}{ⓔɕpɤr} 
\classe{idph.1} 
\begin{définition}\pfra{pleurer soudainement}\end{définition}
\begin{définition}\pcmn{突然大声哭出来的声音}\end{définition}
\begin{exemple}\pjya{tɤ-pɤtso ɕpɤr ʑo ɲɯ-ɣɤwu}\hspace{5pt}\pcmn{小孩子哇的一声就哭了}\end{exemple}\end{entrée}

\begin{entrée}{ɕpɤrɕpɤr/\variante{ɕpɤɕpɤr}}{}{ⓔɕpɤrɕpɤr} 
\classe{idph.2} 
\begin{définition}\pfra{rond et large}\end{définition}
\begin{définition}\pcmn{形容又圆又大的样子(难看)}\end{définition}
\begin{exemple}\pjya{nɯŋaqe ɕpɤɕpɤr ʑo ɲɯ-pa}\hspace{5pt}\pcmn{牛屎又圆又大}\end{exemple}
\begin{exemple}\pjya{jlaqe ɕpɤɕpɤr ʑo ɲɯ-pa}\hspace{5pt}\pcmn{犏牛屎又圆又大}\end{exemple}
\begin{exemple}\pjya{tɯ-ci mbala ɯ-ʁrɯ ɕpɤɕpɤr ʑo ɲɯpa}\hspace{5pt}\pcmn{水牛的角又圆又宽}\end{exemple}
\begin{exemple}\pjya{ɯ-rŋa ra ɲɯ-wxti ɕpɤrɕpɤr ʑo}\hspace{5pt}\pcmn{他脸很大,不漂亮}\end{exemple}\end{entrée}

\begin{entrée}{ɕpɤrnɤlɤr}{}{ⓔɕpɤrnɤlɤr} 
\classe{idph.4} 
\begin{définition}\pfra{voix très forte}\end{définition}
\begin{définition}\pcmn{讲话声音很大,不注意场合,也不注意礼貌,说话很放肆}\end{définition}
\begin{exemple}\pjya{ɕpɤrnɤlɤr ɲɯ-ʑɣɤstu}\hspace{5pt}\pcmn{他讲话声音很大,不注意礼貌}\end{exemple}\relationsémantique{参考}{\lien{ⓔɣɤɕpɤɕpɤr}{ɣɤɕpɤɕpɤr}}\end{entrée}

\begin{entrée}{ɕpɤtnɤɕpɤt}{}{ⓔɕpɤtnɤɕpɤt} 
\classe{idph.3} 
\begin{définition}\pfra{éloquent, sans hésitation}\end{définition}
\begin{définition}\pcmn{形容说话流利不中断的样子}\end{définition}
\begin{exemple}\pjya{ɕpɤɕpɤt nɤ ɕpɤɕpɤt pɯ-fɕat-a}\hspace{5pt}\pcmn{我讲故事讲得滔滔不绝}\end{exemple}\end{entrée}

\begin{entrée}{ɕpɣo}{}{ⓔɕpɣo} 
\classe{n} 
\begin{définition}\pfra{dix boisseaux}\end{définition}
\begin{définition}\pcmn{十升}\end{définition}\end{entrée}

\begin{entrée}{ɕpɣowoʁ}{}{ⓔɕpɣowoʁ} 
\classe{n} 
\begin{définition}\pfra{grande jarre}\end{définition}
\begin{définition}\pcmn{大坛子}\end{définition}\end{entrée}

\begin{entrée}{ɕphɤβɕphɤβ}{}{ⓔɕphɤβɕphɤβ} 
\classe{idph.2} 
\begin{définition}\pfra{couché par terre sans bouger}\end{définition}
\begin{définition}\pcmn{形容卧在地上不动的样子}\end{définition}
\begin{exemple}\pjya{nɯ tɯrme ɕphɤβɕphɤβ ɲɯ-nɯ-rŋgɯ}\hspace{5pt}\pcmn{那个人躺在地上不动}\end{exemple}
\begin{exemple}\pjya{jla ɕphɤβɕphɤβ ko-nɯ-rŋgɯ}\hspace{5pt}\pcmn{犏牛躺在地上不动了}\end{exemple}
\begin{exemple}\pjya{ɯ-thoʁ ɕphɤβɕphɤβ ʑo ɲɯ-rɤʑi, kɤ-nɯqambɯmbjom mɯ́j-cha}\hspace{5pt}\pcmn{(鸟)躺在地上,飞不起来了}\end{exemple}\end{entrée}

\begin{entrée}{ɕphɤɕphɤt}{}{ⓔɕphɤɕphɤt} 
\classe{idph.2} 
\begin{définition}\pfra{en lamelle}\end{définition}
\begin{définition}\pcmn{形容很薄的片状物体}\end{définition}
\begin{exemple}\pjya{tɤrɤm ɕphɤɕphɤt ʑo kɯ-pa ɲɤ-βzu}\hspace{5pt}\pcmn{他作了很薄的木片}\end{exemple}\end{entrée}

\begin{entrée}{ɕphɤɣnɤɕphɤɣ}{}{ⓔɕphɤɣnɤɕphɤɣ} 
\classe{idph.3} 
\begin{définition}\pfra{bruit (produit en frappant les habits avec force)}\end{définition}
\begin{définition}\pcmn{啪啪声(洗衣服时,使劲地拍打衣服)}\end{définition}
\begin{exemple}\pjya{tɯ-ŋga ɲɯ-ɤsɯ-χtɕi tɕe, ɕphɤɣnɤɕphɤɣ ɲɯ-ɤsɯ-stu}\hspace{5pt}\pcmn{她在洗衣服,发出“啪啪”声}\end{exemple}
\begin{exemple}\pjya{tɯ-ŋga kɯ-jaʁ nɯ ra ɲɯ́-wɣ-χtɕi tɕe ɕphɤɣnɤɕphɤɣ pjɯ́-wɣ-rtɤβ tɕe ɲɯ-ɕo ɕti}\hspace{5pt}\pcmn{洗厚一点的衣服的时候,要使劲地把衣服拍打才会干净}\end{exemple}\relationsémantique{参考}{\lien{ⓔsɤɕphɤɣɕphɤɣ}{sɤɕphɤɣɕphɤɣ}}\end{entrée}

\begin{entrée}{ɕphɤrɕphɤr}{}{ⓔɕphɤrɕphɤr} 
\classe{idph.2} 
\begin{définition}\pfra{large et mou}\end{définition}
\begin{définition}\pcmn{形容宽而柔软,向下垂的样子}\end{définition}
\begin{exemple}\pjya{loŋbutɕhi ɯ-rna nɯ ɲɯ-wxti ɕphɤrɕphɤr ʑo}\hspace{5pt}\pcmn{大象的耳朵又宽又大,软软地耷拉着}\end{exemple}
\begin{exemple}\pjya{qhɤjmbaʁ ɯ-jwaʁ nɯ ɕphɤrɕphɤr ʑo pa}\hspace{5pt}\pcmn{酸模的叶子又宽又大}\end{exemple}\end{entrée}

\begin{entrée}{ɕphɤt}{}{ⓔɕphɤt} 
\classe{vt} \paradigme{dir}{kɤ-}\paradigme{dir}{pɯ-}\paradigme{dir}{kɤ-}\paradigme{dir}{pɯ-}
\begin{définition}\pfra{réparer (un habit)}\end{définition}
\begin{définition}\pcmn{补}\end{définition}
\begin{définition}\pfra{réparer les habits}\end{définition}
\begin{définition}\pcmn{补衣服}\end{définition}
\begin{exemple}\pjya{kɤ-ɕphat-a, pɯ-ɕphat-a}\hspace{5pt}\pcmn{我补了}\end{exemple}
\begin{exemple}\pjya{ka-ɕphɤt}\hspace{5pt}\pcmn{他补了}\end{exemple}
\begin{exemple}\pjya{tɯ-ŋga pjɤ-ɴɢraʁ tɕe, kɤ-ɕphɤt}\hspace{5pt}\pcmn{衣服破了,你把它补一下}\end{exemple}\relationsémantique{参考}{\lien{ⓔtɤ-ɕphɤt}{tɤ-ɕphɤt}}
\begin{sous-entrée}{rɤɕphɤt}{ⓔɕphɤtⓝrɤɕphɤt} 
\classe{vi}  
\grammaire{apass} \end{sous-entrée}

\end{entrée}

\begin{entrée}{ɕphɣo}{}{ⓔɕphɣo} 
\classe{vt}  
\grammaire{caus}
\grammaire{refl} \paradigme{dir}{\_}\paradigme{dir}{\_}
\begin{définition}\pfra{emporter}\end{définition}
\begin{définition}\pcmn{拿走(不让人家发现)}\end{définition}
\begin{exemple}\pjya{jɤ-ɕphɣo-t-a, ɯʑo kɯ ja-ɕphɣo, jɤ-ɕphɣɤm}\hspace{5pt}\pcmn{我拿走了,他拿走了,你拿走吧}\end{exemple}
\begin{exemple}\pjya{kɯki laχtɕha ki nɤʑɯɣ ɯ́-ra nɤ jɤ-ɕphɣɤm ma tha mɤ-tɯ-βɟɤt}\hspace{5pt}\pcmn{这个东西,如果你需要的话,你就拿走吧,不然你就得不到}\end{exemple}
\begin{exemple}\pjya{a-sroʁ ju-ɕphɣam-a}\hspace{5pt}\pcmn{我要逃命}\end{exemple}
\begin{exemple}\pjya{kɤ-ɕphɣo nɯ, laχtɕha tú-wɣ-ndo tɕe jú-wɣ-tsɯm}\hspace{5pt}\pcmn{\lien{ⓔ\_ɕphɣo}{ɕphɣo}的意思就是拿了东西就带走了}\end{exemple}\relationsémantique{参考}{\lien{ⓔphɣo}{phɣo}}
\begin{sous-entrée}{ʑɣɤɕphɣo}{ⓔɕphɣoⓝʑɣɤɕphɣo} 
\classe{vi} \end{sous-entrée}

\begin{définition}\pfra{se sauver}\end{définition}
\begin{définition}\pcmn{逃命}\end{définition}
\begin{exemple}\pjya{jɤ-ʑɣɤɕphɣo-a}\hspace{5pt}\pcmn{我逃命了}\end{exemple}\end{entrée}

\begin{entrée}{ɕphɯɣɕphɯɣ}{}{ⓔɕphɯɣɕphɯɣ} 
\classe{idph.2} 
\begin{définition}\pfra{trempé}\end{définition}
\begin{définition}\pcmn{湿润(衣服)}\end{définition}
\begin{exemple}\pjya{a-ŋga ɕphɯɣɕphɯɣ ɲɤ-k-ɤci-ci}\hspace{5pt}\pcmn{我的衣服湿了}\end{exemple}
\begin{sous-entrée}{ɕphɯɣnɤɕphɯɣ}{ⓔɕphɯɣɕphɯɣⓝɕphɯɣnɤɕphɯɣ} 
\classe{idph.3} 
\begin{exemple}\pjya{a-taʁ tɯ-ci ɕphɯɣnɤɕphɯɣ ta-lɤt}\hspace{5pt}\pcmn{(互相泼水),他朝我洒水了(一次又一次)}\end{exemple}\end{sous-entrée}

\end{entrée}

\begin{entrée}{ɕplaɕpla}{}{ⓔɕplaɕpla} 
\classe{idph.2} \sens{1}
\begin{définition}\pfra{sans poil, chauve}\end{définition}
\begin{définition}\pcmn{形容光秃无毛的样子}\end{définition}\sens{2}
\begin{définition}\pfra{obstiné}\end{définition}
\begin{définition}\pcmn{形容固执的样子}\end{définition}
\begin{exemple}\pjya{ɕplaɕpla ma-tɯ-ʑɣɤstu}\hspace{5pt}\pcmn{你不要那么固执,不听指挥}\end{exemple}\end{entrée}

\begin{entrée}{ɕploʁɕploʁ}{}{ⓔɕploʁɕploʁ} 
\classe{idph.2} 
\begin{définition}\pfra{rond et lisse}\end{définition}
\begin{définition}\pcmn{形容圆而光滑的样子}\end{définition}
\begin{exemple}\pjya{nɤ-rte nɯ ɕploʁɕploʁ ʑo ɲɯ-pa}\hspace{5pt}\pcmn{你的帽子很圆,没有帽边}\end{exemple}\relationsémantique{参考}{\lien{}{χploχploʁ}}\end{entrée}

\begin{entrée}{ɕpɯt}{}{ⓔɕpɯt} 
\classe{vt} \sens{1}\paradigme{dir}{thɯ-}
\begin{définition}\pfra{élever}\end{définition}
\begin{définition}\pcmn{养育;抚养成大}\end{définition}
\begin{exemple}\pjya{ɯʑo kɯ tɯrme ɯ-rɟit ci tha-ɕpɯt}\hspace{5pt}\pcmn{他养了别人的孩子}\end{exemple}\sens{2}\paradigme{dir}{kɤ-}
\begin{définition}\pfra{adopter}\end{définition}
\begin{définition}\pcmn{收养}\end{définition}
\begin{exemple}\pjya{thɯ-ɕpɯt-a, kɤ-ɕpɯt, kɤ-tɯ-ɕpɯt, ɯʑo kɯ ka-ɕpɯt}\hspace{5pt}\pcmn{我收养了,你收养了,他收养了}\end{exemple}
\begin{exemple}\pjya{kɤ-ɕpɯt sɤcha}\hspace{5pt}\pcmn{可以养}\end{exemple}
\begin{exemple}\pjya{χpɤltɕin kɯ @wugui ka-ɕpɯt}\hspace{5pt}\pcmn{柏尔青养了乌龟}\end{exemple}
\begin{exemple}\pjya{ɯʑo kɯ fsapaʁ ci ka-ɕpɯt}\hspace{5pt}\pcmn{他养了动物}\end{exemple}
\begin{exemple}\pjya{rɯdaʁ cho fsapaʁ ra kɤ-ɕpɯt ɯ-tɯ-ɴqa mɤ-naχtɕɯɣ, rɯdaʁ kɤ-ɕpɯt ɴqa, fsapaʁ ra kɤ-ɕpɯt mbat}\hspace{5pt}\pcmn{养家畜跟养野生动物难度不同,养家畜容易,养野生动物难。}\end{exemple}\end{entrée}

\begin{entrée}{ɕqɤjɤr}{}{ⓔɕqɤjɤr} 
\classe{n} 
\begin{définition}\pfra{personne qui louche}\end{définition}
\begin{définition}\pcmn{斜眼}\end{définition}\end{entrée}

\begin{entrée}{ɕqɤnɕqɤn}{}{ⓔɕqɤnɕqɤn} 
\classe{idph.2} 
\begin{définition}\pfra{rouge}\end{définition}
\begin{définition}\pcmn{形容红色,没有光的样子}\end{définition}
\begin{exemple}\pjya{prɤɲi nɯ ɕqɤnɕqɤn ʑo ɲɯ-ɣɯrni}\hspace{5pt}\pcmn{晚霞是红色的}\end{exemple}\end{entrée}

\begin{entrée}{ɕqɤt}{}{ⓔɕqɤt} 
\classe{idph.1} \sens{1}
\begin{définition}\pfra{bruit d'une pierre jetée contre une surface dure}\end{définition}
\begin{définition}\pcmn{石头扔在硬的平面上发出的声音}\end{définition}\sens{2}
\begin{définition}\pfra{douleur soudaine}\end{définition}
\begin{définition}\pcmn{形容突然痛起来的感觉}\end{définition}
\begin{exemple}\pjya{ɕqɤt ʑo ɲɯ-ti}\hspace{5pt}\pcmn{突然痛起来}\end{exemple}
\begin{exemple}\pjya{a-ku rdɤstaʁ ɕqɤt ʑo ta-lɤt}\hspace{5pt}\pcmn{他朝我扔了一块石头,啪的一声打到我头上}\end{exemple}\end{entrée}

\begin{entrée}{ɕquɕqu}{}{ⓔɕquɕqu} 
\classe{idph.2} 
\begin{définition}\pfra{en fronçant les sourcils}\end{définition}
\begin{définition}\pcmn{形容皱着眉头的样子}\end{définition}
\begin{exemple}\pjya{a-rŋa ɕquɕqu ʑo ku-ru ɲɯ-ŋu}\hspace{5pt}\pcmn{他皱着眉头地盯着我}\end{exemple}\relationsémantique{参考}{\lien{ⓔqlɯqlɯ}{qlɯqlɯ}}\relationsémantique{参考}{\lien{ⓔɕqhɯɕqhi}{ɕqhɯɕqhi}}\end{entrée}

\begin{entrée}{ɕqudoŋ}{}{ⓔɕqudoŋ} 
\classe{n} 
\begin{définition}\pfra{aveugle (insulte)}\end{définition}
\begin{définition}\pcmn{瞎子(骂人的话)}\end{définition}\end{entrée}

\begin{entrée}{ɕqhaloʁ}{}{ⓔɕqhaloʁ} 
\classe{n} 
\begin{définition}\pfra{bâton qui sert à caler la porte}\end{définition}
\begin{définition}\pcmn{门闩}\end{définition}
\begin{exemple}\pjya{ɕqhaloʁ nɯ-lat-a}\hspace{5pt}\pcmn{我拴了门}\end{exemple}\end{entrée}

\begin{entrée}{ɕqhlɤt}{}{ⓔɕqhlɤt} 
\classe{vi}  
\grammaire{refl}
\grammaire{caus} \paradigme{dir}{pɯ-}\paradigme{dir}{\_}\sens{1}
\begin{définition}\pfra{disparaître}\end{définition}
\begin{définition}\pcmn{消失}\end{définition}
\begin{exemple}\pjya{nɤ-skɤt nɯ ju-ɕqhlɤt kɯ-fse li ju-nɯɬoʁ kɯ-fse ɲɯ-ŋu}\hspace{5pt}\pcmn{你的声音好像一会儿消失,一会儿又出现了(电话连接有问题)}\end{exemple}\sens{2}
\begin{définition}\pfra{tomber dans}\end{définition}
\begin{définition}\pcmn{掉进去}\end{définition}
\begin{exemple}\pjya{jɤ-ari tɕe jɤ-ɕqhlɤt}\hspace{5pt}\pcmn{他去了就掉进去了}\end{exemple}
\begin{exemple}\pjya{nɤki tʂu kɯspoʁ nɯtɕu tɯ-ɕqhlɤt nɤ !}\hspace{5pt}\pcmn{路中有洞,小心不要掉进去}\end{exemple}
\begin{exemple}\pjya{aʑo kɯspoʁ mɯ-pjɤ-mto-t-a tɕe pjɤ-ɕqhlat-a}\hspace{5pt}\pcmn{我没有看见有洞,就掉进去了}\end{exemple}\sens{3}\paradigme{dir}{\_}
\begin{définition}\pfra{sombrer}\end{définition}
\begin{définition}\pcmn{沉下去}\end{définition}
\begin{sous-entrée}{sɯɕqhlɤt}{ⓔɕqhlɤtⓢ3ⓝsɯɕqhlɤt} 
\classe{vt} 
\begin{définition}\pfra{faire disparaître}\end{définition}
\begin{définition}\pcmn{使消失}\end{définition}\end{sous-entrée}

\begin{sous-entrée}{ʑɣɤsɯɕqhlɤt}{ⓔɕqhlɤtⓢ3ⓝʑɣɤsɯɕqhlɤt} 
\classe{vi} \end{sous-entrée}

\begin{définition}\pfra{faire en sorte de disparaître}\end{définition}
\begin{définition}\pcmn{使自己消失}\end{définition}
\begin{exemple}\pjya{ɯʑo kɤ-ari nɤ kɤ-ari tɕe, kɤ-ʑɣɤsɯɕqhlɤt}\hspace{5pt}\pcmn{他走下去,走下去,最后不见了踪影}\end{exemple}\end{entrée}

\begin{entrée}{ɕqhɯɕqhi}{}{ⓔɕqhɯɕqhi} 
\classe{idph.2} 
\begin{définition}\pfra{regardant en travers}\end{définition}
\begin{définition}\pcmn{形容斜着眼睛看的样子}\end{définition}
\begin{exemple}\pjya{ɕqhɯɕqhi ʑo ku-ru ɲɯ-ŋu}\hspace{5pt}\pcmn{他斜着眼睛看着他}\end{exemple}\relationsémantique{参考}{\lien{ⓔɕquɕqu}{ɕquɕqu}}\relationsémantique{参考}{\lien{ⓔqlɯqlɯ}{qlɯqlɯ}}\end{entrée}

\begin{entrée}{ɕqlɤβɕqlɤβ}{}{ⓔɕqlɤβɕqlɤβ} 
\classe{idph.2} 
\begin{définition}\pfra{ayant un air mécontent}\end{définition}
\begin{définition}\pcmn{形容不高兴的眼神的样子}\end{définition}
\begin{exemple}\pjya{ɯʑo ɯ-sɯm mɯ́j-ɕe tɕe, a-ɕki ɕqlɤβɕqlɤβ ʑo ju-ru ɲɯ-ŋu}\hspace{5pt}\pcmn{他用不高兴的眼神看着我}\end{exemple}\end{entrée}

\begin{entrée}{ɕqlɯβnɤɕqlɯβ}{}{ⓔɕqlɯβnɤɕqlɯβ} 
\classe{idph.3} 
\begin{définition}\pfra{bruit de personne qui marche dans l'eau, bruit produit lorsque l'on craque des doigts}\end{définition}
\begin{définition}\pcmn{在水里走的声音,大口地喝水的声音}\end{définition}
\begin{exemple}\pjya{ɕqlɯβnɤɕqlɯβ kɤ-tshi-t-a}\hspace{5pt}\pcmn{我咕噜咕噜地喝了}\end{exemple}
\begin{exemple}\pjya{ɯ-jaʁndzu ɕqlɯβnɤɕqlɯβ ʑo ta-ʑmbri}\hspace{5pt}\pcmn{他把手指关节拉得嘎嘣嘎嘣响}\end{exemple}\relationsémantique{参考}{\lien{ⓔqlɯβ}{qlɯβ}}\end{entrée}

\begin{entrée}{ɕqraʁ}{}{ⓔɕqraʁ} 
\classe{vs} \paradigme{dir}{tɤ-}\paradigme{dir}{tɤ-}\paradigme{dir}{tɤ-}
\begin{définition}\pfra{intelligent}\end{définition}
\begin{définition}\pcmn{聪明}\end{définition}
\begin{définition}\pfra{rendre intelligent}\end{définition}
\begin{définition}\pcmn{令……变聪明}\end{définition}
\begin{définition}\pfra{se rendre intelligent}\end{définition}
\begin{définition}\pcmn{令自己变聪明}\end{définition}
\begin{exemple}\pjya{ɲɯ-ɕqraʁ}\hspace{5pt}\pcmn{他很聪明}\end{exemple}
\begin{exemple}\pjya{tɤkhe kɤ-sɯɕqraʁ ra}\hspace{5pt}\pcmn{要令笨蛋变聪明}\end{exemple}
\begin{exemple}\pjya{tɯʑo tu-kɯ-ʑɣɤsɯɕqraʁ ra}\hspace{5pt}\pcmn{要想办法令自己变聪明}\end{exemple}\relationsémantique{参考}{\lien{ⓔznɤɕqɯɕqraʁ}{znɤɕqɯɕqraʁ}}
\begin{sous-entrée}{sɯɕqraʁ}{ⓔɕqraʁⓝsɯɕqraʁ} 
\classe{vt} \end{sous-entrée}

\begin{sous-entrée}{nɤɕqraʁ}{ⓔɕqraʁⓝnɤɕqraʁ} 
\classe{vt}  
\grammaire{trop} 
\begin{définition}\pfra{trouver intelligent}\end{définition}
\begin{définition}\pcmn{觉得聪明}\end{définition}
\begin{exemple}\pjya{ɯʑo ndɤre, aʑo wuma ɲɯ-nɤɕqraʁ-a}\hspace{5pt}\pcmn{我觉得他很聪明}\end{exemple}\end{sous-entrée}

\begin{sous-entrée}{ʑɣɤsɯɕqraʁ}{ⓔɕqraʁⓝʑɣɤsɯɕqraʁ} 
\classe{vi} \end{sous-entrée}

\end{entrée}

\begin{entrée}{ɕquwa}{}{ⓔɕquwa} 
\classe{n} 
\begin{définition}\pfra{aveugle}\end{définition}
\begin{définition}\pcmn{瞎子}\end{définition}
\begin{exemple}\pjya{ɕquwa nɯ-aβzu-a}\hspace{5pt}\pcmn{我瞎了}\end{exemple}\relationsémantique{参考}{\lien{ⓔaɕquwa}{aɕquwa}}\end{entrée}

\begin{entrée}{ɕur}{}{ⓔɕur} 
\classe{vi} \paradigme{dir}{pɯ-}
\begin{définition}\pfra{payer une amende}\end{définition}
\begin{définition}\pcmn{罚款}\end{définition}\paradigme{dir}{pɯ-}
\begin{définition}\pfra{faire payer une amende}\end{définition}
\begin{définition}\pcmn{让人交罚款}\end{définition}
\begin{exemple}\pjya{tʂu tɕe, ɯ-stu tu-kɯ-ŋke ra ma @fakuan kɯ-fse ɕur}\hspace{5pt}\pcmn{路上不要违反交通规则,不然就会(被)罚款}\end{exemple}
\begin{exemple}\pjya{ɯʑo kɯ @qiche ko-sɯrpu tɕe pjɤ-ɕur}\hspace{5pt}\pcmn{他撞了车,所以被罚款}\end{exemple}
\begin{exemple}\pjya{pɕawtʂɯ ɲo-kho pjɤ-ra, pɕawtsɯ khro pjɤ-ɕur}\hspace{5pt}\pcmn{罚了很多钱}\end{exemple}
\begin{exemple}\pjya{ɕ-to-olɯlɤt tɕe, rŋɯl ɣurʑa pjɤ-ɕur}\hspace{5pt}\pcmn{因为他跟别人打架了,被罚了一百块}\end{exemple}
\begin{sous-entrée}{sɯxɕur}{ⓔɕurⓝsɯxɕur} 
\classe{vt} \end{sous-entrée}

\étymologie{ɕor}\end{entrée}

\begin{entrée}{ɕri}{₁}{ⓔɕriⓗ1} 
\classe{vs} 
\begin{définition}\pfra{avoir une fuite}\end{définition}
\begin{définition}\pcmn{(从小的洞或缝里)漏出来}\end{définition}
\begin{exemple}\pjya{tɯ-ci pjɤ-ɕri}\hspace{5pt}\pcmn{漏了水}\end{exemple}
\begin{exemple}\pjya{qale ɲɤ-ɕri}\hspace{5pt}\pcmn{漏了空气}\end{exemple}
\begin{exemple}\pjya{tɯthɯ cho tɯŋgu ni ndʑi-pɤrthɤβ nɯtɕu, tɤjlɤβ mɤ-kɯ-ɕri tu-βzu-nɯ}\hspace{5pt}\pcmn{把锅子和锅子之间的缝隙密封住,不让水蒸气冒出来}\end{exemple}\relationsémantique{参考}{\lien{ⓔaɕoʁri}{aɕoʁri}}\end{entrée}

\begin{entrée}{ɕri}{₂}{ⓔɕriⓗ2} 
\classe{vt} \paradigme{dir}{pɯ-}\paradigme{dir}{nɯ-}
\begin{définition}\pfra{coudre}\end{définition}
\begin{définition}\pcmn{交叉着缝}\end{définition}
\begin{exemple}\pjya{aʑo tɤ-sno ɯ-jaʁ pɯ-ɕri-t-a}\hspace{5pt}\pcmn{我缝了马鞍的垫子}\end{exemple}\end{entrée}

\begin{entrée}{ɕʁɯznɤɕʁɯz}{}{ⓔɕʁɯznɤɕʁɯz} 
\classe{idph.3} 
\begin{définition}\pfra{croquant}\end{définition}
\begin{définition}\pcmn{形容吃东西发出声音,干而脆的样子}\end{définition}\end{entrée}

\begin{entrée}{ɕte}{}{ⓔɕte} 
\classe{vt} \paradigme{dir}{nɯ-}
\begin{définition}\pfra{contaminer, infecter}\end{définition}
\begin{définition}\pcmn{传染}\end{définition}
\begin{exemple}\pjya{ɲɤ-ɕte-t-a, ɲɤ-tɯ-ɕte-t, ɲɤ-ɕte}\hspace{5pt}\pcmn{我传染给他,你传染给他,他传染给他}\end{exemple}
\begin{exemple}\pjya{tɯ́-wɣ-ɕte}\hspace{5pt}\pcmn{他会传染给你}\end{exemple}
\begin{exemple}\pjya{ɯʑo ɲɯ-nɯtɕhomba tɕe ɲɤ́-wɣ-ɕte-a}\hspace{5pt}\pcmn{他把感冒传染给我了}\end{exemple}
\begin{exemple}\pjya{a-tɕhomba ɣɤʑu tɕe, ɲɤ-ɕte-t-a}\hspace{5pt}\pcmn{我感冒了,传染给他(传染给你了)}\end{exemple}
\begin{sous-entrée}{aɕtɯɕte}{ⓔɕteⓝaɕtɯɕte} 
\classe{vi} 
\begin{définition}\pfra{se contaminer les uns les autres}\end{définition}
\begin{définition}\pcmn{互相传染}\end{définition}
\begin{exemple}\pjya{tɯ-ŋgo nɯ ɲɤ-k-ɤɕtɯɕte-nɯ-ci}\hspace{5pt}\pcmn{他们互相传染了}\end{exemple}\end{sous-entrée}

\begin{sous-entrée}{sɤɕte}{ⓔɕteⓝsɤɕte} 
\classe{vs} 
\begin{définition}\pfra{qui se transmet facilement (maladie)}\end{définition}
\begin{définition}\pcmn{容易传染}\end{définition}
\begin{exemple}\pjya{ki tɯ-ŋgo ki ɲɯ-sɤɕte tɕe kɤ-rɯndzaŋspa ɲɯ-ra}\hspace{5pt}\pcmn{这种病容易传染,要注意}\end{exemple}\end{sous-entrée}

\end{entrée}

\begin{entrée}{ɕthrɤβɕthrɤβ}{}{ⓔɕthrɤβɕthrɤβ} 
\classe{idph.2} 
\begin{définition}\pfra{objet long et mou}\end{définition}
\begin{définition}\pcmn{形容又破烂又长的样子}\end{définition}
\begin{exemple}\pjya{tɯ-ŋga ɕthrɤβɕthrɤβ ʑo kɯ-pa to-ŋga}\hspace{5pt}\pcmn{他穿了又破烂又长的衣服}\end{exemple}
\begin{exemple}\pjya{tɤ-jwaʁ pjɤ-lni ʑo ɕthrɤβɕthrɤβ}\hspace{5pt}\pcmn{叶子受热后耷拉着,显得很柔软的样子}\end{exemple}\relationsémantique{参考}{\lien{ⓔʑdrɤβʑdrɤβ}{ʑdrɤβʑdrɤβ}}\end{entrée}

\begin{entrée}{ɕthɯz}{}{ⓔɕthɯz} 
\classe{vt} \sens{1}\paradigme{dir}{\_}
\begin{définition}\pfra{tourner vers}\end{définition}
\begin{définition}\pcmn{朝向;对着}\end{définition}
\begin{exemple}\pjya{ɯʑo kɯ ko-ɕthɯz, ka-ɕthɯz}\hspace{5pt}\pcmn{对着他,朝向他}\end{exemple}
\begin{exemple}\pjya{nɤ-ɕki ku-ɕthɯz-a}\hspace{5pt}\pcmn{我把它朝向你}\end{exemple}
\begin{exemple}\pjya{nɤki kɤ-ɕthɯz tɕe a-pɯ-mtam-a}\hspace{5pt}\pcmn{给我看一下这个东西}\end{exemple}
\begin{exemple}\pjya{ɯʑo kɯ tɯ-pɤr nɯ ka-ɕthɯz tɕe pɯ-mto-t-a}\hspace{5pt}\pcmn{他把照片对着(我),我就看了一下}\end{exemple}
\begin{exemple}\pjya{ɕɤɣ ɯ-khɯ tɤ-nɯ-ɕthɯz-a}\hspace{5pt}\pcmn{我闻了柏树的烟子(治病的方法)}\end{exemple}\sens{2}
\begin{définition}\pfra{prendre (eau)}\end{définition}
\begin{définition}\pcmn{接住(水)}\end{définition}
\begin{sous-entrée}{nɤɕthɯɕthɯz}{ⓔɕthɯzⓢ2ⓝnɤɕthɯɕthɯz} 
\classe{vt} 
\begin{définition}\pfra{montrer à tout le monde}\end{définition}
\begin{définition}\pcmn{给大家看}\end{définition}\relationsémantique{参考}{\lien{ⓔʑɣɤɕthɯz}{ʑɣɤɕthɯz}}\end{sous-entrée}

\end{entrée}

\begin{entrée}{ɕtraŋɕtraŋ}{}{ⓔɕtraŋɕtraŋ} 
\classe{idph.2} 
\begin{définition}\pfra{mou et long}\end{définition}
\begin{définition}\pcmn{形容又软又长的样子}\end{définition}
\begin{exemple}\pjya{razmbe ɕtraŋɕtraŋ ʑo ɲɯ-ɴqoʁ}\hspace{5pt}\pcmn{烂布条又长又脏地挂在那里}\end{exemple}
\begin{sous-entrée}{ɕtraŋnɤɕtraŋ}{ⓔɕtraŋɕtraŋⓝɕtraŋnɤɕtraŋ} 
\classe{idph.3} 
\begin{exemple}\pjya{ɕtraŋnɤɕtraŋ ʑo ɲɯ-ŋke}\end{exemple}\end{sous-entrée}

\begin{sous-entrée}{ɕtrɯŋɯɕtraŋi}{ⓔɕtraŋɕtraŋⓝɕtrɯŋɯɕtraŋi} 
\classe{idph.8} 
\begin{exemple}\pjya{tɯ-mbri ɕtrɯŋɯɕtraŋi ɲɯ-xcat}\hspace{5pt}\pcmn{绳子很多,很凌乱}\end{exemple}\end{sous-entrée}

\end{entrée}

\begin{entrée}{ɕtrɤβɕtrɤβ}{}{ⓔɕtrɤβɕtrɤβ} 
\classe{idph.2} 
\begin{définition}\pfra{objet long et mou}\end{définition}
\begin{définition}\pcmn{形容长而软,没有精神的样子}\end{définition}
\begin{exemple}\pjya{ɯʑo lo-βzi tɕe, ɕtrɤβɕtrɤβ ʑo ɲɯ-rŋgɯ}\hspace{5pt}\pcmn{他喝醉了,无精打采地瘫在那里}\end{exemple}\end{entrée}

\begin{entrée}{ɕtriɕtri}{}{ⓔɕtriɕtri} 
\classe{idph.2} 
\begin{définition}\pfra{mou, long et fin}\end{définition}
\begin{définition}\pcmn{形容柔软,细而长的样子}\end{définition}
\begin{exemple}\pjya{mɯntoʁ ɲɤ-lni ɕtriɕtri ʑo}\hspace{5pt}\pcmn{花晒蔫了,立不起来}\end{exemple}
\begin{exemple}\pjya{ɕtriɕtri ma-tɯ-ʑɣɤstu}\hspace{5pt}\pcmn{你不要这么蔫不拉叽}\end{exemple}\end{entrée}

\begin{entrée}{ɕtʂaŋɕtʂaŋ}{}{ⓔɕtʂaŋɕtʂaŋ} 
\classe{idph.2} \paradigme{dir}{tɤ-}\paradigme{dir}{tɤ-}
\begin{définition}\pfra{accroché}\end{définition}
\begin{définition}\pcmn{吊着}\end{définition}
\begin{définition}\pfra{se balancer}\end{définition}
\begin{définition}\pcmn{摇摆;摇晃(吊着的东西)}\end{définition}
\begin{définition}\pfra{balancer}\end{définition}
\begin{définition}\pcmn{甩来甩去}\end{définition}
\begin{exemple}\pjya{ɕtʂaŋɕtʂaŋ ɲɯ-ɴqoʁ}\hspace{5pt}\pcmn{吊着}\end{exemple}
\begin{exemple}\pjya{to-ɣɤɕtʂaŋlaŋ}\hspace{5pt}\pcmn{摇晃了}\end{exemple}
\begin{exemple}\pjya{laχtɕha to-ɕɯɴqoʁ tɕe, ɲɯ-ɣɤɕtʂaŋlaŋ}\hspace{5pt}\pcmn{挂着的东西在摇晃}\end{exemple}
\begin{exemple}\pjya{nɯ mɯ-pɯ-fse ri to-ɣɤɕtʂaŋlaŋ}\hspace{5pt}\pcmn{原来不是这样,现在就在那里吊着摇晃}\end{exemple}
\begin{exemple}\pjya{ɯ-jaʁ ci ɲɯ-sɤɕtʂaŋlaŋ}\hspace{5pt}\pcmn{他把手甩来甩去}\end{exemple}
\begin{sous-entrée}{ɕtʂaŋnɤɕtʂaŋ}{ⓔɕtʂaŋɕtʂaŋⓝɕtʂaŋnɤɕtʂaŋ} 
\classe{idph.3} 
\begin{exemple}\pjya{ɯ-jaʁ ɕtʂaŋnɤɕtʂaŋ ɲɯ-ɤsɯ-stu kɤ-ari}\hspace{5pt}\pcmn{他甩着就去了}\end{exemple}\end{sous-entrée}

\begin{sous-entrée}{ɕtʂaŋnɤlaŋ}{ⓔɕtʂaŋɕtʂaŋⓝɕtʂaŋnɤlaŋ} 
\classe{idph.4} 
\begin{exemple}\pjya{tɯ-nga to-ɕɯɴqoʁ-a tɕe, qale kɯ ɕtʂaŋnɤlaŋ ɲɯ-ɤsɯ-stu}\hspace{5pt}\pcmn{我挂了衣服,被风吹来吹去}\end{exemple}\end{sous-entrée}

\begin{sous-entrée}{ɣɤɕtʂaŋlaŋ}{ⓔɕtʂaŋɕtʂaŋⓝɣɤɕtʂaŋlaŋ} 
\classe{vi} \end{sous-entrée}

\begin{sous-entrée}{sɤɕtʂaŋlaŋ}{ⓔɕtʂaŋɕtʂaŋⓝsɤɕtʂaŋlaŋ} 
\classe{vt} \end{sous-entrée}

\end{entrée}

\begin{entrée}{ɕtʂat}{}{ⓔɕtʂat} 
\classe{vt}  
\grammaire{apass}
\grammaire{refl} \paradigme{dir}{nɯ-}\paradigme{dir}{thɯ-}\paradigme{dir}{nɯ-}
\begin{définition}\pfra{économiser}\end{définition}
\begin{définition}\pcmn{节省}\end{définition}
\begin{exemple}\pjya{nɯ-ɕtʂat-a, nɯ-tɯ-ɕtʂat, na-ɕtʂat}\hspace{5pt}\pcmn{我节约了,你节约了,他节约了}\end{exemple}
\begin{exemple}\pjya{kɯki laχtɕha nɯ kɤ-ɕtʂat ra}\hspace{5pt}\pcmn{这些东西要省着点用}\end{exemple}
\begin{exemple}\pjya{rŋɯl ra kɤ-ɕtʂat ra ma ʑatsa arɕo}\hspace{5pt}\pcmn{钱要省着点用,不然很快就用光了}\end{exemple}
\begin{exemple}\pjya{tɯ-ŋga nɯ-ɕtʂat}\hspace{5pt}\pcmn{你要节约穿衣}\end{exemple}
\begin{sous-entrée}{rɤɕtʂat}{ⓔɕtʂatⓝrɤɕtʂat} 
\classe{vi} \end{sous-entrée}

\paradigme{dir}{nɯ-}
\begin{définition}\pfra{économiser}\end{définition}
\begin{définition}\pcmn{节约用东西}\end{définition}
\begin{exemple}\pjya{ɯʑo wuma ɲɯ-rɤɕtʂat}\hspace{5pt}\pcmn{他很节约}\end{exemple}
\begin{sous-entrée}{ʑɣɤɕtʂat}{ⓔɕtʂatⓝʑɣɤɕtʂat} 
\classe{vi} \end{sous-entrée}

\begin{définition}\pfra{économiser ses forces}\end{définition}
\begin{définition}\pcmn{节省自己的体力}\end{définition}
\begin{exemple}\pjya{ɕɤxɕo ɲɯ-ʑɣɤɕtʂat-a ɲɯ-ntshi ma a-phoŋbu mɤ-cha ɲɯ-ŋu}\hspace{5pt}\pcmn{我这几天要节省自己的体力,因为身体不行}\end{exemple}\end{entrée}

\begin{entrée}{ɕtʂo}{}{ⓔɕtʂo} 
\classe{vt} \sens{1}\paradigme{dir}{tɤ-}
\begin{définition}\pfra{mesurer (avec une louche)}\end{définition}
\begin{définition}\pcmn{用瓢量(食物)}\end{définition}\sens{2}\paradigme{dir}{nɯ-}
\begin{définition}\pfra{mesurer}\end{définition}
\begin{définition}\pcmn{量}\end{définition}
\begin{exemple}\pjya{tɤ-ɕtʂɤm}\hspace{5pt}\pcmn{你量一下}\end{exemple}
\begin{exemple}\pjya{tɯ-kɤ-ndza nɯ ra kɤ-ɕtʂo kɯ-ra ɕti}\hspace{5pt}\pcmn{食物要量过(才能吃)}\end{exemple}
\begin{exemple}\pjya{kɯɕɯŋgɯ tɕe, rɟama pjɤ-me tɕe tú-wɣ-ɕtʂo pjɤ-ŋu}\hspace{5pt}\pcmn{古时候没有秤,都用瓢来量食物}\end{exemple}
\begin{sous-entrée}{rɤɕtʂo}{ⓔɕtʂoⓢ2ⓝrɤɕtʂo} 
\classe{vi}  
\grammaire{apass} 
\begin{définition}\pfra{mesurer des choses}\end{définition}
\begin{définition}\pcmn{量东西}\end{définition}\end{sous-entrée}

\begin{exemple}\pjya{ʑara ɣɯ ku-rɤɕtʂo-a}\hspace{5pt}\pcmn{我给他们量东西}\end{exemple}\end{entrée}

\begin{entrée}{ɕtʂɯ}{}{ⓔɕtʂɯ} 
\classe{vt} \paradigme{dir}{kɤ-}\paradigme{dir}{kɤ-}
\begin{définition}\pfra{confier, déposer}\end{définition}
\begin{définition}\pcmn{寄存}\end{définition}
\begin{définition}\pfra{confier, déposer}\end{définition}
\begin{définition}\pcmn{寄存}\end{définition}
\begin{exemple}\pjya{kɤ-ɕtʂɯ-t-a, kɤ-tɯ-ɕtʂɯt, ka-ɕtʂɯ}\hspace{5pt}\pcmn{我存在他那里了,你存在他那里了,他存在他那里了}\end{exemple}
\begin{exemple}\pjya{kɯki laχtɕha ki a-kɤ-tɯ́-wɣ-ɕtʂɯ tɕe, nɤʑo ɯ-pɯ tɤ-pe}\hspace{5pt}\pcmn{把这个东西存在你那里吧,你把它保管好}\end{exemple}
\begin{exemple}\pjya{kɯki laχtɕha ki nɤʑo ku-ta-ɕtʂɯ}\hspace{5pt}\pcmn{我把这个东西存在你那里}\end{exemple}
\begin{exemple}\pjya{ɕ-kú-wɣ-ɕtʂɯ}\hspace{5pt}\pcmn{有人去寄存}\end{exemple}
\begin{exemple}\pjya{ɕ-kɤ-ɕtʂi}\hspace{5pt}\pcmn{你存在他那里吧}\end{exemple}
\begin{exemple}\pjya{laχtɕha kɤ-rɤɕtʂɯ-a}\hspace{5pt}\pcmn{我把东西寄存起来了}\end{exemple}
\begin{sous-entrée}{rɤɕtʂɯ}{ⓔɕtʂɯⓝrɤɕtʂɯ} 
\grammaire{apass} \end{sous-entrée}

\end{entrée}

\begin{entrée}{ɕtʂɯɣɕtʂɯɣ}{}{ⓔɕtʂɯɣɕtʂɯɣ} 
\classe{idph.2} 
\begin{définition}\pfra{en suspension, pendant (objet)}\end{définition}
\begin{définition}\pcmn{形容吊着的东西}\end{définition}
\begin{sous-entrée}{ɕtʂɯɣnɤɕtʂɯɣ}{ⓔɕtʂɯɣɕtʂɯɣⓝɕtʂɯɣnɤɕtʂɯɣ} 
\classe{idph.3} 
\begin{définition}\pfra{balancer}\end{définition}
\begin{définition}\pcmn{摆动着}\end{définition}
\begin{exemple}\pjya{ɯ-jaʁ ɕtʂɯɣnɤɕtʂɯɣ ɲɯ-ɤsɯ-stu kɤ-ari}\hspace{5pt}\pcmn{他摆动着手就去了}\end{exemple}\end{sous-entrée}

\begin{sous-entrée}{sɤɕtʂɯlɯɣ}{ⓔɕtʂɯɣɕtʂɯɣⓝsɤɕtʂɯlɯɣ} 
\classe{vt} 
\begin{définition}\pfra{balancer}\end{définition}
\begin{définition}\pcmn{摆动着}\end{définition}
\begin{exemple}\pjya{ɯ-jaʁ laχtɕha ɲɯ-sɤɕtʂɯlɯɣ kɤ-ari}\hspace{5pt}\pcmn{他手上摆动着东西就去了}\end{exemple}\end{sous-entrée}

\end{entrée}

\begin{entrée}{ɕɯ}{}{ⓔɕɯ} 
\classe{pro} 
\begin{définition}\pfra{qui}\end{définition}
\begin{définition}\pcmn{谁}\end{définition}
\begin{exemple}\pjya{andi tɯrme ci jo-ɣi tɕe, ɕɯ ci ku-nnɯ-ŋu kɯma?}\hspace{5pt}\pcmn{那边来了人,是谁呢?}\end{exemple}
\begin{sous-entrée}{ɕɯmɤɕɯ}{ⓔɕɯⓝɕɯmɤɕɯ}
\begin{définition}\pfra{qui que ce soit}\end{définition}
\begin{définition}\pcmn{无论谁都……}\end{définition}
\begin{exemple}\pjya{(ɯ-xtsa) ɲɯ-nɯrge tɕe ɕɯmɤɕɯ ʑo tu-sɯrtoʁ ŋu.}\hspace{5pt}\pcmn{他很喜欢他的鞋子,到处炫耀给别人看}\end{exemple}\relationsémantique{参考}{\lien{ⓔŋotɕuŋondɤt}{ŋotɕuŋondɤt}}\end{sous-entrée}

\end{entrée}

\begin{entrée}{ɕɯchapaja}{}{ⓔɕɯchapaja} 
\classe{adv} 
\begin{définition}\pfra{lutter pour ne pas être le dernier}\end{définition}
\begin{définition}\pcmn{争先恐后}\end{définition}
\begin{exemple}\pjya{ɕɯchapaja ʑo jo-phɣo-nɯ}\hspace{5pt}\pcmn{他们争先恐后地逃了}\end{exemple}\end{entrée}

\begin{entrée}{ɕɯdo}{}{ⓔɕɯdo} 
\classe{n} 
\begin{définition}\pfra{bol en bois}\end{définition}
\begin{définition}\pcmn{木碗}\end{définition}\end{entrée}

\begin{entrée}{ɕɯfka}{}{ⓔɕɯfka}\relationsémantique{参考}{\lien{ⓔfkaⓗ1}{fka₁}}\end{entrée}

\begin{entrée}{ɕɯfkaβ}{}{ⓔɕɯfkaβ} 
\classe{vt} \relationsémantique{参考}{\lien{ⓔfkaβ}{fkaβ}}\end{entrée}

\begin{entrée}{ɕɯftaʁ}{}{ⓔɕɯftaʁ} 
\classe{vt} \paradigme{dir}{kɤ-}
\begin{définition}\pfra{se souvenir}\end{définition}
\begin{définition}\pcmn{记得;记住}\end{définition}
\begin{exemple}\pjya{kɤ-ɕɯftaʁ-a, kɤ-tɯ-ɕɯftaʁ, ka-ɕɯftaʁ}\hspace{5pt}\pcmn{我记住了,你记住了,他记住了}\end{exemple}
\begin{exemple}\pjya{tɤ-scoz nɯ koŋla ʑo kú-wɣ-rtoʁ tɕe, ɲɯ́-wɣ-sɯɣʑaʁ mɤɕtʂa kɤ-ɕɯftaʁ mɤ-sɤcha}\hspace{5pt}\pcmn{书要反复看,反复复习才能记得住}\end{exemple}
\begin{exemple}\pjya{tɯ-rju kɤ-ɕɯftaʁ ɴqa}\hspace{5pt}\pcmn{句子很难记住}\end{exemple}
\begin{exemple}\pjya{pɯ-kɯ-sɯxɕat-a nɯ ku-ɕɯftaʁ-a ɲɯ-ra}\hspace{5pt}\pcmn{我要记住你教给我的(知识)}\end{exemple}\relationsémantique{反义词}{\lien{ⓔjmɯt}{jmɯt}}
\begin{sous-entrée}{ɣɤɕɯftaʁ}{ⓔɕɯftaʁⓝɣɤɕɯftaʁ} 
\classe{vs}  
\grammaire{facil} 
\begin{définition}\pfra{avoir une bonne mémoire}\end{définition}
\begin{définition}\pcmn{记性好}\end{définition}
\begin{exemple}\pjya{ɲɯ-ɕqraʁ tɕe ɲɯ-ɣɤɕɯftaʁ}\hspace{5pt}\pcmn{他很聪明,记性很好}\end{exemple}\end{sous-entrée}

\begin{sous-entrée}{nɯɣɯɕɯftaʁ}{ⓔɕɯftaʁⓝnɯɣɯɕɯftaʁ} 
\classe{vs} 
\begin{définition}\pfra{facile à mémoriser}\end{définition}
\begin{définition}\pcmn{容易记住}\end{définition}\relationsémantique{反义词}{\lien{ⓔjmɯtⓝnɯɣɯjmɯt}{nɯɣɯjmɯt}}\end{sous-entrée}

\étymologie{btags}\end{entrée}

\begin{entrée}{ɕɯftɯɣ}{}{ⓔɕɯftɯɣ} 
\classe{vt} \paradigme{dir}{pɯ-}\paradigme{dir}{tɤ-}
\begin{définition}\pfra{achever}\end{définition}
\begin{définition}\pcmn{完成}\end{définition}
\begin{exemple}\pjya{pɯ-ɕɯftɯɣ-a, pɯ-tɯ-ɕɯftɯɣ, pa-ɕɯftɯɣ}\hspace{5pt}\pcmn{我完成了,你完成了,他完成了}\end{exemple}
\begin{exemple}\pjya{kɯki ɯ-ro kɯ-dɤn me tɕe, pɯ-ɕɯftɯɣ}\hspace{5pt}\pcmn{剩下的不多,你把它完成吧}\end{exemple}
\begin{exemple}\pjya{nɤ-kɤ-nɤma pjɯ-kɤ-ɕɯftɯɣ ci pɯ-ri}\hspace{5pt}\pcmn{你的工作只剩下快要结束的那一段}\end{exemple}
\begin{exemple}\pjya{@xingqiliu tɕɤn ɕɯftɯɣ-tɕi}\hspace{5pt}\pcmn{我们俩星期六(把这个工作)做完}\end{exemple}\relationsémantique{参考}{\lien{}{caus}}\relationsémantique{参考}{\lien{ⓔftɯɣ}{ftɯɣ}}\end{entrée}

\begin{entrée}{ɕɯɣɕɯɣ}{}{ⓔɕɯɣɕɯɣ} 
\classe{idph.2} 
\begin{définition}\pfra{en silence}\end{définition}
\begin{définition}\pcmn{形容安静,毫不作声的样子}\end{définition}\end{entrée}

\begin{entrée}{ɕɯɣmu}{}{ⓔɕɯɣmu} 
\classe{vt}  
\grammaire{caus} \paradigme{dir}{nɯ-}\sens{1}
\begin{définition}\pfra{effrayer}\end{définition}
\begin{définition}\pcmn{吓唬}\end{définition}
\begin{exemple}\pjya{nɤ-sɤ-ɕɯɣmu tu-βze-a pɯ-ɕti ma ɯ-stu pɯ-maʁ}\hspace{5pt}\pcmn{我只是吓唬一下,不是真的}\end{exemple}
\begin{exemple}\pjya{ɯ-sɤ-ɕɯɣmu to-βzu}\hspace{5pt}\pcmn{对他说了吓人的话,做了吓人的动作}\end{exemple}\sens{2}
\begin{définition}\pfra{menacer}\end{définition}
\begin{définition}\pcmn{威胁}\end{définition}
\begin{exemple}\pjya{ɯʑo to-mɯrkɯ tɕe, aʑo kɯ ɕɯ́-wɣ-ndʑɯ-a ɲɯ-sɯsɤm tɕe, a-sɤ-ɕɯɣmu ɲɯ-ɤsɯ-βzu (ɲɯ́-wɣ-ɕɯɣmu-a ɲɯ-ŋu)}\hspace{5pt}\pcmn{他偷了东西,怕我去告状就威胁了我}\end{exemple}
\begin{sous-entrée}{ʑɣɤɕɯɣmu}{ⓔɕɯɣmuⓢ2ⓝʑɣɤɕɯɣmu} 
\classe{vi}  
\grammaire{refl}
\grammaire{caus} 
\begin{définition}\pfra{s'effrayer soi-même}\end{définition}
\begin{définition}\pcmn{自己吓唬自己}\end{définition}
\begin{exemple}\pjya{nɤʑo tɯ-ʑɣɤɕɯɣmu mɤ-ra ma mɤ-ʁdɯɣ}\hspace{5pt}\pcmn{你不用害怕,没有事}\end{exemple}\relationsémantique{参考}{\lien{ⓔmuⓗ1}{mu₁}}\end{sous-entrée}

\end{entrée}

\begin{entrée}{ɕɯɣra}{}{ⓔɕɯɣra} 
\classe{n} 
\begin{définition}\pfra{crible à gros trous}\end{définition}
\begin{définition}\pcmn{粗罗筛}\end{définition}\relationsémantique{同义词}{\lien{ⓔtshaʁ}{tshaʁ}}\relationsémantique{参考}{\lien{ⓔsɯɕɯɣra}{sɯɕɯɣra}}\end{entrée}

\begin{entrée}{ɕɯjaʁ}{}{ⓔɕɯjaʁ} 
\classe{n} 
\begin{définition}\pfra{poutre}\end{définition}
\begin{définition}\pcmn{走檐上的横梁}\end{définition}\relationsémantique{同义词}{\lien{ⓔtɤsthoʁsi}{tɤsthoʁsi}}\end{entrée}

\begin{entrée}{ɕɯkhuj}{}{ⓔɕɯkhuj} 
\classe{n} 
\begin{définition}\pfra{petite bassine avec un verseur}\end{définition}
\begin{définition}\pcmn{有嘴的小盆子}\end{définition}\end{entrée}

\begin{entrée}{ɕɯm}{}{ⓔɕɯm} 
\classe{vt} \paradigme{dir}{kɤ-}\paradigme{dir}{pɯ-}
\begin{définition}\pfra{couver}\end{définition}
\begin{définition}\pcmn{孵}\end{définition}
\begin{définition}\pfra{dormir avec (un enfant)}\end{définition}
\begin{définition}\pcmn{(跟孩子)一起睡}\end{définition}
\begin{exemple}\pjya{tɤ-pɤtso pɯ-ɕɯm-a}\hspace{5pt}\pcmn{我让孩子跟我一起睡}\end{exemple}
\begin{exemple}\pjya{kumpɣa kɯ tɤ-ŋgɯm ko-ɕɯm}\hspace{5pt}\pcmn{母鸡把蛋孵出来了}\end{exemple}\end{entrée}

\begin{entrée}{ɕɯmɤɕɯ}{}{ⓔɕɯmɤɕɯ}\relationsémantique{参考}{\lien{ⓔɕɯ}{ɕɯ}}\end{entrée}

\begin{entrée}{ɕɯmbɣom}{}{ⓔɕɯmbɣom} 
\classe{vt}  
\grammaire{caus}
\grammaire{caus}
\grammaire{refl} \paradigme{dir}{tɤ-}\paradigme{dir}{tɤ-}
\begin{définition}\pfra{faire plus vite}\end{définition}
\begin{définition}\pcmn{加快速度}\end{définition}
\begin{exemple}\pjya{kɯki tɤ-scoz ɲɯ-mbɣom, tɤ-ɕɯmbɣom ʑo pɯ-rɤt}\hspace{5pt}\pcmn{这封信很急,你要写得快些}\end{exemple}
\begin{exemple}\pjya{ki laχtɕha ki kɤ-ndo ra ŋu ŋu nɤ, tɤ-ɕɯmbɣom ʑo jɤ-ɣɯt ma aʑo ɕe-a ŋu}\hspace{5pt}\pcmn{这个东西要带的话,你快一点带来,不然我要走了}\end{exemple}\relationsémantique{参考}{\lien{ⓔmbɣom}{mbɣom}}
\begin{sous-entrée}{ʑɣɤɕɯmbɣom}{ⓔɕɯmbɣomⓝʑɣɤɕɯmbɣom} 
\classe{vi} \end{sous-entrée}

\begin{définition}\pfra{se presser}\end{définition}
\begin{définition}\pcmn{赶紧……}\end{définition}\end{entrée}

\begin{entrée}{ɕɯmɕɯm}{}{ⓔɕɯmɕɯm} 
\classe{idph.2} \sens{1}
\begin{définition}\pfra{formant une couche fine}\end{définition}
\begin{définition}\pcmn{构成了薄薄的一层}\end{définition}\sens{2}
\begin{définition}\pfra{calme et agréable}\end{définition}
\begin{définition}\pcmn{很安静;很舒服}\end{définition}
\begin{exemple}\pjya{tɤjpa ɕɯmɕɯm ci ko-lɤt}\hspace{5pt}\pcmn{下了薄薄的雪}\end{exemple}
\begin{exemple}\pjya{tɯ-mɯ ɕɯmɕɯm nɤ ɕɯmɕɯm ɲɯ-ɤsɯ-lɤt}\hspace{5pt}\pcmn{在下毛毛雨}\end{exemple}
\begin{exemple}\pjya{a-βri pɯ-nɯ-χtɕi-t-a tɕe, ɕɯmɕɯm ʑo ɲo-pa}\hspace{5pt}\pcmn{我洗完澡,很舒服}\end{exemple}
\begin{sous-entrée}{ɕɯmɯmi}{ⓔɕɯmɕɯmⓢ2ⓝɕɯmɯmi} 
\classe{idph.7} 
\begin{exemple}\pjya{tɯ-mɯ ɕɯmɯmi ʑo ɲɯ-lɤt}\hspace{5pt}\pcmn{雨下得又细又密(很安静)}\end{exemple}\end{sous-entrée}

\begin{sous-entrée}{sɤɕɯmɕɯm}{ⓔɕɯmɕɯmⓢ2ⓝsɤɕɯmɕɯm} 
\classe{vt} 
\begin{exemple}\pjya{tɯ-mɯ ɲɯ-sɤɕɯmɕɯm ʑo ɲɯ-ɤsɯ-lɤt}\hspace{5pt}\pcmn{雨下得又细又密}\end{exemple}\end{sous-entrée}

\end{entrée}

\begin{entrée}{ɕɯmnɤm}{}{ⓔɕɯmnɤm} 
\classe{vt}  
\grammaire{caus} \paradigme{dir}{nɯ-}
\begin{définition}\pfra{causer une odeur}\end{définition}
\begin{définition}\pcmn{导致有味道}\end{définition}
\begin{exemple}\pjya{thamakha ɯ-di nɯ-tɯ-ɕɯmnɤm}\hspace{5pt}\pcmn{(因为你抽了烟),把家里弄得有烟味}\end{exemple}
\begin{exemple}\pjya{cha ɯ-di na-ɕɯmnɤm}\hspace{5pt}\pcmn{(因为喝了酒),就把家里弄得有酒味}\end{exemple}
\begin{exemple}\pjya{kha nɯ tɕu ɕɤɣ pjɯ́-wɣ-sɤkhɯ tɕe, ɯ-di kɤ-ɕɯmnɤm ɲɯ-ra}\hspace{5pt}\pcmn{要烧柏树让家里有香味}\end{exemple}\relationsémantique{参考}{\lien{ⓔmnɤm}{mnɤm}}\relationsémantique{参考}{\lien{ⓔnɤmnɤm}{nɤmnɤm}}\end{entrée}

\begin{entrée}{ɕɯmɲatsa}{}{ⓔɕɯmɲatsa} 
\classe{n} 
\begin{définition}\pfra{violon à deux cordes}\end{définition}
\begin{définition}\pcmn{胡琴}\end{définition}\end{entrée}

\begin{entrée}{ɕɯmŋɤm}{}{ⓔɕɯmŋɤm} 
\classe{vt}  
\grammaire{caus} \paradigme{dir}{tɤ-}
\begin{définition}\pfra{faire mal}\end{définition}
\begin{définition}\pcmn{弄痛}\end{définition}
\begin{exemple}\pjya{tɤ-ɕɯmŋam-a, tɤ-tɯ-ɕɯmŋɤm, ta-ɕɯmŋɤm}\hspace{5pt}\pcmn{我弄痛了,你弄痛了,他弄痛了}\end{exemple}
\begin{exemple}\pjya{a-mgɯr tɤŋkhɯt ta-lɤt tɕe, ta-ɕɯmŋɤm}\hspace{5pt}\pcmn{他在我背上打了一拳,把我打得很痛}\end{exemple}\relationsémantique{参考}{\lien{ⓔmŋɤm}{mŋɤm}}\end{entrée}

\begin{entrée}{ɕɯmthu}{}{ⓔɕɯmthu} 
\classe{vi} \paradigme{dir}{tɤ-}\paradigme{dir}{thɯ-}
\begin{définition}\pfra{poser plein de questions}\end{définition}
\begin{définition}\pcmn{问很多问题}\end{définition}
\begin{exemple}\pjya{ɕɯmthu-a, tɯ-ɕɯmthu, ɕɯmthu}\hspace{5pt}\pcmn{我问了很多问题,你问了很多问题,他问了很多问题}\end{exemple}
\begin{exemple}\pjya{tɤ-tɯ-ɕɯmthu, to-ɕɯmthu}\hspace{5pt}\pcmn{你问了很多问题,他问了很多问题}\end{exemple}
\begin{exemple}\pjya{mɯ-tɤ-tɯ-tso tɕe, a-tɤ-tɯ-ɕɯmthu}\hspace{5pt}\pcmn{你没有懂的话,可以随便问一下}\end{exemple}
\begin{exemple}\pjya{ɯʑo kha lo-nɯɕe tɕe, thɯ-ɕɯmthu-a}\hspace{5pt}\pcmn{他回家了,我就问了很多问题}\end{exemple}
\begin{exemple}\pjya{nɤʑo ndɤre nɤ-tɯ-ɕɯmthu nɯ!}\hspace{5pt}\pcmn{你倒是个爱问问题的人}\end{exemple}\relationsémantique{参考}{\lien{ⓔthuⓗ1}{thu₁}}\end{entrée}

\begin{entrée}{ɕɯmthuspoʁ}{}{ⓔɕɯmthuspoʁ} 
\classe{n} 
\begin{définition}\pfra{enfant qui aimer poser des questions sans cesse}\end{définition}
\begin{définition}\pcmn{不停问东问西的孩子}\end{définition}\relationsémantique{参考}{\lien{ⓔɕɯmthu}{ɕɯmthu}}\end{entrée}

\begin{entrée}{ɕɯmɯma}{}{ⓔɕɯmɯma} 
\classe{postp} 
\begin{définition}\pfra{juste au moment où}\end{définition}
\begin{définition}\pcmn{正当……的时候}\end{définition}
\begin{sous-entrée}{nɯɕɯmɯma}{ⓔɕɯmɯmaⓝnɯɕɯmɯma}
\begin{définition}\pfra{immédiatement}\end{définition}
\begin{définition}\pcmn{马上就}\end{définition}\end{sous-entrée}

\end{entrée}

\begin{entrée}{ɕɯngo}{}{ⓔɕɯngo} 
\classe{vt}  
\grammaire{caus} \paradigme{dir}{tɤ-}
\begin{définition}\pfra{rendre malade}\end{définition}
\begin{définition}\pcmn{使生病}\end{définition}
\begin{exemple}\pjya{ta-ɕɯngo}\hspace{5pt}\pcmn{他让他生病了}\end{exemple}
\begin{exemple}\pjya{kɤndza nɯ mɯ-nɯ-kɯ-sna ra a-mɤ-tɤ́-wɣ-ndza ra ma kɯ-ɕɯngo}\hspace{5pt}\pcmn{变了味的食物不能吃,不然会因此生病}\end{exemple}\relationsémantique{参考}{\lien{ⓔngo}{ngo}}\end{entrée}

\begin{entrée}{ɕɯnŋo}{}{ⓔɕɯnŋo} 
\classe{vt}  
\grammaire{caus} \paradigme{dir}{pɯ-}
\begin{définition}\pfra{gagner, vaincre}\end{définition}
\begin{définition}\pcmn{赢;打败;战胜}\end{définition}
\begin{exemple}\pjya{ta-ɕɯnŋo}\hspace{5pt}\pcmn{我赢了你}\end{exemple}
\begin{exemple}\pjya{kɯ-ɕɯnŋo-a}\hspace{5pt}\pcmn{你赢了我}\end{exemple}
\begin{exemple}\pjya{tɕiʑo ni tɤ-nɯ-saχɕɯβ-tɕi ri, pɯ́-wɣ-ɕɯnŋo-a}\hspace{5pt}\pcmn{我们俩比赛的时候,他打败了我}\end{exemple}
\begin{exemple}\pjya{ɯʑo kɯ ɯ-zda pa-ɕɯnŋo}\hspace{5pt}\pcmn{我把对方打败了}\end{exemple}
\begin{exemple}\pjya{nɤʑo pɯ-ta-ɕɯnŋo}\hspace{5pt}\pcmn{我打败了你}\end{exemple}
\begin{exemple}\pjya{tɕiʑo tɤ-aʑɯʑutɕi pɯ-ta-ɕɯnŋo}\hspace{5pt}\pcmn{在角力的时候,我把你打败了}\end{exemple}
\begin{exemple}\pjya{tɤ-amɯti-tɕi pɯ-ta-ɕɯnŋo}\hspace{5pt}\pcmn{我把你讲赢了}\end{exemple}
\begin{exemple}\pjya{ɕu pɯ-lɤt-tɕi tɕe pɯ-ta-ɕɯnŋo}\hspace{5pt}\pcmn{我们俩打牌的时候,我把你打输了}\end{exemple}
\begin{exemple}\pjya{a-χti pɯ-ɕɯnŋo-t-a}\hspace{5pt}\pcmn{我把对方打败了}\end{exemple}
\begin{sous-entrée}{ʑɣɤɕɯnŋo}{ⓔɕɯnŋoⓝʑɣɤɕɯnŋo} 
\classe{vi}  
\grammaire{refl}
\grammaire{caus} 
\begin{définition}\pfra{causer soi-même sa propre défaite}\end{définition}
\begin{définition}\pcmn{使自己失败}\end{définition}\relationsémantique{同义词}{\lien{ⓔkoⓗ2}{ko}}\relationsémantique{参考}{\lien{ⓔnŋo}{nŋo}}\end{sous-entrée}

\end{entrée}

\begin{entrée}{ɕɯntaβ}{}{ⓔɕɯntaβ}\relationsémantique{参考}{\lien{ⓔntaβ}{ntaβ}}\end{entrée}

\begin{entrée}{ɕɯŋarɯra}{}{ⓔɕɯŋarɯra} 
\classe{pro} 
\begin{définition}\pfra{meilleurs les uns que les autres}\end{définition}
\begin{définition}\pcmn{一个比一个好}\end{définition}\end{entrée}

\begin{entrée}{ɕɯŋɕɯŋ}{}{ⓔɕɯŋɕɯŋ} 
\classe{idph.2} 
\begin{définition}\pfra{bruit de friction métallique}\end{définition}
\begin{définition}\pcmn{铁皮摩擦的声音(如锯子锯东西的时候)}\end{définition}
\begin{sous-entrée}{ɣɤɕɯŋɕɯŋ}{ⓔɕɯŋɕɯŋⓝɣɤɕɯŋɕɯŋ} 
\classe{vi} 
\begin{exemple}\pjya{rɟaŋsoʁ ɯ-zgra ɲɯ-ɣɤɕɯŋɕɯŋ}\hspace{5pt}\pcmn{锯子发出铁皮摩擦声}\end{exemple}\end{sous-entrée}

\end{entrée}

\begin{entrée}{ɕɯŋgɯ}{₁}{ⓔɕɯŋgɯⓗ1} 
\classe{postp} 
\begin{définition}\pfra{avant}\end{définition}
\begin{définition}\pcmn{之前}\end{définition}
\begin{exemple}\pjya{ku-ɣi ɕɯŋgɯ χsɯ-sŋi tɕe a-@dianhua tu-lɤt ɯ-ŋu?}\hspace{5pt}\pcmn{他来之前三天会给我打电话是吗?}\end{exemple}\relationsémantique{参考}{\lien{ⓔɯ-ŋgɯ}{ɯ-ŋgɯ}}\end{entrée}

\begin{entrée}{ɕɯŋgɯ}{₂}{ⓔɕɯŋgɯⓗ2}\relationsémantique{参考}{\lien{ⓔtɤ-mbrɯ,ŋgɯ}{tɤ-mbrɯ,ŋgɯ}}\end{entrée}

\begin{entrée}{ɕɯŋgɯmɯr}{}{ⓔɕɯŋgɯmɯr} 
\classe{n} 
\begin{définition}\pfra{la veille}\end{définition}
\begin{définition}\pcmn{前一天}\end{définition}\relationsémantique{参考}{\lien{ⓔɕɯŋgɯⓗ1}{ɕɯŋgɯ}}\relationsémantique{参考}{\lien{ⓔtɯ-ɣmɯr}{tɯ-ɣmɯr}}\end{entrée}

\begin{entrée}{ɕɯŋke}{}{ⓔɕɯŋke}\relationsémantique{参考}{\lien{ⓔŋke}{ŋke}}\end{entrée}

\begin{entrée}{ɕɯɴqoʁ}{}{ⓔɕɯɴqoʁ} 
\classe{vt}  
\grammaire{caus} \paradigme{dir}{tɤ-}\paradigme{dir}{pɯ-}\paradigme{dir}{tɤ-}
\begin{définition}\pfra{accrocher}\end{définition}
\begin{définition}\pcmn{挂(在上面)}\end{définition}
\begin{définition}\pfra{se laisser pendre}\end{définition}
\begin{définition}\pcmn{让自己身体悬吊着}\end{définition}
\begin{exemple}\pjya{tɤ-ɕɯɴqoʁ-a, ta-ɕɯɴqoʁ}\hspace{5pt}\pcmn{我挂了,他挂了(这个东西)}\end{exemple}
\begin{exemple}\pjya{kɯki laχtɕha ki tɕɤtu nɯ tɕu tɤ-ɕɯɴqoʁ}\hspace{5pt}\pcmn{你把这个东西挂在上面}\end{exemple}
\begin{exemple}\pjya{tɯpɤr znde ɯ-taʁ tɤ-ɕɯɴqoʁ-a}\hspace{5pt}\pcmn{我在墙上挂了一幅画}\end{exemple}
\begin{exemple}\pjya{ɯ-jme pa-ɕɯɴqoʁ nɤ, tɤ́-wɣ-rɤɕi ɲɯ-ŋu.}\hspace{5pt}\pcmn{(狐狸)把尾巴伸进(洞)里,把弟弟拉上来了(狐狸.153)}\end{exemple}
\begin{exemple}\pjya{tɤ-pɤtso to-ʑɣɤɕɯɴqoʁ tɕe, ɲɯ-ɤnɯɣro, ɲɯ-ɣɤɕtʂaŋlaŋ ʑo}\hspace{5pt}\pcmn{小孩子把自己挂上去了,在那里吊着玩}\end{exemple}
\begin{sous-entrée}{ʑɣɤɕɯɴqoʁ}{ⓔɕɯɴqoʁⓝʑɣɤɕɯɴqoʁ} 
\classe{vi}  
\grammaire{refl}
\grammaire{caus} \end{sous-entrée}

\end{entrée}

\begin{entrée}{ɕɯphɣo}{}{ⓔɕɯphɣo}\relationsémantique{参考}{\lien{ⓔphɣo}{phɣo}}\end{entrée}

\begin{entrée}{ɕɯrɕɯr}{}{ⓔɕɯrɕɯr} 
\classe{idph.2} 
\begin{définition}\pfra{calme}\end{définition}
\begin{définition}\pcmn{形容静悄悄的,一点声音也没有}\end{définition}
\begin{exemple}\pjya{ɕɯrɕɯr ʑo lo-pa lo-fsoʁ}\hspace{5pt}\pcmn{天亮都没有亮}\end{exemple}
\begin{exemple}\pjya{jiɕqha ɕɯrɕɯr ʑo ɲɯ-pa mɯ-ɲɯ-ɤrju-nɯ}\hspace{5pt}\pcmn{很安静,没有人讲话}\end{exemple}
\begin{exemple}\pjya{ɕɯrɕɯr ʑo lo-pa tɕe tɤ-rɤru-a}\hspace{5pt}\pcmn{天亮都没有亮我就起床了}\end{exemple}
\begin{exemple}\pjya{tɯrme ra ko-nɯrŋgɯ-nɯ tɕe, ɕɯrɕɯr ʑo ɲɯ-pa}\hspace{5pt}\pcmn{人们睡着了以后,很安静}\end{exemple}
\begin{exemple}\pjya{ɕɤr tɕe, kha ɯ-ŋgɯ ɕɯrɕɯr ʑo ɲɯ-pa}\hspace{5pt}\pcmn{晚上家里很安静}\end{exemple}\end{entrée}

\begin{entrée}{ɕɯrdɯm}{}{ⓔɕɯrdɯm} 
\classe{n} 
\begin{définition}\pfra{bois de chauffage non coupé}\end{définition}
\begin{définition}\pcmn{没有劈开的木柴}\end{définition}\relationsémantique{反义词}{\lien{ⓔsɯpaⓗ2}{sɯpa}}\end{entrée}

\begin{entrée}{ɕɯrga}{}{ⓔɕɯrga} 
\classe{vt}  
\grammaire{caus} \paradigme{dir}{nɯ-}
\begin{définition}\pfra{rendre qqn content}\end{définition}
\begin{définition}\pcmn{让别人高兴}\end{définition}
\begin{exemple}\pjya{ta-ɕɯrga, ɣɯ́-ɕɯrga-a}\hspace{5pt}\pcmn{我让你高兴,他让我高兴}\end{exemple}
\begin{exemple}\pjya{nɯ-tɯ́-wɣ-ɕɯrga}\hspace{5pt}\pcmn{他让你高兴了}\end{exemple}
\begin{exemple}\pjya{aʑo mɤ-pe-a qhe, kɤ-ɕɯrga mɤ-cha-a}\hspace{5pt}\pcmn{我不好,不能让他高兴}\end{exemple}
\begin{exemple}\pjya{kɯ-pe tú-wɣ-nɤma tɕe, tɯ-zda kɤ-ɕɯrga sɤcha}\hspace{5pt}\pcmn{把工作做好就可以让自己的朋友高兴}\end{exemple}
\begin{exemple}\pjya{nɯ-ɕɯrga-t-a}\hspace{5pt}\pcmn{我让他高兴了}\end{exemple}\relationsémantique{参考}{\lien{ⓔrgaⓗ1ⓝrga}{rga}}
\begin{sous-entrée}{ʑɣɤɕɯrga}{ⓔɕɯrgaⓝʑɣɤɕɯrga} 
\classe{vi}  
\grammaire{caus}
\grammaire{refl} 
\begin{définition}\pfra{se faire plaisir}\end{définition}
\begin{définition}\pcmn{让自己开心}\end{définition}
\begin{exemple}\pjya{laχtɕha ɯ-kɤ-sɯso nɯ to-χtɯ ndɤre ɲɯ-ʑɣɤɕɯrga}\hspace{5pt}\pcmn{他买到了他想买的东西,很开心}\end{exemple}\end{sous-entrée}

\end{entrée}

\begin{entrée}{ɕɯrŋgɯ}{}{ⓔɕɯrŋgɯ} 
\classe{vt}  
\grammaire{caus} \sens{1}\paradigme{dir}{kɤ-}
\begin{définition}\pfra{faire dormir, fermenter le vin}\end{définition}
\begin{définition}\pcmn{使人睡;使人躺下;发酵(酒)}\end{définition}
\begin{exemple}\pjya{kɤ-ɕɯrŋgɯ-t-a, kɤ-tɯ-ɕɯrŋgɯ-t, kaɕɯrŋgɯ}\hspace{5pt}\pcmn{我让他睡了,你让他睡了,他让他睡了}\end{exemple}
\begin{exemple}\pjya{tɤ-pɤtso nɯʑɯβ ɲɯ-ŋu tɕe kɤ-ɕɯrŋgɯ-t-a}\hspace{5pt}\pcmn{小孩子快睡着了,我就让他躺下休息}\end{exemple}\sens{2}
\begin{définition}\pfra{laisser le vin fermenter}\end{définition}
\begin{définition}\pcmn{让……发酵(酒)}\end{définition}
\begin{exemple}\pjya{cha kɤ-sqa-t-a tɕe kɤ-ɕɯrŋgɯ-t-a}\hspace{5pt}\pcmn{我让酒发酵了}\end{exemple}\relationsémantique{参考}{\lien{ⓔrŋgɯⓗ1}{rŋgɯ₁}}\end{entrée}

\begin{entrée}{ɕɯrŋo}{}{ⓔɕɯrŋo} 
\classe{vt}  
\grammaire{caus} \paradigme{dir}{nɯ-}
\begin{définition}\pfra{prêter (un objet)}\end{définition}
\begin{définition}\pcmn{借给别人(假定以后能归还原物)}\end{définition}
\begin{exemple}\pjya{ta-ɕɯrŋo, kɯ-ɕɯrŋo-a, ɣɯ́-ɕɯrŋgo-a}\hspace{5pt}\pcmn{我借给你,你借给我,他借给我}\end{exemple}\relationsémantique{参考}{\lien{ⓔrŋo}{rŋo}}\relationsémantique{参考}{\lien{ⓔnɤŋgɯⓝznɤŋgɯ}{znɤŋgɯ}}\end{entrée}

\begin{entrée}{ɕɯrɴɢo}{}{ⓔɕɯrɴɢo} 
\classe{n} 
\begin{définition}\pfra{Anisodus tanguticus}\end{définition}
\begin{définition}\pcmn{山茛菪}\end{définition}
\begin{exemple}\pjya{ɕɯrɴɢo nɯ sɯjno kɯ-wxti kɯ-jpumqa ci ŋu. ɯ-mɯntoʁ kɯnɤ kɯ-jaʁjɯ kɯ-wxti tsa ci tshaŋlaŋ taʁ tɤ-kɯ-ɕthɯz kɯ-fse ci ŋu, ɯ-mɯntoʁ wɣrum, ɯ-taʁ kɯ-ɣɯrni kɯ-ɤkhra tu, ɯ-mɯntoʁ ɯ-rqhu nɯ wuma ʑo jaʁ cho rko. ɯ-mɯntoʁ pɯ-rom ɯ-qhu kɯnɤ ɯ-rqhu nɯ tu tɕe, ɯ-rɣi nɯ ku-mphɯr, ku-sɤsɯɣ ʑo ŋu, tɕe mɯ-pɯ́-wɣ-tɕɣaʁ mɤɕtʂa ɯ-rɣi mɤ-nɯɬoʁ. ɯ-mat thɯ-tɯt tɕe, ɯ-rɣi ndɯβ cho dɤn. ɯ-zrɤm wuma ʑo ngɯt, kɤ-phɯt ɴqa. tɤ-rɤku ɯ-ŋgɯ tu-ɬoʁ mɤ-pe ma ɯ-sta ɴqa tɕe tɤ-rɤku mɤ-sɤpe. ɯ-jwaʁ cho ɯ-ru ra aɣɯmdoʁ. nɯŋa kɤ-mbi sna, nɯŋa ɲɯ́-wɣ-mbi tɕe ɯ-lu dɤn tu-ti-nɯ ɲɯ-ŋu.}\hspace{5pt}\pcmn{莨菪是粗壮的大草,花也很厚实,像朝上的铃铛。花是白色的,上面红色斑点,花萼又厚又硬。花凋谢了之后,花萼还在,包住种子,包得很紧,只有把它挤破种子才能出来。果实成熟后,种子细而多。根长得很结实,难以拔掉。长在庄稼地里不好,因为所占地面积宽,会影响庄稼。叶子和茎颜色一样。可以喂奶牛,据说奶牛吃了奶多。}\end{exemple}\end{entrée}

\begin{entrée}{ɕɯrʁom}{}{ⓔɕɯrʁom} 
\classe{n} 
\begin{définition}\pfra{poils de yak épais}\end{définition}
\begin{définition}\pcmn{牦牛的粗毛}\end{définition}\relationsémantique{参考}{\lien{ⓔrʁom}{rʁom}}\end{entrée}

\begin{entrée}{ɕɯʁno}{}{ⓔɕɯʁno} 
\classe{vt} \paradigme{dir}{kɤ-}
\begin{définition}\pfra{accuser à tort}\end{définition}
\begin{définition}\pcmn{冤枉}\end{définition}
\begin{exemple}\pjya{ma-kɤ-kɯ-ɕɯʁno-a!}\hspace{5pt}\pcmn{你不要冤枉我!}\end{exemple}\end{entrée}

\begin{entrée}{ɕɯwa}{}{ⓔɕɯwa} 
\classe{n} 
\begin{définition}\pfra{teigne}\end{définition}
\begin{définition}\pcmn{癣}\end{définition}\end{entrée}

\begin{entrée}{ɕɯxtɯ}{}{ⓔɕɯxtɯ} 
\classe{n} 
\begin{définition}\pfra{bord de l'âtre}\end{définition}
\begin{définition}\pcmn{火塘边(铁或者石头制成的);长方形}\end{définition}\end{entrée}

\begin{entrée}{ɕɯxtɯrɟɤxtsa}{}{ⓔɕɯxtɯrɟɤxtsa} 
\classe{n} 
\begin{définition}\pfra{une plante}\end{définition}
\begin{définition}\pcmn{植物的一种}\end{définition}
\begin{exemple}\pjya{ɕɯxtɯ rɟɤxtsa nɯ ruŋgu, tʂɤrkɯ, stɤmku aʁɤndɯndɤt tu-ɬoʁ ŋu, ɯ-zrɤm wxti, tɯ-boʁ nɤ tɯ-boʁ tu-ɬoʁ tɕe, ɯ-ru tɯ-ldʑa ɯ-taʁ ɯ-jwaʁ ɯ-rme ʑo kɯ-fse kɯ-dɤn ɲɯ-ɬoʁ ŋu, ɯ-ru ngɯt, ɯ-mɯntoʁ kɯ-ndɯ-ndɯβ kɯ-dɯ-dɤn ʑo ɲɯ-lɤt ŋu.}\hspace{5pt}\pcmn{\lien{}{ɕɯxtɯ rɟɤxtsa}在草山、草地、路边都可以生长,根大,丛生。茎上的叶子像毛一样密。茎很结实,茎的顶上开小而密的花。}\end{exemple}\end{entrée}

\newpage\caractère{d}

\begin{entrée}{dal}{}{ⓔdal} 
\classe{adv} 
\begin{définition}\pfra{plus tard}\end{définition}
\begin{définition}\pcmn{晚一点}\end{définition}
\begin{exemple}\pjya{nɤʑo jiɕqha dal tsa ri kɤ-tɯ-nɯtʂha ɕti}\hspace{5pt}\pcmn{你晚一点才吃了早餐}\end{exemple}
\begin{exemple}\pjya{japa dal ri}\hspace{5pt}\pcmn{几年前}\end{exemple}
\begin{exemple}\pjya{jɯfɕɯndʐi dal ri}\hspace{5pt}\pcmn{几天前}\end{exemple}\end{entrée}

\begin{entrée}{daltsa}{}{ⓔdaltsa} 
\classe{n} 
\begin{définition}\pfra{lentement}\end{définition}
\begin{définition}\pcmn{慢}\end{définition}
\begin{exemple}\pjya{kɤ-rɯɕmi daltsa tɤ-pe ɲɯ-ra}\hspace{5pt}\pcmn{请你说话稍微慢一点}\end{exemple}\étymologie{dal}\end{entrée}

\begin{entrée}{daltsɯtsa}{}{ⓔdaltsɯtsa} 
\classe{n} 
\begin{définition}\pfra{lentement}\end{définition}
\begin{définition}\pcmn{慢慢}\end{définition}\relationsémantique{参考}{\lien{ⓔdaltsa}{daltsa}}\end{entrée}

\begin{entrée}{dɤlie}{}{ⓔdɤlie} 
\classe{vi} 
\begin{définition}\pfra{bienvenue}\end{définition}
\begin{définition}\pcmn{欢迎光临,快回家(我们在等着你)}\end{définition}
\begin{exemple}\pjya{dɤlie-ndʑi}\hspace{5pt}\pcmn{你们俩快回家}\end{exemple}\end{entrée}

\begin{entrée}{dɤn}{}{ⓔdɤn} 
\classe{vs} \paradigme{dir}{tɤ-}\paradigme{dir}{nɯ-}
\begin{définition}\pfra{nombreux}\end{définition}
\begin{définition}\pcmn{多}\end{définition}
\begin{exemple}\pjya{tɯrme ɲɯ-dɤn}\hspace{5pt}\pcmn{人很多}\end{exemple}
\begin{exemple}\pjya{pɯ-dɤn-i}\hspace{5pt}\pcmn{(当时)我们人很多}\end{exemple}
\begin{exemple}\pjya{kɯ-dɤn me-j}\hspace{5pt}\pcmn{我们人不多}\end{exemple}\relationsémantique{同义词}{\lien{ⓔxcat}{xcat}}
\begin{sous-entrée}{nɤdɤn}{ⓔdɤnⓝnɤdɤn} 
\classe{vt}  
\grammaire{trop} 
\begin{définition}\pfra{trouver nombreux}\end{définition}
\begin{définition}\pcmn{觉得很多}\end{définition}\end{sous-entrée}

\étymologie{ldan}\end{entrée}

\begin{entrée}{dɤrʁɯ}{}{ⓔdɤrʁɯ} 
\classe{n} 
\begin{définition}\pfra{fougère}\end{définition}
\begin{définition}\pcmn{蕨苔}\end{définition}
\begin{exemple}\pjya{dɤrʁɯ nɯ sɤtɕha thamtɕɤt ʑo tu-maʁ, sɤtɕha ɴqiaβ tsa sɤjku kɯ-tu ɣɯ pɕoʁ nɯ ra tu-ɬoʁ ŋu. phaʁzla jarma tu-ɬoʁ tɕe tʂɯɣpa ɲɯ-ɤrɕo ɕɯŋgɯ tɕe tu-qalpɕa tɕe tu-rgɤz ɕti. tu-qalpɕa ɕɯŋgɯ nɯ ɯ-ru ɯ-jwaʁ nɯ tɯrme kɯ tɤŋkhɯt tɤ-kɤ-βzu fse. ɯ-ru nɯ mpɯ tɕe kɤ-ndza mɯm.}\hspace{5pt}\pcmn{蕨苔不是所有的地方都有,只有在山阴能有白桦树的地方才生长。一般五月开始生长到六月底叶子就展开,蕨苔也就老了。叶子展开前,茎上的叶子长得像人握着的拳。茎很柔嫩,好吃。}\end{exemple}\end{entrée}

\begin{entrée}{dɣɤrdɣɤr}{}{ⓔdɣɤrdɣɤr} 
\classe{idph.2} 
\begin{définition}\pfra{bête}\end{définition}
\begin{définition}\pcmn{形容不聪明,发呆的模样}\end{définition}
\begin{exemple}\pjya{jiɕqha tɯrme nɯ dɣɤrdɣɤr ʑo ɲɯ-rɤʑi ɲɯ-khe ʑo}\hspace{5pt}\pcmn{那个人在那里发呆}\end{exemple}\relationsémantique{参考}{\lien{ⓔɣɤrɣɤr}{ɣɤrɣɤr}}\end{entrée}

\begin{entrée}{dioʁdioʁ}{}{ⓔdioʁdioʁ} 
\classe{idph.2} 
\begin{définition}\pfra{bien mélangé}\end{définition}
\begin{définition}\pcmn{形容非常均匀}\end{définition}
\begin{exemple}\pjya{tɤjlu pɯ́-wɣ-rɤpɣi tɕe, tɯ-ci cho tɤjlu ni tú-wɣ-sɤtʂoʁloʁ dioʁdioʁ ʑo ra ma nɯ mɤɕtʂa mɤ-mɯm}\hspace{5pt}\pcmn{和面的时候,要把水和面和得很均匀}\end{exemple}\end{entrée}

\begin{entrée}{do}{}{ⓔdo} 
\classe{vs} \paradigme{dir}{thɯ-}
\begin{définition}\pfra{fibreuse (plante)}\end{définition}
\begin{définition}\pcmn{老(植物)}\end{définition}
\begin{exemple}\pjya{pɤjka cho-do}\hspace{5pt}\pcmn{白瓜老了(可以吃了)}\end{exemple}\end{entrée}

\begin{entrée}{doŋdoŋ}{}{ⓔdoŋdoŋ} 
\classe{idph.2} 
\begin{définition}\pfra{long et épais, cylindrique}\end{définition}
\begin{définition}\pcmn{形容圆柱形物体粗而长的样子}\end{définition}
\begin{exemple}\pjya{ɕoŋtɕa ɲɯ-jpum doŋdoŋ}\hspace{5pt}\pcmn{木料非常粗}\end{exemple}\end{entrée}

\begin{entrée}{drɤβdrɤβ}{}{ⓔdrɤβdrɤβ} 
\classe{idph.2} 
\begin{définition}\pfra{plein de saleté (eau)}\end{définition}
\begin{définition}\pcmn{形容浑浊、装满渣滓、没有搅匀的样子}\end{définition}
\begin{exemple}\pjya{tɯtshi drɤβdrɤβ ʑo ɲɯ-pa}\hspace{5pt}\pcmn{粥很粘稠(没有搅匀)}\end{exemple}
\begin{exemple}\pjya{tɯ-mɯ chɤ-qandʐi drɤβdrɤβ ʑo}\hspace{5pt}\pcmn{天空布满了乌云}\end{exemple}
\begin{exemple}\pjya{tɯ-ci chɤ-qarndɯm drɤβdrɤβ ʑo}\hspace{5pt}\pcmn{水变浑浊了}\end{exemple}\end{entrée}

\begin{entrée}{droŋdroŋ}{}{ⓔdroŋdroŋ} 
\classe{idph.2} 
\begin{définition}\pfra{gros et sale}\end{définition}
\begin{définition}\pcmn{形容粗、大而脏的样子}\end{définition}
\begin{exemple}\pjya{ɯ-ɕŋaβ droŋdroŋ ʑo chɤ-nɤndzɣi}\hspace{5pt}\pcmn{他的鼻涕一大根一大根地吊在那里,显得很脏}\end{exemple}
\begin{exemple}\pjya{ɯ-phoŋbu ɲɯ-sɤjloʁ droŋdroŋ}\hspace{5pt}\pcmn{她身材又胖又高,不好看}\end{exemple}\end{entrée}

\begin{entrée}{drɯβ}{}{ⓔdrɯβ} 
\classe{idph.1} 
\begin{définition}\pfra{percer et laisser couler un liquide}\end{définition}
\begin{définition}\pcmn{刺破貌(脓包)}\end{définition}
\begin{exemple}\pjya{taqaβ drɯβ ʑo tɤ-lat-a tɕe, tɤ-spɯ tɤ-tɕat-a}\hspace{5pt}\pcmn{我用针刺了一下,把脓排出来了}\end{exemple}
\begin{sous-entrée}{drɯβnɤdrɯβ}{ⓔdrɯβⓝdrɯβnɤdrɯβ} 
\classe{idph.3} 
\begin{exemple}\pjya{tɤ-spɯ drɯβnɤdrɯβ ta-tɕɣaʁ}\hspace{5pt}\pcmn{他把脓挤了出来}\end{exemple}\relationsémantique{参考}{\lien{ⓔnɯdrɯβ}{nɯdrɯβ}}\end{sous-entrée}

\end{entrée}

\begin{entrée}{dɯdɯt}{}{ⓔdɯdɯt} 
\classe{n} 
\begin{définition}\pfra{tourterelle}\end{définition}
\begin{définition}\pcmn{斑鸠}\end{définition}
\begin{exemple}\pjya{dɯdɯt nɯ pɣa khro mɤ-kɯ-wxti ci ŋu, qro jamar ma me, ɯ-tshɯɣa ra qro fse, ɯ-mdoʁ nɯ kɯ-ɤrŋi ɯ-ŋgɯz kɯnɤ kɯ-pɣi tsa ŋu, tu-mbri tɕe, `du dɯt cɯ ɯ-ŋgɯ lɤɣ' tu-ti ɲɯ-ŋu, tɤ-rɤku cho qajɯ ndze, tɤ-rɤku kɤ-ndza χɕu, kha tɤ-kɤ-nɯ-sɤro ʑo nɯ ɣɯ-tu-ndze ŋu.}\hspace{5pt}\pcmn{斑鸠是一种不太大的鸟,只有鸽子那么大,样子也像鸽子,颜色蓝里带灰,叫声是\lien{}{du dɯt cɯ ɯ-ŋgɯ lɤɣ}。吃粮食和虫子,吃粮食尤其厉害,专门来吃摆放在家里的粮食。}\end{exemple}\end{entrée}

\begin{entrée}{dɯɣ}{}{ⓔdɯɣ} 
\classe{vi}  
\grammaire{caus} \paradigme{dir}{tɤ-}\paradigme{dir}{tɤ-}
\begin{définition}\pfra{en avoir assez}\end{définition}
\begin{définition}\pcmn{厌倦;厌烦;觉得麻烦}\end{définition}
\begin{exemple}\pjya{ɲɯ-tɯ-dɯɣ}\hspace{5pt}\pcmn{你不想坚持}\end{exemple}
\begin{exemple}\pjya{ɯʑo to-dɯɣ}\hspace{5pt}\pcmn{他厌倦了}\end{exemple}
\begin{exemple}\pjya{tɯtun pjɯ-ŋgrɯ tɤ-ra tɕe kɯ-ɴqa kɤ-nɤma kɤ-dɯɣ mɤ-βze}\hspace{5pt}\pcmn{要达到自己的目标的话,不能怕辛苦}\end{exemple}
\begin{exemple}\pjya{kɤ-nɤma tɤ-ɴqa tɕe, tu-dɯɣ ɲɯ-ɕti}\hspace{5pt}\pcmn{工作进行得很辛苦,他灰心了}\end{exemple}
\begin{exemple}\pjya{kɤ-ɤmdzɯ ɲɯ-dɯɣ-a}\hspace{5pt}\pcmn{我坐烦了}\end{exemple}
\begin{sous-entrée}{sɯɣdɯɣ}{ⓔdɯɣⓝsɯɣdɯɣ} 
\classe{vt} \end{sous-entrée}

\begin{définition}\pfra{énerver, fatiguer}\end{définition}
\begin{définition}\pcmn{令人不耐烦;厌倦}\end{définition}
\begin{exemple}\pjya{nɤ-kɤ-ti ɯ-tɯ-dɤn kɯ ɲɯ-kɯ-sɯɣdɯɣ-a}\hspace{5pt}\pcmn{你讲得太多,让我不耐烦了}\end{exemple}
\begin{exemple}\pjya{nɤ-kɤ-rɯrawa ɯ-tɯ-dɤn kɯ ɲɯ-kɯ-sɯɣdɯɣ-a}\hspace{5pt}\pcmn{你要求得太多,让我不耐烦了}\end{exemple}
\begin{sous-entrée}{sɤɣdɯɣ}{ⓔdɯɣⓝsɤɣdɯɣ} 
\classe{vs}  
\grammaire{deexp} 
\begin{définition}\pfra{désagréable, détestable}\end{définition}
\begin{définition}\pcmn{辛苦;令人心烦}\end{définition}\relationsémantique{参考}{\lien{ⓔnɤsɤɣdɯɣ}{nɤsɤɣdɯɣ}}\end{sous-entrée}

\end{entrée}

\begin{entrée}{dɯrnɤrdɯr}{}{ⓔdɯrnɤrdɯr} 
\classe{idph.2} 
\begin{définition}\pfra{battement de tambour lointain}\end{définition}
\begin{définition}\pcmn{形容轻微的敲鼓声}\end{définition}
\begin{exemple}\pjya{tɤrmbɣo dɯnɤrdɯr ʑo ɲɯ-mbri}\hspace{5pt}\pcmn{听得到轻微的敲鼓声}\end{exemple}\relationsémantique{参考}{\lien{ⓔɣdoŋnɤɣdoŋ}{ɣdoŋnɤɣdoŋ}}\relationsémantique{参考}{\lien{ⓔɣdɯɣnɤɣdɯɣ}{ɣdɯɣnɤɣdɯɣ}}\end{entrée}

\begin{entrée}{dɯxpa}{}{ⓔdɯxpa} 
\classe{vs} \sens{1}
\begin{définition}\pfra{pauvre de ...}\end{définition}
\begin{définition}\pcmn{可怜;倒霉}\end{définition}
\begin{exemple}\pjya{tɕiʑo ndɤ dɯxpa-tɕi nɤ!}\hspace{5pt}\pcmn{我们俩很倒霉!}\end{exemple}\sens{2}
\begin{définition}\pfra{avoir pris la peine de faire}\end{définition}
\begin{définition}\pcmn{让……费心;……辛苦了}\end{définition}
\begin{exemple}\pjya{dɯxpa ma nɯ ɯ-pɤro ra jo-ɣɯt}\hspace{5pt}\pcmn{让他费心了,带来了礼物}\end{exemple}\end{entrée}

\begin{entrée}{dwaŋdwaŋ}{}{ⓔdwaŋdwaŋ} 
\classe{idph.2} 
\begin{définition}\pfra{ne pas être en possession de ses moyens}\end{définition}
\begin{définition}\pcmn{神志不清}\end{définition}
\begin{exemple}\pjya{ɲɯ-ngo tɕe, dwaŋdwaŋ ʑo ɲɯ-rɤʑi}\hspace{5pt}\pcmn{他生病了,神志不清}\end{exemple}
\begin{exemple}\pjya{cha kú-wɣ-tshi tɕe tɯ-ku dwaŋdwaŋ ʑo pa}\hspace{5pt}\pcmn{喝酒后,头脑就会不清醒}\end{exemple}\relationsémantique{同义词}{\lien{ⓔjaŋjaŋ}{jaŋjaŋ}}\end{entrée}

\begin{entrée}{dzambaɬa}{}{ⓔdzambaɬa} 
\classe{n} 
\begin{définition}\pfra{type de mammifère}\end{définition}
\begin{définition}\pcmn{动物的一种(像黄鼠狼)}\end{définition}\end{entrée}

\begin{entrée}{dzaŋdzaŋ}{}{ⓔdzaŋdzaŋ} 
\classe{idph.2} \sens{1}
\begin{définition}\pfra{dru}\end{définition}
\begin{définition}\pcmn{形容又茂盛又粗糙的样子(植物的刺、头发等)}\end{définition}
\begin{exemple}\pjya{ɯ-ku dzaŋdzaŋ ʑo to-stu}\hspace{5pt}\pcmn{他头发蓬乱}\end{exemple}
\begin{exemple}\pjya{si tɯ-phɯ dzaŋdzaŋ ɣɤʑu}\hspace{5pt}\pcmn{有一棵枝桠茂盛的大树}\end{exemple}
\begin{exemple}\pjya{ɯ-ku to-rpɯ dzaŋdzaŋ ʑo}\hspace{5pt}\pcmn{他的头发又长又脏,乱蓬蓬的}\end{exemple}\sens{2}
\begin{définition}\pfra{ne pas être en possession de ses facultés}\end{définition}
\begin{définition}\pcmn{神志不清(喝醉了,病了)}\end{définition}
\begin{exemple}\pjya{ɲɯ-ŋgo-a tɕe, dzaŋdzaŋ ɲɯ-pa-a}\hspace{5pt}\pcmn{我喝醉了,神志不清}\end{exemple}
\begin{exemple}\pjya{lo-βzi-a tɕe, dzaŋdzaŋ ʑo ɲɯ-pa-a}\hspace{5pt}\pcmn{我喝醉了,神志不清}\end{exemple}
\begin{sous-entrée}{ɣɤdzaŋdzaŋ}{ⓔdzaŋdzaŋⓢ2ⓝɣɤdzaŋdzaŋ} 
\classe{vi} 
\begin{définition}\pfra{avoir les poils longs, en désordre}\end{définition}
\begin{définition}\pcmn{毛又长又乱}\end{définition}
\begin{exemple}\pjya{mbroχpa khɯna ɲɯ-ɣɤdzaŋdzaŋ thɯ-ɣe}\hspace{5pt}\pcmn{藏獒竖起毛扑过来了}\end{exemple}\end{sous-entrée}

\begin{sous-entrée}{ɣɤdzaŋlaŋ}{ⓔdzaŋdzaŋⓢ2ⓝɣɤdzaŋlaŋ} 
\classe{vi} 
\begin{exemple}\pjya{qambrɯ ɲɯ-ɣɤdzaŋlaŋ ntsɯ ɲɯ-rɤβʁa}\hspace{5pt}\pcmn{牦牛一身毛乱蓬蓬地大声叫}\end{exemple}\relationsémantique{参考}{\lien{ⓔzoŋzoŋ}{zoŋzoŋ}}\relationsémantique{参考}{\lien{ⓔzaŋzaŋ}{zaŋzaŋ}}\end{sous-entrée}

\end{entrée}

\begin{entrée}{dzɤjdzɤj}{}{ⓔdzɤjdzɤj}\relationsémantique{参考}{\lien{ⓔzɤjzɤj}{zɤjzɤj}}\end{entrée}

\begin{entrée}{dzoŋdzoŋ}{}{ⓔdzoŋdzoŋ} 
\classe{idph.2} 
\begin{définition}\pfra{ébouriffé}\end{définition}
\begin{définition}\pcmn{形容毛发蓬松、竖起来的样子}\end{définition}
\begin{exemple}\pjya{ɯ-rme dzoŋdzoŋ ʑo cho-ɬoʁ}\hspace{5pt}\pcmn{他的毛发竖起来了}\end{exemple}
\begin{exemple}\pjya{nɤ-kɤrme pɯ-sɤɕɤt ma dzoŋdzoŋ ʑo ɲɯ-pa}\hspace{5pt}\pcmn{你要梳头,不然你的头发毛松松的}\end{exemple}
\begin{exemple}\pjya{tɯrgi paʁtsa kɯ ɯ-jme dzoŋdzoŋ ʑo tu-tse tɕe tu-nɤŋkɯŋke ŋu}\hspace{5pt}\pcmn{松鼠竖起尾巴走动,看起来蓬蓬松松的}\end{exemple}
\begin{sous-entrée}{dzoŋnɤdzoŋ}{ⓔdzoŋdzoŋⓝdzoŋnɤdzoŋ} 
\classe{idph.3} 
\begin{définition}\pfra{sensation désagréable ressentie lorsque l'on marche avec une jambe engourdie}\end{définition}
\begin{définition}\pcmn{形容脚麻木,走得很难受的感觉}\end{définition}
\begin{exemple}\pjya{tu-ŋke-a tɕe a-mi dzoŋnɤdzoŋ ʑo ɲɯ-ti ma chɤ-ndʑɯrpɯt}\hspace{5pt}\pcmn{我走路走得很难受,因为脚麻木了}\end{exemple}\end{sous-entrée}

\end{entrée}

\begin{entrée}{dzoʁ}{}{ⓔdzoʁ} 
\classe{idph.1} 
\begin{définition}\pfra{tout d'un coup (s'agenouiller)}\end{définition}
\begin{définition}\pcmn{一下子(跪下)}\end{définition}
\begin{exemple}\pjya{ɯ-χpɯm dzoʁ ʑo pjɤ-tshoʁ}\hspace{5pt}\pcmn{他一下子跪下了(很恭敬的样子)}\end{exemple}\relationsémantique{同义词}{\lien{ⓔgoʁ}{goʁ}}\end{entrée}

\begin{entrée}{dzɯɣdzɯɣ}{}{ⓔdzɯɣdzɯɣ} 
\classe{idph.2} 
\begin{définition}\pfra{fournis (poils)}\end{définition}
\begin{définition}\pcmn{形容毛多而密,茂盛的样子}\end{définition}
\begin{exemple}\pjya{ɯ-mtɕhɯrme dzɯɣdzɯɣ ʑo ɲɯ-pa}\hspace{5pt}\pcmn{他的胡须很茂盛}\end{exemple}
\begin{exemple}\pjya{tshɤrtɯl ɯ-rme dzɯɣdzɯɣ ʑo ɲɯ-pa}\hspace{5pt}\pcmn{羔羊皮袄的毛很茂盛}\end{exemple}\end{entrée}

\begin{entrée}{dzɯrdzɯr}{}{ⓔdzɯrdzɯr} 
\classe{idph.2} 
\begin{définition}\pfra{bien droit}\end{définition}
\begin{définition}\pcmn{形容端正的样子(又小又乖)}\end{définition}
\begin{exemple}\pjya{dzɯrdzɯr ʑo ɲɯ-rɤʑi}\hspace{5pt}\pcmn{他端端正正地(站)在那里}\end{exemple}
\begin{exemple}\pjya{dzɯrdzɯr ʑo ɲɯ-ɤmdzɯt}\hspace{5pt}\pcmn{他坐得很端正}\end{exemple}
\begin{sous-entrée}{dzɯr}{ⓔdzɯrdzɯrⓝdzɯr}
\begin{définition}\pfra{prompt, agile}\end{définition}
\begin{définition}\pcmn{利索(很受规矩的样子)}\end{définition}
\begin{exemple}\pjya{ɯ-χpɯm dzɯr ʑo ta-nɯ-tshoʁ}\hspace{5pt}\pcmn{他很利索地跪了}\end{exemple}\end{sous-entrée}

\begin{sous-entrée}{dzɯrnɤdzɯr}{ⓔdzɯrdzɯrⓝdzɯrnɤdzɯr} 
\classe{idph.3} 
\begin{exemple}\pjya{dzɯrnɤdzɯr ɲɯ-ŋke}\hspace{5pt}\pcmn{他在端端正正地走}\end{exemple}\end{sous-entrée}

\end{entrée}

\begin{entrée}{dʑaŋdʑaŋ}{}{ⓔdʑaŋdʑaŋ} 
\classe{idph.2} 
\begin{définition}\pfra{long et fin}\end{définition}
\begin{définition}\pcmn{形容细而长的样子}\end{définition}\end{entrée}

\begin{entrée}{dʑɤrdʑɤr}{}{ⓔdʑɤrdʑɤr} 
\classe{idph.2} 
\begin{définition}\pfra{debout tout droit}\end{définition}
\begin{définition}\pcmn{站得很直;身子又细又高;看起来很单薄又孤零零的样子}\end{définition}
\begin{exemple}\pjya{si dʑɤrdʑɤr ta-sɯɣndzur-a}\hspace{5pt}\pcmn{我把树立起来了}\end{exemple}
\begin{sous-entrée}{dʑɤrnɤdʑɤr}{ⓔdʑɤrdʑɤrⓝdʑɤrnɤdʑɤr} 
\classe{idph.3} 
\begin{exemple}\pjya{dʑɤrnɤdʑɤr kɤ-ari}\hspace{5pt}\pcmn{他瘦瘦高高的,走了。}\end{exemple}
\begin{exemple}\pjya{tɤ-pɤtso dʑɤrdʑɤr nɤ dʑɤrdʑɤr ɲɯ-ɤnɯɣro}\hspace{5pt}\pcmn{小孩子一个人(在草坪上)玩}\end{exemple}
\begin{exemple}\pjya{nɤki nɯ ɯ-tɯ-nɯɲɤmkhe kɯ tu-ŋke tɕe dʑɤrnɤdʑɤr ʑo pa ɕti wo}\hspace{5pt}\pcmn{那个人很瘦,走路的时候显得又细又高}\end{exemple}\end{sous-entrée}

\end{entrée}

\begin{entrée}{dʑɯβdʑɯβ}{}{ⓔdʑɯβdʑɯβ} 
\classe{idph.2} 
\begin{définition}\pfra{rugueux et pointu}\end{définition}
\begin{définition}\pcmn{形容粗糙、尖而密的样子}\end{définition}
\begin{exemple}\pjya{tɯrgi ɣɯ ɯ-jwaʁ dʑɯβdʑɯβ ʑo ɲɯ-pa tɕe, ɲɯ-sɤmtsɯɣ}\hspace{5pt}\pcmn{杉树的针叶很尖,会刺到人}\end{exemple}\end{entrée}

\begin{entrée}{dʑɯpdʑɯp}{}{ⓔdʑɯpdʑɯp} 
\classe{idph.2} 
\begin{définition}\pfra{très épineux}\end{définition}
\begin{définition}\pcmn{形容植物的刺密集的样子}\end{définition}
\begin{exemple}\pjya{tɤ-mdzu dʑɯpdʑɯp ʑo ɲɯ-pa}\hspace{5pt}\pcmn{刺很密集,长得很茂盛}\end{exemple}\end{entrée}

\begin{entrée}{dʐoŋdʐoŋ}{}{ⓔdʐoŋdʐoŋ} 
\classe{idph.2} 
\begin{définition}\pfra{mou, long et épais}\end{définition}
\begin{définition}\pcmn{形容软、粗而长的样子}\end{définition}
\begin{exemple}\pjya{tɯ-pu dʐoŋdʐoŋ ʑo ɲɯ-pa}\hspace{5pt}\pcmn{肠子又软又粗又长}\end{exemple}\end{entrée}

\begin{entrée}{dʐɯβdʐɯβ}{}{ⓔdʐɯβdʐɯβ} 
\classe{idph.2} 
\begin{définition}\pfra{tendre}\end{définition}
\begin{définition}\pcmn{嫩}\end{définition}
\begin{exemple}\pjya{ki @bocai ki dʐɯβdʐɯβ ʑo ɲɯ-pa}\hspace{5pt}\pcmn{这个菠菜长得又粗又嫩}\end{exemple}\end{entrée}

\begin{entrée}{dʐɯβnɤdʐɯβ}{}{ⓔdʐɯβnɤdʐɯβ} 
\classe{idph.3} 
\begin{définition}\pfra{mou sous la dent}\end{définition}
\begin{définition}\pcmn{形容食物吃起来软的样子}\end{définition}\end{entrée}

\begin{entrée}{dʐɯɣdʐɯɣ}{}{ⓔdʐɯɣdʐɯɣ} 
\classe{idph.2} 
\begin{définition}\pfra{très fort (thé)}\end{définition}
\begin{définition}\pcmn{酽}\end{définition}
\begin{exemple}\pjya{tʂha tɤ-lu dʐɯɣdʐɯɣ ɲɯ-pa}\hspace{5pt}\pcmn{茶很酽}\end{exemple}\relationsémantique{参考}{\lien{ⓔldɯɣldɯɣ}{ldɯɣldɯɣ}}\end{entrée}

\newpage\caractère{f}

\begin{entrée}{fɕafɕɤt}{}{ⓔfɕafɕɤt} 
\classe{n} 
\begin{définition}\pfra{discours, parole}\end{définition}
\begin{définition}\pcmn{演讲;言论}\end{définition}
\begin{exemple}\pjya{ɯ-grɤl kɯ-me fɕafɕɤt ntsɯ ma-tɯ-βze!}\hspace{5pt}\pcmn{你不要总是胡说八道}\end{exemple}\relationsémantique{参考}{\lien{ⓔfɕɤtⓗ1}{fɕɤt₁}}\end{entrée}

\begin{entrée}{fɕaʁ}{₁}{ⓔfɕaʁⓗ1} 
\classe{vt} \paradigme{dir}{nɯ-}
\begin{définition}\pfra{se repentir, rembourser}\end{définition}
\begin{définition}\pcmn{忏悔;赔偿}\end{définition}
\begin{exemple}\pjya{ɯʑo kɯ kɯmaʁ ta-nɤma ra a-ɕki na-fɕaʁ ɕti}\hspace{5pt}\pcmn{他向我赔偿了他的过失}\end{exemple}
\begin{exemple}\pjya{pɯ-kɯ-ɣɤtɕa tɕe, mɤ-kɤ-fɕaʁ mɤ-khɯ}\hspace{5pt}\pcmn{有了过错就不能不赔偿}\end{exemple}
\begin{exemple}\pjya{ɯ-sɤ-fɕaʁ me}\hspace{5pt}\pcmn{我没有东西来赔偿}\end{exemple}\étymologie{bɕags}\end{entrée}

\begin{entrée}{fɕaʁ}{₂}{ⓔfɕaʁⓗ2} 
\classe{vs} 
\begin{définition}\pfra{être suffisant, satisfaisant}\end{définition}
\begin{définition}\pcmn{足够}\end{définition}
\begin{exemple}\pjya{kɯki jamar ɲɯ-fɕaʁ}\hspace{5pt}\pcmn{这么多大概够了吧}\end{exemple}
\begin{sous-entrée}{mɯ́j-fɕaʁ}{ⓔfɕaʁⓗ2ⓝmɯ́j-fɕaʁ}
\begin{définition}\pfra{il faut non seulement ..}\end{définition}
\begin{définition}\pcmn{不但要……}\end{définition}
\begin{exemple}\pjya{tɯ-rɣi pjɯ-tu mɯ́j-fɕaʁ ma tɯ-ɣli kɯnɤ pjɯ-tu ra ma nɯ maʁ nɤ tɤ-rɤku tu-pe mɯ́j-cha}\hspace{5pt}\pcmn{不但要有种子,也要有肥料,不然庄稼不会长好}\end{exemple}\end{sous-entrée}

\end{entrée}

\begin{entrée}{fɕɤl}{}{ⓔfɕɤl} 
\classe{vi} \paradigme{dir}{nɯ-}
\begin{définition}\pfra{avoir la diarrhée}\end{définition}
\begin{définition}\pcmn{拉肚子}\end{définition}\paradigme{dir}{nɯ-}
\begin{définition}\pfra{causer une diarrhée}\end{définition}
\begin{définition}\pcmn{令人拉肚子}\end{définition}
\begin{exemple}\pjya{a-xtu ɲɯ-fɕɤl (=ɲɯ-nɯtɯfɕal-a)}\hspace{5pt}\pcmn{我拉肚子}\end{exemple}
\begin{exemple}\pjya{a-xtu nɯ-fɕɤl, a-xtu pɯ-fɕɤl}\hspace{5pt}\pcmn{我拉了肚子}\end{exemple}
\begin{exemple}\pjya{qaɕti tɤ-ndza-t-a tɕe a-xtu na-sɯfɕɤl}\hspace{5pt}\pcmn{我吃了桃子,令我拉了肚子}\end{exemple}\relationsémantique{参考}{\lien{ⓔnɯtɯfɕɤl}{nɯtɯfɕɤl}}
\begin{sous-entrée}{sɯfɕɤl}{ⓔfɕɤlⓝsɯfɕɤl} 
\classe{vt} \end{sous-entrée}

\étymologie{bɕal}\end{entrée}

\begin{entrée}{fɕɤm}{}{ⓔfɕɤm} 
\classe{vt} \paradigme{dir}{tɤ-}\paradigme{dir}{thɯ-}
\begin{définition}\pfra{étendre}\end{définition}
\begin{définition}\pcmn{摆出来}\end{définition}
\begin{exemple}\pjya{ɯʑo kɯ laχtɕha kɤ-ntsɣe ɯ-spa nɯ chɤ-fɕɤm}\hspace{5pt}\pcmn{我把卖的东西全部摆出来了}\end{exemple}\relationsémantique{同义词}{\lien{ⓔɕkho}{ɕkho}}\étymologie{bɕams}\end{entrée}

\begin{entrée}{fɕɤt}{₁}{ⓔfɕɤtⓗ1} 
\classe{vt} \sens{1}\paradigme{dir}{pɯ-}
\begin{définition}\pfra{raconter}\end{définition}
\begin{définition}\pcmn{讲故事;讲述}\end{définition}
\begin{exemple}\pjya{nɤʑo pɯ-tɯ-fɕɤt, ɯʑo kɯ pa-fɕɤt}\hspace{5pt}\pcmn{你讲述了,他讲述了}\end{exemple}
\begin{exemple}\pjya{a-χpi pa-fɕɤt}\hspace{5pt}\pcmn{他给我讲了故事}\end{exemple}\sens{1}\paradigme{dir}{pɯ-}
\begin{définition}\pfra{discuter}\end{définition}
\begin{définition}\pcmn{谈;聊天}\end{définition}
\begin{exemple}\pjya{ku-fɕɤt-tɕi}\hspace{5pt}\pcmn{我们俩在谈}\end{exemple}\sens{2}\paradigme{dir}{pɯ-}\paradigme{dir}{kɤ-}\paradigme{dir}{nɯ-}\paradigme{dir}{tɤ-}
\begin{définition}\pfra{danser pour}\end{définition}
\begin{définition}\pcmn{表演(舞蹈)}\end{définition}
\begin{définition}\pfra{raconter, rapporter une information}\end{définition}
\begin{définition}\pcmn{告诉;转告}\end{définition}
\begin{définition}\pfra{rapporter sa situation à qqun}\end{définition}
\begin{définition}\pcmn{把自己的情况告诉别人}\end{définition}
\begin{exemple}\pjya{a-tɯrɟaʁ ci kɤ-fɕɤt}\hspace{5pt}\pcmn{她为我表演了舞蹈}\end{exemple}
\begin{exemple}\pjya{a-ɕki tɤ-ʑɣɤfɕɤt}\hspace{5pt}\pcmn{他把他的情况跟我说了}\end{exemple}\relationsémantique{参考}{\lien{ⓔfɕafɕɤt}{fɕafɕɤt}}\relationsémantique{参考}{\lien{ⓔtɯfɕɤtⓗ1}{tɯfɕɤt}}
\begin{sous-entrée}{rɤfɕɤt}{ⓔfɕɤtⓗ1ⓢ2ⓝrɤfɕɤt} 
\classe{vi}  
\grammaire{apass} \end{sous-entrée}

\begin{sous-entrée}{afɕɤt}{ⓔfɕɤtⓗ1ⓢ2ⓝafɕɤt}
\begin{exemple}\pjya{kɯki mɤ-afɕɤt}\hspace{5pt}\pcmn{这个(故事)还没有讲}\end{exemple}\end{sous-entrée}

\begin{sous-entrée}{nɤfɕɯfɕɤt}{ⓔfɕɤtⓗ1ⓢ2ⓝnɤfɕɯfɕɤt} 
\classe{vt} 
\begin{définition}\pfra{raconter partout}\end{définition}
\begin{définition}\pcmn{到处传送}\end{définition}
\begin{exemple}\pjya{ki χpi ki kɤ-nɤfɕɯfɕɤt ci ŋu}\hspace{5pt}\pcmn{这个故事很出名}\end{exemple}\end{sous-entrée}

\begin{sous-entrée}{nɯɣɯfɕɤt}{ⓔfɕɤtⓗ1ⓢ2ⓝnɯɣɯfɕɤt} 
\classe{vs} 
\begin{définition}\pfra{facile à raconter}\end{définition}
\begin{définition}\pcmn{容易讲述}\end{définition}
\begin{exemple}\pjya{mɯ́j-nɯɣɯfɕɤt}\hspace{5pt}\pcmn{(这个故事)很难讲述,这件事情很难交代}\end{exemple}\end{sous-entrée}

\begin{sous-entrée}{ʑɣɤfɕɤt}{ⓔfɕɤtⓗ1ⓢ2ⓝʑɣɤfɕɤt} 
\classe{vi} \end{sous-entrée}

\begin{sous-entrée}{koŋla tú-wɣ-fɕɤt a-pɯ-ŋu tɕe}{ⓔfɕɤtⓗ1ⓢ2ⓝkoŋla tú-wɣ-fɕɤt a-pɯ-ŋu tɕe}
\begin{définition}\pfra{en fait}\end{définition}
\begin{définition}\pcmn{实际上;说白了;}\end{définition}\end{sous-entrée}

\étymologie{bɕad}\end{entrée}

\begin{entrée}{fɕɤt}{₂}{ⓔfɕɤtⓗ2} 
\classe{vi} \sens{1}\paradigme{dir}{tɤ-}
\begin{définition}\pfra{avoir cette chance}\end{définition}
\begin{définition}\pcmn{享受这个福分}\end{définition}
\begin{exemple}\pjya{ɲɯ-fɕɤt, mɯ́j-fɕɤt}\hspace{5pt}\pcmn{他有这个福分,没有这个福分}\end{exemple}
\begin{exemple}\pjya{aʑɯɣ mɤ-fɕɤt}\hspace{5pt}\pcmn{我没有资格享受这个福分}\end{exemple}\sens{2}\paradigme{dir}{kɤ-}
\begin{exemple}\pjya{lɯski kɤ-fɕɤt}\hspace{5pt}\pcmn{当然可以!(满口答应)}\end{exemple}\end{entrée}

\begin{entrée}{fɕɤtpa}{}{ⓔfɕɤtpa} 
\classe{n} 
\begin{définition}\pfra{fanfaronnade}\end{définition}
\begin{définition}\pcmn{大话}\end{définition}
\begin{exemple}\pjya{fɕɤtpa khro ma-pɯ-tɯ-lɤt, ɯ-ma nɯ tɤ-nɤme}\end{exemple}
\begin{exemple}\pjya{fɕɤtpa ɯ-tshɤt pɯ-lɤt tɕe ɯ-ma nɯ tɤ-nɤme}\hspace{5pt}\pcmn{少说大话,多办实事}\end{exemple}\end{entrée}

\begin{entrée}{fɕi}{}{ⓔfɕi} 
\classe{vl} \paradigme{dir}{thɯ-}
\begin{définition}\pfra{forger, travailler le métal}\end{définition}
\begin{définition}\pcmn{打铁;铸造}\end{définition}
\begin{exemple}\pjya{aʑo kɤ-fɕi spe-a}\hspace{5pt}\pcmn{我会打铁}\end{exemple}
\begin{exemple}\pjya{kɯki aj thɯ-fɕi-t-a ŋu}\hspace{5pt}\pcmn{这是我铸造的}\end{exemple}
\begin{exemple}\pjya{kɯki ɯʑo kɯ tha-fɕi}\hspace{5pt}\pcmn{这是他铸造的}\end{exemple}
\begin{exemple}\pjya{aʑo kɯre ku-fɕi-a}\hspace{5pt}\pcmn{我在这里打铁}\end{exemple}\relationsémantique{参考}{\lien{ⓔnɯɕɣɤthɯt}{nɯɕɣɤthɯt}}\relationsémantique{参考}{\lien{ⓔrɤlɤt}{rɤlɤt}}\end{entrée}

\begin{entrée}{fɕur}{}{ⓔfɕur} 
\classe{vt} \paradigme{dir}{tɤ-}\paradigme{dir}{pɯ-}
\begin{définition}\pfra{filtrer le thé, verser lentement}\end{définition}
\begin{définition}\pcmn{过滤茶叶,慢慢地倒下去}\end{définition}
\begin{exemple}\pjya{ɯʑo kɯ ta-fɕur}\hspace{5pt}\pcmn{他过滤了}\end{exemple}
\begin{exemple}\pjya{tʂha tɤ-fɕur-a}\hspace{5pt}\pcmn{我过滤了茶}\end{exemple}
\begin{exemple}\pjya{tɯ-ci tɤ-fɕur-a}\hspace{5pt}\pcmn{我过滤了水}\end{exemple}
\begin{exemple}\pjya{nɤʑo ji-tʂha tɤ-fɕur}\hspace{5pt}\pcmn{你过滤我们的茶吧}\end{exemple}\end{entrée}

\begin{entrée}{fɕɯɣ}{}{ⓔfɕɯɣ} 
\classe{vt} \sens{1}\paradigme{dir}{thɯ-}
\begin{définition}\pfra{déchirer (habits)}\end{définition}
\begin{définition}\pcmn{拆(衣服)}\end{définition}
\begin{exemple}\pjya{(tɯ-ŋga) tha-fɕɯɣ}\hspace{5pt}\pcmn{他拆了(衣服)}\end{exemple}\sens{2}\paradigme{dir}{pɯ-}\paradigme{dir}{tɤ-}
\begin{définition}\pfra{déchirer, démolir}\end{définition}
\begin{définition}\pcmn{拆(房子)}\end{définition}
\begin{exemple}\pjya{pɯ-fɕɯɣ-a}\hspace{5pt}\pcmn{我拆了}\end{exemple}
\begin{exemple}\pjya{pɯ-tɯ-fɕɯɣ}\hspace{5pt}\pcmn{你拆了}\end{exemple}
\begin{exemple}\pjya{pa-fɕɯɣ}\hspace{5pt}\pcmn{他拆了(房子)}\end{exemple}
\begin{exemple}\pjya{kha pjɤ-fɕɯɣ-nɯ}\hspace{5pt}\pcmn{他们把房子拆了}\end{exemple}
\begin{exemple}\pjya{ɕoŋβzu kɯ tʂɤm ɲɤ-fɕɯɣ}\hspace{5pt}\pcmn{木匠把板壁拆下来了}\end{exemple}
\begin{exemple}\pjya{kɯfɕi kɯ laχtɕha chɤ-fɕɯɣ}\hspace{5pt}\pcmn{铁匠把东西拆下来了}\end{exemple}\relationsémantique{同义词}{\lien{ⓔqia}{qia}}\end{entrée}

\begin{entrée}{fka}{₂}{ⓔfkaⓗ2} 
\classe{n} 
\begin{définition}\pfra{ordre}\end{définition}
\begin{définition}\pcmn{命令}\end{définition}\étymologie{bka}\end{entrée}

\begin{entrée}{fka}{₁}{ⓔfkaⓗ1} 
\classe{vi} \paradigme{dir}{tɤ-}\paradigme{dir}{nɯ-}\paradigme{dir}{tɤ-}\paradigme{dir}{tɤ-}
\begin{définition}\pfra{être rassasié}\end{définition}
\begin{définition}\pcmn{饱}\end{définition}
\begin{définition}\pfra{être gonflé}\end{définition}
\begin{définition}\pcmn{胀起来;鼓起来}\end{définition}
\begin{définition}\pfra{permettre à qqn de manger à sa faim}\end{définition}
\begin{définition}\pcmn{喂饱}\end{définition}
\begin{définition}\pfra{manger à sa faim}\end{définition}
\begin{définition}\pcmn{吃饱}\end{définition}
\begin{exemple}\pjya{tɤ-fka-a}\hspace{5pt}\pcmn{我饱了}\end{exemple}
\begin{exemple}\pjya{tɤ-tɯ-fka}\hspace{5pt}\pcmn{你饱了}\end{exemple}
\begin{exemple}\pjya{ɯʑo tɤ-fka}\hspace{5pt}\pcmn{他饱了}\end{exemple}
\begin{exemple}\pjya{ɯʑo a-tɤ-fka}\hspace{5pt}\pcmn{让他饱吧!}\end{exemple}
\begin{exemple}\pjya{tɤ-fkɯm ɲo-fka}\hspace{5pt}\pcmn{口袋鼓起来了}\end{exemple}
\begin{exemple}\pjya{tɤ-ɕɯfka-t-a}\hspace{5pt}\pcmn{我把他喂饱了}\end{exemple}
\begin{exemple}\pjya{tɤ-tɯ-ɕɯfka-t}\hspace{5pt}\pcmn{你喂饱了}\end{exemple}
\begin{exemple}\pjya{ɯʑo kɯ ta-ɕɯfka}\hspace{5pt}\pcmn{他喂饱了}\end{exemple}
\begin{exemple}\pjya{tɤ-pɤtso kɤ-ngu-t-a tɕe tɤ-ɕɯfka-t-a}\hspace{5pt}\pcmn{我把孩子喂饱了}\end{exemple}
\begin{exemple}\pjya{ɯ-ɣmba qale kɯ ɲɤ-ɕɯfka}\hspace{5pt}\pcmn{他鼓起了腮帮子}\end{exemple}
\begin{exemple}\pjya{aʑo tɤ-ʑɣɤɕɯfka-a}\hspace{5pt}\pcmn{我吃饱了}\end{exemple}
\begin{exemple}\pjya{tɤ-ʑɣɤɕɯfka je}\hspace{5pt}\pcmn{你要吃饱啊!不用客气!}\end{exemple}\relationsémantique{参考}{\lien{ⓔɯ-xtɤfka}{ɯ-xtɤfka}}
\begin{sous-entrée}{ɕɯfka}{ⓔfkaⓗ1ⓝɕɯfka} 
\classe{vt}  
\grammaire{caus} \end{sous-entrée}

\begin{sous-entrée}{ʑɣɤɕɯfka}{ⓔfkaⓗ1ⓝʑɣɤɕɯfka} 
\classe{vi}  
\grammaire{caus}
\grammaire{refl} \end{sous-entrée}

\begin{sous-entrée}{ɣɤfka}{ⓔfkaⓗ1ⓝɣɤfka} 
\classe{vs}  
\grammaire{facil} 
\begin{définition}\pfra{être facilement rassasié}\end{définition}
\begin{définition}\pcmn{容易饱}\end{définition}\end{sous-entrée}

\begin{sous-entrée}{sɤfka}{ⓔfkaⓗ1ⓝsɤfka} 
\classe{vs} 
\begin{définition}\pfra{qui rassasie facilement}\end{définition}
\begin{définition}\pcmn{吃……吃得饱}\end{définition}
\begin{exemple}\pjya{rɟɤɣi tú-wɣ-ndza tɕe sɤfka}\hspace{5pt}\pcmn{吃糌粑容易吃饱}\end{exemple}\end{sous-entrée}

\end{entrée}

\begin{entrée}{fkaβ}{}{ⓔfkaβ} 
\classe{vt} \paradigme{dir}{pɯ-}\paradigme{dir}{kɤ-}
\begin{définition}\pfra{couvrir}\end{définition}
\begin{définition}\pcmn{盖}\end{définition}\paradigme{dir}{pɯ-}\paradigme{dir}{kɤ-}
\begin{définition}\pfra{couvrir (avec quelque chose)}\end{définition}
\begin{définition}\pcmn{(用某个东西)盖}\end{définition}
\begin{exemple}\pjya{kɤ-fkaβ-a}\hspace{5pt}\pcmn{我盖了}\end{exemple}
\begin{exemple}\pjya{tɯ-mɯ pɯ-pa-fkaβ ʑo ɕ-tɤ-khat-a}\hspace{5pt}\pcmn{我走遍了天下}\end{exemple}
\begin{exemple}\pjya{ka-fkaβ}\hspace{5pt}\pcmn{他盖了}\end{exemple}
\begin{exemple}\pjya{pɯ-fkaβ-a}\hspace{5pt}\pcmn{我盖了}\end{exemple}
\begin{exemple}\pjya{tɯthɯ ka-fkaβ}\hspace{5pt}\pcmn{他把锅子盖上了}\end{exemple}
\begin{exemple}\pjya{tɕoχtsi pɯ-fkaβ}\hspace{5pt}\pcmn{你盖一下桌子吧!}\end{exemple}
\begin{exemple}\pjya{ɯ-pɤloʁ kɯ ɯ-zgrɯ kɯ-xtɕɯ-xtɕi chɯ-fkaβ jamar ɲɯ-ŋu ma nɯ ma mɯ́j-zri}\hspace{5pt}\pcmn{他的袖子不完全盖住他的肘,只有那么长}\end{exemple}
\begin{exemple}\pjya{a-pɯ-ɕɯfkaβ}\hspace{5pt}\pcmn{把它盖了吧}\end{exemple}
\begin{exemple}\pjya{kɤntɕhoz mɯ-nɯ-ra tɕe, pɯ-ɕɯfkaβ}\hspace{5pt}\pcmn{(这个锅子)不需要用了,(用锅盖)把他盖上吧!}\end{exemple}
\begin{exemple}\pjya{khɤlɤβ kɯ tɯthɯ pɯ-ɕɯfkaβ-a}\hspace{5pt}\pcmn{我用锅盖盖了锅子}\end{exemple}
\begin{sous-entrée}{ɕɯfkaβ}{ⓔfkaβⓝɕɯfkaβ} 
\classe{vt}  
\grammaire{caus} \end{sous-entrée}

\étymologie{bkab}\end{entrée}

\begin{entrée}{fkɤn}{}{ⓔfkɤn} 
\classe{vs} \sens{1}\paradigme{dir}{tɤ-}
\begin{définition}\pfra{sur (endroit)}\end{définition}
\begin{définition}\pcmn{安全(地方)}\end{définition}
\begin{exemple}\pjya{ki sɤtɕha wuma ʑo ɲɯ-fkɤn}\hspace{5pt}\pcmn{这个地方很安全}\end{exemple}\sens{2}
\begin{définition}\pfra{fiable (personne)}\end{définition}
\begin{définition}\pcmn{可靠(人)}\end{définition}\relationsémantique{参考}{\lien{ⓔrkɤl}{rkɤl}}\end{entrée}

\begin{entrée}{fkot}{}{ⓔfkot} 
\classe{vt} \paradigme{dir}{tɤ-}\paradigme{dir}{pɯ-}
\begin{définition}\pfra{établir}\end{définition}
\begin{définition}\pcmn{创造;创建;设计}\end{définition}
\begin{exemple}\pjya{pɯ-fkot-a}\hspace{5pt}\pcmn{我创建了}\end{exemple}
\begin{exemple}\pjya{ɯʑo kɯ ta-fkot}\hspace{5pt}\pcmn{他创建了}\end{exemple}
\begin{exemple}\pjya{nɤʑo tɤ-tɯ-fkot}\hspace{5pt}\pcmn{你创建了}\end{exemple}
\begin{exemple}\pjya{ɕoŋβzu kɯ kha kɤ-βzu pjɤ-fkot}\hspace{5pt}\pcmn{木匠设计了房子}\end{exemple}
\begin{exemple}\pjya{βdiwa ɲɤ-fkot}\hspace{5pt}\pcmn{他做了善事}\end{exemple}\étymologie{bkod}\end{entrée}

\begin{entrée}{fkoz}{}{ⓔfkoz}\relationsémantique{参考}{\lien{ⓔβgoz}{βgoz}}\end{entrée}

\begin{entrée}{fkur}{}{ⓔfkur} 
\classe{vt} \paradigme{dir}{tɤ-}\paradigme{dir}{tɤ-}
\begin{définition}\pfra{porter sur le dos}\end{définition}
\begin{définition}\pcmn{背}\end{définition}
\begin{définition}\pfra{aller et revenir en portant sur le dos}\end{définition}
\begin{définition}\pcmn{背来背去}\end{définition}
\begin{exemple}\pjya{tɤ-tɯ-fkur}\hspace{5pt}\pcmn{你背了}\end{exemple}
\begin{exemple}\pjya{ɯʑo kɯ ta-fkur}\hspace{5pt}\pcmn{他背了}\end{exemple}
\begin{exemple}\pjya{aʑo mɯ́j-cha-a tɕe nɤj tɤ-fkur}\hspace{5pt}\pcmn{我不行,你背吧}\end{exemple}
\begin{sous-entrée}{nɤfkɯfkur}{ⓔfkurⓝnɤfkɯfkur} 
\classe{vt} \end{sous-entrée}

\end{entrée}

\begin{entrée}{fkra}{}{ⓔfkra} 
\classe{vs} \paradigme{dir}{thɯ-}
\begin{définition}\pfra{magnifique (peau d'animal)}\end{définition}
\begin{définition}\pcmn{彩色斑斓}\end{définition}
\begin{exemple}\pjya{qachɣɤndʐi ɲɯ-fkra}\hspace{5pt}\pcmn{狐狸皮子彩色斑斓}\end{exemple}
\begin{exemple}\pjya{kɯrtsɤɣndʐi ɲɯ-fkra}\hspace{5pt}\pcmn{豹子皮子彩色斑斓}\end{exemple}\étymologie{bkra}\end{entrée}

\begin{entrée}{fkri}{}{ⓔfkri} 
\classe{vt} \paradigme{dir}{pɯ-}
\begin{définition}\pfra{ajouter une poudre dans un liquide}\end{définition}
\begin{définition}\pcmn{在液体里放粉状的物体然后搅拌(例如在汤里放盐)}\end{définition}
\begin{exemple}\pjya{pɯ-fkri-t-a}\hspace{5pt}\pcmn{我(在汤里)放了(盐、香料等)}\end{exemple}
\begin{exemple}\pjya{pɯ-tɯ-fkri-t}\hspace{5pt}\pcmn{你放了}\end{exemple}
\begin{exemple}\pjya{pa-fkri}\hspace{5pt}\pcmn{他放了}\end{exemple}
\begin{exemple}\pjya{tɤjlu pɯ-fkri-t-a}\hspace{5pt}\pcmn{我放了面粉}\end{exemple}
\begin{exemple}\pjya{@xiangliao pɯ-fkri-t-a}\hspace{5pt}\pcmn{我放了香料}\end{exemple}
\begin{exemple}\pjya{tsha pɯ-fkri-t-a}\hspace{5pt}\pcmn{我放了盐}\end{exemple}\end{entrée}

\begin{entrée}{fkro/\variante{fkrɤm}}{}{ⓔfkro} 
\classe{vt} \paradigme{dir}{\_}
\begin{définition}\pfra{mettre en ordre (des objets identiques)}\end{définition}
\begin{définition}\pcmn{排列整齐(相同的东西)}\end{définition}
\begin{exemple}\pjya{aʑo pɯ-fkro-t-a}\hspace{5pt}\pcmn{我布置了}\end{exemple}
\begin{exemple}\pjya{aʑo nɯ-βzdɤr pjɯ-fkram-a}\hspace{5pt}\pcmn{我把酥油分给大家}\end{exemple}\étymologie{bkram}\end{entrée}

\begin{entrée}{fkrɯz}{}{ⓔfkrɯz} 
\classe{vs} \paradigme{dir}{tɤ-}
\begin{définition}\pfra{avide, vorace}\end{définition}
\begin{définition}\pcmn{贪吃;贪得无厌}\end{définition}
\begin{exemple}\pjya{paʁ ɲɯ-fkrɯz}\hspace{5pt}\pcmn{猪很贪吃}\end{exemple}
\begin{exemple}\pjya{qapar ɲɯ-fkrɯz}\hspace{5pt}\pcmn{豺狗很贪吃}\end{exemple}
\begin{exemple}\pjya{jiɕqha tɯrme kɯ-fkrɯz ci ɲɯ-ŋu}\hspace{5pt}\pcmn{他是个贪吃的人}\end{exemple}
\begin{sous-entrée}{sɯfkrɯz}{ⓔfkrɯzⓝsɯfkrɯz}
\begin{exemple}\pjya{tu-ta-sɯ-fkrɯz}\hspace{5pt}\pcmn{我引发你的食欲}\end{exemple}\end{sous-entrée}

\étymologie{bkres}\end{entrée}

\begin{entrée}{fkurzʁe}{}{ⓔfkurzʁe} 
\classe{n} 
\begin{définition}\pfra{action de transporter sur le dos}\end{définition}
\begin{définition}\pcmn{背东西}\end{définition}
\begin{exemple}\pjya{fkurzʁe tɤ-βzu-t-a}\hspace{5pt}\pcmn{我背了很多东西}\end{exemple}\relationsémantique{参考}{\lien{ⓔnɯfkurzʁe}{nɯfkurzʁe}}\relationsémantique{参考}{\lien{ⓔfkur}{fkur}}\relationsémantique{参考}{\lien{ⓔnɯzʁe}{nɯzʁe}}\end{entrée}

\begin{entrée}{fkɯm}{}{ⓔfkɯm} 
\classe{vi} 
\begin{définition}\pfra{qui peut contenir (un liquide)}\end{définition}
\begin{définition}\pcmn{不漏水,能够装水}\end{définition}
\begin{exemple}\pjya{kɯki tɯthɯ ki mɯ́j-spoʁ tɕe tɯ-ci ɲɯ-fkɯm}\hspace{5pt}\pcmn{这个锅子没有洞,不漏水}\end{exemple}
\begin{exemple}\pjya{ɲɯ-spoʁ tɕe tɯ-ci mɯ́j-fkɯm ma pjɯ-nɯ-ɬoʁ ɲɯ-ɕti}\hspace{5pt}\pcmn{有个洞,不能装水否则会漏}\end{exemple}\end{entrée}

\begin{entrée}{frtɤn}{}{ⓔfrtɤn} 
\classe{vs} \paradigme{dir}{tɤ-}
\begin{définition}\pfra{fiable}\end{définition}
\begin{définition}\pcmn{可靠;耿直}\end{définition}
\begin{exemple}\pjya{tɯrme ɲɯ-frtɤn}\hspace{5pt}\pcmn{那个人很可靠}\end{exemple}\relationsémantique{同义词}{\lien{ⓔfkɤn}{fkɤn}}\étymologie{brtan}\end{entrée}

\begin{entrée}{fsaŋ}{}{ⓔfsaŋ} 
\classe{n} 
\begin{définition}\pfra{fumigation}\end{définition}
\begin{définition}\pcmn{烟(“求烟”、供神的烟)}\end{définition}
\begin{exemple}\pjya{fsaŋ la-ta}\hspace{5pt}\pcmn{他求了烟子}\end{exemple}
\begin{exemple}\pjya{fsaŋ tɤ-ta-t-a}\hspace{5pt}\pcmn{我求了烟子}\end{exemple}\relationsémantique{参考}{\lien{ⓔsɯfsaŋ}{sɯfsaŋ}}\relationsémantique{参考}{\lien{ⓔtɤfsaŋ}{tɤfsaŋ}}\étymologie{bsaŋ}\end{entrée}

\begin{entrée}{fsaŋkaŋ}{}{ⓔfsaŋkaŋ} 
\classe{n} 
\begin{définition}\pfra{cheminée}\end{définition}
\begin{définition}\pcmn{烟囱}\end{définition}\relationsémantique{参考}{\lien{ⓔfsaŋ}{fsaŋ}}\end{entrée}

\begin{entrée}{fsaŋkhɯ}{}{ⓔfsaŋkhɯ} 
\classe{n} 
\begin{définition}\pfra{fumigation}\end{définition}
\begin{définition}\pcmn{拜神的烟子}\end{définition}
\begin{exemple}\pjya{fsaŋkhɯ ta-tɕɤt}\hspace{5pt}\pcmn{他求了烟子}\end{exemple}\end{entrée}

\begin{entrée}{fsaŋmtɕhɤt}{}{ⓔfsaŋmtɕhɤt} 
\classe{n} 
\begin{définition}\pfra{fumigations}\end{définition}
\begin{définition}\pcmn{烧香拜佛冒出来的烟}\end{définition}\étymologie{bsaŋ.mtɕʰod}\end{entrée}

\begin{entrée}{fsapaʁ}{}{ⓔfsapaʁ} 
\classe{n} 
\begin{définition}\pfra{bétail}\end{définition}
\begin{définition}\pcmn{牲畜}\end{définition}\relationsémantique{参考}{\lien{ⓔarɯfsapaʁ}{arɯfsapaʁ}}\end{entrée}

\begin{entrée}{fsapaʁɣli}{}{ⓔfsapaʁɣli} 
\classe{n} 
\begin{définition}\pfra{purin}\end{définition}
\begin{définition}\pcmn{粪}\end{définition}\relationsémantique{参考}{\lien{ⓔfsapaʁ}{fsapaʁ}}\relationsémantique{参考}{\lien{ⓔtɯ-ɣli}{tɯ-ɣli}}\end{entrée}

\begin{entrée}{fsaqhe}{}{ⓔfsaqhe} 
\classe{n} 
\begin{définition}\pfra{l'année prochaine}\end{définition}
\begin{définition}\pcmn{明年}\end{définition}\relationsémantique{参考}{\lien{ⓔfso}{fso}}\end{entrée}

\begin{entrée}{fsɤl}{}{ⓔfsɤl} 
\classe{vt} 
\begin{exemple}\pjya{ji-kɯjŋu nɯ ɲɯ-fsal-a ɲɯ-sɯsam-a}\hspace{5pt}\pcmn{我要兑现我们的誓言}\end{exemple}\étymologie{bsal}\end{entrée}

\begin{entrée}{fsɤndɤpa}{}{ⓔfsɤndɤpa} 
\classe{n} 
\begin{définition}\pfra{l'année d'après}\end{définition}
\begin{définition}\pcmn{后年}\end{définition}\relationsémantique{参考}{\lien{ⓔfsɤndi}{fsɤndi}}\end{entrée}

\begin{entrée}{fsɤndi}{}{ⓔfsɤndi} 
\classe{adv} 
\begin{définition}\pfra{après demain}\end{définition}
\begin{définition}\pcmn{后天}\end{définition}
\begin{exemple}\pjya{fsɤndi tɕe li kɯmaʁ ji-kɤ-nɤma ɣɤʑu}\hspace{5pt}\pcmn{后天我们这里有另外一件事要做}\end{exemple}\end{entrée}

\begin{entrée}{fse}{₁}{ⓔfseⓗ1} 
\classe{vs} \paradigme{dir}{tɤ-}\sens{1}
\begin{définition}\pfra{ressembler}\end{définition}
\begin{définition}\pcmn{像}\end{définition}
\begin{exemple}\pjya{ɯ-mu fse}\hspace{5pt}\pcmn{他像他母亲}\end{exemple}
\begin{exemple}\pjya{ki kɯ-fse}\hspace{5pt}\pcmn{这样}\end{exemple}
\begin{exemple}\pjya{nɯ kɯ-fse}\hspace{5pt}\pcmn{那样}\end{exemple}
\begin{exemple}\pjya{nɯnɯ kɯ-fse kɯ-tu maŋe}\hspace{5pt}\pcmn{这不算什么}\end{exemple}
\begin{exemple}\pjya{mɤ-kɤ-nɯ-fse ʑo kɯ-me}\hspace{5pt}\pcmn{各种各样、五花八门}\end{exemple}
\begin{exemple}\pjya{a-tɯ-mbro nɤʑo ɯ-ɲɯ́-fse-a}\hspace{5pt}\pcmn{我跟你一样高吗?}\end{exemple}\sens{2}
\begin{définition}\pfra{être / se passer de cette manière}\end{définition}
\begin{définition}\pcmn{这样发生}\end{définition}
\begin{exemple}\pjya{ɯʑo kɯ nɯra fse mɯ-ɲɤ-sɯso}\hspace{5pt}\pcmn{我没有想到会那样}\end{exemple}
\begin{exemple}\pjya{tɕhi nɯ mɤ-nɯ-fse}\hspace{5pt}\pcmn{有什么不好?}\end{exemple}
\begin{exemple}\pjya{nɤ-stu tɤ-fse}\hspace{5pt}\pcmn{你小心一点}\end{exemple}
\begin{exemple}\pjya{rɟɤlpu fka ɕti tɕe, nɯ mɤ-tɯ-fse mɤ-jɤɣ}\hspace{5pt}\pcmn{这是皇上的旨意,不能不照办}\end{exemple}
\begin{exemple}\pjya{maka nɯ fse mɯ-ɲɤ-sɯso ri}\hspace{5pt}\pcmn{出乎意料(完全没有想到会那样)}\end{exemple}
\begin{sous-entrée}{nɤfse}{ⓔfseⓗ1ⓢ2ⓝnɤfse} 
\classe{vt} 
\begin{définition}\pfra{trouver ressemblant}\end{définition}
\begin{définition}\pcmn{觉得很像}\end{définition}\relationsémantique{参考}{\lien{ⓔsɤfse}{sɤfse}}\relationsémantique{参考}{\lien{ⓔamɯfse}{amɯfse}}\relationsémantique{参考}{\lien{ⓔnɯfseⓗ1}{nɯfse₁}}\relationsémantique{参考}{\lien{ⓔnɯfseⓗ2}{nɯfse₂}}\end{sous-entrée}

\end{entrée}

\begin{entrée}{fse}{₂}{ⓔfseⓗ2} 
\classe{vt} \sens{1}\paradigme{dir}{thɯ-}
\begin{définition}\pfra{aiguiser}\end{définition}
\begin{définition}\pcmn{磨(刀)}\end{définition}
\begin{exemple}\pjya{mbrɯtɕɯ thɯ-fse-t-a}\hspace{5pt}\pcmn{我磨了刀子}\end{exemple}
\begin{exemple}\pjya{thɯ-tɯ-fse-t}\hspace{5pt}\pcmn{你磨了}\end{exemple}
\begin{exemple}\pjya{tha-fse}\hspace{5pt}\pcmn{他磨了}\end{exemple}
\begin{exemple}\pjya{ldɯɣɯ tha-fse}\hspace{5pt}\pcmn{他磨了弯刀}\end{exemple}
\begin{exemple}\pjya{tɯrpa tha-fse}\hspace{5pt}\pcmn{他磨了斧头}\end{exemple}\sens{2}\paradigme{dir}{tɤ-}\paradigme{dir}{thɯ-}
\begin{définition}\pfra{frotter}\end{définition}
\begin{définition}\pcmn{搓}\end{définition}
\begin{définition}\pfra{aiguiser les couteaux}\end{définition}
\begin{définition}\pcmn{磨刀}\end{définition}
\begin{exemple}\pjya{a-jaʁ tɤ-fse-t-a}\hspace{5pt}\pcmn{我搓了一下手}\end{exemple}
\begin{sous-entrée}{rɤfse}{ⓔfseⓗ2ⓢ2ⓝrɤfse} 
\classe{vi}  
\grammaire{apass} \end{sous-entrée}

\begin{sous-entrée}{sɤfsɯfse}{ⓔfseⓗ2ⓢ2ⓝsɤfsɯfse}
\begin{définition}\pfra{frotter ... l'un contre l'autre}\end{définition}
\begin{définition}\pcmn{把……互相摩擦}\end{définition}
\begin{exemple}\pjya{ɯ-jaʁ to-sɤfsɯfse}\hspace{5pt}\pcmn{他搓了搓手}\end{exemple}\end{sous-entrée}

\end{entrée}

\begin{entrée}{fsjɤnɤfsjɤt}{}{ⓔfsjɤnɤfsjɤt} 
\classe{idph.3} 
\begin{définition}\pfra{qui marche sans efforts}\end{définition}
\begin{définition}\pcmn{形容七八岁的小孩子走路很轻松的样子}\end{définition}
\begin{exemple}\pjya{tɤ-pɤtso fsjɤnɤfsjɤt ʑo lɤ-ari}\hspace{5pt}\pcmn{小孩子很轻松地进去了}\end{exemple}\end{entrée}

\begin{entrée}{fsjitnɤfsjat}{}{ⓔfsjitnɤfsjat} 
\classe{idph.3} 
\begin{définition}\pfra{sifflement}\end{définition}
\begin{définition}\pcmn{吹口哨的声音}\end{définition}
\begin{exemple}\pjya{fsapaʁ ra nɯ-qhu tɕe fsjitnɤfsjat ʑo ɲɯ-ti tɕe ja-no}\hspace{5pt}\pcmn{他在牛后面“嗖—嗖”地吹口哨赶牛}\end{exemple}\end{entrée}

\begin{entrée}{fskɤr}{}{ⓔfskɤr} 
\classe{vt} \paradigme{dir}{tɤ-}
\begin{définition}\pfra{contourner}\end{définition}
\begin{définition}\pcmn{绕道;围着走}\end{définition}\paradigme{dir}{tɤ-}
\begin{définition}\pfra{tourner autour}\end{définition}
\begin{définition}\pcmn{转来转去}\end{définition}
\begin{exemple}\pjya{ɯʑo kɯ a-tɤ-fskɤr}\hspace{5pt}\pcmn{他绕过去吧!}\end{exemple}
\begin{exemple}\pjya{aʑo tu-fskar-a}\hspace{5pt}\pcmn{我绕过去}\end{exemple}
\begin{exemple}\pjya{ki rdɤstaʁ kɯ-wxti ci ɣɤʑu tɕe kɤ-fskɤr ɲɯ-ra}\hspace{5pt}\pcmn{这块大石头,要绕过去}\end{exemple}
\begin{exemple}\pjya{kɯki tʂu ki ɕɯ-kɤ-fskɤr ra}\hspace{5pt}\pcmn{要绕路}\end{exemple}
\begin{exemple}\pjya{si ɣɯ ɯ-rkɯ tú-wɣ-nɤfskɯfskɤr tɕe tɤjmɤɣ ɣɯ-mto}\hspace{5pt}\pcmn{在树的周围转来转去就会找到菌子}\end{exemple}\relationsémantique{参考}{\lien{ⓔtɯ-tɤfskɤr}{tɯ-tɤfskɤr}}
\begin{sous-entrée}{nɤfskɯfskɤr}{ⓔfskɤrⓝnɤfskɯfskɤr} 
\classe{vt} \end{sous-entrée}

\étymologie{skor}\end{entrée}

\begin{entrée}{fso}{}{ⓔfso} 
\classe{n} \sens{1}
\begin{définition}\pfra{demain}\end{définition}
\begin{définition}\pcmn{明天}\end{définition}
\begin{exemple}\pjya{ɯ-fso}\hspace{5pt}\pcmn{第二天}\end{exemple}\sens{2}
\begin{définition}\pfra{à l'avenir}\end{définition}
\begin{définition}\pcmn{将来}\end{définition}
\begin{exemple}\pjya{fso thɯ-tɯ-rgɤz tɕe kɤ-ɣi mɤ-tɯ-cha}\hspace{5pt}\pcmn{你将来老了就不再能来}\end{exemple}\end{entrée}

\begin{entrée}{fsomɯr}{}{ⓔfsomɯr} 
\classe{n} 
\begin{définition}\pfra{demain soir}\end{définition}
\begin{définition}\pcmn{明晚}\end{définition}
\begin{exemple}\pjya{ɯ-fsomɯr}\hspace{5pt}\pcmn{第二天晚上}\end{exemple}\end{entrée}

\begin{entrée}{fsoʁ}{₁}{ⓔfsoʁⓗ1} 
\classe{vt} \paradigme{dir}{nɯ-}
\begin{définition}\pfra{gagner (de l'argent)}\end{définition}
\begin{définition}\pcmn{挣(钱)}\end{définition}
\begin{exemple}\pjya{(tɯ-rɟɯ) nɯ-fso-ʁa}\hspace{5pt}\pcmn{我挣了(钱)}\end{exemple}
\begin{exemple}\pjya{nɯ-tɯ-fsoʁ}\hspace{5pt}\pcmn{你挣了}\end{exemple}
\begin{exemple}\pjya{na-fsoʁ}\hspace{5pt}\pcmn{他挣了}\end{exemple}
\begin{exemple}\pjya{tɯ-rɟɯ kɤ-fsoʁ ɴqa}\hspace{5pt}\pcmn{挣钱发财是一件很难的事情}\end{exemple}
\begin{exemple}\pjya{ɯʑo kɯ tɯ-rɟɯ kɤfsoʁ cha}\hspace{5pt}\pcmn{他很会挣钱}\end{exemple}
\begin{exemple}\pjya{qaʑo kú-wɣ-ftɕa tɕe, pɕawtsɯ sɤ-fsoʁ ŋu}\hspace{5pt}\pcmn{饲养羊是一种赚钱的方式}\end{exemple}
\begin{sous-entrée}{rɤfsoʁ}{ⓔfsoʁⓗ1ⓝrɤfsoʁ} 
\classe{vi}  
\grammaire{apass} 
\begin{exemple}\pjya{ɯʑo kɤ-rɤfsoʁ ɲɯ-rkaŋ}\hspace{5pt}\pcmn{他很会挣钱}\end{exemple}\end{sous-entrée}

\étymologie{bsogs}\end{entrée}

\begin{entrée}{fsoʁ}{₂}{ⓔfsoʁⓗ2} 
\classe{vs} \paradigme{dir}{nɯ-}\paradigme{dir}{lɤ-}\paradigme{dir}{nɯ-}\paradigme{dir}{nɯ-}
\begin{définition}\pfra{clair (ciel)}\end{définition}
\begin{définition}\pcmn{亮(天色)}\end{définition}
\begin{définition}\pfra{rendre clair}\end{définition}
\begin{définition}\pcmn{使变亮}\end{définition}
\begin{définition}\pfra{rendre clair}\end{définition}
\begin{définition}\pcmn{使变亮}\end{définition}
\begin{exemple}\pjya{kɯki ɲɯ-fsoʁ}\hspace{5pt}\pcmn{这个很亮}\end{exemple}
\begin{exemple}\pjya{tɤtʂu ɲɯ-fsoʁ}\hspace{5pt}\pcmn{灯很亮}\end{exemple}
\begin{exemple}\pjya{tɯ-mɯ ɲɤ-fsoʁ}\hspace{5pt}\pcmn{天亮开了(原来是阴天)}\end{exemple}
\begin{exemple}\pjya{kha ɲo-fsoʁ}\hspace{5pt}\pcmn{房间变亮了}\end{exemple}
\begin{exemple}\pjya{tɯtshot kɯtʂɤɣ ko-zɣɯt tɕe lo-fsoʁ}\hspace{5pt}\pcmn{到了六点钟,天就亮}\end{exemple}
\begin{exemple}\pjya{kɯki kha ɲɯ-qanɯ ri, kɤ-ɣɤfsoʁ khɯ}\hspace{5pt}\pcmn{这个房间很黑,可以令它变亮}\end{exemple}\relationsémantique{参考}{\lien{ⓔtɯfsɤkha}{tɯfsɤkha}}\relationsémantique{反义词}{\lien{ⓔqanɯ}{qanɯ}}
\begin{sous-entrée}{ɣɤfsoʁ}{ⓔfsoʁⓗ2ⓝɣɤfsoʁ} 
\classe{vt} \end{sous-entrée}

\begin{sous-entrée}{sɯfsoʁ}{ⓔfsoʁⓗ2ⓝsɯfsoʁ} 
\classe{vt} \end{sous-entrée}

\end{entrée}

\begin{entrée}{fsosɲɯm}{}{ⓔfsosɲɯm} 
\classe{n} 
\begin{définition}\pfra{aumônes (aux moines)}\end{définition}
\begin{définition}\pcmn{布施}\end{définition}
\begin{exemple}\pjya{wortɕhi ʑo, a-fsosɲɯm ci pɯ-lɤt}\hspace{5pt}\pcmn{请你给我布施}\end{exemple}
\begin{exemple}\pjya{fsosɲɯm nɯ-sɤmbi-a}\hspace{5pt}\pcmn{我讨布施了}\end{exemple}\end{entrée}

\begin{entrée}{fsraŋ/\variante{fsroŋ}}{}{ⓔfsraŋ} 
\classe{vt} \paradigme{dir}{tɤ-}\sens{1}
\begin{définition}\pfra{sauver}\end{définition}
\begin{définition}\pcmn{救……一命}\end{définition}
\begin{exemple}\pjya{tɤ-fsraŋ-a}\hspace{5pt}\pcmn{我救了他}\end{exemple}
\begin{exemple}\pjya{ta-fsraŋ}\hspace{5pt}\pcmn{他救了他}\end{exemple}
\begin{exemple}\pjya{tɤ-kɯ-fsraŋ-a}\hspace{5pt}\pcmn{你救了我}\end{exemple}
\begin{exemple}\pjya{aʑo a-kɯ-mŋɤm pɯ-tu ri, ɯʑo kɯ tɤ́-wɣ-fsraŋ-a ŋu}\hspace{5pt}\pcmn{在我生病的时候,他救了我的命}\end{exemple}
\begin{exemple}\pjya{a-sroʁ pɯ-sɤɣʑɯr ri tɤ́-wɣ-fsroŋ-a ŋu}\hspace{5pt}\pcmn{在我生命垂危的时候,他救了我}\end{exemple}\sens{2}\paradigme{dir}{tɤ-}
\begin{définition}\pfra{protéger}\end{définition}
\begin{définition}\pcmn{保护}\end{définition}
\begin{définition}\pfra{se protéger}\end{définition}
\begin{définition}\pcmn{保护自己}\end{définition}\relationsémantique{同义词}{\lien{ⓔtɯ-sroʁ,ri}{tɯ-sroʁ,ri}}
\begin{sous-entrée}{ʑɣɤfsraŋ}{ⓔfsraŋⓢ2ⓝʑɣɤfsraŋ} 
\classe{vi}  
\grammaire{refl} \end{sous-entrée}

\étymologie{bsruŋ}\end{entrée}

\begin{entrée}{fsraŋma/\variante{sraŋma}}{}{ⓔfsraŋma} 
\classe{n} 
\begin{définition}\pfra{divinité protectrice}\end{définition}
\begin{définition}\pcmn{神仙}\end{définition}\étymologie{bsruŋ.ma}\end{entrée}

\begin{entrée}{fsusqi}{}{ⓔfsusqi} 
\classe{num} 
\begin{définition}\pfra{trente}\end{définition}
\begin{définition}\pcmn{三十}\end{définition}\end{entrée}

\begin{entrée}{fsusqipa}{}{ⓔfsusqipa}\relationsémantique{参考}{\lien{ⓔfsusqi}{fsusqi}}\relationsémantique{参考}{\lien{ⓔtɯ-xpa}{tɯ-xpa}}\end{entrée}

\begin{entrée}{fstɤt}{}{ⓔfstɤt} 
\classe{vt} \paradigme{dir}{tɤ-}
\begin{définition}\pfra{louer}\end{définition}
\begin{définition}\pcmn{称赞;抬举}\end{définition}
\begin{exemple}\pjya{tu-ta-fstɤt}\hspace{5pt}\pcmn{我称赞你}\end{exemple}
\begin{exemple}\pjya{to-fstɤt}\hspace{5pt}\pcmn{他称赞了他}\end{exemple}
\begin{exemple}\pjya{nɯsthɯci tu-kɯ-fstat-a mɤ-ra wo!}\hspace{5pt}\pcmn{你过奖了}\end{exemple}\relationsémantique{同义词}{\lien{ⓔɣɤmɯ}{ɣɤmɯ}}\étymologie{bstod}\end{entrée}

\begin{entrée}{fstɯn}{}{ⓔfstɯn} 
\classe{vt}  
\grammaire{apass} \paradigme{dir}{kɤ-}
\begin{définition}\pfra{servir, s'occuper de}\end{définition}
\begin{définition}\pcmn{伺候;照顾}\end{définition}\paradigme{dir}{kɤ-}
\begin{exemple}\pjya{kɤ-fstɯn-a}\hspace{5pt}\pcmn{我照顾了他}\end{exemple}
\begin{exemple}\pjya{kɤ-tɯ-fstɯn}\hspace{5pt}\pcmn{你照顾了他}\end{exemple}
\begin{exemple}\pjya{ɯʑo kɯ kɤ́-wɣ-fstɯna}\hspace{5pt}\pcmn{他照顾了我}\end{exemple}
\begin{sous-entrée}{ʑɣɤfstɯn}{ⓔfstɯnⓝʑɣɤfstɯn} 
\classe{vi}  
\grammaire{refl} 
\begin{définition}\pfra{s'occuper de soi-même}\end{définition}
\begin{définition}\pcmn{照顾自己}\end{définition}\end{sous-entrée}

\begin{sous-entrée}{sɤfstɯn}{ⓔfstɯnⓝsɤfstɯn} 
\classe{vi} \end{sous-entrée}

\begin{définition}\pfra{servir}\end{définition}
\begin{définition}\pcmn{伺候}\end{définition}
\begin{exemple}\pjya{nɤʑo kɤ-sɤfstɯn tɕhi tu-tɯ-fse ŋu}\hspace{5pt}\pcmn{你是怎么伺候人家的呢?}\end{exemple}\étymologie{bstun}\end{entrée}

\begin{entrée}{fsɯɣ}{}{ⓔfsɯɣ} 
\classe{vt} \paradigme{dir}{nɯ-}\paradigme{dir}{thɯ-}
\begin{définition}\pfra{rendre la monnaie}\end{définition}
\begin{définition}\pcmn{退还多余的部分,退零钱}\end{définition}
\begin{exemple}\pjya{aʑo nɯ-fsɯɣ-a}\hspace{5pt}\pcmn{我还了}\end{exemple}
\begin{exemple}\pjya{nɤʑo nɯ-tɯ-fsɯɣ}\hspace{5pt}\pcmn{你还了}\end{exemple}
\begin{exemple}\pjya{ɯʑo kɯ na-fsɯɣ}\hspace{5pt}\pcmn{他还了}\end{exemple}
\begin{exemple}\pjya{ɯ-phɯ ɯ-kho nɯ-kɯ-ro nɯnɯ nɯ́-wɣ-fsɯɣ-a}\hspace{5pt}\pcmn{他把多给的钱还给我了}\end{exemple}
\begin{exemple}\pjya{ki ɯ-phɯ ki kɤ-kho ɲɤ-ro tɕe ɲɯ-ta-fsɯɣ}\hspace{5pt}\pcmn{你多给了钱,我找一下零钱给你。}\end{exemple}
\begin{exemple}\pjya{pɕawtsɯ ɲɤ-fsɯɣ}\hspace{5pt}\pcmn{他找了零钱}\end{exemple}\relationsémantique{同义词}{\lien{ⓔɕɣɤz}{ɕɣɤz}}\étymologie{gsug.pa}\end{entrée}

\begin{entrée}{fsɯr}{}{ⓔfsɯr} 
\classe{vs} \paradigme{dir}{tɤ-}
\begin{définition}\pfra{avoir besoin de viande}\end{définition}
\begin{définition}\pcmn{需要吃肉}\end{définition}
\begin{exemple}\pjya{to-ngo tɕe ɲɯ-fsɯr}\hspace{5pt}\pcmn{他生病了,需要吃肉}\end{exemple}
\begin{exemple}\pjya{nɤʑo tɤ-mthɯm tu-tɯ-ndze ntsɯ ɕti ri, nɯ kɯnɤ ɲɯ-tɯ-fsɯr!}\hspace{5pt}\pcmn{你总是吃这么多肉,还是那么馋嘴、}\end{exemple}\end{entrée}

\begin{entrée}{fsɯz}{}{ⓔfsɯz} 
\classe{vi} \paradigme{dir}{pɯ-}
\begin{définition}\pfra{possible}\end{définition}
\begin{définition}\pcmn{可能}\end{définition}
\begin{exemple}\pjya{fsɯfsɯz ʑo}\hspace{5pt}\pcmn{千方百计}\end{exemple}
\begin{exemple}\pjya{aʑo a-pɯ-fsɯz tɕe, sɲikuku ʑo ɣi-a ɕti}\hspace{5pt}\pcmn{只有我可以,我天天来}\end{exemple}
\begin{exemple}\pjya{ɯʑo a-pɯ-fsɯz tɕe, ɯʑo laχtɕha ra nɯ-ndɤm, tɯrme laχtɕha kɯnɤ nɯ-ndɤm}\hspace{5pt}\pcmn{只要他有机会,他不但会拿自己的东西,也会拿别人的}\end{exemple}\end{entrée}

\begin{entrée}{ftɕa}{}{ⓔftɕa} 
\classe{vt} \paradigme{dir}{tɤ-}
\begin{définition}\pfra{posséder}\end{définition}
\begin{définition}\pcmn{拥有}\end{définition}
\begin{exemple}\pjya{jiʑo @shouji tɤ-ftɕa-j}\hspace{5pt}\pcmn{我有了手机}\end{exemple}
\begin{exemple}\pjya{qaʑo kú-wɣ-ftɕa tɕe, pɕawtsɯ sɤ-fsoʁ ŋu}\hspace{5pt}\pcmn{拥有绵羊是赚钱的方式}\end{exemple}\relationsémantique{同义词}{\lien{ⓔaro}{aro}}\end{entrée}

\begin{entrée}{ftɕaka}{}{ⓔftɕaka} 
\classe{n}
\classe{vt} 
\begin{définition}\pfra{méthode}\end{définition}
\begin{définition}\pcmn{办法}\end{définition}\relationsémantique{参考}{\lien{ⓔaftɕaka}{aftɕaka}}\relationsémantique{参考}{\lien{ⓔrɯftɕaka}{rɯftɕaka}}\relationsémantique{参考}{\lien{ⓔsɤftɕaka}{sɤftɕaka}}\relationsémantique{Component 2}{\lien{}{βzu}}
\begin{sous-entrée}{ftɕaka,βzu}{ⓔftɕakaⓝftɕaka,βzu} 
\classe{n} 
\begin{définition}\pfra{essayer par tous les moyens de}\end{définition}
\begin{définition}\pcmn{想办法……}\end{définition}
\begin{exemple}\pjya{tɤ-rɤku kɤ-mɯrkɯ ftɕaka wuma ʑo βze}\hspace{5pt}\pcmn{它想尽办法偷吃庄稼}\end{exemple}\relationsémantique{Component 1}{\lien{ⓔftɕaka}{ftɕaka}}\end{sous-entrée}

\étymologie{btɕa.ka}\end{entrée}

\begin{entrée}{ftɕar}{}{ⓔftɕar} 
\classe{n} 
\begin{définition}\pfra{été, période du printemps à l'été}\end{définition}
\begin{définition}\pcmn{夏天;春夏}\end{définition}
\begin{exemple}\pjya{ftɕar kɤ-ndzoʁ}\end{exemple}
\begin{exemple}\pjya{ftɕar ɯ-qa ka-ta}\hspace{5pt}\pcmn{春天开始了}\end{exemple}\relationsémantique{参考}{\lien{ⓔftɕɤru}{ftɕɤru}}\end{entrée}

\begin{entrée}{ftɕaʁ}{}{ⓔftɕaʁ} 
\classe{vt} \paradigme{dir}{pɯ-}
\begin{définition}\pfra{être une brebis gâleuse}\end{définition}
\begin{définition}\pcmn{一个老鼠屎坏了一锅粥}\end{définition}
\begin{exemple}\pjya{ma-pɯ-kɯ-ftɕaʁ-i}\hspace{5pt}\pcmn{不要成为我们团队里的老鼠屎}\end{exemple}\relationsémantique{参考}{\lien{ⓔsɯftɕaʁ}{sɯftɕaʁ}}
\begin{sous-entrée}{sɤftɕaʁ}{ⓔftɕaʁⓝsɤftɕaʁ} 
\grammaire{apass} 
\begin{exemple}\pjya{ma-ɕɯ-tɯ-sɤftɕaʁ}\hspace{5pt}\pcmn{你不要做老鼠屎}\end{exemple}\end{sous-entrée}

\end{entrée}

\begin{entrée}{ftɕɤfkɤt}{}{ⓔftɕɤfkɤt} 
\classe{n} \sens{1}
\begin{définition}\pfra{organisation}\end{définition}
\begin{définition}\pcmn{指挥;安排(事情)}\end{définition}
\begin{exemple}\pjya{jisŋi kɤ-nɤma kɯkɯra thamtɕɤt aʑo ji-ftɕɤfkɤt tu-βze-a}\hspace{5pt}\pcmn{今天我们要做的工作,我来指挥}\end{exemple}\sens{2}
\begin{définition}\pfra{suggestion, idée, conseil}\end{définition}
\begin{définition}\pcmn{主意}\end{définition}
\begin{exemple}\pjya{nɯ-ftɕɤfkɤt to-tɕɤt}\hspace{5pt}\pcmn{他给他们出了主意(给他们做了安排)}\end{exemple}
\begin{exemple}\pjya{a-ftɕɤfkɤt ci tɤ-tɕɤt}\hspace{5pt}\pcmn{给我出主意吧}\end{exemple}\relationsémantique{同义词}{\lien{ⓔβluβra}{βluβra}}\relationsémantique{参考}{\lien{ⓔrɯftɕɤfkɤt}{rɯftɕɤfkɤt}}\étymologie{btɕa.bkod?}\end{entrée}

\begin{entrée}{ftɕɤl}{}{ⓔftɕɤl} 
\classe{vt} \paradigme{dir}{tɤ-}
\begin{définition}\pfra{demander à qqn de faire qqch pour soi}\end{définition}
\begin{définition}\pcmn{请别人做事}\end{définition}
\begin{exemple}\pjya{tó-wɣ-ftɕala}\hspace{5pt}\pcmn{他请了我}\end{exemple}
\begin{exemple}\pjya{to-tɯ́-wɣ-ftɕɤl}\hspace{5pt}\pcmn{他请了你}\end{exemple}
\begin{exemple}\pjya{a-laχtɕha kɤ-χtɯ ci tu-ta-ftɕɤl}\hspace{5pt}\pcmn{我请你帮我买(某种东西)}\end{exemple}
\begin{sous-entrée}{sɤftɕɤl}{ⓔftɕɤlⓝsɤftɕɤl} 
\classe{vi} 
\begin{définition}\pfra{demander à des gens de faire qqch pour soi}\end{définition}
\begin{définition}\pcmn{请人做事}\end{définition}
\begin{exemple}\pjya{ɲɯ-sɤftɕɤl}\hspace{5pt}\pcmn{他请别人做事}\end{exemple}\relationsémantique{同义词}{\lien{ⓔsqɤr}{sqɤr}}\end{sous-entrée}

\étymologie{btɕol}\end{entrée}

\begin{entrée}{ftɕɤru}{}{ⓔftɕɤru} 
\classe{n} 
\begin{définition}\pfra{chemin au milieu des champs (pour éviter d'abîmer les récoltes)}\end{définition}
\begin{définition}\pcmn{阡陌(夏天为了避免踩坏庄稼而特意在田地里开辟的一条小道)}\end{définition}\relationsémantique{参考}{\lien{ⓔftɕar}{ftɕar}}\relationsémantique{参考}{\lien{ⓔtʂu}{tʂu}}\end{entrée}

\begin{entrée}{ftɕɤt}{}{ⓔftɕɤt} 
\classe{vt} \sens{1}\paradigme{dir}{nɯ-}
\begin{définition}\pfra{s'abstenir, renoncer à une mauvaise habitude}\end{définition}
\begin{définition}\pcmn{戒掉}\end{définition}
\begin{exemple}\pjya{ɯʑo kɯ na-ftɕɤt}\hspace{5pt}\pcmn{他戒掉了}\end{exemple}
\begin{exemple}\pjya{nɤʑo nɯ-tɯ-ftɕɤt}\hspace{5pt}\pcmn{你戒掉了}\end{exemple}
\begin{exemple}\pjya{cha kɤ-tshi mɯ́j-sna tɕe nɯ-ftɕat-a}\hspace{5pt}\pcmn{喝酒是不好的,所以我就戒掉了}\end{exemple}
\begin{exemple}\pjya{thamakha nɯ-ftɕat-a}\hspace{5pt}\pcmn{我戒了烟}\end{exemple}
\begin{exemple}\pjya{thamakha kɤ-sko na-ftɕɤt}\hspace{5pt}\pcmn{他戒了烟}\end{exemple}\sens{2}
\begin{définition}\pfra{subjuger, soumettre}\end{définition}
\begin{définition}\pcmn{征服;使……驯服}\end{définition}\relationsémantique{同义词}{\lien{ⓔftɯlⓗ2}{ftɯl₂}}\étymologie{btɕad}\end{entrée}

\begin{entrée}{ftɕɤz}{}{ⓔftɕɤz} 
\classe{vt} \paradigme{dir}{nɯ-}\paradigme{dir}{nɯ-}
\begin{définition}\pfra{castrer}\end{définition}
\begin{définition}\pcmn{阉割}\end{définition}
\begin{définition}\pfra{castrer les animaux}\end{définition}
\begin{définition}\pcmn{阉割}\end{définition}
\begin{exemple}\pjya{nɯ-ftɕaz-a}\hspace{5pt}\pcmn{我阉割了}\end{exemple}
\begin{exemple}\pjya{mbala na-ftɕɤz}\hspace{5pt}\pcmn{他把公牛阉割了}\end{exemple}
\begin{exemple}\pjya{mbala kɤ-ftɕɤz ra}\hspace{5pt}\pcmn{要把公牛阉割掉}\end{exemple}\relationsémantique{同义词}{\lien{ⓔɣɤβdi}{ɣɤβdi}}
\begin{sous-entrée}{rɤftɕɤz}{ⓔftɕɤzⓝrɤftɕɤz} 
\classe{vi}  
\grammaire{apass} \end{sous-entrée}

\end{entrée}

\begin{entrée}{ftɕhoʁftɕhoʁ}{}{ⓔftɕhoʁftɕhoʁ} 
\classe{idph.2} 
\begin{définition}\pfra{petit et dressé}\end{définition}
\begin{définition}\pcmn{形容小的物体(如耳朵)立起来的样子}\end{définition}
\begin{exemple}\pjya{qala ɯ-rna ftɕhoʁftɕhoʁ ʑo ɲɯ-pa}\hspace{5pt}\pcmn{兔子的耳朵是翘起来的}\end{exemple}\end{entrée}

\begin{entrée}{ftɕhur}{}{ⓔftɕhur} 
\classe{vt} \sens{1}\paradigme{dir}{tɤ-}\paradigme{dir}{lɤ-}
\begin{définition}\pfra{relever, mettre verticalement}\end{définition}
\begin{définition}\pcmn{竖起来;立起来}\end{définition}
\begin{exemple}\pjya{ɯʑo kɯ ta-ftɕhur}\hspace{5pt}\pcmn{他(把那个东西)立起来了}\end{exemple}
\begin{exemple}\pjya{kɯki laʁdɯn ki pjɤ-ndʐaβ tɕe tu-ftɕhur-a}\hspace{5pt}\pcmn{这个工具倒了,我把它立起来}\end{exemple}
\begin{exemple}\pjya{ɕoŋtɕa lo-ftɕhur-a}\hspace{5pt}\pcmn{我把木料竖起来了}\end{exemple}
\begin{exemple}\pjya{khɯna kɯ ɯ-rna to-ftɕhur (=to-znɯndzɯ) ʑo tɕe ɲɯ-sɤŋo}\hspace{5pt}\pcmn{狗把耳朵竖起来听}\end{exemple}\relationsémantique{同义词}{\lien{ⓔnɯndzɯⓝznɯndzɯ}{znɯndzɯ}}\sens{2}\paradigme{dir}{pɯ-}
\begin{définition}\pfra{verser complètement}\end{définition}
\begin{définition}\pcmn{倒干}\end{définition}
\begin{exemple}\pjya{kɯki ɯ-ro ci ɣɤʑu tɕe, pjɯ-ftɕhur-a tɕe kɤ-tshi}\hspace{5pt}\pcmn{剩了一点,我把他倒干了,你喝吧}\end{exemple}
\begin{exemple}\pjya{mɯ-to-rɯdzaŋspa-a tɕe, tʂha ɯ-ro nɯ pjɤ-ftɕhur-a}\hspace{5pt}\pcmn{我不小心把剩下的茶倒干了}\end{exemple}
\begin{exemple}\pjya{laʁjɯɣ to-ftɕhur}\hspace{5pt}\pcmn{他把棍子立起来了}\end{exemple}\relationsémantique{同义词}{\lien{ⓔnɯndzɯⓝznɯndzɯ}{znɯndzɯ}}\end{entrée}

\begin{entrée}{ftɕit}{}{ⓔftɕit} 
\classe{vt} \paradigme{dir}{pɯ-}\sens{1}
\begin{définition}\pfra{prendre en charge}\end{définition}
\begin{définition}\pcmn{掌管}\end{définition}
\begin{exemple}\pjya{ki kha ki nɤʑo pɯ-nɯ-ftɕit tɕe sɤcɯ tɤ-nɯ-ndɤm}\hspace{5pt}\pcmn{这套房子由你来掌管,钥匙就交给你了}\end{exemple}\sens{2}
\begin{définition}\pfra{prendre le contrôle de}\end{définition}
\begin{définition}\pcmn{霸占}\end{définition}\end{entrée}

\begin{entrée}{ftɕɯm}{}{ⓔftɕɯm} 
\classe{vt} \sens{1}
\begin{définition}\pfra{digérer}\end{définition}
\begin{définition}\pcmn{消化(食物)}\end{définition}\sens{2}
\begin{définition}\pfra{apprivoiser}\end{définition}
\begin{définition}\pcmn{驯服}\end{définition}
\begin{exemple}\pjya{jla ki nɯ kóʁmɯz kɤ-sɯxɕɤt ɲɯ-ɕti tɕe kɤ-ftɕɯm ɲɯ-ɴqa}\hspace{5pt}\pcmn{这头犏牛刚刚开始驯养,很难驯服}\end{exemple}
\begin{sous-entrée}{sɯftɕɯm}{ⓔftɕɯmⓝsɯftɕɯm} 
\classe{vt}  
\grammaire{habil} 
\begin{définition}\pfra{être capable de digérer, d'assimiler (un médicament)}\end{définition}
\begin{définition}\pcmn{消化得了}\end{définition}
\begin{exemple}\pjya{smɤn ɲɯ-sna tɕe mɯ́j-sɯftɕɯm-a}\hspace{5pt}\pcmn{药量过多,我吸收不了}\end{exemple}\end{sous-entrée}

\étymologie{btɕom}\end{entrée}

\begin{entrée}{ftɕɯpa}{}{ⓔftɕɯpa} 
\classe{n} 
\begin{définition}\pfra{dixième mois}\end{définition}
\begin{définition}\pcmn{十月}\end{définition}\étymologie{btɕu.pa}\end{entrée}

\begin{entrée}{ftɕɯʁɲiz}{}{ⓔftɕɯʁɲiz} 
\classe{n} 
\begin{définition}\pfra{douzième mois}\end{définition}
\begin{définition}\pcmn{十二月}\end{définition}\étymologie{btɕu.gɲis.pa}\end{entrée}

\begin{entrée}{ftɕɯt}{}{ⓔftɕɯt} 
\classe{vt} \paradigme{dir}{pɯ-}\sens{1}
\begin{définition}\pfra{diriger, régner sur}\end{définition}
\begin{définition}\pcmn{统治}\end{définition}
\begin{exemple}\pjya{rɟɤlpu lo-ndo tɕe sɤtɕha pjɤ-ftɕɯt pjɤ-cha}\hspace{5pt}\pcmn{他登上王位之后统治了国家}\end{exemple}\sens{2}
\begin{définition}\pfra{être en charge de}\end{définition}
\begin{définition}\pcmn{掌管}\end{définition}
\begin{exemple}\pjya{ki kɯm ki, nɤʑo sɤcɯ tɤ-ndɤm tɕe pɯ-nɯ-ftɕɯt}\hspace{5pt}\pcmn{你拿着门的钥匙来掌管(这个房间)}\end{exemple}\end{entrée}

\begin{entrée}{ftɕɯχtɕɯɣ}{}{ⓔftɕɯχtɕɯɣ} 
\classe{n} 
\begin{définition}\pfra{onzième mois}\end{définition}
\begin{définition}\pcmn{十一月}\end{définition}\étymologie{btɕu.gtɕig.pa}\end{entrée}

\begin{entrée}{fte}{}{ⓔfte} 
\classe{vs} \paradigme{dir}{nɯ-}\sens{1}
\begin{définition}\pfra{émacié et livide}\end{définition}
\begin{définition}\pcmn{憔悴,没有血色}\end{définition}
\begin{exemple}\pjya{ɯ-kɯ-mŋɤm pjɤ-thɯ tɕe ɯ-rŋa ra ɲɤ-fte ʑo}\hspace{5pt}\pcmn{他的病严重了,脸上没有血色了}\end{exemple}\sens{2}
\begin{définition}\pfra{se décolorer (habits)}\end{définition}
\begin{définition}\pcmn{褪色(衣服)}\end{définition}
\begin{exemple}\pjya{ɯ-ŋga khro to-ŋga tɕe ɲɤ-fte ʑo}\hspace{5pt}\pcmn{他衣服穿了很久,都褪色了}\end{exemple}\end{entrée}

\begin{entrée}{ftsaʁra}{}{ⓔftsaʁra} 
\classe{n} 
\begin{définition}\pfra{plaque de pierre pour empêcher l'eau du toit de couler}\end{définition}
\begin{définition}\pcmn{用以挡住屋檐漏水的石板或木板}\end{définition}\end{entrée}

\begin{entrée}{ftsɤn}{}{ⓔftsɤn} 
\classe{vs} 
\begin{définition}\pfra{sévère}\end{définition}
\begin{définition}\pcmn{严格}\end{définition}
\begin{exemple}\pjya{khɯna kɯ-ftsɤn ci ɲɯ-ŋu}\hspace{5pt}\pcmn{是一条很凶的狗(看家的)}\end{exemple}
\begin{exemple}\pjya{sɤcɯ ɲɯ-ftsɤn}\hspace{5pt}\pcmn{钥匙很可靠(锁不容易被人家打开)}\end{exemple}
\begin{exemple}\pjya{sɤtɕha ɲɯ-ftsɤn}\hspace{5pt}\pcmn{这个地方安全}\end{exemple}
\begin{exemple}\pjya{@laoshi kɯ-ftsɤn ci nɯ-atɯɣ-i}\hspace{5pt}\pcmn{我们遇到了一位严格的老师}\end{exemple}\relationsémantique{同义词}{\lien{ⓔrkɤl}{rkɤl}}\étymologie{btsan}\end{entrée}

\begin{entrée}{ftsɤnbu}{}{ⓔftsɤnbu} 
\classe{n} 
\begin{définition}\pfra{force}\end{définition}
\begin{définition}\pcmn{强制性的办法}\end{définition}
\begin{exemple}\pjya{kɤ-nɤɲɟɯɲɟu mɯ-mɤ-ɲɯ-khɯ nɤ, ftsɤnbu tú-wɣ-βzu ɲɯ-ɬoʁ}\hspace{5pt}\pcmn{软的不行,就要来硬的}\end{exemple}\étymologie{btsan.po}\end{entrée}

\begin{entrée}{ftshi}{}{ⓔftshi} 
\classe{vs} \paradigme{dir}{tɤ-}\sens{1}
\begin{définition}\pfra{bon à rien, qui ne vaut rien}\end{définition}
\begin{définition}\pcmn{不可靠,没有良心(人)}\end{définition}
\begin{exemple}\pjya{jiɕqha nɯ ʁo ɲɯ-ftshi ɕti}\hspace{5pt}\pcmn{那个倒不怎么好}\end{exemple}
\begin{exemple}\pjya{laχtɕha ɲɯ-ftshi ɕti}\hspace{5pt}\pcmn{那个东西质量不好}\end{exemple}
\begin{exemple}\pjya{tɯrme ɲɯ-ftshi ɕti}\hspace{5pt}\pcmn{那个人不可靠}\end{exemple}\sens{2}
\begin{définition}\pfra{aller mieux}\end{définition}
\begin{définition}\pcmn{变好;减轻}\end{définition}
\begin{exemple}\pjya{smɤnba ɯ-thɯrʑi kɯ, ɯ-kɯ-mŋɤm to-ftshi}\hspace{5pt}\pcmn{多亏医生,他的病减轻了}\end{exemple}\relationsémantique{同义词}{\lien{}{tʂaʁ}}\relationsémantique{同义词}{\lien{ⓔmna}{mna}}
\begin{sous-entrée}{sɯftshi}{ⓔftshiⓢ2ⓝsɯftshi} 
\classe{vt}  
\grammaire{只用于否定式} 
\begin{définition}\pfra{forcer}\end{définition}
\begin{définition}\pcmn{逼迫}\end{définition}\end{sous-entrée}

\paradigme{dir}{tɤ-}
\begin{définition}\pfra{faire aller mieux}\end{définition}
\begin{définition}\pcmn{令……减轻}\end{définition}
\begin{exemple}\pjya{ɯʑo kɯ mɤ́-wɣ-sɯftshi-a tɕe, ta-tɯt nɯ kɤ-fse ɬoʁ}\hspace{5pt}\pcmn{他会逼迫我的,我只好照他说地去办}\end{exemple}
\begin{exemple}\pjya{smɤn nɯ kɯ a-kɯ-mŋɤm to-ɣɤftshi pjɤ-cha}\hspace{5pt}\pcmn{服了药我的病减轻了许多}\end{exemple}\relationsémantique{参考}{\lien{ⓔmɤkɯftshi}{mɤkɯftshi}}
\begin{sous-entrée}{ɣɤftshi}{ⓔftshiⓝɣɤftshi} 
\classe{vt} \end{sous-entrée}

\end{entrée}

\begin{entrée}{ftsoʁ}{}{ⓔftsoʁ} 
\classe{n} 
\begin{définition}\pfra{femelle d'hybride de yak et de vache}\end{définition}
\begin{définition}\pcmn{母犏牛}\end{définition}\end{entrée}

\begin{entrée}{ftsoʁdo}{}{ⓔftsoʁdo} 
\classe{n} 
\begin{définition}\pfra{vieille femelle de yak hybride}\end{définition}
\begin{définition}\pcmn{老母犏牛}\end{définition}\end{entrée}

\begin{entrée}{ftsɯɣ}{}{ⓔftsɯɣ} 
\classe{vt} \sens{1}\paradigme{dir}{kɤ-}
\begin{définition}\pfra{établir (une organisation)}\end{définition}
\begin{définition}\pcmn{成立;建立(组织)}\end{définition}
\begin{exemple}\pjya{jiʑo kɤ-ftsɯɣ-i}\hspace{5pt}\pcmn{我们建立了(某种组织)}\end{exemple}
\begin{exemple}\pjya{ʑara kɯ ka-ftsɯɣ-nɯ}\hspace{5pt}\pcmn{他们建立了(某种组织)}\end{exemple}
\begin{exemple}\pjya{tɯtɯrca kɤ-ftsɯɣ-i}\hspace{5pt}\pcmn{我们一起建立了(某种组织)}\end{exemple}
\begin{exemple}\pjya{@nonghui kɤ-ftsɯɣ-i}\hspace{5pt}\pcmn{我们建立了农会}\end{exemple}
\begin{exemple}\pjya{@renmin @gongshe kɤ-ftsɯɣ-i}\hspace{5pt}\pcmn{我们建立了人民公社}\end{exemple}\sens{2}\paradigme{dir}{tɤ-}
\begin{définition}\pfra{empiler des pierres pour faire une marque}\end{définition}
\begin{définition}\pcmn{把石头立起来做标记}\end{définition}
\begin{exemple}\pjya{zgoku mɤɕtʂa tɤ-ari-a ri, a-χti kɤ-mto maŋe tɕe, thu χsɯm ta-ftsɯɣ-a tɕe pɯ-nɯɣi-a}\hspace{5pt}\pcmn{我走到山顶去接丈夫,但见不到他回来,我立了三块石头做标记就回来了}\end{exemple}\étymologie{btsug}\end{entrée}

\begin{entrée}{ftsɯr}{}{ⓔftsɯr} 
\classe{vt} \paradigme{dir}{pɯ-}\sens{1}
\begin{définition}\pfra{essorer}\end{définition}
\begin{définition}\pcmn{拧干}\end{définition}
\begin{exemple}\pjya{pɯ-ftsɯr-a}\hspace{5pt}\pcmn{我拧干了}\end{exemple}
\begin{exemple}\pjya{ɯʑo kɯ pa-ftsɯr}\hspace{5pt}\pcmn{他拧干了}\end{exemple}
\begin{exemple}\pjya{kɯki ɲɯ-ɤci tɕe, pɯ-ftsɯr}\hspace{5pt}\pcmn{这个湿了,你把它拧干吧}\end{exemple}
\begin{exemple}\pjya{tɯ-ŋga ki ɲɤ-k-ɤci-ci, tɕe pɯ-ftsɯr}\hspace{5pt}\pcmn{这件衣服湿了,你把它拧干吧}\end{exemple}\sens{2}
\begin{définition}\pfra{vider de son eau}\end{définition}
\begin{définition}\pcmn{倒干;让水流干}\end{définition}\étymologie{btsir}\end{entrée}

\begin{entrée}{ftʂi}{}{ⓔftʂi} 
\classe{vt} \paradigme{dir}{nɯ-}\paradigme{dir}{thɯ-}
\begin{définition}\pfra{faire fondre}\end{définition}
\begin{définition}\pcmn{使融化}\end{définition}
\begin{exemple}\pjya{ta-mar kɤ-tʂi ɲɯ-ra}\hspace{5pt}\pcmn{要把酥油融化掉}\end{exemple}
\begin{exemple}\pjya{ta-mar nɯ-ftʂi-t-a}\hspace{5pt}\pcmn{我把酥油融化了}\end{exemple}
\begin{exemple}\pjya{tɤjpɣom na-ftʂi}\hspace{5pt}\pcmn{他把冰融化了}\end{exemple}\relationsémantique{同义词}{\lien{ⓔndʐiⓝsɯɣndʐi}{sɯɣndʐi}}\relationsémantique{参考}{\lien{ⓔndʐi}{ndʐi}}\end{entrée}

\begin{entrée}{ftɯɣ}{}{ⓔftɯɣ} 
\classe{vi} \paradigme{dir}{pɯ-}
\begin{définition}\pfra{être accompli}\end{définition}
\begin{définition}\pcmn{完工}\end{définition}
\begin{exemple}\pjya{kɤ-nɤma mɤ-kɯ-ftɯɣ nɯ kɤ-ɕɯftɯɣ ra}\hspace{5pt}\pcmn{要完成没有完成的工作}\end{exemple}\relationsémantique{参考}{\lien{ⓔɕɯftɯɣ}{ɕɯftɯɣ}}\end{entrée}

\begin{entrée}{ftɯl}{₁}{ⓔftɯlⓗ1} 
\classe{vt} \paradigme{dir}{pɯ-}
\begin{définition}\pfra{apprivoiser}\end{définition}
\begin{définition}\pcmn{驯服}\end{définition}
\begin{exemple}\pjya{jla pɯ-ftɯl-a}\hspace{5pt}\pcmn{我驯服了犏牛}\end{exemple}
\begin{exemple}\pjya{ɯʑo kɯ mbro pa-ftɯl}\hspace{5pt}\pcmn{他驯服了马}\end{exemple}\relationsémantique{同义词}{\lien{ⓔftɕɤt}{ftɕɤt}}
\begin{sous-entrée}{nɯɣɯftɯl}{ⓔftɯlⓗ1ⓝnɯɣɯftɯl} 
\classe{vi}  
\grammaire{facil} 
\begin{définition}\pfra{facile à apprivoiser}\end{définition}
\begin{définition}\pcmn{容易驯服}\end{définition}\relationsémantique{参考}{\lien{ⓔndɯlⓗ1}{ndɯl₁}}\end{sous-entrée}

\étymologie{btul}\end{entrée}

\begin{entrée}{ftɯl}{₂}{ⓔftɯlⓗ2} 
\classe{vi} \paradigme{dir}{nɯ-}
\begin{définition}\pfra{digérer}\end{définition}
\begin{définition}\pcmn{消化}\end{définition}
\begin{exemple}\pjya{kɤ-ndza kɤ-ftɯl mɯ́j-cha}\hspace{5pt}\pcmn{不能消化食物}\end{exemple}
\begin{exemple}\pjya{kɯki kɤ-ndza ki a-mɤ-nɯ-tɕhom ra ma kɤ-ftɯl mɯ́j-sɤcha}\hspace{5pt}\pcmn{这些食物不要吃太多,不然就没有办法消化}\end{exemple}\end{entrée}

\newpage\caractère{g}

\begin{entrée}{gaŋgaŋ}{}{ⓔgaŋgaŋ} 
\classe{idph.2} 
\begin{définition}\pfra{haut et imposant}\end{définition}
\begin{définition}\pcmn{形容高而大的样子}\end{définition}
\begin{exemple}\pjya{ɯ-phoŋbu wxti gaŋgaŋ ʑo pa}\hspace{5pt}\pcmn{他的身体又高又大}\end{exemple}\end{entrée}

\begin{entrée}{gɤgɤɣ}{}{ⓔgɤgɤɣ} 
\classe{idph.2} 
\begin{définition}\pfra{courbé}\end{définition}
\begin{définition}\pcmn{形容不灵活,站得不稳的样子}\end{définition}
\begin{exemple}\pjya{tɯ-ŋga ɕ-tɤ-ɕkho-t-a ri, ɲɯ-ɤqajpɣom tɕe gɤgɤɣ ʑo ɲɤ-pa}\hspace{5pt}\pcmn{我去晒了衣服,冻到了就变硬了}\end{exemple}
\begin{exemple}\pjya{ki tɤ-wɯ ki chɤ-rgɤz tɕe, gɤgɤɣ ʑo ɲɯ-pa}\hspace{5pt}\pcmn{这位老年人变老了,站得不稳}\end{exemple}
\begin{exemple}\pjya{ɯʑo lo-βzi tɕe, gɤgɤɣ ʑo ɲɯ-ndzur}\hspace{5pt}\pcmn{他喝醉了,站不稳}\end{exemple}
\begin{sous-entrée}{gɤɣnɤgɤɣ}{ⓔgɤgɤɣⓝgɤɣnɤgɤɣ} 
\classe{idph.3} 
\begin{exemple}\pjya{gɤgɤɣ nɤ gɤgɤɣ kɤ-anɯri}\hspace{5pt}\pcmn{他很不灵活地去了}\end{exemple}\end{sous-entrée}

\begin{sous-entrée}{gɤɣɯɣi}{ⓔgɤgɤɣⓝgɤɣɯɣi} 
\classe{idph.6} 
\begin{exemple}\pjya{rgɤtpu gɤɣɯɣi ʑo kɤ-ŋke ɲɯ-cha}\hspace{5pt}\pcmn{老年人只能走一点点}\end{exemple}\end{sous-entrée}

\end{entrée}

\begin{entrée}{glɤɣglɤɣ}{}{ⓔglɤɣglɤɣ} 
\classe{idph.2} 
\begin{définition}\pfra{pressé très fort}\end{définition}
\begin{définition}\pcmn{压得很紧}\end{définition}
\begin{exemple}\pjya{glɤɣglɤɣ ʑo ɲɯ-nɯ-rŋgɯ}\hspace{5pt}\pcmn{他睡得熟,动也不动}\end{exemple}
\begin{exemple}\pjya{glɤɣglɤɣ ʑo pjɤ-sɯ-ɲcɤr}\hspace{5pt}\pcmn{压得很紧}\end{exemple}
\begin{exemple}\pjya{mbrɤz tɯfkur tɤ-fkur-a ɯ-tɯ-rʑi glɤɣglɤɣ ʑo ɲɯ-pa}\hspace{5pt}\pcmn{我背了一袋大米,很重地压着我}\end{exemple}\relationsémantique{参考}{\lien{ⓔɣɤglɤglɤɣ}{ɣɤglɤglɤɣ}}\end{entrée}

\begin{entrée}{glinɤgli}{}{ⓔglinɤgli} 
\classe{idph.3} 
\begin{définition}\pfra{craquement d'os}\end{définition}
\begin{définition}\pcmn{形容(骨头)互相摩擦发出的声音}\end{définition}
\begin{exemple}\pjya{a-mke ɯ-ɕɤrɯ ɲɯ-ɤndʑɯgli ɲɯ-ŋu tɕe glinɤgli ʑo ɲɯ-ti}\hspace{5pt}\pcmn{我脖子的骨头互相摩擦发出咯咯声}\end{exemple}\relationsémantique{参考}{\lien{ⓔadʑɯgli}{adʑɯgli}}\end{entrée}

\begin{entrée}{goʁ}{}{ⓔgoʁ} 
\classe{idph.1} 
\begin{définition}\pfra{tout d'un coup (s'agenouiller)}\end{définition}
\begin{définition}\pcmn{一下子(跪下)}\end{définition}
\begin{exemple}\pjya{ɯ-χpɯm goʁ ʑo pjɤ-tshoʁ}\hspace{5pt}\pcmn{他一下子跪下了(很恭敬的样子)}\end{exemple}\relationsémantique{同义词}{\lien{ⓔdzoʁ}{dzoʁ}}\relationsémantique{同义词}{\lien{ⓔzgoʁ}{zgoʁ}}\end{entrée}

\begin{entrée}{graŋgraŋ}{}{ⓔgraŋgraŋ} 
\classe{idph.2} 
\begin{définition}\pfra{grand et mince}\end{définition}
\begin{définition}\pcmn{形容又高又瘦的样子(人)}\end{définition}
\begin{exemple}\pjya{aʑo a-slamaχti ci tɤ-tɕɯ tu tɕe, kɯ-mbro ci graŋgraŋ ʑo ŋu}\hspace{5pt}\pcmn{我有个男同学,个子又高又大}\end{exemple}\end{entrée}

\begin{entrée}{grɯβgrɯβ}{}{ⓔgrɯβgrɯβ} 
\classe{n} 
\begin{définition}\pfra{matsutake}\end{définition}
\begin{définition}\pcmn{松茸}\end{définition}
\begin{exemple}\pjya{grɯβgrɯβ nɯ ɕkrɤz wuma ʑo kɯ-wxti kɯ-ʁjɤr ɯ-ŋgɯ tu-ɬoʁ ŋu. grɯβgrɯβ nɯ ɯ-mgɯr ɯ-qhu nɯ kɯ-qandʐi tsa kɯ-wɣrum tsa ŋu. ɯ-rʑɯɣ pɕoʁ cho ɯ-ru nɯ kɯ-wɣrum ŋu, ɯ-dɯχɯn wuma ʑo tu; kɤ-ndza mɯm wuma ʑo aɣɯsmɤn, wuma ʑo ɯ-koŋ kɯ-wxti tɤjmɤɣ ɲɯ-ŋu}\hspace{5pt}\pcmn{松茸长在茂密的大青冈树林中,它背面带有乌色和白色,菌褶部分和干都是白色的,香味很浓,很好吃,可以入药,是很贵重的菌种。}\end{exemple}\end{entrée}

\begin{entrée}{grɯβgrɯβftsa}{}{ⓔgrɯβgrɯβftsa} 
\classe{n} 
\begin{définition}\pfra{une espèce de champignon}\end{définition}
\begin{définition}\pcmn{一种菌子}\end{définition}
\begin{exemple}\pjya{grɯβgrɯβftsa nɯ ɯ-tshɯɣa cho ɯ-mdoʁ ɯ-sɤɣɬoʁ nɯ ra grɯβgrɯβ cho wuma ʑo naχtɕɯɣ, tɕeri grɯβgrɯβ kɯ-fse ɯ-dɯχɯn me, sɤndɤɣ kɤ-ndza mɤ-sna, kɯ-thɯ nɯ ra pjɯ-si ɲɯ-ŋgrɤl}\hspace{5pt}\pcmn{\lien{ⓔgrɯβgrɯβftsa}{grɯβgrɯβftsa}的形状、颜色和生长的地方都和松茸一样,但是没有松茸那样的香气。有毒,不能吃。中毒严重的话会导致死亡。}\end{exemple}\end{entrée}

\begin{entrée}{grɯɣgrɯɣ}{}{ⓔgrɯɣgrɯɣ} 
\classe{idph.2} 
\begin{définition}\pfra{immobile, très serré}\end{définition}
\begin{définition}\pcmn{动也不动,绷得很紧}\end{définition}
\begin{exemple}\pjya{tɯmbri grɯɣgrɯɣ ʑo ɲɯ-ɤsɯ-ndo}\hspace{5pt}\pcmn{他使劲地抓住绳子不放}\end{exemple}
\begin{exemple}\pjya{jla grɯɣgrɯɣ ɯ-ɕna ɲɯ-ɤsɯ-ndo tɕe mɯ́j-ɕlɯɣ}\hspace{5pt}\pcmn{他使劲地抓住犏牛的鼻绳不放}\end{exemple}
\begin{exemple}\pjya{grɯɣgrɯɣ ʑo ko-βraʁ}\hspace{5pt}\pcmn{拴得很牢}\end{exemple}
\begin{sous-entrée}{grɯɣnɤgrɯɣ}{ⓔgrɯɣgrɯɣⓝgrɯɣnɤgrɯɣ} 
\classe{idph.3} \sens{1}
\begin{définition}\pfra{en se pressant}\end{définition}
\begin{définition}\pcmn{时间抓得很紧}\end{définition}
\begin{exemple}\pjya{ta-ma grɯɣnɤgʁɯɣ ʑo tu-nɤme ɲɯ-ŋu}\hspace{5pt}\pcmn{他把工作时间抓得很紧}\end{exemple}\end{sous-entrée}

\sens{2}
\begin{définition}\pfra{se gratter sans arrêt}\end{définition}
\begin{définition}\pcmn{一次又一次地抠}\end{définition}
\begin{exemple}\pjya{a-mgɯr ʑo grɯɣnɤgrɯɣ nɯ-rɤβraʁ-a}\hspace{5pt}\pcmn{我一次又一次地抠背}\end{exemple}\sens{3}
\begin{définition}\pfra{bruit émit par la meule}\end{définition}
\begin{définition}\pcmn{形容推磨的声音}\end{définition}
\begin{exemple}\pjya{mbrɯtɕɯ grɯɣnɤgrɯɣ ʑo chɤ-fse}\hspace{5pt}\pcmn{他使劲地磨刀}\end{exemple}\end{entrée}

\begin{entrée}{grɯŋgrɯŋ}{}{ⓔgrɯŋgrɯŋ} 
\classe{idph.2} \sens{1}
\begin{définition}\pfra{propre, utilisé jusqu'au bout}\end{définition}
\begin{définition}\pcmn{干净;用完}\end{définition}
\begin{exemple}\pjya{grɯŋgrɯŋ tu-orɕo}\hspace{5pt}\pcmn{完全用完了}\end{exemple}
\begin{exemple}\pjya{ji-tsha thɯ-arɕo grɯŋgrɯŋ ʑo}\hspace{5pt}\pcmn{我们的盐用完了}\end{exemple}\sens{2}
\begin{définition}\pfra{ne pas perdre de temps et bien faire son travail}\end{définition}
\begin{définition}\pcmn{时间安排得很紧凑;工作进行得很塌实;把线拉得很紧}\end{définition}
\begin{exemple}\pjya{phɯntshoʁ rcanɯ grɯɣgrɯɣ kɯ-pa ku-rɤʑi ɕti, tɤ-rʑaʁ ra kɯmɤlɤxso maka mɤ-sɯxɕe, kɤ-nɤma ra tshɯntshɯn tu-ste ɕti}\hspace{5pt}\pcmn{朋措把时间抓得很紧,从不浪费,工作也做得很好}\end{exemple}\relationsémantique{参考}{\lien{ⓔkhrɯŋkhrɯŋ}{khrɯŋkhrɯŋ}}\relationsémantique{参考}{\lien{ⓔχɤlχɤl}{χɤlχɤl}}\end{entrée}

\begin{entrée}{gɯgɯɣ}{}{ⓔgɯgɯɣ} 
\classe{idph.2} 
\begin{définition}\pfra{ciel très noir}\end{définition}
\begin{définition}\pcmn{形容天色很黑的样子}\end{définition}\end{entrée}

\begin{entrée}{gɯɣnɤgɯɣ}{}{ⓔgɯɣnɤgɯɣ} 
\classe{idph.3} 
\begin{définition}\pfra{émettant un bruit grave rythmique}\end{définition}
\begin{définition}\pcmn{发出有节奏的钝音}\end{définition}
\begin{exemple}\pjya{gɯɣnɤgɯɣ pa-xtsɯ}\hspace{5pt}\pcmn{一阵又一阵地砸了}\end{exemple}\end{entrée}

\newpage\caractère{ɣ}

\begin{entrée}{ɣa}{}{ⓔɣa} 
\classe{adv} 
\begin{définition}\pfra{oui}\end{définition}
\begin{définition}\pcmn{是的}\end{définition}\end{entrée}

\begin{entrée}{ɣar}{}{ⓔɣar} 
\classe{vi} \paradigme{dir}{nɯ-}\sens{1}
\begin{définition}\pfra{devenir sauvage}\end{définition}
\begin{définition}\pcmn{变野}\end{définition}
\begin{exemple}\pjya{ɯʑo ɲo-ɣar}\hspace{5pt}\pcmn{它(例如猫)变野了}\end{exemple}
\begin{exemple}\pjya{ɯʑo nɯ-ɣar}\hspace{5pt}\pcmn{他变野了}\end{exemple}
\begin{exemple}\pjya{lɯlu nɯ kha ri pjɯ-χsu-j pɯ-ɕti ri, jɤ-anɯri tɕe ɲo-ɣar}\hspace{5pt}\pcmn{我们以前家里养猫,但是它走了,变野了}\end{exemple}\sens{2}
\begin{définition}\pfra{devenir anormal, marginal}\end{définition}
\begin{définition}\pcmn{变得不正常}\end{définition}
\begin{exemple}\pjya{nɯ tɯrme nɯ kɯɕɯŋgɯ kɯ-pɯ-pe pjɤ-ɕti ri, nɯ ɯ-qhu tɕe ɲo-ɣar}\hspace{5pt}\pcmn{那个人原来好好的,后来就变得不正常。}\end{exemple}\sens{3}
\begin{définition}\pfra{aller vivre tout seul dans un endroit sauvage}\end{définition}
\begin{définition}\pcmn{到野地去生活}\end{définition}\end{entrée}

\begin{entrée}{ɣɤbɤbɤβ}{}{ⓔɣɤbɤbɤβ}\relationsémantique{参考}{\lien{ⓔbɤbɤβ}{bɤbɤβ}}\end{entrée}

\begin{entrée}{ɣɤbɤβlɤβ}{}{ⓔɣɤbɤβlɤβ} 
\classe{vi} 
\begin{définition}\pfra{parler de façon grossière et incompréhensible}\end{définition}
\begin{définition}\pcmn{胡言乱语;语气粗大,说别人听不懂的话}\end{définition}\end{entrée}

\begin{entrée}{ɣɤβdi}{}{ⓔɣɤβdi} 
\classe{vt}
\classe{vt}  
\grammaire{caus} \paradigme{dir}{}\sens{2}\paradigme{dir}{tɤ-}
\begin{définition}\pfra{réparer, bien organiser}\end{définition}
\begin{définition}\pcmn{修理;弄好}\end{définition}
\begin{exemple}\pjya{ɯʑo kɯ ta-ɣɤβdi}\hspace{5pt}\pcmn{他修了}\end{exemple}
\begin{exemple}\pjya{kɯki laχtɕha ki mɯ-to-pe tɕe, tu-ɣɤβdi-a ɲɯ-ntshi}\hspace{5pt}\pcmn{这个东西坏了,我要把它修理一下}\end{exemple}
\begin{exemple}\pjya{mkhɯrlu to-ɣɤβdi tɕe to-ndʐɯm}\hspace{5pt}\pcmn{机器修了以后,转得更快}\end{exemple}
\begin{exemple}\pjya{nɤ-sɯm nɯ-nɯ-ɣɤβdi}\hspace{5pt}\pcmn{你放心吧/你死了这条心吧}\end{exemple}
\begin{exemple}\pjya{kɤ-rɤt mɯ-pjɤ-tɯ-ɣɤβdi-t}\hspace{5pt}\pcmn{你写得不好}\end{exemple}
\begin{exemple}\pjya{a-kha kɤ-ɣɤβdi tɤ-jɤɣ}\hspace{5pt}\pcmn{我的房子修完了}\end{exemple}\sens{2}\paradigme{dir}{nɯ-}\paradigme{dir}{tɤ-}
\begin{définition}\pfra{castrer (verrat)}\end{définition}
\begin{définition}\pcmn{阉割(公猪)}\end{définition}
\begin{définition}\pfra{mettre en ordre}\end{définition}
\begin{définition}\pcmn{整理}\end{définition}
\begin{exemple}\pjya{ta-ɣɤβdoʁβdi}\hspace{5pt}\pcmn{他整理了}\end{exemple}
\begin{exemple}\pjya{kɤ-ɣɤβdoʁβdi ra}\hspace{5pt}\pcmn{要整理}\end{exemple}
\begin{exemple}\pjya{nɤ-laχtɕha ra tɤ-rɤwum tɕe ɯ-rtsɯɣ ra tɤ-ɣɤβdoʁβdi}\hspace{5pt}\pcmn{把你的东西收拾一下,把它们堆整齐}\end{exemple}\relationsémantique{同义词}{\lien{ⓔftɕɤz}{ftɕɤz}}
\begin{sous-entrée}{aβdoʁdi}{ⓔɣɤβdiⓢ2ⓝaβdoʁdi} 
\classe{vs} 
\begin{définition}\pfra{en bonne santé, paisible}\end{définition}
\begin{définition}\pcmn{安宁(身体状况、生活条件)}\end{définition}
\begin{exemple}\pjya{jiʑora ɕɤfɕo ku-oβdoʁβdi-j}\hspace{5pt}\pcmn{我们最近(这几年)身体没有异常现象}\end{exemple}\end{sous-entrée}

\begin{sous-entrée}{ɣɤβdoʁdi}{ⓔɣɤβdiⓢ2ⓝɣɤβdoʁdi} 
\classe{vt} \end{sous-entrée}

\begin{sous-entrée}{ʑɣɤɣɤβdi}{ⓔɣɤβdiⓢ2ⓝʑɣɤɣɤβdi} 
\classe{vi}  
\grammaire{refl}
\grammaire{caus} 
\begin{définition}\pfra{se soulager (en se massant, par exemple)}\end{définition}
\begin{définition}\pcmn{令自己好受一些(例如,揉自己的身体)}\end{définition}
\begin{exemple}\pjya{tɯʑo kɤ-ʑɣɤɣɤβdi mɤ-sɤcha}\hspace{5pt}\pcmn{不能令自己好受一些(因为太痛)}\end{exemple}\end{sous-entrée}

\end{entrée}

\begin{entrée}{ɣɤβdoʁβdi}{}{ⓔɣɤβdoʁβdi}\relationsémantique{参考}{\lien{ⓔɣɤβdi}{ɣɤβdi}}\end{entrée}

\begin{entrée}{ɣɤβlo}{}{ⓔɣɤβlo} 
\classe{vs} \paradigme{dir}{nɯ-}\paradigme{dir}{nɯ-}
\begin{définition}\pfra{lent}\end{définition}
\begin{définition}\pcmn{慢}\end{définition}
\begin{définition}\pfra{ralentir}\end{définition}
\begin{définition}\pcmn{减慢}\end{définition}
\begin{exemple}\pjya{kɤ-rɤma ɲɯ-tɯ-ɣɤβlo}\hspace{5pt}\pcmn{你工作得很慢!}\end{exemple}
\begin{exemple}\pjya{kɤ-nɤma wuma ʑo ɲɯ-mbɣom ri, aʑo mɯ́j-cha-a tɕe nɯ-zɣɤβlo-t-a}\hspace{5pt}\pcmn{工作很急,但是我不会做,所以耽误了时间}\end{exemple}\relationsémantique{反义词}{\lien{ⓔɣɤji}{ɣɤji}}
\begin{sous-entrée}{zɣɤβlo}{ⓔɣɤβloⓝzɣɤβlo} 
\classe{vt}  
\grammaire{caus} \end{sous-entrée}

\end{entrée}

\begin{entrée}{ɣɤβloʁβle/\variante{ɣɤβlɯβle}}{}{ⓔɣɤβloʁβle} 
\classe{vs} \paradigme{dir}{thɯ-}
\begin{définition}\pfra{maladroit}\end{définition}
\begin{définition}\pcmn{动作慢}\end{définition}\end{entrée}

\begin{entrée}{ɣɤβlɯβlɯɣ}{}{ⓔɣɤβlɯβlɯɣ} 
\classe{vi}  
\grammaire{deidph} \paradigme{dir}{tɤ-}
\begin{définition}\pfra{irisé}\end{définition}
\begin{définition}\pcmn{发光,显得耀眼}\end{définition}
\begin{exemple}\pjya{tɤtʂu ɲɯ-ɣɤβlɯβlɯɣ}\hspace{5pt}\pcmn{灯在发光}\end{exemple}
\begin{exemple}\pjya{ɲɯ-nɤmbju ɲɯ-ɣɤβlɯβlɯɣ}\hspace{5pt}\pcmn{在发光}\end{exemple}
\begin{exemple}\pjya{ɯ-ŋga ɯ-tɕhɤz ɯ-tɯ-dɤn kɯ ɲɯ-ɣɤβlɯβlɯɣ ʑo}\hspace{5pt}\pcmn{他衣服的彩色布料很多,显得很耀眼}\end{exemple}
\begin{sous-entrée}{sɤβlɯβlɯɣ}{ⓔɣɤβlɯβlɯɣⓝsɤβlɯβlɯɣ} 
\classe{vt} 
\begin{exemple}\pjya{ɯ-ŋga ɯ-tɕhɤz ɲɯ-sɤβlɯβlɯɣ ʑo}\hspace{5pt}\pcmn{他衣服的彩色布料穿着显得很耀眼}\end{exemple}\relationsémantique{参考}{\lien{ⓔβlɯɣnɤβlɯɣ}{βlɯɣnɤβlɯɣ}}\end{sous-entrée}

\end{entrée}

\begin{entrée}{ɣɤβzaʁlaʁ}{}{ⓔɣɤβzaʁlaʁ} 
\classe{vi} 
\begin{définition}\pfra{parler / se comporter de façon frivole}\end{définition}
\begin{définition}\pcmn{轻佻;说话、动作不严谨}\end{définition}\relationsémantique{参考}{\lien{ⓔβzaʁlu}{βzaʁlu}}\end{entrée}

\begin{entrée}{ɣɤβzi}{}{ⓔɣɤβzi}\relationsémantique{参考}{\lien{ⓔβzi}{βzi}}\end{entrée}

\begin{entrée}{ɣɤcaʁcaʁ}{}{ⓔɣɤcaʁcaʁ} 
\classe{vi} \paradigme{dir}{tɤ-}\paradigme{dir}{thɯ-}
\begin{définition}\pfra{bavard, racontant n'importe quoi}\end{définition}
\begin{définition}\pcmn{多嘴;乱说}\end{définition}
\begin{exemple}\pjya{pɯ-ɣɤcaʁcaʁ-a}\hspace{5pt}\pcmn{我以前多嘴}\end{exemple}
\begin{exemple}\pjya{pɯ-tɯ-ɣɤcaʁcaʁ}\hspace{5pt}\pcmn{你以前多嘴}\end{exemple}
\begin{exemple}\pjya{ɯʑo pɯ-ɣɤcaʁcaʁ}\hspace{5pt}\pcmn{他以前多嘴}\end{exemple}
\begin{exemple}\pjya{jiɕqha nɯ ɯ-mtɕhi dɤn, tu-ɣɤcaʁcaʁ ntsɯ ŋu}\hspace{5pt}\pcmn{刚才那个(人)爱多嘴乱说}\end{exemple}\end{entrée}

\begin{entrée}{ɣɤcɤtcɤt}{}{ⓔɣɤcɤtcɤt} 
\classe{vi} \paradigme{dir}{tɤ-}\sens{1}
\begin{définition}\pfra{piailler (oiseau)}\end{définition}
\begin{définition}\pcmn{叫(鸟)}\end{définition}\sens{2}
\begin{définition}\pfra{prendre la parole sans arrêt (enfant)}\end{définition}
\begin{définition}\pcmn{不停地插嘴(小孩子)}\end{définition}
\begin{exemple}\pjya{ma-tɯ-ɣɤcɤtcɤt}\hspace{5pt}\pcmn{你不要不停地插嘴}\end{exemple}
\begin{exemple}\pjya{ɣɤcɤtcat-a}\hspace{5pt}\pcmn{我不停地插嘴}\end{exemple}\end{entrée}

\begin{entrée}{ɣɤchi}{}{ⓔɣɤchi}\relationsémantique{参考}{\lien{ⓔchi}{chi}}\end{entrée}

\begin{entrée}{ɣɤchrɤβchrɤβ}{}{ⓔɣɤchrɤβchrɤβ} 
\classe{vi} \sens{1}
\begin{définition}\pfra{avoir la gorge enrouée}\end{définition}
\begin{définition}\pcmn{嗓子哑了}\end{définition}
\begin{exemple}\pjya{ɲɯ-nɯtɕhomba tɕe, ɯ-rqo ɲɯ-ɣɤchrɤβchrɤβ}\hspace{5pt}\pcmn{他感冒了,嗓子哑了}\end{exemple}\sens{2}
\begin{définition}\pfra{émettre du bruit en roulant (petites pierres)}\end{définition}
\begin{définition}\pcmn{小石头滚下来发出声音}\end{définition}
\begin{exemple}\pjya{praʁ ɲɯ-mbɯt ɲɯ-ɣɤchrɤβchrɤβ}\hspace{5pt}\pcmn{悬崖塌下来了,小石头滚下来发出声音}\end{exemple}
\begin{sous-entrée}{ɣɤchrɤβlɤβ}{ⓔɣɤchrɤβchrɤβⓢ2ⓝɣɤchrɤβlɤβ}
\begin{définition}\pfra{émettre un bruit de mucus dans la gorge}\end{définition}
\begin{définition}\pcmn{(喉咙里)有痰的声音}\end{définition}
\begin{exemple}\pjya{ɲɯ-tɯ-ɤɕqhe tɕe nɤ-rqo ɲɯ-ɣɤchrɤβlɤβ}\hspace{5pt}\pcmn{你咳嗽,嗓子发出嘶哑声}\end{exemple}\end{sous-entrée}

\begin{sous-entrée}{sɤschrɤβlɤβ}{ⓔɣɤchrɤβchrɤβⓢ2ⓝsɤschrɤβlɤβ} 
\classe{vt} \sens{1}
\begin{définition}\pfra{émettre un bruit de mucus dans la gorge}\end{définition}
\begin{définition}\pcmn{放出喉咙里有痰的声音}\end{définition}
\begin{exemple}\pjya{nɤ-rqo ma-tɯ-sɤchrɤβlɤβ}\hspace{5pt}\pcmn{你嗓子不要发出嘶哑声}\end{exemple}\end{sous-entrée}

\sens{2}
\begin{définition}\pfra{émettre du bruit en faisant s'entrechoquer des petits objets}\end{définition}
\begin{définition}\pcmn{令小东西互相碰撞,发出声音}\end{définition}
\begin{exemple}\pjya{laχtɕha ɲɯ-sɤschrɤβlɤβ}\hspace{5pt}\pcmn{他把很多小东西装在一起,拿起来的时候发出互相碰撞的声音}\end{exemple}\relationsémantique{参考}{\lien{ⓔchrɤβchrɤβ}{chrɤβchrɤβ}}\end{entrée}

\begin{entrée}{ɣɤchrɤβlɤβ}{}{ⓔɣɤchrɤβlɤβ}\relationsémantique{参考}{\lien{ⓔɣɤchrɤβchrɤβ}{ɣɤchrɤβchrɤβ}}\end{entrée}

\begin{entrée}{ɣɤchɯchrɯɣ/\variante{ɣɤchrɯɣchrɯɣ}}{}{ⓔɣɤchɯchrɯɣ} 
\classe{vs} \paradigme{dir}{tɤ-}
\begin{définition}\pfra{bruit d'objets durs qui s'entrechoquent}\end{définition}
\begin{définition}\pcmn{很多硬的东西(铁链子的环子,骨头,小石头)互相摩擦和撞击发出的声音}\end{définition}
\begin{exemple}\pjya{ʑɴɢɯloʁ tɤ-fkur-a ɲɯ-ɣɤchrɯchrɯɣ}\hspace{5pt}\pcmn{(口袋里装了核桃),我背起时,(核桃)相碰撞发出的声音}\end{exemple}\end{entrée}

\begin{entrée}{ɣɤcraŋlaŋ}{}{ⓔɣɤcraŋlaŋ} 
\classe{vi} \paradigme{dir}{tɤ-}
\begin{définition}\pfra{crier très fort}\end{définition}
\begin{définition}\pcmn{高声喧哗;大声地叫}\end{définition}
\begin{définition}\pfra{faire crier très fort}\end{définition}
\begin{définition}\pcmn{发出很响的的声音;让……大声地叫}\end{définition}
\begin{exemple}\pjya{khɯna ɲɯ-ɣɤcraŋlaŋ}\hspace{5pt}\pcmn{狗在大声地叫}\end{exemple}
\begin{exemple}\pjya{paʁ ɲɯ-ɣɤcraŋlaŋ}\hspace{5pt}\pcmn{猪在大声地叫}\end{exemple}
\begin{exemple}\pjya{ɯ-tɯ-ɣɤcraŋlaŋ kɯ kɤ-rɤβzjoz koŋla mɯ́j-khɯ}\hspace{5pt}\pcmn{他很吵,根本无法念书}\end{exemple}
\begin{exemple}\pjya{paʁ ɲɯ-ɤsɯ-ntɕha tɕe, ta-sɤcraŋlaŋ}\hspace{5pt}\pcmn{他在宰猪,猪大声地叫}\end{exemple}
\begin{exemple}\pjya{χɕɤl ɲɯ-sɤcraŋlaŋ pa-qrɯ}\hspace{5pt}\pcmn{他把玻璃砸碎了}\end{exemple}
\begin{sous-entrée}{sɤcraŋlaŋ}{ⓔɣɤcraŋlaŋⓝsɤcraŋlaŋ} 
\classe{vt} \end{sous-entrée}

\end{entrée}

\begin{entrée}{ɣɤcrɯɣlɯɣ}{}{ⓔɣɤcrɯɣlɯɣ}\relationsémantique{参考}{\lien{ⓔcrɯɣcrɯɣ}{crɯɣcrɯɣ}}\end{entrée}

\begin{entrée}{ɣɤcɯqhlɯβ}{}{ⓔɣɤcɯqhlɯβ} 
\classe{vs} \paradigme{dir}{nɯ-}\paradigme{dir}{nɯ-}
\begin{définition}\pfra{faire du bruit en s'agitant (eau)}\end{définition}
\begin{définition}\pcmn{发出水晃动声}\end{définition}
\begin{définition}\pfra{agiter l'eau bruyament}\end{définition}
\begin{définition}\pcmn{把水摇晃发出声音}\end{définition}
\begin{exemple}\pjya{tɯ-ci nɯ zɯm ɯ-ŋgɯ ɲɯ-ɣɤcɯqhlɯβ}\hspace{5pt}\pcmn{水在桶里摇晃发出声音}\end{exemple}
\begin{exemple}\pjya{tɯ-ŋga nɯ́-wɣ-χtɕi tɕe khro ɲɯ́-wɣ-sɤcɯqhlɯβ tɕe ɲɯ-ɕo cha}\hspace{5pt}\pcmn{洗衣服的时候要多晃几下就洗得干净}\end{exemple}
\begin{sous-entrée}{sɤcɯqhlɯβ}{ⓔɣɤcɯqhlɯβⓝsɤcɯqhlɯβ}\end{sous-entrée}

\end{entrée}

\begin{entrée}{ɣɤɕu}{}{ⓔɣɤɕu} 
\classe{vi} \sens{1}\paradigme{dir}{thɯ-}
\begin{définition}\pfra{frais}\end{définition}
\begin{définition}\pcmn{凉快}\end{définition}
\begin{exemple}\pjya{jiɕqha pɯ-ɣɯtshɤdɯɣ ri, zdɯm jo-ɣi tɕe ko-ɣɤɕu}\hspace{5pt}\pcmn{刚才很闷热,来了云就凉快一些}\end{exemple}\sens{2}\paradigme{dir}{kɤ-}
\begin{définition}\pfra{être couvert (soleil)}\end{définition}
\begin{définition}\pcmn{天阴}\end{définition}
\begin{exemple}\pjya{tɯ-mɯ chɤ-lɯβ tɕe ɲɯ-ɣɤɕu}\hspace{5pt}\pcmn{天阴了,现在很凉快}\end{exemple}
\begin{exemple}\pjya{jisŋi wuma ʑo ɲɯ-sɤɕke tɕe, kɤ-ɣɤɕu kóʁmɯz nɤ kɤ-nɤma ɲɯ-ɬoʁ}\hspace{5pt}\pcmn{今天天气很热,太阳阴了才能劳动}\end{exemple}\relationsémantique{参考}{\lien{ⓔnɤɕu}{nɤɕu}}\relationsémantique{参考}{\lien{ⓔtɤɕu}{tɤɕu}}\end{entrée}

\begin{entrée}{ɣɤɕaʁɕaʁ}{}{ⓔɣɤɕaʁɕaʁ} 
\classe{vs} 
\begin{définition}\pfra{très amer}\end{définition}
\begin{définition}\pcmn{很苦}\end{définition}
\begin{exemple}\pjya{tʂha ɲɯ-qiaβ ɲɯ-ɣɤɕaʁɕaʁ}\hspace{5pt}\pcmn{茶非常苦}\end{exemple}
\begin{exemple}\pjya{smɤn ɲɯ-qiaβ ɲɯ-ɣɤɕaʁɕaʁ}\hspace{5pt}\pcmn{药非常苦}\end{exemple}\end{entrée}

\begin{entrée}{ɣɤɕe}{}{ⓔɣɤɕe} 
\classe{vi} 
\begin{définition}\pfra{qui va vite (temps)}\end{définition}
\begin{définition}\pcmn{过得快(时间);走得早}\end{définition}
\begin{exemple}\pjya{ki a-tɤ-fse tɕe, nɤ-tɤ-rʑaʁ ɯ-ɲɯ-ɣɤɕe}\hspace{5pt}\pcmn{这样的话,你的时间过得快吗?}\end{exemple}\relationsémantique{参考}{\lien{ⓔɕe}{ɕe}}
\begin{sous-entrée}{nɤɣɤɕe}{ⓔɣɤɕeⓝnɤɣɤɕe} 
\classe{vt} 
\begin{définition}\pfra{trouver que (le temps) va vite}\end{définition}
\begin{définition}\pcmn{觉得时间过得很快}\end{définition}
\begin{exemple}\pjya{tɤ-rʑaʁ mɯ́j-nɤɣɤɕe-a}\hspace{5pt}\pcmn{我觉得时间过得很慢}\end{exemple}\end{sous-entrée}

\end{entrée}

\begin{entrée}{ɣɤɕkɤɣɕkɤɣ}{}{ⓔɣɤɕkɤɣɕkɤɣ} 
\classe{vi}  
\grammaire{deidph} \paradigme{dir}{tɤ-}\paradigme{dir}{tɤ-}
\begin{définition}\pfra{objet dur faisant du bruit lorsqu'on le frappe, glacé au point d'être dur}\end{définition}
\begin{définition}\pcmn{硬东西一敲就发出声音;冻得很硬的样子}\end{définition}
\begin{définition}\pfra{faire du bruit en frappant un objet dur}\end{définition}
\begin{définition}\pcmn{敲打硬的东西}\end{définition}
\begin{exemple}\pjya{ɲɯ-rko ɲɯ-ɣɤɕkɤɣɕkɤɣ ʑo}\hspace{5pt}\pcmn{非常硬,摸起来像石头一样}\end{exemple}
\begin{exemple}\pjya{nɤ-stu tɤ-fse ma ta-sɤɕkɤɣɕkɤɣ}\hspace{5pt}\pcmn{你小心,不然我会整你的!}\end{exemple}
\begin{exemple}\pjya{rdɤstaʁ (si) tɤ-sɤɕkɤɣɕkɤɣ-a}\hspace{5pt}\pcmn{我敲打了石头(木料)}\end{exemple}
\begin{sous-entrée}{sɤɕkɤɣɕkɤɣ}{ⓔɣɤɕkɤɣɕkɤɣⓝsɤɕkɤɣɕkɤɣ} 
\classe{vt} \end{sous-entrée}

\begin{sous-entrée}{ɣɤɕkɤɣlɤɣ}{ⓔɣɤɕkɤɣɕkɤɣⓝɣɤɕkɤɣlɤɣ} 
\classe{vs} 
\begin{définition}\pfra{maigre et vieux, mais qui s'active sans arrêt}\end{définition}
\begin{définition}\pcmn{又瘦又老,还是不停地活动}\end{définition}
\begin{exemple}\pjya{nɯŋa ɲɯ-ɣɤɕkɤɣlɤɣ}\hspace{5pt}\pcmn{奶牛很不听话地乱蹦乱跳}\end{exemple}
\begin{exemple}\pjya{jiɕqha rgɤtpu nɯ ɲɯ-ɣɤɕkɤɣlɤɣ}\hspace{5pt}\pcmn{那个(又瘦又高的)老年人走动}\end{exemple}\relationsémantique{参考}{\lien{ⓔɕkɤɣnɤɕkɤɣ}{ɕkɤɣnɤɕkɤɣ}}\end{sous-entrée}

\end{entrée}

\begin{entrée}{ɣɤɕkɤɣlɤɣ}{}{ⓔɣɤɕkɤɣlɤɣ}\relationsémantique{参考}{\lien{ⓔɣɤɕkɤɣɕkɤɣ}{ɣɤɕkɤɣɕkɤɣ}}\end{entrée}

\begin{entrée}{ɣɤɕnɯɣlɯɣ}{}{ⓔɣɤɕnɯɣlɯɣ}\relationsémantique{参考}{\lien{ⓔɕnɯɣnɤlɯɣ}{ɕnɯɣnɤlɯɣ}}\end{entrée}

\begin{entrée}{ɣɤɕŋaʁɕŋaʁ}{}{ⓔɣɤɕŋaʁɕŋaʁ}\relationsémantique{参考}{\lien{ⓔɕŋaʁɕŋaʁⓗ2}{ɕŋaʁɕŋaʁ₂}}\end{entrée}

\begin{entrée}{ɣɤɕo}{}{ⓔɣɤɕo}\relationsémantique{参考}{\lien{ⓔɕo}{ɕo}}\end{entrée}

\begin{entrée}{ɣɤɕpɤɕpɤr}{}{ⓔɣɤɕpɤɕpɤr} 
\classe{vi} \paradigme{dir}{tɤ-}
\begin{définition}\pfra{émettre un bruit fort}\end{définition}
\begin{définition}\pcmn{乱叫}\end{définition}
\begin{définition}\pfra{exprimer son opinion à haute voix sans se soucier de rien}\end{définition}
\begin{définition}\pcmn{大声说话,不注意场合}\end{définition}
\begin{exemple}\pjya{@laba ɲɯ-ɣɤɕpɤɕpɤr}\hspace{5pt}\pcmn{喇叭在乱叫}\end{exemple}
\begin{exemple}\pjya{ɲɯ-ɣɤɕpɤrlɤr}\hspace{5pt}\pcmn{他在乱吼乱叫}\end{exemple}
\begin{exemple}\pjya{ɲɯ-ɣɤɕpɤrlɤr ɲɯ-rɯɕmi}\hspace{5pt}\pcmn{他在大声说话}\end{exemple}
\begin{exemple}\pjya{nɤki tɕheme nɯ kɯ-ɣɤɕpɤrlɤr ci ŋu}\hspace{5pt}\pcmn{那个女的很多嘴}\end{exemple}
\begin{exemple}\pjya{nɤ-mtɕhi kɤ-ndɤm ma-tɯ-ɣɤɕpɤrlɤr}\hspace{5pt}\pcmn{你闭嘴,不要大声说话}\end{exemple}\relationsémantique{参考}{\lien{ⓔɕpɤrnɤlɤr}{ɕpɤrnɤlɤr}}
\begin{sous-entrée}{ɣɤɕpɤrlɤr}{ⓔɣɤɕpɤɕpɤrⓝɣɤɕpɤrlɤr} 
\classe{vi} \end{sous-entrée}

\end{entrée}

\begin{entrée}{ɣɤɕpɤrlɤr}{}{ⓔɣɤɕpɤrlɤr}\relationsémantique{参考}{\lien{ⓔɣɤɕpɤɕpɤr}{ɣɤɕpɤɕpɤr}}\end{entrée}

\begin{entrée}{ɣɤɕphɤβlɤβ}{}{ⓔɣɤɕphɤβlɤβ} 
\classe{vi} 
\begin{définition}\pfra{faire du bruit (battement d'aile)}\end{définition}
\begin{définition}\pcmn{(鸟)拍翅膀发出声音;抖衣服发出声音}\end{définition}
\begin{exemple}\pjya{pɣa ɲɯ-ɣɤɕphɤβlɤβ}\hspace{5pt}\pcmn{鸟拍翅膀发出声音}\end{exemple}
\begin{sous-entrée}{sɤɕphɤβlɤβ}{ⓔɣɤɕphɤβlɤβⓝsɤɕphɤβlɤβ} 
\classe{vt} 
\begin{exemple}\pjya{tɯ-ŋga ɲɯ-sɤɕphɤβlɤβ}\hspace{5pt}\pcmn{他在抖衣服,发出声音}\end{exemple}\end{sous-entrée}

\end{entrée}

\begin{entrée}{ɣɤɕphɤr}{}{ⓔɣɤɕphɤr} 
\classe{vi} \paradigme{dir}{nɯ-}
\begin{définition}\pfra{concilier}\end{définition}
\begin{définition}\pcmn{劝解}\end{définition}
\begin{exemple}\pjya{ɯʑo nɯ-ɣɤɕphɤr}\hspace{5pt}\pcmn{他劝解了}\end{exemple}
\begin{exemple}\pjya{ku-ɣɤɕphar-a}\hspace{5pt}\pcmn{我劝解}\end{exemple}
\begin{exemple}\pjya{ku-tɯ-ɣɤɕphɤr}\hspace{5pt}\pcmn{你劝解}\end{exemple}
\begin{exemple}\pjya{ɯʑo ku-ɣɤɕphɤr}\hspace{5pt}\pcmn{他劝解}\end{exemple}
\begin{exemple}\pjya{nɯ-pɤrthɤβ nɯ-ɣɤɕphar-a}\hspace{5pt}\pcmn{我在他们之间劝解了}\end{exemple}
\begin{exemple}\pjya{ʑɤni ɲɯ-ɤlɯlɤt-ndʑi tɕe, nɯ-ɣɤɕphar-a (=nɯ-nɯkhɤda-t-a-ndʑi)}\hspace{5pt}\pcmn{他们吵架的时候,我把他们劝开了}\end{exemple}\relationsémantique{同义词}{\lien{ⓔnɯkhɤda}{nɯkhɤda}}\end{entrée}

\begin{entrée}{ɣɤɕqali}{}{ⓔɣɤɕqali} 
\classe{vi} \paradigme{dir}{tɤ-}\paradigme{dir}{tɤ-}
\begin{définition}\pfra{crier}\end{définition}
\begin{définition}\pcmn{嚷;喊}\end{définition}
\begin{exemple}\pjya{aʑo tɤ-ɣɤɕqali-a}\hspace{5pt}\pcmn{我喊了一下}\end{exemple}
\begin{exemple}\pjya{nɤʑo tɤ-tɯ-ɣɤɕqali}\hspace{5pt}\pcmn{你喊了一下}\end{exemple}
\begin{exemple}\pjya{ɯʑo tɤ-ɣɤɕqali}\hspace{5pt}\pcmn{他喊了一下}\end{exemple}
\begin{exemple}\pjya{pɯ-nɤscar-a tɕe, tɤ-ɣɤɕqali-a pɯ-ra}\hspace{5pt}\pcmn{我吓了一跳,忍不住喊了一声}\end{exemple}
\begin{exemple}\pjya{a-@shouji ɲɯ-ɣɤɕqali}\hspace{5pt}\pcmn{我的手机在响}\end{exemple}
\begin{exemple}\pjya{ɯʑo kɯ ɯ-skɤt ta-sɤɕqali ʑo (=ta-nɤxɕɤt)}\hspace{5pt}\pcmn{他大声地喊了一下}\end{exemple}\relationsémantique{参考}{\lien{ⓔtɤɕqali}{tɤɕqali}}
\begin{sous-entrée}{sɤɕqali}{ⓔɣɤɕqaliⓝsɤɕqali} 
\classe{vt} \end{sous-entrée}

\begin{sous-entrée}{sɯsɤɕqali}{ⓔɣɤɕqaliⓝsɯsɤɕqali} 
\classe{vt} 
\begin{définition}\pfra{faire crier}\end{définition}
\begin{définition}\pcmn{让……喊}\end{définition}
\begin{exemple}\pjya{ɯʑo kɯ tɤ-pɤtso ta-ʁndɯ tɕe ta-sɯsɤɕqali ʑo}\hspace{5pt}\pcmn{他打了小孩子,让他大叫了一声}\end{exemple}\end{sous-entrée}

\end{entrée}

\begin{entrée}{ɣɤɕtʂaŋlaŋ}{}{ⓔɣɤɕtʂaŋlaŋ}\relationsémantique{参考}{\lien{ⓔɕtʂaŋɕtʂaŋ}{ɕtʂaŋɕtʂaŋ}}\end{entrée}

\begin{entrée}{ɣɤɕɯβɕɯβ}{}{ⓔɣɤɕɯβɕɯβ} 
\classe{vi} \paradigme{dir}{tɤ-}
\begin{définition}\pfra{murmurer}\end{définition}
\begin{définition}\pcmn{悄声说话,偷偷说话}\end{définition}
\begin{exemple}\pjya{tɤ-ɣɤɕɯβɕɯβ-a}\hspace{5pt}\pcmn{我悄声说话了}\end{exemple}
\begin{exemple}\pjya{tɤ-tɯ-ɣɤɕɯβɕɯβ}\hspace{5pt}\pcmn{你悄声说话了}\end{exemple}
\begin{exemple}\pjya{tɤ-ɣɤɕɯβɕɯβ}\hspace{5pt}\pcmn{他悄声说话了}\end{exemple}
\begin{exemple}\pjya{ɯʑo ku-ɣɤɕɯβɕɯβ}\hspace{5pt}\pcmn{他正在悄声说话}\end{exemple}
\begin{exemple}\pjya{tɯrme ra kɯ pjɯ-mtshɤm-nɯ mɯ-tɤ-pe tɕe, tu-kɯ-ɣɤɕɯβɕɯβ tɕe phɤn}\hspace{5pt}\pcmn{如果不想被别人听见,悄声说话就可以了}\end{exemple}\end{entrée}

\begin{entrée}{ɣɤɕɯftaʁ}{}{ⓔɣɤɕɯftaʁ}\relationsémantique{参考}{\lien{ⓔɕɯftaʁ}{ɕɯftaʁ}}\end{entrée}

\begin{entrée}{ɣɤɕɯŋɕɯŋ}{}{ⓔɣɤɕɯŋɕɯŋ}\relationsémantique{参考}{\lien{ⓔɕɯŋɕɯŋ}{ɕɯŋɕɯŋ}}\end{entrée}

\begin{entrée}{ɣɤdɤn}{}{ⓔɣɤdɤn} 
\classe{vt} \paradigme{dir}{tɤ-}\paradigme{dir}{nɯ-}
\begin{définition}\pfra{augmenter}\end{définition}
\begin{définition}\pcmn{增多;增加}\end{définition}
\begin{exemple}\pjya{tɤ-ɣɤdan-a}\hspace{5pt}\pcmn{我加了}\end{exemple}
\begin{exemple}\pjya{tɤ-tɯ-ɣɤdɤn}\hspace{5pt}\pcmn{你加了}\end{exemple}
\begin{exemple}\pjya{ɯʑo kɯ ta-ɣɤdɤn}\hspace{5pt}\pcmn{他加了}\end{exemple}
\begin{exemple}\pjya{tɯ-rju nɯ kɤ-ɣɤdɤn mɤ-ra}\hspace{5pt}\pcmn{一句都不要加}\end{exemple}
\begin{exemple}\pjya{pɕawtsɯ nɯ ɯ-tsa nɯ ta-rku ma, nɯ ma mɯ-ta-ɣɤdɤn}\hspace{5pt}\pcmn{他装了必要的钱,没有装更多}\end{exemple}
\begin{exemple}\pjya{mɯ-na-ɣɤdɤn}\hspace{5pt}\pcmn{话没有多说}\end{exemple}\end{entrée}

\begin{entrée}{ɣɤdi}{}{ⓔɣɤdi} 
\classe{vs} \paradigme{dir}{nɯ-}\paradigme{dir}{nɯ-}
\begin{définition}\pfra{mauvaise (odeur), puer}\end{définition}
\begin{définition}\pcmn{臭;变味}\end{définition}
\begin{définition}\pfra{laisser puer}\end{définition}
\begin{définition}\pcmn{令东西有臭味}\end{définition}
\begin{exemple}\pjya{tɤ-mthɯm ɲɯ-ɣɤdi}\hspace{5pt}\pcmn{肉变味了}\end{exemple}
\begin{exemple}\pjya{ɯβrɤ-ɲɯ-ɣɤdi}\hspace{5pt}\pcmn{有没有变味?}\end{exemple}
\begin{exemple}\pjya{tɤ-mthɯm ɲɤ-zɣɤdi-t-a}\hspace{5pt}\pcmn{我(不小心)把肉(忘在那里了),令它有臭味}\end{exemple}\relationsémantique{参考}{\lien{ⓔtɤ-di}{tɤ-di}}
\begin{sous-entrée}{zɣɤdi}{ⓔɣɤdiⓝzɣɤdi} 
\classe{vt}  
\grammaire{caus} \end{sous-entrée}

\end{entrée}

\begin{entrée}{ɣɤdoŋdoŋ}{}{ⓔɣɤdoŋdoŋ} 
\classe{vi} \paradigme{dir}{pɯ-}
\begin{définition}\pfra{couler bruyamment (eau)}\end{définition}
\begin{définition}\pcmn{水管或某种物体中流出来的水又急又多,发出响声}\end{définition}
\begin{exemple}\pjya{tɯ-ci ɲɯ-ɣɤdoŋdoŋ ʑo}\hspace{5pt}\pcmn{水流出来很响}\end{exemple}
\begin{sous-entrée}{sɤdoŋdoŋ}{ⓔɣɤdoŋdoŋⓝsɤdoŋdoŋ}
\begin{définition}\pfra{jeter, verser beaucoup d'eau}\end{définition}
\begin{définition}\pcmn{泼很多水,倒很多水(发出很多声音)}\end{définition}
\begin{exemple}\pjya{qhajŋgɯ ri tɯ-ci tɤ-sɤdoŋdoŋ-a pɯ-lat-a}\hspace{5pt}\pcmn{我把水倒进引水槽了(声音大,水多)}\end{exemple}\end{sous-entrée}

\end{entrée}

\begin{entrée}{ɣɤdzaŋdzaŋ}{}{ⓔɣɤdzaŋdzaŋ}\relationsémantique{参考}{\lien{ⓔdzaŋdzaŋ}{dzaŋdzaŋ}}\end{entrée}

\begin{entrée}{ɣɤdzɯlɯt/\variante{\_ɣɤdzɯlɯz}}{}{ⓔɣɤdzɯlɯt} 
\classe{vi} \paradigme{dir}{nɯ-}\paradigme{dir}{tɤ-}
\begin{définition}\pfra{s'agiter}\end{définition}
\begin{définition}\pcmn{动来动去;摇动;蠕动}\end{définition}
\begin{exemple}\pjya{ku-ɣɤdzɯlɯz-a}\end{exemple}
\begin{exemple}\pjya{tu-ɣɤdzɯlɯt-a}\hspace{5pt}\pcmn{我正在动来动去}\end{exemple}
\begin{exemple}\pjya{ɲɯ-tɯ-ɣɤdzɯlɯz}\hspace{5pt}\pcmn{你动来动去}\end{exemple}
\begin{exemple}\pjya{kɤ-nɤma tɤra tɕe, ku-ɣɤdzɯlɯt ra}\hspace{5pt}\pcmn{要工作了,要活跃起来}\end{exemple}
\begin{exemple}\pjya{tɤ-pɤtso ɲɯ-ɣɤdzɯlɯt}\hspace{5pt}\pcmn{小孩子在动来动去}\end{exemple}
\begin{exemple}\pjya{nɤ-mi ɲɯ-ɣɤdzɯlɯt}\hspace{5pt}\pcmn{你的脚在动}\end{exemple}
\begin{exemple}\pjya{qajɯ ci ɲɯ-ɣɤdzɯlɯt}\hspace{5pt}\pcmn{有一条虫在动}\end{exemple}\relationsémantique{参考}{\lien{ⓔsɤdzɯlɯt}{sɤdzɯlɯt}}\end{entrée}

\begin{entrée}{ɣɤdʑɯdʑaŋ}{}{ⓔɣɤdʑɯdʑaŋ} 
\classe{vi} \paradigme{dir}{tɤ-}\paradigme{dir}{tɤ-}
\begin{définition}\pfra{glisser rapidement (d'un gros morceau de bois)}\end{définition}
\begin{définition}\pcmn{又粗又长的木料很快地滑下来的样子}\end{définition}
\begin{définition}\pfra{faire glisser rapidement (un gros morceau de bois)}\end{définition}
\begin{définition}\pcmn{令又粗又长的木料很快地滑下来}\end{définition}
\begin{exemple}\pjya{si ta-sɤdʑɯdʑaŋ ʑo pa-βde}\hspace{5pt}\pcmn{把长的树枝从高处任意地摔下来了}\end{exemple}
\begin{sous-entrée}{sɤdʑɯdʑaŋ}{ⓔɣɤdʑɯdʑaŋⓝsɤdʑɯdʑaŋ} 
\classe{vt} \end{sous-entrée}

\end{entrée}

\begin{entrée}{ɣɤfka}{}{ⓔɣɤfka}\relationsémantique{参考}{\lien{ⓔfkaⓗ1}{fka₁}}\end{entrée}

\begin{entrée}{ɣɤfsoʁ}{}{ⓔɣɤfsoʁ}\relationsémantique{参考}{\lien{ⓔfsoʁⓗ2}{fsoʁ₂}}\end{entrée}

\begin{entrée}{ɣɤftshi}{}{ⓔɣɤftshi}\relationsémantique{参考}{\lien{ⓔftshi}{ftshi}}\end{entrée}

\begin{entrée}{ɣɤglɤglɤɣ}{}{ⓔɣɤglɤglɤɣ} 
\classe{vi} 
\begin{définition}\pfra{bruyant}\end{définition}
\begin{définition}\pcmn{很响(一阵一阵敲打声)}\end{définition}
\begin{exemple}\pjya{@qiche ɲɯ-ɣɤglɤglɤɣ}\hspace{5pt}\pcmn{汽车很响}\end{exemple}\relationsémantique{参考}{\lien{ⓔsɤglɤglɤɣ}{sɤglɤglɤɣ}}\end{entrée}

\begin{entrée}{ɣɤgo}{}{ⓔɣɤgo} 
\classe{vs} 
\begin{définition}\pfra{naïf, honnête}\end{définition}
\begin{définition}\pcmn{笨;老实}\end{définition}
\begin{exemple}\pjya{jiɕqha tɯrme kɯ-ɣɤgo ci ɲɯ-ŋu}\hspace{5pt}\pcmn{他是老实人}\end{exemple}
\begin{exemple}\pjya{ɯʑo kɯ-ɣɤgo ci ɕti tɕe, kɯŋu nɤme ɕti}\hspace{5pt}\pcmn{他这个人很老实,相信他会把事情做好}\end{exemple}\end{entrée}

\begin{entrée}{ɣɤgɯgɯɣ}{}{ⓔɣɤgɯgɯɣ} 
\classe{vi} \paradigme{dir}{tɤ-}
\begin{définition}\pfra{faire du bruit en démarrant (moteur), faire du bruit en soufflant (vent)}\end{définition}
\begin{définition}\pcmn{发出声音(如汽车发动,吹大风等的声音)}\end{définition}
\begin{exemple}\pjya{mbɣɯrloʁ ɲɯ-ɣɤgɯgɯɣ}\hspace{5pt}\pcmn{雷很响}\end{exemple}\relationsémantique{参考}{\lien{ⓔgɯɣnɤgɯɣ}{gɯɣnɤgɯɣ}}\relationsémantique{同义词}{\lien{ⓔɣɤŋgɯrŋgɯr}{ɣɤŋgɯrŋgɯr}}\end{entrée}

\begin{entrée}{ɣɤɣɤjɣɤj}{}{ⓔɣɤɣɤjɣɤj} 
\classe{vi} 
\begin{définition}\pfra{avoir la tête qui tourne, ne pas pouvoir tenir sur ses jambe}\end{définition}
\begin{définition}\pcmn{不停的摇晃的感觉,脚都站不稳}\end{définition}
\begin{sous-entrée}{ɣɤɣɤjlɤj}{ⓔɣɤɣɤjɣɤjⓝɣɤɣɤjlɤj} 
\classe{vi} \end{sous-entrée}

\end{entrée}

\begin{entrée}{ɣɤɣɤjlɤj}{}{ⓔɣɤɣɤjlɤj}\relationsémantique{参考}{\lien{ⓔɣɤɣɤjɣɤj}{ɣɤɣɤjɣɤj}}\end{entrée}

\begin{entrée}{ɣɤɣɤmbrɯ}{}{ⓔɣɤɣɤmbrɯ} 
\classe{vs}  
\grammaire{facil} 
\begin{définition}\pfra{s'énerver facilement}\end{définition}
\begin{définition}\pcmn{容易生气}\end{définition}
\begin{exemple}\pjya{ɲɯ-ɣɤɣɤmbrɯ}\hspace{5pt}\pcmn{他容易生气}\end{exemple}
\begin{exemple}\pjya{ɲɯ-tɯ-ɣɤɣɤmbrɯ}\hspace{5pt}\pcmn{你容易生气}\end{exemple}\relationsémantique{参考}{\lien{ⓔsɤmbrɯ}{sɤmbrɯ}}\relationsémantique{参考}{\lien{ⓔsɤzmbrɯ}{sɤzmbrɯ}}\end{entrée}

\begin{entrée}{ɣɤɣɤtɕɯɣ}{}{ⓔɣɤɣɤtɕɯɣ}\relationsémantique{参考}{\lien{ⓔɣɤtɕɯɣ}{ɣɤtɕɯɣ}}\end{entrée}

\begin{entrée}{ɣɤɣɤwu}{}{ⓔɣɤɣɤwu}\relationsémantique{参考}{\lien{ⓔɣɤwu}{ɣɤwu}}\end{entrée}

\begin{entrée}{ɣɤɣi}{}{ⓔɣɤɣi}\relationsémantique{参考}{\lien{ⓔɣi}{ɣi}}\end{entrée}

\begin{entrée}{ɣɤɣɯrɣɯr}{}{ⓔɣɤɣɯrɣɯr} 
\classe{vi} \paradigme{dir}{tɤ-}\sens{1}
\begin{définition}\pfra{ardent (feu)}\end{définition}
\begin{définition}\pcmn{旺盛(火)}\end{définition}
\begin{exemple}\pjya{smi ɲɯ-ɣɤɣɯrɣɯr ɲɯ-nɯt}\hspace{5pt}\pcmn{火烧得很旺}\end{exemple}\sens{2}
\begin{définition}\pfra{animé, bruyant}\end{définition}
\begin{définition}\pcmn{嘈杂(声音);闹哄哄;熙熙攘攘}\end{définition}
\begin{exemple}\pjya{tɯrme ra ɲɯ-ɣɤɣɯrɣɯr-nɯ}\hspace{5pt}\pcmn{人们很吵}\end{exemple}
\begin{exemple}\pjya{ɲɯ-ɣɤɕqali-nɯ tɕe ɲɯ-ɣɤɣɯrɣɯr-nɯ}\hspace{5pt}\pcmn{他们在吼叫,很吵}\end{exemple}
\begin{sous-entrée}{sɤɣɯrɣɯr}{ⓔɣɤɣɯrɣɯrⓢ2ⓝsɤɣɯrɣɯr} 
\classe{vt} \end{sous-entrée}

\end{entrée}

\begin{entrée}{ɣɤjaŋri}{}{ⓔɣɤjaŋri} 
\classe{vi} \paradigme{dir}{pɯ-}
\begin{définition}\pfra{aller ça et là}\end{définition}
\begin{définition}\pcmn{来回走动}\end{définition}
\begin{exemple}\pjya{pɯ-ɣɤjaŋri-a ntsɯ ma tɤ-χtɯ-t-a me}\hspace{5pt}\pcmn{我在街上随便走动,没有买什么东西}\end{exemple}\end{entrée}

\begin{entrée}{ɣɤjaʁ}{}{ⓔɣɤjaʁ}\relationsémantique{参考}{\lien{ⓔjaʁ}{jaʁ}}\end{entrée}

\begin{entrée}{ɣɤjɤβjɤβ}{}{ⓔɣɤjɤβjɤβ} 
\classe{vi} \paradigme{dir}{nɯ-}
\begin{définition}\pfra{toucher à tout, s'amuser avec les petits objets}\end{définition}
\begin{définition}\pcmn{到处乱摸;东摸西摸(偷东西)}\end{définition}
\begin{exemple}\pjya{βɣɤza ɲɯ-ɣɤjɤβjɤβ}\hspace{5pt}\pcmn{苍蝇到处乱爬(令人发痒)}\end{exemple}
\begin{exemple}\pjya{ma-tɯ-ɣɤjɤβjɤβ, koŋla nɯ-sɤŋo}\hspace{5pt}\pcmn{别乱摸东西,认真听}\end{exemple}
\begin{exemple}\pjya{kɯ-mɯrkɯ ɯ-jaʁ ɣɤjɤβjɤβ}\hspace{5pt}\pcmn{小偷在乱摸}\end{exemple}\end{entrée}

\begin{entrée}{ɣɤjɤrjɤr}{}{ⓔɣɤjɤrjɤr}\relationsémantique{参考}{\lien{ⓔjɤrjɤr}{jɤrjɤr}}\end{entrée}

\begin{entrée}{ɣɤji}{}{ⓔɣɤji} 
\classe{vs} \paradigme{dir}{tɤ-}\paradigme{dir}{tɤ-}
\begin{définition}\pfra{rapide}\end{définition}
\begin{définition}\pcmn{快(动作)}\end{définition}
\begin{définition}\pfra{accélérer}\end{définition}
\begin{définition}\pcmn{加快}\end{définition}
\begin{exemple}\pjya{tɤ-ɣɤji tsa ɲɯ-ra}\hspace{5pt}\pcmn{要快点}\end{exemple}
\begin{exemple}\pjya{ɯʑo kɤ-nɤma ra ɲɯ-ɣɤji}\hspace{5pt}\pcmn{他劳动做得很快}\end{exemple}
\begin{exemple}\pjya{aʑo ɣɤji-a}\hspace{5pt}\pcmn{我很快}\end{exemple}
\begin{exemple}\pjya{kɤ-rɯndzɤtshi ɲɯ-ɣɤji}\hspace{5pt}\pcmn{他吃饭吃得很快}\end{exemple}
\begin{exemple}\pjya{ʑara ɲɯ-rɤma-nɯ tɕe, aʑo kɯ-qur jɤ-ari-a tɕe, tɤ-zɣɤji-t-a-nɯ}\hspace{5pt}\pcmn{他们在工作,我去帮忙,令他们工作得更快}\end{exemple}
\begin{sous-entrée}{zɣɤji}{ⓔɣɤjiⓝzɣɤji} 
\classe{vt}  
\grammaire{caus} \end{sous-entrée}

\end{entrée}

\begin{entrée}{ɣɤjiz}{}{ⓔɣɤjiz} 
\classe{adv} 
\begin{définition}\pfra{temporairement}\end{définition}
\begin{définition}\pcmn{暂时}\end{définition}
\begin{exemple}\pjya{tɤ-rʑaʁ kɯ-zri ɯ-sɯso kɤ-lɤt ra ma ɣɤjiz ɯ-βlɯβlu kɤ-lɤt mɤ-pe}\hspace{5pt}\pcmn{只顾短暂不顾长远是不好的}\end{exemple}\end{entrée}

\begin{entrée}{ɣɤjka}{}{ⓔɣɤjka} 
\classe{vi} 
\begin{définition}\pfra{bégayer}\end{définition}
\begin{définition}\pcmn{结巴}\end{définition}
\begin{exemple}\pjya{jiɕqha tɯrme ɲɯ-ɣɤjka}\hspace{5pt}\pcmn{这个人是结巴}\end{exemple}
\begin{exemple}\pjya{kɤ-rɯɕmi ɲɯ-ɣɤjka}\hspace{5pt}\pcmn{他说话结巴}\end{exemple}\end{entrée}

\begin{entrée}{ɣɤjlu}{}{ⓔɣɤjlu} 
\classe{vs}  
\grammaire{denom} 
\begin{définition}\pfra{cru}\end{définition}
\begin{définition}\pcmn{没有炒熟,吃的时候有生味}\end{définition}
\begin{exemple}\pjya{tɤ-jlu ɣɤjlu}\hspace{5pt}\pcmn{面粉没有炒熟}\end{exemple}
\begin{exemple}\pjya{mɯ́j-smi tɕe ɲɯ-ɣɤjlu}\hspace{5pt}\pcmn{没有炒熟,有生味}\end{exemple}
\begin{exemple}\pjya{stoʁ ɲɯ-ɣɤjlu}\hspace{5pt}\pcmn{胡豆没有炒熟}\end{exemple}\relationsémantique{参考}{\lien{ⓔtɤjlu}{tɤjlu}}\end{entrée}

\begin{entrée}{ɣɤjmŋo}{}{ⓔɣɤjmŋo} 
\classe{vt}  
\grammaire{denom} \paradigme{dir}{pɯ-}\paradigme{dir}{kɤ-}
\begin{définition}\pfra{rêver, rêver de}\end{définition}
\begin{définition}\pcmn{做梦}\end{définition}
\begin{exemple}\pjya{kɤ-ɣɤjmŋo-t-a, pɯ-ɣɤjmŋo-t-a}\hspace{5pt}\pcmn{我梦见他了}\end{exemple}
\begin{exemple}\pjya{pɯ-tɯ-ɣɤjmŋo-t}\hspace{5pt}\pcmn{你梦见他了}\end{exemple}
\begin{exemple}\pjya{pa-ɣɤjmŋo}\hspace{5pt}\pcmn{他梦见他了}\end{exemple}
\begin{exemple}\pjya{jɯfɕɯr pɯ-ta-ɣɤjmŋo}\hspace{5pt}\pcmn{我昨天梦见你了}\end{exemple}
\begin{exemple}\pjya{ɯʑo kɯ pjɤ́-wɣ-ɣɤmŋo-a}\hspace{5pt}\pcmn{他梦见我了}\end{exemple}
\begin{exemple}\pjya{kɯrɯ skɤt pjɯ-kɯ-sɯxɕat-a ɲɯ-ŋu pɯ-ɣɤjmŋo-t-a}\hspace{5pt}\pcmn{我做梦你在教我藏语}\end{exemple}\relationsémantique{参考}{\lien{ⓔtɯ-jmŋo}{tɯ-jmŋo}}\end{entrée}

\begin{entrée}{ɣɤjmɯt}{}{ⓔɣɤjmɯt}\relationsémantique{参考}{\lien{ⓔjmɯt}{jmɯt}}\end{entrée}

\begin{entrée}{ɣɤjom}{}{ⓔɣɤjom} 
\classe{vt} \paradigme{dir}{nɯ-}
\begin{définition}\pfra{élargir}\end{définition}
\begin{définition}\pcmn{扩大;修宽}\end{définition}
\begin{exemple}\pjya{tʂu jiʑora nɯ-ɣɤjom-i}\hspace{5pt}\pcmn{我们把路扩大了}\end{exemple}
\begin{exemple}\pjya{kɤntɕhaʁ ɣɯ tʂu nɯra ɲɯ-ɣɤjom-nɯ ɲɯ-ŋu tɕe, tɕe nɯtɕu ku-nɤma-nɯ tɕe, tɤ-zgra ɲɯ-wxti wo!}\hspace{5pt}\pcmn{他们在把路修宽一点,所以很吵}\end{exemple}\relationsémantique{参考}{\lien{ⓔjom}{jom}}\end{entrée}

\begin{entrée}{ɣɤjpum}{}{ⓔɣɤjpum}\relationsémantique{参考}{\lien{ⓔjpum}{jpum}}\end{entrée}

\begin{entrée}{ɣɤjqaʁ}{}{ⓔɣɤjqaʁ} 
\classe{vt} \paradigme{dir}{pɯ-}
\begin{définition}\pfra{se débarrasser}\end{définition}
\begin{définition}\pcmn{摆脱}\end{définition}
\begin{exemple}\pjya{pɯ-ɣɤjqaʁ-a}\hspace{5pt}\pcmn{我摆脱了}\end{exemple}
\begin{exemple}\pjya{ɯʑo kɯ pjɤ-ɣɤjqaʁ}\hspace{5pt}\pcmn{他摆脱了}\end{exemple}
\begin{exemple}\pjya{kɤ-ɣɤjqaʁ me}\hspace{5pt}\pcmn{无法摆脱}\end{exemple}
\begin{exemple}\pjya{kɯ-ɲɟo tɤ-apa tɕe kɤ-ɣɤjqaʁ me}\hspace{5pt}\pcmn{一旦灾难来了就无法摆脱}\end{exemple}\end{entrée}

\begin{entrée}{ɣɤjru}{}{ⓔɣɤjru} 
\classe{vs} \paradigme{dir}{tɤ-}
\begin{définition}\pfra{aux gestes rapides}\end{définition}
\begin{définition}\pcmn{动作勤快}\end{définition}
\begin{exemple}\pjya{ta-ma ɲɯ-ɣɤjru}\hspace{5pt}\pcmn{他劳动的时候动作勤快}\end{exemple}
\begin{exemple}\pjya{ta-ma ɲɯ-tɯ-ɣɤjru}\hspace{5pt}\pcmn{你劳动的时候动作勤快}\end{exemple}\relationsémantique{反义词}{\lien{ⓔɣɯlaj}{ɣɯlaj}}\end{entrée}

\begin{entrée}{ɣɤjtɯ}{}{ⓔɣɤjtɯ}\relationsémantique{参考}{\lien{ⓔajtɯ}{ajtɯ}}\end{entrée}

\begin{entrée}{ɣɤjɯ}{}{ⓔɣɤjɯ} 
\classe{vt} \paradigme{dir}{pɯ-}\paradigme{dir}{tɤ-}
\begin{définition}\pfra{ajouter}\end{définition}
\begin{définition}\pcmn{加,添加}\end{définition}
\begin{exemple}\pjya{pɯ-ɣɤjɯ-t-a}\hspace{5pt}\pcmn{我加了}\end{exemple}
\begin{exemple}\pjya{pɯ-tɯ-ɣɤjɯ-t}\hspace{5pt}\pcmn{你加了}\end{exemple}
\begin{exemple}\pjya{pa-ɣɤjɯ}\hspace{5pt}\pcmn{他加了}\end{exemple}
\begin{exemple}\pjya{aʑo tɤ-ɣɤjɯ-t-a}\hspace{5pt}\pcmn{我加了}\end{exemple}
\begin{exemple}\pjya{ɯʑo ɣɯ ɯ-rŋɯl ɣurʑa ɣɤʑu tɕe, kɯβdɤ-sqi tɤ-ɣɤjɯ-t-a tɕe, tɕe lonba ɣurʑa kɯβdɤ-sqi tɤ-tu}\hspace{5pt}\pcmn{他本来有一百元,我添了四十,他现在一共有了一百四十元}\end{exemple}
\begin{exemple}\pjya{qro kɯ tɤ-kɤ-fɕɤt mɤʑɯ pjɯ́-wɣ-ɣɤjɯ ɲɯ-khɯ}\hspace{5pt}\pcmn{(编故事的时候)蚂蚁这个人物所讲的话,可以多加几句}\end{exemple}
\begin{exemple}\pjya{a-@dian thɯ-ɣɤjɯ-t-a tɕe tɤ-amdzɯ-a, ku-ta-nɤjo}\hspace{5pt}\pcmn{我已经充了电,坐下来了,我在等你}\end{exemple}\end{entrée}

\begin{entrée}{ɣɤjwaʁ}{}{ⓔɣɤjwaʁ} 
\classe{vs} \paradigme{dir}{nɯ-}
\begin{définition}\pfra{pousser des feuilles}\end{définition}
\begin{définition}\pcmn{长出叶子}\end{définition}
\begin{exemple}\pjya{χɕitka jɤ-ɣe tɕe, sɯku ɲɯ-ɣɤjwaʁ ɲɯ-ŋu}\hspace{5pt}\pcmn{到了春天,树长出叶子}\end{exemple}\relationsémantique{参考}{\lien{ⓔtɤ-jwaʁ}{tɤ-jwaʁ}}\relationsémantique{参考}{\lien{ⓔrɤjwaʁ}{rɤjwaʁ}}\end{entrée}

\begin{entrée}{ɣɤjwɤrlɤr}{}{ⓔɣɤjwɤrlɤr} 
\classe{vi} 
\begin{définition}\pfra{être secoué}\end{définition}
\begin{définition}\pcmn{摇晃;摇摆不稳}\end{définition}
\begin{exemple}\pjya{ʑmbrɯ nɯ tɯ-ɣɤjwɤrlɤr to-ʑa}\hspace{5pt}\pcmn{船开始摇晃了}\end{exemple}
\begin{sous-entrée}{sɤjwɤrlɤr}{ⓔɣɤjwɤrlɤrⓝsɤjwɤrlɤr} 
\classe{vt} 
\begin{exemple}\pjya{khɯtsa ma-tɯ-sɤjwɤrlɤr ma tɯ-lwoʁ}\hspace{5pt}\pcmn{碗不要摇,会(把水)倒出来}\end{exemple}\end{sous-entrée}

\end{entrée}

\begin{entrée}{ɣɤɟaʁ}{}{ⓔɣɤɟaʁ} 
\classe{vt} \paradigme{dir}{kɤ-}
\begin{définition}\pfra{cajoler un enfant}\end{définition}
\begin{définition}\pcmn{哄}\end{définition}
\begin{exemple}\pjya{tɤ-pɤtso kɤ-ɣɤɟaʁ-a}\hspace{5pt}\pcmn{我哄了小孩子}\end{exemple}
\begin{exemple}\pjya{kɤ-tɯ-ɣɤɟaʁ}\hspace{5pt}\pcmn{你哄了他}\end{exemple}
\begin{exemple}\pjya{ka-ɣɤɟaʁ}\hspace{5pt}\pcmn{他哄了他}\end{exemple}
\begin{exemple}\pjya{tɤ-pɤtso kú-wɣ-ɣɤɟaʁ tɕe rga}\hspace{5pt}\pcmn{小孩子被哄就高兴}\end{exemple}\end{entrée}

\begin{entrée}{ɣɤɟɯɣɟɯɣ}{}{ⓔɣɤɟɯɣɟɯɣ} 
\classe{vi} \paradigme{dir}{\_}
\begin{définition}\pfra{trembler, grouiller}\end{définition}
\begin{définition}\pcmn{发抖;蠕动}\end{définition}
\begin{exemple}\pjya{ɯ-re ɲɯ-ɬoʁ tɕe ɲɯ-ɣɤɟɯɣɟɯɣ ʑo}\hspace{5pt}\pcmn{他想笑,一身都在发抖}\end{exemple}
\begin{exemple}\pjya{ɲɯ-ɣɤɟɯɣɟɯɣ ɲɯ-nɤre}\hspace{5pt}\pcmn{他笑着(不发出声音、全身发抖)}\end{exemple}
\begin{sous-entrée}{sɤɟɯɣɟɯɣ}{ⓔɣɤɟɯɣɟɯɣⓝsɤɟɯɣɟɯɣ} 
\classe{vt} 
\begin{définition}\pfra{se tourner dans tous les sens}\end{définition}
\begin{définition}\pcmn{扭动}\end{définition}
\begin{exemple}\pjya{jla kɯ ɯ-βri ɲɯ-sɤɟɯɣɟɯɣ}\hspace{5pt}\pcmn{犏牛扭动它的身体(驱赶苍蝇)}\end{exemple}
\begin{exemple}\pjya{ɯ-tʂɯmpari ɲɯ-sɤɟɯɣɟɯɣ}\end{exemple}
\begin{exemple}\pjya{jɤlwa ɲɯ-sɤɟɯɣɟɯɣ}\hspace{5pt}\pcmn{他在扭动门帘}\end{exemple}
\begin{exemple}\pjya{tɤɕi tɤ-rku-t-a tɕeri, mɯ́j-xtɕhɯt tɕe nɯ-sɤɟɯɣɟɯɣ-a tɕe mɤʑɯ tɤ-xtɕhɯt}\hspace{5pt}\pcmn{我把青稞装在口袋里,装不下,抖动了一下就装得下了}\end{exemple}\end{sous-entrée}

\end{entrée}

\begin{entrée}{ɣɤɟɯɣlɯɣ}{}{ⓔɣɤɟɯɣlɯɣ} 
\classe{vi} 
\begin{définition}\pfra{se tortiller}\end{définition}
\begin{définition}\pcmn{扭来扭去}\end{définition}\end{entrée}

\begin{entrée}{ɣɤɟɯɟrɯɣ/\variante{ɣɤɟrɯɣɟrɯɣ}}{}{ⓔɣɤɟɯɟrɯɣ} 
\classe{vs} \paradigme{dir}{tɤ-}\paradigme{dir}{tɤ-}
\begin{définition}\pfra{gargouiller (ventre)}\end{définition}
\begin{définition}\pcmn{肚子咕噜叫}\end{définition}
\begin{définition}\pfra{faire du bruit (en démolissant un mur)}\end{définition}
\begin{définition}\pcmn{发出声音(拆墙的时候)}\end{définition}
\begin{exemple}\pjya{znde ɲɯ-ɣɤɟɯɟrɯɣ pɯ-mbɯt}\hspace{5pt}\pcmn{墙慢慢地塌下来了}\end{exemple}
\begin{exemple}\pjya{a-xtu ɲɯ-ɣɤɟrɯɣɟrɯɣ}\hspace{5pt}\pcmn{我肚子咕噜咕噜叫}\end{exemple}
\begin{exemple}\pjya{znde tɤ-sɤɟɯɟrɯɣa pɯ-phɯt-a}\hspace{5pt}\pcmn{我把墙拆了(发出很响的声音)}\end{exemple}\relationsémantique{参考}{\lien{ⓔɟrɯɣɟrɯɣ}{ɟrɯɣɟrɯɣ}}
\begin{sous-entrée}{sɤɟɯɟrɯɣ}{ⓔɣɤɟɯɟrɯɣⓝsɤɟɯɟrɯɣ} 
\classe{vt} \end{sous-entrée}

\end{entrée}

\begin{entrée}{ɣɤkɤβjɤβ}{}{ⓔɣɤkɤβjɤβ} 
\classe{vi} \paradigme{dir}{nɯ-}\paradigme{dir}{tɤ-}
\begin{définition}\pfra{bouger dans tous les coins sans savoir quoi faire}\end{définition}
\begin{définition}\pcmn{急得到处乱动}\end{définition}
\begin{exemple}\pjya{ɲɯ-mbɣom tɕe ɲɯ-ɣɤkɤβjɤβ}\hspace{5pt}\pcmn{他很急,到处乱动}\end{exemple}
\begin{exemple}\pjya{tɤ-mbɣom-a ra tɤ-tɯt-a tɕe (tɤ-ɕɯmbɣom-a tɕe) tɤ-sɤkɤβjaβ-a ʑo}\hspace{5pt}\pcmn{我叫他快点,(令)他(急得)到处乱动}\end{exemple}\relationsémantique{参考}{\lien{ⓔɣɤqhɤβjɤβ}{ɣɤqhɤβjɤβ}}
\begin{sous-entrée}{sɤkɤβjɤβ}{ⓔɣɤkɤβjɤβⓝsɤkɤβjɤβ} 
\classe{vt} \end{sous-entrée}

\end{entrée}

\begin{entrée}{ɣɤkhe}{}{ⓔɣɤkhe}\relationsémantique{参考}{\lien{ⓔkhe}{khe}}\end{entrée}

\begin{entrée}{ɣɤkhrɤβjɤβ}{}{ⓔɣɤkhrɤβjɤβ} 
\classe{vi} 
\begin{définition}\pfra{émettre un bruit de grattement incessant}\end{définition}
\begin{définition}\pcmn{不停地抓东西发出的声音}\end{définition}
\begin{exemple}\pjya{ɲɯ-ɣɤkhrɤβjɤβ ntsɯ}\hspace{5pt}\pcmn{他不停地做事}\end{exemple}
\begin{exemple}\pjya{βʑɯ ɲɯ-ɣɤkhrɤβjɤβ tɕe kɤ-ʑɣɤsɯndo}\hspace{5pt}\pcmn{老鼠不停地抓东西发出声音,最后(被猫)抓到了}\end{exemple}
\begin{exemple}\pjya{kɤ-rɤʑi mɯ́j-cha tɕe pjɯ-ɣɤkhrɤβjɤβ ntsɯ ŋu}\hspace{5pt}\pcmn{他不能待在那里,不停地做事}\end{exemple}
\begin{sous-entrée}{sɤkhrɤβjɤβ}{ⓔɣɤkhrɤβjɤβⓝsɤkhrɤβjɤβ} 
\classe{vt} 
\begin{exemple}\pjya{laχtɕha ra ma-tɯ-sɤkhrɤβjɤβ}\hspace{5pt}\pcmn{你不要不停地弄那些东西,发出声音}\end{exemple}\end{sous-entrée}

\end{entrée}

\begin{entrée}{ɣɤkhrɤβkhrɤβ}{}{ⓔɣɤkhrɤβkhrɤβ} 
\classe{vi} 
\begin{définition}\pfra{émettre du bruit (en secouant un récipient qui contient de petits objets durs)}\end{définition}
\begin{définition}\pcmn{发出撞击的声音}\end{définition}\end{entrée}

\begin{entrée}{ɣɤkhrɯɣlɯɣ}{}{ⓔɣɤkhrɯɣlɯɣ}\relationsémantique{参考}{\lien{ⓔkhrɯɣnɤkhrɯɣ}{khrɯɣnɤkhrɯɣ}}\end{entrée}

\begin{entrée}{ɣɤkhɯ}{₁}{ⓔɣɤkhɯⓗ1} 
\classe{vi}  
\grammaire{denom} \paradigme{dir}{tɤ-}
\begin{définition}\pfra{être enfumé}\end{définition}
\begin{définition}\pcmn{有烟;冒烟}\end{définition}
\begin{exemple}\pjya{kha ɲɯ-ɣɤkhɯ}\hspace{5pt}\pcmn{满屋子都是烟}\end{exemple}
\begin{exemple}\pjya{smi chɯ́-wɣ-βlɯ tɕe ɣɤkhɯ}\hspace{5pt}\pcmn{烧火就会冒烟}\end{exemple}
\begin{exemple}\pjya{ɲɯ-tɯ-ɣɤkhɯ}\hspace{5pt}\pcmn{你家在冒烟(知道你在家里)}\end{exemple}\relationsémantique{参考}{\lien{ⓔnɤkhɯ}{nɤkhɯ}}\relationsémantique{参考}{\lien{ⓔtɤ-khɯ}{tɤ-khɯ}}\relationsémantique{参考}{\lien{ⓔsɤkhɯ}{sɤkhɯ}}\end{entrée}

\begin{entrée}{ɣɤkhɯ}{₂}{ⓔɣɤkhɯⓗ2} 
\classe{vt}  
\grammaire{caus} \sens{1}\paradigme{dir}{tɤ-}
\begin{définition}\pfra{forcer quelqu'un à faire quelque chose qu'il n'a pas envie de faire}\end{définition}
\begin{définition}\pcmn{让别人做 (他不愿意做的事情)}\end{définition}
\begin{exemple}\pjya{tɤ-ɣɤkhɯ-t-a}\hspace{5pt}\pcmn{我让他同意了}\end{exemple}
\begin{exemple}\pjya{tɤ-ndzɯmbra-t-a tɕe tɤ-ɣɤkhɯ-t-a}\hspace{5pt}\pcmn{经过我的劝说,他就同意了}\end{exemple}\sens{2}
\begin{définition}\pfra{rendre possible}\end{définition}
\begin{définition}\pcmn{使……可以……}\end{définition}
\begin{exemple}\pjya{kɯm tɤ-ɣɤβdi-t-a tɕe, kɤ-cɯ tɤ-ɣɤkhɯ-t-a}\hspace{5pt}\pcmn{我修了门,使它可以打开了}\end{exemple}\relationsémantique{参考}{\lien{ⓔkhɯⓗ1}{khɯ₁}}\end{entrée}

\begin{entrée}{ɣɤla}{}{ⓔɣɤla} 
\classe{vt}  
\grammaire{caus}
\grammaire{refl} \paradigme{dir}{pɯ-}\paradigme{dir}{pɯ-}
\begin{définition}\pfra{mouiller, tremper, plonger dans l'eau}\end{définition}
\begin{définition}\pcmn{泡软;浸泡在水里}\end{définition}
\begin{exemple}\pjya{pɯ-ɣɤla-t-a}\hspace{5pt}\pcmn{我把它泡软了}\end{exemple}
\begin{exemple}\pjya{pɯ-ɣɤle}\hspace{5pt}\pcmn{你把它泡一下吧!}\end{exemple}
\begin{exemple}\pjya{tɯ-ci ɯ-ŋgɯ zɯ tɯ-ndʐi pɯ-ɣɤla-t-a}\hspace{5pt}\pcmn{我把皮子浸泡在水里了}\end{exemple}\relationsémantique{参考}{\lien{ⓔla}{la}}
\begin{sous-entrée}{ʑɣɤɣɤla}{ⓔɣɤlaⓝʑɣɤɣɤla} 
\classe{vi}  
\grammaire{refl} \end{sous-entrée}

\begin{définition}\pfra{se baigner}\end{définition}
\begin{définition}\pcmn{沐浴}\end{définition}\end{entrée}

\begin{entrée}{ɣɤlɤt}{}{ⓔɣɤlɤt} 
\classe{vt} \paradigme{dir}{pɯ-}
\begin{définition}\pfra{fermer à clé}\end{définition}
\begin{définition}\pcmn{锁门}\end{définition}
\begin{exemple}\pjya{pɯ-ɣɤlat-a}\hspace{5pt}\pcmn{我锁了}\end{exemple}
\begin{exemple}\pjya{pa-ɣɤlɤt}\hspace{5pt}\pcmn{他锁了}\end{exemple}
\begin{exemple}\pjya{sɤcɯ pɯ-ɣɤlat-a}\hspace{5pt}\pcmn{我锁了}\end{exemple}
\begin{exemple}\pjya{kɯm nɯ pɯ-ɣɤ-lat-a}\hspace{5pt}\pcmn{我锁了门}\end{exemple}
\begin{exemple}\pjya{khɯzgɯr pɯ-ɣɤlat-a}\hspace{5pt}\pcmn{我锁了}\end{exemple}\relationsémantique{参考}{\lien{ⓔlɤtⓗ1}{lɤt₁}}\end{entrée}

\begin{entrée}{ɣɤloŋloŋ}{}{ⓔɣɤloŋloŋ}\relationsémantique{参考}{\lien{ⓔloŋloŋ}{loŋloŋ}}\end{entrée}

\begin{entrée}{ɣɤltshɤltshɤt}{}{ⓔɣɤltshɤltshɤt}\relationsémantique{参考}{\lien{ⓔsɤltshɤltshɤt}{sɤltshɤltshɤt}}\end{entrée}

\begin{entrée}{ɣɤlɯrlɯr}{}{ⓔɣɤlɯrlɯr}\relationsémantique{参考}{\lien{ⓔlɯrlɯr}{lɯrlɯr}}\end{entrée}

\begin{entrée}{ɣɤlɯzlɯz}{}{ⓔɣɤlɯzlɯz} 
\classe{vi} \paradigme{dir}{nɯ-}
\begin{définition}\pfra{se secouer}\end{définition}
\begin{définition}\pcmn{(自动地)摇动}\end{définition}
\begin{définition}\pfra{secouer}\end{définition}
\begin{définition}\pcmn{摇动}\end{définition}
\begin{sous-entrée}{sɤlɯzlɯz}{ⓔɣɤlɯzlɯzⓝsɤlɯzlɯz} 
\classe{vt} \end{sous-entrée}

\end{entrée}

\begin{entrée}{ɣɤlwɤlwɤt}{}{ⓔɣɤlwɤlwɤt} 
\classe{vi} \paradigme{dir}{tɤ-}
\begin{définition}\pfra{s’agiter}\end{définition}
\begin{définition}\pcmn{飘动}\end{définition}\relationsémantique{参考}{\lien{ⓔsɤlwɤlwɤt}{sɤlwɤlwɤt}}\end{entrée}

\begin{entrée}{ɣɤɬɤt}{}{ⓔɣɤɬɤt} 
\classe{vt} 
\begin{définition}\pfra{détendre}\end{définition}
\begin{définition}\pcmn{放松}\end{définition}
\begin{exemple}\pjya{a-rɕa ɲɯ-ɣɤɬat-a ɲɯ-ra}\hspace{5pt}\pcmn{我要放松一下}\end{exemple}
\begin{exemple}\pjya{nɤ-rɕa nɯ-ɣɤɬɤt tɕe nɤ-mgɯr mɤ-mŋɤm}\hspace{5pt}\pcmn{你放松一下,你的背就不会痛了}\end{exemple}\end{entrée}

\begin{entrée}{ɣɤmaʁ}{}{ⓔɣɤmaʁ} 
\classe{vt} \paradigme{dir}{nɯ-}\sens{1}
\begin{définition}\pfra{retirer}\end{définition}
\begin{définition}\pcmn{取消;让……没有……}\end{définition}
\begin{exemple}\pjya{kɯki nɤj nɤ-kho nɯ-ɣɤmaʁ-a}\hspace{5pt}\pcmn{我让你没有这个房子了}\end{exemple}\sens{2}
\begin{définition}\pfra{relever de ses fonctions}\end{définition}
\begin{définition}\pcmn{免职}\end{définition}
\begin{exemple}\pjya{ɯ-khɯrthaŋ nɯ-ɣɤmaʁ-a}\hspace{5pt}\pcmn{我罢免了他的官职}\end{exemple}\relationsémantique{参考}{\lien{ⓔmaʁⓗ2}{maʁ}}\end{entrée}

\begin{entrée}{ɣɤmba}{}{ⓔɣɤmba}\relationsémantique{参考}{\lien{ⓔmba}{mba}}\end{entrée}

\begin{entrée}{ɣɤmbat}{}{ⓔɣɤmbat}\relationsémantique{参考}{\lien{ⓔmbat}{mbat}}\end{entrée}

\begin{entrée}{ɣɤmbɤr}{}{ⓔɣɤmbɤr} 
\classe{vt}  
\grammaire{caus} \paradigme{dir}{pɯ-}
\begin{définition}\pfra{abaisser, rendre moins haut}\end{définition}
\begin{définition}\pcmn{弄低;弄矮}\end{définition}
\begin{exemple}\pjya{ɯʑo kɯ ɯ-phoŋbu pa-ɣɤmbɤr}\hspace{5pt}\pcmn{他稍微低下了身子}\end{exemple}
\begin{exemple}\pjya{kha nɯ pjɯ́-ɣw-ɣɤmbɤr tsa jɤɣ}\hspace{5pt}\pcmn{要把房子弄得矮一点}\end{exemple}
\begin{exemple}\pjya{ɕomskrɯt tɯ-ŋga ɯ-sɤ-ɕkho tɤ-kɤ-βzu nɯ ɲɯ-mbro tɕe, pɯ-ɣɤ-mbara}\hspace{5pt}\pcmn{因为晒衣服的铁丝弄的太高,我把它放低}\end{exemple}
\begin{exemple}\pjya{ɯ-koŋ pɯ-ɣɤ-mbara}\hspace{5pt}\pcmn{我减价了}\end{exemple}
\begin{exemple}\pjya{pa-nɯβʑit tɕe pa-ɣɤ-mbɤr}\hspace{5pt}\pcmn{他锯了一段,把它弄矮了一点}\end{exemple}\relationsémantique{参考}{\lien{ⓔmbɤrⓗ1}{mbɤr₁}}\end{entrée}

\begin{entrée}{ɣɤmbɤrmbɤr}{}{ⓔɣɤmbɤrmbɤr}\relationsémantique{参考}{\lien{ⓔɣɤɲɟɤrɲɟɤr}{ɣɤɲɟɤrɲɟɤr}}\end{entrée}

\begin{entrée}{ɣɤmbɣaʁ}{}{ⓔɣɤmbɣaʁ}\relationsémantique{参考}{\lien{ⓔmbɣaʁ}{mbɣaʁ}}\end{entrée}

\begin{entrée}{ɣɤmbɣo}{}{ⓔɣɤmbɣo} 
\classe{vs}  
\grammaire{denom} \paradigme{dir}{kɤ-}
\begin{définition}\pfra{être sourd}\end{définition}
\begin{définition}\pcmn{聋}\end{définition}
\begin{exemple}\pjya{ɯ-rna mɯ́j-mtshɤm ɲɯ-ɣɤmbɣo}\hspace{5pt}\pcmn{他耳朵听不见,他是聋的}\end{exemple}\relationsémantique{参考}{\lien{ⓔtɤmbɣo}{tɤmbɣo}}\end{entrée}

\begin{entrée}{ɣɤmbɣomru}{}{ⓔɣɤmbɣomru} 
\classe{vs} 
\begin{définition}\pfra{impatient, pressé}\end{définition}
\begin{définition}\pcmn{急躁}\end{définition}
\begin{exemple}\pjya{nɤʑo dal tɤ-pe ma-tɯ-ɣɤmbɣomru}\hspace{5pt}\pcmn{你慢慢做,不要急躁}\end{exemple}\relationsémantique{参考}{\lien{ⓔmbɣom}{mbɣom}}\end{entrée}

\begin{entrée}{ɣɤmbro}{}{ⓔɣɤmbro}\relationsémantique{参考}{\lien{ⓔmbroⓗ2}{mbro}}\end{entrée}

\begin{entrée}{ɣɤmdzu}{}{ⓔɣɤmdzu} 
\classe{vs}  
\grammaire{denom} \paradigme{dir}{tɤ-}
\begin{définition}\pfra{avoir beaucoup d'épines}\end{définition}
\begin{définition}\pcmn{有刺}\end{définition}
\begin{exemple}\pjya{zɲɟa ɲɯ-ɣɤmdzu}\hspace{5pt}\pcmn{黄刺泡儿有刺}\end{exemple}\relationsémantique{参考}{\lien{ⓔtɤ-mdzu}{tɤ-mdzu}}\end{entrée}

\begin{entrée}{ɣɤme}{}{ⓔɣɤme} 
\classe{vt}  
\grammaire{caus} \paradigme{dir}{nɯ-}\paradigme{dir}{nɯ-}
\begin{définition}\pfra{perdre}\end{définition}
\begin{définition}\pcmn{弄丢}\end{définition}
\begin{définition}\pfra{disparaître}\end{définition}
\begin{définition}\pcmn{消失(躲起来,不让别人发现自己)}\end{définition}
\begin{exemple}\pjya{nɯ-ɣɤme-t-a}\hspace{5pt}\pcmn{我丢了}\end{exemple}
\begin{exemple}\pjya{nɯ-tɯ-ɣɤme-t}\hspace{5pt}\pcmn{你丢了}\end{exemple}
\begin{exemple}\pjya{na-ɣɤme}\hspace{5pt}\pcmn{他丢了}\end{exemple}
\begin{exemple}\pjya{jɯɣi ɯ-tɯ-ɣɤme nɯ ?}\hspace{5pt}\pcmn{那本书那么快就丢了?}\end{exemple}
\begin{exemple}\pjya{a-laχtɕha ɲɤ-ɣɤme}\hspace{5pt}\pcmn{我把我的东西弄丢了}\end{exemple}\relationsémantique{参考}{\lien{ⓔmeⓗ2}{me}}
\begin{sous-entrée}{ɯ-rca,ɣɤme}{ⓔɣɤmeⓝɯ-rca,ɣɤme}
\begin{définition}\pfra{mettre en désordre}\end{définition}
\begin{définition}\pcmn{弄得很乱,令人无从做起}\end{définition}
\begin{exemple}\pjya{a-rca ci na-ɣɤme}\hspace{5pt}\pcmn{他把事情弄得很乱,令我无从做起}\end{exemple}
\begin{exemple}\pjya{kɯki kɤ-nɤma ki tu-sɤpe-a nɯ-sɯso-t-a pɯ-ŋu ri, chɤ-nɯkɯmaʁ-a tɕe ɯ-rca ci ɲɤ-ɣɤme-t-a}\hspace{5pt}\pcmn{我本来以为会把这个工作做好,但是弄错了,弄得很乱了}\end{exemple}\end{sous-entrée}

\begin{sous-entrée}{sna,ɣɤme}{ⓔɣɤmeⓝsna,ɣɤme}
\begin{définition}\pfra{abîmer}\end{définition}
\begin{définition}\pcmn{弄烂}\end{définition}
\begin{exemple}\pjya{tɯ-ŋga tɤ-ŋga-t-a tɕe, thɯ-ɴɢraʁ tɕe sna nɯ-ɣɤme-t-a}\hspace{5pt}\pcmn{我穿衣服的时候就破了,把它弄烂了}\end{exemple}\end{sous-entrée}

\begin{sous-entrée}{ʑɣɤɣɤme}{ⓔɣɤmeⓝʑɣɤɣɤme} 
\classe{vi}  
\grammaire{caus}
\grammaire{refl} \end{sous-entrée}

\end{entrée}

\begin{entrée}{ɣɤmi}{}{ⓔɣɤmi} 
\classe{vt}  
\grammaire{caus} \sens{1}\paradigme{dir}{nɯ-}
\begin{définition}\pfra{éteindre la lumière}\end{définition}
\begin{définition}\pcmn{关(灯)}\end{définition}
\begin{exemple}\pjya{nɯ-ɣɤmi-t-a}\hspace{5pt}\pcmn{我关了(灯)}\end{exemple}
\begin{exemple}\pjya{nɯ-tɯ-ɣɤmi-t}\hspace{5pt}\pcmn{你关了(灯)}\end{exemple}
\begin{exemple}\pjya{tɤtʂu a-nɯ-ɣɤmi}\hspace{5pt}\pcmn{要关灯}\end{exemple}\sens{2}\paradigme{dir}{pɯ-}
\begin{définition}\pfra{éteindre un feu}\end{définition}
\begin{définition}\pcmn{灭(火)}\end{définition}
\begin{exemple}\pjya{smi pɯ-ɣɤmi-t-a}\hspace{5pt}\pcmn{我灭了火}\end{exemple}\relationsémantique{参考}{\lien{ⓔmiⓗ1}{mi₁}}\end{entrée}

\begin{entrée}{ɣɤmna}{}{ⓔɣɤmna} 
\classe{vt}  
\grammaire{caus} \paradigme{dir}{tɤ-}
\begin{définition}\pfra{guérir une maladie}\end{définition}
\begin{définition}\pcmn{治病}\end{définition}
\begin{exemple}\pjya{tɤ-ɣɤmna-t-a}\hspace{5pt}\pcmn{我治了病}\end{exemple}
\begin{exemple}\pjya{tɤ-tɯ-ɣɤmna-t}\hspace{5pt}\pcmn{你治了病}\end{exemple}
\begin{exemple}\pjya{ta-ɣɤmna}\hspace{5pt}\pcmn{他治了病}\end{exemple}
\begin{exemple}\pjya{a-kɯ-mŋɤm ta-ɣɤmna}\hspace{5pt}\pcmn{他治了我的病}\end{exemple}
\begin{exemple}\pjya{smɤnba kɯ ɯ-kɯ-mŋɤm ta-ɣɤmna}\hspace{5pt}\pcmn{医生治了他的病}\end{exemple}
\begin{exemple}\pjya{lonba kɤ-ɣɤmna mɯ́j-khɯ}\hspace{5pt}\pcmn{不能完全治好}\end{exemple}
\begin{sous-entrée}{ɣɤmna}{ⓔɣɤmnaⓝɣɤmna} 
\classe{vs} 
\begin{définition}\pfra{guérir facilement}\end{définition}
\begin{définition}\pcmn{容易痊愈}\end{définition}\relationsémantique{参考}{\lien{ⓔmna}{mna}}\end{sous-entrée}

\end{entrée}

\begin{entrée}{ɣɤmɲɤt}{}{ⓔɣɤmɲɤt}\relationsémantique{参考}{\lien{ⓔmɲɤt}{mɲɤt}}\end{entrée}

\begin{entrée}{ɣɤmɲi}{}{ⓔɣɤmɲi}\relationsémantique{参考}{\lien{ⓔmɲi}{mɲi}}\end{entrée}

\begin{entrée}{ɣɤmpɕu}{}{ⓔɣɤmpɕu}\relationsémantique{参考}{\lien{ⓔmpɕu}{mpɕu}}\end{entrée}

\begin{entrée}{ɣɤmpɕɤr}{}{ⓔɣɤmpɕɤr}\relationsémantique{参考}{\lien{ⓔmpɕɤr}{mpɕɤr}}\end{entrée}

\begin{entrée}{ɣɤmpja}{}{ⓔɣɤmpja} 
\classe{vt}  
\grammaire{caus} \paradigme{dir}{tɤ-}\paradigme{dir}{kɤ-}
\begin{définition}\pfra{chauffer, réchauffer}\end{définition}
\begin{définition}\pcmn{加热}\end{définition}
\begin{exemple}\pjya{tɯ-ndza tɤ-ɣɤmpje}\hspace{5pt}\pcmn{你把饭热了吧}\end{exemple}
\begin{exemple}\pjya{smi ɯ-taʁ kú-wɣ-ɣɤmpja}\hspace{5pt}\pcmn{在火上加热}\end{exemple}\end{entrée}

\begin{entrée}{ɣɤmpɯ}{}{ⓔɣɤmpɯ}\relationsémantique{参考}{\lien{ⓔmpɯ}{mpɯ}}\end{entrée}

\begin{entrée}{ɣɤmqrɯz}{}{ⓔɣɤmqrɯz} 
\classe{vi} \paradigme{dir}{tɤ-}
\begin{définition}\pfra{qui fait mal aux pieds lorsque l'on marche dessus}\end{définition}
\begin{définition}\pcmn{硌脚,地面很粗糙的时候,不穿鞋子走上去感到痛、不舒服的感觉}\end{définition}
\begin{exemple}\pjya{ɯ-thoʁ nɯ ɲɯ-ɣɤmqrɯz}\hspace{5pt}\pcmn{地面很硌脚}\end{exemple}
\begin{sous-entrée}{znɤmqrɯz}{ⓔɣɤmqrɯzⓝznɤmqrɯz} 
\classe{vt} \end{sous-entrée}

\sens{1}
\begin{définition}\pfra{faire mal aux pieds (d'une surface rugueuse, lorsque l'on marche dessus sans chaussures)}\end{définition}
\begin{définition}\pcmn{硌脚,地面很粗糙的时候,不穿鞋子走上去感到痛,不舒服的感觉}\end{définition}
\begin{exemple}\pjya{ɯ-thoʁ ɲɯ-rʁom tɕe, a-mi ɲɯ-znɤmqrɯz ma a-xtsa maŋe}\hspace{5pt}\pcmn{地面很粗糙,令我的脚很痛因为我没有穿鞋子}\end{exemple}\sens{2}
\begin{définition}\pfra{causer une sensation acide}\end{définition}
\begin{définition}\pcmn{酸到}\end{définition}
\begin{exemple}\pjya{paχɕi kɯ-tɕur tɤ-ndza-t-a tɕe a-ɕɣa to-znɤmqrɯz}\hspace{5pt}\pcmn{我吃了酸的苹果,牙齿被酸到了}\end{exemple}\end{entrée}

\begin{entrée}{ɣɤmtɕoʁ}{}{ⓔɣɤmtɕoʁ}\relationsémantique{参考}{\lien{ⓔmtɕoʁ}{mtɕoʁ}}\end{entrée}

\begin{entrée}{ɣɤmthu}{}{ⓔɣɤmthu} 
\classe{vt} \paradigme{dir}{nɯ-}
\begin{définition}\pfra{rendre faible}\end{définition}
\begin{définition}\pcmn{令……虚弱}\end{définition}
\begin{exemple}\pjya{a-kɯ-mŋɤm kɯ mɯ-nɯ́-wɣ-ɣɤmthu-a}\hspace{5pt}\pcmn{我的病令我变得很虚弱}\end{exemple}\end{entrée}

\begin{entrée}{ɣɤmto}{}{ⓔɣɤmto}\relationsémantique{参考}{\lien{ⓔmtoⓝmto}{mto}}\end{entrée}

\begin{entrée}{ɣɤmtsɯr}{}{ⓔɣɤmtsɯr}\relationsémantique{参考}{\lien{ⓔmtsɯr}{mtsɯr}}\end{entrée}

\begin{entrée}{ɣɤmɯ}{}{ⓔɣɤmɯ} 
\classe{vt} \paradigme{dir}{pɯ-}
\begin{définition}\pfra{louer}\end{définition}
\begin{définition}\pcmn{称赞;表扬}\end{définition}
\begin{exemple}\pjya{aʑo kɯ pjɯ-ɣɤmi-a}\hspace{5pt}\pcmn{我表扬他}\end{exemple}
\begin{exemple}\pjya{ɲɯ-mkhɤz tɤ-tɯt-a tɕe, pɯ-ɣɤmɯt-a}\hspace{5pt}\pcmn{我说“他很厉害”,表扬了他}\end{exemple}
\begin{exemple}\pjya{kɯki kɤ-βzjoz ɯʑo ɲɯ-mkhɤz, tɕe pjɤ-ɣɤmɯ}\hspace{5pt}\pcmn{他学得很好,所以他(另外一个人)表扬了他}\end{exemple}
\begin{exemple}\pjya{ɯ-kɯmdza (ɯ-βzaŋsa) ɲɯ-ɣɤmi}\hspace{5pt}\pcmn{他称赞他的亲戚(朋友)}\end{exemple}
\begin{sous-entrée}{sɤɣɤmɯ/\variante{sɤzɣɤmɯ}}{ⓔɣɤmɯⓝsɤɣɤmɯ} 
\classe{vi}  
\grammaire{apass} 
\begin{définition}\pfra{louer des gens}\end{définition}
\begin{définition}\pcmn{表扬人}\end{définition}\end{sous-entrée}

\end{entrée}

\begin{entrée}{ɣɤmɯm}{}{ⓔɣɤmɯm}\relationsémantique{参考}{\lien{ⓔmɯm}{mɯm}}\end{entrée}

\begin{entrée}{ɣɤmɯrmɯr}{}{ⓔɣɤmɯrmɯr} 
\classe{vs} 
\begin{définition}\pfra{(surface de l'eau) ayant des rides}\end{définition}
\begin{définition}\pcmn{有轻微的波纹}\end{définition}
\begin{exemple}\pjya{tɯ-ci ɲɯ-ɣɤmɯrmɯr ʑo}\hspace{5pt}\pcmn{水面上有轻微的波纹}\end{exemple}\end{entrée}

\begin{entrée}{ɣɤmɯt}{}{ⓔɣɤmɯt} 
\classe{vt} \paradigme{dir}{thɯ-}
\begin{définition}\pfra{souffler}\end{définition}
\begin{définition}\pcmn{吹(灰)}\end{définition}
\begin{exemple}\pjya{thɯ-ɣɤmɯt-a}\hspace{5pt}\pcmn{我吹了}\end{exemple}
\begin{exemple}\pjya{ɯʑo kɯ tha-ɣɤmɯt}\hspace{5pt}\pcmn{他吹了}\end{exemple}\end{entrée}

\begin{entrée}{ɣɤndɣɤndɣɤt}{}{ⓔɣɤndɣɤndɣɤt} 
\classe{vs}  
\grammaire{deidph} \paradigme{dir}{nɯ-}
\begin{définition}\pfra{trembler}\end{définition}
\begin{définition}\pcmn{颤抖;震动}\end{définition}
\begin{sous-entrée}{sɤndɣɤndɣɤt}{ⓔɣɤndɣɤndɣɤtⓝsɤndɣɤndɣɤt} 
\classe{vt} 
\begin{définition}\pfra{faire trembler}\end{définition}
\begin{définition}\pcmn{使震动}\end{définition}
\begin{exemple}\pjya{kha ɲɯ-sɤndɣɤndɣɤt}\hspace{5pt}\pcmn{他令屋子震动}\end{exemple}
\begin{exemple}\pjya{khɤxtu ɲɯ-nɤmdɯmdar tɕe ɲɯ-sɤndɣɤndɣɤt}\hspace{5pt}\pcmn{他在房背上东跳西跳,令屋子震动}\end{exemple}\relationsémantique{参考}{\lien{ⓔndɣɤndɣɤt}{ndɣɤndɣɤt}}\end{sous-entrée}

\end{entrée}

\begin{entrée}{ɣɤndɯβ}{}{ⓔɣɤndɯβ} 
\classe{vt}  
\grammaire{caus} \paradigme{dir}{pɯ-}\paradigme{dir}{nɯ-}
\begin{définition}\pfra{écraser}\end{définition}
\begin{définition}\pcmn{弄碎}\end{définition}
\begin{exemple}\pjya{aʑo pɯ-ɣɤndɯβ-a}\hspace{5pt}\pcmn{我弄碎了}\end{exemple}
\begin{exemple}\pjya{ɯʑo kɯ pa-ɣɤndɯβ}\hspace{5pt}\pcmn{他弄碎了}\end{exemple}
\begin{exemple}\pjya{paʁndza kɤ-rɤkrɯ pa-ɣɤndɯβ}\hspace{5pt}\pcmn{他把猪草切得很细}\end{exemple}\relationsémantique{参考}{\lien{ⓔndɯβ}{ndɯβ}}\end{entrée}

\begin{entrée}{ɣɤndɯl}{}{ⓔɣɤndɯl}\relationsémantique{参考}{\lien{ⓔndɯlⓗ1}{ndɯl₁}}\end{entrée}

\begin{entrée}{ɣɤndziaʁ}{}{ⓔɣɤndziaʁ}\relationsémantique{参考}{\lien{ⓔndziaʁ}{ndziaʁ}}\end{entrée}

\begin{entrée}{ɣɤndzɯrndzɯr}{}{ⓔɣɤndzɯrndzɯr} 
\classe{vs} \paradigme{dir}{nɯ-}
\begin{définition}\pfra{trembler}\end{définition}
\begin{définition}\pcmn{发抖}\end{définition}
\begin{exemple}\pjya{ɯ-βri ɲɯ-ɣɤndzɯrndzɯr}\hspace{5pt}\pcmn{他的身体在发抖}\end{exemple}
\begin{exemple}\pjya{ɲo-mu tɕe ɲɯ-ɣɤndzɯrndzɯr}\hspace{5pt}\pcmn{他怕了,全身在发抖}\end{exemple}
\begin{sous-entrée}{sɤndzɯrndzɯr}{ⓔɣɤndzɯrndzɯrⓝsɤndzɯrndzɯr} 
\classe{vt} 
\begin{définition}\pfra{faire trembler}\end{définition}
\begin{définition}\pcmn{使发抖}\end{définition}\relationsémantique{同义词}{\lien{ⓔɣɤthɣɤthɣɤt}{ɣɤthɣɤthɣɤt}}\end{sous-entrée}

\end{entrée}

\begin{entrée}{ɣɤndʑɤm}{}{ⓔɣɤndʑɤm} 
\classe{vt}  
\grammaire{caus} \paradigme{dir}{tɤ-}
\begin{définition}\pfra{chauffer la nourriture}\end{définition}
\begin{définition}\pcmn{热(饭)}\end{définition}
\begin{exemple}\pjya{ɯʑo kɯ ta-ɣɤndʑɤm}\hspace{5pt}\pcmn{你热了一下}\end{exemple}
\begin{exemple}\pjya{aʑo tu-ɣɤndʑam-a}\hspace{5pt}\pcmn{我热一下}\end{exemple}
\begin{exemple}\pjya{ki tɤ-lu ki nɤʑo tɤ-ɣɤndʑɤm}\hspace{5pt}\pcmn{你把牛奶热一下}\end{exemple}\relationsémantique{参考}{\lien{ⓔndʑɤm}{ndʑɤm}}\end{entrée}

\begin{entrée}{ɣɤndʑɣɤrlɤr}{}{ⓔɣɤndʑɣɤrlɤr} 
\classe{vi} 
\begin{définition}\pfra{piétiner de partout}\end{définition}
\begin{définition}\pcmn{到处乱踩;动作不雅观}\end{définition}
\begin{exemple}\pjya{a-tɯji ɯ-ŋgɯ ma-tɯ-ɣɤndʑɣɤrlɤr ma a-kɤrkɯm tɯ-rɤtɕaʁ}\hspace{5pt}\pcmn{你不要在我的田地里乱踩,会把我的菜苗踩死}\end{exemple}\end{entrée}

\begin{entrée}{ɣɤndʐo}{}{ⓔɣɤndʐo} 
\classe{vs} \paradigme{dir}{thɯ-}
\begin{définition}\pfra{froid (temps)}\end{définition}
\begin{définition}\pcmn{冷(天气)}\end{définition}
\begin{exemple}\pjya{qartsɯ jo-ɣi ɲɯ-ɣɤndʐo}\hspace{5pt}\pcmn{到了冬天,天气很冷}\end{exemple}
\begin{exemple}\pjya{qale ɲɤ-sɯβzu, ɲɯ-ɣɤndʐo}\hspace{5pt}\pcmn{在刮风,很冷}\end{exemple}\relationsémantique{参考}{\lien{ⓔnɤndʐo}{nɤndʐo}}\relationsémantique{参考}{\lien{ⓔtɤndʐo}{tɤndʐo}}\end{entrée}

\begin{entrée}{ɣɤngɯt}{}{ⓔɣɤngɯt}\relationsémantique{参考}{\lien{ⓔngɯt}{ngɯt}}\end{entrée}

\begin{entrée}{ɣɤnmu}{}{ⓔɣɤnmu}\relationsémantique{参考}{\lien{ⓔnmu}{nmu}}\end{entrée}

\begin{entrée}{ɣɤntaβ}{}{ⓔɣɤntaβ} 
\classe{vt}  
\grammaire{caus} \paradigme{dir}{\_}
\begin{définition}\pfra{poser}\end{définition}
\begin{définition}\pcmn{放置}\end{définition}
\begin{exemple}\pjya{ka-ɣɤntaβ}\hspace{5pt}\pcmn{他放了}\end{exemple}
\begin{exemple}\pjya{aʑo kɯki laχtɕha ki tɕɤkɯ zɯ kɤ-ɣɤntaβ-a}\hspace{5pt}\pcmn{我把这个东西放在那边了}\end{exemple}
\begin{exemple}\pjya{@chabei tɕɤkɯ ka-ɣɤntaβ}\hspace{5pt}\pcmn{他把茶杯放在那边了}\end{exemple}
\begin{exemple}\pjya{kɤ-nɤma mɯ-mɤ-ɲɯ-tɯ-cha nɤ, nɯ-ɣɤntaβ jɤɣ}\hspace{5pt}\pcmn{如果你不会做的话,可以放在那里}\end{exemple}
\begin{exemple}\pjya{ɯ-sɯm nɯ-ɣɤntaβ-a}\hspace{5pt}\pcmn{我令他放心了}\end{exemple}\relationsémantique{参考}{\lien{ⓔɕɯntaβ}{ɕɯntaβ}}\relationsémantique{参考}{\lien{ⓔntaβ}{ntaβ}}
\begin{sous-entrée}{ʑɣɤɣɤntaβ}{ⓔɣɤntaβⓝʑɣɤɣɤntaβ} 
\classe{vi}  
\grammaire{refl} 
\begin{définition}\pfra{rester sans bouger}\end{définition}
\begin{définition}\pcmn{坐稳,动都不动}\end{définition}
\begin{exemple}\pjya{tɤ-ndzur, nɯtɕu ma-nɯ-tɯ-ʑɣɤɣɤntaβ ʑo kɯ}\hspace{5pt}\pcmn{你站起来,不要在那里动都不动}\end{exemple}\end{sous-entrée}

\end{entrée}

\begin{entrée}{ɣɤnɯndzɯlŋɯz}{}{ⓔɣɤnɯndzɯlŋɯz}\relationsémantique{参考}{\lien{ⓔnɯndzɯlŋɯz}{nɯndzɯlŋɯz}}\end{entrée}

\begin{entrée}{ɣɤnɯʑɯβ}{}{ⓔɣɤnɯʑɯβ}\relationsémantique{参考}{\lien{ⓔnɯʑɯβ}{nɯʑɯβ}}\end{entrée}

\begin{entrée}{ɣɤɲɤβɲɤβ}{}{ⓔɣɤɲɤβɲɤβ} 
\classe{vi} \paradigme{dir}{tɤ-}
\begin{définition}\pfra{parler sans arrêt}\end{définition}
\begin{définition}\pcmn{不由自主地流出来;不停地唠叨}\end{définition}
\begin{exemple}\pjya{ma-tɯ-ɣɤɲɤβɲɤβ}\hspace{5pt}\pcmn{你不要啰嗦}\end{exemple}
\begin{exemple}\pjya{nɤ-mtɕhi kɤ-ndɤm, ma-tɯ-ɣɤɲɤβɲɤβ}\hspace{5pt}\pcmn{你闭嘴,不要啰嗦了}\end{exemple}
\begin{exemple}\pjya{aʑo tɤ-ɣɤɲɤβɲɤβ-a tɕe, ɯʑo kɯ nɯ ta-stu}\hspace{5pt}\pcmn{我重复讲很多次,他最后还是照做了}\end{exemple}
\begin{sous-entrée}{sɤɲɤβɲɤβ}{ⓔɣɤɲɤβɲɤβⓝsɤɲɤβɲɤβ} 
\classe{vt} 
\begin{exemple}\pjya{ɯʑo pɯ-rɯndzɤtshi ri toʁde tɕe kɤ-ndza ra ɲɯ-sɤɲɤβɲɤβ ʑo pa-tɕɤt}\hspace{5pt}\pcmn{他在吃东西,突然把嘴里的东西不由自主地吐出来了}\end{exemple}\end{sous-entrée}

\end{entrée}

\begin{entrée}{ɣɤɲcɣɤlɤt}{}{ⓔɣɤɲcɣɤlɤt} 
\classe{vs}  
\grammaire{deidph} \paradigme{dir}{tɤ-}
\begin{définition}\pfra{bruyant}\end{définition}
\begin{définition}\pcmn{吵}\end{définition}
\begin{exemple}\pjya{ma-tɯ-ɣɤɲcɣɤlɤt-nɯ, tɕe aʑo pjɯ-nɯʑɯβ-a}\hspace{5pt}\pcmn{你们不要高声喧哗,让我睡觉吧}\end{exemple}\relationsémantique{参考}{\lien{ⓔɲcɣɤɲcɣɤt}{ɲcɣɤɲcɣɤt}}\end{entrée}

\begin{entrée}{ɣɤɲcɣɤɲcɣɤt}{}{ⓔɣɤɲcɣɤɲcɣɤt} 
\classe{vs} \paradigme{dir}{thɯ-}\sens{1}
\begin{définition}\pfra{animé, bruyant}\end{définition}
\begin{définition}\pcmn{嘈杂(声音)、闹哄哄}\end{définition}
\begin{exemple}\pjya{kɤntɕhaʁ ɲɯ-ɣɤɲcɣɤɲcɣɤt}\hspace{5pt}\pcmn{街上很热闹}\end{exemple}\sens{2}\paradigme{dir}{thɯ-}
\begin{définition}\pfra{ardent (feu)}\end{définition}
\begin{définition}\pcmn{旺盛(火)}\end{définition}
\begin{définition}\pfra{rendre très ardent}\end{définition}
\begin{définition}\pcmn{使火烧得更旺盛}\end{définition}
\begin{exemple}\pjya{tɯrma tɯβlɯ chɤ-nɯ-sɤɲcɣɤɲcɣɤt-nɯ kɤ-ti ɲɯ-ŋu}\hspace{5pt}\pcmn{他们过着幸福的生活(传统故事结尾)}\end{exemple}\relationsémantique{参考}{\lien{ⓔɲcɣɤɲcɣɤt}{ɲcɣɤɲcɣɤt}}
\begin{sous-entrée}{sɤɲcɣɤɲcɣɤt}{ⓔɣɤɲcɣɤɲcɣɤtⓢ2ⓝsɤɲcɣɤɲcɣɤt} 
\classe{vt} \end{sous-entrée}

\end{entrée}

\begin{entrée}{ɣɤɲizɲiz}{}{ⓔɣɤɲizɲiz} 
\classe{vs} \paradigme{dir}{tɤ-}
\begin{définition}\pfra{bonne à rien, qui passe son temps à parler pour ne rien dire (fille)}\end{définition}
\begin{définition}\pcmn{唠叨,不停地讲废话(女孩子)}\end{définition}
\begin{exemple}\pjya{nɤki nɯ ɯ-rju ɲɯ-dɤn tɕe, ɲɯ-ɣɤɲizɲiz ntsɯ}\hspace{5pt}\pcmn{那个女孩子很爱说话,总是在唠叨}\end{exemple}
\begin{exemple}\pjya{ma-tɯ-ɣɤɲizɲiz}\hspace{5pt}\pcmn{你不要唠叨}\end{exemple}
\begin{sous-entrée}{sɤɲizɲiz}{ⓔɣɤɲizɲizⓝsɤɲizɲiz} 
\classe{vt} 
\begin{définition}\pfra{tomber sans arrêt, mais en faible quantité (pluie)}\end{définition}
\begin{définition}\pcmn{雨下个不停但下得不多}\end{définition}
\begin{exemple}\pjya{tɯ-mɯ ɲɯ-sɤɲizɲiz ntsɯ}\hspace{5pt}\pcmn{毛毛雨下个不停}\end{exemple}\end{sous-entrée}

\end{entrée}

\begin{entrée}{ɣɤɲɟɤrɲɟɤr}{}{ⓔɣɤɲɟɤrɲɟɤr} 
\classe{vs} \paradigme{dir}{tɤ-}
\begin{définition}\pfra{qui frétille}\end{définition}
\begin{définition}\pcmn{(很庞大的东西)在抖动}\end{définition}
\begin{exemple}\pjya{ɲɯ-ɣɤrcoʁ tɕe ɲɯ-ɣɤɲɟɤrɲɟɤr ʑo}\hspace{5pt}\pcmn{地面堆了很多泥浆,某个地方一触到,整个地面都在抖动}\end{exemple}
\begin{exemple}\pjya{ɯ-tɯ-tshu ɲɯ-ɣɤɲɟɤrɲɟɤr}\hspace{5pt}\pcmn{胖得肉都在抖动}\end{exemple}\relationsémantique{同义词}{\lien{ⓔɣɤmbɤrmbɤr}{ɣɤmbɤrmbɤr}}\relationsémantique{参考}{\lien{ⓔɲɟɤrɲɟɤr}{ɲɟɤrɲɟɤr}}\end{entrée}

\begin{entrée}{ɣɤɲɟɣɤrɲɟɣɤr}{}{ⓔɣɤɲɟɣɤrɲɟɣɤr} 
\classe{vs} 
\begin{définition}\pfra{long et instable}\end{définition}
\begin{définition}\pcmn{不稳固}\end{définition}
\begin{exemple}\pjya{romɲa ɲɯ-mpɯ tɕe, ɯ-taʁ tu-kɯ-ŋke tɕe ɲɯ-ɣɤɲɟɣɤrɲɟɣɤr}\hspace{5pt}\pcmn{小梁不稳固,走在上面一晃一晃的}\end{exemple}\end{entrée}

\begin{entrée}{ɣɤŋɤn}{}{ⓔɣɤŋɤn}\relationsémantique{参考}{\lien{ⓔŋɤn}{ŋɤn}}\end{entrée}

\begin{entrée}{ɣɤŋgi}{}{ⓔɣɤŋgi} 
\classe{vi}  
\grammaire{refl} \paradigme{dir}{pɯ-}\paradigme{dir}{pɯ-}\paradigme{dir}{pɯ-}
\begin{définition}\pfra{avoir raison}\end{définition}
\begin{définition}\pcmn{说得对}\end{définition}
\begin{définition}\pfra{être d'accord avec, dire que quelqu'un a raison}\end{définition}
\begin{définition}\pcmn{同意,说别人是对的}\end{définition}
\begin{exemple}\pjya{aʑo pɯ-ɣɤŋgi-a}\hspace{5pt}\pcmn{我是对的}\end{exemple}
\begin{exemple}\pjya{nɤʑo kɯ nɯ ɲɯ-tɯ-ti, ɲɯ-tɯ-ɣɤŋgi}\hspace{5pt}\pcmn{你说那些,你是对的}\end{exemple}
\begin{exemple}\pjya{kɯ-rkɯn sɤznɤ kɯ-dɤn nɯ ra pjɯ-ɣɤŋgi-nɯ ra}\hspace{5pt}\pcmn{比起少数,多数是对的}\end{exemple}\relationsémantique{反义词}{\lien{ⓔɣɤtɕa}{ɣɤtɕa}}
\begin{sous-entrée}{zɣɤŋgi}{ⓔɣɤŋgiⓝzɣɤŋgi} 
\classe{vt} \end{sous-entrée}

\begin{sous-entrée}{ʑɣɤɣɤŋgi}{ⓔɣɤŋgiⓝʑɣɤɣɤŋgi} 
\classe{vi} \end{sous-entrée}

\begin{définition}\pfra{dire que l'on a raison}\end{définition}
\begin{définition}\pcmn{说自己是对的}\end{définition}\end{entrée}

\begin{entrée}{ɣɤŋgrɯ}{}{ⓔɣɤŋgrɯ} 
\classe{vt}  
\grammaire{caus} \paradigme{dir}{pɯ-}
\begin{définition}\pfra{permettre de réussir}\end{définition}
\begin{définition}\pcmn{让……成功,得逞}\end{définition}
\begin{exemple}\pjya{nɤʑo nɤ-kɤ-sɯso nɯ aʑo pjɯ-ɣɤŋgri-a jɤɣ.}\hspace{5pt}\pcmn{我可以实现你的愿望}\end{exemple}\relationsémantique{参考}{\lien{ⓔŋgrɯ}{ŋgrɯ}}\end{entrée}

\begin{entrée}{ɣɤŋgɯrŋgɯr}{}{ⓔɣɤŋgɯrŋgɯr} 
\classe{vs}  
\grammaire{deidph} \paradigme{dir}{tɤ-}
\begin{définition}\pfra{fort (bruit)}\end{définition}
\begin{définition}\pcmn{发出隆隆的响声}\end{définition}
\begin{exemple}\pjya{mbɣɯrloʁ ɲɯ-ɣɤŋgɯrŋgɯr}\hspace{5pt}\pcmn{雷发出隆隆的声音}\end{exemple}\relationsémantique{参考}{\lien{ⓔŋgɯrⓗ2}{ŋgɯr₂}}\end{entrée}

\begin{entrée}{ɣɤŋoʁ}{}{ⓔɣɤŋoʁ} 
\classe{vt}  
\grammaire{recip} \sens{1}\paradigme{dir}{tɤ-}
\begin{définition}\pfra{saluer}\end{définition}
\begin{définition}\pcmn{打招呼}\end{définition}
\begin{exemple}\pjya{tɤ-ɣɤŋoʁ-a}\hspace{5pt}\pcmn{我给他打了招呼}\end{exemple}
\begin{exemple}\pjya{aʑo @xiangbolin ɯ-phe tɤ-ɣɤŋoʁ-a}\hspace{5pt}\pcmn{我给向柏霖打了招呼}\end{exemple}\sens{2}\paradigme{dir}{\_}\paradigme{dir}{tɤ-}
\begin{définition}\pfra{chasser (animaux)}\end{définition}
\begin{définition}\pcmn{驱逐,赶走(动物)}\end{définition}
\begin{exemple}\pjya{jɤ-ɣɤŋoʁ-a}\hspace{5pt}\pcmn{我把它赶走了}\end{exemple}
\begin{exemple}\pjya{ja-ɣɤŋoʁ}\hspace{5pt}\pcmn{他把它赶走了}\end{exemple}
\begin{exemple}\pjya{lɯlu jɤ-ɣɤŋoʁ-a ma tɤ-mthɯm tu-mɯrki ɲɯ-ŋu}\hspace{5pt}\pcmn{我把猫赶走了因为它在偷肉}\end{exemple}
\begin{sous-entrée}{aɣɤŋɯŋoʁ}{ⓔɣɤŋoʁⓢ2ⓝaɣɤŋɯŋoʁ} 
\classe{vi} \end{sous-entrée}

\begin{définition}\pfra{se saluer les uns les autres}\end{définition}
\begin{définition}\pcmn{互相打招呼}\end{définition}\end{entrée}

\begin{entrée}{ɣɤŋoʁle}{}{ⓔɣɤŋoʁle} 
\classe{vs} 
\begin{définition}\pfra{extraverti}\end{définition}
\begin{définition}\pcmn{外向,爱说话}\end{définition}
\begin{exemple}\pjya{ɯʑo kɯ-ɣɤŋoʁle ci ŋu}\hspace{5pt}\pcmn{他是个爱打交道的人}\end{exemple}\relationsémantique{同义词}{\lien{ⓔɣɤχalala}{ɣɤχalala}}\end{entrée}

\begin{entrée}{ɣɤɴɢrɯ}{}{ⓔɣɤɴɢrɯ}\relationsémantique{参考}{\lien{ⓔɴɢrɯ}{ɴɢrɯ}}\end{entrée}

\begin{entrée}{ɣɤɴqa}{}{ⓔɣɤɴqa}\relationsémantique{参考}{\lien{ⓔɴqa}{ɴqa}}\end{entrée}

\begin{entrée}{ɣɤɴqhi}{}{ⓔɣɤɴqhi}\relationsémantique{参考}{\lien{ⓔɴqhi}{ɴqhi}}\end{entrée}

\begin{entrée}{ɣɤpaʁpaʁ}{}{ⓔɣɤpaʁpaʁ} 
\classe{vs} 
\begin{définition}\pfra{acide}\end{définition}
\begin{définition}\pcmn{很酸;很辣}\end{définition}
\begin{exemple}\pjya{ɲɯ-tɕur ɲɯ-ɣɤpaʁpaʁ ʑo}\hspace{5pt}\pcmn{很酸}\end{exemple}
\begin{exemple}\pjya{ɲɯ-mɤrtsaβ ɲɯ-ɣɤpaʁpaʁ ʑo}\hspace{5pt}\pcmn{很辣}\end{exemple}\end{entrée}

\begin{entrée}{ɣɤpɣo}{}{ⓔɣɤpɣo} 
\classe{vt}  
\grammaire{denom} \paradigme{dir}{tɤ-}
\begin{définition}\pfra{empiler}\end{définition}
\begin{définition}\pcmn{堆起来}\end{définition}
\begin{exemple}\pjya{ta-ɣɤpɣo}\hspace{5pt}\pcmn{他堆起来了}\end{exemple}
\begin{exemple}\pjya{ki si kutɕu tɤ-ɣɤpɣo-t-a}\hspace{5pt}\pcmn{我把这些木头在这里堆起来}\end{exemple}
\begin{exemple}\pjya{tɤ-fkɯm ɯ-ŋgɯ tɯ-rdoʁ chɯ́-wɣ-rku tɕe tú-wɣ-ɣɤpɣo}\hspace{5pt}\pcmn{把粮食装在口袋里堆起来了}\end{exemple}\end{entrée}

\begin{entrée}{ɣɤphɤn}{}{ⓔɣɤphɤn}\relationsémantique{参考}{\lien{ⓔphɤn}{phɤn}}\end{entrée}

\begin{entrée}{ɣɤphrɤβphrɤβ}{}{ⓔɣɤphrɤβphrɤβ}\relationsémantique{参考}{\lien{ⓔphrɤβ}{phrɤβ}}\end{entrée}

\begin{entrée}{ɣɤphɯɕlaʁ}{}{ⓔɣɤphɯɕlaʁ} 
\classe{vi} \paradigme{dir}{tɤ-}\paradigme{dir}{tɤ-}
\begin{définition}\pfra{aux mouvements rapides}\end{définition}
\begin{définition}\pcmn{勤快,动作伶俐}\end{définition}
\begin{définition}\pfra{agir avec zèle}\end{définition}
\begin{définition}\pcmn{做得很勤快;马上做完}\end{définition}
\begin{exemple}\pjya{tɤ-tɕɯ kɯ-ɣɤphɯɕlaʁ ci ɲɯ-ŋu}\hspace{5pt}\pcmn{他是一个勤快的孩子}\end{exemple}
\begin{exemple}\pjya{tɤ-sɤphɯɕlaʁ-a tɤ-nɤma-t-a}\hspace{5pt}\pcmn{我很快就(把工作)做好了}\end{exemple}
\begin{exemple}\pjya{mɯ́j-sɤphɯɕlaʁ}\hspace{5pt}\pcmn{他做得很慢}\end{exemple}\relationsémantique{参考}{\lien{ⓔɕlaʁ}{ɕlaʁ}}
\begin{sous-entrée}{sɤphɯɕlaʁ}{ⓔɣɤphɯɕlaʁⓝsɤphɯɕlaʁ} 
\classe{vt} \end{sous-entrée}

\end{entrée}

\begin{entrée}{ɣɤploʁploʁ}{}{ⓔɣɤploʁploʁ}\relationsémantique{参考}{\lien{ⓔploʁploʁ}{ploʁploʁ}}\end{entrée}

\begin{entrée}{ɣɤpɯplɯɣ}{}{ⓔɣɤpɯplɯɣ} 
\classe{vi}  
\grammaire{deidph} \paradigme{dir}{tɤ-}
\begin{définition}\pfra{briller par intermittence, bouillir (eau)}\end{définition}
\begin{définition}\pcmn{闪光;水沸腾翻滚的样子}\end{définition}
\begin{exemple}\pjya{tʂha ɲɯ-ɤla ɲɯ-ɣɤpɯplɯɣ}\hspace{5pt}\pcmn{茶在沸腾}\end{exemple}
\begin{exemple}\pjya{tɯ-ci tu-ola tu-ɣɤpɯplɯɣ ɲɯ-ŋu}\hspace{5pt}\pcmn{水在沸腾}\end{exemple}
\begin{exemple}\pjya{tɤrmbja ɲɯ-ɣɤpɯplɯɣ ʑo}\hspace{5pt}\pcmn{闪电一闪一闪地发光}\end{exemple}\end{entrée}

\begin{entrée}{ɣɤqhɤβjɤβ}{}{ⓔɣɤqhɤβjɤβ} 
\classe{vi} \paradigme{dir}{nɯ-}
\begin{définition}\pfra{chercher partout}\end{définition}
\begin{définition}\pcmn{到处乱搜;乱找}\end{définition}
\begin{exemple}\pjya{aʁɤndɯndɤt ɲɯ-ɣɤqhɤβjɤβ ʑo tu-kɯ-stu ra ma mɯrkɯ}\hspace{5pt}\pcmn{他到处乱找东西,要注意不然他会偷东西}\end{exemple}
\begin{exemple}\pjya{laχtɕha ma-tɯ-ɣɤqhɤβjɤβ}\hspace{5pt}\pcmn{你不要乱找东西}\end{exemple}\relationsémantique{参考}{\lien{ⓔɣɤkɤβjɤβ}{ɣɤkɤβjɤβ}}\end{entrée}

\begin{entrée}{ɣɤqlaʁqlaʁ}{}{ⓔɣɤqlaʁqlaʁ} 
\classe{vs} \sens{1}
\begin{définition}\pfra{clair (ciel)}\end{définition}
\begin{définition}\pcmn{晴}\end{définition}
\begin{exemple}\pjya{tɯ-mɯ ɲɯ-jɯm ɲɯ-ɣɤqlaʁqlaʁ ʑo}\hspace{5pt}\pcmn{天很晴}\end{exemple}\sens{2}
\begin{définition}\pfra{très dur}\end{définition}
\begin{définition}\pcmn{硬邦邦}\end{définition}
\begin{exemple}\pjya{stoʁ ɲɯ-rko ɲɯ-ɣɤqlaʁqlaʁ, kɤ-ndza mɯ́j-sɤsphɯt}\hspace{5pt}\pcmn{胡豆是硬邦邦的,吃不动}\end{exemple}\end{entrée}

\begin{entrée}{ɣɤqrɤβqrɤβ}{}{ⓔɣɤqrɤβqrɤβ} 
\classe{vi} 
\begin{définition}\pfra{rauque, enroué}\end{définition}
\begin{définition}\pcmn{嗓子嘶哑的}\end{définition}
\begin{exemple}\pjya{ɯ-rqo ɲɯ-ɣɤqrɤβqrɤβ}\hspace{5pt}\pcmn{他嗓子嘶哑地讲话}\end{exemple}\end{entrée}

\begin{entrée}{ɣɤqɯβqɯβ}{}{ⓔɣɤqɯβqɯβ} 
\classe{vs} 
\begin{définition}\pfra{murmurer (eau)}\end{définition}
\begin{définition}\pcmn{潺潺流水}\end{définition}
\begin{exemple}\pjya{tɯ-ci ɲɯ-ɣɤqɯβqɯβ ʑo ɲɯ-ɣi}\hspace{5pt}\pcmn{潺潺流水}\end{exemple}\relationsémantique{参考}{\lien{ⓔɣɤrɕɯrɕɯβ}{ɣɤrɕɯrɕɯβ}}\relationsémantique{参考}{\lien{ⓔɣɤqɯrqɯr}{ɣɤqɯrqɯr}}\end{entrée}

\begin{entrée}{ɣɤqɯrqɯr}{}{ⓔɣɤqɯrqɯr} 
\classe{vs} 
\begin{définition}\pfra{murmurer (eau)}\end{définition}
\begin{définition}\pcmn{潺潺流水}\end{définition}
\begin{exemple}\pjya{tɯ-ci ɲɯ-ɣɤqɯrqɯr ʑo ɲɯ-ɣi}\hspace{5pt}\pcmn{潺潺流水}\end{exemple}\relationsémantique{同义词}{\lien{ⓔɣɤrɕɯrɕɯβ}{ɣɤrɕɯrɕɯβ}}\relationsémantique{参考}{\lien{ⓔɣɤqɯβqɯβ}{ɣɤqɯβqɯβ}}\end{entrée}

\begin{entrée}{ɣɤra}{}{ⓔɣɤra} 
\classe{vt}  
\grammaire{caus} \paradigme{dir}{\_}
\begin{définition}\pfra{rendre nécessaire}\end{définition}
\begin{définition}\pcmn{使……需要}\end{définition}
\begin{exemple}\pjya{a-ku a-kɤrme pɯ-zri ri, pa-qrɤz-nɯ tɕe a-rte ta-ɣɤra}\hspace{5pt}\pcmn{原来我的头发很长,他们剪了之后,我需要戴帽子}\end{exemple}\relationsémantique{参考}{\lien{ⓔraⓗ1}{ra₁}}\end{entrée}

\begin{entrée}{ɣɤrɤβ}{}{ⓔɣɤrɤβ} 
\classe{vs} \paradigme{dir}{tɤ-}
\begin{définition}\pfra{pentu}\end{définition}
\begin{définition}\pcmn{陡峭}\end{définition}
\begin{exemple}\pjya{sɤtɕha ɲɯ-ɣɤrɤβ}\hspace{5pt}\pcmn{(那个)地方很陡峭}\end{exemple}
\begin{exemple}\pjya{tɤ-ɣmbaj ɲɯ-ɣɤrɤβ}\hspace{5pt}\pcmn{山的那一面很陡峭}\end{exemple}
\begin{exemple}\pjya{zgo ɲɯ-ɣɤrɤβ}\hspace{5pt}\pcmn{山坡很陡}\end{exemple}\relationsémantique{同义词}{\lien{}{ɣɤʑin}}\étymologie{rab}\end{entrée}

\begin{entrée}{ɣɤrɤru}{}{ⓔɣɤrɤru} 
\classe{vi} \paradigme{dir}{tɤ-}
\begin{définition}\pfra{se lever tôt (se lever dès qu'on est réveillé)}\end{définition}
\begin{définition}\pcmn{起得早(叫了马上就起来)}\end{définition}
\begin{exemple}\pjya{tɤ-rɟit ɲɯ-ɣɤrɤru}\hspace{5pt}\pcmn{孩子一叫就起床}\end{exemple}
\begin{exemple}\pjya{jiɕqha nɯ, toʁde ku-nɯ-rŋgɯ, nɯɕɯmɯma chɯ-rɤru tɕe kɯ-ɣɤrɤru ci ɲɯ-ŋu}\hspace{5pt}\pcmn{这个人睡得早,起得早,是个爱早起的人}\end{exemple}\relationsémantique{参考}{\lien{ⓔrɤru}{rɤru}}\end{entrée}

\begin{entrée}{ɣɤrɤt}{}{ⓔɣɤrɤt} 
\classe{vt} \paradigme{dir}{thɯ-}\paradigme{dir}{\_}
\begin{définition}\pfra{jeter}\end{définition}
\begin{définition}\pcmn{扔}\end{définition}
\begin{exemple}\pjya{aʑo thɯ-ɣɤrat-a}\hspace{5pt}\pcmn{我扔了}\end{exemple}
\begin{exemple}\pjya{jiɕqha laχtɕha nɯ tha-ɣɤrɤt}\hspace{5pt}\pcmn{他扔了那个东西}\end{exemple}
\begin{exemple}\pjya{qha kɤ-ŋga mɯ́j-sna, tha-ɣɤrɤt}\hspace{5pt}\pcmn{这件(衣服)不能穿,他就扔了}\end{exemple}
\begin{exemple}\pjya{ki sna maŋe, aj thɯ-ɣɤrat-a}\hspace{5pt}\pcmn{这个没有用,我就扔了}\end{exemple}
\begin{exemple}\pjya{khɯɣɲɟɯ ɯ-pɕi na-ɣɤrɤt}\hspace{5pt}\pcmn{他把它扔到窗子外面去了}\end{exemple}
\begin{exemple}\pjya{ɲɤ-tsɣi tɕe thɯ-ɣɤrat-a}\hspace{5pt}\pcmn{(苹果)烂了,所以我把它扔了}\end{exemple}\end{entrée}

\begin{entrée}{ɣɤrchɤrchɤt}{}{ⓔɣɤrchɤrchɤt}\relationsémantique{参考}{\lien{ⓔrchɤrchɤt}{rchɤrchɤt}}\end{entrée}

\begin{entrée}{ɣɤrchɯɣlɯɣ}{}{ⓔɣɤrchɯɣlɯɣ}\relationsémantique{参考}{\lien{ⓔrchɯɣrchɯɣ}{rchɯɣrchɯɣ}}\end{entrée}

\begin{entrée}{ɣɤrchɯɣrchɯɣ}{}{ⓔɣɤrchɯɣrchɯɣ}\relationsémantique{参考}{\lien{ⓔrchɯɣrchɯɣ}{rchɯɣrchɯɣ}}\end{entrée}

\begin{entrée}{ɣɤrcoʁ}{}{ⓔɣɤrcoʁ} 
\classe{vs}  
\grammaire{denom} 
\begin{définition}\pfra{boueux}\end{définition}
\begin{définition}\pcmn{泥泞;满是污泥}\end{définition}
\begin{exemple}\pjya{tʂu ɲɯ-ɣɤrcoʁ}\hspace{5pt}\pcmn{路很泥泞}\end{exemple}
\begin{exemple}\pjya{ɲɯ-ɣɤrcoʁ tɕe ɲɯ-ɣɤɲɟɤrɲɟɤr}\hspace{5pt}\pcmn{泥巴很稀}\end{exemple}\relationsémantique{参考}{\lien{ⓔtɤrcoʁ}{tɤrcoʁ}}\relationsémantique{参考}{\lien{ⓔrɤrcoʁ}{rɤrcoʁ}}\end{entrée}

\begin{entrée}{ɣɤrɕo}{}{ⓔɣɤrɕo}\relationsémantique{参考}{\lien{ⓔarɕo}{arɕo}}\end{entrée}

\begin{entrée}{ɣɤrɕɯrɕiz}{}{ⓔɣɤrɕɯrɕiz} 
\classe{vs} 
\begin{définition}\pfra{murmurer de façon intermittente (eau)}\end{définition}
\begin{définition}\pcmn{水流动时发出断断续续的声音}\end{définition}
\begin{exemple}\pjya{tɯ-mɯ kɯ-ɣɤrɕɯrɕiz ʑo ɲɯ-ɤsɯ-lɤt}\hspace{5pt}\pcmn{下雨,发出断断续续的声音}\end{exemple}\relationsémantique{参考}{\lien{ⓔɣɤrɕɯrɕɯβ}{ɣɤrɕɯrɕɯβ}}\end{entrée}

\begin{entrée}{ɣɤrɕɯrɕɯβ/\variante{ɣɤrɕɯβrɕɯβ}}{}{ⓔɣɤrɕɯrɕɯβ} 
\classe{vi} \sens{1}
\begin{définition}\pfra{émettre un bruit de froissement du papier}\end{définition}
\begin{définition}\pcmn{发出沙沙声}\end{définition}\sens{2}
\begin{définition}\pfra{murmurer (eau)}\end{définition}
\begin{définition}\pcmn{潺潺流水}\end{définition}
\begin{exemple}\pjya{tɯ-ci ɲɯ-ɣɤrɕɯrɕɯβ ʑo ɲɯ-ɣi}\hspace{5pt}\pcmn{潺潺流水}\end{exemple}
\begin{exemple}\pjya{tɯ-mɯ ɲɯ-ɣɤrɕɯrɕɯβ ʑo ɲɯ-ɤsɯ-lɤt}\hspace{5pt}\pcmn{唰唰地下雨}\end{exemple}
\begin{sous-entrée}{sɤrɕɯβrɕɯβ}{ⓔɣɤrɕɯrɕɯβⓢ2ⓝsɤrɕɯβrɕɯβ} 
\classe{vt} 
\begin{exemple}\pjya{tɯ-mɯ ɲɯ-sɤrɕɯβrɕɯβ ʑo ɲɯ-ɤsɯ-lɤt}\hspace{5pt}\pcmn{唰唰地下雨}\end{exemple}\relationsémantique{参考}{\lien{ⓔɣɤrsɯβrsɯβ}{ɣɤrsɯβrsɯβ}}\relationsémantique{参考}{\lien{ⓔɣɤrɕɯrɕiz}{ɣɤrɕɯrɕiz}}\relationsémantique{参考}{\lien{ⓔɣɤqɯβqɯβ}{ɣɤqɯβqɯβ}}\end{sous-entrée}

\end{entrée}

\begin{entrée}{ɣɤrdɯl}{}{ⓔɣɤrdɯl} 
\classe{vs} 
\begin{définition}\pfra{plein de poussière}\end{définition}
\begin{définition}\pcmn{充满灰尘}\end{définition}
\begin{exemple}\pjya{kɤntɕhaʁ ɯ-tɯ-ɣɤrdɯl ɲɯ-sɤre ʑo}\hspace{5pt}\pcmn{街上很多灰尘}\end{exemple}\relationsémantique{参考}{\lien{ⓔnɯrdɯl}{nɯrdɯl}}\relationsémantique{参考}{\lien{ⓔrdɯl}{rdɯl}}\end{entrée}

\begin{entrée}{ɣɤrgɤz}{}{ⓔɣɤrgɤz}\relationsémantique{参考}{\lien{ⓔrgɤz}{rgɤz}}\end{entrée}

\begin{entrée}{ɣɤrɣɤβrɣɤβ}{}{ⓔɣɤrɣɤβrɣɤβ}\relationsémantique{参考}{\lien{ⓔrɣɤβrɣɤβ}{rɣɤβrɣɤβ}}\end{entrée}

\begin{entrée}{ɣɤrɣɤr}{}{ⓔɣɤrɣɤr} 
\classe{idph.2} 
\begin{définition}\pfra{bête}\end{définition}
\begin{définition}\pcmn{发呆}\end{définition}
\begin{exemple}\pjya{ɣɤrɣɤr ʑo ma-tɤ-tɯ-ʑɣɤstu}\hspace{5pt}\pcmn{你不要在那里发呆}\end{exemple}\relationsémantique{参考}{\lien{ⓔdɣɤrdɣɤr}{dɣɤrdɣɤr}}\end{entrée}

\begin{entrée}{ɣɤrjɤlɤt}{}{ⓔɣɤrjɤlɤt}\relationsémantique{参考}{\lien{ⓔrjɤrjɤt}{rjɤrjɤt}}\end{entrée}

\begin{entrée}{ɣɤrkhɤrkhɤt}{}{ⓔɣɤrkhɤrkhɤt} 
\classe{vi} 
\begin{définition}\pfra{faire de petits coups légers}\end{définition}
\begin{définition}\pcmn{发出弹起来的声音,发出轻轻的敲击声}\end{définition}
\begin{sous-entrée}{sɤrkhɤrkhɤt}{ⓔɣɤrkhɤrkhɤtⓝsɤrkhɤrkhɤt} 
\classe{vt} 
\begin{exemple}\pjya{@yangyu ɲɯ-ɤz-rɤkrɯ tɕe ɲɯ-sɤrkhɤrkhɤt}\hspace{5pt}\pcmn{他切土豆发出轻轻的敲击声}\end{exemple}\relationsémantique{参考}{\lien{ⓔrkhɤrkhɤt}{rkhɤrkhɤt}}\end{sous-entrée}

\end{entrée}

\begin{entrée}{ɣɤrkɯn}{}{ⓔɣɤrkɯn} 
\classe{vt}  
\grammaire{caus} \paradigme{dir}{pɯ-}\paradigme{dir}{nɯ-}
\begin{définition}\pfra{diminuer}\end{définition}
\begin{définition}\pcmn{减少}\end{définition}
\begin{exemple}\pjya{pɯ-ɣɤrkɯn-a}\hspace{5pt}\pcmn{我减少了}\end{exemple}
\begin{exemple}\pjya{ɯʑo kɯ pa-ɣɤrkɯn}\hspace{5pt}\pcmn{他减少了}\end{exemple}
\begin{exemple}\pjya{tɤ-rʑaʁ ɲo-ɣɤrkɯn}\hspace{5pt}\pcmn{他把时间减少了}\end{exemple}\relationsémantique{参考}{\lien{ⓔrkɯn}{rkɯn}}\étymologie{dkon}\end{entrée}

\begin{entrée}{ɣɤrlaʁ}{}{ⓔɣɤrlaʁ} 
\classe{vt}  
\grammaire{caus} \paradigme{dir}{nɯ-}\paradigme{dir}{thɯ-}
\begin{définition}\pfra{détruire (une famille)}\end{définition}
\begin{définition}\pcmn{消灭(家族),弄丢}\end{définition}
\begin{sous-entrée}{aɣɤrlɯrlaʁ}{ⓔɣɤrlaʁⓝaɣɤrlɯrlaʁ} 
\classe{vi}  
\grammaire{recip}
\grammaire{caus} 
\begin{définition}\pfra{s'évincer mutuellement}\end{définition}
\begin{définition}\pcmn{互相倾轧}\end{définition}
\begin{exemple}\pjya{kɯɕɯŋgɯ tɕe, tɤru ra tu-onɯsnɯɲɯɲaʁ-nɯ tɕe chɯ-ɤɣɤrlɯrlaʁ-nɯ pjɤ-ŋgrɤl}\hspace{5pt}\pcmn{古时候,贵族们一直互相倾轧}\end{exemple}\relationsémantique{参考}{\lien{ⓔrlaʁ}{rlaʁ}}\end{sous-entrée}

\end{entrée}

\begin{entrée}{ɣɤrlɤrlɤɣ}{}{ⓔɣɤrlɤrlɤɣ} 
\classe{vi} \paradigme{dir}{tɤ-}\paradigme{dir}{tɤ-}
\begin{définition}\pfra{rouler (objet rond)}\end{définition}
\begin{définition}\pcmn{滚;扭动(圆东西)}\end{définition}
\begin{exemple}\pjya{jiɕqha laχtɕha nɯ ɲɯ-ɣɤrlɤrlɤɣ}\hspace{5pt}\pcmn{那个东西在滚}\end{exemple}
\begin{exemple}\pjya{khɯtsa ɯ-ta mɯ-ɲɤ-βdi ɲɯ-ɣɤrlɤrlɤɣ}\end{exemple}
\begin{exemple}\pjya{ɯ-ku ɲɯ-sɤrlɤrlɤɣ}\hspace{5pt}\pcmn{他摇头}\end{exemple}
\begin{sous-entrée}{sɤrlɤrlɤɣ}{ⓔɣɤrlɤrlɤɣⓝsɤrlɤrlɤɣ} 
\classe{vt} \end{sous-entrée}

\end{entrée}

\begin{entrée}{ɣɤrmɤβrmɤβ}{}{ⓔɣɤrmɤβrmɤβ} 
\classe{vs} \paradigme{dir}{tɤ-}\paradigme{dir}{tɤ-}
\begin{définition}\pfra{scintiller}\end{définition}
\begin{définition}\pcmn{一闪一闪}\end{définition}
\begin{exemple}\pjya{@dian mɯ-tɤ-pe tɕe, @dianshi tu-ɣɤrmɤβrmɤβ ʑo ɲɯ-ŋu}\hspace{5pt}\pcmn{电不好的时候,电视一闪一闪}\end{exemple}
\begin{exemple}\pjya{ɯ-mɲaʁ tu-sɤrmɤβrmɤβ ʑo ɲɯ-ŋu}\hspace{5pt}\pcmn{他眼睛一眨一眨,眨得很快}\end{exemple}
\begin{sous-entrée}{sɤrmɤβrmɤβ}{ⓔɣɤrmɤβrmɤβⓝsɤrmɤβrmɤβ} 
\classe{vt} \end{sous-entrée}

\end{entrée}

\begin{entrée}{ɣɤrmɤrmɤβ}{}{ⓔɣɤrmɤrmɤβ} 
\classe{vi} 
\begin{définition}\pfra{qui passe rapidement (comme un éclair)}\end{définition}
\begin{définition}\pcmn{闪得很快}\end{définition}
\begin{exemple}\pjya{nɤ-@diannao ɲɯ-ɣɤrmɤrmɤβ ʑo ri, nɯ kɯnɤ tɤ-scoz ɲɯ-tɯ-mtɤm}\hspace{5pt}\pcmn{虽然电脑屏幕上的字闪的很快,你还是看得见}\end{exemple}\end{entrée}

\begin{entrée}{ɣɤrndi}{}{ⓔɣɤrndi} 
\classe{vt}  
\grammaire{refl} \sens{1}\paradigme{dir}{kɤ-}
\begin{définition}\pfra{soutenir, déployer}\end{définition}
\begin{définition}\pcmn{搀扶;扶起来;撑住;稳住}\end{définition}
\begin{exemple}\pjya{aʑo ku-ɣɤrndi-a}\hspace{5pt}\pcmn{我把他扶起来}\end{exemple}
\begin{exemple}\pjya{ki laχtɕha ki kɤ-ɣɤrndi}\hspace{5pt}\pcmn{你把这个东西扶起来}\end{exemple}
\begin{exemple}\pjya{ki @huatong ki a-mɤ-pɯ-ndʐaβ, kɤ-ɣɤrndi}\hspace{5pt}\pcmn{不要让这个话筒掉下去,你把它扶起来吧}\end{exemple}
\begin{exemple}\pjya{jiɕqha tɯrme nɯ hanɯni ngo, aj kɤ-ɣɤrndi-t-a}\hspace{5pt}\pcmn{这个人身体有点不舒服,我把他扶起来了}\end{exemple}
\begin{exemple}\pjya{ndʐaβ ɲɯ-ŋu, kɤ-ɣɤrndi-t-a}\hspace{5pt}\pcmn{差一点掉下去了,我把它扶起来了}\end{exemple}
\begin{exemple}\pjya{a-wa lo-βzi tɕe, kɤ-ɣɤrndi-t-a}\hspace{5pt}\pcmn{我父亲醉了,我把他扶起来了}\end{exemple}\sens{2}\paradigme{dir}{kɤ-}\paradigme{dir}{kɤ-}
\begin{définition}\pfra{calmer}\end{définition}
\begin{définition}\pcmn{冷静}\end{définition}
\begin{exemple}\pjya{nɤ-sɯm nɯ kɤ-ɣɤrndi ɲɯ-ra}\hspace{5pt}\pcmn{你要镇静下来}\end{exemple}
\begin{sous-entrée}{tɤ-βɣe ɯ-ku ɣɤrndi}{ⓔɣɤrndiⓢ2ⓝtɤ-βɣe ɯ-ku ɣɤrndi} 
\classe{np,np,vt} 
\begin{définition}\pfra{recueillir un orphelin}\end{définition}
\begin{définition}\pcmn{收养孤儿}\end{définition}
\begin{exemple}\pjya{a-βɣe ɯ-ku kɤ-tɯ-ɣɤrndi-t ŋu}\hspace{5pt}\pcmn{你在我最痛苦的时候收养了我这个孤儿}\end{exemple}\end{sous-entrée}

\begin{sous-entrée}{ʑɣɤɣɤrndi}{ⓔɣɤrndiⓢ2ⓝʑɣɤɣɤrndi} 
\classe{vi} \end{sous-entrée}

\begin{définition}\pfra{se calmer}\end{définition}
\begin{définition}\pcmn{镇静}\end{définition}
\begin{exemple}\pjya{nɤʑo kɤ-ʑɣɤɣɤrndi ɲɯ-ra}\hspace{5pt}\pcmn{你要镇静下来}\end{exemple}\end{entrée}

\begin{entrée}{ɣɤrɲɟi}{₂}{ⓔɣɤrɲɟiⓗ2} 
\classe{vt}
\classe{vs}  
\grammaire{caus}
\grammaire{facil} \paradigme{dir}{nɯ-}\paradigme{dir}{pɯ-}\paradigme{dir}{thɯ-}
\begin{définition}\pfra{allonger}\end{définition}
\begin{définition}\pcmn{弄长}\end{définition}
\begin{exemple}\pjya{jiɕqha tɤ-ri nɯ nɯ-ɣɤrɲɟi-t-a}\hspace{5pt}\pcmn{我把这根线拉长了}\end{exemple}
\begin{exemple}\pjya{si kɤ-ʁndzɤr pɯ-ɣɤrɲɟi-t-a}\hspace{5pt}\pcmn{我把木头锯得太长了}\end{exemple}
\begin{exemple}\pjya{tɯ-ŋga nɯ-qrɯ-t-a tɕe nɯ-ɣɤrɲɟi-t-a}\hspace{5pt}\pcmn{我把衣服剪得太长了}\end{exemple}
\begin{exemple}\pjya{ɯ-mke cho-ɣɤrɲɟi}\hspace{5pt}\pcmn{他伸了脖子}\end{exemple}
\begin{sous-entrée}{ɣɤrɲɟi}{ⓔɣɤrɲɟiⓝɣɤrɲɟi}\end{sous-entrée}

\paradigme{dir}{nɯ-}
\begin{définition}\pfra{s'allonger facilement}\end{définition}
\begin{définition}\pcmn{容易变长}\end{définition}\relationsémantique{参考}{\lien{ⓔrɲɟi}{rɲɟi}}\end{entrée}

\begin{entrée}{ɣɤrɲɯɣrɲɯɣ}{}{ⓔɣɤrɲɯɣrɲɯɣ} 
\classe{vi} \paradigme{dir}{tɤ-}\sens{1}
\begin{définition}\pfra{ramper}\end{définition}
\begin{définition}\pcmn{蠕动}\end{définition}
\begin{exemple}\pjya{qapri ɲɯ-ɣɤrɲɯɣrɲɯɣ nɯ-ari}\hspace{5pt}\pcmn{蛇蠕动着过去了}\end{exemple}\sens{2}
\begin{définition}\pfra{percer rapidement}\end{définition}
\begin{définition}\pcmn{钻得很快}\end{définition}
\begin{exemple}\pjya{mkhɯrlɤmnɯ ɲɯ-ɣɤrɲɯrɲɯɣ}\hspace{5pt}\pcmn{钻子在钻}\end{exemple}
\begin{exemple}\pjya{ɕomskrɯt nɯ ŋgɤm ɯ-taʁ kɤ-sɤtsa-t-a tɕe, ɲɯ-nɤmpi tɕe, ɲɯ-ɣɤrɲɯɣrɲɯɣ ʑo lɤ-ari}\hspace{5pt}\pcmn{我把铁丝插在土墙上,因为很软,所以钻得很快}\end{exemple}\end{entrée}

\begin{entrée}{ɣɤrŋa}{}{ⓔɣɤrŋa} 
\classe{vs}  
\grammaire{denom} 
\begin{définition}\pfra{être possible}\end{définition}
\begin{définition}\pcmn{有……的可能,有……的危险}\end{définition}
\begin{exemple}\pjya{khɤxtu ɯ-ndo ʑo ma-thɯ-tɯ-ɕe ma kɯ-ɤtɤr ɲɯ-ɣɤrŋa}\hspace{5pt}\pcmn{你不要靠近房背的边缘,有掉下去的危险}\end{exemple}
\begin{exemple}\pjya{tɯ-mɯ chɤ-ɲɟɯr tɕe, zdɯm kɯ-ɲaʁ jo-ɣi tɕe, tɯ-mɯ kɤ-lɤt ɲɯ-ɣɤrŋa}\hspace{5pt}\pcmn{变天了,来了乌云,有可能会下雨}\end{exemple}
\begin{exemple}\pjya{nɤ-jaʁ kɤ-xtsɯɣ ɲɯ-ɣɤrŋa}\hspace{5pt}\pcmn{你有切到手的危险}\end{exemple}\relationsémantique{参考}{\lien{ⓔtɯ-rŋa}{tɯ-rŋa}}\end{entrée}

\begin{entrée}{ɣɤro}{}{ⓔɣɤro} 
\classe{vt}  
\grammaire{caus} \paradigme{dir}{tɤ-}\paradigme{dir}{nɯ-}
\begin{définition}\pfra{ajouter, accomplir sa tâche au delà des exigences}\end{définition}
\begin{définition}\pcmn{增加;完成任务超过标准}\end{définition}
\begin{exemple}\pjya{ɯʑo kɯ ta-ɣɤro}\hspace{5pt}\pcmn{他增加了}\end{exemple}
\begin{exemple}\pjya{jiɕqha ɯ-khrɤt nɯ staʁ tɤ-ɣɤro-t-a}\hspace{5pt}\pcmn{我做得比刚才规定的多一些}\end{exemple}
\begin{exemple}\pjya{ɯʑo kɯ a-sci ɲɯ-khɤm ra, ri nɯ staʁ ta-ɣɤro}\hspace{5pt}\pcmn{他还给我的那些,还多了一点}\end{exemple}
\begin{exemple}\pjya{a-phoʁ ta-ɣɤro-nɯ}\hspace{5pt}\pcmn{他们给我增加了工资}\end{exemple}
\begin{exemple}\pjya{a-ma nɯ-ɣɤro-t-a}\hspace{5pt}\pcmn{我完成任务超过标准}\end{exemple}\end{entrée}

\begin{entrée}{ɣɤroʁroʁ}{}{ⓔɣɤroʁroʁ} 
\classe{vi} 
\begin{définition}\pfra{couler sans arrêt}\end{définition}
\begin{définition}\pcmn{不停地流}\end{définition}
\begin{exemple}\pjya{tɤ-se ɲɯ-ɣɤroʁroʁ ʑo}\hspace{5pt}\pcmn{血流个不停}\end{exemple}\end{entrée}

\begin{entrée}{ɣɤrphɤrphɤβ}{}{ⓔɣɤrphɤrphɤβ} 
\classe{vs} \paradigme{dir}{tɤ-}
\begin{définition}\pfra{battre des ailes}\end{définition}
\begin{définition}\pcmn{拍打翅膀}\end{définition}
\begin{exemple}\pjya{tsɯʁot ɲɯ-ɣɤrphɤrphɤβ}\hspace{5pt}\pcmn{野鸡在拍翅膀发出声音}\end{exemple}
\begin{exemple}\pjya{ʁdɯskɤr ɲɯ-ɣɤrphɤrphɤβ}\hspace{5pt}\pcmn{旗子在飘动,发出声音}\end{exemple}\relationsémantique{参考}{\lien{ⓔʁarphɤβ}{ʁarphɤβ}}\relationsémantique{参考}{\lien{ⓔnɤʁarphɤβ}{nɤʁarphɤβ}}
\begin{sous-entrée}{sɤrphɤrphɤβ}{ⓔɣɤrphɤrphɤβⓝsɤrphɤrphɤβ} 
\classe{vt} 
\begin{définition}\pfra{battre des ailes}\end{définition}
\begin{définition}\pcmn{拍打翅膀}\end{définition}
\begin{exemple}\pjya{kumpɣa kɯ ɯ-ʁar ɲɯ-sɤrphɤrphɤβ}\hspace{5pt}\pcmn{鸡在拍打翅膀}\end{exemple}\end{sous-entrée}

\end{entrée}

\begin{entrée}{ɣɤrpi}{}{ⓔɣɤrpi} 
\classe{vt}  
\grammaire{denom} \sens{1}\paradigme{dir}{nɯ-}
\begin{définition}\pfra{exécuter une cérémonie religieuse chez soi}\end{définition}
\begin{définition}\pcmn{请和尚来家里念经}\end{définition}
\begin{exemple}\pjya{jisŋi jiʑo jikha ɲɯ-ɣɤrpi-j ŋu}\hspace{5pt}\pcmn{今天我们请了和尚到家里念经}\end{exemple}\sens{2}\paradigme{dir}{tɤ-}
\begin{définition}\pfra{frapper}\end{définition}
\begin{définition}\pcmn{打(比喻和尚念经的时候打鼓)}\end{définition}
\begin{exemple}\pjya{nɤ-stu tɤ-fse ma ta-ɣɤrpi}\hspace{5pt}\pcmn{你规矩一点,不然我就打你(对小孩子说的)}\end{exemple}
\begin{exemple}\pjya{paχtsa nɯ aʑo nɯ-ɣɤrpi-t-a}\hspace{5pt}\pcmn{我把小猪打死了}\end{exemple}\relationsémantique{参考}{\lien{ⓔtɤ-rpi}{tɤ-rpi}}\end{entrée}

\begin{entrée}{ɣɤrqhoŋloŋ}{}{ⓔɣɤrqhoŋloŋ} 
\classe{vi} \paradigme{dir}{tɤ-}
\begin{définition}\pfra{émettre un bruit (objets durs qui se cognent)}\end{définition}
\begin{définition}\pcmn{硬的东西相撞的时候发出声音}\end{définition}
\begin{exemple}\pjya{atu ra tshitsuku ɲɯ-ɤz-nɤma-nɯ tɕe tɯrnda ɯ-taʁ ɲɯ-ɣɤrqhoŋloŋ-nɯ ʑo}\hspace{5pt}\pcmn{上面的那些人不知道在做什么,在地板上很吵}\end{exemple}\end{entrée}

\begin{entrée}{ɣɤrqhoʁrqhoʁ}{}{ⓔɣɤrqhoʁrqhoʁ} 
\classe{vs}  
\grammaire{deidph} \paradigme{dir}{tɤ-}\paradigme{dir}{tɤ-}\paradigme{dir}{pɯ-}
\begin{définition}\pfra{bruit de choc d'un objet dur sur une surface dure}\end{définition}
\begin{définition}\pcmn{硬的东西敲到木板上时发出声音}\end{définition}
\begin{définition}\pfra{frapper (objet dur)}\end{définition}
\begin{définition}\pcmn{敲(硬的东西)}\end{définition}
\begin{définition}\pfra{tirer au fusil sans arrêt}\end{définition}
\begin{définition}\pcmn{不停地打枪}\end{définition}
\begin{exemple}\pjya{ʑɴɢɯloʁ pɯ-nɯɕlɯɣ-a tɕe, ɲɯ-ɣɤrqhoʁrqhoʁ ʑo}\hspace{5pt}\pcmn{我失手,把核桃掉到地上发出声音}\end{exemple}
\begin{exemple}\pjya{tʂha ɲɯ-ɤla ɲɯ-ɣɤrqhoʁrqhoʁ ʑo}\hspace{5pt}\pcmn{茶开了发出沸腾的声音}\end{exemple}
\begin{exemple}\pjya{a-ku ta-sɤrqhoʁrqhoʁ}\hspace{5pt}\pcmn{他敲了我的头}\end{exemple}
\begin{exemple}\pjya{ɕɤmɯɣdɯ ɲɯ-ɤz-nɯrqhoʁ ʑo ɲɯ-ɤsɯ-lɤt}\hspace{5pt}\pcmn{他啪啪地射枪}\end{exemple}
\begin{sous-entrée}{sɤrqhoʁrqhoʁ}{ⓔɣɤrqhoʁrqhoʁⓝsɤrqhoʁrqhoʁ} 
\classe{vt} \end{sous-entrée}

\begin{sous-entrée}{nɯrqhoʁ}{ⓔɣɤrqhoʁrqhoʁⓝnɯrqhoʁ} 
\classe{vi} \end{sous-entrée}

\end{entrée}

\begin{entrée}{ɣɤrqhɯβrqhɯβ/\variante{ɣɤrqhɯrqhɯβ}}{}{ⓔɣɤrqhɯβrqhɯβ} 
\classe{vi} 
\begin{définition}\pfra{bruit d'objets durs et secs qui s'entrechoquent}\end{définition}
\begin{définition}\pcmn{又干又硬的东西相撞发出声音}\end{définition}
\begin{exemple}\pjya{tɯɲɤt thɯ-ɣe tɕe, rdɤstaʁ ra ɲɯ-ɣɤrqhɯrqhɯβ ɲɯ-ɤmɯrpu-nɯ}\hspace{5pt}\pcmn{泥石流下来了,里面的大小石头互相撞击}\end{exemple}\relationsémantique{参考}{\lien{ⓔsɤrqhɯrqhɯβ}{sɤrqhɯrqhɯβ}}\end{entrée}

\begin{entrée}{ɣɤrʁaʁ}{}{ⓔɣɤrʁaʁ} 
\classe{vi}  
\grammaire{denom} \paradigme{dir}{tɤ-}\paradigme{dir}{thɯ-}
\begin{définition}\pfra{chasser}\end{définition}
\begin{définition}\pcmn{打猎}\end{définition}
\begin{exemple}\pjya{kɯ-ɣɤrʁaʁ jɤ-ɕe}\hspace{5pt}\pcmn{你去打猎吧}\end{exemple}
\begin{exemple}\pjya{ji-lɯlu ɲɯ-ɣɤrʁaʁ}\hspace{5pt}\pcmn{我们的猫在捉(老鼠)}\end{exemple}
\begin{sous-entrée}{kɯɣɤrʁaʁ}{ⓔɣɤrʁaʁⓝkɯɣɤrʁaʁ} 
\classe{n} 
\begin{définition}\pfra{chasseur}\end{définition}
\begin{définition}\pcmn{猎人}\end{définition}\relationsémantique{参考}{\lien{ⓔnɤrʁaʁ}{nɤrʁaʁ}}\end{sous-entrée}

\end{entrée}

\begin{entrée}{ɣɤrʁɤβjɤβ}{}{ⓔɣɤrʁɤβjɤβ} 
\classe{vi} \paradigme{dir}{tɤ-}\paradigme{dir}{tɤ-}
\begin{définition}\pfra{se débattre}\end{définition}
\begin{définition}\pcmn{东抓西抓}\end{définition}
\begin{définition}\pfra{faire se débattre}\end{définition}
\begin{définition}\pcmn{使人东抓西抓}\end{définition}
\begin{exemple}\pjya{tɤ-ɣɤrʁɤβjaβ-a}\hspace{5pt}\pcmn{我抓狂了}\end{exemple}
\begin{exemple}\pjya{jiɕqha tsɯʁot nɯnɯ pɤjkhu mɯ-pjɤ-si, ɲɯ-ɣɤrʁɤβjɤβ}\hspace{5pt}\pcmn{那只野鸡还没有死,正在挣扎}\end{exemple}
\begin{exemple}\pjya{pjɤ-ndʐaβ tɕe ɲɯ-ɣɤrʁɤβjɤβ}\hspace{5pt}\pcmn{他摔倒了,正在挣扎}\end{exemple}
\begin{exemple}\pjya{aʑo tɕɣom tɤ-ndzat-a tɕe, pɯ́-wɣ-sɯmkɯt-a tɤ́-wɣ-sɤrʁɤβjaβ-a}\hspace{5pt}\pcmn{我吃了花椒,把我呛到了,让我东抓西抓}\end{exemple}
\begin{sous-entrée}{sɤrʁɤβjɤβ}{ⓔɣɤrʁɤβjɤβⓝsɤrʁɤβjɤβ}\end{sous-entrée}

\end{entrée}

\begin{entrée}{ɣɤrsɯβrsɯβ}{}{ⓔɣɤrsɯβrsɯβ} 
\classe{vs} 
\begin{définition}\pfra{émettre un bruit de froissement}\end{définition}
\begin{définition}\pcmn{发出沙沙声}\end{définition}
\begin{exemple}\pjya{soʁma ɯ-ŋgɯ ɲɯ-ɣɤrsɯβrsɯβ}\hspace{5pt}\pcmn{干草里有沙沙声}\end{exemple}\relationsémantique{参考}{\lien{}{ɣɤrɕɯβrɕɯβ}}\end{entrée}

\begin{entrée}{ɣɤrtaʁ}{}{ⓔɣɤrtaʁ} 
\classe{vt}  
\grammaire{caus} \paradigme{dir}{tɤ-}
\begin{définition}\pfra{rendre suffisant}\end{définition}
\begin{définition}\pcmn{加够;添够;使其具备条件}\end{définition}
\begin{exemple}\pjya{@cai ɯ-spa mɯ́j-rtaʁ tɕe, @yangyu lɤ-tɕat-a tɕe tɤ-ɣɤrtaʁ-a}\hspace{5pt}\pcmn{菜的材料不够,所以我拿了土豆,这样就够了}\end{exemple}
\begin{exemple}\pjya{stoʁ nɯ-phɯt-a tɕe tɤ-ɣɤrtaʁ-a}\hspace{5pt}\pcmn{我拿了胡豆,这样就够了}\end{exemple}\end{entrée}

\begin{entrée}{ɣɤrtɕhɣaʁ}{}{ⓔɣɤrtɕhɣaʁ} 
\classe{vi} \paradigme{dir}{nɯ-}
\begin{définition}\pfra{chicaner}\end{définition}
\begin{définition}\pcmn{计较;找毛病}\end{définition}\relationsémantique{同义词}{\lien{ⓔɣɤtɕɯqaʁ}{ɣɤtɕɯqaʁ}}\relationsémantique{参考}{\lien{ⓔsɤrtɕhɣaʁ}{sɤrtɕhɣaʁ}}\relationsémantique{参考}{\lien{ⓔtɤ-rtɕhɣaʁ,tɕɤt}{tɤ-rtɕhɣaʁ,tɕɤt}}\end{entrée}

\begin{entrée}{ɣɤrthɯɣlɯɣ}{}{ⓔɣɤrthɯɣlɯɣ} 
\classe{vi} 
\begin{définition}\pfra{gêner les gens en bougeant sans arrêt}\end{définition}
\begin{définition}\pcmn{没有目地走动,影响周边的人}\end{définition}
\begin{exemple}\pjya{tɤ-pɤtso ɯ-rkɯ chiz ɲɯ-ɣɤrthɯɣlɯɣ tɕe jɤ-sɯxɕe-t-a}\hspace{5pt}\pcmn{那个小孩子在身边影响我,我叫他走开}\end{exemple}\end{entrée}

\begin{entrée}{ɣɤrti}{}{ⓔɣɤrti} 
\classe{n}  
\grammaire{n.lieu} 
\begin{définition}\pfra{l'un des hameaux de Kamnyu}\end{définition}
\begin{définition}\pcmn{干木鸟的大队之一}\end{définition}\end{entrée}

\begin{entrée}{ɣɤrtshɯɣlɯɣ}{}{ⓔɣɤrtshɯɣlɯɣ} 
\classe{vi} 
\begin{définition}\pfra{être grossier}\end{définition}
\begin{définition}\pcmn{说粗话}\end{définition}
\begin{exemple}\pjya{ɯ-mbrɯ ɲɯ-ŋgɯ tɕe, ɯ-zda ra nɯ-ɕki ɲɯ-ɣɤrtshɯɣlɯɣ}\hspace{5pt}\pcmn{他在生气,对别人说粗话}\end{exemple}\end{entrée}

\begin{entrée}{ɣɤrtɯm}{}{ⓔɣɤrtɯm} 
\classe{vt}  
\grammaire{caus} \paradigme{dir}{tɤ-}
\begin{définition}\pfra{enrouler ensemble les fils}\end{définition}
\begin{définition}\pcmn{缠线}\end{définition}
\begin{exemple}\pjya{tɤ-ri tɤ-ɣɤrtɯm-a}\hspace{5pt}\pcmn{我缠了线}\end{exemple}\end{entrée}

\begin{entrée}{ɣɤrɯβrɯβ}{}{ⓔɣɤrɯβrɯβ} 
\classe{vi}  
\grammaire{deidph} \sens{1}
\begin{définition}\pfra{couler sans arrêt}\end{définition}
\begin{définition}\pcmn{不停地往下流}\end{définition}
\begin{exemple}\pjya{tɯ-ci ɲɯ-ɣɤrɯβrɯβ ɲɯ-nɯɬoʁ}\hspace{5pt}\pcmn{一滴一滴地不停地漏水}\end{exemple}
\begin{exemple}\pjya{tɯ-ci ɲɯ-ɣɤrɯβrɯβ}\hspace{5pt}\pcmn{水不断地往下流}\end{exemple}
\begin{exemple}\pjya{a-tɕɯ ɣɯ ɯ-mci ɲɯ-ɣɤrɯβrɯβ}\hspace{5pt}\pcmn{我儿子在流口水}\end{exemple}\sens{2}
\begin{définition}\pfra{radoter}\end{définition}
\begin{définition}\pcmn{不停地唠叨}\end{définition}
\begin{exemple}\pjya{ma-tɯ-ɣɤrɯβrɯβ}\hspace{5pt}\pcmn{你不要唠叨}\end{exemple}\relationsémantique{同义词}{\lien{ⓔɣɤtʂɯtʂɯt}{ɣɤtʂɯtʂɯt}}\end{entrée}

\begin{entrée}{ɣɤrɯri}{}{ⓔɣɤrɯri} 
\classe{vi} 
\begin{définition}\pfra{souffler sans cesse (vent)}\end{définition}
\begin{définition}\pcmn{(风)吹得很紧,不停地吹}\end{définition}
\begin{exemple}\pjya{qale ɲɯ-ɣɤrɯri ʑo}\hspace{5pt}\pcmn{风吹得很紧}\end{exemple}\end{entrée}

\begin{entrée}{ɣɤrwɤrwɤt}{}{ⓔɣɤrwɤrwɤt} 
\classe{vs} 
\begin{définition}\pfra{bavard}\end{définition}
\begin{définition}\pcmn{说话滔滔不绝}\end{définition}
\begin{exemple}\pjya{ɯ-tɯ-nɯɕmɯrga kɯ ɲɯ-ɣɤrwɤrwɤt ʑo}\hspace{5pt}\pcmn{他很爱说话,说得滔滔不绝}\end{exemple}
\begin{sous-entrée}{sɤrwɤrwɤt}{ⓔɣɤrwɤrwɤtⓝsɤrwɤrwɤt} 
\classe{vt} 
\begin{exemple}\pjya{nɤki nɯ jazɣɯt rcanɯ, ɯ-tɯ-rɯɕmi kɯ ɲɯ-sɤrwɤrwɤt ʑo}\hspace{5pt}\pcmn{那个人到了,讲话讲得滔滔不绝}\end{exemple}
\begin{exemple}\pjya{kɤ-ndɯn ɲɯ-mkhɤz tɕe, ɲɯ-sɤrwɤrwɤt ʑo ɕti}\hspace{5pt}\pcmn{他读得很熟练}\end{exemple}\end{sous-entrée}

\end{entrée}

\begin{entrée}{ɣɤrzɤβrzɤβ}{}{ⓔɣɤrzɤβrzɤβ}\relationsémantique{参考}{\lien{ⓔrzɤβrzɤβ}{rzɤβrzɤβ}}\end{entrée}

\begin{entrée}{ɣɤrʑi}{}{ⓔɣɤrʑi}\relationsémantique{参考}{\lien{ⓔrʑi}{rʑi}}\end{entrée}

\begin{entrée}{ɣɤrʑo}{}{ⓔɣɤrʑo} 
\classe{vt} \paradigme{dir}{pɯ-}\paradigme{dir}{nɯ-}
\begin{définition}\pfra{ne pas exiger que qqn repaie sa dette}\end{définition}
\begin{définition}\pcmn{免去(别人欠的钱)}\end{définition}
\begin{exemple}\pjya{nɯ-ɣɤrʑo-t-a}\hspace{5pt}\pcmn{我没有要他还钱}\end{exemple}
\begin{exemple}\pjya{nɤʑo kɯ aʑo pɯ-kɯ-ɣɤrʑo-a}\hspace{5pt}\pcmn{你没有要我还钱}\end{exemple}
\begin{exemple}\pjya{jiɕqha nɯ kɯ a-sci pɕawtsɯ sqɯ-mpɕar ɲɯ-khɤm pɯ-ra ri, kɯmŋu-mpɕar ma mɯ-nɯ-mɟat-a pɯ-ɣɤrʑo-t-a}\hspace{5pt}\pcmn{他本来应该给我十块钱,但是我只拿到了五块,我没有要他还钱}\end{exemple}
\begin{exemple}\pjya{jiɕqha nɯ kɯ tɤɕi sqɯ-tɯrpa ɲɯ-khɤm pɯ-ra ri, kɯmŋu-tɯrpa ma mɯ-nɯ-mɟa-t-a tɕe pɯ-ɣɤrʑo-t-a}\hspace{5pt}\pcmn{他本来应该给我十斤青稞,但是我只拿到了五斤,我没有要他还钱}\end{exemple}
\begin{exemple}\pjya{ɲɯ-kɯ-ɣɤrʑo-a ɯ́-jɤɣ?}\hspace{5pt}\pcmn{求你免我的债}\end{exemple}
\begin{exemple}\pjya{kɯki laχtɕha ki ɯ-phɯ khro tsa ɲɯ-βze ɕti ri, tsuku ɲɯ-ta-ɣɤrʑo jɤɣ}\hspace{5pt}\pcmn{这个东西价钱有点高,可以给你降低一点}\end{exemple}\end{entrée}

\begin{entrée}{ɣɤrʑɯɣlɯɣ}{}{ⓔɣɤrʑɯɣlɯɣ}\relationsémantique{参考}{\lien{ⓔrʑɯɣrʑɯɣ}{rʑɯɣrʑɯɣ}}\end{entrée}

\begin{entrée}{ɣɤʁjɤβlɤβ}{}{ⓔɣɤʁjɤβlɤβ} 
\classe{vi} 
\begin{définition}\pfra{agir en cachette}\end{définition}
\begin{définition}\pcmn{偷偷摸摸;鬼鬼祟祟}\end{définition}
\begin{exemple}\pjya{jiɕqha nɯ kɯ-ɣɤʁjɤβlɤβ ci ɲɯ-ŋu}\hspace{5pt}\pcmn{那个(人)是个鬼鬼祟祟的人}\end{exemple}
\begin{exemple}\pjya{ɯʑo ɲɯ-ɣɤʁjɤβlɤβ tɕe, ɯ-stu mɤ-nɤme}\hspace{5pt}\pcmn{他这样鬼鬼祟祟的样子,不会做什么好事}\end{exemple}\relationsémantique{参考}{\lien{ⓔɣɤqhɤβjɤβ}{ɣɤqhɤβjɤβ}}\end{entrée}

\begin{entrée}{ɣɤʁnɤt}{}{ⓔɣɤʁnɤt}\relationsémantique{参考}{\lien{ⓔʁnɤt}{ʁnɤt}}\end{entrée}

\begin{entrée}{ɣɤʁre}{}{ⓔɣɤʁre} 
\classe{vs}  
\grammaire{caus} \paradigme{dir}{tɤ-}
\begin{définition}\pfra{être respecté}\end{définition}
\begin{définition}\pcmn{受人尊重}\end{définition}
\begin{exemple}\pjya{ɲɯ-ɣɤʁre (=ɯ-ʁre)}\hspace{5pt}\pcmn{他受人尊重}\end{exemple}
\begin{exemple}\pjya{jiɕqha nɯ kɯ-ɣɤʁre ci ŋu}\hspace{5pt}\pcmn{他是一个受尊重的人}\end{exemple}\relationsémantique{同义词}{\lien{ⓔŋgro}{ŋgro}}\relationsémantique{参考}{\lien{ⓔsaʁre}{saʁre}}\relationsémantique{参考}{\lien{ⓔnaʁre}{naʁre}}\relationsémantique{参考}{\lien{ⓔɯ-ʁre}{ɯ-ʁre}}
\begin{sous-entrée}{zɣɤʁre}{ⓔɣɤʁreⓝzɣɤʁre} 
\classe{vt} \end{sous-entrée}

\begin{définition}\pfra{respecter}\end{définition}
\begin{définition}\pcmn{尊重}\end{définition}
\begin{exemple}\pjya{tu-kɯ-zɣɤʁre-a}\hspace{5pt}\pcmn{你尊重我}\end{exemple}
\begin{sous-entrée}{azɣɤʁrɯʁre}{ⓔɣɤʁreⓝazɣɤʁrɯʁre} 
\classe{vi}  
\grammaire{refl} 
\begin{définition}\pfra{se respecter les uns les autres}\end{définition}
\begin{définition}\pcmn{互相尊重}\end{définition}
\begin{exemple}\pjya{tɕiʑo ʑɤŋgɯz azɣɤʁrɯʁre-tɕi}\hspace{5pt}\pcmn{我们俩互相尊重}\end{exemple}\end{sous-entrée}

\end{entrée}

\begin{entrée}{ɣɤʁrɯ}{}{ⓔɣɤʁrɯ} 
\classe{vi}  
\grammaire{denom} \paradigme{dir}{nɯ-}\paradigme{dir}{lɤ-}
\begin{définition}\pfra{faire des pousses}\end{définition}
\begin{définition}\pcmn{发芽(种子)}\end{définition}
\begin{exemple}\pjya{jiɕqha qaj nɯ tɯ-mɯ kɯ pjɤ-χtɕi tɕe ɲɤ-ɣɤʁrɯ}\hspace{5pt}\pcmn{雨把小麦打湿了就生芽了}\end{exemple}\relationsémantique{参考}{\lien{ⓔta-ʁrɯ}{ta-ʁrɯ}}\end{entrée}

\begin{entrée}{ɣɤʁrɯqa}{}{ⓔɣɤʁrɯqa} 
\classe{vi} \paradigme{dir}{pɯ-}
\begin{définition}\pfra{chicaner}\end{définition}
\begin{définition}\pcmn{计较;反驳}\end{définition}\relationsémantique{参考}{\lien{ⓔsɤtɕɯqaʁ}{sɤtɕɯqaʁ}}\end{entrée}

\begin{entrée}{ɣɤʁzɤβ}{}{ⓔɣɤʁzɤβ} 
\classe{vs} 
\begin{définition}\pfra{grandiose}\end{définition}
\begin{définition}\pcmn{隆重}\end{définition}
\begin{exemple}\pjya{tɤʁaʁ ɲɯ-ɣɤʁzɤβ}\hspace{5pt}\pcmn{聚会很隆重}\end{exemple}
\begin{exemple}\pjya{khɤjhwi ɲɯ-ɣɤʁzɤβ}\hspace{5pt}\pcmn{开会很隆重}\end{exemple}\étymologie{gzab}\end{entrée}

\begin{entrée}{ɣɤsa}{}{ⓔɣɤsa}\relationsémantique{参考}{\lien{ⓔsa}{sa}}\end{entrée}

\begin{entrée}{ɣɤsɤmbrɯ}{}{ⓔɣɤsɤmbrɯ} 
\classe{vi} 
\begin{définition}\pfra{s'énerver facilement}\end{définition}
\begin{définition}\pcmn{容易生气}\end{définition}
\begin{exemple}\pjya{jiɕqha nɯ wuma kɯ-ɣɤsɤmbrɯ ci ɲɯ-ŋu}\hspace{5pt}\pcmn{这个人脾气不好,容易生气}\end{exemple}\relationsémantique{参考}{\lien{ⓔsɤmbrɯ}{sɤmbrɯ}}\end{entrée}

\begin{entrée}{ɣɤscɤscɤt}{}{ⓔɣɤscɤscɤt} 
\classe{vi} 
\begin{définition}\pfra{(frapper, crier) à toute vitesse}\end{définition}
\begin{définition}\pcmn{(叫、拍)得很快}\end{définition}
\begin{exemple}\pjya{taχphe ɲɯ-sɤscɤscɤt ʑo ta-lɤt}\hspace{5pt}\pcmn{他拍手拍得很快}\end{exemple}\end{entrée}

\begin{entrée}{ɣɤscraʁscraʁ}{}{ⓔɣɤscraʁscraʁ}\relationsémantique{参考}{\lien{ⓔscraʁscraʁ}{scraʁscraʁ}}\end{entrée}

\begin{entrée}{ɣɤsmi}{₁₂}{ⓔɣɤsmiⓗ1ⓗ2} 
\classe{vt}
\classe{vs}  
\grammaire{caus} \paradigme{dir}{kɤ-}
\begin{définition}\pfra{faire cuire}\end{définition}
\begin{définition}\pcmn{煮熟}\end{définition}
\begin{exemple}\pjya{jaŋjy kɤ-sqa-t-a ri mɯ́j-smi tɕe, mɤʑɯ kɤ-ɣɤsmi-t-a}\hspace{5pt}\pcmn{我煮了土豆没有煮熟,所以又煮了一次}\end{exemple}
\begin{exemple}\pjya{ndzɤtshi kɤ-βzu kɤ-ɣɤsmi ɲɯ-ra}\hspace{5pt}\pcmn{做饭的时候要煮熟}\end{exemple}
\begin{exemple}\pjya{kɤ-sqa-t-a tɕe, kɤ-ɣɤsmi-t-a}\hspace{5pt}\pcmn{我煮熟了}\end{exemple}
\begin{sous-entrée}{ɣɤsmi}{ⓔɣɤsmiⓗ1ⓝɣɤsmi}\end{sous-entrée}

\begin{définition}\pfra{facile à cuire}\end{définition}
\begin{définition}\pcmn{容易煮}\end{définition}\relationsémantique{参考}{\lien{ⓔsmiⓗ1}{smi₁}}\end{entrée}

\begin{entrée}{ɣɤsna}{}{ⓔɣɤsna} 
\classe{vt}  
\grammaire{caus} \sens{1}\paradigme{dir}{tɤ-}
\begin{définition}\pfra{réparer}\end{définition}
\begin{définition}\pcmn{修理}\end{définition}
\begin{exemple}\pjya{qaʁ ɯ-jɯ mɯ́j-sna tɕe nɯ-βzu-t-a tɕe tɤ-ɣɤsna-t-a}\hspace{5pt}\pcmn{锄头的把子不好,我把它修好了}\end{exemple}
\begin{exemple}\pjya{a-khɯtsa ɲɤ-spoʁ tɕe, na-sti tɕe ta-ɣɤsna}\hspace{5pt}\pcmn{我的碗破了个洞,他塞住了就可以用了}\end{exemple}\sens{2}
\begin{définition}\pfra{faire du thé trop fort}\end{définition}
\begin{définition}\pcmn{熬得太浓}\end{définition}
\begin{exemple}\pjya{tʂha kɤ-ta-t-a tɕe kɤ-ɣɤsna-t-a}\hspace{5pt}\pcmn{我熬了茶熬得很浓}\end{exemple}\end{entrée}

\begin{entrée}{ɣɤsŋaβ}{}{ⓔɣɤsŋaβ} 
\classe{vs} \paradigme{dir}{tɤ-}\paradigme{dir}{thɯ-}
\begin{définition}\pfra{étouffant}\end{définition}
\begin{définition}\pcmn{闷热}\end{définition}
\begin{exemple}\pjya{jisŋi ɲɯ-ɣɤsŋaβ}\hspace{5pt}\pcmn{今天很闷热}\end{exemple}
\begin{exemple}\pjya{kha ɯ-ŋgɯ smi chɯ́-wɣ-βlɯ tɕe ɲɯ-ɣɤsŋaβ}\hspace{5pt}\pcmn{屋子里烧了火,很闷热}\end{exemple}\relationsémantique{同义词}{\lien{ⓔɣɯtshɤdɯɣ}{ɣɯtshɤdɯɣ}}\end{entrée}

\begin{entrée}{ɣɤsoŋsoŋ}{}{ⓔɣɤsoŋsoŋ} 
\classe{vs} \paradigme{dir}{tɤ-}
\begin{définition}\pfra{émettre un bruit d'ébullition}\end{définition}
\begin{définition}\pcmn{发出沸腾的声音}\end{définition}
\begin{exemple}\pjya{tɤ-lu tɤ-ala tɕe, tu-ɣɤsoŋsoŋ ʑo tu-mbuz ŋu, wuma ʑo ɣɤji}\hspace{5pt}\pcmn{牛奶烧开的时候发出沸腾的声音,速度很快地溢出来}\end{exemple}
\begin{exemple}\pjya{tɯ-ci to-ɣɤsoŋsoŋ ʑo to-mbuz}\hspace{5pt}\pcmn{水发出沸腾声溢出来了}\end{exemple}\relationsémantique{参考}{\lien{ⓔɣɤsthɯsthoŋ}{ɣɤsthɯsthoŋ}}\end{entrée}

\begin{entrée}{ɣɤstaŋlaŋ}{}{ⓔɣɤstaŋlaŋ} 
\classe{vi} 
\begin{définition}\pfra{sautiller}\end{définition}
\begin{définition}\pcmn{跳来跳去(大的圆形的东西)}\end{définition}
\begin{exemple}\pjya{rɯdaʁ ra ɲɯ-ɣɤstaŋlaŋ-nɯ ʑo ɲɯ-phɣo-nɯ ɲɯ-ŋu}\hspace{5pt}\pcmn{野兽一跳一跳地逃跑}\end{exemple}
\begin{exemple}\pjya{@piqiu ɲɯ-ɣɤstaŋlaŋ}\hspace{5pt}\pcmn{皮球在弹来弹去}\end{exemple}\relationsémantique{同义词}{\lien{ⓔstɯrstɯr}{stɯrstɯr}}\relationsémantique{同义词}{\lien{ⓔstɤjstɤj}{stɤjstɤj}}\end{entrée}

\begin{entrée}{ɣɤstɤjlɤj}{}{ⓔɣɤstɤjlɤj}\relationsémantique{参考}{\lien{ⓔstɤjstɤj}{stɤjstɤj}}\end{entrée}

\begin{entrée}{ɣɤsthɯsthoŋ}{}{ⓔɣɤsthɯsthoŋ} 
\classe{vs} \paradigme{dir}{tɤ-}
\begin{définition}\pfra{bruit de l'eau qui bout}\end{définition}
\begin{définition}\pcmn{发出沸腾的声音}\end{définition}
\begin{exemple}\pjya{tɯ-ci tala tɕe ɲɯɣɤsthɯtshoŋ ʑo ɲɯ-mbuz}\hspace{5pt}\pcmn{水发出沸腾的声音就溢出来了}\end{exemple}\relationsémantique{参考}{\lien{ⓔɣɤsoŋsoŋ}{ɣɤsoŋsoŋ}}\end{entrée}

\begin{entrée}{ɣɤsthɯsthrɯβ}{}{ⓔɣɤsthɯsthrɯβ} 
\classe{vi} 
\begin{sous-entrée}{sɤsthɯsthrɯβ}{ⓔɣɤsthɯsthrɯβⓝsɤsthɯsthrɯβ} 
\classe{vt} 
\begin{définition}\pfra{se moucher le nez bruyamment}\end{définition}
\begin{définition}\pcmn{擤鼻涕的时候发出声音}\end{définition}
\begin{exemple}\pjya{ɯ-ɕnaβ ta-sɤsthɯsthrɯβ ʑo tha-pɕiz}\hspace{5pt}\pcmn{他擤了鼻涕,发出了很响的声音}\end{exemple}\relationsémantique{参考}{\lien{ⓔsthrɯβ}{sthrɯβ}}\end{sous-entrée}

\end{entrée}

\begin{entrée}{ɣɤstɯrlɯr}{}{ⓔɣɤstɯrlɯr}\relationsémantique{参考}{\lien{ⓔstɯrstɯr}{stɯrstɯr}}\end{entrée}

\begin{entrée}{ɣɤsɯβsɯβ}{}{ⓔɣɤsɯβsɯβ}\relationsémantique{参考}{\lien{ⓔsɯβsɯβ}{sɯβsɯβ}}\end{entrée}

\begin{entrée}{ɣɤsɯɣsɯɣ}{}{ⓔɣɤsɯɣsɯɣ} 
\classe{vi}  
\grammaire{deidph} \paradigme{dir}{nɯ-}
\begin{définition}\pfra{gigoter}\end{définition}
\begin{définition}\pcmn{不停地乱动,令人讨厌}\end{définition}
\begin{exemple}\pjya{tɤ-pɤtso mɯ́j-khɯ tɕe ɲɯ-ɣɤsɯɣsɯɣ ʑo}\hspace{5pt}\pcmn{小孩子不听话,不停地乱动}\end{exemple}
\begin{exemple}\pjya{ma-tɯ-ɣɤsɯɣsɯɣ kɯ kɤ-rɤʑi}\hspace{5pt}\pcmn{不要乱动}\end{exemple}\end{entrée}

\begin{entrée}{ɣɤʂɲɯɣlɯɣ}{}{ⓔɣɤʂɲɯɣlɯɣ} 
\classe{vi} 
\begin{définition}\pfra{gigoter dans tous les sens}\end{définition}
\begin{définition}\pcmn{动来动去}\end{définition}
\begin{exemple}\pjya{ma-tɯ-ɣɤʂɲɯɣlɯɣ}\hspace{5pt}\pcmn{你不要动来动去}\end{exemple}\relationsémantique{参考}{\lien{ⓔʂɲɯɣnɤʂɲɯɣ}{ʂɲɯɣnɤʂɲɯɣ}}
\begin{sous-entrée}{sɤʂɲɯɣlɯɣ}{ⓔɣɤʂɲɯɣlɯɣⓝsɤʂɲɯɣlɯɣ} 
\classe{vt} 
\begin{définition}\pfra{gigoter dans tous les sens}\end{définition}
\begin{définition}\pcmn{动来动去}\end{définition}\end{sous-entrée}

\end{entrée}

\begin{entrée}{ɣɤʂχaβʂχaβ}{}{ⓔɣɤʂχaβʂχaβ} 
\classe{vi} 
\begin{définition}\pfra{émettre un bruit de froissement}\end{définition}
\begin{définition}\pcmn{发出沙沙声}\end{définition}
\begin{exemple}\pjya{ɲɯ-ɣɤʂχaβʂχaβ}\hspace{5pt}\pcmn{(皮子)发出沙沙声}\end{exemple}\relationsémantique{参考}{\lien{ⓔɣɤʂχɯʂχɯβ}{ɣɤʂχɯʂχɯβ}}\end{entrée}

\begin{entrée}{ɣɤʂχɯʂχɯβ}{}{ⓔɣɤʂχɯʂχɯβ}\relationsémantique{参考}{\lien{ⓔsɤʂχɯʂχɯβ}{sɤʂχɯʂχɯβ}}\end{entrée}

\begin{entrée}{ɣɤtu}{}{ⓔɣɤtu} 
\classe{vt} \paradigme{dir}{tɤ-}
\begin{définition}\pfra{faire exister}\end{définition}
\begin{définition}\pcmn{使其存在}\end{définition}
\begin{exemple}\pjya{nɤ-skɤt tɤ-ɣɤte}\hspace{5pt}\pcmn{你要发出声音}\end{exemple}
\begin{exemple}\pjya{nɤ-χsoŋχsɤz tɤ-ɣɤte}\hspace{5pt}\pcmn{你要振作精神}\end{exemple}
\begin{exemple}\pjya{nɤ-sŋɯro tɤ-ɣɤte}\hspace{5pt}\pcmn{你要发出呼吸声}\end{exemple}
\begin{exemple}\pjya{nɤ-kɤ-ntɕhoz kɯ-kɯ-ra nɯ tɤ-ɣɤtu-t-a}\hspace{5pt}\pcmn{我给准备了你所有要用的东西}\end{exemple}\relationsémantique{反义词}{\lien{ⓔɣɤme}{ɣɤme}}\relationsémantique{参考}{\lien{ⓔtu}{tu}}\end{entrée}

\begin{entrée}{ɣɤtaʁ}{}{ⓔɣɤtaʁ}\relationsémantique{参考}{\lien{ⓔtaʁⓗ2}{taʁ₂}}\end{entrée}

\begin{entrée}{ɣɤtɕa}{}{ⓔɣɤtɕa} 
\classe{vi}  
\grammaire{refl} \paradigme{dir}{pɯ-}\paradigme{dir}{pɯ-}\paradigme{dir}{pɯ-}
\begin{définition}\pfra{avoir tord}\end{définition}
\begin{définition}\pcmn{错;犯错误}\end{définition}
\begin{définition}\pfra{ne pas être d'accord avec}\end{définition}
\begin{définition}\pcmn{不同意,认为别人是错的}\end{définition}
\begin{exemple}\pjya{pɯ-ɣɤtɕa-a}\hspace{5pt}\pcmn{我错了}\end{exemple}
\begin{exemple}\pjya{ɯʑo kɯ kɯmaʁ tu-ti ɲɯ-ŋu tɕe ɲɯ-ɣɤtɕa}\hspace{5pt}\pcmn{他说得不对,他是错的}\end{exemple}
\begin{exemple}\pjya{aʑo kɯmaʁ to-ti-a tɕe pjɤ-ɣɤtɕa-a}\hspace{5pt}\pcmn{我说得不对,我错了}\end{exemple}
\begin{exemple}\pjya{pɯ-ɣɤtɕa-a tɤ-ti}\hspace{5pt}\pcmn{你要承认自己做错了}\end{exemple}
\begin{exemple}\pjya{nɯ tɤ-tɯt-a ri ɯʑo kɯ pɯ́-wɣ-zɣɤtɕa}\hspace{5pt}\pcmn{他说了那句说,但是他表示不同意}\end{exemple}\relationsémantique{反义词}{\lien{ⓔɣɤŋgi}{ɣɤŋgi}}\relationsémantique{参考}{\lien{ⓔnɯɣɤtɕa}{nɯɣɤtɕa}}
\begin{sous-entrée}{zɣɤtɕa}{ⓔɣɤtɕaⓝzɣɤtɕa} 
\classe{vt} \end{sous-entrée}

\begin{sous-entrée}{ʑɣɤɣɤtɕa/\variante{ʑɣɤzɣɤtɕa}}{ⓔɣɤtɕaⓝʑɣɤɣɤtɕa} 
\classe{vi} \end{sous-entrée}

\begin{définition}\pfra{admettre sa faute}\end{définition}
\begin{définition}\pcmn{承认自己的错}\end{définition}
\begin{exemple}\pjya{pjɤ-ʑɣɤɣɤtɕa}\hspace{5pt}\pcmn{他承认了自己的错误}\end{exemple}\end{entrée}

\begin{entrée}{ɣɤtɕɤt}{}{ⓔɣɤtɕɤt} 
\classe{vt} \sens{1}\paradigme{dir}{tɤ-}
\begin{définition}\pfra{choisir parmi (un homme)}\end{définition}
\begin{définition}\pcmn{从中挑选(人)}\end{définition}
\begin{exemple}\pjya{nɯ-ŋgɯz nɯ tɕu kɯ-mna ʁnɯz tú-wɣ-ɣɤtɕɤt ɲɯ-ra}\hspace{5pt}\pcmn{他们当中要选两个当领导}\end{exemple}\sens{2}\paradigme{dir}{pɯ-}\paradigme{dir}{tɤ-}\paradigme{dir}{tɤ-}
\begin{définition}\pfra{fournir}\end{définition}
\begin{définition}\pcmn{提供}\end{définition}
\begin{définition}\pfra{se porter volontaire pour}\end{définition}
\begin{définition}\pcmn{自告奋勇}\end{définition}
\begin{exemple}\pjya{nɤʑo kha jɤ-ɣi jɤɣ ma nɤ-kɤndza cho nɤ-kɯ-ra pjɯ-ɣɤtɕat-a jɤɣ}\hspace{5pt}\pcmn{你可以到我家来,我可以给你提供食物和你所需要的东西}\end{exemple}
\begin{exemple}\pjya{kɯ-rɤma kɤ-ɕe ɲɯ-ra tɕe, aʑo tɤ-ʑɣɤɣɤtɕat-a}\hspace{5pt}\pcmn{需要人,我自告奋勇去做了}\end{exemple}
\begin{sous-entrée}{ʑɣɤɣɤtɕɤt}{ⓔɣɤtɕɤtⓢ2ⓝʑɣɤɣɤtɕɤt} 
\classe{vi}  
\grammaire{refl} \end{sous-entrée}

\end{entrée}

\begin{entrée}{ɣɤtɕɣɤrtɕɣɤr}{}{ⓔɣɤtɕɣɤrtɕɣɤr}\relationsémantique{参考}{\lien{ⓔtɕɣɤrtɕɣɤr}{tɕɣɤrtɕɣɤr}}\end{entrée}

\begin{entrée}{ɣɤtɕhaʁ}{}{ⓔɣɤtɕhaʁ} 
\classe{vt}  
\grammaire{caus} 
\begin{définition}\pfra{faire diminuer}\end{définition}
\begin{définition}\pcmn{减}\end{définition}\relationsémantique{同义词}{\lien{ⓔsɯxtɕhaʁ}{sɯxtɕhaʁ}}\relationsémantique{参考}{\lien{ⓔtɕhaʁ}{tɕhaʁ}}\end{entrée}

\begin{entrée}{ɣɤtɕhom}{}{ⓔɣɤtɕhom} 
\classe{vt}  
\grammaire{caus} \paradigme{dir}{\_}
\begin{définition}\pfra{être excessif, dépasser la mesure}\end{définition}
\begin{définition}\pcmn{过分;过量;走过头}\end{définition}
\begin{exemple}\pjya{tɤ-}\end{exemple}
\begin{exemple}\pjya{aʑo tɤ-ari-a ri to-ɣɤtɕhom-a}\hspace{5pt}\pcmn{我走过头了}\end{exemple}
\begin{exemple}\pjya{aʑo jiɕqha rɟɤɣi ʁnɯz tɤ-ndzat-a to-ɣɤtɕhom-a}\hspace{5pt}\pcmn{我吃了两坨糌粑,吃多了}\end{exemple}
\begin{exemple}\pjya{cha (kɤ-tshi, ɯ-tshi) ko-ɣɤtɕhom}\hspace{5pt}\pcmn{酒喝多了}\end{exemple}\relationsémantique{参考}{\lien{ⓔtɕhom}{tɕhom}}\end{entrée}

\begin{entrée}{ɣɤtɕhɯβtɕhɯβ}{}{ⓔɣɤtɕhɯβtɕhɯβ}\relationsémantique{参考}{\lien{ⓔtɕhɯβtɕhɯβ}{tɕhɯβtɕhɯβ}}\end{entrée}

\begin{entrée}{ɣɤtɕur}{}{ⓔɣɤtɕur}\relationsémantique{参考}{\lien{ⓔtɕurⓗ1}{tɕur₁}}\end{entrée}

\begin{entrée}{ɣɤtɕɯɣ}{}{ⓔɣɤtɕɯɣ} 
\classe{vi}  
\grammaire{facil} \paradigme{dir}{nɯ-}\paradigme{dir}{nɯ-}
\begin{définition}\pfra{germer (arbre)}\end{définition}
\begin{définition}\pcmn{发芽(树)}\end{définition}
\begin{exemple}\pjya{ʑmbri ɲɤ-ɣɤtɕɯɣ}\hspace{5pt}\pcmn{柳树发芽了}\end{exemple}
\begin{exemple}\pjya{mi ɲɤ-ɣɤtɕɯɣ}\hspace{5pt}\pcmn{杨树发芽了}\end{exemple}
\begin{sous-entrée}{ɣɤɣɤtɕɯɣ}{ⓔɣɤtɕɯɣⓝɣɤɣɤtɕɯɣ} 
\classe{vs} \end{sous-entrée}

\begin{définition}\pfra{germer précocement (arbre)}\end{définition}
\begin{définition}\pcmn{发芽发得早(树)}\end{définition}
\begin{exemple}\pjya{ʑmbri ɲɯ-ɣɤɣɤtɕɯɣ}\hspace{5pt}\pcmn{柳树发芽发得很早}\end{exemple}
\begin{exemple}\pjya{mi ɲɯ-ɣɤɣɤtɕɯɣ}\hspace{5pt}\pcmn{杨树发芽发得很早}\end{exemple}\relationsémantique{参考}{\lien{ⓔtɤ-tɕɯɣ}{tɤ-tɕɯɣ}}\end{entrée}

\begin{entrée}{ɣɤtɕɯqaʁ}{}{ⓔɣɤtɕɯqaʁ}\relationsémantique{参考}{\lien{ⓔsɤtɕɯqaʁ}{sɤtɕɯqaʁ}}\end{entrée}

\begin{entrée}{ɣɤthɤβjɤβ}{}{ⓔɣɤthɤβjɤβ} 
\classe{vi} 
\begin{définition}\pfra{chercher des choses n'importe comment}\end{définition}
\begin{définition}\pcmn{乱找东西(小孩子)}\end{définition}
\begin{exemple}\pjya{ma-tɯ-ɣɤthɤβjɤβ}\hspace{5pt}\pcmn{你不要乱找东西}\end{exemple}\end{entrée}

\begin{entrée}{ɣɤthɤβthɤβ}{}{ⓔɣɤthɤβthɤβ}\relationsémantique{参考}{\lien{ⓔsɤthɤβthɤβ}{sɤthɤβthɤβ}}\end{entrée}

\begin{entrée}{ɣɤthɣɤthɣɤt}{}{ⓔɣɤthɣɤthɣɤt} 
\classe{vi} 
\begin{définition}\pfra{trembler}\end{définition}
\begin{définition}\pcmn{发抖}\end{définition}
\begin{exemple}\pjya{ɲɯ-ngo tɕe ɲɯ-ɣɤthɣɤthɣɤt}\hspace{5pt}\pcmn{他病了,在发抖}\end{exemple}
\begin{sous-entrée}{sɤthɣɤthɣɤt}{ⓔɣɤthɣɤthɣɤtⓝsɤthɣɤthɣɤt} 
\classe{vt} 
\begin{définition}\pfra{gigoter}\end{définition}
\begin{définition}\pcmn{不停地乱动}\end{définition}
\begin{exemple}\pjya{ɯ-mi ɲɯ-sɤthɣɤthɣɤt}\hspace{5pt}\pcmn{他在抖脚}\end{exemple}
\begin{exemple}\pjya{ma-tɯ-sɤthɣɤthɣɤt}\hspace{5pt}\pcmn{你不要不停地乱动}\end{exemple}\relationsémantique{同义词}{\lien{ⓔɣɤndzɯrndzɯr}{ɣɤndzɯrndzɯr}}\end{sous-entrée}

\end{entrée}

\begin{entrée}{ɣɤthɯ}{}{ⓔɣɤthɯ}\relationsémantique{参考}{\lien{ⓔthɯⓗ2}{thɯ₂}}\end{entrée}

\begin{entrée}{ɣɤtsɤngo}{}{ⓔɣɤtsɤngo} 
\classe{vs} 
\begin{définition}\pfra{tomber souvent malade}\end{définition}
\begin{définition}\pcmn{经常生病}\end{définition}
\begin{exemple}\pjya{ɲɯ-ɣɤtsɤngo-a}\hspace{5pt}\pcmn{我经常生病}\end{exemple}\relationsémantique{参考}{\lien{ⓔngo}{ngo}}\end{entrée}

\begin{entrée}{ɣɤtsɣaʁtsɣaʁ}{}{ⓔɣɤtsɣaʁtsɣaʁ} 
\classe{vs}  
\grammaire{deidph} \paradigme{dir}{tɤ-}
\begin{définition}\pfra{pénible à supporter (comme la piqûre d'une aiguille )}\end{définition}
\begin{définition}\pcmn{令人痛、难受(像扎针一样)}\end{définition}
\begin{exemple}\pjya{ɲɯ-mɤrtsaβ ɲɯ-ɣɤtsɣaʁtsɣaʁ}\hspace{5pt}\pcmn{太辣,很难受}\end{exemple}
\begin{exemple}\pjya{ɯ-tɯ-sɤɕke kɯ ɲɯ-ɣɤtsɣaʁtsɣaʁ}\hspace{5pt}\pcmn{太烫,很难受}\end{exemple}\end{entrée}

\begin{entrée}{ɣɤtshu}{}{ⓔɣɤtshu} 
\classe{vt}  
\grammaire{caus} \paradigme{dir}{thɯ-}
\begin{définition}\pfra{faire grossir}\end{définition}
\begin{définition}\pcmn{使变胖}\end{définition}
\begin{exemple}\pjya{paʁ nɯ (stoʁ) khro pɯ-mbi-j tɕe thɯ-ɣɤtshu-j}\hspace{5pt}\pcmn{我们给猪喂了很多胡豆,把它养肥了}\end{exemple}
\begin{exemple}\pjya{chó-wɣ-z-nɯɣmbaβ-a ɕti ma chó-wɣ-ɣɤtshu-a maʁ}\hspace{5pt}\pcmn{(蚊子)令我肿了,没有令我变胖(笑话)}\end{exemple}\relationsémantique{参考}{\lien{ⓔtshu}{tshu}}\end{entrée}

\begin{entrée}{ɣɤtsha}{}{ⓔɣɤtsha}\relationsémantique{参考}{\lien{ⓔɯ-rɕa,tshaⓝɯ-rɕa,ɣɤtsha}{ɯ-rɕa,ɣɤtsha}}\end{entrée}

\begin{entrée}{ɣɤtshoz}{}{ⓔɣɤtshoz}\relationsémantique{参考}{\lien{ⓔtshoz}{tshoz}}\end{entrée}

\begin{entrée}{ɣɤtsri}{}{ⓔɣɤtsri} 
\classe{vt} \paradigme{dir}{pɯ-}\paradigme{dir}{lɤ-}
\begin{définition}\pfra{saler}\end{définition}
\begin{définition}\pcmn{放盐;令……变得更咸}\end{définition}
\begin{exemple}\pjya{pɯ-ɣɤtsri-t-a}\hspace{5pt}\pcmn{我放了盐}\end{exemple}\relationsémantique{参考}{\lien{ⓔtsri}{tsri}}\end{entrée}

\begin{entrée}{ɣɤtsrɯ}{}{ⓔɣɤtsrɯ} 
\classe{vs}  
\grammaire{denom} \paradigme{dir}{nɯ-}\paradigme{dir}{tɤ-}
\begin{définition}\pfra{germer}\end{définition}
\begin{définition}\pcmn{发芽}\end{définition}
\begin{exemple}\pjya{tɤɕi to-ɣɤtsrɯ}\hspace{5pt}\pcmn{青稞发芽了}\end{exemple}
\begin{exemple}\pjya{stoʁ ɲɯ-ɣɤtsrɯ}\hspace{5pt}\pcmn{胡豆在发芽}\end{exemple}\relationsémantique{参考}{\lien{ⓔtɤ-tsrɯ}{tɤ-tsrɯ}}\end{entrée}

\begin{entrée}{ɣɤtsɯr}{}{ⓔɣɤtsɯr} 
\classe{vi}  
\grammaire{denom} \paradigme{dir}{pɯ-}
\begin{définition}\pfra{se fêler, avoir des gerçures}\end{définition}
\begin{définition}\pcmn{裂口;龟裂}\end{définition}
\begin{exemple}\pjya{ki znde ki pjɤ-ɣɤtsɯr}\hspace{5pt}\pcmn{这堵墙裂了}\end{exemple}
\begin{exemple}\pjya{a-jaʁ pjɤ-ɣɤtsɯr}\hspace{5pt}\pcmn{我的手裂了}\end{exemple}\relationsémantique{参考}{\lien{ⓔtɤ-tsɯr}{tɤ-tsɯr}}\end{entrée}

\begin{entrée}{ɣɤtsɯtsrɯɣ}{}{ⓔɣɤtsɯtsrɯɣ} 
\classe{vs}  
\grammaire{deidph} \paradigme{dir}{tɤ-}
\begin{définition}\pfra{grincer (bois)}\end{définition}
\begin{définition}\pcmn{发出尖锐的嘎吱声(木头)}\end{définition}
\begin{exemple}\pjya{tɯ-sta ɲɯ-ɣɤtsɯtsrɯɣ}\hspace{5pt}\pcmn{床在嘎吱嘎吱响}\end{exemple}
\begin{exemple}\pjya{si ɲɯ-ɣɤtsɯtsrɯɣ}\hspace{5pt}\pcmn{木头在嘎吱嘎吱响}\end{exemple}\end{entrée}

\begin{entrée}{ɣɤtʂɤjɤt}{}{ⓔɣɤtʂɤjɤt} 
\classe{vi} \paradigme{dir}{tɤ-}
\begin{définition}\pfra{bouger dans tous les sens (enfant)}\end{définition}
\begin{définition}\pcmn{乱动(孩子)}\end{définition}\end{entrée}

\begin{entrée}{ɣɤtʂhɯtʂhɯt/\variante{ɣɤtʂhɯtʂhɯz}}{}{ⓔɣɤtʂhɯtʂhɯt} 
\classe{vi}  
\grammaire{deidph} 
\begin{définition}\pfra{crépiter (feu)}\end{définition}
\begin{définition}\pcmn{火烧的时候,发出噼啪噼啪声}\end{définition}
\begin{exemple}\pjya{smi ɲɯ-ɣɤtʂhɯtʂhɯz}\hspace{5pt}\pcmn{火噼啪噼啪作响}\end{exemple}\end{entrée}

\begin{entrée}{ɣɤtʂot}{}{ⓔɣɤtʂot} 
\classe{vt} \paradigme{dir}{tɤ-}\paradigme{dir}{\_}
\begin{définition}\pfra{rendre clair}\end{définition}
\begin{définition}\pcmn{弄清楚}\end{définition}
\begin{exemple}\pjya{ɯʑo kɯ pa-ɣɤtʂot}\hspace{5pt}\pcmn{他弄清楚了}\end{exemple}
\begin{exemple}\pjya{jiɕqha tɤ-scoz nɯ pɯ-tɯ-rɤt nɯ mɯ́j-tʂot tɕe aʑo pɯ-ɣɤtʂo-t-a}\hspace{5pt}\pcmn{你字写得不清楚,我把它写清楚了}\end{exemple}
\begin{exemple}\pjya{kɯki tɯrkɤz to-βzu mɯ́j-tʂot tɕe mɤʑɯ kɤ-ɣɤtʂo-t-a}\hspace{5pt}\pcmn{花纹刻得不清楚,我又把它刻清楚了}\end{exemple}\relationsémantique{参考}{\lien{ⓔtʂot}{tʂot}}\end{entrée}

\begin{entrée}{ɣɤtʂɯtʂɯt}{}{ⓔɣɤtʂɯtʂɯt} 
\classe{vi}  
\grammaire{deidph} \paradigme{dir}{tɤ-}
\begin{définition}\pfra{radoter sans arrêt}\end{définition}
\begin{définition}\pcmn{不停地唠叨}\end{définition}
\begin{exemple}\pjya{ma-tɯ-ɣɤtʂɯtʂɯt ma ɲɯ-sɤɣdɯɣ}\hspace{5pt}\pcmn{你不要不停地唠叨,很讨厌}\end{exemple}\relationsémantique{同义词}{\lien{ⓔɣɤrɯβrɯβ}{ɣɤrɯβrɯβ}}\end{entrée}

\begin{entrée}{ɣɤtɯɣ}{}{ⓔɣɤtɯɣ} 
\classe{vt} \sens{1}\paradigme{dir}{thɯ-}\paradigme{dir}{lɤ-}
\begin{définition}\pfra{maintenir fermée en appuyant avec un bâton (porte)}\end{définition}
\begin{définition}\pcmn{(用棍子)把门顶住}\end{définition}
\begin{exemple}\pjya{kɯm thɯ-ɣɤtɯɣ-a (lɤ-ɣɤtɯɣ-a)}\hspace{5pt}\pcmn{我拴了门}\end{exemple}\sens{2}\paradigme{dir}{lɤ-}
\begin{définition}\pfra{soutenir en appuyant avec un bâton}\end{définition}
\begin{définition}\pcmn{用棍子顶住}\end{définition}
\begin{exemple}\pjya{jiɕqha nɯnɯ lɤ-ɣɤtɯɣ tɕe a-mɤ-thɯ-ndʐaβ}\hspace{5pt}\pcmn{你把它顶住,不要让它跌倒}\end{exemple}
\begin{exemple}\pjya{jiɕqha laχtɕha nɯ ndʐaβ ɲɯ-ŋu tɕe, aj lɤ-ɣɤtɯɣ-a}\hspace{5pt}\pcmn{这个东西差一点倒了,我用棍子把它顶住了}\end{exemple}
\begin{sous-entrée}{aɣɤtɯɣ}{ⓔɣɤtɯɣⓢ2ⓝaɣɤtɯɣ} 
\classe{vi}  
\grammaire{pass} 
\begin{exemple}\pjya{kɯm pjɤ-k-ɤɣɤtɯɣ-ci ɕti tɕe, kɤ-cɯ mɯ-pjɤ-khɯ}\hspace{5pt}\pcmn{门是(被棍子)顶着的,打不开}\end{exemple}\end{sous-entrée}

\end{entrée}

\begin{entrée}{ɣɤtɯt}{}{ⓔɣɤtɯt}\relationsémantique{参考}{\lien{ⓔtɯt}{tɯt}}\end{entrée}

\begin{entrée}{ɣɤwu}{}{ⓔɣɤwu} 
\classe{vi} \paradigme{dir}{nɯ-}\sens{1}
\begin{définition}\pfra{pleurer}\end{définition}
\begin{définition}\pcmn{哭}\end{définition}
\begin{exemple}\pjya{aʑo nɯ-ɣɤwu-a}\hspace{5pt}\pcmn{我哭了}\end{exemple}
\begin{exemple}\pjya{ɯʑo nɯ-ɣɤwu}\hspace{5pt}\pcmn{他哭了}\end{exemple}
\begin{exemple}\pjya{jiɕqha tɤ-pɤtso nɯ ɲɯ-ɣɤwu}\hspace{5pt}\pcmn{这个小孩子在哭}\end{exemple}
\begin{exemple}\pjya{jiɕqha nɯ ɲɯ-nɯzdɯɣ-a tɕe, nɯ-ɣɤwu-a}\hspace{5pt}\pcmn{我为他担心所以哭了}\end{exemple}\sens{2}\paradigme{dir}{nɯ-}
\begin{définition}\pfra{crier (chat, bœuf, cochon, mouton, loup)}\end{définition}
\begin{définition}\pcmn{叫(猫、牛、猪、羊、狼)}\end{définition}
\begin{définition}\pfra{faire pleurer}\end{définition}
\begin{définition}\pcmn{让人哭}\end{définition}
\begin{exemple}\pjya{qachɣa ɲɯ-ɣɤwu}\hspace{5pt}\pcmn{狐狸叫}\end{exemple}
\begin{exemple}\pjya{ɕkɤrɯ ɲɯ-ɣɤwu}\hspace{5pt}\pcmn{鬣羚叫}\end{exemple}
\begin{exemple}\pjya{xsar ɲɯ-ɣɤwu}\hspace{5pt}\pcmn{青羊叫}\end{exemple}
\begin{exemple}\pjya{qaʑo ra ɲɯ-mtsɯr-nɯ tɕe ɲɯ-ɣɤwu-nɯ}\hspace{5pt}\pcmn{绵羊饿了就叫}\end{exemple}
\begin{exemple}\pjya{ɯʑo nɯ-zɣɤwu-t-a}\hspace{5pt}\pcmn{我让他哭了(我把他整哭了)}\end{exemple}
\begin{sous-entrée}{zɣɤwu}{ⓔɣɤwuⓢ2ⓝzɣɤwu} 
\classe{vt} \end{sous-entrée}

\begin{sous-entrée}{ɣɤɣɤwu}{ⓔɣɤwuⓢ2ⓝɣɤɣɤwu} 
\classe{vs} 
\begin{définition}\pfra{qui pleure tout le temps}\end{définition}
\begin{définition}\pcmn{爱哭;容易哭}\end{définition}
\begin{exemple}\pjya{nɤ-tɯ-ɣɤɣɤwu nɯ!}\hspace{5pt}\pcmn{你真爱哭呢!}\end{exemple}\relationsémantique{参考}{\lien{ⓔnɤwu}{nɤwu}}\relationsémantique{参考}{\lien{}{tɤ-wu}}\end{sous-entrée}

\end{entrée}

\begin{entrée}{ɣɤwɤt}{}{ⓔɣɤwɤt} 
\classe{vs}  
\grammaire{denom} \paradigme{dir}{nɯ-}
\begin{définition}\pfra{s'ouvrir (fleur)}\end{définition}
\begin{définition}\pcmn{开花}\end{définition}
\begin{exemple}\pjya{khɯjŋga ɲɤ-ɣɤwɤt}\hspace{5pt}\pcmn{杜鹃花开花了}\end{exemple}\end{entrée}

\begin{entrée}{ɣɤwxti}{}{ⓔɣɤwxti}\relationsémantique{参考}{\lien{ⓔwxti}{wxti}}\end{entrée}

\begin{entrée}{ɣɤxpra}{}{ⓔɣɤxpra} 
\classe{vt}  
\grammaire{denom} \paradigme{dir}{tɤ-}
\begin{définition}\pfra{ordonner}\end{définition}
\begin{définition}\pcmn{指使}\end{définition}
\begin{exemple}\pjya{tɤ-ɣɤxprat-a}\hspace{5pt}\pcmn{我使唤了他}\end{exemple}
\begin{exemple}\pjya{tɤ́-wɣ-ɣɤxpra}\hspace{5pt}\pcmn{他使唤了我}\end{exemple}
\begin{exemple}\pjya{tɤ-pɤtso tɤ-ɣɤxpra-t-a}\hspace{5pt}\pcmn{我使唤了小孩子}\end{exemple}
\begin{exemple}\pjya{tɤ-ɣɤxpra-t-a tɕe tɕe tɕɤndi laχtɕha ɯ-kɯ-ru nɯ-sɤɣri-t-a}\hspace{5pt}\pcmn{我使唤他去那边拿东西}\end{exemple}\relationsémantique{参考}{\lien{ⓔtɤpra}{tɤpra}}\relationsémantique{参考}{\lien{ⓔnɤpra}{nɤpra}}
\begin{sous-entrée}{sɤɣɤxpra}{ⓔɣɤxpraⓝsɤɣɤxpra} 
\classe{vi}  
\grammaire{apass} \end{sous-entrée}

\end{entrée}

\begin{entrée}{ɣɤxtɕɤβ}{}{ⓔɣɤxtɕɤβ} 
\classe{vi} \paradigme{dir}{pɯ-}\paradigme{dir}{pɯ-}
\begin{définition}\pfra{couper l'herbe}\end{définition}
\begin{définition}\pcmn{割饲草(左手抓住草,右手拿镰刀割)}\end{définition}
\begin{définition}\pfra{couper l'herbe}\end{définition}
\begin{définition}\pcmn{割饲草}\end{définition}
\begin{exemple}\pjya{pɯ-ɣɤxtɕɤβ}\hspace{5pt}\pcmn{他割了草}\end{exemple}
\begin{exemple}\pjya{sɯjno kutɕu ɲɯ-dɤn tɕe ɕ-pɯ-ɣɤxtɕaβ-a}\hspace{5pt}\pcmn{这里草很多,我去割了草}\end{exemple}
\begin{exemple}\pjya{rirɤβ kɤ-ɣɤxtɕɤβ mɤ-sɤcha}\hspace{5pt}\pcmn{不可能一手就把泰山割掉}\end{exemple}\relationsémantique{参考}{\lien{ⓔtɤxtɕɤβ}{tɤxtɕɤβ}}
\begin{sous-entrée}{nɤxtɕɤβ}{ⓔɣɤxtɕɤβⓝnɤxtɕɤβ} 
\classe{vt} \end{sous-entrée}

\end{entrée}

\begin{entrée}{ɣɤxtɕhɯxtɕhɯβ}{}{ⓔɣɤxtɕhɯxtɕhɯβ}\relationsémantique{参考}{\lien{ⓔsɤxtɕhɯxtɕhɯβ}{sɤxtɕhɯxtɕhɯβ}}\end{entrée}

\begin{entrée}{ɣɤxtɕi}{}{ⓔɣɤxtɕi}\relationsémantique{参考}{\lien{ⓔxtɕi}{xtɕi}}\end{entrée}

\begin{entrée}{ɣɤxtshɯm}{}{ⓔɣɤxtshɯm}\relationsémantique{参考}{\lien{ⓔxtshɯm}{xtshɯm}}\end{entrée}

\begin{entrée}{ɣɤxtɯt}{}{ⓔɣɤxtɯt}\relationsémantique{参考}{\lien{ⓔxtɯtⓗ2}{xtɯt}}\end{entrée}

\begin{entrée}{ɣɤxɯβxɯβ}{}{ⓔɣɤxɯβxɯβ}\relationsémantique{参考}{\lien{ⓔxɯβxɯβ}{xɯβxɯβ}}\end{entrée}

\begin{entrée}{ɣɤxɯrxɯr}{}{ⓔɣɤxɯrxɯr}\relationsémantique{参考}{\lien{ⓔxɯrxɯr}{xɯrxɯr}}\end{entrée}

\begin{entrée}{ɣɤxɯxɯɣ}{}{ⓔɣɤxɯxɯɣ} 
\classe{vi} 
\begin{définition}\pfra{faire la sourde oreille}\end{définition}
\begin{définition}\pcmn{当作耳边风}\end{définition}
\begin{exemple}\pjya{ɯ-rna ɯ-ɣmbaj ɲɯ-ɣɤxɯxɯɣ}\hspace{5pt}\pcmn{他当作耳风}\end{exemple}\relationsémantique{参考}{\lien{ⓔsɤxɯxɯɣ}{sɤxɯxɯɣ}}\end{entrée}

\begin{entrée}{ɣɤχalala}{}{ⓔɣɤχalala} 
\classe{vs} \paradigme{dir}{tɤ-}
\begin{définition}\pfra{extraverti}\end{définition}
\begin{définition}\pcmn{外向}\end{définition}
\begin{exemple}\pjya{jiɕqha tɯrme ɲɯ-ɣɤχalala, kɯ-rɯɕmi ci ɲɯ-ŋu}\hspace{5pt}\pcmn{那个人很外向,爱说话}\end{exemple}\relationsémantique{同义词}{\lien{ⓔɣɤŋoʁle}{ɣɤŋoʁle}}\end{entrée}

\begin{entrée}{ɣɤχɤlχɤl}{}{ⓔɣɤχɤlχɤl} 
\classe{vi} \paradigme{dir}{tɤ-}
\begin{définition}\pfra{amical}\end{définition}
\begin{définition}\pcmn{热情}\end{définition}
\begin{exemple}\pjya{azo jɤ-ɣe-a tɕe, ɯʑo a-ɕki ɲɯ-ɣɤχɤlχɤl ʑo}\hspace{5pt}\pcmn{我来的时候,他对我很热情}\end{exemple}\relationsémantique{参考}{\lien{ⓔχɤlχɤl}{χɤlχɤl}}\end{entrée}

\begin{entrée}{ɣɤχsrɯ}{}{ⓔɣɤχsrɯ} 
\classe{vs} 
\begin{définition}\pfra{beau}\end{définition}
\begin{définition}\pcmn{好看;英俊}\end{définition}
\begin{exemple}\pjya{kɯ-ɣɤχsrɯ ci tɤ-χtɯ-t-a}\hspace{5pt}\pcmn{我买了好看的(东西)}\end{exemple}
\begin{exemple}\pjya{tɯrme ɲɯ-ɣɤχsrɯ}\hspace{5pt}\pcmn{人很英俊}\end{exemple}
\begin{exemple}\pjya{jla ɲɯ-ɣɤχsrɯ}\hspace{5pt}\pcmn{犏牛很好看}\end{exemple}\end{entrée}

\begin{entrée}{ɣɤzbaʁ}{₁₂}{ⓔɣɤzbaʁⓗ1ⓗ2} 
\classe{vt}
\classe{vs}  
\grammaire{caus} \paradigme{dir}{tɤ-}
\begin{définition}\pfra{sécher}\end{définition}
\begin{définition}\pcmn{弄干}\end{définition}
\begin{exemple}\pjya{tɯ-ŋga thɯ-ɕkho-t-a tɕe, tɤ-ɣɤzbaʁ-a}\hspace{5pt}\pcmn{我把衣服晒干了}\end{exemple}
\begin{sous-entrée}{ɣɤzbaʁ}{ⓔɣɤzbaʁⓗ1ⓝɣɤzbaʁ}\end{sous-entrée}

\begin{définition}\pfra{se sécher facilement}\end{définition}
\begin{définition}\pcmn{容易干}\end{définition}
\begin{exemple}\pjya{qale a-pɯ-tu tɕe, ɲɯ-ɣɤzbaʁ}\hspace{5pt}\pcmn{有风的话就容易干}\end{exemple}\relationsémantique{参考}{\lien{ⓔzbaʁ}{zbaʁ}}\end{entrée}

\begin{entrée}{ɣɤzda}{}{ⓔɣɤzda} 
\classe{vt} \paradigme{dir}{\_}
\begin{définition}\pfra{saluer (sur le chemin)}\end{définition}
\begin{définition}\pcmn{打招呼(在路上碰见的时候)}\end{définition}
\begin{exemple}\pjya{ɕ-kɤ-ta-ɣɤzda ri kɤ-mtshɤm mataŋe}\hspace{5pt}\pcmn{我叫了你一声,但是你没有听到}\end{exemple}
\begin{exemple}\pjya{ɕ-kɤ-ta-ɣɤzda tɕe nɯ-tɯ-ɣe}\hspace{5pt}\pcmn{我叫了你,你就来了}\end{exemple}\relationsémantique{同义词}{\lien{ⓔɣɤŋoʁ}{ɣɤŋoʁ}}
\begin{sous-entrée}{aɣɤzdɯzda}{ⓔɣɤzdaⓝaɣɤzdɯzda} 
\classe{vi}  
\grammaire{recip} 
\begin{définition}\pfra{se saluer les uns les autres}\end{définition}
\begin{définition}\pcmn{互相打招呼}\end{définition}\relationsémantique{参考}{\lien{ⓔtɯ-zda}{tɯ-zda}}\relationsémantique{参考}{\lien{ⓔrɤzda}{rɤzda}}\relationsémantique{参考}{\lien{ⓔsɤzda}{sɤzda}}\relationsémantique{参考}{\lien{ⓔnɤzda}{nɤzda}}\end{sous-entrée}

\end{entrée}

\begin{entrée}{ɣɤzdoʁloʁ}{}{ⓔɣɤzdoʁloʁ} 
\classe{vi} \paradigme{dir}{tɤ-}
\begin{définition}\pfra{être tout petit et vif}\end{définition}
\begin{définition}\pcmn{做出一副小巧玲珑的样子}\end{définition}
\begin{exemple}\pjya{ma-tɯ-ɣɤzdoʁloʁ}\hspace{5pt}\pcmn{你不要一副小巧玲珑的样子}\end{exemple}\relationsémantique{参考}{\lien{ⓔzdoʁzdoʁ}{zdoʁzdoʁ}}\end{entrée}

\begin{entrée}{ɣɤzdɯzdɯr/\variante{ɣɤzdɯrzdɯr}}{}{ⓔɣɤzdɯzdɯr} 
\classe{vi} 
\begin{définition}\pfra{sautiller, rebondir (petits objets ronds)}\end{définition}
\begin{définition}\pcmn{弹来弹去(豌豆、珠子等)}\end{définition}
\begin{exemple}\pjya{staχpɯ ɲɯ-ɣɤzdɯzdɯr}\hspace{5pt}\pcmn{豌豆在弹来弹去}\end{exemple}\relationsémantique{参考}{\lien{ⓔzdɯzdɯr}{zdɯzdɯr}}\end{entrée}

\begin{entrée}{ɣɤzgrɤɣlɤɣ}{}{ⓔɣɤzgrɤɣlɤɣ} 
\classe{vs}  
\grammaire{deidph} \paradigme{dir}{tɤ-}
\begin{définition}\pfra{bruit de saut incessant}\end{définition}
\begin{définition}\pcmn{不停地跳动的声音}\end{définition}
\begin{exemple}\pjya{ɲɯ-ɣɤzgrɤɣlɤɣ}\hspace{5pt}\pcmn{他在跳动,发出很响的声音}\end{exemple}
\begin{exemple}\pjya{pɯ-ɣɤzgrɤɣlɤɣ-nɯ pɯ-rɟaʁ-nɯ}\hspace{5pt}\pcmn{他们跳舞,不停地跳动,发出很响的声音}\end{exemple}\end{entrée}

\begin{entrée}{ɣɤzɣɤrlɤr}{}{ⓔɣɤzɣɤrlɤr} 
\classe{vi}
\classe{vt} \paradigme{dir}{tɤ-}\paradigme{dir}{tɤ-}
\begin{définition}\pfra{avoir la tête qui tourne}\end{définition}
\begin{définition}\pcmn{头晕,脚都站不稳的感觉}\end{définition}
\begin{exemple}\pjya{ɲɯ-ɣɤzɣɤrlar-a ʑo}\hspace{5pt}\pcmn{我脚都站不稳}\end{exemple}
\begin{sous-entrée}{sɤzɣɤrlɤr}{ⓔɣɤzɣɤrlɤrⓝsɤzɣɤrlɤr}\end{sous-entrée}

\begin{définition}\pfra{qui fait tourner la tête}\end{définition}
\begin{définition}\pcmn{摇晃,令……头晕}\end{définition}\end{entrée}

\begin{entrée}{ɣɤzɣɯt}{}{ⓔɣɤzɣɯt}\relationsémantique{参考}{\lien{ⓔzɣɯt}{zɣɯt}}\end{entrée}

\begin{entrée}{ɣɤzjaŋlaŋ}{}{ⓔɣɤzjaŋlaŋ} 
\classe{vi} 
\begin{définition}\pfra{se balancer}\end{définition}
\begin{définition}\pcmn{摇晃}\end{définition}
\begin{exemple}\pjya{tʂu mɯ́j-pe tɕe, @qiche ɯ-ŋgɯ ku-kɯ-ɤmdzɯ tɕe tu-ɣɤzjaŋlaŋ ɲɯ-ŋu}\hspace{5pt}\pcmn{路不好,所以坐车的时候摇摇晃晃}\end{exemple}
\begin{sous-entrée}{sɤzjaŋlaŋ}{ⓔɣɤzjaŋlaŋⓝsɤzjaŋlaŋ} 
\classe{vt} 
\begin{définition}\pfra{balancer}\end{définition}
\begin{définition}\pcmn{摇动}\end{définition}
\begin{exemple}\pjya{laʁjɯɣ ɲɯ-sɤzjaŋlaŋ}\hspace{5pt}\pcmn{他乱动棍子}\end{exemple}\end{sous-entrée}

\begin{sous-entrée}{ɣɤzjaŋzjaŋ}{ⓔɣɤzjaŋlaŋⓝɣɤzjaŋzjaŋ} 
\classe{vi} \end{sous-entrée}

\begin{sous-entrée}{sɤzjaŋzjaŋ}{ⓔɣɤzjaŋlaŋⓝsɤzjaŋzjaŋ} 
\classe{vt} 
\begin{exemple}\pjya{mbro ɲɯ-sɤzjaŋzjaŋ ʑo ɲɯ-ɤz-nɯmbrɤpɯ}\hspace{5pt}\pcmn{他骑着马显得很高}\end{exemple}\end{sous-entrée}

\begin{sous-entrée}{nɯzjaŋ}{ⓔɣɤzjaŋlaŋⓝnɯzjaŋ} 
\classe{vt} \end{sous-entrée}

\end{entrée}

\begin{entrée}{ɣɤzjaŋzjaŋ}{}{ⓔɣɤzjaŋzjaŋ}\relationsémantique{参考}{\lien{ⓔɣɤzjaŋlaŋ}{ɣɤzjaŋlaŋ}}\end{entrée}

\begin{entrée}{ɣɤzjɤɣlɤɣ}{}{ⓔɣɤzjɤɣlɤɣ}\relationsémantique{参考}{\lien{ⓔzjɤɣzjɤɣ}{zjɤɣzjɤɣ}}\end{entrée}

\begin{entrée}{ɣɤzoŋzoŋ}{}{ⓔɣɤzoŋzoŋ} 
\classe{vs}  
\grammaire{deidph} \paradigme{dir}{thɯ-}\paradigme{dir}{tɤ-}
\begin{définition}\pfra{sensation d'engourdissement}\end{définition}
\begin{définition}\pcmn{觉得麻木}\end{définition}
\begin{définition}\pfra{engourdir}\end{définition}
\begin{définition}\pcmn{令……麻木}\end{définition}
\begin{exemple}\pjya{nɤ-mi cho-ɣɤzoŋzoŋ}\hspace{5pt}\pcmn{你的脚麻了}\end{exemple}
\begin{exemple}\pjya{nɤ-mi to-ndʑɯrpɯt tɕe, ɲɯ-ɣɤzoŋzoŋ loβtɕi}\hspace{5pt}\pcmn{你的脚麻了,是吧}\end{exemple}
\begin{sous-entrée}{sɤzoŋzoŋ}{ⓔɣɤzoŋzoŋⓝsɤzoŋzoŋ} 
\classe{vt} \end{sous-entrée}

\end{entrée}

\begin{entrée}{ɣɤzri}{₁}{ⓔɣɤzriⓗ1} 
\classe{vt}  
\grammaire{caus} \paradigme{dir}{nɯ-}
\begin{définition}\pfra{allonger}\end{définition}
\begin{définition}\pcmn{弄长}\end{définition}
\begin{exemple}\pjya{tɯ-ŋga nɯ-qrɯ-t-a tɕe nɯ-ɣɤzri-t-a}\hspace{5pt}\pcmn{我把衣服剪得太长了}\end{exemple}\relationsémantique{参考}{\lien{ⓔzri}{zri}}\end{entrée}

\begin{entrée}{ɣɤzri}{₂}{ⓔɣɤzriⓗ2} 
\classe{vs} 
\begin{définition}\pfra{qui s'allonge vite}\end{définition}
\begin{définition}\pcmn{长得快;容易变长}\end{définition}
\begin{sous-entrée}{nɤɣɤzri}{ⓔɣɤzriⓗ2ⓝnɤɣɤzri} 
\classe{vt} 
\begin{définition}\pfra{trouver que ...s'allonge vite}\end{définition}
\begin{définition}\pcmn{觉得……长得快}\end{définition}
\begin{exemple}\pjya{a-ndzrɯ ɲɯ-nɤɣɤzri-a tɕe tshɯrɟɯn ɲɯ-phɯt-a ŋu}\hspace{5pt}\pcmn{我觉得指甲长得很快,要经常剪}\end{exemple}\relationsémantique{参考}{\lien{ⓔzri}{zri}}\end{sous-entrée}

\end{entrée}

\begin{entrée}{ɣɤzɯβzɯβ}{}{ⓔɣɤzɯβzɯβ} 
\classe{vi} \paradigme{dir}{nɯ-}\paradigme{dir}{tɤ-}
\begin{définition}\pfra{astringent}\end{définition}
\begin{définition}\pcmn{涩;麻}\end{définition}
\begin{sous-entrée}{sɤzɯβzɯβ}{ⓔɣɤzɯβzɯβⓝsɤzɯβzɯβ}
\begin{exemple}\pjya{tɕɣom tɤ-ndza-t-a tɕe, a-mtɕhi sɤzɯβzɯβ}\hspace{5pt}\pcmn{我吃了花椒,嘴里发麻}\end{exemple}\end{sous-entrée}

\end{entrée}

\begin{entrée}{ɣɤzɯrzɯr}{}{ⓔɣɤzɯrzɯr} 
\classe{vi} 
\begin{définition}\pfra{ressentir une démangeaison}\end{définition}
\begin{définition}\pcmn{感觉到很痒}\end{définition}
\begin{exemple}\pjya{a-mtɕhi ɲɯ-ɣɤzɯzɯr}\hspace{5pt}\pcmn{我的嘴很痒}\end{exemple}\end{entrée}

\begin{entrée}{ɣɤʑu}{}{ⓔɣɤʑu} 
\classe{vi} 
\begin{définition}\pfra{y avoir, exister (sensoriel)}\end{définition}
\begin{définition}\pcmn{有(亲见)}\end{définition}
\begin{exemple}\pjya{nɯ ma mɤ-kɯ-pe ɯ-ɣɤ́ʑu?}\hspace{5pt}\pcmn{还有没有错误?}\end{exemple}\relationsémantique{参考}{\lien{ⓔmaŋe}{maŋe}}\forme{2s}{ɣɤtɤʑu}\end{entrée}

\begin{entrée}{ɣɤʑɯn}{}{ⓔɣɤʑɯn} 
\classe{vs} \paradigme{dir}{tɤ-}
\begin{définition}\pfra{pentu}\end{définition}
\begin{définition}\pcmn{陡峭}\end{définition}
\begin{exemple}\pjya{jiɕqha sɤtɕha ɣɤʑɯn}\hspace{5pt}\pcmn{那个地方很陡}\end{exemple}
\begin{exemple}\pjya{praʁ ɲɯ-ɣɤʑɯn}\hspace{5pt}\pcmn{悬崖很陡}\end{exemple}\relationsémantique{参考}{\lien{ⓔnɤʑɯn}{nɤʑɯn}}\relationsémantique{同义词}{\lien{ⓔɣɤrɤβ}{ɣɤrɤβ}}\end{entrée}

\begin{entrée}{ɣdɤɣdɤt}{}{ⓔɣdɤɣdɤt} 
\classe{idph.2} 
\begin{définition}\pfra{court et grassouillet}\end{définition}
\begin{définition}\pcmn{形容胖而短,看起来很可爱的样子}\end{définition}
\begin{exemple}\pjya{tɤ-pɤtso ɲɯ-tshu, ɯ-mɤlɤjaʁ ra ɣdɤɣdɤt ʑo ɲɯ-pa}\hspace{5pt}\pcmn{小孩子胖胖的,手脚又粗又短}\end{exemple}\end{entrée}

\begin{entrée}{ɣdɤso}{}{ⓔɣdɤso} 
\classe{n} 
\begin{définition}\pfra{ver marron}\end{définition}
\begin{définition}\pcmn{虫的一种}\end{définition}\end{entrée}

\begin{entrée}{ɣdoŋnɤɣdoŋ}{}{ⓔɣdoŋnɤɣdoŋ} 
\classe{idph.2} 
\begin{définition}\pfra{battement de tambour}\end{définition}
\begin{définition}\pcmn{形容很响的敲鼓声}\end{définition}\relationsémantique{参考}{\lien{}{dɯrnɤdɯr}}\relationsémantique{参考}{\lien{ⓔɣdɯɣnɤɣdɯɣ}{ɣdɯɣnɤɣdɯɣ}}
\begin{sous-entrée}{sɤɣdoŋɣdoŋ}{ⓔɣdoŋnɤɣdoŋⓝsɤɣdoŋɣdoŋ} 
\classe{vt} 
\begin{exemple}\pjya{tɤrmbɣo ɲɯ-sɤɣdoŋɣdoŋ ʑo}\hspace{5pt}\pcmn{打鼓打得很响}\end{exemple}\relationsémantique{同义词}{\lien{ⓔsɤndɤrndɤr}{sɤndɤrndɤr}}\end{sous-entrée}

\end{entrée}

\begin{entrée}{ɣdɯ}{}{ⓔɣdɯ} 
\classe{n} 
\begin{définition}\pfra{jarre de vin}\end{définition}
\begin{définition}\pcmn{酒坛子}\end{définition}\relationsémantique{同义词}{\lien{ⓔtɕhɤɣdɯ}{tɕhɤɣdɯ}}\relationsémantique{同义词}{\lien{ⓔtɕhorzi}{tɕhorzi}}\end{entrée}

\begin{entrée}{ɣdɯβɣdɯβ}{}{ⓔɣdɯβɣdɯβ} 
\classe{idph.2} 
\begin{définition}\pfra{court et épais}\end{définition}
\begin{définition}\pcmn{形容又粗又短的样子}\end{définition}
\begin{exemple}\pjya{ɯ-rtshɯm ci ɣdɯβɣdɯβ ɣɤʑu}\hspace{5pt}\pcmn{树墩又粗又短}\end{exemple}\end{entrée}

\begin{entrée}{ɣdɯɣnɤɣdɯɣ}{}{ⓔɣdɯɣnɤɣdɯɣ} 
\classe{idph.2} 
\begin{définition}\pfra{battement de tambour}\end{définition}
\begin{définition}\pcmn{形容轻轻的敲鼓声}\end{définition}\relationsémantique{参考}{\lien{}{dɯrnɤdɯr}}\relationsémantique{参考}{\lien{ⓔɣdoŋnɤɣdoŋ}{ɣdoŋnɤɣdoŋ}}\end{entrée}

\begin{entrée}{ɣe}{}{ⓔɣe} 
\classe{part} 
\begin{définition}\pfra{n'est ce pas ?}\end{définition}
\begin{définition}\pcmn{是不是?}\end{définition}
\begin{exemple}\pjya{nɤki tɕheme nɯ ɲɯ-mpɕɤr ɣe}\hspace{5pt}\pcmn{那个女孩子挺漂亮的,是不是?}\end{exemple}
\begin{exemple}\pjya{jisŋi tɯ-mɯ ɲɯ-jɯm ɣe}\hspace{5pt}\pcmn{今天天气挺好的,对不对?}\end{exemple}\end{entrée}

\begin{entrée}{ɣi}{}{ⓔɣi} 
\classe{vi}
\classe{vi} \paradigme{dir}{jɤ-}\paradigme{past stem}{ɣe}\paradigme{construction}{subj.part}
\begin{définition}\pfra{venir}\end{définition}
\begin{définition}\pcmn{来}\end{définition}
\begin{exemple}\pjya{jiɕqha kɤ-ari-tɕi tɕe, li nɯ-ɣe-tɕi}\hspace{5pt}\pcmn{我们俩去了然后就回来了}\end{exemple}
\begin{exemple}\pjya{pjɯ-tɯ-ɣi mɤ-ra}\hspace{5pt}\pcmn{你不用下来(送我们)}\end{exemple}
\begin{exemple}\pjya{a-mu ɯ-lɯz thɯ-ɣe}\hspace{5pt}\pcmn{我母亲年龄大了}\end{exemple}
\begin{exemple}\pjya{ɯ-ftsa ci chɯ-ɕe ɲɯ-ŋu tɕe, ``thɯ-ɣi ma a-zda me" ɲɯ-ti tɕe thɯ-ari}\hspace{5pt}\pcmn{他的侄子去成都就说“你来吧,没有人陪我”,他就一同去了}\end{exemple}\relationsémantique{Component 2}{\lien{ⓔɣi}{ɣi}}\relationsémantique{参考}{\lien{ⓔɣɤnɯʑɯβ}{ɣɤnɯʑɯβ}}\relationsémantique{参考}{\lien{ⓔtɯ-ʑɯβ}{tɯ-ʑɯβ}}\relationsémantique{参考}{\lien{ⓔtɯ-χsɯmχsozⓝɯ-χsoŋχsɤz,ɣi}{ɯ-χsoŋχsɤz,ɣi}}\relationsémantique{参考}{\lien{ⓔtɯ-ʁjiz,ɣi}{tɯ-ʁjiz,ɣi}}\relationsémantique{参考}{\lien{ⓔtɯ-lɯzⓝtɯ-lɯz,ɣi}{tɯ-lɯz,ɣi}}
\begin{sous-entrée}{ɣɤɣi}{ⓔɣiⓝɣɤɣi} 
\classe{vs} 
\begin{définition}\pfra{qui vient facilement}\end{définition}
\begin{définition}\pcmn{容易来}\end{définition}
\begin{exemple}\pjya{ɯ-ʑɯβ ɲɯ-ɣɤɣi}\hspace{5pt}\pcmn{他容易入眠}\end{exemple}\end{sous-entrée}

\begin{sous-entrée}{tɯ-jaʁ,ɣi}{ⓔɣiⓝtɯ-jaʁ,ɣi} 
\classe{np} 
\begin{définition}\pfra{obtenir}\end{définition}
\begin{définition}\pcmn{拿到;抓到;收到}\end{définition}
\begin{exemple}\pjya{ɯ-jaʁ mɯ-jɤ-ɣe}\hspace{5pt}\pcmn{没有到手}\end{exemple}
\begin{exemple}\pjya{ɯ-jaʁ tɕhi nɯ-kɯ-ɣe ʑo tu-ndze ɲɯ-ɕti}\hspace{5pt}\pcmn{它能抓到什么就吃什么}\end{exemple}\relationsémantique{Component 1}{\lien{ⓔtɯ-jaʁ}{tɯ-jaʁ}}\end{sous-entrée}

\end{entrée}

\begin{entrée}{ɣɟaβ}{}{ⓔɣɟaβ} 
\classe{vt} \paradigme{dir}{nɯ-}\paradigme{dir}{pɯ-}
\begin{définition}\pfra{battre le lait}\end{définition}
\begin{définition}\pcmn{搅牛奶;打酥油}\end{définition}
\begin{exemple}\pjya{pɯ-ɣɟaβ-a}\hspace{5pt}\pcmn{我搅了(牛奶)}\end{exemple}
\begin{exemple}\pjya{ɯʑo kɯ tɤ-lu na-ɣɟaβ}\hspace{5pt}\pcmn{他搅了牛奶}\end{exemple}
\begin{exemple}\pjya{tɤ-lu pjɯ́-wɣ-ɣɟaβ kóʁmɯz ta-mar tu-nɯɬoʁ ŋu}\hspace{5pt}\pcmn{搅了牛奶就有酥油出来}\end{exemple}\relationsémantique{参考}{\lien{ⓔtɯɣɟaβ}{tɯɣɟaβ}}\end{entrée}

\begin{entrée}{ɣɟɯ}{}{ⓔɣɟɯ} 
\classe{n} 
\begin{définition}\pfra{tour de garde}\end{définition}
\begin{définition}\pcmn{碉楼}\end{définition}\end{entrée}

\begin{entrée}{ɣɟɯʁar}{}{ⓔɣɟɯʁar} 
\classe{n} 
\begin{définition}\pfra{monstre}\end{définition}
\begin{définition}\pcmn{魔鬼;高大的人,全身长着黑毛}\end{définition}\end{entrée}

\begin{entrée}{ɣɟɯthoʁ}{}{ⓔɣɟɯthoʁ} 
\classe{n}  
\grammaire{n.lieu} 
\begin{définition}\pfra{hameau de Ercha près du fleuve}\end{définition}
\begin{définition}\pcmn{二茶村的河坝地区}\end{définition}\end{entrée}

\begin{entrée}{ɣɟɯtshapa}{}{ⓔɣɟɯtshapa} 
\classe{n}  
\grammaire{n.lieu} 
\begin{définition}\pfra{Ercha (village de Gdongbrgyad)}\end{définition}
\begin{définition}\pcmn{二茶村}\end{définition}\end{entrée}

\begin{entrée}{ɣle}{}{ⓔɣle} 
\classe{vt} \paradigme{dir}{pɯ-}\paradigme{dir}{tɤ-}
\begin{définition}\pfra{pétrir, frotter}\end{définition}
\begin{définition}\pcmn{揉}\end{définition}
\begin{exemple}\pjya{pɯ-ɣle-t-a}\hspace{5pt}\pcmn{我揉了}\end{exemple}
\begin{exemple}\pjya{pɯ-tɯ-ɣle-t}\hspace{5pt}\pcmn{你揉了}\end{exemple}
\begin{exemple}\pjya{pa-ɣle}\hspace{5pt}\pcmn{他揉了}\end{exemple}
\begin{exemple}\pjya{tɤjlu wuma ʑo pa-ɣle tɕe ɲɯ-mɯm}\hspace{5pt}\pcmn{他揉了面,面就好吃了}\end{exemple}
\begin{sous-entrée}{nɤɣlɤɣle}{ⓔɣleⓝnɤɣlɤɣle} 
\classe{vt} 
\begin{définition}\pfra{frotter dans tous les sens}\end{définition}
\begin{définition}\pcmn{揉来揉去}\end{définition}\end{sous-entrée}

\end{entrée}

\begin{entrée}{ɣlɯntɯ}{}{ⓔɣlɯntɯ} 
\classe{n} 
\begin{définition}\pfra{bouse de vache sèche dans la montagne}\end{définition}
\begin{définition}\pcmn{山上的干牛屎}\end{définition}\end{entrée}

\begin{entrée}{ɣlɯtɕɤt}{}{ⓔɣlɯtɕɤt} 
\classe{n} 
\begin{définition}\pfra{action de retirer le purin de l'étable}\end{définition}
\begin{définition}\pcmn{出圈粪}\end{définition}
\begin{exemple}\pjya{ɣlɯtɕɤt wuma ʑo ɴqa}\hspace{5pt}\pcmn{出圈是很辛苦的}\end{exemple}\relationsémantique{参考}{\lien{ⓔtɯ-ɣli}{tɯ-ɣli}}\relationsémantique{参考}{\lien{ⓔtɕɤt}{tɕɤt}}\relationsémantique{参考}{\lien{ⓔɣɯɣlɯtɕɤt}{ɣɯɣlɯtɕɤt}}\end{entrée}

\begin{entrée}{ɣnɤsqi}{}{ⓔɣnɤsqi} 
\classe{num} 
\begin{définition}\pfra{vingt}\end{définition}
\begin{définition}\pcmn{二十}\end{définition}\end{entrée}

\begin{entrée}{ɣnda}{}{ⓔɣnda} 
\classe{vt} \paradigme{dir}{pɯ-}
\begin{définition}\pfra{frapper (avec un marteau), marteler, tasser}\end{définition}
\begin{définition}\pcmn{捶打;夯结实}\end{définition}
\begin{exemple}\pjya{pɯ-ɣnda-t-a}\hspace{5pt}\pcmn{我捶打了}\end{exemple}
\begin{exemple}\pjya{pɯ-tɯ-ɣnda-t}\hspace{5pt}\pcmn{你捶打了}\end{exemple}
\begin{exemple}\pjya{pa-ɣnda}\hspace{5pt}\pcmn{他捶打了}\end{exemple}
\begin{exemple}\pjya{khɤxtu pjɯ́-wɣ-ɣnda tɕe, tɯftsaʁ mɤ-ɣi}\hspace{5pt}\pcmn{把房背夯结实以后,下雨的时候不会漏水}\end{exemple}
\begin{exemple}\pjya{tɤtshoʁ pɯ-ɣnda-t-a}\hspace{5pt}\pcmn{我钉了钉子}\end{exemple}\end{entrée}

\begin{entrée}{ɣndʑɤβ}{}{ⓔɣndʑɤβ} 
\classe{n} 
\begin{définition}\pfra{feu, incendie}\end{définition}
\begin{définition}\pcmn{损坏性的火}\end{définition}
\begin{exemple}\pjya{ɣndʑɤβ tɤ-ta-t-a}\hspace{5pt}\pcmn{我放了火}\end{exemple}
\begin{exemple}\pjya{ɣndʑɤβ to-lɯɣ}\hspace{5pt}\pcmn{失火}\end{exemple}\relationsémantique{参考}{\lien{ⓔɣndʑɤβta}{ɣndʑɤβta}}\relationsémantique{参考}{\lien{ⓔndʑɤβ}{ndʑɤβ}}\end{entrée}

\begin{entrée}{ɣndʑɤβta}{}{ⓔɣndʑɤβta} 
\classe{n} 
\begin{définition}\pfra{feu}\end{définition}
\begin{définition}\pcmn{火}\end{définition}
\begin{exemple}\pjya{ɣndʑɤβta pɯ-tu}\hspace{5pt}\pcmn{烧火了}\end{exemple}\relationsémantique{参考}{\lien{ⓔɣɯɣndʑɤβta}{ɣɯɣndʑɤβta}}\end{entrée}

\begin{entrée}{ɣndʑɯr}{}{ⓔɣndʑɯr} 
\classe{vt} \paradigme{dir}{thɯ-}\paradigme{dir}{nɯ-}\paradigme{dir}{pɯ-}
\begin{définition}\pfra{moudre}\end{définition}
\begin{définition}\pcmn{磨}\end{définition}
\begin{exemple}\pjya{thɯ-ɣndʑɯr-a}\hspace{5pt}\pcmn{我磨了}\end{exemple}
\begin{exemple}\pjya{ɯʑo kɯ tha-ɣndʑɯr}\hspace{5pt}\pcmn{他磨了}\end{exemple}
\begin{exemple}\pjya{tɯjpu chɯ́-wɣ-ndʑɯr mɤɕtʂa kɤ-ndza mɤ-sna}\hspace{5pt}\pcmn{在没磨之前,不能吃粮食}\end{exemple}
\begin{exemple}\pjya{smɤn nɯ-ɣndʑɯr-a}\hspace{5pt}\pcmn{我磨了药}\end{exemple}
\begin{exemple}\pjya{tɤjlu thɯ-ɣndʑɯr-a}\hspace{5pt}\pcmn{我磨了面}\end{exemple}
\begin{exemple}\pjya{pɯ-ɣndʑɯr-a}\hspace{5pt}\pcmn{我磨了(青稞)}\end{exemple}\relationsémantique{参考}{\lien{ⓔsɤrŋɤɣndʑɯr}{sɤrŋɤɣndʑɯr}}\relationsémantique{参考}{\lien{}{tɯ-ɣndʑɯr}}\end{entrée}

\begin{entrée}{ɣni}{}{ⓔɣni} 
\classe{n} 
\begin{définition}\pfra{renard volant}\end{définition}
\begin{définition}\pcmn{飞鼠}\end{définition}
\begin{exemple}\pjya{ɣni nɯ stɤmku cho sɯŋgɯ ku-rɤʑi ŋu, ɯ-mdoʁ nɯ βʑɯ ɯ-mdoʁ cho naχtɕɯɣ, ɯ-mgɯr ɯ-χcɤl nɯ tɕu kɯ-ɲaʁ tɯ-ʂɯl tu, ɯ-mi ɯ-ru me, ɯ-mɤpɤl ma me, ɯ-ndzrɯ tu, kú-wɣ-rtoʁ tɕe, ɯ-mɤlɤjaʁ stɤsmɤt ɯ-pɤrthɤβ nɯ tɯ-ndʐi kɯ ɲɯ-ɤlɤɣɯ ɕti, tɕe tɯ-ɕe jɤ-ʑa tɕe, ɯ-ndʐi nɯ ɯ-mɤlɤjaʁ kɯ ɲɯ-sqhiar tɕe sɤtɕha ɯ-taʁ pjɤ-sthaβ ʑo tɕe ɲɯ-nɯqambɯmbjom kɯ-fse ɲɯ-ŋu. kɤ-ŋke nɯ mɯ́j-khɯ rca, ɯ-jme kɯ-xtɕɯ-xtɕi ci ɣɤʑu, ɯ-ku nɯ βʑɯ ɯ-ku ɲɯ-fse tɕe ri ɯ-rna kɯ-saχsɤl maŋe, tu-mbri tɕe, tɤ-pɤtso nɯ-kɤ-ɣɤwu ɯ-skɤt kɯ-fse tu-lɤt ɲɯ-ŋu, mɤʑɯ kɯmaʁ kɤntɕhɯ-tɯphu ɯ-skɤt tu-lɤt ɲɯ-ŋgrɤl}\hspace{5pt}\pcmn{飞鼠生活在草山和森林里,(皮毛的)颜色和老鼠的一样,背上中间有一条黑色的纹路,(好像)没有腿,只有脚板和爪子。看起来前腿和后腿之间的皮子是连在一起的。它开始走动的时候,皮子就用四肢来展开,贴住地面(离地面很近)飞行。它可能不会走。有小尾巴,头有点像老鼠的头,但看不出耳朵。喊叫的时候,发出和小孩子哭一样的声音,还能发出其它好几种动物的叫声。}\end{exemple}\end{entrée}

\begin{entrée}{ɣot}{}{ⓔɣot} 
\classe{n} 
\begin{définition}\pfra{lumière et chaleur (du soleil)}\end{définition}
\begin{définition}\pcmn{光和热量(太阳的)}\end{définition}
\begin{exemple}\pjya{tɤŋe ɯ-ɣot}\hspace{5pt}\pcmn{太阳光}\end{exemple}\relationsémantique{参考}{\lien{ⓔsmɯɣot}{smɯɣot}}\étymologie{ɦod}\end{entrée}

\begin{entrée}{ɣrɤmu/\variante{rɤmu}}{}{ⓔɣrɤmu} 
\classe{n} 
\begin{définition}\pfra{Thlaspi arvense}\end{définition}
\begin{définition}\pcmn{菥蓂【苦苦菜】}\end{définition}
\begin{exemple}\pjya{ɣrɤmu nɯ sɯjno kɯ-xtɕɯ-xtɕi ci ŋu, ɯ-jwaʁ nɯ tɕɤr, ɯ-ku tɕe lu-ortɯm tsa ŋu ma mɤ-amtɕoʁ, ɯ-jwaʁ mpɯ. ɯ-jwaʁ cho ɯ-ru ra arŋi. ɯ-χcɤl ɯ-ru tu-ɬoʁ tɕe, ɯ-jwaʁ tu-oʑɯrja ŋu. ɯ-jwaʁ tɤ-arɕo tɕe ɯ-mɯntoʁ tu-oʑɯrja tɕe, ɯ-mat chɯ-βze ŋu. ɯ-mat nɯ kɯ-ɤrtɯm tɕe kɯ-ɤɕpɯɕpa ŋu. ɯ-mɯntoʁ wɣrum. ɯ-mat ɯ-ŋgɯ ɯ-rɣi wuma ʑo dɤn, ndɯβ. ɯ-rɣi ɯ-taʁ tɤ-rʑɯɣ kɯ-fse tu, tu-mbro mɤ-cha, ɯ-ru tu-ɬoʁ ɕɯŋgɯ tɕe, ɯ-jwaʁ kɤ-ndza sna, kɯ-xtɕɯ-xtɕi qiaβ. thɯ-tɯt tɕe, ɯ-ru cho ɯ-mat ra lonba ɲɯ-qarŋe, ɯ-jwaʁ ra lonba pjɯ-ŋgra ŋu. ɯ-ru cho ɯ-mat ma ɲɯ-me ŋu.}\hspace{5pt}\pcmn{苦苦菜是长得很小的草,叶子细,顶端是圆形的,不尖,很嫩。叶子和茎都是绿色的。中间长茎,叶子排列在茎上。叶子长到一定的高度,然后花排列在上面一段,就结果。果实圆而扁,花是白色的,果实里种子很多,很小。种子表面有皱纹。长不高。茎长出来之前,叶子可以吃,有点苦。成熟以后,茎和果实全变黄,叶子落光,只剩下茎和果实了。}\end{exemple}\end{entrée}

\begin{entrée}{ɣro}{}{ⓔɣro} 
\classe{vi} \paradigme{dir}{pɯ-}\paradigme{dir}{thɯ-}\paradigme{dir}{pɯ-}
\begin{définition}\pfra{s'étouffer}\end{définition}
\begin{définition}\pcmn{呛到}\end{définition}
\begin{définition}\pfra{étouffer}\end{définition}
\begin{définition}\pcmn{呛}\end{définition}
\begin{exemple}\pjya{tɯsqar (tɯ-ɣndʑɤr) tɤ-moʁ-a tɕe pɯ-ɣro-a}\hspace{5pt}\pcmn{我吃糌粑的时候呛到了}\end{exemple}
\begin{exemple}\pjya{pɯ́-wɣ-sɯɣro-a}\hspace{5pt}\pcmn{我被呛到了}\end{exemple}
\begin{exemple}\pjya{ɯ-tshɤt kɤ-tshi ma tɯ́-wɣ-sɯɣro}\hspace{5pt}\pcmn{不要喝得太多,不然会呛着}\end{exemple}
\begin{sous-entrée}{sɯɣro}{ⓔɣroⓝsɯɣro} 
\classe{vt}  
\grammaire{caus} \end{sous-entrée}

\end{entrée}

\begin{entrée}{ɣurʑa}{}{ⓔɣurʑa} 
\classe{n} 
\begin{définition}\pfra{cent}\end{définition}
\begin{définition}\pcmn{100}\end{définition}
\begin{exemple}\pjya{ɣurʑa cho tɯ-rdoʁ}\hspace{5pt}\pcmn{一百零一}\end{exemple}
\begin{exemple}\pjya{ɯ-ɣurʑa-xpa}\hspace{5pt}\pcmn{好几百年}\end{exemple}\relationsémantique{同义词}{\lien{ⓔtɯ-ri}{tɯ-ri}}\end{entrée}

\begin{entrée}{ɣɯβɣɯβ}{}{ⓔɣɯβɣɯβ} 
\classe{idph.2} 
\begin{définition}\pfra{gens attroupés autour de quelque chose}\end{définition}
\begin{définition}\pcmn{围拢起来,人与人之间没有缝隙}\end{définition}
\begin{exemple}\pjya{tɤjmɤɣ ɯ-kɯ-ntsɣe jo-ɣi-nɯ tɕe, kɤntɕhaʁ tɯrme ra ɣɯβɣɯβ ʑo ko-nɤrkhar-nɯ}\hspace{5pt}\pcmn{卖菌子的人来了,街上很多人围着看}\end{exemple}
\begin{exemple}\pjya{ɲɯ-mpja ɣɯβɣɯβ ʑo}\hspace{5pt}\pcmn{暖烘烘}\end{exemple}\relationsémantique{参考}{\lien{ⓔnɤɣɯβɣɯβ}{nɤɣɯβɣɯβ}}\end{entrée}

\begin{entrée}{ɣɯcɤno}{}{ⓔɣɯcɤno} 
\classe{vi}  
\grammaire{incorp} \paradigme{dir}{\_}
\begin{définition}\pfra{faire de la chasse à courre}\end{définition}
\begin{définition}\pcmn{围猎}\end{définition}\relationsémantique{参考}{\lien{ⓔca}{ca}}\relationsémantique{参考}{\lien{ⓔno}{no}}\end{entrée}

\begin{entrée}{ɣɯchɤtshi}{}{ⓔɣɯchɤtshi} 
\classe{vi}  
\grammaire{incorp} \paradigme{dir}{kɤ-}
\begin{définition}\pfra{boire du vin (ensemble)}\end{définition}
\begin{définition}\pcmn{喝酒(几个人一起)}\end{définition}
\begin{exemple}\pjya{pɯ-ɣɯchɤtshi-j}\hspace{5pt}\pcmn{我们在喝酒(过去)}\end{exemple}
\begin{exemple}\pjya{kɤ-ɣɯchɤtshi-j}\hspace{5pt}\pcmn{我们喝了酒}\end{exemple}\relationsémantique{参考}{\lien{ⓔchɤtshi}{chɤtshi}}\end{entrée}

\begin{entrée}{ɣɯcɯphɯt}{}{ⓔɣɯcɯphɯt} 
\classe{vi}  
\grammaire{incorp} \paradigme{dir}{nɯ-}\paradigme{dir}{pɯ-}
\begin{définition}\pfra{ramasser les pierres}\end{définition}
\begin{définition}\pcmn{拣石头(庄稼地)}\end{définition}
\begin{exemple}\pjya{jisŋi tɯ-ji ɯ-ŋgɯ pɯ-ɣɯcɯphɯt-a}\hspace{5pt}\pcmn{今天我在田地里捡石头}\end{exemple}
\begin{exemple}\pjya{jiʑo ji-ji ɯ-ŋgɯ pɯ-ɣɯcɯphɯt-a}\hspace{5pt}\pcmn{我在我们田里捡了石头}\end{exemple}\relationsémantique{参考}{\lien{ⓔcɯphɯt}{cɯphɯt}}\end{entrée}

\begin{entrée}{ɣɯɕkat}{}{ⓔɣɯɕkat} 
\classe{vt}  
\grammaire{denom} \paradigme{dir}{tɤ-}
\begin{définition}\pfra{mettre les charges sur les animaux}\end{définition}
\begin{définition}\pcmn{上驮子}\end{définition}
\begin{exemple}\pjya{mbala tɤ-ɣɯɕkat-i}\hspace{5pt}\pcmn{我们给牛上了驮子}\end{exemple}
\begin{exemple}\pjya{jla tɤ-ɣɯɕkat-i}\hspace{5pt}\pcmn{我们给犏牛上了驮子}\end{exemple}
\begin{exemple}\pjya{mbro tɤ-ɣɯɕkat-i}\hspace{5pt}\pcmn{我们给马上了驮子}\end{exemple}
\begin{exemple}\pjya{jla tɤ-ɣɯɕkat-a}\hspace{5pt}\pcmn{我给犏牛上了驮子}\end{exemple}
\begin{exemple}\pjya{laχtɕha tɤ-ɣɯɕkat-a}\hspace{5pt}\pcmn{我驮了东西}\end{exemple}\relationsémantique{参考}{\lien{ⓔtɯ-ɕkat}{tɯ-ɕkat}}\relationsémantique{参考}{\lien{ⓔnɯɕkat}{nɯɕkat}}
\begin{sous-entrée}{aɣɯɕkat}{ⓔɣɯɕkatⓝaɣɯɕkat} 
\classe{vi} 
\begin{définition}\pfra{porter comme charge}\end{définition}
\begin{définition}\pcmn{驮着}\end{définition}
\begin{exemple}\pjya{ki tɤrka ki tɤrɤku aɣɯɕkat}\hspace{5pt}\pcmn{这头驴子驮的是粮食}\end{exemple}\end{sous-entrée}

\end{entrée}

\begin{entrée}{ɣɯɕoŋtɕa}{}{ⓔɣɯɕoŋtɕa} 
\classe{vi}  
\grammaire{denom} \paradigme{dir}{pɯ-}
\begin{définition}\pfra{couper du bois}\end{définition}
\begin{définition}\pcmn{砍木头}\end{définition}
\begin{exemple}\pjya{pɯ-ɣɯɕoŋtɕa-a}\hspace{5pt}\pcmn{我砍木头了}\end{exemple}
\begin{exemple}\pjya{ɕ-pɯ-ɣɯɕoŋtɕa-a}\hspace{5pt}\pcmn{我去砍木头了}\end{exemple}\relationsémantique{参考}{\lien{ⓔɕoŋtɕa}{ɕoŋtɕa}}\end{entrée}

\begin{entrée}{ɣɯɕɯ}{}{ⓔɣɯɕɯ} 
\classe{vs} \paradigme{dir}{thɯ-}
\begin{définition}\pfra{âgé et respecté, calme et avisé}\end{définition}
\begin{définition}\pcmn{稳重;沉着}\end{définition}
\begin{exemple}\pjya{iɕqha tɯrme nɯ kɯ-ɣɯɕɯ ci ɲɯ-ŋu, ɲɯ-ɣɯɕɯ}\hspace{5pt}\pcmn{那个人很稳重}\end{exemple}\end{entrée}

\begin{entrée}{ɣɯfkɯm}{}{ⓔɣɯfkɯm} 
\classe{vt}  
\grammaire{denom} \paradigme{dir}{tɤ-}\sens{1}
\begin{définition}\pfra{mettre dans sa poche}\end{définition}
\begin{définition}\pcmn{装在口袋里}\end{définition}
\begin{exemple}\pjya{tɯjpu nɯ tɤ-ɣɯfkɯm-a}\hspace{5pt}\pcmn{我把粮食装在口袋里了}\end{exemple}
\begin{exemple}\pjya{qha laχtɕha nɯ tɤ-ɣɯfkɯm-a}\hspace{5pt}\pcmn{我把那个东西装在口袋里了}\end{exemple}
\begin{exemple}\pjya{qha laχtɕha nɯra ɣɯfkɯm-i}\hspace{5pt}\pcmn{我们要把这些东西装在口袋里}\end{exemple}\sens{2}
\begin{définition}\pfra{conserver dans le grenier}\end{définition}
\begin{définition}\pcmn{储存在仓库里}\end{définition}\relationsémantique{参考}{\lien{ⓔtɤ-fkɯm}{tɤ-fkɯm}}\end{entrée}

\begin{entrée}{ɣɯfsu}{}{ⓔɣɯfsu} 
\classe{n} 
\begin{définition}\pfra{ami}\end{définition}
\begin{définition}\pcmn{朋友}\end{définition}
\begin{exemple}\pjya{ɣɯfsu to-nɯpa-ndʑi}\hspace{5pt}\pcmn{我们交了朋友}\end{exemple}\relationsémantique{参考}{\lien{ⓔɯ-fsu}{ɯ-fsu}}\end{entrée}

\begin{entrée}{ɣɯɣlɯtɕɤt}{}{ⓔɣɯɣlɯtɕɤt} 
\classe{vi}  
\grammaire{incorp} \paradigme{dir}{thɯ-}
\begin{définition}\pfra{retirer le purin de l'étable pour en faire de l'engrais}\end{définition}
\begin{définition}\pcmn{出圈粪}\end{définition}
\begin{exemple}\pjya{thɯ-ɣɯɣlɯtɕɤt-i, pɯ-ɣɯɣlɯtɕɤt-i}\hspace{5pt}\pcmn{我们出圈粪}\end{exemple}\relationsémantique{参考}{\lien{ⓔɣlɯtɕɤt}{ɣlɯtɕɤt}}\end{entrée}

\begin{entrée}{ɣɯɣndʑɤβta}{}{ⓔɣɯɣndʑɤβta} 
\classe{vi}  
\grammaire{incorp} \paradigme{dir}{tɤ-}
\begin{définition}\pfra{défricher par le feu}\end{définition}
\begin{définition}\pcmn{烧荒}\end{définition}
\begin{exemple}\pjya{jɯfɕɯr ɕ-pɯ-ɣɯɣndʑɤβta-j}\hspace{5pt}\pcmn{我们昨天去烧荒了}\end{exemple}\relationsémantique{参考}{\lien{ⓔɣndʑɤβta}{ɣndʑɤβta}}\relationsémantique{参考}{\lien{ⓔɣndʑɤβta}{ɣndʑɤβta}}\end{entrée}

\begin{entrée}{ɣɯjpa}{}{ⓔɣɯjpa} 
\classe{adv} 
\begin{définition}\pfra{cette année}\end{définition}
\begin{définition}\pcmn{今年}\end{définition}\end{entrée}

\begin{entrée}{ɣɯjru}{}{ⓔɣɯjru} 
\classe{vt} \paradigme{dir}{kɤ-}
\begin{définition}\pfra{cuire (poterie)}\end{définition}
\begin{définition}\pcmn{炙烤(泥制品)}\end{définition}
\begin{exemple}\pjya{tɕhorzi kɤ-ɣɯjru-t-a}\hspace{5pt}\pcmn{我把坛子炙烤了}\end{exemple}
\begin{exemple}\pjya{tɕhorzi ɯ-ɣɯjru mɯ-ko-rtaʁ}\hspace{5pt}\pcmn{坛子没有烤熟}\end{exemple}\end{entrée}

\begin{entrée}{ɣɯjtsi}{}{ⓔɣɯjtsi} 
\classe{vt} \paradigme{dir}{tɤ-}
\begin{définition}\pfra{soutenir}\end{définition}
\begin{définition}\pcmn{(用柱子)顶住}\end{définition}
\begin{exemple}\pjya{kɯki tɤ-ɣɯjtsi-t-a}\hspace{5pt}\pcmn{我把这个顶住了}\end{exemple}
\begin{exemple}\pjya{tɕɤtu nɯ ɯ-pa pjɯ-nɯɣi ɲɯ-ŋu tɕe tɤ-ɣɯjtsi-t-a}\hspace{5pt}\pcmn{上面的那个东西往下面来,我把往下来的部分顶住了}\end{exemple}\relationsémantique{参考}{\lien{ⓔtɤ-jtsi}{tɤ-jtsi}}\end{entrée}

\begin{entrée}{ɣɯkhɯtshoʁ}{}{ⓔɣɯkhɯtshoʁ} 
\classe{vi}  
\grammaire{incorp} \paradigme{dir}{pɯ-}\paradigme{dir}{thɯ-}
\begin{définition}\pfra{lâcher les chiens (à la chasse)}\end{définition}
\begin{définition}\pcmn{放狗打猎}\end{définition}
\begin{exemple}\pjya{jɯfɕo kɯ-ɣɯkhɯtshoʁ jɤ-ari-a}\hspace{5pt}\pcmn{我今天早上去放狗打猎了}\end{exemple}
\begin{exemple}\pjya{ɕ-pɯ-ɣɯkhɯtshoʁ-a}\hspace{5pt}\pcmn{我去放狗打猎了}\end{exemple}
\begin{exemple}\pjya{kɯ-ɣɤrʁaʁ ci jo-ɣi tɕe, khɯna to-ndo, ta-ʁrɯ chɤ-ʑmbri tɕe, chɤ-ɣɯkhɯtshoʁ}\hspace{5pt}\pcmn{来了一个猎人,带着猎狗,吹了螺号,然后把狗放了}\end{exemple}\relationsémantique{参考}{\lien{ⓔkhɯtshoʁ}{khɯtshoʁ}}\relationsémantique{参考}{\lien{ⓔtshoʁⓢ2ⓝkhɯna,tshoʁ}{khɯna,tshoʁ}}\end{entrée}

\begin{entrée}{ɣɯlaj}{}{ⓔɣɯlaj} 
\classe{vs} 
\begin{définition}\pfra{aux gestes}\end{définition}
\begin{définition}\pcmn{动作慢}\end{définition}
\begin{exemple}\pjya{ta-ma ɲɯ-tɯ-ɣɯlaj}\hspace{5pt}\pcmn{你劳动的时候动作慢}\end{exemple}\relationsémantique{反义词}{\lien{ⓔɣɤjru}{ɣɤjru}}\end{entrée}

\begin{entrée}{ɣɯlɤn}{}{ⓔɣɯlɤn} 
\classe{vt}  
\grammaire{denom} \paradigme{dir}{nɯ-}\paradigme{dir}{tɤ-}
\begin{définition}\pfra{répondre}\end{définition}
\begin{définition}\pcmn{回答}\end{définition}
\begin{exemple}\pjya{ɯʑo kɯ na-ɣɯlɤn}\hspace{5pt}\pcmn{他回答了}\end{exemple}
\begin{exemple}\pjya{jiɕqha ɲɯ-ɤkhu tɕe tɤ-ɣɯlan-a}\hspace{5pt}\pcmn{他叫了,我就回答了}\end{exemple}
\begin{exemple}\pjya{ɲo-sɯthu tɕe tɤ-ɣɯlan-a}\hspace{5pt}\pcmn{他问了我就回答了}\end{exemple}\relationsémantique{参考}{\lien{ⓔtɯ-lɤn}{tɯ-lɤn}}\étymologie{len}\end{entrée}

\begin{entrée}{ɣɯpɕawtsɯfsoʁ}{}{ⓔɣɯpɕawtsɯfsoʁ} 
\classe{vi}  
\grammaire{incorp} \paradigme{dir}{nɯ-}
\begin{définition}\pfra{gagner de l'argent}\end{définition}
\begin{définition}\pcmn{挣钱}\end{définition}
\begin{exemple}\pjya{ʑ-nɯ-ɣɯpɕawtsɯfsoʁ}\hspace{5pt}\pcmn{你去挣钱吧}\end{exemple}
\begin{exemple}\pjya{kɯ-nɯpɕawtsɯfsoʁ jɤ-ari}\hspace{5pt}\pcmn{他去打工挣钱了}\end{exemple}
\begin{exemple}\pjya{kɯ-nɯhɯɲi jɤ-ari tɕe ɲɯ-ɣɯpɕawtsɯfsoʁ}\hspace{5pt}\pcmn{他去打工挣钱了}\end{exemple}\end{entrée}

\begin{entrée}{ɣɯrɣɯr}{}{ⓔɣɯrɣɯr} 
\classe{idph.2} \paradigme{dir}{tɤ-}
\begin{définition}\pfra{beaucoup de gens rassemblés}\end{définition}
\begin{définition}\pcmn{形容人密集的样子}\end{définition}
\begin{exemple}\pjya{ɣɯrɣɯr ʑo pjɤ-k-ɤkhar-nɯ-ci}\hspace{5pt}\pcmn{很多人围着他}\end{exemple}
\begin{sous-entrée}{ɣɯrnɤɣɯr}{ⓔɣɯrɣɯrⓝɣɯrnɤɣɯr} 
\classe{idph.3} \end{sous-entrée}

\begin{sous-entrée}{ɣɤɣɯrɣɯr}{ⓔɣɯrɣɯrⓝɣɤɣɯrɣɯr} 
\classe{vi} \end{sous-entrée}

\sens{1}
\begin{définition}\pfra{animé, bruyant}\end{définition}
\begin{définition}\pcmn{嘈杂(声音);闹哄哄;熙熙攘攘}\end{définition}
\begin{exemple}\pjya{tɯrme ra ɲɯ-ɣɤɣɯrɣɯr-nɯ}\hspace{5pt}\pcmn{人们很吵}\end{exemple}
\begin{exemple}\pjya{ɲɯ-ɣɤɕqali-nɯ tɕe ɲɯ-ɣɤɣɯrɣɯr-nɯ}\hspace{5pt}\pcmn{他们在吼叫,很吵}\end{exemple}\sens{2}
\begin{définition}\pfra{ardent (feu)}\end{définition}
\begin{définition}\pcmn{旺盛(火)}\end{définition}
\begin{exemple}\pjya{smi ɲɯ-ɣɤɣɯrɣɯr ɲɯ-nɯt}\hspace{5pt}\pcmn{火烧得很旺盛}\end{exemple}\end{entrée}

\begin{entrée}{ɣɯri}{}{ⓔɣɯri} 
\classe{vt}  
\grammaire{denom} \sens{1}\paradigme{dir}{thɯ-}
\begin{définition}\pfra{insérer des perles sur un fil}\end{définition}
\begin{définition}\pcmn{穿珠}\end{définition}
\begin{exemple}\pjya{tɤ-rɣe thɯ-ɣɯri}\hspace{5pt}\pcmn{你穿珠子吧!}\end{exemple}
\begin{exemple}\pjya{rɤjndoʁ (kɯ-fse) cho-ɣɯri}\hspace{5pt}\pcmn{他穿了芜菁根(用杨柳条子穿,穿了以后晒干)}\end{exemple}\sens{2}\paradigme{dir}{nɯ-}
\begin{définition}\pfra{mettre un fil dans le chas d'une aiguille}\end{définition}
\begin{définition}\pcmn{穿针}\end{définition}
\begin{exemple}\pjya{taqaβrna nɯ-ɣɯri-t-a}\hspace{5pt}\pcmn{我穿了针}\end{exemple}\relationsémantique{参考}{\lien{ⓔtɤ-ri}{tɤ-ri}}\end{entrée}

\begin{entrée}{ɣɯrɟɤn}{}{ⓔɣɯrɟɤn} 
\classe{vt} \paradigme{dir}{tɤ-}\paradigme{dir}{kɤ-}
\begin{définition}\pfra{décoré}\end{définition}
\begin{définition}\pcmn{装饰}\end{définition}
\begin{exemple}\pjya{kha nɯ kɯ-wɣrum to-lɤt-nɯ tɕe to-ɣɯrɟɤn-nɯ}\hspace{5pt}\pcmn{他们把家涂成白色,这样装饰了}\end{exemple}
\begin{exemple}\pjya{si ɯ-taʁ qarma ɯ-muj ko-tshoʁ-nɯ tɕe ko-ɣɯrɟɤn-nɯ}\hspace{5pt}\pcmn{他们用马鸡的羽毛装饰了枝桠}\end{exemple}\end{entrée}

\begin{entrée}{ɣɯrɟɯfsoʁ/\variante{rɯrɟɯfsoʁ}}{}{ⓔɣɯrɟɯfsoʁ} 
\classe{vi}  
\grammaire{incorp} \paradigme{dir}{nɯ-}\paradigme{dir}{tɤ-}
\begin{définition}\pfra{gagner de l'argent}\end{définition}
\begin{définition}\pcmn{挣钱;财产}\end{définition}
\begin{exemple}\pjya{nɯ-ɣɯrɟɯfsoʁ-a}\hspace{5pt}\pcmn{我挣了钱}\end{exemple}\relationsémantique{参考}{\lien{ⓔrɟɯfsoʁ}{rɟɯfsoʁ}}\end{entrée}

\begin{entrée}{ɣɯrni}{}{ⓔɣɯrni} 
\classe{vs} \paradigme{dir}{nɯ-}\paradigme{dir}{thɯ-}
\begin{définition}\pfra{rouge}\end{définition}
\begin{définition}\pcmn{红}\end{définition}
\begin{exemple}\pjya{a-rŋa ɯ-ɲó-ɣɯrni}\hspace{5pt}\pcmn{我的脸变红了吗}\end{exemple}
\begin{exemple}\pjya{tɤ-se pjɤ-ɣɯrni}\hspace{5pt}\pcmn{出血了}\end{exemple}
\begin{sous-entrée}{zɣɯrni}{ⓔɣɯrniⓝzɣɯrni} 
\classe{vt}  
\grammaire{caus} 
\begin{définition}\pfra{rendre rouge}\end{définition}
\begin{définition}\pcmn{使其变红}\end{définition}
\begin{exemple}\pjya{nɯ-zɣɯrni-t-a}\hspace{5pt}\pcmn{他令它变红了}\end{exemple}\end{sous-entrée}

\end{entrée}

\begin{entrée}{ɣɯrŋi}{}{ⓔɣɯrŋi} 
\classe{vs} 
\begin{définition}\pfra{vert (bois)}\end{définition}
\begin{définition}\pcmn{未干;生(材)}\end{définition}
\begin{exemple}\pjya{si ɲɯ-ɣɯrŋi}\hspace{5pt}\pcmn{柴没干}\end{exemple}\relationsémantique{反义词}{\lien{ⓔrom}{rom}}\end{entrée}

\begin{entrée}{ɣɯscur}{}{ⓔɣɯscur} 
\classe{vt} \paradigme{dir}{tɤ-}
\begin{définition}\pfra{tenir dans les deux mains}\end{définition}
\begin{définition}\pcmn{捧}\end{définition}
\begin{exemple}\pjya{mbrɤz pjɤ-ʁndɤr tɕe tɤ-ɣɯscur-a}\hspace{5pt}\pcmn{米撒了,我就捧起来了}\end{exemple}
\begin{exemple}\pjya{tɤ-ŋgɯm a-mɤ-pɯ-ɴɢrɯ ɲɯ-ra tɕe, tɤ-ɣɯscur-a}\hspace{5pt}\pcmn{为了不要把鸡蛋打烂,我把它捧起了}\end{exemple}\end{entrée}

\begin{entrée}{ɣɯsɯphɯt}{}{ⓔɣɯsɯphɯt} 
\classe{vi}  
\grammaire{incorp} \paradigme{dir}{pɯ-}\paradigme{dir}{lɤ-}
\begin{définition}\pfra{couper du bois pour faire du feu}\end{définition}
\begin{définition}\pcmn{打柴;砍柴}\end{définition}
\begin{exemple}\pjya{Yingchun kɯ-ɣɯsɯphɯt jo-ɕe}\hspace{5pt}\pcmn{迎春去砍柴了}\end{exemple}
\begin{exemple}\pjya{jisŋi pɯ-ɣɯsɯphɯt-a}\hspace{5pt}\pcmn{我今天砍了柴}\end{exemple}
\begin{exemple}\pjya{nɯʑo ɯ-tɯ́-ɣɯsɯphɯt-nɯ}\hspace{5pt}\pcmn{你们要砍柴吗?}\end{exemple}
\begin{exemple}\pjya{ɕɯ-ɣɯsɯphɯt-a ŋu}\hspace{5pt}\pcmn{我去砍柴}\end{exemple}\relationsémantique{参考}{\lien{ⓔsɯphɯt}{sɯphɯt}}\end{entrée}

\begin{entrée}{ɣɯt}{}{ⓔɣɯt} 
\classe{vt} \paradigme{dir}{jɤ-}
\begin{définition}\pfra{amener}\end{définition}
\begin{définition}\pcmn{带来}\end{définition}
\begin{exemple}\pjya{tɯji kɯ tɤ-rɤku tɤ-kɤ-ɣɯt nɯ pe}\hspace{5pt}\pcmn{那块地里长出的庄稼好}\end{exemple}
\begin{exemple}\pjya{nɤki tɯji kɯ tɤ-rɤku kɯ-pɯ-pe ʑo to-ɣɯt}\hspace{5pt}\pcmn{这块地长出了很好的庄稼}\end{exemple}
\begin{exemple}\pjya{@chezi ɯ-ŋgɯ kɤ-mdzɯt mɤ-cha ma @yunche ɯ-tɯ-βzu saχaʁ ʑo qhe, tɕe kɤ-ɣɯt mɤ-sna wo.}\hspace{5pt}\pcmn{她不能坐车,因为晕车晕得很厉害,所以不能把她带来}\end{exemple}
\begin{exemple}\pjya{tɤ-scoz ɯ-kɯ-ɣɯt ci pɯ-tu}\hspace{5pt}\pcmn{有人送信了}\end{exemple}
\begin{exemple}\pjya{ku-ɣɯt-a ɕi? ɲɯ-pe, a-kɤ-tɯ-ɣɯt}\hspace{5pt}\pcmn{我要不要带来?好的,带来吧}\end{exemple}\relationsémantique{参考}{\lien{ⓔɣi}{ɣi}}\end{entrée}

\begin{entrée}{ɣɯtɕha}{}{ⓔɣɯtɕha} 
\classe{vt}  
\grammaire{denom} \paradigme{dir}{nɯ-}
\begin{définition}\pfra{répondre}\end{définition}
\begin{définition}\pcmn{回答}\end{définition}
\begin{exemple}\pjya{jiɕqha tɤ-tɯ-thu-t nɯ nɯ-ta-ɣɯtɕha}\hspace{5pt}\pcmn{我已经回答了你刚才问的那些问题}\end{exemple}\relationsémantique{参考}{\lien{ⓔtɯ-tɕhaⓗ1}{tɯ-tɕha₁}}\end{entrée}

\begin{entrée}{ɣɯthaʁ}{}{ⓔɣɯthaʁ} 
\classe{vt} \paradigme{dir}{nɯ-}
\begin{définition}\pfra{séparer deux objets en mettant un autre objet entre eux}\end{définition}
\begin{définition}\pcmn{在中间放一个东西隔开}\end{définition}
\begin{exemple}\pjya{nɯ-ɣɯthaʁ-a}\hspace{5pt}\pcmn{我隔开了;我分开了}\end{exemple}
\begin{exemple}\pjya{tɤrɤm ɯ-pɤrthɤβ tɕe ndʑu ci pjɯ́-wɣ-rku tɕe ɲɯ́-wɣ-z-ɣɯthaʁ ŋu}\hspace{5pt}\pcmn{在两个木板中间插了一根木棒把木板隔开}\end{exemple}
\begin{exemple}\pjya{mbɣɤsroʁ nɯ kɯ mbɣɤru cho mbɣopɤl ɲɯ́-wɣ-z-ɣɯthaʁ ra}\hspace{5pt}\pcmn{木棒要把犁干和犁头两边顶住(使它们俩稳固)}\end{exemple}\end{entrée}

\begin{entrée}{ɣɯtshɤdɯɣ}{}{ⓔɣɯtshɤdɯɣ} 
\classe{vi} \paradigme{dir}{thɯ-}
\begin{définition}\pfra{chaud (temps)}\end{définition}
\begin{définition}\pcmn{很热(天气)}\end{définition}
\begin{exemple}\pjya{tɯ-mɯ ɲɯ-jɯm, ɲɯ-ɣɯtshɤdɯɣ}\hspace{5pt}\pcmn{天晴了,很热}\end{exemple}\étymologie{tsʰa.dug}\end{entrée}

\begin{entrée}{ɣɯtʂɤmtshi}{}{ⓔɣɯtʂɤmtshi} 
\classe{vi}  
\grammaire{incorp} \paradigme{dir}{jɤ-}
\begin{définition}\pfra{conduire le chemin}\end{définition}
\begin{définition}\pcmn{带路;引路}\end{définition}
\begin{exemple}\pjya{jo-ɣɯtʂɤmtshi}\hspace{5pt}\pcmn{他带路了}\end{exemple}
\begin{exemple}\pjya{jɤ-ɣɯtʂɤmtshi-a}\hspace{5pt}\pcmn{我带路了}\end{exemple}
\begin{exemple}\pjya{jiɕqha tʂu mɯ́j-sɯz tɕe, ɯ-kɯ-ɣɯtʂɤmtshi jo-ɕe}\hspace{5pt}\pcmn{他不认识路,给他带路了}\end{exemple}\relationsémantique{参考}{\lien{ⓔtʂu}{tʂu}}\relationsémantique{参考}{\lien{ⓔmtshi}{mtshi}}\relationsémantique{参考}{\lien{ⓔtʂɤmtshi}{tʂɤmtshi}}\end{entrée}

\begin{entrée}{ɣɯtʂhɤtshi}{}{ⓔɣɯtʂhɤtshi} 
\classe{vi}  
\grammaire{incorp} \paradigme{dir}{kɤ-}\paradigme{dir}{pɯ-}
\begin{définition}\pfra{boire du thé}\end{définition}
\begin{définition}\pcmn{喝茶}\end{définition}
\begin{exemple}\pjya{jisŋi rcanɯ pɯ-ɣɯtʂhɤtshi-tɕi ko}\hspace{5pt}\pcmn{我们今天喝了茶}\end{exemple}
\begin{exemple}\pjya{tshi}\end{exemple}\relationsémantique{参考}{\lien{ⓔtʂha}{tʂha}}\end{entrée}

\begin{entrée}{ɣɯχsɤl}{}{ⓔɣɯχsɤl} 
\classe{vt} \paradigme{dir}{tɤ-}
\begin{définition}\pfra{prouver, réfuter}\end{définition}
\begin{définition}\pcmn{证实;用证据来反驳错误的说法}\end{définition}
\begin{exemple}\pjya{jɯfɕɯr ji-ŋgumdʑɯɣ kɯ "mɯ-to-tɯ-sɤpe-t" ɲɯ-ti tɕe, jisŋi tɕe aʑo tɤ-nɯ-ɣɯχsal-a}\hspace{5pt}\pcmn{昨天领导说我做错了事,今天我就证明了我没有错}\end{exemple}
\begin{exemple}\pjya{kɯki tɤ-ndɤm tɕe, ɯ-sɤz-ɣɯχsɤl a-pɯ-ŋu}\hspace{5pt}\pcmn{你拿着当做证据}\end{exemple}\relationsémantique{参考}{\lien{}{χsɤl}}\end{entrée}

\begin{entrée}{ɣzɤn}{}{ⓔɣzɤn} 
\classe{n} 
\begin{définition}\pfra{appât, leurre}\end{définition}
\begin{définition}\pcmn{鱼饵;诱饵}\end{définition}
\begin{exemple}\pjya{ɯ-fsa ɯ-ŋgɯ ɯ-ɣzɤn pɯ-pɯ-me nɤ mɤ-ɕe}\hspace{5pt}\pcmn{如果在陷阱里没有诱饵,(动物)是不会进去的}\end{exemple}\étymologie{gzan}\end{entrée}

\begin{entrée}{ɣzɯ}{}{ⓔɣzɯ} 
\classe{n} 
\begin{définition}\pfra{singe}\end{définition}
\begin{définition}\pcmn{猴子}\end{définition}\end{entrée}

\begin{entrée}{ɣzɯlu}{}{ⓔɣzɯlu} 
\classe{n} 
\begin{définition}\pfra{année du singe}\end{définition}
\begin{définition}\pcmn{猴年}\end{définition}\relationsémantique{参考}{\lien{ⓔɣzɯ}{ɣzɯ}}\end{entrée}

\begin{entrée}{ɣzɯɬa/\variante{βzɯɬa}}{}{ⓔɣzɯɬa} 
\classe{n} 
\begin{définition}\pfra{pika}\end{définition}
\begin{définition}\pcmn{崖兔【岩兔儿】}\end{définition}
\begin{exemple}\pjya{ɣzɯɬa nɯ zndɤrchɤβ cho praʁ ɯ-rchɤβ ku-rɤʑi ŋu, ɯ-rme ɯ-mdoʁ nɯ βʑɯ ɯ-mdoʁ asɯndo, kɯ-pɣi ŋu. ɯ-tshɯɣa nɯ ra qala fse, ɯ-rna rɲɟi, ɯ-mtɕhi amtɕoʁ ɯ-jme xtɯt, ɯ-jaʁ xtɯt, ɯ-mi rɲɟi, xɕaj tu-ndze ŋu, tɯ-ji ɯ-rkɯ tɤ-rɤku ra tu-ndze ŋgrɤl, ɯ-zda ra nɯ ɯ-taʁ mɤ-ʁnɤt, ɯʑo ɯ-kɯ-ndza dɤn.}\hspace{5pt}\pcmn{岩兔住在墙壁缝里和岩洞里,毛的颜色和老鼠的一样,是灰色的。形状像兔子,耳朵长,嘴尖,尾巴短,前腿短,后腿长,吃草,也吃地里的庄稼。不危害其它动物,但是是很多动物的猎物。}\end{exemple}\end{entrée}

\begin{entrée}{ɣzɯthɯz}{}{ⓔɣzɯthɯz} 
\classe{n} 
\begin{définition}\pfra{Selaginella sp.}\end{définition}
\begin{définition}\pcmn{卷柏}\end{définition}
\begin{exemple}\pjya{ɣzɯthɯz nɯ praʁ ɯ-taʁ ku-ndzoʁ ŋu, cɤmi pɕoʁ sɤtɕha kɯ-mpja nɯ tɕu tu-ɬoʁ ɲɯ-ŋu. praʁ ɯ-taʁ ku-oɲɟoʁ ku-fse tu-ɬoʁ, tɕe ɯ-ru maŋe, ɯ-mɯntoʁ me, ɯ-jwaʁ nɯ ɕɤɣ ɣɯ ɯ-jwaʁ tsa ɲɯ-fse, ɯ-jwaʁ ɯ-χcɤl ɯ-ŋgɯ chu tu-ŋgɤɣ, tɕe ɯ-χcɤl pɕoʁ ku-wum kɯ-fse ɲɯ-ŋu. ɯ-mdoʁ nɯ ftɕar ɲɯ-ɤrŋi, qartsɯ ɲɯ-pɣi ɲɯ-ŋu ri, ɲɯ-rom mɯ́j-cha. kɯɕɯŋgɯ tɕe, ɲɯ-phɯt-nɯ tɕe, tɯ-thɯ ɯ-sɤ-χtɕi tu-βzu-nɯ pɯ-ŋu. tham tɕe ɯ-kɯ-ntɕhoz me.}\hspace{5pt}\pcmn{卷柏生长在岩石上,在河坝气候温和的地方生长。长的样子好像是贴着岩石,没有茎,没有花,叶子有些类似柏树叶,叶子中间部分往里卷,合在一起。叶子的颜色夏绿冬灰,但不会干枯。以前,人们拆下来用来洗锅子,现在就没有人用了。}\end{exemple}\end{entrée}

\begin{entrée}{ɣʑɤndza}{}{ⓔɣʑɤndza} 
\classe{n} 
\begin{définition}\pfra{Agastache rugosa}\end{définition}
\begin{définition}\pcmn{藿香}\end{définition}
\begin{exemple}\pjya{ɣʑɤndza nɯ sɯjno ŋu, mɤ-mbro, tɤ-rɤku ɯ-rchɤβ kɤ-ɬoʁ rga ɯ-mdoʁ kɯ-ɤrŋi ɯ-ŋgɯz kɯnɤ kɯ-ɤɲaʁndzɯm tsa ŋu, ɯ-rtaʁ dɤn, ɯ-mɯntoʁ nɯ phaʁrzi ɯ-tshɯɣa kɯ-fse tu, kɯ-ɤlɯlju ɯ-tshɯɣa tu, mɯntoʁ dɤn, ɣʑo wuma ʑo rga tɕe ɣʑɤndza rmi, ɯ-di wuma ʑo χɕɤβ}\hspace{5pt}\pcmn{藿香是一种植物,长得不高,生长在田地之间,绿色里面带有暗绿色,很多枝桠,花的形状像牙刷,是圆柱形的,花很多,蜜蜂特别喜欢,所以叫\lien{ⓔɣʑɤndza}{ɣʑɤndza}(蜜蜂的食物),味道很浓。}\end{exemple}\end{entrée}

\begin{entrée}{ɣʑɤzga}{}{ⓔɣʑɤzga} 
\classe{n} 
\begin{définition}\pfra{miel}\end{définition}
\begin{définition}\pcmn{蜂蜜}\end{définition}
\begin{sous-entrée}{ɣʑɤzga ɯ-rqhu}{ⓔɣʑɤzgaⓝɣʑɤzga ɯ-rqhu} 
\classe{n} 
\begin{définition}\pfra{alvéole}\end{définition}
\begin{définition}\pcmn{蜂房}\end{définition}
\begin{exemple}\pjya{ɣʑɤzga pɯ-phɯt-a / kɤ-tɕat-a}\hspace{5pt}\pcmn{我找了蜂蜜}\end{exemple}\end{sous-entrée}

\end{entrée}

\begin{entrée}{ɣʑo}{}{ⓔɣʑo} 
\classe{n} 
\begin{définition}\pfra{abeille}\end{définition}
\begin{définition}\pcmn{蜜蜂}\end{définition}
\begin{exemple}\pjya{ɣʑo nɯ kɤntɕhɯ-tɯphu tu, ɯ-zga kɯ-tu ci tu, sɤtɕha ɣʑo kɤ-ti ci tu, ndzɯrnaʁ kɤ-ti ci tu, li nɯ ɣʑo ɕti, tɕe ɯ-zga kɯ-tu nɯ khro mɤ-wxti tɕe tɯ-khɤl kɯ-dɯdɤn ʑo ku-rɤʑi ɕti, ftɕar mɯntoʁ nɯ-ʁaʁ tɕe, nɯ-tɯ-mbɣom saχaʁ, tɕe ku-rɤzga-nɯ ŋu, tɕe khɤrka ri tu-ndzoʁ-nɯ tɕe, nɯ-zga ku-tshoʁ-nɯ ŋu tɕe tɯ-xpa tɕe zgo kɯ-fse nɯ kɯngɯsqi jamar ku-tshoʁ ɲɯ-cha, tɕe pjɯ́-wɣ-phɯt tɕe, ɕnɤcat tɯ-rpa jamar tu-ŋgrɤl tɕe ɣʑo nɯ ɯ-ku kɯ-ɤmtɕoʁ ci ɯ-mthɤɣ kɯ-xtshɯm ci ɯ-xtu kɯ-wxti tsa ci ŋu, ɯ-ʁar kɯ-nɤmbju kɯ-mbɯ-mba ci ŋu ɯ-rɯmu tu, ɯ-mɤlɤjaʁ kɯtʂɤ-ldʑa tu, tɕeri sɤmtsɯɣ tɕe ɯ-mdzu nɯ ɯ-mphɯz ri ku-ndzoʁ ŋu, tɕe kɤ-kɯ-mtsɯɣ tɕe, ɯʑo ju-nɯɕe ŋu ri, ɯ-mdzu nɯ tɯ-ɕa ɯ-ŋgɯ ku-raʁ tɕe ɯ-mphɯz ɯ-ntɕhɯr ɲɯ-nɯ-phɯt ŋu tɕe ɯʑo pjɯ-si ŋu tu-kɯ-ti ŋu. tɕe kɤ-kɯ-mtsɯɣ tɕe aɣɯtɯɣ tɕe tɯ-ɕa ra ɯ-kho kɯ-jom chɯ-z-nɯɣmbɤβ cha. ɯ-zga nɯ wuma ʑo chi, smɤn kɯ-sna ɲɯ-ŋu. sɤtɕha ɣʑo nɯ sɤtɕha ɯ-ŋgɯ kɯ-spoʁ kɯ-tu nɯ tɕu ku-rɤʑi ɲɯ-ŋu, tɕe ɲɯ-dɤn.}\hspace{5pt}\pcmn{蜂有几种,一种叫蜜蜂,另一种住在地洞里,还有一种叫马蜂,也是蜂的一种。蜜蜂不是很大,一个地方住着很多只。春天花开的时候,它们忙着酿蜜。它们在天花板上筑巢,在那里酿蜜。一年可以产九十排蜂蜜,取下来时,至少都有七八斤。蜜蜂头部尖、腰部细、肚子大,翅膀很薄、有光泽、有纹路。有六只脚,蜇人,刺位于尾部。蜇人时,把刺插进人的皮肤后,自己飞走,但刺卡住在皮肤里,尾部的一块会留在刺上,据说蜜蜂会因而丧命。蜇后因为毒性大,皮肉上会起比较大的包。蜂蜜很甜,是一种好药。另一种蜂(\lien{}{sɤtɕha ɣʑo})成群住在地洞里。}\end{exemple}\end{entrée}

\begin{entrée}{ɣʑokha}{}{ⓔɣʑokha} 
\classe{n} 
\begin{définition}\pfra{ruche}\end{définition}
\begin{définition}\pcmn{蜂窝}\end{définition}\end{entrée}

\newpage\caractère{h}

\begin{entrée}{hu}{}{ⓔhu} 
\classe{n} 
\begin{définition}\pfra{souffle}\end{définition}
\begin{définition}\pcmn{吹热气}\end{définition}
\begin{exemple}\pjya{hu nɤ hu ʑo tɤ-tɯt-a tɕe nɯ-ɣɤmpja-t-a}\hspace{5pt}\pcmn{我吹了几下取暖}\end{exemple}\end{entrée}

\begin{entrée}{hajtsu}{}{ⓔhajtsu} 
\classe{n} 
\begin{définition}\pfra{piment}\end{définition}
\begin{définition}\pcmn{辣椒}\end{définition}\end{entrée}

\begin{entrée}{hanɯni}{}{ⓔhanɯni} 
\classe{adv} 
\begin{définition}\pfra{un peu}\end{définition}
\begin{définition}\pcmn{稍微}\end{définition}\end{entrée}

\begin{entrée}{hatsɯtsi}{}{ⓔhatsɯtsi} 
\classe{adv} 
\begin{définition}\pfra{un peu}\end{définition}
\begin{définition}\pcmn{稍微,一点点}\end{définition}\end{entrée}

\begin{entrée}{hwɤl}{}{ⓔhwɤl} 
\classe{idph.3} 
\begin{définition}\pfra{qui est découvert d'un seul coup}\end{définition}
\begin{définition}\pcmn{形容一下子被掀开的样子}\end{définition}
\begin{exemple}\pjya{qale ɲɯ-ɤsɯ-βzu tɕe, a-tʂɯmpa hwɤl ʑo ta-pɣaʁ}\hspace{5pt}\pcmn{风把我的围腰帕一下子吹开了}\end{exemple}\end{entrée}

\begin{entrée}{hwɤrhwɤr}{}{ⓔhwɤrhwɤr} 
\classe{idph.2} 
\begin{définition}\pfra{évasé}\end{définition}
\begin{définition}\pcmn{形容口朝外展开的样子}\end{définition}
\begin{exemple}\pjya{ki a-rte ki hwɤrhwɤr ʑo ɲɯ-pa}\hspace{5pt}\pcmn{我的帽子的口朝外展开}\end{exemple}\relationsémantique{同义词}{\lien{ⓔwɤrwɤr}{wɤrwɤr}}\end{entrée}

\newpage\caractère{j}

\begin{entrée}{ja}{₁}{ⓔjaⓗ1} 
\classe{vt} \sens{1}\paradigme{dir}{kɤ-}
\begin{définition}\pfra{enfermer}\end{définition}
\begin{définition}\pcmn{关}\end{définition}
\begin{exemple}\pjya{nɯŋa kɤ-je}\hspace{5pt}\pcmn{你把牛关起来吧}\end{exemple}
\begin{exemple}\pjya{paʁ kɤ-je}\hspace{5pt}\pcmn{你把猪关起来吧}\end{exemple}
\begin{exemple}\pjya{qaʑo kɤ-ja-t-a}\hspace{5pt}\pcmn{我把羊关起来了}\end{exemple}\relationsémantique{参考}{\lien{ⓔnɤrkɤja}{nɤrkɤja}}\sens{2}\paradigme{dir}{lɤ-}
\begin{définition}\pfra{amasser les grains}\end{définition}
\begin{définition}\pcmn{把粮食堆在仓库里}\end{définition}
\begin{exemple}\pjya{tɤɕi lɤ-ja-t-a}\hspace{5pt}\pcmn{我把青稞堆在仓库里了}\end{exemple}
\begin{sous-entrée}{aja}{ⓔjaⓗ1ⓢ2ⓝaja} 
\classe{vi}  
\grammaire{pass} 
\begin{définition}\pfra{être enfermé}\end{définition}
\begin{définition}\pcmn{被关(牲畜)}\end{définition}
\begin{exemple}\pjya{nɯtɕu aja}\hspace{5pt}\pcmn{被关在那里}\end{exemple}\end{sous-entrée}

\end{entrée}

\begin{entrée}{ja}{₂}{ⓔjaⓗ2} 
\classe{vs} \paradigme{dir}{kɤ-}
\begin{définition}\pfra{se faner}\end{définition}
\begin{définition}\pcmn{凋谢}\end{définition}
\begin{exemple}\pjya{ɯ-mɯntoʁ ko-ja}\hspace{5pt}\pcmn{花凋谢了}\end{exemple}\relationsémantique{同义词}{\lien{}{rŋil}}\end{entrée}

\begin{entrée}{jaftɕɯn}{}{ⓔjaftɕɯn} 
\classe{n} 
\begin{définition}\pfra{étriers}\end{définition}
\begin{définition}\pcmn{马镫}\end{définition}\étymologie{job.tɕan}\end{entrée}

\begin{entrée}{jaftɕɯnkha}{}{ⓔjaftɕɯnkha} 
\classe{n} 
\begin{définition}\pfra{cheville}\end{définition}
\begin{définition}\pcmn{踝关节}\end{définition}\end{entrée}

\begin{entrée}{jamar}{}{ⓔjamar} 
\classe{adv} \sens{1}
\begin{définition}\pfra{environ}\end{définition}
\begin{définition}\pcmn{左右}\end{définition}\sens{2}
\begin{définition}\pfra{à ce moment}\end{définition}
\begin{définition}\pcmn{在那个时候}\end{définition}\étymologie{jar.mar}\end{entrée}

\begin{entrée}{jaŋjaŋ}{}{ⓔjaŋjaŋ} 
\classe{idph.2} \sens{1}
\begin{définition}\pfra{calme}\end{définition}
\begin{définition}\pcmn{安静}\end{définition}\sens{2}
\begin{définition}\pfra{complètement plat}\end{définition}
\begin{définition}\pcmn{又平坦又宽阔(田地、草地)}\end{définition}
\begin{exemple}\pjya{jisŋi kɯ-ɣɤɕqali ri maŋe tɕe, jaŋjaŋ ʑo ɲɯ-pa}\hspace{5pt}\pcmn{今天没有人大声说话,很安静}\end{exemple}
\begin{exemple}\pjya{tɯji jaŋjaŋ ʑo ɲɯ-pa, ɲɯ-jom}\hspace{5pt}\pcmn{田地很宽阔}\end{exemple}\end{entrée}

\begin{entrée}{jaŋntsɤrpa}{}{ⓔjaŋntsɤrpa} 
\classe{n} 
\begin{définition}\pfra{hache que l'on peut tenir d'une seule main}\end{définition}
\begin{définition}\pcmn{单手斧头}\end{définition}\end{entrée}

\begin{entrée}{jaŋpho}{}{ⓔjaŋpho} 
\classe{n} 
\begin{définition}\pfra{fusil}\end{définition}
\begin{définition}\pcmn{枪}\end{définition}\étymologie{fn:洋炮}\end{entrée}

\begin{entrée}{japa}{}{ⓔjapa} 
\classe{adv} 
\begin{définition}\pfra{l'année dernière}\end{définition}
\begin{définition}\pcmn{去年}\end{définition}
\begin{exemple}\pjya{japa to-mɯɕtaʁ tsa ma ɣɯjpa mɯ́j-mɯɕtaʁ}\end{exemple}\end{entrée}

\begin{entrée}{japandʐi}{}{ⓔjapandʐi} 
\classe{n} 
\begin{définition}\pfra{l'année d'avant}\end{définition}
\begin{définition}\pcmn{前年}\end{définition}\end{entrée}

\begin{entrée}{jaqhɤrŋgɤβ,lɤt}{}{ⓔjaqhɤrŋgɤβ,lɤt} 
\classe{n}
\classe{n}
\classe{vt} \paradigme{dir}{nɯ-}
\begin{définition}\pfra{attacher les mains derrière le dos}\end{définition}
\begin{définition}\pcmn{把手捆在背后}\end{définition}
\begin{exemple}\pjya{aʑo jaqhɤrŋgɤβ nɯ́-wɣ-lat-a-nɯ}\hspace{5pt}\pcmn{他们把我的手捆在背后了}\end{exemple}
\begin{exemple}\pjya{a-jaqhɤrŋgɤβ na-lɤt-nɯ}\hspace{5pt}\pcmn{他们把我的手捆在背后了}\end{exemple}\relationsémantique{Component 1}{\lien{}{jaqhɤrŋgɤβ}}\relationsémantique{Component 2}{\lien{}{lɤt}}\relationsémantique{参考}{\lien{ⓔlɤtⓗ1}{lɤt₁}}\end{entrée}

\begin{entrée}{jaramara}{}{ⓔjaramara} 
\classe{n} 
\begin{définition}\pfra{chercher à gagner de l'argent par tout les moyens}\end{définition}
\begin{définition}\pcmn{拉关系,到处想办法赚钱}\end{définition}
\begin{exemple}\pjya{jaramara to-βzu-j}\hspace{5pt}\pcmn{我们拉了关系想办法到处赚钱}\end{exemple}\étymologie{jar.mar}\end{entrée}

\begin{entrée}{jaʁ}{}{ⓔjaʁ} 
\classe{vs} \paradigme{dir}{tɤ-}\paradigme{dir}{tɤ-}
\begin{définition}\pfra{épais}\end{définition}
\begin{définition}\pcmn{厚}\end{définition}
\begin{exemple}\pjya{tɤjpa ɲɯ-jaʁ}\hspace{5pt}\pcmn{雪很厚}\end{exemple}
\begin{exemple}\pjya{tɯ-ŋga tú-wɣ-ɣɤjaʁ ma ɲɯ-ɣɤndʐo}\hspace{5pt}\pcmn{衣服要穿厚一些,天气很冷}\end{exemple}
\begin{exemple}\pjya{a-ŋga tɤ-ɣɤjaʁ-a}\hspace{5pt}\pcmn{我衣服穿厚一些}\end{exemple}\relationsémantique{参考}{\lien{ⓔnɤjaʁⓗ2}{nɤjaʁ₂}}\relationsémantique{反义词}{\lien{ⓔmba}{mba}}
\begin{sous-entrée}{ɣɤjaʁ}{ⓔjaʁⓝɣɤjaʁ} 
\classe{vt}  
\grammaire{caus} \end{sous-entrée}

\end{entrée}

\begin{entrée}{jaʁjɯ}{}{ⓔjaʁjɯ} 
\classe{vs} \paradigme{dir}{tɤ-}\paradigme{dir}{thɯ-}
\begin{définition}\pfra{épais et résistant}\end{définition}
\begin{définition}\pcmn{厚实}\end{définition}
\begin{exemple}\pjya{ŋgɤjpan kɯ-jaʁjɯ ci ɲɯ-ŋu}\hspace{5pt}\pcmn{这是厚实的木板}\end{exemple}\relationsémantique{参考}{\lien{ⓔjaʁ}{jaʁ}}\end{entrée}

\begin{entrée}{jaʁlu}{}{ⓔjaʁlu} 
\classe{n} 
\begin{définition}\pfra{manchot}\end{définition}
\begin{définition}\pcmn{独臂的人}\end{définition}\end{entrée}

\begin{entrée}{jaʁmɤzdoʁzdoʁ}{}{ⓔjaʁmɤzdoʁzdoʁ} 
\classe{n} 
\begin{définition}\pfra{espèce d'oiseau}\end{définition}
\begin{définition}\pcmn{一种鸟}\end{définition}
\begin{exemple}\pjya{jarmɤzdoʁzdoʁ nɯ pɣɤtɕɯ kɯ-xtɕɯ-xtɕi ci ŋu, ɯ-mdoʁ pɣi, ɯ-ʁar ɯ-rkɯ ra hanɯni ɲaʁ, ɲɯ-nɯqambɯmbjom tɕe wuma ʑo ɣɤji, ɯ-mtsioʁ kɯnɤ kɯ-xtɯ-xtɯt ma me, qajɯ tu-ndze ŋu, ɯʑo sti ma kɤ-mto me. ɯ-mi nɯ ra kɯ-pɣi ɯ-ŋgɯz kɯ-qarŋɯrŋe tsa ŋu.}\hspace{5pt}\pcmn{\lien{}{jarmɤzdoʁzdoʁ}是很小的鸟,灰色,翅膀边缘略黑。起飞非常快,嘴很短,吃虫子。只能看见它单独飞行。脚也是黑里带黄的。}\end{exemple}\end{entrée}

\begin{entrée}{jaʁmba}{}{ⓔjaʁmba} 
\classe{n} 
\begin{définition}\pfra{épaisseur}\end{définition}
\begin{définition}\pcmn{厚薄;厚度}\end{définition}\relationsémantique{参考}{\lien{ⓔjaʁ}{jaʁ}}\relationsémantique{参考}{\lien{ⓔmba}{mba}}\end{entrée}

\begin{entrée}{jasa,ta}{}{ⓔjasa,ta} 
\classe{vt}
\classe{n}
\classe{vt} \paradigme{dir}{tɤ-}
\begin{définition}\pfra{respect}\end{définition}
\begin{définition}\pcmn{尊重人}\end{définition}
\begin{exemple}\pjya{tɯrme kɯ-wxti ra jasa tú-wɣ-ta ra}\hspace{5pt}\pcmn{要尊重长辈}\end{exemple}\relationsémantique{同义词}{\lien{ⓔɣɤʁreⓝzɣɤʁre}{zɣɤʁre}}\relationsémantique{Component 1}{\lien{}{jasa}}\relationsémantique{Component 2}{\lien{ⓔta}{ta}}\end{entrée}

\begin{entrée}{jaχpɤtar}{}{ⓔjaχpɤtar} 
\classe{n} 
\begin{définition}\pfra{frapper dans mains, applaudir}\end{définition}
\begin{définition}\pcmn{拍掌,鼓掌}\end{définition}
\begin{exemple}\pjya{jaχpɤtar to-lɤt-nɯ}\hspace{5pt}\pcmn{他们鼓掌了}\end{exemple}\relationsémantique{参考}{\lien{ⓔtɤtar}{tɤtar}}\end{entrée}

\begin{entrée}{jɤβjɤβ}{}{ⓔjɤβjɤβ} 
\classe{idph.2} 
\begin{définition}\pfra{faible lueur (au lever du jour)}\end{définition}
\begin{définition}\pcmn{蒙蒙亮}\end{définition}
\begin{exemple}\pjya{soz jɤβjɤβ ʑo lɤ-pa tɕe aʑo tɤ-ndzu-a}\hspace{5pt}\pcmn{今天天蒙蒙亮的时候我就出发了}\end{exemple}\end{entrée}

\begin{entrée}{jɤɣ}{}{ⓔjɤɣ} 
\classe{vs} \sens{1}\paradigme{dir}{pɯ-}\paradigme{dir}{tɤ-}
\begin{définition}\pfra{s'accomplir, se finir}\end{définition}
\begin{définition}\pcmn{完成}\end{définition}
\begin{exemple}\pjya{ji-kha pɤjkhu mɯ-tɤ-jɤɣ}\hspace{5pt}\pcmn{我们家在装修,还没有完成}\end{exemple}
\begin{exemple}\pjya{aʑɯɣ pɯ-jɤɣ}\hspace{5pt}\pcmn{我成功了}\end{exemple}
\begin{exemple}\pjya{a-kɤ-nɤma pɯ-jɤɣ}\hspace{5pt}\pcmn{我把工作做完了}\end{exemple}
\begin{exemple}\pjya{ɯ-kɤ-nɤma jɤɣ mɤ-jɤɣ ʑo tɕe li ɲɤ-khrɤt}\hspace{5pt}\pcmn{工作还没有做完又给他布置了(另外一个任务)}\end{exemple}
\begin{exemple}\pjya{ɯ-rju jɤɣ mɤ-jɤɣ ʑo tɕe to-nɤla}\hspace{5pt}\pcmn{话没有说完就答应了}\end{exemple}
\begin{exemple}\pjya{ɯ-pa tɤ-jɤɣ tɕe mɤ-tɯ-mɤrʑaβ mɤ-khɯ}\hspace{5pt}\pcmn{已经商量好了,你不能不嫁人}\end{exemple}\sens{2}
\begin{définition}\pfra{être possible}\end{définition}
\begin{définition}\pcmn{可以}\end{définition}
\begin{exemple}\pjya{tɤ-ti jɤɣ ma a-ʁa tu}\hspace{5pt}\pcmn{你可以说,我有空}\end{exemple}\relationsémantique{参考}{\lien{ⓔsɯɣjɤɣ}{sɯɣjɤɣ}}
\begin{sous-entrée}{mɤ-kɯ-jɤɣ kɯ}{ⓔjɤɣⓢ2ⓝmɤ-kɯ-jɤɣ kɯ} 
\classe{adv} 
\begin{définition}\pfra{non seulement}\end{définition}
\begin{définition}\pcmn{不但}\end{définition}\end{sous-entrée}

\end{entrée}

\begin{entrée}{jɤɣɤrna}{}{ⓔjɤɣɤrna} 
\classe{n} 
\begin{définition}\pfra{partie du balcon où l'on fait sécher la nourriture}\end{définition}
\begin{définition}\pcmn{上走檐(晒粮食用的)}\end{définition}\end{entrée}

\begin{entrée}{jɤɣɤt}{}{ⓔjɤɣɤt} 
\classe{n} 
\begin{définition}\pfra{construction suspendue au deuxième étage des maisons tibétaines, servant à faire sécher la nourriture}\end{définition}
\begin{définition}\pcmn{在藏式房屋二楼的位置上一个向外伸出的悬空部分,外围有木架围着。木架是由几根竖着的木杆和加在上面的许多横杆构成,另外再借着这些垂直穿插着一些细树枝。这个通风的架构可用来晾晒食物或其他物品。【走缘】}\end{définition}
\begin{exemple}\pjya{jɤɣɤt jɤ-ari-a}\hspace{5pt}\pcmn{我上了厕所}\end{exemple}
\begin{exemple}\pjya{jɤɣɤt nɯ znde ɯ-taʁ tɯ-mɢɯt kú-wɣ-sɤtsa tɕe ɯ-taʁ nɯ tɕu romɲa chɯ́-wɣ-lɤt, romɲa nɯ tɯ-mɢɯt cho pjɯ́-wɣ-sɯ-ɤqɤtʂha tɕe pjɯ́-wɣ-ta ra. romɲa ɯ-taʁ nɯ tɕu χɕaʁ chɯ́-wɣ-ta, tɕe nɯ ɯ-taʁ tɕe tɤ-lmɯz chɯ́-wɣ-ta, nɯ ɯ-taʁ tɕe thɤlwa chɯ́-wɣ-lɤt tɕe tɯ-mɢɯt ɯ-ɕnɤz ɯ-taʁ nɯ tɕu, koʑi komɤl tɯ-ldʑa ka kú-wɣ-lɤt, koʑi komɤl cho romɲa ni pjɯ-ɤnɯpɕɯ-pɕoʁ ɲɯ-ra. nɯ ɯ-taʁ nɯ tɕu jɤɣɤt laχtsɯ pjɯ́-wɣ-tshoʁ tɕe, tɯ-mɢɯt tɯ-ldza ɯ-taʁ nɯ tɕu jɤɣɤt laχtsɯ nɯ tɯ-ldʑa pjɯ-tu ra, nɯ ɯ-taʁ nɯ tɕu li koʑi komɤl pjɯ́-wɣ-tshoʁ, nɯ ɯ-taʁ nɯ tɕu, tɤsthoʁsi chɯ́-wɣ-lɤt tɕe tɤsthoʁsi nɯ tɯ-mɢɯt tú-wɣ-z-nɯjɯn tɕe pjɯ́-wɣ-ta, tɯ-ldʑa pjɯ-tu khɯ, ʁnɯ-ldʑa pɯ-tu kɯnɤ khɯ, tɕe jɤɣɤt laχtsɯ raŋri ɯ-taʁ chɯ́-wɣ-ta ra. nɯ ɯ-taʁ nɯ tɕu li romɲa tɤsthoʁsi cho pjɯ́-wɣ-sɯ-ɤqɤtʂha tɕe pjɯ́-wɣ-ta ra, nɯ ɯ-taʁ nɯ tɕu cupa chɯ́-wɣ-ta, tɕe thɤlwa chɯ́-wɣ-lɤt, tɕe pjɯ́-wɣ-ɣnda, tɕe khɤxtu ɲɯ-βze ŋu. tɕe jɤɣɤt laχtsɯ ɣɯ ɯ-kɯ-spoʁ tu tɕe, nɯ tɕu rorʁe ɲɯ́-wɣ-rʁe tɕe, ftɕar tɕe tɯ-ɣro tú-wɣ-sɤro ŋu, stonka tɕe tɤ-rɤku tú-wɣ-sɤro ŋu tɕe rasti kɯnɤ nɯ tɕu tú-wɣ-sɯɣrom khɯ. tɕe jɤɣɤt nɯ ŋkhorwapa ra ɣɯ nɯ-ɲɤm wuma phɤn.}\hspace{5pt}\pcmn{在外墙上斜插小木梁,在上面交叉着放上横梁(作为走缘的地面)。在(横梁)上面铺一层劈好了的木料,再铺上麦草、豌豆草或者枝桠,然后铺上泥土。然后在小木梁的一头放上\lien{ⓔkoʑi}{koʑi}和\lien{ⓔkomɤl}{komɤl}各一条,\lien{ⓔkoʑi}{koʑi} \lien{ⓔkomɤl}{komɤl}和\lien{ⓔromɲa}{romɲa}朝同一个方向。上面插上柱子,每一根\lien{}{tɯ-mɢɯt}上面要有一根柱子。上面又插上\lien{}{koʑi komɤl},在上面就放上\lien{ⓔtɤsthoʁsi}{tɤsthoʁsi},\lien{}{tɤ-sthoʁsi}要顺着\lien{}{tɯ-mɢɯt}的方向装上去,可以是一根一根的,也可以是两根两根的,每根柱子上都要放。在那上面还要交叉着放\lien{ⓔromɲa}{romɲa},上面铺上石板,再铺上泥土,然后把泥土锤紧,也就成了房背。柱子上面有洞,在那里穿上横干,春天可以架饲草,秋天可以架粮食和圆根苗。走缘对农民用处很大。}\end{exemple}\relationsémantique{参考}{\lien{ⓔrɯjɤɣɤt}{rɯjɤɣɤt}}\end{entrée}

\begin{entrée}{jɤlwa}{}{ⓔjɤlwa} 
\classe{n} 
\begin{définition}\pfra{rideau}\end{définition}
\begin{définition}\pcmn{帘}\end{définition}\étymologie{jol.ba}\end{entrée}

\begin{entrée}{jɤmtsa}{}{ⓔjɤmtsa} 
\classe{n} 
\begin{définition}\pfra{casserole en fer}\end{définition}
\begin{définition}\pcmn{炒菜锅(用生铁制成的)}\end{définition}
\begin{exemple}\pjya{jɤmtsa pjɤ-ɲɟɤβ}\hspace{5pt}\pcmn{炒菜锅凹进去了}\end{exemple}\end{entrée}

\begin{entrée}{jɤnlaʁ}{}{ⓔjɤnlaʁ} 
\classe{n} 
\begin{définition}\pfra{nom commun aux balcons et aux terrasses des maisons tibétaines}\end{définition}
\begin{définition}\pcmn{房背、走缘的通称}\end{définition}
\begin{exemple}\pjya{kha ɣɯ jɤɣɤt cho khɤxtu nɯ ra jɤnlaʁ rmi}\hspace{5pt}\pcmn{房子的走缘和房背都叫\lien{ⓔjɤnlaʁ}{jɤnlaʁ}}\end{exemple}\étymologie{jan.lag}\end{entrée}

\begin{entrée}{jɤntɤn}{}{ⓔjɤntɤn} 
\classe{n} 
\begin{définition}\pfra{savoir}\end{définition}
\begin{définition}\pcmn{技术;知识}\end{définition}
\begin{exemple}\pjya{nɤ-jɤntɤn, @qiche kɤ-lɤt tu}\hspace{5pt}\pcmn{你的技术是开汽车}\end{exemple}\relationsémantique{参考}{\lien{ⓔnɯjɤntɤn}{nɯjɤntɤn}}\étymologie{jon.tan}\end{entrée}

\begin{entrée}{jɤŋkhɤphɯt}{}{ⓔjɤŋkhɤphɯt} 
\classe{n} 
\begin{définition}\pfra{coup avec le dos de la main}\end{définition}
\begin{définition}\pcmn{用手背打}\end{définition}
\begin{exemple}\pjya{ɯ-taʁ jɤŋkhɤphɯt ci tɤ-lat-a}\hspace{5pt}\pcmn{我用手背打了一下他}\end{exemple}\end{entrée}

\begin{entrée}{jɤrjɤr}{}{ⓔjɤrjɤr} 
\classe{idph.2} \paradigme{dir}{nɯ-}
\begin{définition}\pfra{allure de qqn portant une lourde charge}\end{définition}
\begin{définition}\pcmn{形容背着很重的东西,很难受的样子;歪歪斜斜的样子}\end{définition}
\begin{définition}\pfra{porter une lourde charge (à plusieurs)}\end{définition}
\begin{définition}\pcmn{(几个人)抬很重的东西}\end{définition}
\begin{exemple}\pjya{jɤrjɤr ʑo ɲɯ-ɤsɯ-ndo}\hspace{5pt}\pcmn{他拿着很重的东西}\end{exemple}
\begin{exemple}\pjya{jɤrjɤr ʑo ɲɯ-rɤʑi}\hspace{5pt}\pcmn{他背着很重的东西站在那里}\end{exemple}
\begin{exemple}\pjya{jiɕqha tɕheme ɲɤ-sɤfɕi ɯ-xtu jɤrjɤr ɲɤ-pa}\hspace{5pt}\pcmn{那个女怀孕了,肚子显得很重}\end{exemple}
\begin{exemple}\pjya{jɤrjɤr ʑo ma-tɯ-ʑɣɤstu}\hspace{5pt}\pcmn{你别做出这么一副笨拙的样子!}\end{exemple}
\begin{exemple}\pjya{ɯ-fkur ɲɯ-nɤrʑi tɕe, jɤrjɤr ʑo tɤ-ʑɣɤstu}\hspace{5pt}\pcmn{他觉得背的东西很重,显得很难受}\end{exemple}
\begin{exemple}\pjya{kɯki lʁa ki ɯ-ŋgɯ ɲɯ-mtshɤt tɕe, nɯ-nɤjɤrjɤr-tɕi tɕe, nɯ-sɤzɣɯt-tɕi}\hspace{5pt}\pcmn{这个袋子满了,我们俩把它抬回来了}\end{exemple}
\begin{sous-entrée}{jɤrnɤjɤr}{ⓔjɤrjɤrⓝjɤrnɤjɤr} 
\classe{idph.3} 
\begin{exemple}\pjya{jɤrjɤr nɤ jɤrjɤr ka-tsɯm}\hspace{5pt}\pcmn{他带走了(很重的东西)}\end{exemple}\end{sous-entrée}

\begin{sous-entrée}{nɤjɤrjɤr}{ⓔjɤrjɤrⓝnɤjɤrjɤr} 
\classe{vt} \end{sous-entrée}

\begin{sous-entrée}{ɣɤjɤrjɤr}{ⓔjɤrjɤrⓝɣɤjɤrjɤr} 
\classe{vi} 
\begin{définition}\pfra{chancelant, instable (lorsque l'on porte une lourde charge)}\end{définition}
\begin{définition}\pcmn{形容摇摇晃晃的样子(因为背着很重的东西)}\end{définition}\end{sous-entrée}

\begin{sous-entrée}{sɤjɤrjɤr}{ⓔjɤrjɤrⓝsɤjɤrjɤr} 
\classe{vt} 
\begin{exemple}\pjya{ɯ-fkur ɲɯ-nɤrʑi tɕe ɲɯ-sɤjɤrjɤr}\hspace{5pt}\pcmn{背的东西很重,显得路都走不稳}\end{exemple}\end{sous-entrée}

\end{entrée}

\begin{entrée}{jɤxtshi}{}{ⓔjɤxtshi} 
\classe{adv} 
\begin{définition}\pfra{cette fois-ci}\end{définition}
\begin{définition}\pcmn{这一次}\end{définition}\end{entrée}

\begin{entrée}{jɤznɤ}{}{ⓔjɤznɤ} 
\classe{post} 
\begin{définition}\pfra{pendant}\end{définition}
\begin{définition}\pcmn{在……的时候}\end{définition}
\begin{exemple}\pjya{andi kɤ-ɣe-a jɤznɤ tɯ-mɯ pɯ-asɯ-lɤt ɕti}\hspace{5pt}\pcmn{我从那边来到这里一直都在下雨}\end{exemple}
\begin{exemple}\pjya{jɯfɕɯr jɤznɤ tɯ-mɯ pɯ-asɯ-lɤt tɕe tham to-stat}\hspace{5pt}\pcmn{昨天开始下雨,现在就停了}\end{exemple}
\begin{exemple}\pjya{aʑo @zhongguo ju-ɣi-a ɕɯŋgɯ jɤznɤ pɯ-aɕqhe-a ɕti}\hspace{5pt}\pcmn{我来中国之前已经在咳嗽}\end{exemple}
\begin{exemple}\pjya{tɯ-kɯ-mŋɤm tu-ʑe ɕɯŋgɯ jɤznɤ tú-wɣ-z-nɯsmɤn ra}\hspace{5pt}\pcmn{在开始生病之前就要治疗}\end{exemple}\end{entrée}

\begin{entrée}{jɤzɯlu}{}{ⓔjɤzɯlu} 
\classe{n} 
\begin{définition}\pfra{année du lapin}\end{définition}
\begin{définition}\pcmn{兔年}\end{définition}\étymologie{yos.lo}\end{entrée}

\begin{entrée}{jɣɤt}{}{ⓔjɣɤt} 
\classe{vi}
\classe{vt}  
\grammaire{caus} \paradigme{dir}{\_}\paradigme{dir}{\_}
\begin{définition}\pfra{revenir}\end{définition}
\begin{définition}\pcmn{转回}\end{définition}
\begin{exemple}\pjya{jiɕqha nɯ-ɣe ri li kɤ-jɣɤt}\hspace{5pt}\pcmn{他来了,又回去了}\end{exemple}
\begin{exemple}\pjya{jiɕqha nɯ-ɣe-a tɕe li kɤ-jɣat-a}\hspace{5pt}\pcmn{我来了又回去了}\end{exemple}
\begin{exemple}\pjya{tɕiʑo kɯ-nɯsaχsɯ kɤ-ari-tɕi tɕe, li nɯ-nɯ-jɣɤt-tɕi}\hspace{5pt}\pcmn{我们去吃中午餐,又回来了}\end{exemple}
\begin{sous-entrée}{sɯjɣɤt}{ⓔjɣɤtⓝsɯjɣɤt}\end{sous-entrée}

\begin{définition}\pfra{faire revenir}\end{définition}
\begin{définition}\pcmn{令……回来}\end{définition}
\begin{exemple}\pjya{ɯʑo lɤ-ari ri, thɯ-sɯjɣat-a}\hspace{5pt}\pcmn{虽然他往上游去了,但是我让他转回来了}\end{exemple}\end{entrée}

\begin{entrée}{ji}{}{ⓔji} 
\classe{vt} \paradigme{dir}{lɤ-}\paradigme{dir}{pɯ-}
\begin{définition}\pfra{planter}\end{définition}
\begin{définition}\pcmn{种(菜、植物等)、播种}\end{définition}
\begin{exemple}\pjya{jiɕqha jaŋjy nɯ lɤ-ji-t-a}\hspace{5pt}\pcmn{我种了土豆}\end{exemple}
\begin{exemple}\pjya{si tɯβli nɯ ɕ-pɯ-ji-t-a}\hspace{5pt}\pcmn{我种了树苗}\end{exemple}
\begin{exemple}\pjya{tɯrgi tɯβli ɕ-pɯ-ji-t-a}\hspace{5pt}\pcmn{我种了杉树苗}\end{exemple}\relationsémantique{参考}{\lien{ⓔrɤji}{rɤji}}\end{entrée}

\begin{entrée}{jiɕqha}{}{ⓔjiɕqha} 
\classe{adv} 
\begin{définition}\pfra{à l'instant}\end{définition}
\begin{définition}\pcmn{刚才}\end{définition}\end{entrée}

\begin{entrée}{jima}{}{ⓔjima} 
\classe{n} 
\begin{définition}\pfra{maïs}\end{définition}
\begin{définition}\pcmn{玉米}\end{définition}\étymologie{fn:御麦}\end{entrée}

\begin{entrée}{jinɤji}{}{ⓔjinɤji} 
\classe{idph.3} 
\begin{définition}\pfra{son de gémissement}\end{définition}
\begin{définition}\pcmn{形容不停地呻吟的样子}\end{définition}
\begin{exemple}\pjya{ɲɯ-nɤmŋɤm tɕe jinɤji ʑo ɲɯ-ti}\hspace{5pt}\pcmn{他很痛,不停地呻吟}\end{exemple}\end{entrée}

\begin{entrée}{jinde}{}{ⓔjinde} 
\classe{n} 
\begin{définition}\pfra{maintenant}\end{définition}
\begin{définition}\pcmn{现在}\end{définition}\end{entrée}

\begin{entrée}{jisŋi}{}{ⓔjisŋi} 
\classe{n} 
\begin{définition}\pfra{aujourd'hui}\end{définition}
\begin{définition}\pcmn{今天}\end{définition}\end{entrée}

\begin{entrée}{jit}{}{ⓔjit} 
\classe{vi} \paradigme{dir}{pɯ-}
\begin{définition}\pfra{couler}\end{définition}
\begin{définition}\pcmn{流出来;溢出来;倒出来;洒(水)}\end{définition}
\begin{exemple}\pjya{tɯ-rŋa sɤ-mar nɯ pjɤ-jit}\hspace{5pt}\pcmn{擦脸的(液体)溢出来了}\end{exemple}
\begin{exemple}\pjya{jiɕqha phoŋ ɯ-ŋgɯ cha nɯ pjɤ-jit}\hspace{5pt}\pcmn{酒从这个瓶子溢出来了}\end{exemple}
\begin{exemple}\pjya{phoŋ ɯ-ŋgɯ tɤ-lu nɯ pjɤ-jit}\hspace{5pt}\pcmn{牛奶从瓶子溢出来了}\end{exemple}
\begin{exemple}\pjya{jiɕqha a-tʂha kɯre pɯ-ata ri pjɤ-jit}\hspace{5pt}\pcmn{茶刚才放在这里,不小心倒了}\end{exemple}\end{entrée}

\begin{entrée}{jiʑo}{}{ⓔjiʑo} 
\classe{pro} 
\begin{définition}\pfra{nous}\end{définition}
\begin{définition}\pcmn{我们}\end{définition}\end{entrée}

\begin{entrée}{jkrɯt}{}{ⓔjkrɯt} 
\classe{vi} \paradigme{dir}{kɤ-}
\begin{définition}\pfra{se solidifier}\end{définition}
\begin{définition}\pcmn{凝固}\end{définition}
\begin{exemple}\pjya{tɯkri nɯ-sɤla-t-a tɕe ɲo-mɯɕtaʁ tɕe ko-jkrɯt}\hspace{5pt}\pcmn{我烧了油,冷了就凝固了}\end{exemple}\end{entrée}

\begin{entrée}{jla}{}{ⓔjla} 
\classe{n} 
\begin{définition}\pfra{hybride de yak et de vache}\end{définition}
\begin{définition}\pcmn{犏牛}\end{définition}
\begin{exemple}\pjya{jla nɯ fsapaʁ kɯ-wxti ci ŋu, ɯ-rme jndʐɤz, khro mɤ-rɲɟi, jla ɯ-ɲɤm a-pɯ-pe tɕe ɯ-rme nɤmbju, ɯ-ʁrɯ jpum tsa rɲɟi, ʑɯrɯʑɤri lu-omtɕoʁ ŋu, ɯ-jme ɣɯ ɯ-rme nɯ khro mɤ-dɤn, rɲɟi, ɯ-ʁrɯ ɯ-rca pa tsa ri ɯ-rna ku-ndzoʁ ŋu. nɯ ɯ-pa ri ɯ-mɲaʁ ŋu, jla ɯ-ku wxti, ɯ-mtɕi jpum, ɯ-ɕna ɲɯ́-wɣ-sɯ-spoʁ tɕe, nɯ tɕu ɯ-si tɤ-loʁ ɲɯ́-wɣ-βzu tɕe, tɕe nɯ ɲɯ́-wɣ-rʁe, tɕe nɯ tɕu tɯ-mbri kú-wɣ-tshoʁ tɕe nɯ ɕnɤri ŋu tɕe, ɣɯ-mtshi tɤ-ra tɕe, tɯ-mbri nɯ kú-wɣ-ndo tɕe jla nɯ tɯ-qhu ju-ɣi ŋu, tɕe jla ju-ɕe mɯ-tɤ-jɤɣ tɕe nɯ tɯ-mbri nɯ cischiz kú-wɣ-βraʁ tɕe jla ku-nɯ-rɤʑi ŋu. jla ju-nɯɕɯ-ce tɤ-jɤɣ tɕe, nɯ-ɕnɤri nɯ jla ɯ-ʁrɯ ɯ-taʁ tú-wɣ-rtɤβ tɕe tɤ-mtɯ tú-wɣ-lɤt jla ju-nɯɕe ɕti. jla nɯ kɯ-rɯŋkhorwa ra ɣɯ kɤ-rɤji cho kɤ-ɕlu pjɯ-me mɤ-kɯ-khɯ ʑo ɕti ma tɯ-ɕlu ɯ-kɯ-rɤɕi nɯ ɕti. nɯ a-pɯ-maʁ tɕe, tɤ-rɤku kɤ-ji mɤ-khɯ. tɕe jla sna mɤ-sna cho taŋ mɤ-taŋ nɯ ɯ-jme ɯ-rme ɯ-ʁrɯ ɯ-qɯ-qataʁrɯ ɯ-mtɕhi nɯ ra ku-χpjɤt ŋu. jla nɯ sɯjno ndze, tɯ-jpu ndze, tɯ-kri tshi, tɕeri li ɯ-ŋgo tu tɕe pjɯ-nɯtɕhɯtɯɣ ŋgrɤl, pjɯ-nɯrtsatɯɣ ŋgrɤl. jla kɯmɯxte kɯ-ɲaʁ ŋu tɕeri li kɯ-wɣrum kɯ-ɣɯrni kɯ-pɣi nɯ ra tu, ʑakastaka nɯ-rmi tu : raχtɕoŋ, sɯli, mdzukɤr, mdzumɤr, mdzusŋun, kanaʁ, ɕasca, rɟamar, rɟandzi nɯ ra kɤ-ti tu-ŋgrɤl}\hspace{5pt}\pcmn{犏牛是高大的牲畜,毛很粗,不长。如果膘好,毛就有光泽。脚又粗又长,逐渐变尖,尾巴上的毛不多、很长,角的下面紧接着就是耳朵,再下面就是眼睛。犏牛的头很大,嘴巴很粗。把鼻子打穿后,就可以穿上鼻圈,再系上一根绳子,那就是鼻索。需要牵走的时候,就拿着那根绳子,犏牛就跟在后面走,不要它走的时候,就把绳子随便拴在什么地方,它就会待在那里。要让它走时,就把绳子缠在角上,打上结,就会走了。犏牛是农民耕田播种不可缺少的,因为只有它能拉犁耕地,没有它,就不能种庄稼。判断犏牛品种的好坏与真假要观察尾巴、毛、角、嘴和蹄子。犏牛吃草,吃粮食,喝油,但是它也有一些病,如水中毒、草中毒等。大部分犏牛是黑色的,但是也有白色、红色和灰色的,各种都有自己的名称。}\end{exemple}\end{entrée}

\begin{entrée}{jlaʁar}{}{ⓔjlaʁar} 
\classe{n} 
\begin{définition}\pfra{veau d'hybride de yak et de vache}\end{définition}
\begin{définition}\pcmn{犏牛犊}\end{définition}\end{entrée}

\begin{entrée}{jlɤβndʑu}{}{ⓔjlɤβndʑu} 
\classe{n} 
\begin{définition}\pfra{bâton duquel on défile la trame}\end{définition}
\begin{définition}\pcmn{纬线的杆子}\end{définition}\relationsémantique{参考}{\lien{ⓔtɯ-jlɤβ}{tɯ-jlɤβ}}\relationsémantique{参考}{\lien{ⓔndʑu}{ndʑu}}\end{entrée}

\begin{entrée}{jlɤdo}{}{ⓔjlɤdo} 
\classe{n} 
\begin{définition}\pfra{vieux yak hybride}\end{définition}
\begin{définition}\pcmn{老犏牛}\end{définition}\relationsémantique{参考}{\lien{ⓔjla}{jla}}\relationsémantique{参考}{\lien{ⓔɯ-do}{ɯ-do}}\end{entrée}

\begin{entrée}{jlɤkrɯ}{}{ⓔjlɤkrɯ} 
\classe{n} 
\begin{définition}\pfra{hotte}\end{définition}
\begin{définition}\pcmn{背篼}\end{définition}\end{entrée}

\begin{entrée}{jlɤmtshi}{}{ⓔjlɤmtshi} 
\classe{n} 
\begin{définition}\pfra{action de mener un yak hybride}\end{définition}
\begin{définition}\pcmn{牵犏牛}\end{définition}\relationsémantique{参考}{\lien{ⓔnɯjlɤmtshi}{nɯjlɤmtshi}}\end{entrée}

\begin{entrée}{jlɤndʐi}{}{ⓔjlɤndʐi} 
\classe{n} 
\begin{définition}\pfra{peau de yak hybride}\end{définition}
\begin{définition}\pcmn{犏牛皮子}\end{définition}\relationsémantique{参考}{\lien{ⓔjla}{jla}}\relationsémantique{参考}{\lien{ⓔtɯ-ndʐi}{tɯ-ndʐi}}\end{entrée}

\begin{entrée}{jlɤprɤm}{}{ⓔjlɤprɤm} 
\classe{n} 
\begin{définition}\pfra{poudre pour nourrir les yaks hybrides}\end{définition}
\begin{définition}\pcmn{喂犏牛的粉}\end{définition}\relationsémantique{参考}{\lien{ⓔjla}{jla}}\relationsémantique{参考}{\lien{ⓔtɤ-prɤm}{tɤ-prɤm}}\end{entrée}

\begin{entrée}{jlɤtɯndzɯm}{}{ⓔjlɤtɯndzɯm} 
\classe{n} 
\begin{définition}\pfra{paire de yak hybrides (pour tirer la charrue)}\end{définition}
\begin{définition}\pcmn{一对犏牛(拖犁头的)}\end{définition}\relationsémantique{参考}{\lien{ⓔjla}{jla}}\relationsémantique{参考}{\lien{ⓔtɯ-tɯndzɯm}{tɯ-tɯndzɯm}}\end{entrée}

\begin{entrée}{jmɤɣni}{}{ⓔjmɤɣni} 
\classe{n} 
\begin{définition}\pfra{russule rouge}\end{définition}
\begin{définition}\pcmn{红菇【杉木菌】}\end{définition}
\begin{exemple}\pjya{jmɤɣni nɯ tɯrgi ɯ-ŋgɯ tu-ɬoʁ ŋu, kɤ-ndza mɯm ɯ-mdoʁ nɯ kɯ-ɤɣɯrnɯɕɯr tsa ŋu. ftɕar tɕe tɯrgi kɯ-xtɕi ɯ-ŋgɯ ɯ-ŋgɯ ŋu stonka tɕe tɯrgi kɯ-wxti ɯ-ŋgɯ tu-ɬoʁ ŋu}\hspace{5pt}\pcmn{杉木菌长在杉木林里,好吃,颜色带有点红色,夏天长在小杉木林里,秋天长在大杉木林里。}\end{exemple}\relationsémantique{参考}{\lien{ⓔɣɯrni}{ɣɯrni}}\relationsémantique{参考}{\lien{ⓔtɤjmɤɣ}{tɤjmɤɣ}}\end{entrée}

\begin{entrée}{jmɤjmɤr}{}{ⓔjmɤjmɤr} 
\classe{idph.2} 
\begin{définition}\pfra{doux}\end{définition}
\begin{définition}\pcmn{形容柔软的样子}\end{définition}\end{entrée}

\begin{entrée}{jmɤlu}{}{ⓔjmɤlu} 
\classe{n} 
\begin{définition}\pfra{sans queue}\end{définition}
\begin{définition}\pcmn{没有尾巴(的动物)}\end{définition}\relationsémantique{参考}{\lien{ⓔtɤ-jme}{tɤ-jme}}\end{entrée}

\begin{entrée}{jmɤrtaʁ}{}{ⓔjmɤrtaʁ} 
\classe{n} 
\begin{définition}\pfra{espèce d'insecte}\end{définition}
\begin{définition}\pcmn{虫子的一种}\end{définition}\relationsémantique{参考}{\lien{ⓔtɤ-jme}{tɤ-jme}}\relationsémantique{参考}{\lien{ⓔartaʁ}{artaʁ}}\end{entrée}

\begin{entrée}{jmɤtɤsti}{}{ⓔjmɤtɤsti} 
\classe{n} 
\begin{définition}\pfra{espèce de champignon}\end{définition}
\begin{définition}\pcmn{菌子的一种}\end{définition}\relationsémantique{参考}{\lien{ⓔjmɤɣni}{jmɤɣni}}\end{entrée}

\begin{entrée}{jmɯt}{}{ⓔjmɯt} 
\classe{vt} \paradigme{dir}{nɯ-}
\begin{définition}\pfra{oublier}\end{définition}
\begin{définition}\pcmn{忘记}\end{définition}
\begin{exemple}\pjya{ma-nɯ-tɯ-jmɯt}\hspace{5pt}\pcmn{你不要忘记}\end{exemple}
\begin{exemple}\pjya{ɲɤ-jmɯt}\hspace{5pt}\pcmn{他忘记了}\end{exemple}
\begin{exemple}\pjya{ɲo-nɯ-jmɯt}\hspace{5pt}\pcmn{我忘了}\end{exemple}
\begin{exemple}\pjya{ɲo-kɯ-jmɯt-a}\hspace{5pt}\pcmn{你把我忘了}\end{exemple}
\begin{exemple}\pjya{mɯ́j-tɯ-cha wo ma nɤ-ma ntsɯ, nɤ-koŋtso ntsɯ tu-tɯ-nɤme qhe, kɯmaʁ ra ɲɯ-tɯ-nɯ-jmɯt ɲɯ-ɕti}\hspace{5pt}\pcmn{你不行吧,因为你只顾着你的工作,会忘记其它事情}\end{exemple}\relationsémantique{反义词}{\lien{ⓔɕɯftaʁ}{ɕɯftaʁ}}
\begin{sous-entrée}{ɣɤjmɯt}{ⓔjmɯtⓝɣɤjmɯt} 
\classe{vs}  
\grammaire{facil} 
\begin{définition}\pfra{qui a une mauvaise mémoire}\end{définition}
\begin{définition}\pcmn{记性不好;溶剂忘记}\end{définition}
\begin{exemple}\pjya{a-rɯxpa ɲɯ-khe tɕe ɲɯ-ɣɤjmɯt-a}\hspace{5pt}\pcmn{我记性不好,容易忘记}\end{exemple}
\begin{exemple}\pjya{a-rɯxpa pe tɕe mɤ-ɣɤjmɯt-a}\hspace{5pt}\pcmn{我记性很好,不容易忘记}\end{exemple}
\begin{exemple}\pjya{mɯ́j-ɣɤjmɯt}\hspace{5pt}\pcmn{他不容易忘记}\end{exemple}\end{sous-entrée}

\begin{sous-entrée}{nɯɣɯjmɯt}{ⓔjmɯtⓝnɯɣɯjmɯt} 
\classe{vs} 
\begin{définition}\pfra{facile à oublier}\end{définition}
\begin{définition}\pcmn{容易忘记(事情)}\end{définition}
\begin{exemple}\pjya{ɯʑo kɯ ta-tɯt nɯ kɯ-nɯɣɯjmɯt ci ɲɯ-ŋu}\hspace{5pt}\pcmn{他讲的话很容易忘记}\end{exemple}\relationsémantique{反义词}{\lien{ⓔɕɯftaʁⓝnɯɣɯɕɯftaʁ}{nɯɣɯɕɯftaʁ}}\end{sous-entrée}

\end{entrée}

\begin{entrée}{jndʐɤz}{}{ⓔjndʐɤz} 
\classe{vs} \paradigme{dir}{tɤ-}\paradigme{dir}{thɯ-}
\begin{définition}\pfra{épaisse (poudre)}\end{définition}
\begin{définition}\pcmn{粗(粉状)}\end{définition}
\begin{exemple}\pjya{ʑɴɢɯloʁ ɲɯ-jndʐɤz}\hspace{5pt}\pcmn{核桃很大}\end{exemple}
\begin{exemple}\pjya{tɯsqar chɤ-jndʐɤz}\hspace{5pt}\pcmn{糌粑磨得很粗}\end{exemple}\end{entrée}

\begin{entrée}{jndʐɯɣ}{}{ⓔjndʐɯɣ} 
\classe{vi} \paradigme{dir}{nɯ-}
\begin{définition}\pfra{ruminer}\end{définition}
\begin{définition}\pcmn{反刍}\end{définition}
\begin{exemple}\pjya{nɯŋa ɲɯ-jndʐɯɣ}\hspace{5pt}\pcmn{牛在反刍}\end{exemple}\end{entrée}

\begin{entrée}{jnom}{}{ⓔjnom} 
\classe{vs} 
\begin{définition}\pfra{flexible}\end{définition}
\begin{définition}\pcmn{柔软;有弹性}\end{définition}\relationsémantique{参考}{\lien{ⓔldʑɯz}{ldʑɯz}}\end{entrée}

\begin{entrée}{jŋaʁjŋaʁ}{}{ⓔjŋaʁjŋaʁ} 
\classe{idph.2} 
\begin{définition}\pfra{au corps fin et élancé (femme)}\end{définition}
\begin{définition}\pcmn{形容身子苗条的样子(女子)}\end{définition}
\begin{exemple}\pjya{iɕqha tɕheme nɯ ɯ-phoŋbu jŋaʁjŋaʁ ʑo ɲɯ-pa}\hspace{5pt}\pcmn{那个女子身体很苗条}\end{exemple}\end{entrée}

\begin{entrée}{jom}{}{ⓔjom} 
\classe{vs} \paradigme{dir}{nɯ-}
\begin{définition}\pfra{large}\end{définition}
\begin{définition}\pcmn{宽;辽阔}\end{définition}
\begin{exemple}\pjya{sɤtɕha ɲɯ-jom}\hspace{5pt}\pcmn{地方很宽}\end{exemple}
\begin{exemple}\pjya{ɯ-sɤ-ta ɲɯ-jom}\hspace{5pt}\pcmn{放的地方很宽}\end{exemple}
\begin{exemple}\pjya{kha ɲɯ-jom}\hspace{5pt}\pcmn{屋子很宽}\end{exemple}
\begin{exemple}\pjya{ɯ-ro jom (ɯ-sɯmrɟɤz)}\hspace{5pt}\pcmn{他心胸宽阔}\end{exemple}
\begin{exemple}\pjya{ɯ-mɲaʁsta jom}\hspace{5pt}\pcmn{他心胸宽阔}\end{exemple}\relationsémantique{参考}{\lien{ⓔɣɤjom}{ɣɤjom}}\end{entrée}

\begin{entrée}{joʁ}{}{ⓔjoʁ} 
\classe{vt} \sens{1}\paradigme{dir}{tɤ-}\paradigme{dir}{thɯ-}\paradigme{dir}{lɤ-}
\begin{définition}\pfra{lever}\end{définition}
\begin{définition}\pcmn{抬起;举起;放在上面}\end{définition}
\begin{exemple}\pjya{kɯki laχtɕha tɤ-joʁ-a}\hspace{5pt}\pcmn{我把这根东西放上去了}\end{exemple}
\begin{exemple}\pjya{ɯʑo kɯ @luyinji pɯ-asɯ-ndo tɕe la-joʁ}\hspace{5pt}\pcmn{他把手上拿着的录音机放在上面了}\end{exemple}
\begin{exemple}\pjya{a-jaʁ tɤ-joʁ-a}\hspace{5pt}\pcmn{我把手举起来了}\end{exemple}
\begin{exemple}\pjya{ɯ-stɤt chɤ-joʁ}\hspace{5pt}\pcmn{他起了半身}\end{exemple}\sens{2}\paradigme{dir}{\_}
\begin{définition}\pfra{ranger, mettre de côté}\end{définition}
\begin{définition}\pcmn{收起来;收在一边}\end{définition}
\begin{exemple}\pjya{nɤ-ŋga nɯ-tɕɤt tɕe ju-joʁ-a}\hspace{5pt}\pcmn{你把衣服脱下来,我给你收起来}\end{exemple}
\begin{exemple}\pjya{khɯtsa ra tɤ-wum tɕe lɤ-joʁ!}\hspace{5pt}\pcmn{你把这些碗收起来放在(柜子里)}\end{exemple}\relationsémantique{同义词}{\lien{ⓔwum}{wum}}\end{entrée}

\begin{entrée}{joʁβzɯr}{}{ⓔjoʁβzɯr} 
\classe{n} 
\begin{définition}\pfra{rangement}\end{définition}
\begin{définition}\pcmn{收拾}\end{définition}
\begin{exemple}\pjya{joʁβzɯr tɤ-βzu-t-a}\hspace{5pt}\pcmn{我收拾}\end{exemple}
\begin{exemple}\pjya{kɯra laχtɕha ra ɲɯ-dɤn tɕe, joʁβzu kɤ-βzu ɲɯ-ra}\hspace{5pt}\pcmn{这里东西很多,需要收拾}\end{exemple}\relationsémantique{参考}{\lien{ⓔrɤjoʁβzɯr}{rɤjoʁβzɯr}}\end{entrée}

\begin{entrée}{jpɣom}{}{ⓔjpɣom} 
\classe{vi} \paradigme{dir}{kɤ-}
\begin{définition}\pfra{geler}\end{définition}
\begin{définition}\pcmn{冻}\end{définition}
\begin{exemple}\pjya{ɕɤfɕo ɲɯ-mɯɕtaʁ tɕe tɯ-ci ko-jpɣom}\hspace{5pt}\pcmn{这几天很冷,水结成冰了}\end{exemple}
\begin{exemple}\pjya{nɤki ɯ-pɕi ma-nɯ-tɯ-te ma ɲɯ-mɯɕtaʁ tɕe jpɣom}\hspace{5pt}\pcmn{你别把这个东西放在外面,现在很冷就会冻坏}\end{exemple}
\begin{sous-entrée}{sɯjpɣom}{ⓔjpɣomⓝsɯjpɣom} 
\classe{vt} 
\begin{définition}\pfra{geler}\end{définition}
\begin{définition}\pcmn{冻到}\end{définition}
\begin{exemple}\pjya{kɯ-sɯjpɣom ʑo ɲɯ-ŋu}\hspace{5pt}\pcmn{冷得把人冻僵}\end{exemple}\relationsémantique{参考}{\lien{ⓔtɤjpɣom}{tɤjpɣom}}\relationsémantique{参考}{\lien{ⓔqajpɣom}{qajpɣom}}\end{sous-entrée}

\end{entrée}

\begin{entrée}{jpum}{}{ⓔjpum} 
\classe{vs} \paradigme{dir}{nɯ-}\paradigme{dir}{nɯ-}
\begin{définition}\pfra{large (diamètre)}\end{définition}
\begin{définition}\pcmn{粗(直径)}\end{définition}
\begin{définition}\pfra{rendre plus large}\end{définition}
\begin{définition}\pcmn{使变粗}\end{définition}
\begin{exemple}\pjya{si ɲɯ-jpum}\hspace{5pt}\pcmn{树很粗}\end{exemple}
\begin{exemple}\pjya{tɯ-pu thɯ-ɣɤmɯta tɕe, nɯ-ɣɤjpum-a}\hspace{5pt}\pcmn{我把(猪)肠子吹胀起来了}\end{exemple}\relationsémantique{参考}{\lien{ⓔjpumqa}{jpumqa}}\relationsémantique{反义词}{\lien{ⓔxtshɯm}{xtshɯm}}
\begin{sous-entrée}{ɣɤjpum}{ⓔjpumⓝɣɤjpum} 
\classe{vt} \end{sous-entrée}

\end{entrée}

\begin{entrée}{jpumqa}{}{ⓔjpumqa} 
\classe{vs} \paradigme{dir}{thɯ-}
\begin{définition}\pfra{épais}\end{définition}
\begin{définition}\pcmn{粗壮}\end{définition}
\begin{exemple}\pjya{kɯ-jpumqa ci ɲɯ-ŋu}\hspace{5pt}\pcmn{是粗壮的}\end{exemple}\relationsémantique{同义词}{\lien{ⓔaβrdaβrdoŋ}{aβrdaβrdoŋ}}\relationsémantique{反义词}{\lien{ⓔtɕale}{tɕale}}\relationsémantique{参考}{\lien{ⓔjpum}{jpum}}\end{entrée}

\begin{entrée}{jpumxtshɯm}{}{ⓔjpumxtshɯm} 
\classe{n} 
\begin{définition}\pfra{épaisseur}\end{définition}
\begin{définition}\pcmn{粗细;粗度}\end{définition}\relationsémantique{参考}{\lien{ⓔjpum}{jpum}}\relationsémantique{参考}{\lien{ⓔxtshɯm}{xtshɯm}}\relationsémantique{参考}{\lien{ⓔajpomxtshɯm}{ajpomxtshɯm}}\end{entrée}

\begin{entrée}{jqu}{}{ⓔjqu} 
\classe{vt} \paradigme{dir}{pɯ-}\sens{1}
\begin{définition}\pfra{être capable de soulever}\end{définition}
\begin{définition}\pcmn{背得起}\end{définition}
\begin{exemple}\pjya{ɲɯ-jqe-a}\hspace{5pt}\pcmn{我背得起}\end{exemple}
\begin{exemple}\pjya{ɯ-ɲɯ́-tɯ-jqe?}\hspace{5pt}\pcmn{你背得起吗?}\end{exemple}
\begin{exemple}\pjya{mɯ-ɕɯ-tɯ-jqe nɯ-sɯ-so-t-a}\hspace{5pt}\pcmn{我怕你背不起}\end{exemple}
\begin{exemple}\pjya{jiɕqha laχtɕha nɯ tɤ-fkur-a ri pɯ-jqut-a}\hspace{5pt}\pcmn{我把那个东西背起来了,我背得动(原来担心背不动)}\end{exemple}
\begin{exemple}\pjya{tɯkhɯr ɲɯ-rʑi ri, ɲɯ-jqe-a}\hspace{5pt}\pcmn{责任很重但是我背得起}\end{exemple}
\begin{exemple}\pjya{pɤjkhu ɯ-ku mɤ-jqe}\hspace{5pt}\pcmn{(婴儿的脖子)还支撑不了他的头}\end{exemple}
\begin{exemple}\pjya{ɯ-ku to-nɯ-jqu}\hspace{5pt}\pcmn{(婴儿的脖子)支撑得了他的头}\end{exemple}\sens{2}
\begin{définition}\pfra{être capable de supporter}\end{définition}
\begin{définition}\pcmn{承受得了}\end{définition}
\begin{exemple}\pjya{nɯ sthɯci kɯ-mŋɤm mɤ-jqe}\hspace{5pt}\pcmn{他承受不了那么痛}\end{exemple}
\begin{sous-entrée}{sɤjqu}{ⓔjquⓢ2ⓝsɤjqu} 
\classe{vs} 
\begin{définition}\pfra{qui peut être soulevé}\end{définition}
\begin{définition}\pcmn{背得动的}\end{définition}
\begin{exemple}\pjya{kɯki tɯjpu ki kɤ-fkur ɲɯ-sɤjqu}\hspace{5pt}\pcmn{这袋粮食可以背得动}\end{exemple}\end{sous-entrée}

\end{entrée}

\begin{entrée}{jtshi}{}{ⓔjtshi} 
\classe{vt} \paradigme{dir}{nɯ-}\paradigme{dir}{nɯ-}
\begin{définition}\pfra{donner à boire}\end{définition}
\begin{définition}\pcmn{给别人喝}\end{définition}
\begin{définition}\pfra{demander à boire}\end{définition}
\begin{définition}\pcmn{请人给自己喝}\end{définition}
\begin{exemple}\pjya{tʂha nɯ-kɯ-jtshi-a}\hspace{5pt}\pcmn{你给我喝茶了}\end{exemple}
\begin{exemple}\pjya{cha nɯ-kɯ-jtshi-a}\hspace{5pt}\pcmn{你给我喝酒了}\end{exemple}
\begin{exemple}\pjya{aʑo tʂha ci ɲɯ-sɤjtshi-a}\hspace{5pt}\pcmn{我让人给我喝茶}\end{exemple}
\begin{exemple}\pjya{tɕethi thɯ-ɕe tɕe, tʂha ci nɯ-sɤjtshi}\hspace{5pt}\pcmn{你往下游去,要个茶喝}\end{exemple}
\begin{sous-entrée}{sɤjtshi}{ⓔjtshiⓝsɤjtshi} 
\classe{vt} \end{sous-entrée}

\begin{sous-entrée}{rɤjtshi}{ⓔjtshiⓝrɤjtshi} 
\classe{vt}  
\grammaire{apass} 
\begin{définition}\pfra{donner à boire à quelqu'un}\end{définition}
\begin{définition}\pcmn{给别人喝}\end{définition}
\begin{exemple}\pjya{aʑo tɤtɕɯ ra nɯ-ɕki cha ɲɯ-rɤjtshi-a ŋgrɤl}\hspace{5pt}\pcmn{我给男生们喝酒}\end{exemple}\end{sous-entrée}

\begin{sous-entrée}{ajtshi}{ⓔjtshiⓝajtshi} 
\classe{vi} 
\begin{exemple}\pjya{nɤʑo ɲɯ-tɯ-jtshi mɤ-ra ma ajtshi}\hspace{5pt}\pcmn{你不用给他喝了,已经给了}\end{exemple}
\begin{exemple}\pjya{smɤn ɯ-j-ajtshi}\hspace{5pt}\pcmn{给他喝药了没有}\end{exemple}
\begin{exemple}\pjya{tɤ-pɤtso ra tɤ-lu ajtshi}\hspace{5pt}\pcmn{牛奶已经给小孩喝了}\end{exemple}\end{sous-entrée}

\end{entrée}

\begin{entrée}{jtsraβ}{}{ⓔjtsraβ} 
\classe{vi} \paradigme{dir}{nɯ-}
\begin{définition}\pfra{retarder son départ}\end{définition}
\begin{définition}\pcmn{推迟出发时间}\end{définition}
\begin{exemple}\pjya{toʁde nɯ-jtsraβ-a}\hspace{5pt}\pcmn{我推迟了一下时间}\end{exemple}
\begin{exemple}\pjya{ɯʑo toʁde nɯ-jtsraβ}\hspace{5pt}\pcmn{他推迟了一下时间}\end{exemple}
\begin{exemple}\pjya{ɯʑo ju-ɕe pɯ-ɕti ri, hanɯni ci mɯ́j-ndzu tɕe nɯ-jtsraβ pɯ-ra}\hspace{5pt}\pcmn{他本来是要走的,但是因为没有准备好,只好推迟了出发的时间}\end{exemple}\end{entrée}

\begin{entrée}{jtʂhɤβnɤjtʂhɤβ}{}{ⓔjtʂhɤβnɤjtʂhɤβ} 
\classe{idph.3} 
\begin{définition}\pfra{qui tire et arrache dans tous les sens}\end{définition}
\begin{définition}\pcmn{形容乱扯的样子}\end{définition}
\begin{exemple}\pjya{ɯ-ŋga jtʂhɤβnɤjtʂhɤβ ʑo jo-rɤɕi}\hspace{5pt}\pcmn{他乱扯了他的衣服}\end{exemple}\end{entrée}

\begin{entrée}{jɯβjɯβ}{}{ⓔjɯβjɯβ} 
\classe{idph.2} 
\begin{définition}\pfra{beaucoup d'animaux}\end{définition}
\begin{définition}\pcmn{很多(动物)密密麻麻地站着}\end{définition}
\begin{exemple}\pjya{fsapaʁ jɯβjɯβ ʑo ɲɯ-rɤʑinɯ}\hspace{5pt}\pcmn{很多牲畜在那里密密麻麻地站着}\end{exemple}\relationsémantique{参考}{\lien{ⓔʑɯβʑɯβ}{ʑɯβʑɯβ}}\end{entrée}

\begin{entrée}{jɯfɕɯmɯr}{}{ⓔjɯfɕɯmɯr} 
\classe{n} 
\begin{définition}\pfra{hier soir}\end{définition}
\begin{définition}\pcmn{昨晚}\end{définition}\relationsémantique{参考}{\lien{ⓔjɯfɕɯr}{jɯfɕɯr}}\relationsémantique{参考}{\lien{ⓔtɯ-ɣmɯr}{tɯ-ɣmɯr}}\end{entrée}

\begin{entrée}{jɯfɕɯndʐi}{}{ⓔjɯfɕɯndʐi} 
\classe{n} 
\begin{définition}\pfra{ces derniers jours}\end{définition}
\begin{définition}\pcmn{前几天}\end{définition}\end{entrée}

\begin{entrée}{jɯfɕɯndʐɯɕɯŋgɯpa}{}{ⓔjɯfɕɯndʐɯɕɯŋgɯpa} 
\classe{adv} 
\begin{définition}\pfra{il y a trois ans}\end{définition}
\begin{définition}\pcmn{三年前}\end{définition}\end{entrée}

\begin{entrée}{jɯfɕɯndʐɯpa}{}{ⓔjɯfɕɯndʐɯpa} 
\classe{adv} 
\begin{définition}\pfra{l'année d'avant}\end{définition}
\begin{définition}\pcmn{前年}\end{définition}\end{entrée}

\begin{entrée}{jɯfɕɯndʐɯsŋi}{}{ⓔjɯfɕɯndʐɯsŋi} 
\classe{adv} 
\begin{définition}\pfra{avant-hier}\end{définition}
\begin{définition}\pcmn{前天}\end{définition}\end{entrée}

\begin{entrée}{jɯfɕɯr}{}{ⓔjɯfɕɯr} 
\classe{adv} 
\begin{définition}\pfra{hier}\end{définition}
\begin{définition}\pcmn{昨天}\end{définition}\end{entrée}

\begin{entrée}{jɯɣi}{}{ⓔjɯɣi} 
\classe{n} 
\begin{définition}\pfra{lettre}\end{définition}
\begin{définition}\pcmn{信;书}\end{définition}\étymologie{ji.ge}\end{entrée}

\begin{entrée}{jɯɣmɯr}{}{ⓔjɯɣmɯr} 
\classe{adv} 
\begin{définition}\pfra{ce soir}\end{définition}
\begin{définition}\pcmn{今晚}\end{définition}\end{entrée}

\begin{entrée}{jɯl}{}{ⓔjɯl} 
\classe{n} 
\begin{définition}\pfra{village}\end{définition}
\begin{définition}\pcmn{村庄(有地、有房子的地方)}\end{définition}\étymologie{jul}\end{entrée}

\begin{entrée}{jɯlco}{}{ⓔjɯlco} 
\classe{n} 
\begin{définition}\pfra{voisin}\end{définition}
\begin{définition}\pcmn{邻居}\end{définition}\relationsémantique{参考}{\lien{ⓔtɤ-rɣa}{tɤ-rɣa}}\end{entrée}

\begin{entrée}{jɯlpa}{}{ⓔjɯlpa} 
\classe{n} 
\begin{définition}\pfra{villageois}\end{définition}
\begin{définition}\pcmn{村民}\end{définition}\étymologie{jul.pa}\end{entrée}

\begin{entrée}{jɯm}{₂}{ⓔjɯmⓗ2} 
\classe{n} 
\begin{définition}\pfra{épouse de sprulsku}\end{définition}
\begin{définition}\pcmn{活佛的妻子}\end{définition}\étymologie{jum}\end{entrée}

\begin{entrée}{jɯm}{₁}{ⓔjɯmⓗ1} 
\classe{vs} \paradigme{dir}{nɯ-}\paradigme{dir}{nɯ-}
\begin{définition}\pfra{être clair}\end{définition}
\begin{définition}\pcmn{天晴}\end{définition}
\begin{définition}\pfra{apporter le beau temps (par magie)}\end{définition}
\begin{définition}\pcmn{(用法术)令天变晴}\end{définition}
\begin{exemple}\pjya{jisŋi ɲɯ-jɯm}\hspace{5pt}\pcmn{今天天晴}\end{exemple}
\begin{exemple}\pjya{tɤŋe ɲɯ-jɯm}\hspace{5pt}\pcmn{太阳很好}\end{exemple}
\begin{sous-entrée}{sɯɣjɯm}{ⓔjɯmⓗ1ⓝsɯɣjɯm} 
\classe{vt} \end{sous-entrée}

\end{entrée}

\begin{entrée}{jɯnbalazɯ}{}{ⓔjɯnbalazɯ} 
\classe{cnj} 
\begin{définition}\pfra{non seulement... mais...}\end{définition}
\begin{définition}\pcmn{既……又……}\end{définition}
\begin{exemple}\pjya{jisŋi ɲɯ-jɯm jɯnbalazɯ, qale ɣɤʑu}\hspace{5pt}\pcmn{今天既是晴天又刮风}\end{exemple}\étymologie{jin.pa.la}\end{entrée}

\begin{entrée}{jɯxɕo}{}{ⓔjɯxɕo} 
\classe{adv} 
\begin{définition}\pfra{ce matin}\end{définition}
\begin{définition}\pcmn{今天早上}\end{définition}\end{entrée}

\begin{entrée}{jwajwa}{}{ⓔjwajwa} 
\classe{idph.2} 
\begin{définition}\pfra{très fin}\end{définition}
\begin{définition}\pcmn{形容薄的样子}\end{définition}\end{entrée}

\begin{entrée}{jwɤjwɤr}{}{ⓔjwɤjwɤr} 
\classe{idph.2} 
\begin{définition}\pfra{de travers (chapeau)}\end{définition}
\begin{définition}\pcmn{歪的,不周正(帽子)}\end{définition}
\begin{exemple}\pjya{nɤ-rte tɤ-sɤste ma jwɤjwɤr ʑo ɲɯ-pa}\hspace{5pt}\pcmn{你把帽子戴正吧,是歪的}\end{exemple}\relationsémantique{参考}{\lien{ⓔʑwɤʑwɤr}{ʑwɤʑwɤr}}\end{entrée}

\begin{entrée}{jxɤjxɤt}{}{ⓔjxɤjxɤt} 
\classe{idph.2} 
\begin{définition}\pfra{au corps fin et élancé, qui est très active (femme)}\end{définition}
\begin{définition}\pcmn{形容身子苗条、很勤劳的样子(女子)}\end{définition}
\begin{exemple}\pjya{iɕqha tɕheme nɯ ɯ-phoŋbu ra jxɤjxɤt ʑo ɲɯ-pe}\hspace{5pt}\pcmn{这个女子身体苗条,很勤劳的样子}\end{exemple}\end{entrée}

\newpage\caractère{ɟ}

\begin{entrée}{ɟu}{}{ⓔɟu} 
\classe{n} 
\begin{définition}\pfra{bambou}\end{définition}
\begin{définition}\pcmn{竹子}\end{définition}\end{entrée}

\begin{entrée}{ɟar}{}{ⓔɟar} 
\classe{vt} \paradigme{dir}{nɯ-}
\begin{définition}\pfra{séparer le navet et les feuilles du navet}\end{définition}
\begin{définition}\pcmn{把圆根的根和叶子切开}\end{définition}
\begin{sous-entrée}{rɤɟar}{ⓔɟarⓝrɤɟar} 
\classe{vi}  
\grammaire{apass} 
\begin{exemple}\pjya{ku-rɤɟar-a}\hspace{5pt}\pcmn{我在切圆根}\end{exemple}\end{sous-entrée}

\end{entrée}

\begin{entrée}{ɟɤr}{}{ⓔɟɤr} 
\classe{n} 
\begin{définition}\pfra{colle}\end{définition}
\begin{définition}\pcmn{胶}\end{définition}\étymologie{ⁿbʲar}\end{entrée}

\begin{entrée}{ɟɤrɯru}{}{ⓔɟɤrɯru} 
\classe{n} 
\begin{définition}\pfra{tsampa}\end{définition}
\begin{définition}\pcmn{糌粑的一种吃法}\end{définition}
\begin{exemple}\pjya{khɯtsa ɯ-ŋgɯ tʂha tú-wɣ-rku tɕe, tɕhɯrtsɤm sɤznɤ a-pɯ-dɤn tsa tɕe ɯ-taʁ tɯ-sqar pjɯ́-wɣ-lɤt, tɯ-sqar nɯ tɯ-ci sɤznɤ a-pɯ-dɤn tsa tɕe ɲo-ɕmi tɕe rwoʁrwoʁ ʑo a-nɯ-pa tɕe tú-wɣ-ndza tɕe nɯ ɟɤrɯru rmi.}\hspace{5pt}\pcmn{在碗里倒茶,比\lien{ⓔtɕhɯrtsɤm}{tɕhɯrtsɤm}多一点,再放上糌粑,那个糌粑比水多一点,搅到出现很多小球就可以吃,这种吃法叫做\lien{ⓔɟɤrɯru}{ɟɤrɯru}}\end{exemple}\end{entrée}

\begin{entrée}{ɟɤwɤt}{}{ⓔɟɤwɤt} 
\classe{n} 
\begin{définition}\pfra{espèce d'herbe}\end{définition}
\begin{définition}\pcmn{草的一种}\end{définition}\end{entrée}

\begin{entrée}{ɟuli/\variante{ɟɯlij}}{}{ⓔɟuli} 
\classe{n} 
\begin{définition}\pfra{flûte}\end{définition}
\begin{définition}\pcmn{笛子}\end{définition}
\begin{exemple}\pjya{ɟuli tɤ-ʑmbri-t-a}\hspace{5pt}\pcmn{我吹了笛子}\end{exemple}\relationsémantique{参考}{\lien{ⓔrɯɟuli}{rɯɟuli}}\end{entrée}

\begin{entrée}{ɟrɯɣɟrɯɣ}{}{ⓔɟrɯɣɟrɯɣ} 
\classe{idph.2} \sens{1}
\begin{définition}\pfra{mal au ventre}\end{définition}
\begin{définition}\pcmn{形容肚子不舒服的感觉(快要拉肚子的感觉)}\end{définition}
\begin{exemple}\pjya{a-xtu ɟrɯɣɟrɯɣ ʑo ɲɯ-pa ma ɲɯ-sɤɣdɯɣ}\hspace{5pt}\pcmn{我肚子不舒服}\end{exemple}\relationsémantique{参考}{\lien{ⓔɣɤɟɯɟrɯɣ}{ɣɤɟɯɟrɯɣ}}\sens{2}
\begin{définition}\pfra{en désordre}\end{définition}
\begin{définition}\pcmn{形容凌乱,不整齐的样子}\end{définition}
\begin{exemple}\pjya{laχtɕha ɟrɯɣɟrɯɣ ʑo ko-rmbɯ}\hspace{5pt}\pcmn{他把东西得乱七八糟}\end{exemple}\relationsémantique{参考}{\lien{ⓔtɯ-ɟrɯɣ}{tɯ-ɟrɯɣ}}\relationsémantique{参考}{\lien{ⓔcrɯɣcrɯɣ}{crɯɣcrɯɣ}}\end{entrée}

\begin{entrée}{ɟɯ}{}{ⓔɟɯ} 
\classe{vi} \paradigme{dir}{thɯ-}
\begin{définition}\pfra{être digéré}\end{définition}
\begin{définition}\pcmn{消化(缩小)}\end{définition}
\begin{exemple}\pjya{tɤ-ndza-t-a nɯra chɤ-ɟɯ}\hspace{5pt}\pcmn{我吃的东西已经消化好了}\end{exemple}\relationsémantique{参考}{\lien{ⓔnɯɣɟɯ}{nɯɣɟɯ}}\end{entrée}

\begin{entrée}{ɟɯga}{}{ⓔɟɯga} 
\classe{n} 
\begin{définition}\pfra{chemin tortueux}\end{définition}
\begin{définition}\pcmn{弯路}\end{définition}\end{entrée}

\begin{entrée}{ɟɯgɯɟɯga}{}{ⓔɟɯgɯɟɯga} 
\classe{n} 
\begin{définition}\pfra{(chemin) tortueux}\end{définition}
\begin{définition}\pcmn{弯弯曲曲}\end{définition}
\begin{exemple}\pjya{tʂu ɟɯgɯɟɯga ʑo tu-βze ɲɯ-ŋu}\hspace{5pt}\pcmn{录是弯弯曲曲的}\end{exemple}\relationsémantique{参考}{\lien{ⓔɟɯga}{ɟɯga}}\end{entrée}

\begin{entrée}{ɟɯɣɟɯɣ}{}{ⓔɟɯɣɟɯɣ} 
\classe{idph.2} \sens{1}
\begin{définition}\pfra{meuble (terre), déluré}\end{définition}
\begin{définition}\pcmn{形容土地松散,或衣冠不整的样子}\end{définition}\sens{2}\paradigme{dir}{tɤ-}
\begin{définition}\pfra{nombreux (objets longs)}\end{définition}
\begin{définition}\pcmn{很多(条状的物体)}\end{définition}
\begin{définition}\pfra{(faire) en grand nombre, en groupe}\end{définition}
\begin{définition}\pcmn{很多……一起……;成群}\end{définition}
\begin{exemple}\pjya{laʑu ɟɯɣɟɯɣ ʑo ɲɯ-ɴqoʁ}\hspace{5pt}\pcmn{很多条腊肉挂在那里}\end{exemple}
\begin{exemple}\pjya{nɤ-ŋga ɯ-ŋga tɤ-ɣɤβdi ɟɯɣɟɯɣ ʑo ma-tɯ-ʑɣɤ-stu}\hspace{5pt}\pcmn{你把衣服穿好一点,不要松松垮垮的}\end{exemple}
\begin{exemple}\pjya{tɯ-mɯ kɤ-lɤt pa-ʑa tɕe ɲɯ-nɯɟɯɣɟɯɣ-nɯ ʑo kha ɯ-ŋgɯ lɤ-ari-nɯ}\hspace{5pt}\pcmn{开始下雨的时候,他们成群地回家了}\end{exemple}
\begin{sous-entrée}{ɟɯɣnɤɟɯɣ}{ⓔɟɯɣɟɯɣⓢ2ⓝɟɯɣnɤɟɯɣ} 
\classe{idph.3} 
\begin{définition}\pfra{grouillant et très nombreux}\end{définition}
\begin{définition}\pcmn{形容很多(虫子)乱动}\end{définition}
\begin{exemple}\pjya{tɯ-ci ɯ-ŋgɯ zɯ qapri ɟɯɣnɤɟɯɣ ʑo ɲɯ-pa}\hspace{5pt}\pcmn{水里很多蛇在乱动}\end{exemple}
\begin{exemple}\pjya{tɯ-ci ɯ-ɣmbaj zɯ tɤjpɣom ɟɯɣnɤɟɯɣ ʑo ɲɯ-pa}\hspace{5pt}\pcmn{水面上有很多冰}\end{exemple}\end{sous-entrée}

\begin{sous-entrée}{nɯɟɯɣɟɯɣ}{ⓔɟɯɣɟɯɣⓢ2ⓝnɯɟɯɣɟɯɣ} 
\classe{vi} \end{sous-entrée}

\begin{sous-entrée}{sɤɟɯɣɟɯɣ}{ⓔɟɯɣɟɯɣⓢ2ⓝsɤɟɯɣɟɯɣ} 
\classe{vt} 
\begin{définition}\pfra{frétiller}\end{définition}
\begin{définition}\pcmn{抖动}\end{définition}
\begin{exemple}\pjya{jla kɯ ɯ-βri ɲɯ-sɤɟɯɣɟɯɣ}\hspace{5pt}\pcmn{犏牛在抖动}\end{exemple}
\begin{exemple}\pjya{ɲɯ-nɤŋkɯŋke tɕe, ɯ-tʂɯmpɤri ra ɲɯ-sɤɟɯɣɟɯɣ ʑo}\hspace{5pt}\pcmn{她走路的时候围裙腰带的须须跟着抖动}\end{exemple}\end{sous-entrée}

\end{entrée}

\begin{entrée}{ɟɯmɢom}{}{ⓔɟɯmɢom} 
\classe{n} 
\begin{définition}\pfra{pincette en bambou}\end{définition}
\begin{définition}\pcmn{竹子制成的夹子}\end{définition}\relationsémantique{参考}{\lien{ⓔtamɢom}{tamɢom}}\relationsémantique{同义词}{\lien{ⓔrɟɯmtsɯ}{rɟɯmtsɯ}}\relationsémantique{参考}{\lien{ⓔɟu}{ɟu}}\end{entrée}

\begin{entrée}{ɟɯthoʁ}{}{ⓔɟɯthoʁ} 
\classe{n}  
\grammaire{n.lieu} 
\begin{définition}\pfra{l'un des hameaux de Gyutshapa}\end{définition}
\begin{définition}\pcmn{二茶村的大队之一}\end{définition}\end{entrée}

\newpage\caractère{k}

\begin{entrée}{ka}{}{ⓔka} 
\classe{adv} 
\begin{définition}\pfra{chaque}\end{définition}
\begin{définition}\pcmn{各自}\end{définition}\end{entrée}

\begin{entrée}{ka}{}{ⓔka} 
\classe{vi} \paradigme{dir}{tɤ-}\paradigme{dir}{lɤ-}
\begin{définition}\pfra{quitter le sol (nuage)}\end{définition}
\begin{définition}\pcmn{离开地面(云雾)}\end{définition}
\begin{exemple}\pjya{zdɯm zgo ɯ-taʁ pjɤ-ndzoʁ tɕe mɯ-to-ka}\hspace{5pt}\pcmn{云贴在山尖上,还没有离开地面}\end{exemple}\end{entrée}

\begin{entrée}{kaβ}{}{ⓔkaβ} 
\classe{vi} \paradigme{dir}{jɤ-}
\begin{définition}\pfra{porter de l'eau sur son dos}\end{définition}
\begin{définition}\pcmn{背水}\end{définition}
\begin{exemple}\pjya{aj ɕ-ku-kaβ-a (ʑ-ɲɯ-kaβ-a)}\hspace{5pt}\pcmn{我去背水}\end{exemple}\relationsémantique{参考}{\lien{ⓔsakaβ}{sakaβ}}\end{entrée}

\begin{entrée}{kachijmɤɣ}{}{ⓔkachijmɤɣ} 
\classe{n} 
\begin{définition}\pfra{une espèce de champignon}\end{définition}
\begin{définition}\pcmn{一种菌子}\end{définition}
\begin{exemple}\pjya{kachijmɤɣ kɤ-ndza sna}\hspace{5pt}\pcmn{“甜菌”可以吃}\end{exemple}\end{entrée}

\begin{entrée}{kaɣɯ}{}{ⓔkaɣɯ} 
\classe{n} 
\begin{définition}\pfra{collier avec pendentif en argent}\end{définition}
\begin{définition}\pcmn{胸章}\end{définition}\end{entrée}

\begin{entrée}{kaldʐa}{}{ⓔkaldʐa} 
\classe{n} 
\begin{définition}\pfra{une espèce d'arbrisseau}\end{définition}
\begin{définition}\pcmn{灌木的一种}\end{définition}
\begin{exemple}\pjya{kaldʐa nɯ si mɤ-jpum ri kɯ-mbro ci ŋu, zgoku kɯ-mbro tɕe, si kɯ-wxti ra nɯ-rca tu-ɬoʁ ŋu. ɯ-ru nɯ ɲaʁ, ɯ-jwaʁ nɯ kɯ-ɤrtɯm kɯ-zri tsa tɕe kɯ-ndɯ-ndɯβ ʑo ŋu. ɯ-mɯntoʁ kɯ-wɣrum ɲɯ-lɤt tɕe, ɯ-mɯntoʁ ɯ-jɯ nɯ zri tsa tɕe, tɯ-ɣnɤsqi jamar tɯtɯrca ku-ndzoʁ ŋu. ɯ-mat thɯ-tɯt tɕe, kɯ-ɣrɯ-wɣrum ŋu, staχpɯ kɯ-jndʐɤz ʑo fse, ɯ-si nɯ wuma ʑo ngɯt tɕe laʁdɯn ɯ-jɯ ɯ-spa wuma ʑo pe.}\hspace{5pt}\pcmn{\lien{ⓔkaldʐa}{kaldʐa}是长得不粗但很高的树种,生长在高山的乔木林里。树干是黑色的,叶子是椭圆形的,很小。花是白色的,花梗比较长,一、二十朵长在一起。果实成熟后也是白色的,很像大豌豆,木质结实,是作农具把子的好材料。}\end{exemple}\end{entrée}

\begin{entrée}{kamda}{}{ⓔkamda} 
\classe{n} 
\begin{définition}\pfra{brochette de tranches de navets}\end{définition}
\begin{définition}\pcmn{穿成一串的芜菁片}\end{définition}\relationsémantique{参考}{\lien{ⓔrɤjndoʁ}{rɤjndoʁ}}\end{entrée}

\begin{entrée}{kanaʁ}{}{ⓔkanaʁ} 
\classe{n} 
\begin{définition}\pfra{bovidé de couleur noire dont le ventre et les pattes sont blancs}\end{définition}
\begin{définition}\pcmn{全身是黑色的,肚皮和脚白色的牛}\end{définition}\étymologie{dkar.nag}\end{entrée}

\begin{entrée}{kaŋi}{}{ⓔkaŋi} 
\classe{n} 
\begin{définition}\pfra{endroit où l'on place la nourriture préparée}\end{définition}
\begin{définition}\pcmn{寄放成品食物的地方}\end{définition}\end{entrée}

\begin{entrée}{kaŋkaŋ}{}{ⓔkaŋkaŋ} 
\classe{n} 
\begin{définition}\pfra{tasse (avec une poignée)}\end{définition}
\begin{définition}\pcmn{带把的杯子}\end{définition}\étymologie{fn:缸缸}\end{entrée}

\begin{entrée}{karɣi}{}{ⓔkarɣi} 
\classe{n} 
\begin{définition}\pfra{graine de navet}\end{définition}
\begin{définition}\pcmn{【菜子】}\end{définition}
\begin{exemple}\pjya{karɣi nɯ rasti rɣi kɤ-ti ŋu}\hspace{5pt}\pcmn{菜子是圆根的种子}\end{exemple}\end{entrée}

\begin{entrée}{karɟɯ}{}{ⓔkarɟɯ} 
\classe{n}  
\grammaire{n.lieu} 
\begin{définition}\pfra{l'un des hameaux de Gyutshapa}\end{définition}
\begin{définition}\pcmn{二茶村的大队之一}\end{définition}\end{entrée}

\begin{entrée}{karkɯm}{}{ⓔkarkɯm} 
\classe{n} 
\begin{définition}\pfra{cotylédon du navet}\end{définition}
\begin{définition}\pcmn{圆根的子叶}\end{définition}\relationsémantique{参考}{\lien{ⓔɯ-rkɯm}{ɯ-rkɯm}}\end{entrée}

\begin{entrée}{katɕa}{}{ⓔkatɕa} 
\classe{n}  
\grammaire{n.lieu} 
\begin{définition}\pfra{Katcha (village de Gdongbrgyad)}\end{définition}
\begin{définition}\pcmn{尕渣村}\end{définition}\end{entrée}

\begin{entrée}{kawa}{}{ⓔkawa} 
\classe{n} 
\begin{définition}\pfra{bovidé à tête blanche et au corps noir}\end{définition}
\begin{définition}\pcmn{头色的,而身体黑色的牛}\end{définition}\end{entrée}

\begin{entrée}{kɤɕŋaʁ}{}{ⓔkɤɕŋaʁ} 
\classe{n} 
\begin{définition}\pfra{avoir l'impression d'entendre}\end{définition}
\begin{définition}\pcmn{仿佛听见}\end{définition}
\begin{exemple}\pjya{ɯ-pɕi tɕe, kɤkhu ɣɤʑu rcama kɤɕŋaʁ ci ɣɤʑu}\hspace{5pt}\pcmn{外面仿佛听到喊声}\end{exemple}
\begin{exemple}\pjya{jiɕqha ri kɤɕŋaʁ ci pɯ-tu}\hspace{5pt}\pcmn{刚才仿佛听见声音}\end{exemple}\end{entrée}

\begin{entrée}{kɤɕoʁ}{}{ⓔkɤɕoʁ} 
\classe{n} 
\begin{définition}\pfra{nœud}\end{définition}
\begin{définition}\pcmn{活结}\end{définition}
\begin{exemple}\pjya{kɯki tɤmtɯɲaʁ mɤ-tú-wɣ-lɤt, kɤɕoʁ tú-wɣ-lɤt ra}\hspace{5pt}\pcmn{不要打死结,要打活结}\end{exemple}\end{entrée}

\begin{entrée}{kɤɣ}{}{ⓔkɤɣ} 
\classe{vt} \paradigme{dir}{pɯ-}
\begin{définition}\pfra{courber}\end{définition}
\begin{définition}\pcmn{弄弯}\end{définition}
\begin{exemple}\pjya{jiɕqha si nɯ pɯ-kaɣ-a}\hspace{5pt}\pcmn{我把那棵树弄弯了}\end{exemple}
\begin{exemple}\pjya{laʁjɯɣ pɯ-kaɣ-a}\hspace{5pt}\pcmn{我把棍子弄弯了}\end{exemple}
\begin{exemple}\pjya{tɯdi kɤ-kɤɣ ɯ-ɲɯ́-tɯ-wɣ-sɯxcha}\hspace{5pt}\pcmn{这把弓你拉得动吗?}\end{exemple}
\begin{exemple}\pjya{ɕom kɤ-kɤɣ}\hspace{5pt}\pcmn{把铁弄弯}\end{exemple}\relationsémantique{同义词}{\lien{ⓔajʁuⓝsɤjʁu}{sɤjʁu}}\relationsémantique{参考}{\lien{ⓔŋgɤɣ}{ŋgɤɣ}}\end{entrée}

\begin{entrée}{kɤjpɯ}{}{ⓔkɤjpɯ} 
\classe{n} 
\begin{définition}\pfra{poulain}\end{définition}
\begin{définition}\pcmn{马驹,小马}\end{définition}\end{entrée}

\begin{entrée}{kɤlu}{}{ⓔkɤlu} 
\classe{adv} 
\begin{définition}\pfra{sans tête}\end{définition}
\begin{définition}\pcmn{无头}\end{définition}\relationsémantique{参考}{\lien{ⓔtɯ-ku}{tɯ-ku}}\end{entrée}

\begin{entrée}{kɤlɤthɤftɕa}{}{ⓔkɤlɤthɤftɕa} 
\classe{n} 
\begin{définition}\pfra{ustensiles de cuisine}\end{définition}
\begin{définition}\pcmn{厨房用品}\end{définition}\end{entrée}

\begin{entrée}{kɤlɯmlɯm}{}{ⓔkɤlɯmlɯm} 
\classe{adv} 
\begin{définition}\pfra{ensemble}\end{définition}
\begin{définition}\pcmn{一起;完整地}\end{définition}
\begin{exemple}\pjya{ɯ-kɤrme cho ɯ-ɕa kɤlɯmlɯm ʑo ka-ndo}\hspace{5pt}\pcmn{他把头发和皮肉一起抓了起来}\end{exemple}\end{entrée}

\begin{entrée}{kɤmbrɤβraʁ}{}{ⓔkɤmbrɤβraʁ} 
\classe{n} 
\begin{définition}\pfra{relation}\end{définition}
\begin{définition}\pcmn{联系;瓜葛;牵连}\end{définition}
\begin{exemple}\pjya{nɤj nɤ-kɤ-nɤma cho aʑo a-kɤ-nɤma nɯ ʑaka ɕti ma kɤmbrɤβraʁ me}\hspace{5pt}\pcmn{你的工作和我的工作没有什么瓜葛}\end{exemple}\end{entrée}

\begin{entrée}{kɤmda}{}{ⓔkɤmda} 
\classe{n} 
\begin{définition}\pfra{navet séché}\end{définition}
\begin{définition}\pcmn{用柳条穿起来,晾干了的芜菁根}\end{définition}\end{entrée}

\begin{entrée}{kɤmɲɯ}{}{ⓔkɤmɲɯ} 
\classe{n}  
\grammaire{n.lieu} 
\begin{définition}\pfra{Kamnyu}\end{définition}
\begin{définition}\pcmn{干木鸟村}\end{définition}
\begin{exemple}\pjya{ʑatsa kɤmɲɯ nɯɕe-a}\hspace{5pt}\pcmn{我很快就要回干木鸟了}\end{exemple}\end{entrée}

\begin{entrée}{kɤmoʁ}{}{ⓔkɤmoʁ} 
\classe{n} 
\begin{définition}\pfra{tsampa destinée à être consommée sèche}\end{définition}
\begin{définition}\pcmn{香糌粑}\end{définition}\relationsémantique{参考}{\lien{ⓔmoʁ}{moʁ}}\end{entrée}

\begin{entrée}{kɤndzɤtshi}{}{ⓔkɤndzɤtshi} 
\classe{n} 
\begin{définition}\pfra{repas}\end{définition}
\begin{définition}\pcmn{餐}\end{définition}\end{entrée}

\begin{entrée}{kɤndʑɯβzaŋsa}{}{ⓔkɤndʑɯβzaŋsa} 
\classe{n} 
\begin{définition}\pfra{amis}\end{définition}
\begin{définition}\pcmn{朋友们}\end{définition}
\begin{exemple}\pjya{nɤʑo cho aʑo kɤndʑɯβzaŋsa}\hspace{5pt}\pcmn{我们是朋友}\end{exemple}\relationsémantique{参考}{\lien{ⓔβzaŋsa}{βzaŋsa}}\end{entrée}

\begin{entrée}{kɤndʑɯɕaχpu}{}{ⓔkɤndʑɯɕaχpu} 
\classe{n} 
\begin{définition}\pfra{amis}\end{définition}
\begin{définition}\pcmn{朋友}\end{définition}\relationsémantique{同义词}{\lien{ⓔkɤndʑɯβzaŋsa}{kɤndʑɯβzaŋsa}}\relationsémantique{同义词}{\lien{ⓔkɤndʑɯɣɯfsu}{kɤndʑɯɣɯfsu}}\relationsémantique{参考}{\lien{ⓔɕaχpu}{ɕaχpu}}\end{entrée}

\begin{entrée}{kɤndʑɯɣe}{}{ⓔkɤndʑɯɣe} 
\classe{n} 
\begin{définition}\pfra{grand-père et petit-fils}\end{définition}
\begin{définition}\pcmn{祖父孙子}\end{définition}
\begin{exemple}\pjya{a-ɣe cho tɕiʑo kɤndʑɯɣe}\end{exemple}\relationsémantique{参考}{\lien{ⓔtɤ-ɣe}{tɤ-ɣe}}\end{entrée}

\begin{entrée}{kɤndʑɯɣɯfsu}{}{ⓔkɤndʑɯɣɯfsu} 
\classe{n} 
\begin{définition}\pfra{amis}\end{définition}
\begin{définition}\pcmn{朋友}\end{définition}\relationsémantique{同义词}{\lien{ⓔkɤndʑɯβzaŋsa}{kɤndʑɯβzaŋsa}}\relationsémantique{同义词}{\lien{ⓔkɤndʑɯɕaχpu}{kɤndʑɯɕaχpu}}\relationsémantique{参考}{\lien{ⓔɣɯfsu}{ɣɯfsu}}\end{entrée}

\begin{entrée}{kɤndʑɯkɯmdza}{}{ⓔkɤndʑɯkɯmdza} 
\classe{n} 
\begin{définition}\pfra{membres d'une même famille}\end{définition}
\begin{définition}\pcmn{亲戚}\end{définition}
\begin{exemple}\pjya{kɤndʑɯkɯmdza pjɤ-ŋu-ndʑi}\hspace{5pt}\pcmn{他们俩是亲戚}\end{exemple}\relationsémantique{参考}{\lien{ⓔkɯmdza}{kɯmdza}}\end{entrée}

\begin{entrée}{kɤndʑɯmɤtsa}{}{ⓔkɤndʑɯmɤtsa} 
\classe{n} 
\begin{définition}\pfra{cousin (collectif)}\end{définition}
\begin{définition}\pcmn{堂兄弟姐妹}\end{définition}\end{entrée}

\begin{entrée}{kɤndʑɯmbro}{}{ⓔkɤndʑɯmbro} 
\classe{n} 
\begin{définition}\pfra{le cavalier et son cheval}\end{définition}
\begin{définition}\pcmn{马和骑手}\end{définition}\end{entrée}

\begin{entrée}{kɤndʑɯme}{}{ⓔkɤndʑɯme} 
\classe{n} 
\begin{définition}\pfra{mère et fille}\end{définition}
\begin{définition}\pcmn{母女}\end{définition}\relationsémantique{参考}{\lien{ⓔtɯ-me}{tɯ-me}}\end{entrée}

\begin{entrée}{kɤndʑɯɲi}{}{ⓔkɤndʑɯɲi} 
\classe{n} 
\begin{définition}\pfra{tante et neveu}\end{définition}
\begin{définition}\pcmn{姑母侄子}\end{définition}\relationsémantique{参考}{\lien{ⓔtɤ-ɲi}{tɤ-ɲi}}\end{entrée}

\begin{entrée}{kɤndʑɯpɤmdɯ}{}{ⓔkɤndʑɯpɤmdɯ} 
\classe{n} 
\begin{définition}\pfra{oncle et neveu}\end{définition}
\begin{définition}\pcmn{叔叔侄子}\end{définition}\relationsémantique{参考}{\lien{ⓔtɤ-mdɯ}{tɤ-mdɯ}}\end{entrée}

\begin{entrée}{kɤndʑɯrpɯftsa}{}{ⓔkɤndʑɯrpɯftsa} 
\classe{n} 
\begin{définition}\pfra{oncle et neveu}\end{définition}
\begin{définition}\pcmn{舅舅和外甥子}\end{définition}\relationsémantique{参考}{\lien{ⓔtɤ-rpɯ}{tɤ-rpɯ}}\relationsémantique{参考}{\lien{ⓔtɤ-ftsa}{tɤ-ftsa}}\end{entrée}

\begin{entrée}{kɤndʑɯʁi}{}{ⓔkɤndʑɯʁi} 
\classe{n} 
\begin{définition}\pfra{grand frère et petit frère, grande sœur et petite sœur}\end{définition}
\begin{définition}\pcmn{哥哥,姐姐,弟弟和妹妹}\end{définition}\relationsémantique{参考}{\lien{ⓔta-ʁi}{ta-ʁi}}\end{entrée}

\begin{entrée}{kɤndʑɯslamaχti}{}{ⓔkɤndʑɯslamaχti} 
\classe{n} 
\begin{définition}\pfra{ami de classe}\end{définition}
\begin{définition}\pcmn{好同学}\end{définition}\end{entrée}

\begin{entrée}{kɤndʑɯsqhaj}{}{ⓔkɤndʑɯsqhaj} 
\classe{n} 
\begin{définition}\pfra{sœurs (collectif)}\end{définition}
\begin{définition}\pcmn{姐妹}\end{définition}
\begin{exemple}\pjya{jiʑo kɤndʑɯsqhaj ŋu-j}\hspace{5pt}\pcmn{我们是姐妹}\end{exemple}\relationsémantique{参考}{\lien{ⓔtɤ-sqhaj}{tɤ-sqhaj}}\end{entrée}

\begin{entrée}{kɤndʑɯtɤtɕɯχti}{}{ⓔkɤndʑɯtɤtɕɯχti} 
\classe{n} 
\begin{définition}\pfra{amis (garçons)}\end{définition}
\begin{définition}\pcmn{朋友(男孩之间)}\end{définition}\end{entrée}

\begin{entrée}{kɤndʑɯtɕhemɤχti}{}{ⓔkɤndʑɯtɕhemɤχti} 
\classe{n} 
\begin{définition}\pfra{amies}\end{définition}
\begin{définition}\pcmn{朋友(女孩之间)}\end{définition}\end{entrée}

\begin{entrée}{kɤndʑɯwɤɬaʁ}{}{ⓔkɤndʑɯwɤɬaʁ} 
\classe{n} 
\begin{définition}\pfra{tante et neveu}\end{définition}
\begin{définition}\pcmn{婶母侄子}\end{définition}\relationsémantique{参考}{\lien{ⓔtɤ-ɬaʁ}{tɤ-ɬaʁ}}\end{entrée}

\begin{entrée}{kɤndʑɯwɤmɯsnom}{}{ⓔkɤndʑɯwɤmɯsnom} 
\classe{n} 
\begin{définition}\pfra{frères et sœurs}\end{définition}
\begin{définition}\pcmn{兄弟姐妹}\end{définition}\relationsémantique{参考}{\lien{ⓔtɤ-wɤmɯ}{tɤ-wɤmɯ}}\relationsémantique{参考}{\lien{ⓔtɤ-snom}{tɤ-snom}}\end{entrée}

\begin{entrée}{kɤndʑɯxtɤɣ}{}{ⓔkɤndʑɯxtɤɣ} 
\classe{n} 
\begin{définition}\pfra{frères (collectif)}\end{définition}
\begin{définition}\pcmn{兄弟}\end{définition}\relationsémantique{参考}{\lien{ⓔtɤ-xtɤɣ}{tɤ-xtɤɣ}}\end{entrée}

\begin{entrée}{kɤndʑɯχti}{}{ⓔkɤndʑɯχti} 
\classe{n} 
\begin{définition}\pfra{amis}\end{définition}
\begin{définition}\pcmn{朋友们}\end{définition}\relationsémantique{参考}{\lien{ⓔtɯ-χti}{tɯ-χti}}\end{entrée}

\begin{entrée}{kɤndʑɯzda}{}{ⓔkɤndʑɯzda} 
\classe{n} 
\begin{définition}\pfra{compagnons}\end{définition}
\begin{définition}\pcmn{伙伴们}\end{définition}\relationsémantique{参考}{\lien{ⓔtɯ-zda}{tɯ-zda}}\end{entrée}

\begin{entrée}{kɤntɕhaʁ}{}{ⓔkɤntɕhaʁ} 
\classe{n} \sens{1}
\begin{définition}\pfra{rue}\end{définition}
\begin{définition}\pcmn{街}\end{définition}
\begin{exemple}\pjya{kɤntɕhaʁ rɤʑi-a}\hspace{5pt}\pcmn{我在街上}\end{exemple}
\begin{exemple}\pjya{kɤntɕhaʁ ci chɤ-ta}\hspace{5pt}\pcmn{他摆了个摊子}\end{exemple}\sens{2}
\begin{définition}\pfra{village, ville (endroit habité)}\end{définition}
\begin{définition}\pcmn{村子(人住的地方)}\end{définition}\relationsémantique{参考}{\lien{ⓔnɯkɤntɕhaʁ}{nɯkɤntɕhaʁ}}\end{entrée}

\begin{entrée}{kɤntɕhɯkɤβdɤm}{}{ⓔkɤntɕhɯkɤβdɤm} 
\classe{num} 
\begin{définition}\pfra{entre quatre et neuf}\end{définition}
\begin{définition}\pcmn{几个(八九一下,四五个以上}\end{définition}
\begin{exemple}\pjya{jiɕqha ɯ-rɟit kɤntɕhɯkɤβdɤm tu}\hspace{5pt}\pcmn{他有好几个孩子}\end{exemple}\relationsémantique{参考}{\lien{ⓔantɕhɯ}{antɕhɯ}}\end{entrée}

\begin{entrée}{kɤnɯβdɯt/\variante{kɤsɯβdɯt}}{}{ⓔkɤnɯβdɯt} 
\classe{n} 
\begin{définition}\pfra{dépense}\end{définition}
\begin{définition}\pcmn{花费}\end{définition}
\begin{exemple}\pjya{nɤ-βdɯt nɯ-tɕat-a!}\hspace{5pt}\pcmn{让你花费了很多(谢你请我吃饭)}\end{exemple}
\begin{exemple}\pjya{kɤnɯβdɯt pɯ-me, koŋla tɤ-tɯ-ndza-nɯ me}\hspace{5pt}\pcmn{没有花费很多,你们没有吃多少}\end{exemple}\relationsémantique{参考}{\lien{ⓔβdɯtⓗ1}{βdɯt₁}}\end{entrée}

\begin{entrée}{kɤŋgɤxtsa}{}{ⓔkɤŋgɤxtsa} 
\classe{n} 
\begin{définition}\pfra{habits et chaussures}\end{définition}
\begin{définition}\pcmn{衣服和鞋子}\end{définition}\end{entrée}

\begin{entrée}{kɤpa}{}{ⓔkɤpa} 
\classe{n} 
\begin{définition}\pfra{moyen}\end{définition}
\begin{définition}\pcmn{办法}\end{définition}\relationsémantique{参考}{\lien{ⓔpaⓗ3}{pa}}\relationsémantique{同义词}{\lien{ⓔkowa}{kowa}}\end{entrée}

\begin{entrée}{kɤpupu}{}{ⓔkɤpupu} 
\classe{n} 
\begin{définition}\pfra{racine de navet cuite}\end{définition}
\begin{définition}\pcmn{烤熟的芜菁根}\end{définition}
\begin{exemple}\pjya{rɤjndoʁ lú-wɣ-pu tɕe nɯnɯ kɤpupu tu-kɯ-ti ɲɯ-ŋu}\end{exemple}\end{entrée}

\begin{entrée}{kɤpɯpri}{}{ⓔkɤpɯpri} 
\classe{adv} 
\begin{définition}\pfra{sans arrêt, l'un après l'autre}\end{définition}
\begin{définition}\pcmn{连续不断地,一个接着一个}\end{définition}
\begin{exemple}\pjya{ɯ-kɤ-nɤma ɲɯ-dɤn tɕe kɤpɯpri ʑo nɤme ɲɯ-ra}\hspace{5pt}\pcmn{他的工作很多,要连续不断地做}\end{exemple}\end{entrée}

\begin{entrée}{kɤrɤsla}{}{ⓔkɤrɤsla} 
\classe{n} 
\begin{définition}\pfra{plusieurs mois}\end{définition}
\begin{définition}\pcmn{数月}\end{définition}
\begin{exemple}\pjya{kɤrɤsla jɤ-tsu, jɤ-tsu-a}\hspace{5pt}\pcmn{他已经(走了)几个月,我已经几个月了}\end{exemple}\relationsémantique{参考}{\lien{ⓔtɯ-sla}{tɯ-sla}}\end{entrée}

\begin{entrée}{kɤrɤxpa}{}{ⓔkɤrɤxpa} 
\classe{n} 
\begin{définition}\pfra{plusieurs années}\end{définition}
\begin{définition}\pcmn{数年}\end{définition}
\begin{exemple}\pjya{kɤrɤxpa jɤ-tsu-a}\hspace{5pt}\pcmn{我过了很多年}\end{exemple}\relationsémantique{参考}{\lien{ⓔtɯ-xpa}{tɯ-xpa}}\end{entrée}

\begin{entrée}{kɤrjɤl}{}{ⓔkɤrjɤl} 
\classe{n} 
\begin{définition}\pfra{porcelaine}\end{définition}
\begin{définition}\pcmn{瓷碗}\end{définition}
\begin{exemple}\pjya{kɤrjɤl popo, kɤrjɤl tɕhorzi kɤrjɤl khɯtsa}\hspace{5pt}\pcmn{瓷碗}\end{exemple}\étymologie{dkar.jol}\end{entrée}

\begin{entrée}{kɤrŋu}{}{ⓔkɤrŋu} 
\classe{n} 
\begin{définition}\pfra{première période du mois}\end{définition}
\begin{définition}\pcmn{上半月}\end{définition}\étymologie{dkar.ŋo}\end{entrée}

\begin{entrée}{kɤrŋijmɤɣ}{}{ⓔkɤrŋijmɤɣ} 
\classe{n} 
\begin{définition}\pfra{une espèce de champignon}\end{définition}
\begin{définition}\pcmn{一种蘑菇}\end{définition}
\begin{exemple}\pjya{kɤrŋijmɤɣ to-ɬoʁ}\hspace{5pt}\pcmn{蓝菌长出来了}\end{exemple}
\begin{exemple}\pjya{kɤrŋijmɤɣ nɯ tɤjmɤɣ ci ŋu stɤmku ri tu-ɬoʁ tɕe arɤkhɯmkhɤl, kɤrŋijmɤɣ ɯ-sɤ-ɬoʁ stɤmku nɯ kɯroz ʑo arŋi, kɯ-ɤrqhi ʑo ju-kɯ-ru tɕe saχsɤl, kɤrŋijmɤɣ ɯʑo kɯnɤ kɯ-ɤrŋi tsa kɯ-ndɯ-ndɯβ ʑo ŋu, kɤ-ndza wuma ʑo mɯm}\hspace{5pt}\pcmn{蓝菌是长在草地上的蘑菇,不是所有的草地都有,它生长的地方的草特别绿,从很远的地方都可以看到。蓝菌自己带有一点蓝色,长得很小,好吃。}\end{exemple}\relationsémantique{同义词}{\lien{ⓔtɤrmbjajmɤɣ}{tɤrmbjajmɤɣ}}\end{entrée}

\begin{entrée}{kɤrpu}{}{ⓔkɤrpu} 
\classe{n} 
\begin{définition}\pfra{chaux}\end{définition}
\begin{définition}\pcmn{石灰}\end{définition}
\begin{exemple}\pjya{kha ɯ-taʁ kɤrpu ta-lɤt}\hspace{5pt}\pcmn{他抹了墙}\end{exemple}\étymologie{dkar.po}\end{entrée}

\begin{entrée}{kɤrta}{}{ⓔkɤrta} 
\classe{n} 
\begin{définition}\pfra{croix}\end{définition}
\begin{définition}\pcmn{十字形}\end{définition}
\begin{exemple}\pjya{kɤrta pɯ-ta-t-a}\hspace{5pt}\pcmn{我打了叉叉(做记号)}\end{exemple}\end{entrée}

\begin{entrée}{kɤrtsi}{}{ⓔkɤrtsi} 
\classe{n} 
\begin{définition}\pfra{plusieurs (jours, mois, année)}\end{définition}
\begin{définition}\pcmn{好几(天、年)}\end{définition}
\begin{exemple}\pjya{ɯ-sŋi kɤrtsi}\hspace{5pt}\pcmn{数日,好几天}\end{exemple}
\begin{exemple}\pjya{ɯ-xpa kɤrtsi}\hspace{5pt}\pcmn{好几年}\end{exemple}
\begin{exemple}\pjya{kɤ-rɤβzjoz nɯ ɯ-sla kɤrtsi ʑo ra ɕti ma laʁnɯ-rʑaʁ mɤ-fɕaʁ}\hspace{5pt}\pcmn{学习是要好几个月,几天是不够的}\end{exemple}\end{entrée}

\begin{entrée}{kɤsɤri}{}{ⓔkɤsɤri}\relationsémantique{参考}{\lien{ⓔmtɕhɤnmbrɯ}{mtɕhɤnmbrɯ}}\end{entrée}

\begin{entrée}{kɤstu}{₁}{ⓔkɤstuⓗ1} 
\classe{n} 
\begin{définition}\pfra{moyen de s'en sortir}\end{définition}
\begin{définition}\pcmn{办法}\end{définition}
\begin{exemple}\pjya{ɯ-kɤstu maka ɲɤ-me}\hspace{5pt}\pcmn{他再也没有办法了}\end{exemple}\relationsémantique{同义词}{\lien{ⓔkɤpa}{kɤpa}}\end{entrée}

\begin{entrée}{kɤstu}{₂}{ⓔkɤstuⓗ2} 
\classe{n} 
\begin{définition}\pfra{viande séchée conservée dans la peau du pied de cochon}\end{définition}
\begin{définition}\pcmn{把猪腿的皮剥下来,缝成筒形,塞满瘦肉}\end{définition}\end{entrée}

\begin{entrée}{kɤsɯfse/\variante{kɤfsɯfse}}{}{ⓔkɤsɯfse} 
\classe{adv} 
\begin{définition}\pfra{entièrement, tout}\end{définition}
\begin{définition}\pcmn{全部}\end{définition}
\begin{exemple}\pjya{kɤfsɯfse chɤ-k-ɤrɕo-ci}\hspace{5pt}\pcmn{完全用完了}\end{exemple}\end{entrée}

\begin{entrée}{kɤtɕhɯ}{}{ⓔkɤtɕhɯ} 
\classe{n} 
\begin{définition}\pfra{coup de tête}\end{définition}
\begin{définition}\pcmn{用头顶}\end{définition}
\begin{exemple}\pjya{kɤtɕhɯ tɤ-lat-a}\hspace{5pt}\pcmn{我用头顶了他}\end{exemple}\relationsémantique{参考}{\lien{ⓔtɯ-ku}{tɯ-ku}}\relationsémantique{参考}{\lien{ⓔtɕhɯ}{tɕhɯ}}\relationsémantique{参考}{\lien{ⓔnɤkɤtɕhɯ}{nɤkɤtɕhɯ}}\end{entrée}

\begin{entrée}{kɤtsa}{}{ⓔkɤtsa} 
\classe{n} 
\begin{définition}\pfra{parents et enfants}\end{définition}
\begin{définition}\pcmn{父母和孩子}\end{définition}
\begin{exemple}\pjya{tɕheme kɤtsa}\hspace{5pt}\pcmn{母女}\end{exemple}
\begin{exemple}\pjya{tɤ-tɕɯ kɤtsa}\hspace{5pt}\pcmn{父子}\end{exemple}\end{entrée}

\begin{entrée}{kɤtɯm}{}{ⓔkɤtɯm} 
\classe{n} 
\begin{définition}\pfra{pelote de laine}\end{définition}
\begin{définition}\pcmn{线团}\end{définition}
\begin{exemple}\pjya{kɤtɯm tɤ-βze}\end{exemple}\end{entrée}

\begin{entrée}{kɤtɯpa/\variante{kɤtipa}}{₁}{ⓔkɤtɯpaⓗ1} 
\classe{n} 
\begin{définition}\pfra{chicanerie}\end{définition}
\begin{définition}\pcmn{计较}\end{définition}
\begin{exemple}\pjya{ɯ-kɤtɯpa ɲɯ-dɤn}\hspace{5pt}\pcmn{他很喜欢找别人的茬,很爱计较}\end{exemple}
\begin{exemple}\pjya{a-kɤtɯpa dɤn}\hspace{5pt}\pcmn{我爱计较}\end{exemple}
\begin{exemple}\pjya{ɯ-kɤtɯpa me}\hspace{5pt}\pcmn{他不计较}\end{exemple}
\begin{exemple}\pjya{nɤʑo nɤ-kɤtipa rkɯn}\hspace{5pt}\pcmn{你计较得少}\end{exemple}
\begin{exemple}\pjya{nɤʑo nɤ-kɤtipa mɤ-dɤn tɕe pe}\hspace{5pt}\pcmn{你不计较就好}\end{exemple}\relationsémantique{参考}{\lien{ⓔkɤtɯpaⓗ2}{kɤtɯpa₂}}\end{entrée}

\begin{entrée}{kɤtɯpa}{₂}{ⓔkɤtɯpaⓗ2} 
\classe{vt} 
\begin{définition}\pfra{dire}\end{définition}
\begin{définition}\pcmn{说;转告}\end{définition}
\begin{exemple}\pjya{nɤʑo tɤ-tɯ-tɯt nɯ, aʑo kɤtɯpe-a ɕti}\hspace{5pt}\pcmn{你说的话,我会转告(给他)}\end{exemple}
\begin{exemple}\pjya{aʑo tɤ-tɯt-a nɯ, ɯʑo kɯ nɤ-ɕki kɤtɯpe}\hspace{5pt}\pcmn{我说的话,他会转告给你的}\end{exemple}
\begin{exemple}\pjya{kɤtɯpa-tɕi ɕti}\hspace{5pt}\pcmn{我们俩会转告}\end{exemple}\relationsémantique{参考}{\lien{ⓔkɤtɯpaⓗ1}{kɤtɯpa₁}}\end{entrée}

\begin{entrée}{kɕilu}{}{ⓔkɕilu} 
\classe{n} 
\begin{définition}\pfra{année du chien}\end{définition}
\begin{définition}\pcmn{狗年}\end{définition}\étymologie{kʰʲi.lo}\end{entrée}

\begin{entrée}{kuɣrummɤɣ}{}{ⓔkuɣrummɤɣ} 
\classe{n} 
\begin{définition}\pfra{une espèce de champignon}\end{définition}
\begin{définition}\pcmn{一种蘑菇}\end{définition}
\begin{exemple}\pjya{kuɣrummɤɣ nɯ ɕkrɤz kɯ-xtɕi ɯ-ŋgɯ tu-ɬoʁ ŋu, kɯ-wɣrum ʁɟa ʑo ŋu, kɤ-ndza wuma ʑo mɯm. ftɕar tɕe tu-ɬoʁ ŋu.}\hspace{5pt}\pcmn{\lien{ⓔkuɣrummɤɣ}{kuɣrummɤɣ}长在比较矮小的青冈树林里,均呈白色。很好吃,一般在夏天生长。}\end{exemple}\relationsémantique{参考}{\lien{ⓔtɤjmɤɣ}{tɤjmɤɣ}}\relationsémantique{参考}{\lien{ⓔwɣrum}{wɣrum}}\end{entrée}

\begin{entrée}{khu}{}{ⓔkhu} 
\classe{n} 
\begin{définition}\pfra{tigre}\end{définition}
\begin{définition}\pcmn{老虎}\end{définition}\end{entrée}

\begin{entrée}{kha}{}{ⓔkha} 
\classe{n} \sens{1}
\begin{définition}\pfra{maison}\end{définition}
\begin{définition}\pcmn{房子}\end{définition}\sens{2}
\begin{définition}\pfra{place originelle}\end{définition}
\begin{définition}\pcmn{原位,原来放过的地方}\end{définition}\relationsémantique{参考}{\lien{ⓔrɤkha}{rɤkha}}\relationsémantique{参考}{\lien{ⓔqhaqhu}{qhaqhu}}\end{entrée}

\begin{entrée}{khamu}{}{ⓔkhamu} 
\classe{n} 
\begin{définition}\pfra{cuisine}\end{définition}
\begin{définition}\pcmn{炊事}\end{définition}\relationsémantique{参考}{\lien{}{nɯkhama}}\end{entrée}

\begin{entrée}{khamba}{}{ⓔkhamba} 
\classe{n} 
\begin{définition}\pfra{gris clair}\end{définition}
\begin{définition}\pcmn{灰白色}\end{définition}\étymologie{kʰam.pa}\end{entrée}

\begin{entrée}{khambalawa}{}{ⓔkhambalawa} 
\classe{n} 
\begin{définition}\pfra{vêtement en laine grise}\end{définition}
\begin{définition}\pcmn{灰色的毛裙}\end{définition}\end{entrée}

\begin{entrée}{khaŋfkot}{}{ⓔkhaŋfkot} 
\classe{n} 
\begin{définition}\pfra{architecte}\end{définition}
\begin{définition}\pcmn{建筑师,设计房子的人}\end{définition}\étymologie{kʰaŋ.bkod}\end{entrée}

\begin{entrée}{khara}{}{ⓔkhara} 
\classe{n} 
\begin{définition}\pfra{part du produit de la chasse donnée aux amis}\end{définition}
\begin{définition}\pcmn{分给朋友的猎物}\end{définition}
\begin{exemple}\pjya{a-khara a-nɯ-tɯ-βze je!}\hspace{5pt}\pcmn{你给我一份吧}\end{exemple}\end{entrée}

\begin{entrée}{khari}{}{ⓔkhari} 
\classe{n} 
\begin{définition}\pfra{turban}\end{définition}
\begin{définition}\pcmn{包头巾}\end{définition}\étymologie{kʰa.dkris}\end{entrée}

\begin{entrée}{kharwut}{}{ⓔkharwut} 
\classe{n} 
\begin{définition}\pfra{fièvre aphteuse}\end{définition}
\begin{définition}\pcmn{口蹄疫}\end{définition}\étymologie{*kʰa.rbod}\end{entrée}

\begin{entrée}{khatoʁ}{}{ⓔkhatoʁ} 
\classe{n} 
\begin{définition}\pfra{de toutes les couleurs}\end{définition}
\begin{définition}\pcmn{各种颜色}\end{définition}
\begin{exemple}\pjya{raz ɲɯ-mpɕɤr tɕe khatoʁ ʑo ɣɤʑu}\hspace{5pt}\pcmn{布很漂亮,是花色的}\end{exemple}
\begin{exemple}\pjya{thaχtsa chɯ́-wɣ-βzu tɕe, tɤ-ri khatoʁ ʑo pjɯ-tu ra}\hspace{5pt}\pcmn{制作花带的时候要有不同颜色的线}\end{exemple}
\begin{exemple}\pjya{mɯntoʁ ɯ-mdoʁ ɲɯ-mpɕɤr khatoʁ ʑo ɣɤʑu}\hspace{5pt}\pcmn{花的颜色很漂亮,各种颜色都有}\end{exemple}\étymologie{kʰa.dog}\end{entrée}

\begin{entrée}{khatʂu}{}{ⓔkhatʂu} 
\classe{n} 
\begin{définition}\pfra{merci}\end{définition}
\begin{définition}\pcmn{谢谢}\end{définition}\étymologie{kʰa.dro}\end{entrée}

\begin{entrée}{khatʂulɤfsɤm}{}{ⓔkhatʂulɤfsɤm} 
\classe{intj} 
\begin{définition}\pfra{merci infiniment}\end{définition}
\begin{définition}\pcmn{万分感谢}\end{définition}\end{entrée}

\begin{entrée}{khɤβdɤr}{}{ⓔkhɤβdɤr} 
\classe{n} 
\begin{définition}\pfra{blague}\end{définition}
\begin{définition}\pcmn{玩笑}\end{définition}
\begin{exemple}\pjya{khɤβdɤr tɤ-βzu-t-a}\hspace{5pt}\pcmn{我开玩笑了}\end{exemple}\relationsémantique{参考}{\lien{ⓔnɯkhɤβdɤr}{nɯkhɤβdɤr}}\end{entrée}

\begin{entrée}{khɤβdi}{}{ⓔkhɤβdi} 
\classe{n} 
\begin{définition}\pfra{belles paroles}\end{définition}
\begin{définition}\pcmn{好话}\end{définition}
\begin{exemple}\pjya{ɯʑo kɯ khɤβdi tu-βze ɕti}\hspace{5pt}\pcmn{他说得很好听(其实不知道是不是那回事)}\end{exemple}\relationsémantique{同义词}{\lien{ⓔkhɤmpɕɤr}{khɤmpɕɤr}}\end{entrée}

\begin{entrée}{khɤβɣa}{}{ⓔkhɤβɣa} 
\classe{n} 
\begin{définition}\pfra{moulin à main}\end{définition}
\begin{définition}\pcmn{手磨}\end{définition}\relationsémantique{参考}{\lien{ⓔkha}{kha}}\relationsémantique{参考}{\lien{ⓔβɣa}{βɣa}}\end{entrée}

\begin{entrée}{khɤβrda}{}{ⓔkhɤβrda} 
\classe{n} 
\begin{définition}\pfra{parole auspicieuse}\end{définition}
\begin{définition}\pcmn{吉利的话}\end{définition}
\begin{exemple}\pjya{khɤβrda kɯ-sna ta-βzu}\hspace{5pt}\pcmn{他说了吉利的话}\end{exemple}
\begin{exemple}\pjya{nɯnɯ ma-tɯ-ti ma khɤβrda mɤ-kɯ-sna ɲɯ-ŋu}\hspace{5pt}\pcmn{你别这样说,这句话不吉利}\end{exemple}\étymologie{kʰa.brda}\end{entrée}

\begin{entrée}{khɤβsa}{}{ⓔkhɤβsa} 
\classe{n} 
\begin{définition}\pfra{beignet}\end{définition}
\begin{définition}\pcmn{油条}\end{définition}\end{entrée}

\begin{entrée}{khɤβzaŋ}{}{ⓔkhɤβzaŋ} 
\classe{intj} 
\begin{définition}\pfra{formule de politesse pour exprimer son arrivée}\end{définition}
\begin{définition}\pcmn{客人表明自己到来的客套话}\end{définition}\étymologie{kʰa.bzaŋ}\end{entrée}

\begin{entrée}{khɤcɤl}{}{ⓔkhɤcɤl} 
\classe{n} 
\begin{définition}\pfra{sujet de discussion}\end{définition}
\begin{définition}\pcmn{谈话内容}\end{définition}\end{entrée}

\begin{entrée}{khɤɕa}{}{ⓔkhɤɕa} 
\classe{n} 
\begin{définition}\pfra{cerf (cervus elaphus kansuensis)}\end{définition}
\begin{définition}\pcmn{马鹿}\end{définition}\end{entrée}

\begin{entrée}{khɤɕkhɤr}{}{ⓔkhɤɕkhɤr} 
\classe{n} 
\begin{définition}\pfra{place du maître de maison, à l'est}\end{définition}
\begin{définition}\pcmn{主人坐的地方,往东方}\end{définition}\end{entrée}

\begin{entrée}{khɤɕpi}{}{ⓔkhɤɕpi} 
\classe{n} 
\begin{définition}\pfra{petit sac en cuir que l'on attache à la taille}\end{définition}
\begin{définition}\pcmn{拴在腰带上的皮包}\end{définition}\end{entrée}

\begin{entrée}{khɤdaʁ}{}{ⓔkhɤdaʁ} 
\classe{n} 
\begin{définition}\pfra{khatag}\end{définition}
\begin{définition}\pcmn{哈达}\end{définition}
\begin{exemple}\pjya{khɤdaʁ lɤ-lat-a}\hspace{5pt}\pcmn{我给了他哈达}\end{exemple}
\begin{exemple}\pjya{sprɯskɯ kɯ a-khɤdaʁ tha-lɤt}\hspace{5pt}\pcmn{活佛给了我哈达}\end{exemple}\étymologie{kʰa.btags}\end{entrée}

\begin{entrée}{khɤdɤrdɤr}{}{ⓔkhɤdɤrdɤr} 
\classe{n} 
\begin{définition}\pfra{neige en grains}\end{définition}
\begin{définition}\pcmn{雪籽}\end{définition}
\begin{exemple}\pjya{khɤdɤrdɤr pa-lɤt}\hspace{5pt}\pcmn{下了雪籽}\end{exemple}\end{entrée}

\begin{entrée}{khɤdi}{}{ⓔkhɤdi} 
\classe{n} 
\begin{définition}\pfra{place de la maîtresse de maison, au nord}\end{définition}
\begin{définition}\pcmn{女主人坐的地方,往北方}\end{définition}\end{entrée}

\begin{entrée}{khɤfɕɤt}{}{ⓔkhɤfɕɤt} 
\classe{n} 
\begin{définition}\pfra{prière}\end{définition}
\begin{définition}\pcmn{祈祷}\end{définition}
\begin{exemple}\pjya{ʑɯβdaʁ ɯ-ɕki khɤfɕɤt tɤ-βzu-t-a}\hspace{5pt}\pcmn{我向山神祈祷了}\end{exemple}\étymologie{kʰa.bɕad}\end{entrée}

\begin{entrée}{khɤjlɤn}{}{ⓔkhɤjlɤn} 
\classe{n} 
\begin{définition}\pfra{vœux}\end{définition}
\begin{définition}\pcmn{许愿}\end{définition}
\begin{exemple}\pjya{kɤ-qur khɤjlɤn tɤ-nɯβzu-t-a ɕti}\hspace{5pt}\pcmn{我答应帮他,我发誓要帮他}\end{exemple}\relationsémantique{参考}{\lien{ⓔnɯkhɤjlɤn}{nɯkhɤjlɤn}}\étymologie{kʰas.len}\end{entrée}

\begin{entrée}{khɤjmu}{}{ⓔkhɤjmu} 
\classe{n} 
\begin{définition}\pfra{cuisine, premier étage}\end{définition}
\begin{définition}\pcmn{藏式房屋的第二楼(厨房)}\end{définition}
\begin{exemple}\pjya{jiʑo kɯrɯ kha ɣɯ khɤjmu ɯ-ŋgɯ sɤ-ɤmdzɯ ʑakastaka tu tɕe khɤɕkhɤr kɤ-ti ci tu tɕe, nɯ tɕu tɤ-tɕɯ ra cho kɯ-ŋgro ra ku-omdzɯ-nɯ ŋu, ɕaŋlo kɤ-ti ci tu tɕe, nɯtɕu rgɤrgɯn ra cho smi ɯ-kɯ-βlɯ ra ku-omdzɯ-nɯ ŋu, khɤdi nɯ tɕu tɕheme kɯ-nɯkhamu cho kha tɤ-mu nɯ ku-kɯ-ɤmdzɯ ŋu, tɕe nɯ sqhi nɯ pjɯ́-wɣ-nɤkhar ŋu, saŋdi kɤ-ti ci tu tɕe, nɯ ɯ-pɕoʁ nɯ tɕu si kɤ-βlɯ ɯ-spa ɯ-sɤ-ta ŋu, nɯ ɯ-pɕoʁ nɯ tɕu tɯrme kɯ-ɤmdzɯ me. kɤ-βlɯ ɯ-spa ɯ-sɤ-ta ŋu, nɯ ɯ-pɕoʁ nɯ tɕu tɯrme kɯ-ɤmdzɯ me.}\hspace{5pt}\pcmn{我们藏族的厨房里有各种座位,一种叫\lien{ⓔkhɤɕkhɤr}{khɤɕkhɤr},是男人和贵宾的座位,一种叫\lien{ⓔɕaŋloⓗ1ⓗ2}{ɕaŋlo},是老年人和烧火的人的座位,一种叫\lien{ⓔkhɤdi}{khɤdi},是做饭的女子和家庭主妇的座位,这样围着火上的三脚,另一种是\lien{ⓔsaŋdi}{saŋdi},是放烧火柴的地方,没有人在那里坐。}\end{exemple}\relationsémantique{参考}{\lien{ⓔnɯkhɤrŋgɯ}{nɯkhɤrŋgɯ}}\end{entrée}

\begin{entrée}{khɤkɤcu}{}{ⓔkhɤkɤcu} 
\classe{n} 
\begin{définition}\pfra{est de la maison}\end{définition}
\begin{définition}\pcmn{房子的东边}\end{définition}\relationsémantique{参考}{\lien{ⓔkha}{kha}}\relationsémantique{参考}{\lien{ⓔɯ-kɤcu}{ɯ-kɤcu}}\end{entrée}

\begin{entrée}{khɤku raŋri}{}{ⓔkhɤku raŋri} 
\classe{n} 
\begin{définition}\pfra{chaque maison}\end{définition}
\begin{définition}\pcmn{每一户}\end{définition}\end{entrée}

\begin{entrée}{khɤkɯm}{}{ⓔkhɤkɯm} 
\classe{n} 
\begin{définition}\pfra{entrée de la maison}\end{définition}
\begin{définition}\pcmn{门口}\end{définition}\relationsémantique{参考}{\lien{ⓔkha}{kha}}\relationsémantique{参考}{\lien{ⓔkɯm}{kɯm}}\end{entrée}

\begin{entrée}{khɤlɤβ}{}{ⓔkhɤlɤβ} 
\classe{n} 
\begin{définition}\pfra{couvercle}\end{définition}
\begin{définition}\pcmn{锅的盖子}\end{définition}\étymologie{kʰa.leb}\end{entrée}

\begin{entrée}{khɤli,rgi}{}{ⓔkhɤli,rgi} 
\classe{n}
\classe{vs} \paradigme{dir}{tɤ-}
\begin{définition}\pfra{avoir une bonne renommée}\end{définition}
\begin{définition}\pcmn{名声好;受人尊重;受人信任}\end{définition}
\begin{exemple}\pjya{ɯ-khɤli ɲɯ-rgi}\hspace{5pt}\pcmn{他名声很好}\end{exemple}\relationsémantique{Component 1}{\lien{}{khɤli}}\relationsémantique{Component 2}{\lien{}{rgi}}\end{entrée}

\begin{entrée}{khɤmdu}{}{ⓔkhɤmdu} 
\classe{n} 
\begin{définition}\pfra{rênes}\end{définition}
\begin{définition}\pcmn{缰绳}\end{définition}\end{entrée}

\begin{entrée}{khɤmɬa}{}{ⓔkhɤmɬa} 
\classe{n} 
\begin{définition}\pfra{cérémonie}\end{définition}
\begin{définition}\pcmn{庆祝的仪式}\end{définition}\étymologie{kʰams.lha?}\end{entrée}

\begin{entrée}{khɤmpɕɤr}{}{ⓔkhɤmpɕɤr} 
\classe{n} 
\begin{définition}\pfra{belles paroles}\end{définition}
\begin{définition}\pcmn{好话}\end{définition}
\begin{exemple}\pjya{ɯʑo kɯ khɤmpɕɤr tu-βze ɕti}\hspace{5pt}\pcmn{他说得很好听(其实不知道是不是那回事)}\end{exemple}\relationsémantique{同义词}{\lien{ⓔkhɤβdi}{khɤβdi}}\end{entrée}

\begin{entrée}{khɤndɤcu}{}{ⓔkhɤndɤcu} 
\classe{n} 
\begin{définition}\pfra{ouest de la maison}\end{définition}
\begin{définition}\pcmn{房子的西边}\end{définition}\relationsémantique{参考}{\lien{ⓔkha}{kha}}\relationsémantique{参考}{\lien{ⓔɯ-ndɤcu}{ɯ-ndɤcu}}\end{entrée}

\begin{entrée}{khɤndɯn}{}{ⓔkhɤndɯn} 
\classe{n} 
\begin{définition}\pfra{lecture de sutra}\end{définition}
\begin{définition}\pcmn{念经}\end{définition}\étymologie{kʰa.ⁿdon}\end{entrée}

\begin{entrée}{khɤndzo}{}{ⓔkhɤndzo} 
\classe{n} 
\begin{définition}\pfra{étuve}\end{définition}
\begin{définition}\pcmn{蒸笼}\end{définition}\end{entrée}

\begin{entrée}{khɤntshɤm}{}{ⓔkhɤntshɤm} 
\classe{n} 
\begin{définition}\pfra{limite}\end{définition}
\begin{définition}\pcmn{界限;交界地方}\end{définition}
\begin{exemple}\pjya{tɯ-kɤrme ɯ-khɤntshɤm}\hspace{5pt}\pcmn{开始长头发的地方}\end{exemple}\étymologie{kʰa.mtsʰams}\end{entrée}

\begin{entrée}{khɤɴqra}{}{ⓔkhɤɴqra} 
\classe{n} 
\begin{définition}\pfra{maison en ruine}\end{définition}
\begin{définition}\pcmn{烂房子}\end{définition}\relationsémantique{参考}{\lien{ⓔkha}{kha}}\relationsémantique{参考}{\lien{ⓔɯ-ɴqra}{ɯ-ɴqra}}\end{entrée}

\begin{entrée}{khɤpa}{}{ⓔkhɤpa} 
\classe{n} 
\begin{définition}\pfra{rez de chaussée}\end{définition}
\begin{définition}\pcmn{一楼}\end{définition}
\begin{exemple}\pjya{khɤpa ri rɤʑi-a}\hspace{5pt}\pcmn{我在院子里}\end{exemple}\relationsémantique{参考}{\lien{ⓔkha}{kha}}\relationsémantique{参考}{\lien{ⓔpaⓗ3ⓝɯ-pa}{ɯ-pa}}\end{entrée}

\begin{entrée}{khɤphrɯ}{}{ⓔkhɤphrɯ} 
\classe{n} 
\begin{définition}\pfra{action d'asperger de l'eau}\end{définition}
\begin{définition}\pcmn{喷水}\end{définition}
\begin{exemple}\pjya{khɤphrɯ ma-tɤ-tɯ-lɤt}\hspace{5pt}\pcmn{你不要喷水}\end{exemple}\relationsémantique{参考}{\lien{ⓔnɯkhɤphrɯ}{nɯkhɤphrɯ}}\étymologie{kʰa.pʰru}\end{entrée}

\begin{entrée}{khɤpɯ}{}{ⓔkhɤpɯ} 
\classe{n} 
\begin{définition}\pfra{petite cabane}\end{définition}
\begin{définition}\pcmn{小屋子}\end{définition}\relationsémantique{参考}{\lien{ⓔkha}{kha}}\relationsémantique{参考}{\lien{ⓔtɤ-pɯ}{tɤ-pɯ}}\end{entrée}

\begin{entrée}{khɤru}{}{ⓔkhɤru} 
\classe{n} 
\begin{définition}\pfra{porte de la cuisine}\end{définition}
\begin{définition}\pcmn{厨房的门}\end{définition}\end{entrée}

\begin{entrée}{khɤrka}{}{ⓔkhɤrka} 
\classe{n} 
\begin{définition}\pfra{plafond}\end{définition}
\begin{définition}\pcmn{天花板}\end{définition}\end{entrée}

\begin{entrée}{khɤrlɤn}{}{ⓔkhɤrlɤn} 
\classe{n} 
\begin{définition}\pfra{construction, réparation d'une maison}\end{définition}
\begin{définition}\pcmn{装修}\end{définition}
\begin{exemple}\pjya{khɤrlɤn tɤ-βzu-j}\hspace{5pt}\pcmn{我们修房子了}\end{exemple}\relationsémantique{参考}{\lien{ⓔrɯkhɤrlɤn}{rɯkhɤrlɤn}}\end{entrée}

\begin{entrée}{khɤrma}{}{ⓔkhɤrma} 
\classe{n} 
\begin{définition}\pfra{injure}\end{définition}
\begin{définition}\pcmn{咒人的话}\end{définition}
\begin{exemple}\pjya{a-khɤrma pa-βzu}\hspace{5pt}\pcmn{他咒了我}\end{exemple}\relationsémantique{参考}{\lien{ⓔsɯkhɤrma}{sɯkhɤrma}}\relationsémantique{参考}{\lien{ⓔɯ-rɟa}{ɯ-rɟa}}\étymologie{*kʰa.rma}\end{entrée}

\begin{entrée}{khɤrmi}{}{ⓔkhɤrmi} 
\classe{n} 
\begin{définition}\pfra{nom de la maison}\end{définition}
\begin{définition}\pcmn{房名}\end{définition}\relationsémantique{参考}{\lien{ⓔkha}{kha}}\relationsémantique{参考}{\lien{ⓔtɤ-rmi}{tɤ-rmi}}\end{entrée}

\begin{entrée}{khɤrtsɤɣ}{}{ⓔkhɤrtsɤɣ} 
\classe{n} 
\begin{définition}\pfra{étage}\end{définition}
\begin{définition}\pcmn{楼房}\end{définition}\relationsémantique{参考}{\lien{ⓔkha}{kha}}\relationsémantique{参考}{\lien{ⓔtɤ-rtsɤɣ}{tɤ-rtsɤɣ}}\end{entrée}

\begin{entrée}{khɤrɯm}{}{ⓔkhɤrɯm} 
\classe{n} 
\begin{définition}\pfra{ulcère sur la bouche}\end{définition}
\begin{définition}\pcmn{嘴上的疮}\end{définition}
\begin{exemple}\pjya{tɯ-ʑo staʁ kɯ-χtso ra nɯ-khɯ-tsa kú-wɣ-ntɕhoz tɕe, tɯ-mtɕhi maŋtaʁ nɯ tɕu khɤrɯm ɲɯ-ɬoʁ, tɯ-ʑo staʁ kɯ-χtso ra nɯ-khɯtsa kú-wɣ-ntɕhoz tɕe, tɯ-mtɕhi maŋpa nɯ tɕu khɤrɯm ɲɯ-ɬoʁ ŋu, tɯ-ʑo cho kɯ-naχtɕɯɣ ra nɯ-khɯtsa kú-wɣ-ntɕhoz tɕe, tɯ-mtɕhi ɯ-rkɯ nɯ tɕu khɤrɯm ɲɯ-ɬoʁ ŋu tu-kɯ-ti ɲɯ-ŋu}\hspace{5pt}\pcmn{人家说,当你用别人的碗时,就会染上\lien{ⓔkhɤrɯm}{khɤrɯm}这种病,如果拥有碗的那个人比你干净,痘痘长在嘴唇的上面;比你脏,痘痘就会长在嘴唇的下面;跟你一样干净,就会长在嘴唇的两边}\end{exemple}
\begin{exemple}\pjya{kɯmaʁ tɯrme ra nɯ-khɯtsa kú-wɣ-tɕhoz tɕe tɯ-mtɕhi ɯ-taʁ zmbɤr ɲɯ-ɬoʁ ŋgrɤl tɕe, nɯ ʑmbɤr nɯ khɤrɯm rmi}\hspace{5pt}\pcmn{用了别人的碗时嘴上会长一种疮,这种疮叫\lien{ⓔkhɤrɯm}{khɤrɯm}}\end{exemple}\end{entrée}

\begin{entrée}{khɤʁɤri}{}{ⓔkhɤʁɤri} 
\classe{n} 
\begin{définition}\pfra{devant la maison}\end{définition}
\begin{définition}\pcmn{房子的前面}\end{définition}\relationsémantique{参考}{\lien{ⓔkha}{kha}}\relationsémantique{参考}{\lien{ⓔɯ-ʁɤri}{ɯ-ʁɤri}}\end{entrée}

\begin{entrée}{khɤsnɯm}{}{ⓔkhɤsnɯm} 
\classe{n} 
\begin{définition}\pfra{humidification avec la salive}\end{définition}
\begin{définition}\pcmn{用口水弄湿}\end{définition}
\begin{exemple}\pjya{tɤfsɤri ɯ-taʁ khɤsnɯm thɯ-lat-a}\hspace{5pt}\pcmn{我用口水把麻绳弄湿}\end{exemple}\relationsémantique{参考}{\lien{ⓔnɯkhɤsnɯm}{nɯkhɤsnɯm}}\étymologie{*kʰa.snum}\end{entrée}

\begin{entrée}{khɤsta}{}{ⓔkhɤsta} 
\classe{n} 
\begin{définition}\pfra{fondations}\end{définition}
\begin{définition}\pcmn{房基(准备修房子的地方;把房子拆下来了以后,有过房子的地方)}\end{définition}\relationsémantique{参考}{\lien{ⓔkha}{kha}}\relationsémantique{参考}{\lien{ⓔtɯ-sta}{tɯ-sta}}\end{entrée}

\begin{entrée}{khɤt}{}{ⓔkhɤt} 
\classe{vt} \paradigme{dir}{tɤ-}\sens{1}
\begin{définition}\pfra{aller partout}\end{définition}
\begin{définition}\pcmn{到处逛}\end{définition}
\begin{exemple}\pjya{aʁɤndɯndɤt kɤ-nɤmɲo ɕ-to-khɤt}\hspace{5pt}\pcmn{他到处去观看了}\end{exemple}
\begin{exemple}\pjya{alo ʑɯmkhɤm ɕ-ta-khɤt}\hspace{5pt}\pcmn{他山上去逛了}\end{exemple}\sens{2}
\begin{définition}\pfra{faire une action pendant longtemps ou à plusieurs reprises}\end{définition}
\begin{définition}\pcmn{做很多次;发生很多次;持续很久}\end{définition}
\begin{exemple}\pjya{a-ɕki tɯjʁo kɯ ta-khɤt ʑo}\hspace{5pt}\pcmn{他骂我骂了很多次}\end{exemple}
\begin{exemple}\pjya{ta-ma kɯ ta-khɤt ʑo}\hspace{5pt}\pcmn{他劳动得多}\end{exemple}
\begin{exemple}\pjya{ndzɤtshi kɯ ta-khɤt ʑo}\hspace{5pt}\pcmn{他吃得多}\end{exemple}
\begin{exemple}\pjya{khɤcɤl kɯ tɤ-khat-a ʑo}\hspace{5pt}\pcmn{我讲了很久}\end{exemple}\end{entrée}

\begin{entrée}{khɤtaʁ}{}{ⓔkhɤtaʁ} 
\classe{n} 
\begin{définition}\pfra{deuxième étage, au-dessus de la cuisine}\end{définition}
\begin{définition}\pcmn{藏式房屋的第二楼}\end{définition}\relationsémantique{参考}{\lien{ⓔkha}{kha}}\relationsémantique{参考}{\lien{ⓔtaʁⓗ3}{taʁ₃}}\end{entrée}

\begin{entrée}{khɤtɤcu}{}{ⓔkhɤtɤcu} 
\classe{n} 
\begin{définition}\pfra{couloir}\end{définition}
\begin{définition}\pcmn{走廊}\end{définition}\end{entrée}

\begin{entrée}{khɤtɕɯ}{}{ⓔkhɤtɕɯ} 
\classe{n} 
\begin{définition}\pfra{petite chambre}\end{définition}
\begin{définition}\pcmn{小房间}\end{définition}\end{entrée}

\begin{entrée}{khɤthɤβ}{}{ⓔkhɤthɤβ} 
\classe{n} \sens{1}
\begin{définition}\pfra{dans le village, dans la rue}\end{définition}
\begin{définition}\pcmn{街上,村子里}\end{définition}\sens{2}
\begin{définition}\pfra{au bas de l'immeuble}\end{définition}
\begin{définition}\pcmn{楼下}\end{définition}
\begin{exemple}\pjya{aki khɤthɤβ ʑo pɯ-azɣɯt-a}\hspace{5pt}\pcmn{我已经到楼下了}\end{exemple}\relationsémantique{参考}{\lien{ⓔkha}{kha}}\relationsémantique{参考}{\lien{ⓔthɤβ}{thɤβ}}\end{entrée}

\begin{entrée}{khɤtsa}{}{ⓔkhɤtsa} 
\classe{n} 
\begin{définition}\pfra{saleté entre les dents}\end{définition}
\begin{définition}\pcmn{牙垢}\end{définition}\étymologie{kʰa.tsa}\end{entrée}

\begin{entrée}{khɤtshoʁ}{}{ⓔkhɤtshoʁ} 
\classe{n} 
\begin{définition}\pfra{type de pas d'aiguille}\end{définition}
\begin{définition}\pcmn{缝针的方法}\end{définition}\end{entrée}

\begin{entrée}{khɤwɯ}{}{ⓔkhɤwɯ} 
\classe{n} 
\begin{définition}\pfra{troisième étage, où l'on dort}\end{définition}
\begin{définition}\pcmn{四楼,睡觉的地方}\end{définition}\end{entrée}

\begin{entrée}{khɤxtu}{}{ⓔkhɤxtu} 
\classe{n} 
\begin{définition}\pfra{terrasse en haut des maisons tibétaines, toit}\end{définition}
\begin{définition}\pcmn{屋顶平台【房背】}\end{définition}\end{entrée}

\begin{entrée}{khɤxtɤlwɤt}{}{ⓔkhɤxtɤlwɤt} 
\classe{n} 
\begin{définition}\pfra{avant-toit}\end{définition}
\begin{définition}\pcmn{屋檐}\end{définition}
\begin{exemple}\pjya{khɤxtɤlwɤt ɯ-pa}\hspace{5pt}\pcmn{屋檐下}\end{exemple}\end{entrée}

\begin{entrée}{khɤxtɤmbro}{}{ⓔkhɤxtɤmbro} 
\classe{n} 
\begin{définition}\pfra{terrasse la plus haute}\end{définition}
\begin{définition}\pcmn{最高的房背}\end{définition}\end{entrée}

\begin{entrée}{khɤxtɤndo}{}{ⓔkhɤxtɤndo} 
\classe{n} 
\begin{définition}\pfra{bordure du toit}\end{définition}
\begin{définition}\pcmn{房背的边缘}\end{définition}\end{entrée}

\begin{entrée}{khɤxtɤndorɤm}{}{ⓔkhɤxtɤndorɤm} 
\classe{n} 
\begin{définition}\pfra{parapet du toit}\end{définition}
\begin{définition}\pcmn{房背上的栏杆}\end{définition}
\begin{exemple}\pjya{khɤxtɤndo laχtsɯ kú-wɣ-lɤt tɕe rorʁe ʁnɯ-ldʑa ntsɯ kú-wɣ-saχɕɯβ tɕe ɯ-rchɤβ nɯtɕu tɤrɤm kú-wɣ-sɤʑɯrja tɕe khɤxtu kú-wɣ-sɤɣur tɕe nɯ tɤrɤm nɯ khɤxtɤndorɤm rmi}\hspace{5pt}\pcmn{在房背的边缘排几根小柱头,穿上成双的横杆,在中间插上一排木板,把房背拦住,这种木板叫\lien{ⓔkhɤxtɤndorɤm}{khɤxtɤndorɤm}}\end{exemple}\end{entrée}

\begin{entrée}{khɤχpi}{}{ⓔkhɤχpi} 
\classe{n} 
\begin{définition}\pfra{proverbe}\end{définition}
\begin{définition}\pcmn{谚语,俗话}\end{définition}
\begin{exemple}\pjya{khɤχpi kú-wɣ-ta tɕe .... tu-kɯ-ti ɲɯ-ŋgrɤl}\hspace{5pt}\pcmn{俗话说:}\end{exemple}\relationsémantique{参考}{\lien{ⓔχpi}{χpi}}\end{entrée}

\begin{entrée}{khɤzɟi}{}{ⓔkhɤzɟi} 
\classe{n} 
\begin{définition}\pfra{sac pour nourrir les chevaux, que l'on attache à sa bouche}\end{définition}
\begin{définition}\pcmn{喂马的食料袋,系在嘴上}\end{définition}\étymologie{kʰa.sgʲe}\end{entrée}

\begin{entrée}{khe}{}{ⓔkhe} 
\classe{vs} \sens{1}\paradigme{dir}{nɯ-}
\begin{définition}\pfra{stupide}\end{définition}
\begin{définition}\pcmn{蠢}\end{définition}
\begin{exemple}\pjya{kɯ-khe ci ɲɯ-ŋu}\hspace{5pt}\pcmn{他是笨蛋}\end{exemple}
\begin{exemple}\pjya{kɯ-khe ci ɲɯ-tɯ-ŋu}\hspace{5pt}\pcmn{你是笨蛋}\end{exemple}
\begin{exemple}\pjya{tɯrme ɲɯ-khe}\hspace{5pt}\pcmn{那个人很笨}\end{exemple}\sens{2}\paradigme{dir}{tɤ-}\paradigme{dir}{nɯ-}\paradigme{dir}{pɯ-}
\begin{définition}\pfra{sombre}\end{définition}
\begin{définition}\pcmn{天阴}\end{définition}
\begin{définition}\pfra{considérer comme un imbécile}\end{définition}
\begin{définition}\pcmn{把……当做傻瓜,诬蔑}\end{définition}
\begin{définition}\pfra{se moquer de soi-même, faire de l'autodérision}\end{définition}
\begin{définition}\pcmn{贬低自己}\end{définition}
\begin{exemple}\pjya{tɯ-mɯ ɲɯ-khe}\hspace{5pt}\pcmn{天很阴}\end{exemple}
\begin{exemple}\pjya{tɯ-mɯ to-khe}\hspace{5pt}\pcmn{天气变得不好}\end{exemple}
\begin{exemple}\pjya{ɯ-ɲɤm ɲɯ-khe}\hspace{5pt}\pcmn{他很瘦}\end{exemple}
\begin{exemple}\pjya{ma-pɯ-kɯ-ɣɤkhe-a ma ɲɯ-sɤɣdɯɣ}\hspace{5pt}\pcmn{你不要把我当傻瓜,很讨厌}\end{exemple}
\begin{exemple}\pjya{fsapaʁ ra nɯ-ɲɤm ʑo ɲɯ-ɣɤkhe ŋgrɤl}\hspace{5pt}\pcmn{蜱令牲畜变瘦}\end{exemple}
\begin{exemple}\pjya{ma-pɯ-tɯ-ʑɣɤɣɤkhe}\hspace{5pt}\pcmn{不要贬低自己}\end{exemple}\relationsémantique{参考}{\lien{ⓔtɤkhe}{tɤkhe}}\relationsémantique{参考}{\lien{ⓔnɯɲɤmkhe}{nɯɲɤmkhe}}
\begin{sous-entrée}{ɣɤkhe}{ⓔkheⓢ2ⓝɣɤkhe} 
\classe{vt} \end{sous-entrée}

\begin{sous-entrée}{ʑɣɤɣɤkhe}{ⓔkheⓢ2ⓝʑɣɤɣɤkhe} 
\classe{vi} \end{sous-entrée}

\begin{sous-entrée}{sɤzɣɤkhe}{ⓔkheⓢ2ⓝsɤzɣɤkhe} 
\classe{vi} 
\begin{définition}\pfra{considérer les gens comme des imbéciles}\end{définition}
\begin{définition}\pcmn{把别人当做傻瓜,诬蔑别人}\end{définition}\end{sous-entrée}

\end{entrée}

\begin{entrée}{khi}{}{ⓔkhi} 
\classe{part} 
\begin{définition}\pfra{dit-on (ouï-dire)}\end{définition}
\begin{définition}\pcmn{据说}\end{définition}\end{entrée}

\begin{entrée}{khiɤɣ}{}{ⓔkhiɤɣ} 
\classe{idph.1} 
\begin{définition}\pfra{bruit de glissement}\end{définition}
\begin{définition}\pcmn{滑动的声音}\end{définition}
\begin{exemple}\pjya{khiɤɣ ʑo thɯ-ŋgio-a}\hspace{5pt}\pcmn{我嗞溜一声就滑倒了}\end{exemple}\end{entrée}

\begin{entrée}{khiɤt}{}{ⓔkhiɤt} 
\classe{idph.1} 
\begin{définition}\pfra{bruit de glissement}\end{définition}
\begin{définition}\pcmn{滑下来的声音}\end{définition}
\begin{exemple}\pjya{khiɤt ʑo ɲɯ-ti tɕe pɯ-ndʐaβ-a}\hspace{5pt}\pcmn{我嗞溜一声就摔倒了}\end{exemple}\end{entrée}

\begin{entrée}{khikhio}{}{ⓔkhikhio} 
\classe{n} 
\begin{définition}\pfra{rumeurs}\end{définition}
\begin{définition}\pcmn{传言;道听途说}\end{définition}\end{entrée}

\begin{entrée}{khipatsɯt}{}{ⓔkhipatsɯt} 
\classe{n} 
\begin{définition}\pfra{une espèce de chien}\end{définition}
\begin{définition}\pcmn{哈巴狗}\end{définition}\end{entrée}

\begin{entrée}{khulu}{}{ⓔkhulu} 
\classe{n} 
\begin{définition}\pfra{année du tigre}\end{définition}
\begin{définition}\pcmn{虎年}\end{définition}\relationsémantique{参考}{\lien{ⓔkhu}{khu}}\end{entrée}

\begin{entrée}{kho}{₂}{ⓔkhoⓗ2} 
\classe{n} 
\begin{définition}\pfra{chambre}\end{définition}
\begin{définition}\pcmn{房间}\end{définition}\étymologie{kʰaŋ}\end{entrée}

\begin{entrée}{kho}{₁}{ⓔkhoⓗ1} 
\classe{vt} \paradigme{dir}{nɯ-}
\begin{définition}\pfra{donner, passer, transmettre}\end{définition}
\begin{définition}\pcmn{给;递给;传}\end{définition}
\begin{exemple}\pjya{nɯ-khɤm}\hspace{5pt}\pcmn{递给(我)吧}\end{exemple}
\begin{exemple}\pjya{ɕ-kɤ-khɤm}\hspace{5pt}\pcmn{你去给(他)}\end{exemple}
\begin{exemple}\pjya{a-tɕha nɯ-khɤm}\hspace{5pt}\pcmn{把(我需要的消息)告诉我}\end{exemple}
\begin{exemple}\pjya{ɯʑo kɯ a-sci ɲɯ-khɤm ra}\hspace{5pt}\pcmn{他要还给我}\end{exemple}\relationsémantique{参考}{\lien{ⓔsɯkho}{sɯkho}}\relationsémantique{参考}{\lien{ⓔamɟɤkho}{amɟɤkho}}
\begin{sous-entrée}{ʑɣɤkho}{ⓔkhoⓗ1ⓝʑɣɤkho} 
\classe{vi}  
\grammaire{refl} 
\begin{définition}\pfra{se donner à}\end{définition}
\begin{définition}\pcmn{把自己交给}\end{définition}
\begin{exemple}\pjya{nɤʑo ʁgra ɯ-jaʁ nɯtɕu ɲɯ-tɯ-nɯ-ʑɣɤkho ʑo ɯ-mɤ-kɯ-ɕti-ci?}\hspace{5pt}\pcmn{你是不是把自己交到敌人的手里了?}\end{exemple}\end{sous-entrée}

\end{entrée}

\begin{entrée}{khoŋdaʁ}{}{ⓔkhoŋdaʁ} 
\classe{n} 
\begin{définition}\pfra{ancêtre}\end{définition}
\begin{définition}\pcmn{祖宗}\end{définition}\étymologie{kʰaŋ.bdag}\end{entrée}

\begin{entrée}{khoŋrɤl}{}{ⓔkhoŋrɤl} 
\classe{n} 
\begin{définition}\pfra{arbre creux}\end{définition}
\begin{définition}\pcmn{空心树}\end{définition}
\begin{exemple}\pjya{si khoŋrɤl tɤ-kɯ-ɤri}\hspace{5pt}\pcmn{空心树}\end{exemple}\étymologie{kʰoŋ.ral}\end{entrée}

\begin{entrée}{khorca}{}{ⓔkhorca} 
\classe{n} 
\begin{définition}\pfra{sac à dos fait de peau de veau}\end{définition}
\begin{définition}\pcmn{小牛皮整体地剥下来,缝成装行李的背包}\end{définition}\end{entrée}

\begin{entrée}{khru}{}{ⓔkhru} 
\classe{n} 
\begin{définition}\pfra{fer blanc}\end{définition}
\begin{définition}\pcmn{生铁}\end{définition}\étymologie{kʰro}\end{entrée}

\begin{entrée}{khra}{}{ⓔkhra} 
\classe{vt} \paradigme{dir}{pɯ-}
\begin{définition}\pfra{faire une marque}\end{définition}
\begin{définition}\pcmn{划一刀}\end{définition}
\begin{exemple}\pjya{pɯ-khra-t-a}\hspace{5pt}\pcmn{我划了一刀}\end{exemple}
\begin{exemple}\pjya{ɯ-ftaʁ tɤ-βzu-t-a tɕe pɯ-khra-t-a}\hspace{5pt}\pcmn{我划了一刀做记号了}\end{exemple}\relationsémantique{参考}{\lien{ⓔakhra}{akhra}}\relationsémantique{参考}{\lien{ⓔsɤkhra}{sɤkhra}}\end{entrée}

\begin{entrée}{khrakhra}{}{ⓔkhrakhra} 
\classe{n} 
\begin{définition}\pfra{filet}\end{définition}
\begin{définition}\pcmn{网}\end{définition}\end{entrée}

\begin{entrée}{khrala}{}{ⓔkhrala} 
\classe{n} 
\begin{définition}\pfra{chien au pelage bariolé}\end{définition}
\begin{définition}\pcmn{身上有花斑的狗}\end{définition}\étymologie{*kʰra.lʷa}\end{entrée}

\begin{entrée}{khramba}{}{ⓔkhramba} 
\classe{n} 
\begin{définition}\pfra{mensonge}\end{définition}
\begin{définition}\pcmn{谎言}\end{définition}
\begin{exemple}\pjya{khramba to-βzu}\hspace{5pt}\pcmn{他说了谎话}\end{exemple}
\begin{exemple}\pjya{khramba rɟɤlpu}\hspace{5pt}\pcmn{假的国王}\end{exemple}\relationsémantique{参考}{\lien{ⓔnɯkhramba}{nɯkhramba}}\relationsémantique{参考}{\lien{ⓔrɯkhramba}{rɯkhramba}}
\begin{sous-entrée}{khramba,βzu}{ⓔkhrambaⓝkhramba,βzu}
\begin{définition}\pfra{faire semblant}\end{définition}
\begin{définition}\pcmn{假装}\end{définition}
\begin{exemple}\pjya{khramba-nɤre ɲɤ-βzu}\hspace{5pt}\pcmn{他假装笑了}\end{exemple}
\begin{exemple}\pjya{khramba-ɣɤwu ma-tɯ-βze}\hspace{5pt}\pcmn{不要装哭!}\end{exemple}\relationsémantique{同义词}{\lien{}{ʑɣɤpa}}\end{sous-entrée}

\étymologie{kʰram.ba}\end{entrée}

\begin{entrée}{khrambakɯm}{}{ⓔkhrambakɯm} 
\classe{n} 
\begin{définition}\pfra{pommette}\end{définition}
\begin{définition}\pcmn{酒窝}\end{définition}\end{entrée}

\begin{entrée}{khrambaqe}{}{ⓔkhrambaqe} 
\classe{n} 
\begin{définition}\pfra{menteur, personne malhonnête}\end{définition}
\begin{définition}\pcmn{骗子}\end{définition}\relationsémantique{参考}{\lien{ⓔtɯ-qe}{tɯ-qe}}\end{entrée}

\begin{entrée}{khrɤlmu}{}{ⓔkhrɤlmu} 
\classe{n} 
\begin{définition}\pfra{épouse}\end{définition}
\begin{définition}\pcmn{妻子}\end{définition}\étymologie{*kʰrel.mo}\end{entrée}

\begin{entrée}{khrɤlpa}{}{ⓔkhrɤlpa} 
\classe{n} 
\begin{définition}\pfra{époux}\end{définition}
\begin{définition}\pcmn{丈夫}\end{définition}\étymologie{*kʰrel.pa}\end{entrée}

\begin{entrée}{khrɤt}{₁}{ⓔkhrɤtⓗ1} 
\classe{vt} \paradigme{dir}{pɯ-}\paradigme{dir}{thɯ-}
\begin{définition}\pfra{érafler, rayer}\end{définition}
\begin{définition}\pcmn{划破}\end{définition}
\begin{définition}\pfra{érafler}\end{définition}
\begin{définition}\pcmn{划来划去}\end{définition}\relationsémantique{参考}{\lien{ⓔtɯ-tɤkhrɤz}{tɯ-tɤkhrɤz}}
\begin{sous-entrée}{sɯkhrɤt}{ⓔkhrɤtⓗ1ⓝsɯkhrɤt} 
\classe{vt} 
\begin{définition}\pfra{rayer avec}\end{définition}
\begin{définition}\pcmn{用……划破}\end{définition}
\begin{exemple}\pjya{mbrɯtɕɯ kɯ pa-sɯ-khrɤt}\hspace{5pt}\pcmn{他用刀划破了}\end{exemple}\end{sous-entrée}

\begin{sous-entrée}{rɤkhɯkhrɤt/\variante{rɤkhrɯkhrɤt}}{ⓔkhrɤtⓗ1ⓝrɤkhɯkhrɤt} 
\classe{vt} \end{sous-entrée}

\end{entrée}

\begin{entrée}{khrɤt}{₂}{ⓔkhrɤtⓗ2} 
\classe{vt} \paradigme{dir}{nɯ-}
\begin{définition}\pfra{organiser (un travail), planifier}\end{définition}
\begin{définition}\pcmn{布置}\end{définition}
\begin{exemple}\pjya{jɯfɕɯr ŋgumdʑɯɣ kɯ ji-ma pa-khrɤt}\hspace{5pt}\pcmn{昨天领导布置了我们的工作}\end{exemple}
\begin{exemple}\pjya{tɤ-pɤtso ɯ-ma nɯ-khrat-a}\hspace{5pt}\pcmn{我给小孩子布置了这个任务}\end{exemple}\end{entrée}

\begin{entrée}{khri}{}{ⓔkhri} 
\classe{n} \sens{1}
\begin{définition}\pfra{lit}\end{définition}
\begin{définition}\pcmn{床}\end{définition}\sens{2}
\begin{définition}\pfra{siège}\end{définition}
\begin{définition}\pcmn{座位}\end{définition}\étymologie{kʰri}\end{entrée}

\begin{entrée}{khro}{}{ⓔkhro} 
\classe{n} 
\begin{définition}\pfra{beaucoup}\end{définition}
\begin{définition}\pcmn{很多;很长时间}\end{définition}\end{entrée}

\begin{entrée}{khrɯ}{}{ⓔkhrɯ} 
\classe{vs}
\classe{vs}
\classe{vs} \paradigme{dir}{nɯ-}
\begin{définition}\pfra{sec}\end{définition}
\begin{définition}\pcmn{干(干了之后变硬了)}\end{définition}
\begin{exemple}\pjya{tɯndʐi ɲɤ-khrɯ}\hspace{5pt}\pcmn{皮子变干了}\end{exemple}
\begin{exemple}\pjya{sɤtɕha ɲɯ-khrɯ}\hspace{5pt}\pcmn{地很干}\end{exemple}
\begin{exemple}\pjya{tɯthɯ ɯ-ŋgɯ kɤndza pjɤ-khrɯ}\hspace{5pt}\pcmn{锅子里的饭是干的}\end{exemple}\relationsémantique{反义词}{\lien{ⓔnɯrlɤn}{nɯrlɤn}}\relationsémantique{参考}{\lien{ⓔɕɤkhrɯ}{ɕɤkhrɯ}}\relationsémantique{参考}{\lien{ⓔɯ-khrakhrɯ}{ɯ-khrakhrɯ}}\relationsémantique{Component 1}{\lien{ⓔkhrɯ}{khrɯ}}\relationsémantique{Component 2}{\lien{}{jɤβ}}
\begin{sous-entrée}{khrɯ,jɤβ}{ⓔkhrɯⓝkhrɯ,jɤβ}\end{sous-entrée}

\begin{définition}\pfra{très sec}\end{définition}
\begin{définition}\pcmn{非常干燥}\end{définition}
\begin{exemple}\pjya{tɯ-mɯ mɯ́j-lɤt tɕe tɯ-ji ra ɲɯ-khrɯ ɲɯ-jɤβ ʑo (=ɲɯ-khrɯ-jɤβ ʑo)}\hspace{5pt}\pcmn{因为不下雨,这些地方都变得很干燥}\end{exemple}\end{entrée}

\begin{entrée}{khrɯɣnɤkhrɯɣ}{}{ⓔkhrɯɣnɤkhrɯɣ} 
\classe{idph.3} 
\begin{définition}\pfra{bruit (balayage, d'un cochon qui se gratte)}\end{définition}
\begin{définition}\pcmn{扫地、抓痒的声音}\end{définition}
\begin{exemple}\pjya{khrɯɣnɤkhrɯɣ ɲɯ-ŋke}\hspace{5pt}\pcmn{他在走,发出“咯咯”声}\end{exemple}
\begin{exemple}\pjya{paʁ khrɯɣnɤkhrɯɣ ɲɯ-ʑɣɤrɤβraʁ}\hspace{5pt}\pcmn{猪在抓痒,发出“咯咯”声}\end{exemple}
\begin{sous-entrée}{khrɯɣnɤlɯɣ}{ⓔkhrɯɣnɤkhrɯɣⓝkhrɯɣnɤlɯɣ} 
\classe{idph.4} \end{sous-entrée}

\begin{sous-entrée}{ɣɤkhrɯɣlɯɣ}{ⓔkhrɯɣnɤkhrɯɣⓝɣɤkhrɯɣlɯɣ} 
\classe{vi} 
\begin{exemple}\pjya{ɲɯ-ɣɤkhrɯɣlɯɣ ntsɯ}\hspace{5pt}\pcmn{发出“咯咯”声}\end{exemple}\end{sous-entrée}

\begin{sous-entrée}{sɤkhrɯɣkhrɯɣ/\variante{sɤkhɯkhrɯɣ}}{ⓔkhrɯɣnɤkhrɯɣⓝsɤkhrɯɣkhrɯɣ} 
\classe{vt} 
\begin{exemple}\pjya{βʑɯ kɯ @mianban ɲɯ-ɤsɯ-qhrɯt tɕe ɲɯ-sɤkhrɯɣkhrɯɣ}\hspace{5pt}\pcmn{老鼠在啃砧板,发出“咯咯”声}\end{exemple}
\begin{exemple}\pjya{(ɕoŋtɕa) ɲɯ-sɤkhɯkhrɯɣ ʑo ɲɯ-ɤz-rɤɕi}\hspace{5pt}\pcmn{他在拖(木料),发出声音}\end{exemple}\end{sous-entrée}

\end{entrée}

\begin{entrée}{khrɯ,jɤβ}{}{ⓔkhrɯ,jɤβ}\relationsémantique{参考}{\lien{}{jɤβ}}\end{entrée}

\begin{entrée}{khrɯm}{}{ⓔkhrɯm} 
\classe{n} 
\begin{définition}\pfra{châtiment}\end{définition}
\begin{définition}\pcmn{刑}\end{définition}\étymologie{kʰrims}\end{entrée}

\begin{entrée}{khrɯmbjɤm}{}{ⓔkhrɯmbjɤm} 
\classe{n} 
\begin{définition}\pfra{sofa tibétain}\end{définition}
\begin{définition}\pcmn{藏式沙发,坐床}\end{définition}\end{entrée}

\begin{entrée}{khrɯŋkhrɯŋ}{}{ⓔkhrɯŋkhrɯŋ} 
\classe{idph.2} 
\begin{définition}\pfra{propre, utilisé jusqu'au bout}\end{définition}
\begin{définition}\pcmn{干净,用完}\end{définition}
\begin{exemple}\pjya{khrɯŋkhrɯŋ ɯ-tshi pjɤ-ɕkɯt}\hspace{5pt}\pcmn{它把食物吃光了}\end{exemple}
\begin{exemple}\pjya{khɯtsa ɲo-χtɕi khrɯŋkhrɯŋ ʑo}\hspace{5pt}\pcmn{碗洗得干干净净}\end{exemple}\relationsémantique{参考}{\lien{ⓔgrɯŋgrɯŋ}{grɯŋgrɯŋ}}\end{entrée}

\begin{entrée}{khrɯtɕhɯ}{}{ⓔkhrɯtɕhɯ} 
\classe{n}  
\grammaire{n.lieu} 
\begin{définition}\pfra{Khrochu}\end{définition}
\begin{définition}\pcmn{黑水}\end{définition}\end{entrée}

\begin{entrée}{khrɯtsu}{}{ⓔkhrɯtsu} 
\classe{n} 
\begin{définition}\pfra{dix mille}\end{définition}
\begin{définition}\pcmn{10000}\end{définition}
\begin{exemple}\pjya{iɕqha tɯrme nɯ ɯ-khrɯtsu kɯ-ɤro ci ɕti nɤ}\hspace{5pt}\pcmn{那个人很富有,有十万元}\end{exemple}\étymologie{kʰri.tsho}\end{entrée}

\begin{entrée}{khrɯtsusqi}{}{ⓔkhrɯtsusqi} 
\classe{num} 
\begin{définition}\pfra{cent mille}\end{définition}
\begin{définition}\pcmn{十万}\end{définition}\relationsémantique{参考}{\lien{ⓔkhrɯtsu}{khrɯtsu}}\end{entrée}

\begin{entrée}{khrɯtsɯr}{}{ⓔkhrɯtsɯr} 
\classe{n} 
\begin{définition}\pfra{petit récipient en fer utilisé pour cuire la viande pour les personnes âgées}\end{définition}
\begin{définition}\pcmn{生铁铸成的小罐子,有盖子,专门给老人炖肉}\end{définition}\end{entrée}

\begin{entrée}{khrɯzwa}{}{ⓔkhrɯzwa} 
\classe{n} 
\begin{définition}\pfra{riz cuit}\end{définition}
\begin{définition}\pcmn{饭}\end{définition}\end{entrée}

\begin{entrée}{khusri}{}{ⓔkhusri} 
\classe{n}  
\grammaire{n.lieu} 
\begin{définition}\pfra{Lixian}\end{définition}
\begin{définition}\pcmn{理想}\end{définition}\end{entrée}

\begin{entrée}{khɯ}{₂}{ⓔkhɯⓗ2} 
\classe{n} 
\begin{définition}\pfra{poils fins}\end{définition}
\begin{définition}\pcmn{细毛}\end{définition}\end{entrée}

\begin{entrée}{khɯ}{₁}{ⓔkhɯⓗ1} 
\classe{vs} \sens{1}
\begin{définition}\pfra{être possible}\end{définition}
\begin{définition}\pcmn{可以}\end{définition}
\begin{exemple}\pjya{nɯ kɤti mɤ-kɯ-khɯ me}\hspace{5pt}\pcmn{那样说没有什么不可以的}\end{exemple}\sens{2}\paradigme{dir}{tɤ-}
\begin{définition}\pfra{être d'accord}\end{définition}
\begin{définition}\pcmn{肯;听劝}\end{définition}
\begin{exemple}\pjya{mɯ́j-tɯ-khɯ}\hspace{5pt}\pcmn{你不听}\end{exemple}
\begin{exemple}\pjya{aʑo kɤ-zrɤma mɯ-tɤ-nɯ-khɯ-a ɕti tɕe a-kɤ-nɯmɟa a-pɯ-nɯ-me ɬoʁ}\hspace{5pt}\pcmn{我自己不同意做事,没有报酬是应该}\end{exemple}
\begin{exemple}\pjya{nɤʑo mɯ-tɤ-tɯ-nɯ-khɯ ɕti tɕe, ma-tɤ-kɯ-mpɕa-a je}\hspace{5pt}\pcmn{是你自己没有同意,你不要责怪我}\end{exemple}\relationsémantique{参考}{\lien{ⓔɣɤkhɯⓗ2}{ɣɤkhɯ₂}}\end{entrée}

\begin{entrée}{khɯdi}{}{ⓔkhɯdi} 
\classe{n} 
\begin{définition}\pfra{une plante}\end{définition}
\begin{définition}\pcmn{植物的一种}\end{définition}
\begin{exemple}\pjya{khɯdi nɯ rɯŋgu kɯ-mbro tu-kɯ-ɬoʁ sɯjno kɯ-xtɕi ci ŋu, ɯ-jwaʁ nɯ kɯ-zri tɕe kɯ-ɤmtɕoʁ ci ŋu, ɯ-jwaʁ ɯ-qhu nɯ wɣrum, ɯ-jwaʁ mpɕu, ɯ-ru nɯ jima ɯ-ru ʑo fse, ɣɯrni, tɯ-rtsɤɣ tɯ-rtsɤɣ lu-ɬoʁ ŋu. ɯ-mɯntoʁ nɯ ɯ-ru ɯ-kɤχcɤl lu-ɬoʁ ɲɯ-lɤt ŋu. ɯ-mɯntoʁ wɣrum. pɯ-ŋgra tɕe, jima ɯ-mat kɯ-fse ɲɯ-βze ŋu, tɕeri jima kɯ-fse ɯ-rqhu me, kɯ-ɣɯrni ŋu. khɯdi nɯ tɯ-χpɯm ɯ-fsu ɕaŋtaʁ tu-mbro mɤ-cha. ɯ-qa nɯ tɤ-pɤtso ɯ-ŋgo kɯ-phɤn ɲɯ-ŋu khi.}\hspace{5pt}\pcmn{\lien{ⓔkhɯdi}{khɯdi}是长在很高的草地上的小草。叶子长而尖,叶子背面是白色的,叶子光滑。茎像玉米的茎一样,是红色的,是一节一节长出来的。花开在茎的顶端,是白色的。花凋谢后,果实像玉米的一样,但是没有像玉米棒子一样的皮裹着,是红色的。\lien{ⓔkhɯdi}{khɯdi}只能长到人的膝盖那么高。根对孩子的病有疗效。}\end{exemple}\end{entrée}

\begin{entrée}{khɯdo}{}{ⓔkhɯdo} 
\classe{n} 
\begin{définition}\pfra{vieux chien}\end{définition}
\begin{définition}\pcmn{老狗}\end{définition}\relationsémantique{参考}{\lien{ⓔkhɯna}{khɯna}}\relationsémantique{参考}{\lien{ⓔɯ-do}{ɯ-do}}\end{entrée}

\begin{entrée}{khɯɣ}{}{ⓔkhɯɣ} 
\classe{n} 
\begin{définition}\pfra{moule pour les balles de fusils traditionnels}\end{définition}
\begin{définition}\pcmn{(子弹)模型}\end{définition}\end{entrée}

\begin{entrée}{khɯɣɲɟɯ}{}{ⓔkhɯɣɲɟɯ} 
\classe{n} 
\begin{définition}\pfra{fenêtre}\end{définition}
\begin{définition}\pcmn{窗户}\end{définition}
\begin{exemple}\pjya{khɯɣɲɟɯ kɤ-pa-t-a}\hspace{5pt}\pcmn{我关了窗子}\end{exemple}
\begin{exemple}\pjya{khɯɣɲɟɯ nɯ ɯ-pɕi ɲɯ-sɤɣ-ru ŋu}\hspace{5pt}\pcmn{窗户是用来看外面的}\end{exemple}\relationsémantique{参考}{\lien{ⓔɯ-ɣɲɟɯ}{ɯ-ɣɲɟɯ}}\relationsémantique{参考}{\lien{ⓔtɤ-khɯ}{tɤ-khɯ}}\end{entrée}

\begin{entrée}{khɯjŋga}{}{ⓔkhɯjŋga} 
\classe{n} 
\begin{définition}\pfra{rhododendron}\end{définition}
\begin{définition}\pcmn{羊角花}\end{définition}
\begin{exemple}\pjya{khɯjŋga nɯ ʁnɯ-tɯphu tu, sɤtɕha kɯ-mbɤr tu-kɯ-ɬoʁ ci tu, tɕe nɯ khɯjŋga rmi, khɯjŋga nɯ si mɤ-mbro, ɯ-ru mɤ-astu, ɯ-rtaʁ dɤn, ɯ-jwaʁ nɯ ɯ-ʁɤri nɯ ldʑaŋnaʁ ŋu, ɯ-qhu nɯ kɯ-ɤɣrɤɣrum tsa ŋu, ɯ-jwaʁ nɯ kɯ-ɤrtɯm ɯ-ŋgɯz kɯ-rɲɟi tsa ŋu. ɯ-mɯntoʁ nɯ kɯ-wɣrum ʁɟa tu, kɯ-wɣrum ɯ-ŋgɯz kɯ-ɤɣɯrnɯɕɯr tu, ɯ-mɯntoʁ nɯ kɯngɯsqi jamar tɯtɯrca ɲɯ-lɤt ŋu, wuma ʑo mpɕɤr. phaʁzla jamar ɲɯ-rɯmɯntoʁ ŋu. mɤʑɯ tɯ-tɯphu tu tɕe, zgoku wuma ʑo kɯ-ɣɤndʐo tu-ɬoʁ ŋu, ɯ-ru nɯ khɯjŋga cho ɲɯ-naχtɕɯɣ, ɯ-jwaʁ ɯ-ʁɤri nɯ khɯjŋga ɣɯ cho naχtɕɯɣ, tɕeri ɯ-jwaʁ ɯ-qhu chu nɯ kɯ-qarŋe tɯ-ɣndʑɤr kɯ-fse tu, ɲɯ́-wɣ-nɤmɤle tɕe, pjɯ-kɯ-ŋgra ʑo tu. ɯ-mɯntoʁ khɯjŋga ɣɯ sɤznɤ nɤrko ri ndɯβ. nɯnɯ sŋo rmi. ɯ-ru nɯnɯ mɤ-ngɯt ma ndoʁ, nɯ ma mɤ-sna ri kɤ-nɯ-βlɯ pe.}\hspace{5pt}\pcmn{羊角花有两种,一种生长在海拔比较低的地方,叫作\lien{ⓔkhɯjŋga}{khɯjŋga},树干不高,不直,枝桠多,叶子正面是深绿色的,后面带有点白色,叶子是椭圆形的,花有的是纯白的,也有的是粉红色的。花九朵十朵同时开,很美。一般五月份开花。另一种生长在高山比较寒冷的地方,树干和\lien{ⓔkhɯjŋga}{khɯjŋga}的一样,叶子正面也是和\lien{ⓔkhɯjŋga}{khɯjŋga}的一样,但叶子背面有黄色的粉,一碰就会散落。花比和\lien{ⓔkhɯjŋga}{khɯjŋga}的结实,但小一些。这一种叫\lien{ⓔsŋo}{sŋo} 。树干不结实因为很脆,虽然不能用来制造什么东西,但很好烧。}\end{exemple}\end{entrée}

\begin{entrée}{khɯjŋgɯ}{}{ⓔkhɯjŋgɯ} 
\classe{np} 
\begin{définition}\pfra{gamelle (du chien)}\end{définition}
\begin{définition}\pcmn{狗碗}\end{définition}\relationsémantique{参考}{\lien{ⓔɯ-jŋgɯ}{ɯ-jŋgɯ}}\end{entrée}

\begin{entrée}{khɯlu}{}{ⓔkhɯlu} 
\classe{n} 
\begin{définition}\pfra{Euphorbia helioscopia}\end{définition}
\begin{définition}\pcmn{五朵云}\end{définition}
\begin{exemple}\pjya{khɯlu nɯ sɯjno ci ŋu, ɯ-ru kɯ-ɣɯrni ŋu, ɯ-jwaʁ tɯ-tɯ-rdoʁ ma me, ɯ-ru mpɯ, ɯ-kɤχcɤl tɕe ɯ-mɯntoʁ ɲɯ-lɤt ŋu, ɯ-mɯntoʁ dɤn. ɯ-mɯntoʁ tɯ-rdoʁ tɕe, ɯ-mat ʁnɯ-rdoʁ ntsɯ ɲɯ-βze ŋu. pjɯ́-wɣ-qlɯt tɕe, ɯ-lu tu, sɤndɤɣ. ɯ-lu nɯ tɯ-βri nɯ-ɤtɕaʁ tɕe ʑmbɤr ɲɯ-tɕɤt cha.}\hspace{5pt}\pcmn{五朵云是一种草,茎是红色的,叶子只有几片,茎很嫩,顶端开花。花比较多。每一朵花都结两个果实。掰开就有乳汁,有毒性。乳汁粘在皮肤上会生疮。}\end{exemple}\end{entrée}

\begin{entrée}{khɯmŋu}{}{ⓔkhɯmŋu} 
\classe{n} 
\begin{définition}\pfra{bord du bol}\end{définition}
\begin{définition}\pcmn{碗口}\end{définition}
\begin{exemple}\pjya{khɯmŋu mɤ-ɕɤt}\hspace{5pt}\pcmn{(婴儿)还不会用碗口吃东西}\end{exemple}\relationsémantique{参考}{\lien{ⓔkhɯtsa}{khɯtsa}}\relationsémantique{参考}{\lien{ⓔɯ-mŋu}{ɯ-mŋu}}\end{entrée}

\begin{entrée}{khɯmtsɯ}{}{ⓔkhɯmtsɯ} 
\classe{n} 
\begin{définition}\pfra{viande du thorax du cochon}\end{définition}
\begin{définition}\pcmn{猪胸股上的肉}\end{définition}\end{entrée}

\begin{entrée}{khɯna}{}{ⓔkhɯna} 
\classe{n} 
\begin{définition}\pfra{chien}\end{définition}
\begin{définition}\pcmn{狗}\end{définition}\end{entrée}

\begin{entrée}{khɯnajme}{}{ⓔkhɯnajme} 
\classe{n} 
\begin{définition}\pfra{setaria viridis}\end{définition}
\begin{définition}\pcmn{狗尾巴草}\end{définition}
\begin{exemple}\pjya{khɯnajme nɯ sɯjno mɤ-mbro, ɯ-jwaʁ nɯ xsɤrɯ ɯ-jwaʁ fse ri xtɯt, tɯ-ji ɯ-rkɯ aʁɤndɯndɤt tu-ɬoʁ ɕti. ɯ-mat kɯɕnom fse, ɯ-rme ɯ-tshɯɣa nɯ ra xsɤrɯ ɯ-mat fse, xsɤrɯ ɣɯ ɯ-mat pjɯ-ŋgɤɣ ŋu, khɯna-jme ɯ-mat nɯ tu-ndzur kɯ-fse ɕti ma pjɯ-ŋgɤɣ mɤ-cha, pakuku tu-ɬoʁ cha. ɯ-kɯɕnom khɯnajme fse.}\hspace{5pt}\pcmn{狗尾巴草是一种小草,叶子像\lien{ⓔxsɤrɯ}{xsɤrɯ}的叶子,但短一些。在地边随处生长。果实像麦穗,穗上的毛像\lien{ⓔxsɤrɯ}{xsɤrɯ}的果实一样。\lien{ⓔxsɤrɯ}{xsɤrɯ}的果实往下弯,而\lien{ⓔkhɯnajme}{khɯnajme}的果实是立着的,不弯。年年生长。穗像狗的尾巴一样。}\end{exemple}\end{entrée}

\begin{entrée}{khɯnalu}{}{ⓔkhɯnalu} 
\classe{n} 
\begin{définition}\pfra{année du chien}\end{définition}
\begin{définition}\pcmn{狗年}\end{définition}\relationsémantique{参考}{\lien{ⓔkhɯna}{khɯna}}\end{entrée}

\begin{entrée}{khɯndʐi}{}{ⓔkhɯndʐi} 
\classe{n} 
\begin{définition}\pfra{peau de chien}\end{définition}
\begin{définition}\pcmn{狗皮子}\end{définition}\relationsémantique{参考}{\lien{ⓔkhɯna}{khɯna}}\relationsémantique{参考}{\lien{ⓔtɯ-ndʐi}{tɯ-ndʐi}}\end{entrée}

\begin{entrée}{khɯrɕaŋ}{}{ⓔkhɯrɕaŋ} 
\classe{n} 
\begin{définition}\pfra{armature en bois pour porter des charges sur le dos}\end{définition}
\begin{définition}\pcmn{背架}\end{définition}\étymologie{kʰur.ɕiŋ}\end{entrée}

\begin{entrée}{khɯrndɯɣ}{}{ⓔkhɯrndɯɣ} 
\classe{n} 
\begin{définition}\pfra{sanglier solitaire}\end{définition}
\begin{définition}\pcmn{野猪}\end{définition}\end{entrée}

\begin{entrée}{khɯrtshɤz}{}{ⓔkhɯrtshɤz} 
\classe{n} 
\begin{définition}\pfra{espèce de plante}\end{définition}
\begin{définition}\pcmn{一种草}\end{définition}
\begin{exemple}\pjya{khɯrtshɤz nɯ sɯjno ci ŋu, tu-mbro mɤ-cha, ɯ-ru kɯ-ɣɯrni ŋu, ɯ-jwaʁ kɯ-ɤrŋi ŋu, ɯ-mɯntoʁ kɯ-ɣɯrni ŋu, tɤ-rɤku rca kɤ-ɬoʁ rga, paʁ kɤ-mbi sna}\hspace{5pt}\pcmn{\lien{ⓔkhɯrtshɤz}{khɯrtshɤz}是一种植物,长得不高,茎红色,叶子绿色,花红色,一般生长在庄稼地里,可以喂猪。}\end{exemple}\relationsémantique{参考}{\lien{ⓔtɯ-rtshɤz}{tɯ-rtshɤz}}\end{entrée}

\begin{entrée}{khɯrwum}{}{ⓔkhɯrwum} 
\classe{n} 
\begin{définition}\pfra{moisissure}\end{définition}
\begin{définition}\pcmn{霉}\end{définition}
\begin{exemple}\pjya{khɯrwum ɲo-βzu}\hspace{5pt}\pcmn{生了霉}\end{exemple}\relationsémantique{参考}{\lien{ⓔnɯkhɯrwum}{nɯkhɯrwum}}\end{entrée}

\begin{entrée}{khɯtsa}{}{ⓔkhɯtsa} 
\classe{n} 
\begin{définition}\pfra{bol}\end{définition}
\begin{définition}\pcmn{碗}\end{définition}
\begin{exemple}\pjya{tɕi-khɯtsa ɲɯ-χtɕi ɕti wo}\hspace{5pt}\pcmn{他在洗(我们俩的)碗}\end{exemple}\relationsémantique{参考}{\lien{ⓔkhɯmŋu}{khɯmŋu}}\relationsémantique{参考}{\lien{ⓔarɯkhɯtsa}{arɯkhɯtsa}}\relationsémantique{参考}{\lien{ⓔsomo khɯtsa}{somo khɯtsa}}
\begin{sous-entrée}{tɯ-khɯtsa}{ⓔkhɯtsaⓝtɯ-khɯtsa} 
\classe{clf} 
\begin{définition}\pfra{un bol}\end{définition}
\begin{définition}\pcmn{一碗}\end{définition}\end{sous-entrée}

\end{entrée}

\begin{entrée}{khɯtshoʁ}{}{ⓔkhɯtshoʁ} 
\classe{n} 
\begin{définition}\pfra{chasse avec des chiens}\end{définition}
\begin{définition}\pcmn{狩猎(牵着狗)}\end{définition}
\begin{exemple}\pjya{aʑo khɯtshoʁ rga}\hspace{5pt}\pcmn{我喜欢打猎}\end{exemple}\relationsémantique{参考}{\lien{ⓔɣɯkhɯtshoʁ}{ɣɯkhɯtshoʁ}}\end{entrée}

\begin{entrée}{khɯwɯsi}{}{ⓔkhɯwɯsi} 
\classe{n} 
\begin{définition}\pfra{tapis (coloré )}\end{définition}
\begin{définition}\pcmn{(彩色的)地毯}\end{définition}\end{entrée}

\begin{entrée}{khɯzɤpɯ}{}{ⓔkhɯzɤpɯ} 
\classe{n} 
\begin{définition}\pfra{petit chien}\end{définition}
\begin{définition}\pcmn{小狗}\end{définition}\end{entrée}

\begin{entrée}{khɯzgɯr}{}{ⓔkhɯzgɯr} 
\classe{n} 
\begin{définition}\pfra{serrure}\end{définition}
\begin{définition}\pcmn{锁}\end{définition}\end{entrée}

\begin{entrée}{khɯzi}{}{ⓔkhɯzi} 
\classe{n} 
\begin{définition}\pfra{articulation du fléau}\end{définition}
\begin{définition}\pcmn{连枷的接头}\end{définition}\end{entrée}

\begin{entrée}{ki}{}{ⓔki} 
\classe{dem} 
\begin{définition}\pfra{ceci}\end{définition}
\begin{définition}\pcmn{这个}\end{définition}\end{entrée}

\begin{entrée}{kio}{}{ⓔkio} 
\classe{vt} \sens{1}\paradigme{dir}{pɯ-}
\begin{définition}\pfra{faire glisser}\end{définition}
\begin{définition}\pcmn{使滑下来}\end{définition}
\begin{exemple}\pjya{jiɕqha ɕoŋtɕa nɯ pa-kio}\hspace{5pt}\pcmn{他把木料滑下来了}\end{exemple}
\begin{exemple}\pjya{ɕoŋtɕa pɯ-kio-t-a}\hspace{5pt}\pcmn{我把木料滑下来了}\end{exemple}
\begin{exemple}\pjya{ɯ-thoʁ ɲɯ-sɤŋgio tɕe tɕoχtsi ɲɯ́-wɣ-kio ɲɯ-khɯ}\hspace{5pt}\pcmn{地面很滑,可以把桌子推过去}\end{exemple}\sens{2}\paradigme{dir}{\_}
\begin{définition}\pfra{pousser vers un côté (par la foule)}\end{définition}
\begin{définition}\pcmn{挤过去(因为人多,很拥挤)}\end{définition}\end{entrée}

\begin{entrée}{klaŋklaŋ}{}{ⓔklaŋklaŋ} 
\classe{idph.2} 
\begin{définition}\pfra{complètement emmitouflé}\end{définition}
\begin{définition}\pcmn{形容裹得又紧又大的样子}\end{définition}
\begin{exemple}\pjya{klaŋklaŋ ʑo to-ʑɣɤmphɯr}\hspace{5pt}\pcmn{他把自己的全身裹起来了}\end{exemple}\end{entrée}

\begin{entrée}{klɯɣklɯɣ}{}{ⓔklɯɣklɯɣ} 
\classe{idph.2} 
\begin{définition}\pfra{dur et rond}\end{définition}
\begin{définition}\pcmn{形容圆而硬的样子}\end{définition}\relationsémantique{同义词}{\lien{ⓔtslɯɣtslɯɣ}{tslɯɣtslɯɣ}}\end{entrée}

\begin{entrée}{klɯnklɯn}{}{ⓔklɯnklɯn} 
\classe{idph.2} 
\begin{définition}\pfra{très serré}\end{définition}
\begin{définition}\pcmn{形容紧紧包裹的样子}\end{définition}
\begin{exemple}\pjya{tɯrtɯthɯ nɯ klɯnklɯn ʑo chɤ-mphɯr}\hspace{5pt}\pcmn{他把麻布裹得很紧}\end{exemple}\end{entrée}

\begin{entrée}{kumbrɤl}{}{ⓔkumbrɤl} 
\classe{n} 
\begin{définition}\pfra{jeu d'échec}\end{définition}
\begin{définition}\pcmn{棋}\end{définition}\relationsémantique{参考}{\lien{ⓔnɯkumbrɤl}{nɯkumbrɤl}}\étymologie{ⁿbrel}\end{entrée}

\begin{entrée}{kumkɕi/\variante{kɯmxɕi}}{}{ⓔkumkɕi} 
\classe{n} 
\begin{définition}\pfra{chien de garde}\end{définition}
\begin{définition}\pcmn{看门狗}\end{définition}\étymologie{kʰʲi}\end{entrée}

\begin{entrée}{kumpɣa}{}{ⓔkumpɣa} 
\classe{n} 
\begin{définition}\pfra{poulet}\end{définition}
\begin{définition}\pcmn{鸡}\end{définition}\relationsémantique{参考}{\lien{ⓔkɯm}{kɯm}}\relationsémantique{参考}{\lien{ⓔpɣa}{pɣa}}\end{entrée}

\begin{entrée}{kumpɣalu}{}{ⓔkumpɣalu} 
\classe{n} 
\begin{définition}\pfra{année du coq}\end{définition}
\begin{définition}\pcmn{鸡年}\end{définition}\relationsémantique{参考}{\lien{ⓔkumpɣa}{kumpɣa}}\end{entrée}

\begin{entrée}{kumpɣamu}{}{ⓔkumpɣamu} 
\classe{n} 
\begin{définition}\pfra{poule}\end{définition}
\begin{définition}\pcmn{母鸡}\end{définition}\relationsémantique{参考}{\lien{ⓔkumpɣa}{kumpɣa}}\end{entrée}

\begin{entrée}{kumpɣaphu}{}{ⓔkumpɣaphu} 
\classe{n} 
\begin{définition}\pfra{coq}\end{définition}
\begin{définition}\pcmn{公鸡}\end{définition}\relationsémantique{参考}{\lien{ⓔkumpɣa}{kumpɣa}}\end{entrée}

\begin{entrée}{kumpɣapɯ}{}{ⓔkumpɣapɯ} 
\classe{n} 
\begin{définition}\pfra{poussin}\end{définition}
\begin{définition}\pcmn{小鸡}\end{définition}\relationsémantique{参考}{\lien{ⓔkumpɣa}{kumpɣa}}\end{entrée}

\begin{entrée}{kumpɣasta}{}{ⓔkumpɣasta} 
\classe{n} 
\begin{définition}\pfra{poulailler}\end{définition}
\begin{définition}\pcmn{鸡圈}\end{définition}\relationsémantique{参考}{\lien{ⓔkumpɣa}{kumpɣa}}\relationsémantique{参考}{\lien{ⓔtɯ-sta}{tɯ-sta}}\end{entrée}

\begin{entrée}{kumpɣɤŋgɯm}{}{ⓔkumpɣɤŋgɯm} 
\classe{n} 
\begin{définition}\pfra{œuf de poule}\end{définition}
\begin{définition}\pcmn{鸡蛋}\end{définition}\relationsémantique{参考}{\lien{ⓔtɤ-ŋgɯm}{tɤ-ŋgɯm}}\end{entrée}

\begin{entrée}{kumpɣɤtɕɯ}{}{ⓔkumpɣɤtɕɯ} 
\classe{n} 
\begin{définition}\pfra{moineau}\end{définition}
\begin{définition}\pcmn{麻雀}\end{définition}\end{entrée}

\begin{entrée}{kundi}{}{ⓔkundi} 
\classe{n} 
\begin{définition}\pfra{de droite à gauche}\end{définition}
\begin{définition}\pcmn{左右}\end{définition}\end{entrée}

\begin{entrée}{ko}{₂}{ⓔkoⓗ2} 
\classe{part} 
\begin{définition}\pfra{assertif}\end{définition}
\begin{définition}\pcmn{表示确定的语气}\end{définition}
\begin{exemple}\pjya{mɤ-tɯ-cha ko}\hspace{5pt}\pcmn{你肯定不行}\end{exemple}
\begin{exemple}\pjya{tɯ-maqhu ko}\hspace{5pt}\pcmn{你肯定迟到}\end{exemple}\end{entrée}

\begin{entrée}{ko}{₁}{ⓔkoⓗ1} 
\classe{vt}  
\grammaire{caus} \paradigme{dir}{pɯ-}\paradigme{dir}{pɯ-}
\begin{définition}\pfra{vaincre}\end{définition}
\begin{définition}\pcmn{打赢;打败}\end{définition}
\begin{définition}\pfra{causer (involontairement) le malheur}\end{définition}
\begin{définition}\pcmn{让……遭殃(不一定是故意的)}\end{définition}
\begin{exemple}\pjya{pɯ-ko-t-a}\hspace{5pt}\pcmn{我打败了他}\end{exemple}
\begin{exemple}\pjya{tɤ-aʑɯʑu-tɕi tɕe, a-χti pɯ-ko-t-a}\hspace{5pt}\pcmn{在角力时候,我把对方打败了}\end{exemple}\relationsémantique{同义词}{\lien{ⓔɕɯnŋo}{ɕɯnŋo}}
\begin{sous-entrée}{sɯko}{ⓔkoⓗ1ⓝsɯko} 
\classe{vt} \end{sous-entrée}

\begin{exemple}\pjya{pɯ-ta-sɯko}\hspace{5pt}\pcmn{你因为我受了很多苦}\end{exemple}\end{entrée}

\begin{entrée}{kodɤt}{}{ⓔkodɤt} 
\classe{n} 
\begin{définition}\pfra{cave}\end{définition}
\begin{définition}\pcmn{土窖}\end{définition}\end{entrée}

\begin{entrée}{kolɤβ}{}{ⓔkolɤβ} 
\classe{n} 
\begin{définition}\pfra{habit sans manche de moine}\end{définition}
\begin{définition}\pcmn{披衫(没有袖子的袈裟)}\end{définition}\end{entrée}

\begin{entrée}{komɤl}{}{ⓔkomɤl} 
\classe{n} 
\begin{définition}\pfra{poutre}\end{définition}
\begin{définition}\pcmn{横梁}\end{définition}\end{entrée}

\begin{entrée}{komɤr}{}{ⓔkomɤr} 
\classe{n} 
\begin{définition}\pfra{cuir teint en rouge}\end{définition}
\begin{définition}\pcmn{染成红色的皮子}\end{définition}\étymologie{ko.ba.dmar}\end{entrée}

\begin{entrée}{konaʁ}{}{ⓔkonaʁ} 
\classe{n} 
\begin{définition}\pfra{cuir teint en noir}\end{définition}
\begin{définition}\pcmn{染成黑色的皮子}\end{définition}\étymologie{ko.ba.nag}\end{entrée}

\begin{entrée}{kontsɤɣdɯ}{}{ⓔkontsɤɣdɯ} 
\classe{n} 
\begin{définition}\pfra{récipient en cuivre ou en fer}\end{définition}
\begin{définition}\pcmn{红铜;生铁铸成的罐子,有盖子}\end{définition}\end{entrée}

\begin{entrée}{kontsɤrloŋ}{}{ⓔkontsɤrloŋ} 
\classe{n} 
\begin{définition}\pfra{récipient en cuivre ou en fer}\end{définition}
\begin{définition}\pcmn{红铜、生铁铸成的罐子,没有盖子}\end{définition}\end{entrée}

\begin{entrée}{kontsi}{}{ⓔkontsi} 
\classe{n} 
\begin{définition}\pfra{récipient}\end{définition}
\begin{définition}\pcmn{罐子}\end{définition}\étymologie{fn:罐子}\end{entrée}

\begin{entrée}{koŋi}{}{ⓔkoŋi} 
\classe{n} 
\begin{définition}\pfra{joug pour un animal}\end{définition}
\begin{définition}\pcmn{牛轭(单行)}\end{définition}
\begin{exemple}\pjya{jla ɣɯ koŋi}\hspace{5pt}\pcmn{犏牛轭}\end{exemple}\end{entrée}

\begin{entrée}{koŋla/\variante{kuŋula}}{}{ⓔkoŋla} 
\classe{n} \sens{1}
\begin{définition}\pfra{chose vraie}\end{définition}
\begin{définition}\pcmn{真实}\end{définition}
\begin{exemple}\pjya{koŋla tɤ-ti}\hspace{5pt}\pcmn{你要说真话(不要开玩笑)}\end{exemple}\sens{2}
\begin{définition}\pfra{vraiment}\end{définition}
\begin{définition}\pcmn{真地}\end{définition}\end{entrée}

\begin{entrée}{koŋmarɟɤlpu}{}{ⓔkoŋmarɟɤlpu} 
\classe{n} 
\begin{définition}\pfra{empereur}\end{définition}
\begin{définition}\pcmn{皇帝}\end{définition}\étymologie{goŋ.ma rgʲal.po}\end{entrée}

\begin{entrée}{koŋtaʁ}{}{ⓔkoŋtaʁ} 
\classe{n} 
\begin{définition}\pfra{lanière avant de la selle}\end{définition}
\begin{définition}\pcmn{马鞍的前绳}\end{définition}\end{entrée}

\begin{entrée}{kóʁmɯz}{}{ⓔkóʁmɯz} 
\classe{adv} \sens{1}
\begin{définition}\pfra{à l'instant}\end{définition}
\begin{définition}\pcmn{刚才}\end{définition}
\begin{exemple}\pjya{ɬamu kɯ nɯ kóʁmɯz pɯ-kɯ-fse nɯ ra pjɤ-fɕɤt}\hspace{5pt}\pcmn{拉莫给他讲了刚才发生的事}\end{exemple}
\begin{exemple}\pjya{aʑo nɯ kóʁmɯz @xiaban pɯ-βzu-t-a tɕe, kha lɤ-nɯɣe-a}\hspace{5pt}\pcmn{我刚刚下班回家了}\end{exemple}
\begin{exemple}\pjya{nɯ kóʁmɯz nɤ kha jɤ-azɣɯt-a}\hspace{5pt}\pcmn{我刚刚才到家}\end{exemple}\sens{2}
\begin{définition}\pfra{alors seulement}\end{définition}
\begin{définition}\pcmn{这才}\end{définition}\relationsémantique{参考}{\lien{ⓔnóʁmɯz}{nóʁmɯz}}\end{entrée}

\begin{entrée}{kosca}{}{ⓔkosca} 
\classe{n} 
\begin{définition}\pfra{cuir non teint}\end{définition}
\begin{définition}\pcmn{没有染色的皮子}\end{définition}\étymologie{ko.ba.skʲa}\end{entrée}

\begin{entrée}{kota}{}{ⓔkota} 
\classe{n} 
\begin{définition}\pfra{sac en peau que l'on porte à l'épaule}\end{définition}
\begin{définition}\pcmn{毛皮子缝成的挎包}\end{définition}\end{entrée}

\begin{entrée}{kowa}{}{ⓔkowa} 
\classe{n} 
\begin{définition}\pfra{méthode}\end{définition}
\begin{définition}\pcmn{办法}\end{définition}
\begin{exemple}\pjya{ɯʑo ɯ-kowa tu}\hspace{5pt}\pcmn{他有办法}\end{exemple}\relationsémantique{参考}{\lien{ⓔnɯkowa}{nɯkowa}}\étymologie{bkol.ba}\end{entrée}

\begin{entrée}{koxtɕɯn}{}{ⓔkoxtɕɯn} 
\classe{n} 
\begin{définition}\pfra{soie}\end{définition}
\begin{définition}\pcmn{丝绸}\end{définition}
\begin{exemple}\pjya{koxtɕɯn kɯ-qarŋe ci ɲɯ-ŋu}\end{exemple}\étymologie{gos.tɕʰen}\end{entrée}

\begin{entrée}{koxtɕɯnri}{}{ⓔkoxtɕɯnri} 
\classe{n} 
\begin{définition}\pfra{soie}\end{définition}
\begin{définition}\pcmn{丝绸}\end{définition}\étymologie{gos.tɕʰen.ras}\end{entrée}

\begin{entrée}{koʑi}{}{ⓔkoʑi} 
\classe{n} 
\begin{définition}\pfra{poutre}\end{définition}
\begin{définition}\pcmn{横梁}\end{définition}
\begin{exemple}\pjya{koʑi komɤl nɯ jɤɣɤt laχtsɯ ɯ-taʁ tɤpjaʁ nɯ ŋu, nɯ jɤɣɤt ɯ-taʁ tɯ-sthoʁsi ɯ-kɯ-sthoʁ nɯ ŋu, rɟɯɣ sɤznɤ xtshɯm}\hspace{5pt}\pcmn{\lien{}{koʑi komɤl}是走缘的柱头上的木方条,支撑走缘上面的小梁,比\lien{ⓔrɟɯɣⓗ2}{rɟɯɣ}细一些}\end{exemple}\end{entrée}

\begin{entrée}{kupa}{}{ⓔkupa} 
\classe{n} 
\begin{définition}\pfra{chinois}\end{définition}
\begin{définition}\pcmn{汉人}\end{définition}
\begin{exemple}\pjya{kupa skɤt kɤ-βzu a-pɯ-nɯ-me ɲɯ-sɯsam-a}\hspace{5pt}\pcmn{我不想说汉语}\end{exemple}\end{entrée}

\begin{entrée}{kupajmɤɣ}{}{ⓔkupajmɤɣ} 
\classe{n} 
\begin{définition}\pfra{matsutake}\end{définition}
\begin{définition}\pcmn{松茸}\end{définition}\relationsémantique{参考}{\lien{ⓔtɤjmɤɣ}{tɤjmɤɣ}}\end{entrée}

\begin{entrée}{kupaŋga}{}{ⓔkupaŋga} 
\classe{n} 
\begin{définition}\pfra{habits chinois / occidentaux}\end{définition}
\begin{définition}\pcmn{汉装}\end{définition}\relationsémantique{参考}{\lien{ⓔtɯ-ŋga}{tɯ-ŋga}}\relationsémantique{参考}{\lien{ⓔkupa}{kupa}}\end{entrée}

\begin{entrée}{kuparmbatɕɯβ}{}{ⓔkuparmbatɕɯβ} 
\classe{n} 
\begin{définition}\pfra{espèce de plante}\end{définition}
\begin{définition}\pcmn{【冬寒菜】}\end{définition}\end{entrée}

\begin{entrée}{kupastaχpɯ}{}{ⓔkupastaχpɯ} 
\classe{n} 
\begin{définition}\pfra{soja}\end{définition}
\begin{définition}\pcmn{黄豆}\end{définition}\relationsémantique{参考}{\lien{ⓔstaχpɯ}{staχpɯ}}\end{entrée}

\begin{entrée}{kupastoʁ}{}{ⓔkupastoʁ} 
\classe{n} 
\begin{définition}\pfra{pois cultivable toute l'année}\end{définition}
\begin{définition}\pcmn{四季豆}\end{définition}\relationsémantique{参考}{\lien{ⓔstoʁ}{stoʁ}}\end{entrée}

\begin{entrée}{kra}{}{ⓔkra} 
\classe{vt} \paradigme{dir}{pɯ-}
\begin{définition}\pfra{faire tomber}\end{définition}
\begin{définition}\pcmn{打落}\end{définition}
\begin{exemple}\pjya{jɯfɕo jiʑo ji-ʑɴɢɯloʁ ɕ-pɯ-kra-t-a}\hspace{5pt}\pcmn{今天早上我去把我们的核桃打下来了}\end{exemple}
\begin{exemple}\pjya{ʑɴɢɯloʁ ɕ-pɯ-kre}\hspace{5pt}\pcmn{你去把核桃打下来}\end{exemple}\relationsémantique{参考}{\lien{ⓔŋgra}{ŋgra}}\end{entrée}

\begin{entrée}{krɤɣ}{}{ⓔkrɤɣ} 
\classe{vt} \paradigme{dir}{kɤ-}\paradigme{dir}{nɯ-}\paradigme{dir}{thɯ-}
\begin{définition}\pfra{couper}\end{définition}
\begin{définition}\pcmn{割}\end{définition}
\begin{exemple}\pjya{aʑo nɯ-kraɣ-a}\hspace{5pt}\pcmn{我割了(草)}\end{exemple}
\begin{exemple}\pjya{sɯjno na-krɤɣ}\hspace{5pt}\pcmn{他割了草}\end{exemple}
\begin{exemple}\pjya{paʁndza ka-krɤɣ}\hspace{5pt}\pcmn{他割了猪草}\end{exemple}
\begin{exemple}\pjya{qaʑo thɯ-krɤɣ}\hspace{5pt}\pcmn{你给绵羊剃毛}\end{exemple}
\begin{exemple}\pjya{a-ndzrɯ ɲɯ-ɣɤzri tɕe ɲɯ-kraɣ-a ntsɯ ɲɯ-ra}\hspace{5pt}\pcmn{我的指甲长得很快,我必须经常剪}\end{exemple}
\begin{sous-entrée}{sɯkrɤɣ}{ⓔkrɤɣⓝsɯkrɤɣ} 
\classe{vt} 
\begin{définition}\pfra{couper avec}\end{définition}
\begin{définition}\pcmn{用……割}\end{définition}
\begin{exemple}\pjya{tɯɲcɣa kɯ kú-wɣ-sɯkrɤɣ ra}\hspace{5pt}\pcmn{要用镰刀割}\end{exemple}\end{sous-entrée}

\begin{sous-entrée}{nɯɣɯkrɤɣ}{ⓔkrɤɣⓝnɯɣɯkrɤɣ} 
\classe{vs} 
\begin{définition}\pfra{facile à tondre}\end{définition}
\begin{définition}\pcmn{容易剪毛(绵羊)}\end{définition}
\begin{exemple}\pjya{ki qaʑo ki ɲɯ-nɯɣɯkrɤɣ}\hspace{5pt}\pcmn{这只羊容易剪毛}\end{exemple}\end{sous-entrée}

\end{entrée}

\begin{entrée}{krɤlma}{}{ⓔkrɤlma} 
\classe{n} 
\begin{définition}\pfra{colique}\end{définition}
\begin{définition}\pcmn{婴儿的肠病}\end{définition}
\begin{exemple}\pjya{krɤlma nɤrŋi ɯ-ŋgo ŋu, ɲɯ-nɯtɯfɕɤl tɕe ɯ-qe kɯ-qarŋe ŋu, karɣi ɯ-mɯntoʁ fse}\hspace{5pt}\pcmn{\lien{ⓔkrɤlma}{krɤlma}是婴儿的病,他拉肚子时大便是黄色的,像是菜子花。}\end{exemple}\relationsémantique{参考}{\lien{ⓔnɯkrɤlma}{nɯkrɤlma}}\étymologie{grol.ma}\end{entrée}

\begin{entrée}{krul}{}{ⓔkrul} 
\classe{vi} \paradigme{dir}{pɯ-}
\begin{définition}\pfra{finir (cérémonie religieuse)}\end{définition}
\begin{définition}\pcmn{做完(宗教仪式)}\end{définition}
\begin{exemple}\pjya{sɲaŋne kɤ-ndo pɯ-krul-a}\hspace{5pt}\pcmn{我把哑巴经念完了}\end{exemple}
\begin{exemple}\pjya{χpɯn ra kɤ-ɣɤrpi pjɤ-krul-nɯ}\hspace{5pt}\pcmn{和尚们念完经了}\end{exemple}\étymologie{grol}\end{entrée}

\begin{entrée}{kro}{}{ⓔkro} 
\classe{vt} \paradigme{dir}{pɯ-}\paradigme{dir}{thɯ-}\paradigme{dir}{pɯ-}\paradigme{dir}{pɯ-}\paradigme{dir}{pɯ-}\paradigme{dir}{nɯ-}
\begin{définition}\pfra{partager, distribuer}\end{définition}
\begin{définition}\pcmn{分东西}\end{définition}
\begin{définition}\pfra{se partager}\end{définition}
\begin{définition}\pcmn{自己分东西;彼此分东西}\end{définition}
\begin{définition}\pfra{partager avec}\end{définition}
\begin{définition}\pcmn{分给}\end{définition}
\begin{définition}\pfra{se séparer}\end{définition}
\begin{définition}\pcmn{分开(一个群体)}\end{définition}
\begin{définition}\pfra{partager avec tout le monde}\end{définition}
\begin{définition}\pcmn{分来分去}\end{définition}
\begin{exemple}\pjya{pɯ-kro-t-a}\hspace{5pt}\pcmn{我分了}\end{exemple}
\begin{exemple}\pjya{kɯki laχtɕha ki pɯ-krɤm}\hspace{5pt}\pcmn{你把这个东西分(给大家)}\end{exemple}
\begin{exemple}\pjya{tʂha pɯ-nɯkro-tɕi}\hspace{5pt}\pcmn{我们俩分了茶}\end{exemple}
\begin{exemple}\pjya{laχtɕha pɯ-nɯkro-tɕi}\hspace{5pt}\pcmn{我们俩分了东西}\end{exemple}
\begin{exemple}\pjya{kɤ-ndza pɯ-nɯkro-tɕi}\hspace{5pt}\pcmn{我们俩分了东西吃}\end{exemple}
\begin{sous-entrée}{nɯkro}{ⓔkroⓝnɯkro} 
\classe{vt}  
\grammaire{autoben} \end{sous-entrée}

\begin{sous-entrée}{znɯkro}{ⓔkroⓝznɯkro} 
\classe{vt} \end{sous-entrée}

\begin{sous-entrée}{ʑɣɤkro}{ⓔkroⓝʑɣɤkro} 
\classe{vi}  
\grammaire{refl} \end{sous-entrée}

\begin{sous-entrée}{rɤkro}{ⓔkroⓝrɤkro} 
\classe{vi}  
\grammaire{apass} 
\begin{définition}\pfra{partager des choses}\end{définition}
\begin{définition}\pcmn{分东西}\end{définition}\end{sous-entrée}

\begin{sous-entrée}{nɤkɯkro}{ⓔkroⓝnɤkɯkro} 
\classe{vt} \end{sous-entrée}

\end{entrée}

\begin{entrée}{kroŋwa}{}{ⓔkroŋwa} 
\classe{n} 
\begin{définition}\pfra{mal de ventre}\end{définition}
\begin{définition}\pcmn{肚子痛}\end{définition}\end{entrée}

\begin{entrée}{kropa}{}{ⓔkropa} 
\classe{n} 
\begin{définition}\pfra{serviteur}\end{définition}
\begin{définition}\pcmn{仆人}\end{définition}
\begin{exemple}\pjya{βlama mɤ-χsɤl, kropa χsɤl}\hspace{5pt}\pcmn{喇嘛糊涂,仆人清楚}\end{exemple}\end{entrée}

\begin{entrée}{krɯβthoβ}{}{ⓔkrɯβthoβ} 
\classe{n} 
\begin{définition}\pfra{sprulsku qui peut se marier}\end{définition}
\begin{définition}\pcmn{可以娶妻的活佛}\end{définition}\étymologie{grub.thob}\end{entrée}

\begin{entrée}{kɯ}{₁}{ⓔkɯⓗ1} 
\classe{postp} \sens{1}
\begin{définition}\pfra{ergatif}\end{définition}
\begin{définition}\pcmn{施事格}\end{définition}\sens{2}
\begin{définition}\pfra{instrumental}\end{définition}
\begin{définition}\pcmn{工具格}\end{définition}
\begin{exemple}\pjya{mbrɯtɕɯ kɯ ʑɴɢɯloʁ nɯ-sɯphaʁ-a}\hspace{5pt}\pcmn{我用刀子把核桃撬开了}\end{exemple}\sens{3}
\begin{définition}\pfra{conjonction}\end{définition}
\begin{définition}\pcmn{连词}\end{définition}\sens{4}
\begin{définition}\pfra{marque du comparé dans la construction comparative}\end{définition}
\begin{définition}\pcmn{差比句中表示比较主体}\end{définition}
\begin{exemple}\pjya{ɯ-ʁi sɤz ɯ-pi nɯ kɯ mpɕɤr}\hspace{5pt}\pcmn{姐姐比妹妹漂亮}\end{exemple}\end{entrée}

\begin{entrée}{kɯ}{₂}{ⓔkɯⓗ2} 
\classe{part} 
\begin{définition}\pfra{marque de question}\end{définition}
\begin{définition}\pcmn{表示疑问}\end{définition}
\begin{exemple}\pjya{jiɕqha nɯ tɕhi pɯ-rmi kɯ}\hspace{5pt}\pcmn{刚才那个人叫什么名字呢?}\end{exemple}\end{entrée}

\begin{entrée}{kɯβdɤsqi}{}{ⓔkɯβdɤsqi} 
\classe{num} 
\begin{définition}\pfra{quarante}\end{définition}
\begin{définition}\pcmn{四十}\end{définition}\end{entrée}

\begin{entrée}{kɯβde}{}{ⓔkɯβde} 
\classe{num} 
\begin{définition}\pfra{quatre}\end{définition}
\begin{définition}\pcmn{四}\end{définition}\end{entrée}

\begin{entrée}{kɯβʁa}{}{ⓔkɯβʁa} 
\classe{n} 
\begin{définition}\pfra{noble}\end{définition}
\begin{définition}\pcmn{贵族}\end{définition}\end{entrée}

\begin{entrée}{kɯchu}{}{ⓔkɯchu} 
\classe{adv} 
\begin{définition}\pfra{à l'est}\end{définition}
\begin{définition}\pcmn{在东边}\end{définition}
\begin{exemple}\pjya{kɯchu ɯ-rkɯ ri ku-rɤʑi-a}\hspace{5pt}\pcmn{我在东边}\end{exemple}\relationsémantique{参考}{\lien{ⓔɯ-kɤcu}{ɯ-kɤcu}}\end{entrée}

\begin{entrée}{kɯchi}{}{ⓔkɯchi} 
\classe{n} 
\begin{définition}\pfra{sucre}\end{définition}
\begin{définition}\pcmn{糖}\end{définition}\end{entrée}

\begin{entrée}{kɯɕmar}{}{ⓔkɯɕmar} 
\classe{n} 
\begin{définition}\pfra{céréales}\end{définition}
\begin{définition}\pcmn{麦类}\end{définition}\end{entrée}

\begin{entrée}{kɯɕmɤtʂu}{}{ⓔkɯɕmɤtʂu} 
\classe{n} 
\begin{définition}\pfra{allumette}\end{définition}
\begin{définition}\pcmn{火柴}\end{définition}\end{entrée}

\begin{entrée}{kɯɕnɤsqi}{}{ⓔkɯɕnɤsqi} 
\classe{num} 
\begin{définition}\pfra{soixante-dix}\end{définition}
\begin{définition}\pcmn{七十}\end{définition}\end{entrée}

\begin{entrée}{kɯɕnom}{}{ⓔkɯɕnom} 
\classe{n} 
\begin{définition}\pfra{épi}\end{définition}
\begin{définition}\pcmn{穗子}\end{définition}\relationsémantique{参考}{\lien{ⓔrɯkɯɕnom}{rɯkɯɕnom}}\end{entrée}

\begin{entrée}{kɯɕnɯz}{}{ⓔkɯɕnɯz} 
\classe{num} 
\begin{définition}\pfra{sept}\end{définition}
\begin{définition}\pcmn{七}\end{définition}\end{entrée}

\begin{entrée}{kɯɕpaz}{}{ⓔkɯɕpaz} 
\classe{n} 
\begin{définition}\pfra{marmotte}\end{définition}
\begin{définition}\pcmn{旱獭}\end{définition}
\begin{exemple}\pjya{kɯɕpaz nɯ rɯŋgu tsa ku-rɤʑi ŋu, ɯ-ɣɲɟɯ ɯ-ŋgɯ ku-rɤʑi ɲɯ-ŋu, kɯ-wxti kɯ ɣnɤsqi tɯ-rpa jamar tu, ʁzɯɣ kɯ-ɤɣɯrnɯɕɯr tu, ɯ-scɯʁzɯɣ βɣɯz cho naχtɕɯɣ tsa, qartsɯ kɤ-tsu tɕe, ɯ-ɣɲɟɯ ɯ-ŋgɯ lu-cɯ ɲɯ-ŋgrɤl, wuma ʑo tshu ɲɯ-ŋgrɤl, ftɕar tɕe chɯ-nɯɬoʁ, cɯ tɤkha tɕe pɕaʁ tu-βze tɕe ``ɣɯjpa qartsɯ ɕawu rambɯm a-taʁ a-mɤ-ɣɯ-kɤ-rŋgɯ smɯlɤm" tu-ti ɲɯ-ŋgrɤl, ma ɯʑo qartsɯ tɕe wuma ɲɯ-tshu, ɕawu rambɯm wuma ʑo kɯ-wxti tɕe kɯ-mpja ɲɯ-ŋu, tɕe kɯɕpaz ɯ-taʁ ɯ-stu nɯ tɕu kɤ-rŋgɯ tɕe, ɯ-tɯ-mpja kɯ kɯɕpaz nɯ pjɯ́-wɣ-ftʂi ɲɯ-ŋgrɤl, tɕe núndʐa kɯɕpaz khɤfɕɤt tu-βze ɲɯ-ŋgrɤl. kɯ-ɣɤrʁaʁ ra kɯ ɯ-ɣɲɟɯ ɯ-kɯm zɯ tɤ-rcoʁ rɯlɯ thɤstɯɣ kɯ-tu tu-rtsi-nɯ tɕe, ɯ-ŋgɯ kɯɕpaz thɤstɯɣ tu nɯ pjɯ-sɯχsɤl-nɯ ɲɯ-ŋgrɤl. mbroχpa kɯ fsapaʁ ɯ-βraʁ ŋu tu-ti-nɯ ɲɯ-ŋgrɤl, tɕe tú-wɣ-nɤrʁaʁ qha-nɯ, ɯ-sɤ-dɤn nɯ mbroχpa sɤtɕha ŋu.}\hspace{5pt}\pcmn{旱獭生活在高山的洞穴里。大的有二十来斤,是红色的,外貌有点像獾。冬天长肥了的时候,就会在洞穴冬眠。据说要冬眠时,它拱着手祈祷:“今年冬天不希望有独角公鹿来睡在我的上面”,因为它在冬天很肥,独角公鹿很大,热量高,会让旱獭融化掉,所以旱獭要祈祷。猎人们数着旱獭洞口贴着的泥球就知道洞穴里住着多少只旱獭。放牧人说它是牲畜的象征,所以讨厌猎捕。旱獭繁殖比较多的地方在牧区。}\end{exemple}\end{entrée}

\begin{entrée}{kɯɕte}{}{ⓔkɯɕte} 
\classe{adv} 
\begin{définition}\pfra{autre}\end{définition}
\begin{définition}\pcmn{另外;其他}\end{définition}
\begin{exemple}\pjya{kɯɕte nɯ-tɯ-ta-t ŋu ɯ-maʁ}\hspace{5pt}\pcmn{你是不是放在另外一个地方}\end{exemple}
\begin{exemple}\pjya{kɯɕte tɯ-ŋga tɤ-ŋge}\end{exemple}
\begin{exemple}\pjya{tɯ-ŋga kɯɕte tɤ-ŋge}\hspace{5pt}\pcmn{你穿另外一件衣服吧}\end{exemple}\end{entrée}

\begin{entrée}{kɯɕɯŋgɯ}{}{ⓔkɯɕɯŋgɯ} 
\classe{n} 
\begin{définition}\pfra{autrefois}\end{définition}
\begin{définition}\pcmn{古时候}\end{définition}\end{entrée}

\begin{entrée}{kɯfɕi}{}{ⓔkɯfɕi} 
\classe{n} 
\begin{définition}\pfra{forgeron}\end{définition}
\begin{définition}\pcmn{铁匠}\end{définition}
\begin{exemple}\pjya{kɯfɕi ɣɯ ɯ-mbrɯtɕɯ kɤ-ntɕhoz me, ɕoŋβzu ɯ-sɤmdzɯ me}\hspace{5pt}\pcmn{铁匠没有刀子用,木匠没有凳子坐}\end{exemple}\end{entrée}

\begin{entrée}{kɯɣe}{}{ⓔkɯɣe} 
\classe{part} 
\begin{définition}\pfra{question à soi-même}\end{définition}
\begin{définition}\pcmn{自我反问}\end{définition}\end{entrée}

\begin{entrée}{kɯjujmɤlu}{}{ⓔkɯjujmɤlu} 
\classe{n} 
\begin{définition}\pfra{animal sans queue (homme)}\end{définition}
\begin{définition}\pcmn{没有尾巴的动物(人)}\end{définition}
\begin{exemple}\pjya{kɯjujmɤlu ɲɯ-sɲu ŋu tɕe, ma-ɕ-thɯ-tɯ-ʑɣɤ-βde ma ji-kɤ-ndza tu-tu ɕti}\hspace{5pt}\pcmn{(乌鸦说:)你不要去投河自尽,当那个没有尾巴的动物(指人)发疯的时候(指播种子)我们就有吃的了}\end{exemple}\end{entrée}

\begin{entrée}{kɯjka}{}{ⓔkɯjka} 
\classe{n} 
\begin{définition}\pfra{corbeau à bec rouge (pyrrhocorax pyrrhocorax)}\end{définition}
\begin{définition}\pcmn{红嘴山鸦【红嘴老鸦】}\end{définition}
\begin{exemple}\pjya{kɯjka nɯ pɣa kɯ-ɲaʁ ci ŋu, qajdo cho ndʑi-tɯ-wxti ndʑi-tshɯɣa ra naχtɕɯɣ, kɯjka nɯ ɯ-βri nɤmbju, ɯ-mtsioʁ cho ɯ-mi nɯ ra kɯ-ɣɯrni ŋu, jɤɣɤt pa znde ɯ-kɯ-spoʁ kɯ-wxti nɯ ra ku-rɤloʁ ŋu, ɯ-loʁ ɯ-spa nɯ si ɯ-rtaʁ kɯ-xtshɯm tsa z-ju-nɯzʁe tɕe ɯ-kɯr ɯ-ŋgɯ ʁɟa ʑo tu-rke tɕe z-ju-nɯ-zʁe ŋu, ʁzɤmi ni tu-oqurle-ndʑi tɕe ku-rɤloʁ-ndʑi ŋu. tu-mbri tɕe, ``ka ca kɤɣ" ntsɯ tu-ti ŋu. tɯ-mɯ cho sɤrwa lɤt tɤkha tɕe wuma ʑo mbri ma sɤrwa ɯ-kɯ-sɯ-lɤt tɤ-rca ɯ-ku a-rku tu-kɯ-ti ɲɯ-ŋu. tɤ-rɤku kɤ-ndza χɕu tɕe pɯ-kɤ-nɯji ɣɯ ɯ-rɣi ra kɯnɤ tu-nɯ-tɕɤt ɕti. srɯnmɯ ŋu tu-kɯ-ti ŋgrɤl.}\hspace{5pt}\pcmn{红嘴老鸦是一种黑色的鸟,形状和大小和乌鸦一样,毛有光泽,嘴和脚是红色的。在走缘下面和墙壁上比较大的洞里作窝,垒窝的材料是比较细的树枝,全部是用嘴衔来的,公鸦和母鸦一起垒窝。叫声是\lien{}{ka ca kɤɣ}。下雨或下冰雹时,叫声特别欢快,据说它会参与商议下冰雹的事情。它吃粮食很厉害,连种下去的种子也会挖出来吃。人家说它是妖精。}\end{exemple}\end{entrée}

\begin{entrée}{kɯjkɤkɕi}{}{ⓔkɯjkɤkɕi} 
\classe{n} 
\begin{définition}\pfra{belette}\end{définition}
\begin{définition}\pcmn{黄喉貂}\end{définition}
\begin{exemple}\pjya{kɯjkɤkɕi nɯ ʁnɯz nɤ ʁnɯz nɯ tɯtɯrca ku-rɤʑi ɲɯ-ŋgrɤl, ca tu-βɟi ŋgrɤl, ca nɯ sɯku tɤ-ari tɕe ɯ-rcɯrca sɯku tu-ɕe tɕe ɕ-pjɯ-ɣɤrɤt ŋgrɤl. pa-sat tɕe, ca ɯ-rme tu-ndze tu-z-mɤke, ɯ-qhu tɕe nɯ kóʁmɯz ɯ-ɕa tu-ndze chɯ-ɕkɯt-nɯ mɤ-cha. ɯ-mdoʁ nɯ kɯ-ɲaʁ kɯ-ɤɣɯrnɯɕɯr ŋu, kɯ-xtshɯm kɯ-zri kɯ-fse ŋu, ɯ-mtɕhi kɯ-ɤmtɕoʁ, ɯ-rna qachɣa ɯ-rna kɯ-fse ɲɯ-ŋu, ɯ-jme kɯ-zɯ-zri ŋu.}\hspace{5pt}\pcmn{黄喉貂一对一对地呆在一起,会追麝香鹿,麝香鹿上树了,它也跟着上树,最后把麝香鹿甩下来,把它弄死。吃麝香鹿的时候,先吃完麝香鹿的毛再吃肉,所以吃不完。颜色是黑里带红的,身子细而长,嘴尖,耳朵像狐狸的耳朵,尾巴很长。}\end{exemple}\end{entrée}

\begin{entrée}{kɯjŋu}{}{ⓔkɯjŋu} 
\classe{n} 
\begin{définition}\pfra{serment}\end{définition}
\begin{définition}\pcmn{誓言}\end{définition}
\begin{exemple}\pjya{kɯjŋu to-joʁ}\hspace{5pt}\pcmn{他发誓了}\end{exemple}
\begin{exemple}\pjya{kɯjŋu pjɤ-ta}\hspace{5pt}\pcmn{他发誓了}\end{exemple}\relationsémantique{参考}{\lien{ⓔnɯkɯjŋu}{nɯkɯjŋu}}\end{entrée}

\begin{entrée}{kɯjra}{}{ⓔkɯjra} 
\classe{n} 
\begin{définition}\pfra{la plus longue corde d'une tente}\end{définition}
\begin{définition}\pcmn{帐篷最长的拉线}\end{définition}\end{entrée}

\begin{entrée}{kɯjtɯm}{}{ⓔkɯjtɯm} 
\classe{n}  
\grammaire{n.lieu} 
\begin{définition}\pfra{l'un des hameaux de Gyutshapa}\end{définition}
\begin{définition}\pcmn{二茶村的大队之一}\end{définition}\end{entrée}

\begin{entrée}{kɯki}{}{ⓔkɯki} 
\classe{dem} 
\begin{définition}\pfra{ceci}\end{définition}
\begin{définition}\pcmn{这个}\end{définition}\end{entrée}

\begin{entrée}{kɯkrɯ/\variante{krɯkrɯ}}{}{ⓔkɯkrɯ} 
\classe{idph.2} 
\begin{définition}\pfra{découper en morceaux}\end{définition}
\begin{définition}\pcmn{割很多刀,切成几块}\end{définition}
\begin{exemple}\pjya{tɯrme ɯ-ŋga nɯ kɯkrɯ ʑo pjɤ-ta}\hspace{5pt}\pcmn{他在别人的衣服上割了很多刀}\end{exemple}
\begin{exemple}\pjya{paʁ pɯ́-wɣ-ntɕha tɕe kɯkrɯ ʑo ɣɯ-ta ra}\hspace{5pt}\pcmn{宰猪的时候,要切成很多块}\end{exemple}\relationsémantique{参考}{\lien{ⓔrɤkɯkrɯ}{rɤkɯkrɯ}}\end{entrée}

\begin{entrée}{kɯlɤɣpopo}{}{ⓔkɯlɤɣpopo} 
\classe{n} 
\begin{définition}\pfra{coccinelle}\end{définition}
\begin{définition}\pcmn{瓢虫}\end{définition}
\begin{exemple}\pjya{kɯ-lɤɣ popo nɯ qajɯ ci ŋu, tɕe ɯ-βri nɯ ɣɯrni ɯ-taʁ kɯ-ɲaʁ kɯ-ɤkhra tu, rko, ɯ-ʁar ɣɯ ɯ-rqhu ɲɯ-ŋu, ɯ-ʁar nɯ li kɯ-mba tɕe ɯ-rɯmu kɯ-tu ci ŋu, tɕe tɤ-pɤtso ra kɯ nɯ-jaʁ ɯ-taʁ tu-sɯxɕe-nɯ tɕe, ɯ-khɯkha tu-ti-nɯ kɯ ``a-wɯ kɯ-lɤɣ popo, ŋotɕu tɯ-ɕɯ-ɕe nɤ-qhɯ-qhu ɣi-a nɤ-ŋga nɤ-xtsa fkur-a", tɕe nɯ-jaʁndzu ɯ-ku tɤ-nɯɬoʁ tɕe, ju-nɯqambɯmbjom ŋu tɕe, ci ci kɯ-ɤrqhɯ-rqhi ju-ɕe ŋu, ci ci kɯ-ɤrmbɯ-rmbat ku-rɤʑi ŋu tɕe, ŋotɕu jɤ-ari ɯ-pɕoʁ nɯ tɕu thɯ-kɯ-wxti tɕe nɯ-pɕoʁ tɯ-sɤɣ-ɕe ŋu tu-kɯ-ti ɲɯ-ŋgrɤl.}\hspace{5pt}\pcmn{瓢虫是一种虫,身子是红色的,上面有黑点,很硬,是翅膀的外壳。翅膀很薄,有纹路。小孩子们让它在手指上爬,并说:“瓢虫爷爷,不管你去哪里我都会在后面跟着你,我要背上你的衣服、鞋子”。当瓢虫爬到手指的顶端时,就会飞走,有时飞得远,有时在很近的地方停下来。据说飞到哪个方向,小孩子长大以后就会往哪里去。}\end{exemple}\end{entrée}

\begin{entrée}{kɯm}{}{ⓔkɯm} 
\classe{n} 
\begin{définition}\pfra{porte}\end{définition}
\begin{définition}\pcmn{门}\end{définition}\relationsémantique{参考}{\lien{ⓔkumpɣa}{kumpɣa}}\relationsémantique{参考}{\lien{ⓔkumpɣɤtɕɯ}{kumpɣɤtɕɯ}}\end{entrée}

\begin{entrée}{kɯmu}{}{ⓔkɯmu} 
\classe{n} 
\begin{définition}\pfra{tétras (tetraogallus tibetanus)}\end{définition}
\begin{définition}\pcmn{藏雪鸡【贝母鸡】}\end{définition}
\begin{exemple}\pjya{kɯmu nɯ pɤjmu ruŋgu zgoku stu kɯ-mbro ku-rɤʑi ŋu, ɯ-mdoʁ kɯ-pɣi ŋu. kɯ-wxti ra kɯ kɯmu kɯ ``kɯ-ɣɤndʐo nɤ kɯ-ɣɤndʐo a-tɤ-nɯχsɤl, kɯ-mpja nɤ kɯ-mpja a-tɤ-nɯχsɤl" tu-ti ŋu tu-ti-nɯ ŋgrɤl ma tɤ-ɣɤndʐo tɕe rdzari ɯ-ku tu-ɕe ŋu, nɯ-mpja tɕe, co zɯ pjɯ-ɣi ŋu. chɯsɲu tu raŋ tɕe, kɯmu ɯ-skɤt a-pɯ-mtshɤm tɕe, phɤn tu-kɯ-ti ɲɯ-ŋgrɤl, kupa kɯ pɤjmu ɯ-kɯ-tɕɤt pɣa ŋu tu-kɯ-ti ɲɯ-ŋgrɤl, tɕe ŋu maʁ mɤ-xsi. tu-mbri tɕe, ``ku ku ku ku ku" tu-ti ŋu. χsɯm kɯβde tɯtɯrca tu-ŋgrɤl.}\hspace{5pt}\pcmn{贝母鸡栖息在最高山上,颜色是灰的。老人们有个说法:贝母鸡说:“要冷就冷个够,要热就热个够”,因为它在天气寒冷时到高山上去,天气温暖时就在山沟里出现。当有狂犬病的时候,据说患者听见贝母鸡的叫声就会好起来。汉人说是挖贝母的鸡,不知是不是正确的。叫的时候,叫声是\lien{}{ku ku ku ku ku}。贝母鸡一般三四只一起活动。}\end{exemple}\end{entrée}

\begin{entrée}{kɯma}{}{ⓔkɯma} 
\classe{part} 
\begin{définition}\pfra{question}\end{définition}
\begin{définition}\pcmn{会不会}\end{définition}
\begin{exemple}\pjya{kɯtɕu ku-tɯ-rɤʑi tɕe aʑo ju-okhu-a tɯ-mtshɤm ɕi kɯma}\hspace{5pt}\pcmn{你在这里你会不会听见我叫你呢?}\end{exemple}\end{entrée}

\begin{entrée}{kɯmaʁ}{}{ⓔkɯmaʁ} 
\classe{pro} 
\begin{définition}\pfra{autre}\end{définition}
\begin{définition}\pcmn{其他}\end{définition}
\begin{exemple}\pjya{li kɯmaʁ nɯ-ari ɕti ma}\hspace{5pt}\pcmn{(我们)又离题了}\end{exemple}
\begin{exemple}\pjya{kɯmaʁ tɯrme}\hspace{5pt}\pcmn{另一个人}\end{exemple}
\begin{exemple}\pjya{a-me kɯ a-@dianhua kɯmaʁ ta-χtɯ}\hspace{5pt}\pcmn{我女儿给我买了一部新手机}\end{exemple}\relationsémantique{参考}{\lien{ⓔmaʁⓗ1}{maʁ₁}}\end{entrée}

\begin{entrée}{kɯmɤɕtʂa}{}{ⓔkɯmɤɕtʂa} 
\classe{adv} 
\begin{définition}\pfra{jusqu'à maintenant}\end{définition}
\begin{définition}\pcmn{一直到现在}\end{définition}\end{entrée}

\begin{entrée}{kɯmɤlɤxso}{}{ⓔkɯmɤlɤxso} 
\classe{adv} 
\begin{définition}\pfra{pour rien}\end{définition}
\begin{définition}\pcmn{白白,徒劳}\end{définition}
\begin{exemple}\pjya{tɤ-ndze ma kɯmɤlɤxso a-mɤ-thɯ-ɤrɕo}\hspace{5pt}\pcmn{你吃吧,不要浪费}\end{exemple}
\begin{exemple}\pjya{kɯmɤlɤxso a-mɤ-ɕ-thɯ́-wɣ-βde ma nɤja}\hspace{5pt}\pcmn{不要浪费,太可惜了}\end{exemple}\relationsémantique{参考}{\lien{ⓔɯ-xso}{ɯ-xso}}\relationsémantique{参考}{\lien{ⓔso}{so}}\end{entrée}

\begin{entrée}{kɯmɕku}{}{ⓔkɯmɕku} 
\classe{n} 
\begin{définition}\pfra{ail}\end{définition}
\begin{définition}\pcmn{大蒜}\end{définition}
\begin{exemple}\pjya{kɯmɕku nɯ tɯ-ji ɯ-ŋgɯ lu-kɤ-nɯ-ji ci ŋu, ɯ-qa nɯ ɯ-tɯm rmi, ɯ-tɯm ɣɯ ɯ-ndzoʁ tu. ɯ-jwaʁ ma ɯ-ru me, ɯ-jwaʁ nɯ kɯ-pɣi tsa ŋu, kɯ-tɕɤr kɯ-rɲɟi tsa ŋu, ɯ-tho tu ri ɯ-mɯntoʁ me, ɯ-qa nɯ ɕku ɲɯ-βze ŋu, ɯ-zrɤm dɤn tɕe wɣrum, ɯ-qa nɯ-aβzu tɕe, tɯ-ndzoʁ tɯ-ndzoʁ ɯ-spjɯŋ ɯ-taʁ ku-fskɤr ŋu, ɯ-pɕi ɯ-rqhu kɤntɕhɯ-tɤlɤβ kɯ tu-oluj ŋu, ɯ-ndzoʁ raŋri nɯ li ɯ-rqhu tu, ɣɯrni. thɯ-tɯt ɯ-jija nɯ ɲɯ-mba ŋu. ɯ-ndzoʁ tsuku kɯβde ma me, tsuku ɕnɤcɤt jamar kɯ-tu tu, tɕeri ɯ-ndzoʁ tɤ-wxti tɕe rkɯn, ɯ-ndzoʁ tɤ-xtɕi tɕe dɤn. tsuku tɯ-rdoʁ ma kɯ-me tu, tɕe nɯ ɕku phɤri rmi.}\hspace{5pt}\pcmn{大蒜是是自己种在地里的(农作物)。根叫蒜头,蒜头有几个蒜瓣。只有叶子没有茎,叶子带有灰色,又窄又长,有花梗但是不开花,在根部结蒜头,是一瓣一瓣地围着主心干而长的,外面有很多层皮裹着,每个蒜瓣有自己的皮,是红色的。随着大蒜的成熟,外层的皮变得越来越薄。有的只有四个蒜瓣,有的有七八个。蒜瓣越大就越少,越小就越多。有的只有一个,这种叫\lien{}{ɕku phɤri}(对面的葱)。}\end{exemple}\relationsémantique{参考}{\lien{ⓔɕku}{ɕku}}\relationsémantique{参考}{\lien{ⓔkɯm}{kɯm}}\end{entrée}

\begin{entrée}{kɯmdza}{}{ⓔkɯmdza} 
\classe{n} 
\begin{définition}\pfra{membres de la famille}\end{définition}
\begin{définition}\pcmn{亲戚}\end{définition}
\begin{exemple}\pjya{kɯmdza mɤ-arɕɤt-i}\hspace{5pt}\pcmn{我们没有血缘关系}\end{exemple}
\begin{exemple}\pjya{kɯmdza tu-j}\hspace{5pt}\pcmn{我们有血缘关系}\end{exemple}\étymologie{mdza}\étymologie{mdzaɦ}\end{entrée}

\begin{entrée}{kɯmŋu}{}{ⓔkɯmŋu} 
\classe{num} 
\begin{définition}\pfra{cinq}\end{définition}
\begin{définition}\pcmn{五}\end{définition}\end{entrée}

\begin{entrée}{kɯmŋɤsqi}{}{ⓔkɯmŋɤsqi} 
\classe{num} 
\begin{définition}\pfra{cinquante}\end{définition}
\begin{définition}\pcmn{五十}\end{définition}\relationsémantique{参考}{\lien{ⓔsqi}{sqi}}\end{entrée}

\begin{entrée}{kɯmrka}{}{ⓔkɯmrka} 
\classe{n} 
\begin{définition}\pfra{poutre au-dessus de la porte}\end{définition}
\begin{définition}\pcmn{门框的上梁}\end{définition}\end{entrée}

\begin{entrée}{kɯmʁla}{}{ⓔkɯmʁla} 
\classe{n} 
\begin{définition}\pfra{cadre de la porte}\end{définition}
\begin{définition}\pcmn{门框}\end{définition}\end{entrée}

\begin{entrée}{kɯmtɕhoχsɯm}{}{ⓔkɯmtɕhoχsɯm} 
\classe{n} 
\begin{définition}\pfra{Triratna}\end{définition}
\begin{définition}\pcmn{三宝}\end{définition}\étymologie{dkon.mtɕʰog.gsum}\end{entrée}

\begin{entrée}{kɯmtɕhɯ}{}{ⓔkɯmtɕhɯ} 
\classe{n} 
\begin{définition}\pfra{jouet}\end{définition}
\begin{définition}\pcmn{玩具}\end{définition}\relationsémantique{参考}{\lien{ⓔnɯkɯmtɕhɯ}{nɯkɯmtɕhɯ}}\end{entrée}

\begin{entrée}{kɯmthoʁdɤn}{}{ⓔkɯmthoʁdɤn} 
\classe{n} 
\begin{définition}\pfra{seuil}\end{définition}
\begin{définition}\pcmn{门槛}\end{définition}\relationsémantique{参考}{\lien{ⓔtɤ-ʁdɤn}{tɤ-ʁdɤn}}\relationsémantique{参考}{\lien{ⓔkɯm}{kɯm}}\étymologie{gdan}\end{entrée}

\begin{entrée}{kɯnɤ}{}{ⓔkɯnɤ} 
\classe{adv} 
\begin{définition}\pfra{aussi}\end{définition}
\begin{définition}\pcmn{也是}\end{définition}
\begin{exemple}\pjya{ɯʑo ku-nɯkho ɕti ri, nɯ tɕu kɯnɤ ɯ-kɯ-ra ɲɯ-dɤn}\hspace{5pt}\pcmn{他是借住的,居然还有那么多要求}\end{exemple}\end{entrée}

\begin{entrée}{kɯndzarmɯ}{}{ⓔkɯndzarmɯ} 
\classe{n} 
\begin{définition}\pfra{ondée}\end{définition}
\begin{définition}\pcmn{阵雨(以后不再下雨的预兆)}\end{définition}
\begin{exemple}\pjya{kɯndzarmɯ ɲɯ-ɤsɯ-βzu tɕe, ki ɯ-qhu tɕe mɤ-lɤt ɲɯ-ŋu}\hspace{5pt}\pcmn{现在在下阵雨,以后不会再下}\end{exemple}\relationsémantique{参考}{\lien{ⓔtɯ-mɯ}{tɯ-mɯ}}\relationsémantique{参考}{\lien{ⓔndzar}{ndzar}}\end{entrée}

\begin{entrée}{kɯngɯsqi}{}{ⓔkɯngɯsqi} 
\classe{num} 
\begin{définition}\pfra{quatre-vingt dix}\end{définition}
\begin{définition}\pcmn{九十}\end{définition}\relationsémantique{参考}{\lien{ⓔsqi}{sqi}}\end{entrée}

\begin{entrée}{kɯngɯt}{}{ⓔkɯngɯt} 
\classe{num} 
\begin{définition}\pfra{neuf}\end{définition}
\begin{définition}\pcmn{九}\end{définition}\end{entrée}

\begin{entrée}{kɯngɯt tɤrqhɤɴɢaʁ}{}{ⓔkɯngɯt tɤrqhɤɴɢaʁ} 
\classe{n} 
\begin{définition}\pfra{une espèce d'arbrisseau}\end{définition}
\begin{définition}\pcmn{灌木的一种}\end{définition}
\begin{exemple}\pjya{kɯngɯt tɤrqhɤɴɢaʁ nɯ si kɯ-mbro tsa ci ŋu, ɯ-mdoʁ kɯ-pɣi tsa ci ŋu, ɯ-ru ɣɯ ɯ-rqhu pjɯ-ɴɢaʁ nɤ pjɯ-ɴɢaʁ ŋu, wuma ʑo mɤ-mbro tɕe kɤ-ntɕhoz kɯ-sna tu-βze mɤ-cha. ɲɯ́-wɣ-phɯt tɕe, kɤ-nɯ-βlɯ sna. ɯ-rqhu kɯ-dɤn núndʐa, kɯngɯt tɤrqhɤɴɢaʁ ɲɯ-rmi.}\hspace{5pt}\pcmn{\lien{ⓔkɯngɯt tɤrqhɤɴɢaʁ}{kɯngɯt tɤrqhɤɴɢaʁ}是一种长的不高的树,颜色是土灰色的,树干的树皮不断的脱落。因为树长得不高,不能用来做什么。砍下后可以用来烧火。因为树皮多,所以叫\lien{ⓔkɯngɯt tɤrqhɤɴɢaʁ}{kɯngɯt tɤrqhɤɴɢaʁ}(脱了九层皮)}\end{exemple}\end{entrée}

\begin{entrée}{kɯngɯt tɤrtsɤɣ}{}{ⓔkɯngɯt tɤrtsɤɣ} 
\classe{n} 
\begin{définition}\pfra{Leonurus sp.}\end{définition}
\begin{définition}\pcmn{益母草}\end{définition}
\begin{exemple}\pjya{kɯngɯt tɤrtsɤɣ nɯ sɯjno ɯ-rtaʁ mɤ-kɯ-dɤn, mɤ-kɯ-mbro tsa ci ŋu. ɯ-ru nɯ kɯ-ɤβʑɯrdu tu-ɬoʁ ŋu, ɯ-rtsɤɣ tu-oʑɯrja ŋu, ɯ-rtsɤɣ raŋri nɯtɕu ɯ-jwaʁ kɯ-xtɕɯ-xtɕi ɲɯ-βze tɕe, ɯ-rca nɯ tɕu ɯ-mɯntoʁ kɯ-dɯ-dɤn ʑo ku-ofskɤr, tɕe ɯ-mat chɯ-βze ŋu. kɯ-ɕɯŋgɯ tɕe, tɤ-mu ra kɯ kɯngɯt tɤrtsɤɣ ɲɯ-phɯt-nɯ tɕe, tɤ-lu kɤ-kɤ-z-rɤɕom ɯ-taʁ nɯ tɕu ɲɯ-ta-nɯ tɕe, tɤlɤɕom kɯ-jaʁ ku-te cha tu-ti-nɯ pɯ-ŋu. kɯngɯt tɤrtsɤɣ nɯ kha kɯngɯ-rtsɤɣ kɤ-ti ɲɯ-ŋu.}\hspace{5pt}\pcmn{益母草是没有很多枝桠,长得不高的草。茎是四方形的,是由几节组成的,在每一节上,茎的一周长出小叶子,然后长出小花,最后结果。以前,大娘们把它扯一根,放在凉着结奶皮的牛奶上,据说这样可以使奶皮结得好一些。 \lien{ⓔkɯngɯt tɤrtsɤɣ}{kɯngɯt tɤrtsɤɣ} 是“九层楼”的意思。}\end{exemple}\end{entrée}

\begin{entrée}{kɯni}{}{ⓔkɯni} 
\classe{dem} 
\begin{définition}\pfra{ces deux choses}\end{définition}
\begin{définition}\pcmn{这两个}\end{définition}\end{entrée}

\begin{entrée}{kɯnɯkhamu}{}{ⓔkɯnɯkhamu} 
\classe{n} 
\begin{définition}\pfra{cuisinier}\end{définition}
\begin{définition}\pcmn{炊事员}\end{définition}\relationsémantique{参考}{\lien{ⓔnɯkhamu}{nɯkhamu}}\end{entrée}

\begin{entrée}{kɯnɯmar}{}{ⓔkɯnɯmar} 
\classe{n} 
\begin{définition}\pfra{une espèce d'arbrisseau}\end{définition}
\begin{définition}\pcmn{灌木的一种}\end{définition}
\begin{exemple}\pjya{kɯnɯmar nɯ si ci ŋu wuma ʑo tu-mbro cho ɲɯ-jpɯm mɤ-cha. ɯ-jwaʁ nɯ kɯ-tɕɤr kɯ-rɲɟi tsa ŋu, ɯ-ru nɯ kɯ-ɲaʁ ɯ-ŋgɯz kɯ-ɣɯrni tsa ŋu, ɯ-ru ɯ-ŋgɯ nɯ kɯ-wɣrum kɯ-fse xtsoŋxtsoŋ ɕti tɕe, si mɤ-ngɯt. ɯ-mɯntoʁ kɯ-wɣrum kɯ-dɤn ʑo ɲɯ-ɬoʁ ŋu. si ɯ-kɤχcɤl ri ɲɯ-lɤt ŋu.}\hspace{5pt}\pcmn{\lien{ⓔkɯnɯmar}{kɯnɯmar}是一种树,不能长高,也能长粗。叶子细长,树干黑里透红。树干里有白色泡沫,所以木质不结实。开很多白花,集中在树梢上。}\end{exemple}\end{entrée}

\begin{entrée}{kɯnɯwulaʁ}{}{ⓔkɯnɯwulaʁ} 
\classe{n} 
\begin{définition}\pfra{personne assujettie à la corvée}\end{définition}
\begin{définition}\pcmn{专为土司背东西的人(乌拉)}\end{définition}\étymologie{ɦu.lag}\end{entrée}

\begin{entrée}{kɯɲidi}{}{ⓔkɯɲidi} 
\classe{n} 
\begin{définition}\pfra{odeur d'homme}\end{définition}
\begin{définition}\pcmn{人的气味}\end{définition}\end{entrée}

\begin{entrée}{kɯpɤz}{}{ⓔkɯpɤz} 
\classe{n} 
\begin{définition}\pfra{un type de ver}\end{définition}
\begin{définition}\pcmn{虫的一种}\end{définition}\end{entrée}

\begin{entrée}{kɯqurʑŋgri}{}{ⓔkɯqurʑŋgri} 
\classe{n} 
\begin{définition}\pfra{vénus, étoile du matin}\end{définition}
\begin{définition}\pcmn{金星}\end{définition}\relationsémantique{参考}{\lien{ⓔqur}{qur}}\relationsémantique{参考}{\lien{ⓔʑŋgri}{ʑŋgri}}\end{entrée}

\begin{entrée}{kɯra/\variante{kɯkɯra}}{}{ⓔkɯra} 
\classe{dem} 
\begin{définition}\pfra{ces choses}\end{définition}
\begin{définition}\pcmn{这些}\end{définition}\end{entrée}

\begin{entrée}{kɯrcat}{}{ⓔkɯrcat} 
\classe{num} 
\begin{définition}\pfra{huit}\end{définition}
\begin{définition}\pcmn{八}\end{définition}\end{entrée}

\begin{entrée}{kɯrcɤsqi}{}{ⓔkɯrcɤsqi} 
\classe{num} 
\begin{définition}\pfra{quatre-vingt}\end{définition}
\begin{définition}\pcmn{八十}\end{définition}\relationsémantique{参考}{\lien{ⓔsqi}{sqi}}\end{entrée}

\begin{entrée}{kɯre}{}{ⓔkɯre} 
\classe{adv} 
\begin{définition}\pfra{ici}\end{définition}
\begin{définition}\pcmn{在这里}\end{définition}\end{entrée}

\begin{entrée}{kɯrnɯ}{}{ⓔkɯrnɯ} 
\classe{n} 
\begin{définition}\pfra{mite}\end{définition}
\begin{définition}\pcmn{蛀虫}\end{définition}
\begin{exemple}\pjya{kɯrnɯ nɯ qajɯ kɯ-ɣɯrni ci ŋu, ɯ-jme ri ɯ-rme kɯ-tu ci ŋu, zndɤrchɤβ ɯ-thoʁ aʁɤndɯndɤt ʑo tu, ɯ-βri ra lo-rɤrkhɯrkhe kɯ-fse ŋu, ɯʑo xtɕi ri wuma ʑo ŋɤn, ma tɯ-ŋga smɤɣ thɯ-kɤ-βzu kɯ-fse nɯ ra tu-ndze ŋu, tɯ-jpu tu-ndze ŋu, tu-ndze mɤ-kɯ-jɤɣ kɯ ku-sɯ-ɤɴqhi ŋu. tɯ-jpu ra ɯ-ntɕha-ntɕhɯr ɲɯ-sɤβze ŋu, tɕe kɤ-ɣɤme ftɕaka nɯ tɯ-ŋga ɯ-taʁ tɕe qajɯ sɤ-sat ɣɯ smɤn tú-wɣ-lɤt, tɤ-rɤku nɯ tɯ-ci ɯ-ŋgɯ pjɯ́-wɣ-ɣɤla tɕe, tɕe qajɯ cho tɤ-rɤku ɯ-ntɕha-ntɕhɯr nɯ ra tɯ-ci kɯ ɯ-taʁ tu-ɣɯt ŋu ma ʑo tɕe, tɕe tú-wɣ-tɕɤt tɕe ɲɯ́-wɣ-ɣɤme khɯ.}\hspace{5pt}\pcmn{蛀虫是一种红色的虫,尾巴上有一撮毛,墙壁缝里地上到处都有。身子像割了几刀,有类似刀刻的花纹。蛀虫虽然小但是有害,吃毛织品的衣服,吃粮食,不但吃还要把它弄脏,把粮食弄成一块一块的渣滓一样。消灭的办法是在衣服上喷杀虫剂,把粮食泡在水里,虫和粮食渣滓就会浮上水面,这样拿出来就可以消灭它。}\end{exemple}\end{entrée}

\begin{entrée}{kɯrŋi}{}{ⓔkɯrŋi} 
\classe{n} 
\begin{définition}\pfra{fauve}\end{définition}
\begin{définition}\pcmn{猛兽}\end{définition}\end{entrée}

\begin{entrée}{kɯrŋukɯɣndʑɯr}{}{ⓔkɯrŋukɯɣndʑɯr} 
\classe{n} 
\begin{définition}\pfra{faucheur}\end{définition}
\begin{définition}\pcmn{盲蛛}\end{définition}
\begin{exemple}\pjya{kɯrŋu kɯɣndʑɯr ɯ-mi kɯ-rɲɟɯ-rɲɟi ʑo kɯrcɤ-ldʑa tu, ɯ-phoŋbu nɯ kɯ-ɤrtɯ-rtɯm rloʁrloʁ kɯ-xtɕi tsa ŋu, ɯ-taʁ ri ɯ-mɲaʁ tɯ-tɕha ɣɤʑu, ɯ-phoŋbu cho ɯ-mi ra lonba ɲɯ-ɲaʁ, ɯ-mi nɯ ɯ-phoŋbu ku-fskɤr ʑo ku-ndzoʁ ɲɯ-ŋu. ɯ-mi nɯ-mbɯt tɕe tɤ-rʑaʁ kɯ-rɲɟi ʑo tu-mɯnmu ɲɯ-cha. kɯrŋu kɯɣndʑɯr ɯʑo ɯ-kɯ-ndza ɣɤʑu ma ɯʑo kɯ ɯ-zda kɤ-ndza mɯ́j-cha.}\hspace{5pt}\pcmn{盲蛛有八只很长的脚,身子是球形的,很小,上面有一对眼睛,脚和身体全是黑色的,脚处于身体的周围。拔掉了脚后还能动一段时间。有动物吃盲蛛,但它不吃其它动物。}\end{exemple}\relationsémantique{参考}{\lien{ⓔɣndʑɯr}{ɣndʑɯr}}\end{entrée}

\begin{entrée}{kɯrowɤro}{}{ⓔkɯrowɤro} 
\classe{n} 
\begin{définition}\pfra{en trop}\end{définition}
\begin{définition}\pcmn{多余的}\end{définition}
\begin{exemple}\pjya{kɯrowɤro ʑo ma-tɯ-nɤme}\hspace{5pt}\pcmn{你不要做多余的工作}\end{exemple}
\begin{exemple}\pjya{kɯrowɤro ʑo ma-tɯ-nɯskɤt}\hspace{5pt}\pcmn{你不要讲多余的话}\end{exemple}
\begin{exemple}\pjya{tɯ-rju kɯ-rowɤro nɯ ra mɤ-ra}\hspace{5pt}\pcmn{不要有多余的话}\end{exemple}\end{entrée}

\begin{entrée}{kɯroz}{}{ⓔkɯroz} 
\classe{adv} 
\begin{définition}\pfra{spécialement, particulièrement}\end{définition}
\begin{définition}\pcmn{特别}\end{définition}
\begin{exemple}\pjya{ɯʑo kɯroz ʑo mbro}\hspace{5pt}\pcmn{他长得特别高}\end{exemple}\end{entrée}

\begin{entrée}{kɯrtsɤɣ}{}{ⓔkɯrtsɤɣ} 
\classe{n} 
\begin{définition}\pfra{panthère}\end{définition}
\begin{définition}\pcmn{豹子}\end{définition}
\begin{exemple}\pjya{kɯrtsɤɣ nɯ aʁɤndɯndɤt tu ŋgrɤl, ɯ-mdoʁ kɤ-qarŋɯrŋe ɯ-taʁ kɯ-ɲaʁ kɯ-ɤkhra ŋu, ɯ-mtɕhirme kɯ-wɣrum ŋu, ɯ-mɤlɤjaʁ ɯ-ndzrɯ cho lɯlu ɯ-mɤndzrɯ fse, ɯ-jme kɯnɤ akhra, ɯ-jme ɯ-ku tɯ-snaʁ kɯ-wɣrum ŋu. tshɤt qaʑo wuma kɤ-ndza rga, nɯŋa kɯ-fse li pjɯ-sat cha, jla mbro kɯ-fse nɯ ra kɯ-dɤn mɤ-kɤm.}\hspace{5pt}\pcmn{豹子到处都有,颜色是淡黄色,上有黑色的斑点,胡须是白色的,爪子和猫的一样,尾巴也是花的,尾巴的尖端有白点。很喜欢吃羊,也能杀死奶牛,但没有能力打败犏牛和马。}\end{exemple}\étymologie{gzigs}\end{entrée}

\begin{entrée}{kɯrɯ}{}{ⓔkɯrɯ} 
\classe{n} 
\begin{définition}\pfra{tibétain}\end{définition}
\begin{définition}\pcmn{藏族}\end{définition}
\begin{exemple}\pjya{kɯrɯ skɤt nɯ sɤʁʑi ɯ-ku zɯ tɯskɤt stu kɯ-mpɕɤr ɕti.}\hspace{5pt}\pcmn{藏语是世界上最好听的语言}\end{exemple}\end{entrée}

\begin{entrée}{kɯrɯɕɤmɯɣdɯ}{}{ⓔkɯrɯɕɤmɯɣdɯ} 
\classe{n} 
\begin{définition}\pfra{arquebuse}\end{définition}
\begin{définition}\pcmn{民火枪}\end{définition}\relationsémantique{参考}{\lien{ⓔɕɤmɯɣdɯ}{ɕɤmɯɣdɯ}}\end{entrée}

\begin{entrée}{kɯrɯɕoŋβzu}{}{ⓔkɯrɯɕoŋβzu} 
\classe{n} 
\begin{définition}\pfra{menuisier}\end{définition}
\begin{définition}\pcmn{木匠}\end{définition}\étymologie{ɕiŋ.bzo}\end{entrée}

\begin{entrée}{kɯrɯŋga}{}{ⓔkɯrɯŋga} 
\classe{n} 
\begin{définition}\pfra{habits tibétains}\end{définition}
\begin{définition}\pcmn{藏装}\end{définition}\relationsémantique{参考}{\lien{ⓔtɯ-ŋga}{tɯ-ŋga}}\relationsémantique{参考}{\lien{ⓔkɯrɯ}{kɯrɯ}}\end{entrée}

\begin{entrée}{kɯsɤɣru}{}{ⓔkɯsɤɣru} 
\classe{n} 
\begin{définition}\pfra{miroir}\end{définition}
\begin{définition}\pcmn{镜子}\end{définition}
\begin{exemple}\pjya{aʑo kɯsɤɣru ɯ-ŋgɯ kɤ-ru-a}\hspace{5pt}\pcmn{我照了镜子}\end{exemple}\relationsémantique{同义词}{\lien{ⓔχɕɤlzgoŋ}{χɕɤlzgoŋ}}\relationsémantique{参考}{\lien{ⓔruⓗ1}{ru₁}}\end{entrée}

\begin{entrée}{kɯspoʁ}{}{ⓔkɯspoʁ} 
\classe{n} 
\begin{définition}\pfra{trou}\end{définition}
\begin{définition}\pcmn{洞}\end{définition}\relationsémantique{参考}{\lien{ⓔspoʁ}{spoʁ}}\end{entrée}

\begin{entrée}{kɯsthi}{}{ⓔkɯsthi} 
\classe{n} 
\begin{définition}\pfra{autant}\end{définition}
\begin{définition}\pcmn{这么多}\end{définition}\end{entrée}

\begin{entrée}{kɯtaʁ}{}{ⓔkɯtaʁ} 
\classe{n} 
\begin{définition}\pfra{geste}\end{définition}
\begin{définition}\pcmn{动作}\end{définition}\end{entrée}

\begin{entrée}{kɯtɕapɯ}{}{ⓔkɯtɕapɯ} 
\classe{n} 
\begin{définition}\pfra{roturier}\end{définition}
\begin{définition}\pcmn{平民}\end{définition}\end{entrée}

\begin{entrée}{kɯtʂɤɣ}{}{ⓔkɯtʂɤɣ} 
\classe{num} 
\begin{définition}\pfra{six}\end{définition}
\begin{définition}\pcmn{六}\end{définition}\end{entrée}

\begin{entrée}{kɯtʂɤsqi}{}{ⓔkɯtʂɤsqi} 
\classe{num} 
\begin{définition}\pfra{soixante}\end{définition}
\begin{définition}\pcmn{六十}\end{définition}\relationsémantique{参考}{\lien{ⓔsqi}{sqi}}\end{entrée}

\begin{entrée}{kɯz}{}{ⓔkɯz} 
\classe{intj} 
\begin{définition}\pfra{vas-y d'abord!}\end{définition}
\begin{définition}\pcmn{你在前面走!}\end{définition}
\begin{exemple}\pjya{kɯz, tɤ-mbɣom!}\hspace{5pt}\pcmn{走吧,快点!}\end{exemple}\end{entrée}

\begin{entrée}{kɯzɣa}{}{ⓔkɯzɣa} 
\classe{adv} 
\begin{définition}\pfra{très longtemps, de nombreuses fois, vraiment, décidément}\end{définition}
\begin{définition}\pcmn{很久;很多次;确实}\end{définition}
\begin{exemple}\pjya{kɯzɣa ʑo ɲɤ-ɕar}\hspace{5pt}\pcmn{找了很久,费了很大的工夫}\end{exemple}
\begin{exemple}\pjya{kɯzɣa ʑo ɲɯ-tɯ-mtsɯr !}\hspace{5pt}\pcmn{你确实很饿}\end{exemple}\end{entrée}

\begin{entrée}{kɯʑŋgu}{}{ⓔkɯʑŋgu} 
\classe{n}  
\grammaire{n.lieu} 
\begin{définition}\pfra{l'un des hameaux de Gyutshapa}\end{définition}
\begin{définition}\pcmn{二茶村的大队之一}\end{définition}\end{entrée}

\begin{entrée}{kuwu}{}{ⓔkuwu} 
\classe{n} 
\begin{définition}\pfra{gypaète barbu}\end{définition}
\begin{définition}\pcmn{胡兀鹫}\end{définition}\étymologie{ko.bo}\end{entrée}

\begin{entrée}{kwitsɯt}{}{ⓔkwitsɯt} 
\classe{n} 
\begin{définition}\pfra{armoire}\end{définition}
\begin{définition}\pcmn{柜子}\end{définition}\étymologie{fn:柜子}\end{entrée}

\begin{entrée}{kuxtɕo}{}{ⓔkuxtɕo} 
\classe{n} 
\begin{définition}\pfra{hotte}\end{définition}
\begin{définition}\pcmn{背篼}\end{définition}
\begin{exemple}\pjya{ɯ-kɯxtɕɯxtɕo ʑo}\hspace{5pt}\pcmn{一背篼一背篼的}\end{exemple}\end{entrée}

\newpage\caractère{l}

\begin{entrée}{lu}{}{ⓔlu} 
\classe{n} 
\begin{définition}\pfra{signe astrologique}\end{définition}
\begin{définition}\pcmn{生肖}\end{définition}
\begin{exemple}\pjya{nɤʑo tɕhi lu tɯ-ŋu}\hspace{5pt}\pcmn{你属什么?}\end{exemple}
\begin{exemple}\pjya{a-lu na-rtsi}\hspace{5pt}\pcmn{他给我算了命}\end{exemple}\relationsémantique{参考}{\lien{ⓔrtsɯβⓝɯ-lu,rtsɯβ}{ɯ-lu,rtsɯβ}}\étymologie{lo}\end{entrée}

\begin{entrée}{la}{}{ⓔla} 
\classe{vt} \sens{1}\paradigme{dir}{nɯ-}
\begin{définition}\pfra{s'imbiber d'eau}\end{définition}
\begin{définition}\pcmn{泡胀}\end{définition}
\begin{exemple}\pjya{stoʁ nɯ-la ri tɕe ɲɯ-mpɯ ŋu}\hspace{5pt}\pcmn{胡豆浸泡胀了就会变软}\end{exemple}\sens{2}\paradigme{dir}{pɯ-}
\begin{définition}\pfra{tomber dans l'eau}\end{définition}
\begin{définition}\pcmn{掉进水里}\end{définition}
\begin{exemple}\pjya{tɯ-ci ɯ-ŋgɯ pjɤ-la (=pjɤ-ɕe)}\hspace{5pt}\pcmn{他掉进水里了}\end{exemple}\relationsémantique{参考}{\lien{ⓔɣɤla}{ɣɤla}}\end{entrée}

\begin{entrée}{laβde}{}{ⓔlaβde} 
\classe{num} 
\begin{définition}\pfra{à peu près quatre}\end{définition}
\begin{définition}\pcmn{大概四个}\end{définition}\relationsémantique{参考}{\lien{ⓔkɯβde}{kɯβde}}\end{entrée}

\begin{entrée}{laβdelɤŋu}{}{ⓔlaβdelɤŋu} 
\classe{num} 
\begin{définition}\pfra{quatre ou cinq}\end{définition}
\begin{définition}\pcmn{四五个}\end{définition}
\begin{exemple}\pjya{laβdelɤŋu-sŋi}\hspace{5pt}\pcmn{四五天}\end{exemple}\relationsémantique{参考}{\lien{ⓔkɯβde}{kɯβde}}\relationsémantique{参考}{\lien{ⓔkɯmŋu}{kɯmŋu}}\end{entrée}

\begin{entrée}{laβzɣi}{}{ⓔlaβzɣi} 
\classe{n} 
\begin{définition}\pfra{navet cuit à la vapeur}\end{définition}
\begin{définition}\pcmn{蒸了的芜菁根}\end{définition}\relationsémantique{参考}{\lien{ⓔrɤjndoʁ}{rɤjndoʁ}}\end{entrée}

\begin{entrée}{laftaʁ}{}{ⓔlaftaʁ} 
\classe{n} 
\begin{définition}\pfra{caution}\end{définition}
\begin{définition}\pcmn{押金;纪念;信号}\end{définition}\end{entrée}

\begin{entrée}{laftɕɤn}{}{ⓔlaftɕɤn} 
\classe{n} 
\begin{définition}\pfra{chance}\end{définition}
\begin{définition}\pcmn{运气}\end{définition}
\begin{exemple}\pjya{ɯʑo laftɕɤn ci ŋu}\hspace{5pt}\pcmn{他很幸运}\end{exemple}\end{entrée}

\begin{entrée}{lahaŋ}{}{ⓔlahaŋ} 
\classe{n} 
\begin{définition}\pfra{soufre}\end{définition}
\begin{définition}\pcmn{硫磺}\end{définition}\étymologie{fn:硫磺}\end{entrée}

\begin{entrée}{lajɯ}{}{ⓔlajɯ} 
\classe{n} 
\begin{définition}\pfra{chant de montagne}\end{définition}
\begin{définition}\pcmn{山歌}\end{définition}
\begin{définition}\pcmn{他唱了山歌}\end{définition}
\begin{exemple}\pjya{ɯʑo kɯ lajɯ pjɤ-βzu}\end{exemple}\relationsémantique{参考}{\lien{ⓔrɯlajɯ}{rɯlajɯ}}\end{entrée}

\begin{entrée}{laloŋ}{}{ⓔlaloŋ} 
\classe{adv} 
\begin{définition}\pfra{partout}\end{définition}
\begin{définition}\pcmn{统统;到处}\end{définition}
\begin{exemple}\pjya{sɤtɕha laloŋ ʑo ɕ-to-khɤt}\hspace{5pt}\pcmn{他把所有地方都统统走了一遍}\end{exemple}\relationsémantique{参考}{\lien{ⓔsaloŋ}{saloŋ}}\end{entrée}

\begin{entrée}{lankatsɯt}{}{ⓔlankatsɯt} 
\classe{n} 
\begin{définition}\pfra{type d'habit en poil de yak}\end{définition}
\begin{définition}\pcmn{氆氇做成的一种衣服}\end{définition}\end{entrée}

\begin{entrée}{laŋ}{}{ⓔlaŋ} 
\classe{vi} \paradigme{dir}{tɤ-}
\begin{définition}\pfra{se soulever}\end{définition}
\begin{définition}\pcmn{起义}\end{définition}
\begin{exemple}\pjya{mkhɤrmaŋ to-laŋ-nɯ}\hspace{5pt}\pcmn{老百姓起义了}\end{exemple}\étymologie{laŋ}\end{entrée}

\begin{entrée}{laŋgɯ}{}{ⓔlaŋgɯ} 
\grammaire{n.lieu} 
\begin{définition}\pfra{l'un des hameaux de Gyutshapa}\end{définition}
\begin{définition}\pcmn{二茶村的大队之一}\end{définition}\end{entrée}

\begin{entrée}{laŋlaŋ}{}{ⓔlaŋlaŋ} 
\classe{n} 
\begin{définition}\pfra{une espèce de cerisier}\end{définition}
\begin{définition}\pcmn{野樱桃的一种}\end{définition}
\begin{exemple}\pjya{laŋlaŋ nɯ si khro mɤ-mbro, zgoku kɯ-mbɤr tsa tu-ɬoʁ ŋu, ɯ-ru ɣɯrni, ɯ-jwaʁ mɤ-jndʐɤz, kɯ-ɤrtɯm tsa ŋu. χɕitka tɕe, ɯ-jwaʁ ɲɯ-lɤt ɕɯŋgɯ ɯ-mɯntoʁ ɲɯ-lɤt ŋu. ɯ-jwaʁ na-lɤt ɯ-rca tɕe, ɯ-mɯntoʁ pjɯ-ŋgra ŋu, tɕe ɯ-mat tɯ-βzu ɲɯ-ʑe ŋu. ɯ-mat thɯ-tɯt tɕe, ɣɯrni, artɯm rloʁrloʁ ʑo, tú-wɣ-ndza tɕe, kɯ-chi tu, kɯ-tɕur tu, ɯ-ŋgɯ ɯ-rɣi nɯ wxti tsa.}\hspace{5pt}\pcmn{野樱桃长得不高,生长在半山上,树干是红色的,叶子不大,有点圆。春天,在长叶之前就开花。在长叶的同时,花就凋谢了,就开始结果。果实成熟后,是红色的,球形的。吃起来有的很甜,有的酸,里面的种子比较大。}\end{exemple}\end{entrée}

\begin{entrée}{largi}{}{ⓔlargi} 
\classe{n} 
\begin{définition}\pfra{vieux moine}\end{définition}
\begin{définition}\pcmn{老和尚}\end{définition}\end{entrée}

\begin{entrée}{laʁdɯn}{}{ⓔlaʁdɯn} 
\classe{n} 
\begin{définition}\pfra{outil}\end{définition}
\begin{définition}\pcmn{工具}\end{définition}\étymologie{lag.ldan}\end{entrée}

\begin{entrée}{laʁjɤt}{}{ⓔlaʁjɤt} 
\classe{n} 
\begin{définition}\pfra{travail manuel}\end{définition}
\begin{définition}\pcmn{手工}\end{définition}\relationsémantique{参考}{\lien{ⓔrɯlaʁjɤt}{rɯlaʁjɤt}}\end{entrée}

\begin{entrée}{laʁjoʁ}{}{ⓔlaʁjoʁ} 
\classe{n} 
\begin{définition}\pfra{assistant}\end{définition}
\begin{définition}\pcmn{帮手}\end{définition}\relationsémantique{参考}{\lien{ⓔnɯlaʁjoʁ}{nɯlaʁjoʁ}}\étymologie{lag.gjog}\end{entrée}

\begin{entrée}{laʁjɯɣ}{}{ⓔlaʁjɯɣ} 
\classe{n} 
\begin{définition}\pfra{bâton}\end{définition}
\begin{définition}\pcmn{棍子}\end{définition}\étymologie{lag.dbʲug}\end{entrée}

\begin{entrée}{laʁma}{}{ⓔlaʁma} 
\classe{cnj} 
\begin{définition}\pfra{bien que, à part}\end{définition}
\begin{définition}\pcmn{虽然;除此以外}\end{définition}\end{entrée}

\begin{entrée}{laʁnɤ}{}{ⓔlaʁnɤ} 
\classe{conj} 
\begin{définition}\pfra{au moins}\end{définition}
\begin{définition}\pcmn{……倒}\end{définition}
\begin{exemple}\pjya{aʑo laʁnɤ tɤ-nɯmgo-a ma nɤʑo nɤ-ndzɤtshi pɯ-me tɕe tɯ-mtsɯr}\hspace{5pt}\pcmn{我倒是吃过饭了,你没有吃饭就肯定饿了}\end{exemple}\end{entrée}

\begin{entrée}{laʁnɯχsɯm}{}{ⓔlaʁnɯχsɯm} 
\classe{num} 
\begin{définition}\pfra{deux ou trois}\end{définition}
\begin{définition}\pcmn{两三个}\end{définition}\end{entrée}

\begin{entrée}{laʁnɯz}{}{ⓔlaʁnɯz} 
\classe{num} 
\begin{définition}\pfra{un ou deux, quelques}\end{définition}
\begin{définition}\pcmn{一两个}\end{définition}
\begin{exemple}\pjya{laʁnɯ-sŋi}\hspace{5pt}\pcmn{一两天、几天}\end{exemple}\end{entrée}

\begin{entrée}{laʁŋkhɤr}{}{ⓔlaʁŋkhɤr} 
\classe{n} 
\begin{définition}\pfra{moulin à prière}\end{définition}
\begin{définition}\pcmn{手摇转经筒}\end{définition}\étymologie{lag.ⁿkʰor}\end{entrée}

\begin{entrée}{laʁrda}{}{ⓔlaʁrda} 
\classe{n} 
\begin{définition}\pfra{geste}\end{définition}
\begin{définition}\pcmn{手势}\end{définition}\étymologie{lag.brda}\end{entrée}

\begin{entrée}{laʁrdɤβ}{}{ⓔlaʁrdɤβ} 
\classe{n} 
\begin{définition}\pfra{coup de patte avant}\end{définition}
\begin{définition}\pcmn{用前腿打}\end{définition}
\begin{exemple}\pjya{ɯʑo jla ci tu tɕe laʁrdɤβ kɤ-lɤt rga}\hspace{5pt}\pcmn{我们有一头犏牛,特别喜欢用前腿踢人}\end{exemple}\end{entrée}

\begin{entrée}{laʁrŋa}{}{ⓔlaʁrŋa} 
\classe{n} 
\begin{définition}\pfra{tambour à main}\end{définition}
\begin{définition}\pcmn{长柄鼓}\end{définition}\étymologie{lag.rŋa}\end{entrée}

\begin{entrée}{laʁsta}{}{ⓔlaʁsta} 
\classe{n} 
\begin{définition}\pfra{marteau}\end{définition}
\begin{définition}\pcmn{锤子}\end{définition}\end{entrée}

\begin{entrée}{laʁsɯɣma}{}{ⓔlaʁsɯɣma} 
\classe{cnj} 
\begin{définition}\pfra{à part ça}\end{définition}
\begin{définition}\pcmn{除此以外}\end{définition}
\begin{exemple}\pjya{tɯ-mɯ ci kɤ-lɤt laʁsɯɣma, jisŋi pɯ-sɤscit}\hspace{5pt}\pcmn{今天很开心,只不过下了一点雨}\end{exemple}\end{entrée}

\begin{entrée}{laʁsɯɣnɤma}{}{ⓔlaʁsɯɣnɤma} 
\classe{cnj} 
\begin{définition}\pfra{à part ça}\end{définition}
\begin{définition}\pcmn{除此以外;只是}\end{définition}
\begin{exemple}\pjya{nɯ kɤ-kho nɯ ɲɤ-nɯmɟa laʁsɯɣnɤma, nɯ ma mɯ-to-nɯβdaʁ}\hspace{5pt}\pcmn{他只是把送的东西接过去了,其它理也没有理。}\end{exemple}\end{entrée}

\begin{entrée}{laʁzu}{}{ⓔlaʁzu} 
\classe{n} 
\begin{définition}\pfra{type d'offrande au dieux, un des éléments utilisé pour faire les gtorma}\end{définition}
\begin{définition}\pcmn{供奉鬼神的一种物品}\end{définition}\étymologie{lag.bzo}\end{entrée}

\begin{entrée}{lawa}{}{ⓔlawa} 
\classe{n} 
\begin{définition}\pfra{laine}\end{définition}
\begin{définition}\pcmn{羊毛}\end{définition}\end{entrée}

\begin{entrée}{laχɕi}{}{ⓔlaχɕi} 
\classe{n} 
\begin{définition}\pfra{métier (manuel)}\end{définition}
\begin{définition}\pcmn{手艺}\end{définition}
\begin{exemple}\pjya{ɯ-laχɕi tu}\hspace{5pt}\pcmn{他有手艺}\end{exemple}\end{entrée}

\begin{entrée}{laχsɯm}{}{ⓔlaχsɯm} 
\classe{num} 
\begin{définition}\pfra{deux ou trois}\end{définition}
\begin{définition}\pcmn{两三个}\end{définition}
\begin{exemple}\pjya{laχsɯ-sŋi}\hspace{5pt}\pcmn{两三天}\end{exemple}\end{entrée}

\begin{entrée}{laχtɕha}{}{ⓔlaχtɕha} 
\classe{n} 
\begin{définition}\pfra{objet}\end{définition}
\begin{définition}\pcmn{东西}\end{définition}\étymologie{lag.tɕʰa}\end{entrée}

\begin{entrée}{laχthɤβ}{}{ⓔlaχthɤβ} 
\classe{n} 
\begin{définition}\pfra{médecin traditionnel qui répare les fractures}\end{définition}
\begin{définition}\pcmn{专门治疗骨折、脱臼的土医生}\end{définition}\end{entrée}

\begin{entrée}{laχtsɯ}{}{ⓔlaχtsɯ} 
\classe{n} 
\begin{définition}\pfra{poutre du balcon}\end{définition}
\begin{définition}\pcmn{走缘的柱子}\end{définition}\end{entrée}

\begin{entrée}{laʑu}{}{ⓔlaʑu} 
\classe{n} 
\begin{définition}\pfra{viande fumée}\end{définition}
\begin{définition}\pcmn{腊肉}\end{définition}\étymologie{fn:腊肉}\end{entrée}

\begin{entrée}{lɤchu}{}{ⓔlɤchu} 
\classe{adv} 
\begin{définition}\pfra{en amont}\end{définition}
\begin{définition}\pcmn{在上游}\end{définition}\relationsémantique{参考}{\lien{ⓔɯ-locu}{ɯ-locu}}\end{entrée}

\begin{entrée}{lɤftsɤz/\variante{lɤftɕɤz}}{}{ⓔlɤftsɤz} 
\classe{n} 
\begin{définition}\pfra{endroit sur le toit où l'on plante un rlung-rta et où l'on élève un tas de silex}\end{définition}
\begin{définition}\pcmn{屋顶上插有经幡,用白燧石堆积而成的石堆(嘉绒式敖包)}\end{définition}
\begin{exemple}\pjya{lɤftsɤz nɯ khɤxtɤmbro ɣɯ ɯ-qhu znde tu-kɯ-ɣi ɣɯ akɯ andi ɯ-χcɤl li znde kɯ-xtɕi ci kɯ-ɤβʑɯrdu, kha ɯ-znde sɤznɤ tɤ-ʁar jamar tu-ro ra, znde cho ɯ-tɯ-jaʁ kɯ-naχtɕɯɣ thɯ-kɤ-βzu ci ŋu, ɯ-χcɤl kɯ-spoʁ ŋu, ɯ-kɤχcɤl zɯ qapi tú-wɣ-rmbɯ ŋu, ɯ-χcɤl nɯ tɕu tshɤχɕaŋ, rloŋrta, nɯ ra pjɯ́-wɣ-sɤtsa ŋu, kɯrɯ kha ɯ-lɤftsɤz kɯ-me me.}\hspace{5pt}\pcmn{\lien{ⓔlɤftsɤz}{lɤftsɤz}是房背墙顶上左右两边的中间再修一堵和外墙一样厚的四方形小墙,中间有小洞,顶上要堆上白石头,在中间插上\lien{ⓔtshɤχɕaŋ}{tshɤχɕaŋ}、经幡等。所有藏房都有\lien{ⓔlɤftsɤz}{lɤftsɤz}。}\end{exemple}\étymologie{la.btsas}\end{entrée}

\begin{entrée}{lɤftɯɣ}{}{ⓔlɤftɯɣ} 
\classe{n} 
\begin{définition}\pfra{humus}\end{définition}
\begin{définition}\pcmn{腐殖土}\end{définition}\end{entrée}

\begin{entrée}{lɤɣ}{}{ⓔlɤɣ} 
\classe{vl} \paradigme{dir}{nɯ-}
\begin{définition}\pfra{faire paître les animaux}\end{définition}
\begin{définition}\pcmn{放牧}\end{définition}
\begin{exemple}\pjya{a-mu kɯ ji-fsapaʁ ra na-lɤɣ}\hspace{5pt}\pcmn{我母亲把牲畜带去放了}\end{exemple}
\begin{exemple}\pjya{fsapaʁ ra ɕ-pɯ-laɣ-a}\hspace{5pt}\pcmn{我把牲畜带去放了}\end{exemple}\end{entrée}

\begin{entrée}{lɤjmu}{}{ⓔlɤjmu} 
\classe{n} 
\begin{définition}\pfra{servante}\end{définition}
\begin{définition}\pcmn{女仆人}\end{définition}\end{entrée}

\begin{entrée}{lɤn}{}{ⓔlɤn} 
\classe{vi}  
\grammaire{autoben} \paradigme{dir}{pɯ-}\paradigme{dir}{pɯ-}
\begin{définition}\pfra{avoir la responsabilité de, c'est la faute de...}\end{définition}
\begin{définition}\pcmn{要为某事负责任;这件事怪……}\end{définition}
\begin{exemple}\pjya{nɯ ma-tɤ-tɯ-ste nɯ-sɯso-t-a ri mɯ́j-tɯ-khɯ tɕe nɤʑo tɯ-lɤn}\hspace{5pt}\pcmn{我想了“你不要那样做”,你没有听,都怪你了}\end{exemple}
\begin{exemple}\pjya{tɤ-rɟit mɤ-kɯ-pe nɯ phama lɤn}\hspace{5pt}\pcmn{孩子不成材怪父母}\end{exemple}
\begin{exemple}\pjya{laʁtɕha ɲo-me tɕe, nɤʑo pɯ-tɯ-lɤn}\hspace{5pt}\pcmn{东西丢了,你有责任}\end{exemple}
\begin{exemple}\pjya{tɤ-pɤtso mɤ-kɯ-khɯ nɯ chɯ-kɤ-sɯɣli ɲɯ-lɤn}\hspace{5pt}\pcmn{小孩子不听话,就是因为平时惯了他}\end{exemple}
\begin{sous-entrée}{nɯlɤn}{ⓔlɤnⓝnɯlɤn} 
\classe{vi} \end{sous-entrée}

\begin{définition}\pfra{n'avoir à s'en prendre qu'à soi-même}\end{définition}
\begin{définition}\pcmn{自食其果}\end{définition}
\begin{exemple}\pjya{ɯʑo pjɤ-nɯlɤn ɕti}\hspace{5pt}\pcmn{他自食其果}\end{exemple}
\begin{exemple}\pjya{nɤʑo mɯ́j-tɯ-nɯlɤn}\hspace{5pt}\pcmn{不是你的错}\end{exemple}\end{entrée}

\begin{entrée}{lɤntsa}{}{ⓔlɤntsa} 
\classe{n} 
\begin{définition}\pfra{un motif bouddhique}\end{définition}
\begin{définition}\pcmn{佛教的图纹}\end{définition}\end{entrée}

\begin{entrée}{lɤŋu}{}{ⓔlɤŋu} 
\classe{num} 
\begin{définition}\pfra{à peu près cinq}\end{définition}
\begin{définition}\pcmn{大概五个}\end{définition}\relationsémantique{参考}{\lien{ⓔkɯmŋu}{kɯmŋu}}\end{entrée}

\begin{entrée}{lɤŋɤtʂɤɣ}{}{ⓔlɤŋɤtʂɤɣ} 
\classe{num} 
\begin{définition}\pfra{cinq ou six}\end{définition}
\begin{définition}\pcmn{五六个}\end{définition}
\begin{exemple}\pjya{lɤŋɤtʂɤ-sŋi}\hspace{5pt}\pcmn{五六天}\end{exemple}\relationsémantique{参考}{\lien{ⓔkɯmŋu}{kɯmŋu}}\relationsémantique{参考}{\lien{ⓔkɯtʂɤɣ}{kɯtʂɤɣ}}\end{entrée}

\begin{entrée}{lɤpɯɣ}{}{ⓔlɤpɯɣ} 
\classe{n} 
\begin{définition}\pfra{radis}\end{définition}
\begin{définition}\pcmn{萝卜}\end{définition}\étymologie{la.pʰug}\end{entrée}

\begin{entrée}{lɤqhɤtɤmbɤt}{}{ⓔlɤqhɤtɤmbɤt} 
\classe{n} 
\begin{définition}\pfra{(distance de) plusieurs montagnes}\end{définition}
\begin{définition}\pcmn{几座山 (的距离)}\end{définition}
\begin{exemple}\pjya{ji-pɤrthɤβ lɤqhɤtɤmbɤt tu}\hspace{5pt}\pcmn{我们之间相隔着几座山}\end{exemple}\end{entrée}

\begin{entrée}{lɤsɤr}{}{ⓔlɤsɤr} 
\classe{n} 
\begin{définition}\pfra{nouvel an}\end{définition}
\begin{définition}\pcmn{新年}\end{définition}
\begin{exemple}\pjya{lɤsɤr a-pɯ-tɯ-scit-nɯ}\hspace{5pt}\pcmn{新年快乐}\end{exemple}
\begin{exemple}\pjya{lɤsɤr βzaŋ}\hspace{5pt}\pcmn{新年快乐}\end{exemple}
\begin{exemple}\pjya{@eryue @sanhao tɕe ji-lɤsɤr ɲɯ-ŋu, kɯrɯlɤsɤr}\hspace{5pt}\pcmn{我们的新年是二月三号}\end{exemple}\étymologie{lo.gsar}\end{entrée}

\begin{entrée}{lɤsɤr cito}{}{ⓔlɤsɤr cito} 
\classe{n} 
\begin{définition}\pfra{premier jour de l'année}\end{définition}
\begin{définition}\pcmn{年初一}\end{définition}\end{entrée}

\begin{entrée}{lɤskɤr tɕhɯʁɲiz}{}{ⓔlɤskɤr tɕhɯʁɲiz} 
\classe{n} 
\begin{définition}\pfra{les douze signes astrologiques}\end{définition}
\begin{définition}\pcmn{十二生肖}\end{définition}\end{entrée}

\begin{entrée}{lɤt}{₁}{ⓔlɤtⓗ1} 
\classe{vt} \sens{1}\paradigme{dir}{\_}
\begin{définition}\pfra{jeter}\end{définition}
\begin{définition}\pcmn{扔}\end{définition}
\begin{exemple}\pjya{ɯ-mci to-lɤt}\hspace{5pt}\pcmn{他吐了口水}\end{exemple}\sens{2}\paradigme{dir}{tɤ-}
\begin{définition}\pfra{tirer}\end{définition}
\begin{définition}\pcmn{射}\end{définition}
\begin{exemple}\pjya{ɕɤmɯɣdɯ to-lɤt}\hspace{5pt}\pcmn{他开枪了}\end{exemple}
\begin{exemple}\pjya{tɯdi to-lɤt}\hspace{5pt}\pcmn{他射了箭}\end{exemple}\sens{3}\paradigme{dir}{tɤ-}\paradigme{dir}{pɯ-}
\begin{définition}\pfra{frapper}\end{définition}
\begin{définition}\pcmn{打}\end{définition}
\begin{exemple}\pjya{ɯ-ku zɯ tɤŋkhɯt to-lɤt}\hspace{5pt}\pcmn{在他头上打了一拳}\end{exemple}
\begin{exemple}\pjya{tɯ-mɯrtsɯɣ ci to-lɤt}\hspace{5pt}\pcmn{捏了一下}\end{exemple}
\begin{exemple}\pjya{tɯ-qartsɯ ta-lɤt}\hspace{5pt}\pcmn{(马)踢了一脚}\end{exemple}
\begin{exemple}\pjya{ɯ-mke pjɤ-lɤt}\hspace{5pt}\pcmn{他给他割了喉}\end{exemple}\sens{4}\paradigme{dir}{pɯ-}\paradigme{dir}{kɤ-}
\begin{définition}\pfra{mettre, ajouter}\end{définition}
\begin{définition}\pcmn{放(进),加;倒(茶)}\end{définition}
\begin{exemple}\pjya{tsha pjɤ-lɤt, ko-lɤt}\hspace{5pt}\pcmn{放了盐}\end{exemple}
\begin{exemple}\pjya{tɯ-ci pjɤ-lɤt}\hspace{5pt}\pcmn{洒了水}\end{exemple}
\begin{exemple}\pjya{tɯ-ci ko-lɤt}\hspace{5pt}\pcmn{(在菜汤里)加了水}\end{exemple}\sens{5}\paradigme{dir}{tɤ-}
\begin{définition}\pfra{appliquer}\end{définition}
\begin{définition}\pcmn{涂}\end{définition}
\begin{exemple}\pjya{tɤ-mthɯm ɯ-taʁ tsha to-lɤt (=to-mar)}\hspace{5pt}\pcmn{把盐擦在肉上(腌制腊肉的方法)}\end{exemple}\relationsémantique{同义词}{\lien{ⓔmar}{mar}}\sens{6}
\begin{définition}\pfra{utiliser}\end{définition}
\begin{définition}\pcmn{用(工具)}\end{définition}
\begin{exemple}\pjya{mkhɯrlu jo-lɤt}\hspace{5pt}\pcmn{他开了车}\end{exemple}
\begin{exemple}\pjya{mkhɯrlu ɯ-kɯ-lɤt}\hspace{5pt}\pcmn{驾驶员}\end{exemple}
\begin{exemple}\pjya{tɤ-mtsɯ to-lɤt}\hspace{5pt}\pcmn{他扣了扣子}\end{exemple}
\begin{exemple}\pjya{tɤ-mtɯ ko-lɤt, cho-lɤt}\hspace{5pt}\pcmn{他打了(个)结}\end{exemple}
\begin{exemple}\pjya{mbro ɯ-jme nɯ tɤ-mtɯ χsɯm tha-lɤt}\hspace{5pt}\pcmn{在马尾巴上打了三个结}\end{exemple}
\begin{exemple}\pjya{taqaβ to-lɤt}\hspace{5pt}\pcmn{他用针扎了}\end{exemple}
\begin{exemple}\pjya{a-@dianhua ɯ-kɯ-lɤt ci ɣɤʑu}\hspace{5pt}\pcmn{有人打电话给我}\end{exemple}
\begin{exemple}\pjya{nɤ-@dianhua ɲɯ-lat-a}\hspace{5pt}\pcmn{我会给你打电话}\end{exemple}
\begin{exemple}\pjya{pɤjkhu ja-lɤt me}\hspace{5pt}\pcmn{他还没有打(电话)}\end{exemple}
\begin{exemple}\pjya{qraʁ thɯ-lat-a}\hspace{5pt}\pcmn{我铸造了犁铧}\end{exemple}
\begin{exemple}\pjya{sɤcɯ pjɤ-lɤt-ndʑi}\hspace{5pt}\pcmn{他们俩锁了门}\end{exemple}
\begin{exemple}\pjya{tɕhaʁla zɯ tɯmbri ci kɤ-lat-a (kɤ-mtshi-t-a, kɤ-rɤɕi-t-a)}\hspace{5pt}\pcmn{我在院子里拉了一根绳子(晒衣服)}\end{exemple}\sens{7}\paradigme{dir}{kɤ-}
\begin{définition}\pfra{tomber (pluie, neige etc)}\end{définition}
\begin{définition}\pcmn{下(雨、雪等)}\end{définition}
\begin{exemple}\pjya{tɯ-mɯ ko-lɤt}\hspace{5pt}\pcmn{下雨了}\end{exemple}
\begin{exemple}\pjya{tɯ-mɯ ɲɯ-ɤsɯ-lɤt}\hspace{5pt}\pcmn{正在下雨}\end{exemple}
\begin{exemple}\pjya{tɤjpa ko-lɤt}\hspace{5pt}\pcmn{下雪了}\end{exemple}
\begin{exemple}\pjya{sɤrwa cho-lɤt, tha-lɤt}\hspace{5pt}\pcmn{下了冰雹}\end{exemple}\sens{8}\paradigme{dir}{thɯ-}
\begin{définition}\pfra{mettre bas}\end{définition}
\begin{définition}\pcmn{生崽;开(花)}\end{définition}
\begin{exemple}\pjya{paχtsa chɤ-lɤt}\hspace{5pt}\pcmn{下猪崽}\end{exemple}
\begin{exemple}\pjya{mɯntoʁ ɲɤ-lɤt (=ɲɤ-rɯmɯntoʁ)}\hspace{5pt}\pcmn{开花}\end{exemple}\relationsémantique{同义词}{\lien{ⓔrɯmɯntoʁ}{rɯmɯntoʁ}}\sens{9}\paradigme{dir}{nɯ-}
\begin{définition}\pfra{relâcher}\end{définition}
\begin{définition}\pcmn{释放}\end{définition}
\begin{exemple}\pjya{kɯ-nɯkhrɯm nɯ ra ɲo-lɤt}\hspace{5pt}\pcmn{他把囚犯释放了}\end{exemple}
\begin{exemple}\pjya{a-@fangjia lɤt-nɯ}\hspace{5pt}\pcmn{他们要让我放假}\end{exemple}
\begin{exemple}\pjya{ɕɯntɕhi ʁnɯ-sŋi ɲɤ-lɤt-nɯ}\hspace{5pt}\pcmn{他们放了两天假}\end{exemple}\sens{10}
\begin{définition}\pfra{ramener}\end{définition}
\begin{définition}\pcmn{送回}\end{définition}
\begin{exemple}\pjya{kha mɤɕtʂa ɣɯ-jɤ́-wɣ-lat-a}\hspace{5pt}\pcmn{他把我送回家了}\end{exemple}\relationsémantique{参考}{\lien{ⓔnɤscɤlɤt}{nɤscɤlɤt}}\sens{11}
\begin{définition}\pfra{verbe léger}\end{définition}
\begin{définition}\pcmn{助动词}\end{définition}
\begin{exemple}\pjya{aʑo rɤɣo ci pɯ-lat-a}\hspace{5pt}\pcmn{我弹奏了音乐}\end{exemple}\relationsémantique{同义词}{\lien{ⓔɣɤjɯ}{ɣɤjɯ}}\relationsémantique{参考}{\lien{ⓔchɤlɤnnɤ}{chɤlɤnnɤ}}\relationsémantique{参考}{\lien{ⓔɣɤlɤt}{ɣɤlɤt}}\relationsémantique{参考}{\lien{ⓔrɤlɤt}{rɤlɤt}}\relationsémantique{参考}{\lien{ⓔjaqhɤrŋgɤβ,lɤt}{jaqhɤrŋgɤβ,lɤt}}\relationsémantique{参考}{\lien{ⓔtaʁmbra,lɤt}{taʁmbra,lɤt}}\relationsémantique{参考}{\lien{ⓔtɤlɟɣo,lɤt}{tɤlɟɣo,lɤt}}\relationsémantique{参考}{\lien{ⓔtɤqɤt,lɤt}{tɤqɤt,lɤt}}\relationsémantique{参考}{\lien{ⓔtɯ-sɯsoⓝtɯ-sɯso,lɤt}{tɯ-sɯso,lɤt}}\relationsémantique{参考}{\lien{ⓔrpuⓝɯ-rpu,lɤt}{ɯ-rpu,lɤt}}\relationsémantique{参考}{\lien{ⓔtɕhɯⓝɯ-tɕhɯ,lɤt}{ɯ-tɕhɯ,lɤt}}
\begin{sous-entrée}{alɤt}{ⓔlɤtⓗ1ⓢ11ⓝalɤt} 
\classe{vi}  
\grammaire{pass} 
\begin{exemple}\pjya{kɯm sɤcɯ mɤ-alɤt}\hspace{5pt}\pcmn{门没有锁上}\end{exemple}\relationsémantique{参考}{\lien{ⓔsɤlɤt}{sɤlɤt}}\relationsémantique{参考}{\lien{ⓔakɤlɤt}{akɤlɤt}}\end{sous-entrée}

\begin{sous-entrée}{sɯlɤt}{ⓔlɤtⓗ1ⓢ11ⓝsɯlɤt} 
\classe{vt}  
\grammaire{caus} 
\begin{exemple}\pjya{cha tɯ-@beibei to-sɯlɤt (=to-sɯrku)}\hspace{5pt}\pcmn{他请(服务员)给自己倒一杯酒}\end{exemple}\end{sous-entrée}

\end{entrée}

\begin{entrée}{lɤtaŋ}{}{ⓔlɤtaŋ} 
\classe{n} 
\begin{définition}\pfra{conscience}\end{définition}
\begin{définition}\pcmn{良心}\end{définition}
\begin{exemple}\pjya{nɤ-lɤtaŋ ɯ-tɯ-me nɯ!}\hspace{5pt}\pcmn{你真没有良心!}\end{exemple}\end{entrée}

\begin{entrée}{lɤtɕhom}{}{ⓔlɤtɕhom} 
\classe{n} 
\begin{définition}\pfra{baratte}\end{définition}
\begin{définition}\pcmn{打酥油茶的木桶}\end{définition}\relationsémantique{同义词}{\lien{ⓔtɤlɤndʑu}{tɤlɤndʑu}}\end{entrée}

\begin{entrée}{lɤzŋɤn}{}{ⓔlɤzŋɤn} 
\classe{n} 
\begin{définition}\pfra{mauvaise chance}\end{définition}
\begin{définition}\pcmn{霉运;晦气}\end{définition}\étymologie{las.ŋan}\end{entrée}

\begin{entrée}{lɤzŋɤntɕɤn}{}{ⓔlɤzŋɤntɕɤn} 
\classe{n} 
\begin{définition}\pfra{malchanceux}\end{définition}
\begin{définition}\pcmn{最倒霉的人}\end{définition}\étymologie{las.ŋan.tɕan}\end{entrée}

\begin{entrée}{lbjɯlbjɯɣ}{}{ⓔlbjɯlbjɯɣ} 
\classe{idph.2} 
\begin{définition}\pfra{qui pend en grand nombre, mou}\end{définition}
\begin{définition}\pcmn{形容又多又柔软的样子,往下垂吊}\end{définition}
\begin{exemple}\pjya{ʑmbri ɣɯ ɯ-rtaʁ ɲɯ-mpɯ tɕe ɯ-jwaʁ ɲɯ-dɤn pjɯ-ɴqoʁ kɯ-fse tɕe, lbjɯlbjɯɣ ʑo ɲɯ-pa}\hspace{5pt}\pcmn{柳树的枝桠很软,叶子很多,往下垂的样子}\end{exemple}\relationsémantique{同义词}{\lien{ⓔbjɯbjɯɣ}{bjɯbjɯɣ}}\end{entrée}

\begin{entrée}{lchɤlchɤt}{}{ⓔlchɤlchɤt} 
\classe{idph.2} \sens{1}
\begin{définition}\pfra{petit (homme)}\end{définition}
\begin{définition}\pcmn{形容矮墩墩的样子(人)}\end{définition}
\begin{exemple}\pjya{tɯrme ɯ-phoŋbu kɯ-mbɤr ci lchɤlchɤt ɲɯ-ŋu}\hspace{5pt}\pcmn{那个人矮墩墩的}\end{exemple}
\begin{exemple}\pjya{ɯ-phoŋbu mɤ-kɯ-mbro lchɤlchɤt ci ɲɯ-ŋu}\hspace{5pt}\pcmn{他矮墩墩的}\end{exemple}\sens{2}
\begin{définition}\pfra{non rempli}\end{définition}
\begin{définition}\pcmn{形容没有满的样子(口袋)}\end{définition}
\begin{exemple}\pjya{tɤ-fkɯm lchɤlchɤt ci ɲɯ-ŋu}\hspace{5pt}\pcmn{口袋不满}\end{exemple}
\begin{exemple}\pjya{lchɤlchɤt ci ɲɤ-rku}\hspace{5pt}\pcmn{没有装满}\end{exemple}\relationsémantique{参考}{\lien{ⓔlchɯɣlchɯɣ}{lchɯɣlchɯɣ}}\end{entrée}

\begin{entrée}{lchɣaʁlchɣaʁ}{}{ⓔlchɣaʁlchɣaʁ} 
\classe{idph.2} 
\begin{définition}\pfra{souple, agréable à porter}\end{définition}
\begin{définition}\pcmn{形容柔软(衣服、皮子)的触感}\end{définition}
\begin{exemple}\pjya{tɯ-ndʐi lchɣaʁlchɣaʁ ci ɲɯ-ŋu}\hspace{5pt}\pcmn{是很柔软的皮子}\end{exemple}
\begin{exemple}\pjya{ɯ-ŋga lchɣaʁlchɣaʁ ci ɲɯ-ŋu}\hspace{5pt}\pcmn{他的衣服很柔软}\end{exemple}
\begin{exemple}\pjya{lchɣaʁlchɣaʁ ɲɯ-nɯɣɯŋga}\hspace{5pt}\pcmn{很柔软,很好穿}\end{exemple}\end{entrée}

\begin{entrée}{lchɯɣlchɯɣ}{}{ⓔlchɯɣlchɯɣ} 
\classe{idph.2} 
\begin{définition}\pfra{pas rempli}\end{définition}
\begin{définition}\pcmn{没有装满}\end{définition}
\begin{exemple}\pjya{tɤ-fkɯm ɯ-ŋgɯ zɯ stoʁ tɯɣnɤsqɯ-tɯrpa ci ma lchɯɣlchɯɣ maŋe}\hspace{5pt}\pcmn{口袋里只有二十斤胡豆,没有装满}\end{exemple}\end{entrée}

\begin{entrée}{lchɯmlchɯm}{}{ⓔlchɯmlchɯm} 
\classe{idph.2} 
\begin{définition}\pfra{le niveau de l'eau qui baisse lentement}\end{définition}
\begin{définition}\pcmn{形容水位慢慢地降下去的样子}\end{définition}
\begin{exemple}\pjya{mtshu lchɯmlchɯm ʑo pjɤ-schɤt}\hspace{5pt}\pcmn{湖面的水慢慢地降下去}\end{exemple}\end{entrée}

\begin{entrée}{lcɯɣlcɯɣ}{}{ⓔlcɯɣlcɯɣ} 
\classe{idph.2} 
\begin{définition}\pfra{trempé}\end{définition}
\begin{définition}\pcmn{湿透}\end{définition}
\begin{exemple}\pjya{kó-wɣ-sphjaʁ lcɯɣlcɯɣ}\hspace{5pt}\pcmn{湿透了}\end{exemple}
\begin{exemple}\pjya{a-ŋga ra nɯ-aci lcɯɣlcɯɣ ʑo}\hspace{5pt}\pcmn{我的衣服变湿了}\end{exemple}
\begin{exemple}\pjya{a-ŋga ra lcɯɣlcɯɣ ʑo nɯ-pa}\hspace{5pt}\pcmn{我的衣服变湿了}\end{exemple}\end{entrée}

\begin{entrée}{ldɯɣi}{}{ⓔldɯɣi} 
\classe{n} 
\begin{définition}\pfra{bharal (ovis ammon)}\end{définition}
\begin{définition}\pcmn{盘羊}\end{définition}
\begin{exemple}\pjya{ldɯɣi nɯ zgoku stu kɯ-mbro rɯtɕhɤβ ɯ-ŋgɯ zɯ ku-rɤʑi ɲɯ-ŋu, kɯ-dɤn kɯ ɣurʑa tu ɲɯ-ŋgrɤl, kɯ-rkɯn kɯ ʁnɯz χsɯm ku-rɤʑi tu ɲɯ-ŋgrɤl, jɤ-ari-nɯ tɕe stu kɯ-mɤku nɯ ju-ɕe ɯ-qhɯ-qhu zɯ kɯmdi ju-ɕe-nɯ, ɯʑo qaʑo cho kɯ-naχtɕɯɣ ŋu, zgoku-rɤʑi ma co ɯ-ŋgɯ zɯ ku-rɤʑi mɯ́j-ŋgrɤl, ɯ-rme nɯ caʁɕɣɤz kɯ-tu kɯ-wɣrum ɲɯ-ŋu.}\hspace{5pt}\pcmn{盘羊生活在高山上,没有草木的岩石上。多的有一百只左右,少的只有两三只一起生活,它们走动时所有的盘羊跟在一个领头的后面,和绵羊一样。只出现在高山上,不会下河坝来。有很多白色的粗毛。}\end{exemple}\end{entrée}

\begin{entrée}{ldɯɣldɯɣ}{}{ⓔldɯɣldɯɣ} 
\classe{idph.2} 
\begin{définition}\pfra{ciel sombre, rempli de nuages}\end{définition}
\begin{définition}\pcmn{形容浓云密布的样子}\end{définition}
\begin{exemple}\pjya{tɯ-mɯ ldɯɣldɯɣ ʑo ɲɯ-pa}\hspace{5pt}\pcmn{天上的云密密麻麻的}\end{exemple}
\begin{exemple}\pjya{tɤ-lu tʂha ldɯɣldɯɣ ɲɯ-pa}\hspace{5pt}\pcmn{奶茶好喝}\end{exemple}
\begin{exemple}\pjya{jisŋi ɲɯ-lɯβ ldɯɣldɯɣ ʑo}\hspace{5pt}\pcmn{今天天色很阴的样子}\end{exemple}\relationsémantique{参考}{\lien{ⓔdʐɯɣdʐɯɣ}{dʐɯɣdʐɯɣ}}\end{entrée}

\begin{entrée}{ldɯɣɯ}{}{ⓔldɯɣɯ} 
\classe{n} 
\begin{définition}\pfra{couteau courbé}\end{définition}
\begin{définition}\pcmn{弯刀}\end{définition}\end{entrée}

\begin{entrée}{ldɯm}{}{ⓔldɯm} 
\classe{vs} \paradigme{dir}{tɤ-}
\begin{définition}\pfra{sérieux}\end{définition}
\begin{définition}\pcmn{稳重;谨慎}\end{définition}
\begin{exemple}\pjya{ki tɯrme ki kɯ-ldɯm ci ɲɯ-ŋu}\hspace{5pt}\pcmn{这个人很稳重}\end{exemple}
\begin{exemple}\pjya{tɤ-pɤtso ɲɯ-ldɯm}\hspace{5pt}\pcmn{孩子很听话}\end{exemple}\end{entrée}

\begin{entrée}{ldzɣɤβldzɣɤβ}{}{ⓔldzɣɤβldzɣɤβ} 
\classe{idph.2} 
\begin{définition}\pfra{en désordre}\end{définition}
\begin{définition}\pcmn{凌乱(衣服、线)}\end{définition}
\begin{exemple}\pjya{tɤ-ri ɲɤ-k-ɤɬɯt-ci ldzɣɤβldzɣɤβ ʑo}\hspace{5pt}\pcmn{线凌乱了}\end{exemple}
\begin{exemple}\pjya{ɯ-ŋga ɲɯ-ɤdrɤt ldzɣɤβldzɣɤβ ʑo}\hspace{5pt}\pcmn{他衣服放得很凌乱}\end{exemple}
\begin{sous-entrée}{ldzɣɤβnɤldzɣɤβ}{ⓔldzɣɤβldzɣɤβⓝldzɣɤβnɤldzɣɤβ} 
\classe{idph.3} 
\begin{exemple}\pjya{ɯ-ŋga ɯ-mɤ-tɯ-pe ldzɣɤβnɤldzɣɤβ ʑo kɤ-ɕqhlɤt}\hspace{5pt}\pcmn{他衣衫褴褛地过去了}\end{exemple}\end{sous-entrée}

\end{entrée}

\begin{entrée}{ldʑaŋkɯ}{}{ⓔldʑaŋkɯ} 
\classe{n} 
\begin{définition}\pfra{vert}\end{définition}
\begin{définition}\pcmn{绿(布料、线)}\end{définition}\relationsémantique{参考}{\lien{ⓔarɯldʑaŋkɯ}{arɯldʑaŋkɯ}}\étymologie{ldʑaŋ.kʰu}\end{entrée}

\begin{entrée}{ldʑaŋnaʁ}{}{ⓔldʑaŋnaʁ} 
\classe{n} 
\begin{définition}\pfra{vert foncé}\end{définition}
\begin{définition}\pcmn{深绿色}\end{définition}\étymologie{ldʑaŋ.nag}\end{entrée}

\begin{entrée}{ldʑoʁ}{}{ⓔldʑoʁ} 
\classe{vi} \paradigme{dir}{tɤ-}\paradigme{dir}{nɯ-}
\begin{définition}\pfra{parvenir à complétion}\end{définition}
\begin{définition}\pcmn{完成}\end{définition}
\begin{exemple}\pjya{kɤ-nɤma ra tɤ-ldʑoʁ}\hspace{5pt}\pcmn{事情完成了}\end{exemple}
\begin{exemple}\pjya{kha ta-ma kɯ-ldʑoʁ me}\hspace{5pt}\pcmn{家务是做不完的}\end{exemple}
\begin{sous-entrée}{sɯldʑoʁ}{ⓔldʑoʁⓝsɯldʑoʁ} 
\classe{vt} 
\begin{définition}\pfra{mener à complétion}\end{définition}
\begin{définition}\pcmn{完成}\end{définition}
\begin{exemple}\pjya{ta-ma kɤ-sɯldʑoʁ mɯ́j-khɯ ma ɲɯ-dɤn}\hspace{5pt}\pcmn{事情不能一下完成,因为头绪多}\end{exemple}\end{sous-entrée}

\end{entrée}

\begin{entrée}{ldʑɯβ}{}{ⓔldʑɯβ} 
\classe{vt} 
\begin{définition}\pfra{pouvoir aider}\end{définition}
\begin{définition}\pcmn{帮得了}\end{définition}
\begin{exemple}\pjya{aʑo kɯ nɤʑo mɤ-ta-ldʑɯβ}\hspace{5pt}\pcmn{(我是弱者),没有能力帮你}\end{exemple}
\begin{exemple}\pjya{xɕɤndʑu kɯ zdoŋbu mɤ-ldʑɯβ}\hspace{5pt}\pcmn{弱者帮不了强者}\end{exemple}\end{entrée}

\begin{entrée}{ldʑɯŋldʑɯŋ}{}{ⓔldʑɯŋldʑɯŋ} 
\classe{idph.2} 
\begin{définition}\pfra{bleu (ciel)}\end{définition}
\begin{définition}\pcmn{形容天的蓝色}\end{définition}
\begin{exemple}\pjya{tɯ-mɯ ɲɯ-ɤrŋi ldʑɯŋldʑɯŋ ʑo}\hspace{5pt}\pcmn{天空蓝蓝的样子}\end{exemple}\étymologie{ldʑaŋ}\end{entrée}

\begin{entrée}{ldʑɯz}{}{ⓔldʑɯz} 
\classe{vs} \paradigme{dir}{nɯ-}\paradigme{dir}{tɤ-}
\begin{définition}\pfra{flexible (branche)}\end{définition}
\begin{définition}\pcmn{有韧性‘柔韧}\end{définition}
\begin{exemple}\pjya{ki ʑmbri ɲɯ-ldʑɯz}\hspace{5pt}\pcmn{这棵杨树有韧性}\end{exemple}\end{entrée}

\begin{entrée}{ldʐaŋldʐaŋ}{}{ⓔldʐaŋldʐaŋ} 
\classe{idph.2} 
\begin{définition}\pfra{pendu}\end{définition}
\begin{définition}\pcmn{吊着(很大的东西)}\end{définition}
\begin{exemple}\pjya{fsapaʁ to-ʑɣɤtshi tɕe, ldʐaŋldʐaŋ ɲɯ-ɴqoʁ}\hspace{5pt}\pcmn{牲畜(不小心)把自己勒死了,在那里吊着}\end{exemple}\relationsémantique{参考}{\lien{ⓔɕtʂaŋɕtʂaŋ}{ɕtʂaŋɕtʂaŋ}}
\begin{sous-entrée}{ldʐaŋnɤlaŋ}{ⓔldʐaŋldʐaŋⓝldʐaŋnɤlaŋ} 
\classe{idph.4} 
\begin{exemple}\pjya{ldʐaŋnɤlaŋ ɲɯ-ʑɣɤstu}\hspace{5pt}\pcmn{在甩来甩去}\end{exemple}\end{sous-entrée}

\begin{sous-entrée}{ɣɤldʐaŋlaŋ}{ⓔldʐaŋldʐaŋⓝɣɤldʐaŋlaŋ} 
\classe{vi} 
\begin{exemple}\pjya{ɲɯ-ɣɤldʐaŋlaŋ}\hspace{5pt}\pcmn{在甩来甩去}\end{exemple}\end{sous-entrée}

\end{entrée}

\begin{entrée}{ldʐɤβldʐɤβ}{}{ⓔldʐɤβldʐɤβ} 
\classe{idph.2} 
\begin{définition}\pfra{en désordre (fils pendus)}\end{définition}
\begin{définition}\pcmn{形容挂着的线、布条等凌乱的样子}\end{définition}
\begin{exemple}\pjya{ɯ-ŋga chɤ-ɴɢraʁ ldʐɤβldʐɤβ ʑo ri, ɲɯ-ɤsɯ-ŋga}\hspace{5pt}\pcmn{他的衣服破破烂烂的,他还是穿着}\end{exemple}\end{entrée}

\begin{entrée}{ldʐɤpɤldʐɤle}{}{ⓔldʐɤpɤldʐɤle} 
\classe{n} 
\begin{définition}\pfra{habits de mauvaise qualité, abîmés}\end{définition}
\begin{définition}\pcmn{简陋的衣服}\end{définition}
\begin{exemple}\pjya{tɯ-ŋga ldʐɤpɤldʐɤle ʑo tu-nɯ-ŋge ɕti}\hspace{5pt}\pcmn{他不讲究穿着}\end{exemple}\relationsémantique{参考}{\lien{ⓔcɤpɤcrɤle}{cɤpɤcrɤle}}\end{entrée}

\begin{entrée}{lɣa}{}{ⓔlɣa} 
\classe{vt} \paradigme{dir}{lɤ-}\paradigme{dir}{tɤ-}
\begin{définition}\pfra{creuser}\end{définition}
\begin{définition}\pcmn{挖}\end{définition}
\begin{exemple}\pjya{ŋgɤm nɯ lɤ-lɣa-t-a}\hspace{5pt}\pcmn{我挖了土坡}\end{exemple}
\begin{exemple}\pjya{sɤtɕha tɤ-lɣa-t-a}\hspace{5pt}\pcmn{我挖了地}\end{exemple}\end{entrée}

\begin{entrée}{lɣɤβlɣɤβ}{}{ⓔlɣɤβlɣɤβ} 
\classe{idph.2} 
\begin{définition}\pfra{épais (vêtements)}\end{définition}
\begin{définition}\pcmn{沉重;厚实(衣服)}\end{définition}
\begin{exemple}\pjya{a@pugai lɣɤβlɣɤβ ʑo ɲɯ-pa}\hspace{5pt}\pcmn{我的被子很沉重}\end{exemple}
\begin{exemple}\pjya{lɣɤβlɣɤβ ʑo pɯ́-wɣ-ɲcar-a}\hspace{5pt}\pcmn{他重重地压着我}\end{exemple}
\begin{exemple}\pjya{ɲɯ-rʑi lɣɤβlɣɤβ ʑo}\hspace{5pt}\pcmn{比较重}\end{exemple}
\begin{sous-entrée}{lɣɤβnɤlɣɤβ}{ⓔlɣɤβlɣɤβⓝlɣɤβnɤlɣɤβ} 
\classe{idph.3} 
\begin{exemple}\pjya{lɣɤβnɤlɣɤβ pɯ-ŋke-a}\hspace{5pt}\pcmn{我穿着笨重的衣服走路}\end{exemple}\relationsémantique{参考}{\lien{ⓔlxɤβlxɤβ}{lxɤβlxɤβ}}\relationsémantique{参考}{\lien{ⓔlxɯlxi}{lxɯlxi}}\end{sous-entrée}

\end{entrée}

\begin{entrée}{li}{₁}{ⓔliⓗ1} 
\classe{adv} 
\begin{définition}\pfra{encore}\end{définition}
\begin{définition}\pcmn{再}\end{définition}
\begin{exemple}\pjya{li ci tɤ-ti}\hspace{5pt}\pcmn{你再说一遍}\end{exemple}
\begin{exemple}\pjya{li ci tshupa nɯ tɕhi rmi}\hspace{5pt}\pcmn{下一个村子叫什么名字?}\end{exemple}\end{entrée}

\begin{entrée}{li}{₂}{ⓔliⓗ2} 
\classe{n} 
\begin{définition}\pfra{cuivre}\end{définition}
\begin{définition}\pcmn{铜}\end{définition}\étymologie{li}\end{entrée}

\begin{entrée}{li}{₃}{ⓔliⓗ3} 
\classe{vs} \paradigme{dir}{thɯ-}\paradigme{dir}{thɯ-}
\begin{définition}\pfra{mal élevé, gâté}\end{définition}
\begin{définition}\pcmn{娇生惯养,被宠坏}\end{définition}
\begin{définition}\pfra{trop gâter}\end{définition}
\begin{définition}\pcmn{宠坏}\end{définition}
\begin{exemple}\pjya{cho-li}\hspace{5pt}\pcmn{他变得很任性}\end{exemple}
\begin{exemple}\pjya{jiɕqha ɯ-rɟit nɯ wuma ɲɯ-li}\hspace{5pt}\pcmn{他的孩子被宠坏了}\end{exemple}
\begin{exemple}\pjya{ɯ-pɤtso ɲɯ-sɯɣli}\hspace{5pt}\pcmn{他在惯着孩子}\end{exemple}
\begin{exemple}\pjya{thɯ-sɯɣli-t-a}\hspace{5pt}\pcmn{我惯养了他}\end{exemple}
\begin{exemple}\pjya{thɯ́-wɣ-sɯɣli-a}\hspace{5pt}\pcmn{他惯养了我}\end{exemple}
\begin{sous-entrée}{sɯɣli}{ⓔliⓗ3ⓝsɯɣli} 
\classe{vt}  
\grammaire{caus} \end{sous-entrée}

\begin{sous-entrée}{sɤsɯɣli}{ⓔliⓗ3ⓝsɤsɯɣli} 
\classe{vi}  
\grammaire{apass} 
\begin{définition}\pfra{gâter les enfants}\end{définition}
\begin{définition}\pcmn{惯养小孩}\end{définition}\end{sous-entrée}

\end{entrée}

\begin{entrée}{luj}{}{ⓔluj} 
\classe{vt} \paradigme{dir}{tɤ-}\paradigme{dir}{tɤ-}\paradigme{dir}{tɤ-}
\begin{définition}\pfra{recouvrir complètement la surface pour cacher}\end{définition}
\begin{définition}\pcmn{包起来;遮蔽;裹住}\end{définition}
\begin{définition}\pfra{couvrir (avec qqch)}\end{définition}
\begin{définition}\pcmn{遮住(用东西)}\end{définition}
\begin{définition}\pfra{être recouvert}\end{définition}
\begin{définition}\pcmn{被遮住,被裹起来}\end{définition}
\begin{exemple}\pjya{kɯki @huatong ki tú-wɣ-luj}\hspace{5pt}\pcmn{把这个话筒遮住}\end{exemple}
\begin{exemple}\pjya{kɯki ɯ-fkɯm tú-wɣ-βzu tú-wɣ-sɯluj}\hspace{5pt}\pcmn{要做个套子把这个东西掩盖起来}\end{exemple}
\begin{exemple}\pjya{tɯ-ŋga tɤ-nga-t-a tɕe tɤ́-wɣ-sɯluj-a}\hspace{5pt}\pcmn{我穿了衣服,把我遮住了}\end{exemple}
\begin{sous-entrée}{sɯluj}{ⓔlujⓝsɯluj} 
\classe{vt} \end{sous-entrée}

\begin{sous-entrée}{aluj}{ⓔlujⓝaluj} 
\classe{vi} \end{sous-entrée}

\end{entrée}

\begin{entrée}{ljɤɣljɤɣ}{}{ⓔljɤɣljɤɣ} 
\classe{idph.2} 
\begin{définition}\pfra{épais et long}\end{définition}
\begin{définition}\pcmn{形容粗而长,看上去蓬松,但摸起来有些硬的样子}\end{définition}
\begin{exemple}\pjya{paʁ chɤ-tshu ljɤɣljɤɣ ʑo}\hspace{5pt}\pcmn{猪变得又胖又长}\end{exemple}\end{entrée}

\begin{entrée}{ljulju}{}{ⓔljulju} 
\classe{idph.2} 
\begin{définition}\pfra{cylindrique}\end{définition}
\begin{définition}\pcmn{圆柱形}\end{définition}\relationsémantique{参考}{\lien{ⓔalɯlju}{alɯlju}}\end{entrée}

\begin{entrée}{lɟɣaʁ}{}{ⓔlɟɣaʁ} 
\classe{vt} \paradigme{dir}{pɯ-}\paradigme{dir}{\_}
\begin{définition}\pfra{étendre (un habit, une corde) sur un autre objet}\end{définition}
\begin{définition}\pcmn{搭上去(衣服、毛巾、绳子等)}\end{définition}
\begin{exemple}\pjya{nɯki tɯ-ŋga tɤ-lɟɣaʁ}\hspace{5pt}\pcmn{你把衣服搭上去}\end{exemple}
\begin{exemple}\pjya{pa-lɟɣaʁ}\hspace{5pt}\pcmn{他(把衣服)搭了}\end{exemple}
\begin{exemple}\pjya{nɯki tɯ-mbri nɯ kɤcu kɤ-lɟɣaʁ}\hspace{5pt}\pcmn{你把绳子搭在那里}\end{exemple}\end{entrée}

\begin{entrée}{lkɯɣ}{}{ⓔlkɯɣ} 
\classe{vi} \paradigme{dir}{kɤ-}
\begin{définition}\pfra{être ankylosé après avoir manqué d'exercice}\end{définition}
\begin{définition}\pcmn{因为缺乏锻炼,关节僵硬的感觉}\end{définition}
\begin{exemple}\pjya{ko-lkɯɣ-a}\hspace{5pt}\pcmn{我的关节都不灵}\end{exemple}
\begin{exemple}\pjya{nɤ-βri ra a-ʑ-nɯ-tɯ-zmɯnme tɕe a-mɤ-kɤ-tɯ-lkɯɣ}\hspace{5pt}\pcmn{你要锻炼一下身体,这样就不会有关节不灵的感觉}\end{exemple}\end{entrée}

\begin{entrée}{lmɤlmɤr}{}{ⓔlmɤlmɤr} 
\classe{idph.2} 
\begin{définition}\pfra{mou}\end{définition}
\begin{définition}\pcmn{形容很软,好像没有骨头的样子}\end{définition}
\begin{exemple}\pjya{mbro ɯ-mtɕhi lmɤlmɤr ʑo ɲɯ-pa}\hspace{5pt}\pcmn{马的嘴很软}\end{exemple}\end{entrée}

\begin{entrée}{lni}{}{ⓔlni} 
\classe{vi} \paradigme{dir}{nɯ-}
\begin{définition}\pfra{flétrir à cause de la chaleur}\end{définition}
\begin{définition}\pcmn{蔫}\end{définition}
\begin{exemple}\pjya{sɯjno tɤ-phɯt-a tɕe ɲo-lni}\hspace{5pt}\pcmn{我拔了草就蔫了}\end{exemple}
\begin{exemple}\pjya{razti ɲo-lni}\hspace{5pt}\pcmn{圆根蔫了}\end{exemple}\end{entrée}

\begin{entrée}{lɲɯɣlɲɯɣ}{}{ⓔlɲɯɣlɲɯɣ} 
\classe{idph.2} 
\begin{définition}\pfra{habillé de façon négligé}\end{définition}
\begin{définition}\pcmn{形容衣冠不整,(衣服)松,拖得很长的样子}\end{définition}
\begin{exemple}\pjya{kɤ́-ŋgɯ-ŋga lɲɯɣlɲɯɣ kɯ kɯ-rɤma nɤ-scɯʁzɯɣ maʁ}\hspace{5pt}\pcmn{你衣服穿得又松有长,看起来不像是干活的人}\end{exemple}\end{entrée}

\begin{entrée}{lŋaʁlŋaʁ}{}{ⓔlŋaʁlŋaʁ} 
\classe{idph.2} 
\begin{définition}\pfra{petit et mince}\end{définition}
\begin{définition}\pcmn{形容瘦而小,令人讨厌的样子}\end{définition}
\begin{exemple}\pjya{paʁtsa nɯ ɲɯ-ndɯβ-nɯ ʑo lŋaʁlŋaʁ ɕti}\hspace{5pt}\pcmn{猪崽子又小又瘦,令人讨厌}\end{exemple}
\begin{sous-entrée}{lŋaʁnɤlŋaʁ}{ⓔlŋaʁlŋaʁⓝlŋaʁnɤlŋaʁ} 
\classe{idph.3} 
\begin{exemple}\pjya{ɯ-rɟit nɯ mɯ́j-scit tɕe, lŋaʁnɤlŋaʁ ʑo chɯ-ɣɤwu ɲɯ-ŋu}\hspace{5pt}\pcmn{他的孩子不乖,一直在哭,令人讨厌}\end{exemple}\end{sous-entrée}

\end{entrée}

\begin{entrée}{lŋɤβnɤlŋɤβ}{}{ⓔlŋɤβnɤlŋɤβ} 
\classe{idph.3} 
\begin{définition}\pfra{qui a la bouche grande ouverte}\end{définition}
\begin{définition}\pcmn{形容嘴巴开得很大,耷拉着耳朵,不美观的样子}\end{définition}
\begin{exemple}\pjya{khɯna kɯ tɤ-mthɯm lŋɤβnɤlŋɤβ ʑo ɲɯ-ɤsɯ-ndza}\hspace{5pt}\pcmn{狗在大口大口地吃肉}\end{exemple}\relationsémantique{参考}{\lien{ⓔnɯlŋɤβ}{nɯlŋɤβ}}\end{entrée}

\begin{entrée}{lŋɤlŋɤt}{}{ⓔlŋɤlŋɤt} 
\classe{idph.2} 
\begin{définition}\pfra{beaucoup de (fruits) accrochés}\end{définition}
\begin{définition}\pcmn{很多,很大的东西(挂着)}\end{définition}
\begin{exemple}\pjya{ɯ-mat lŋɤlŋɤt ʑo ko-tshoʁ}\hspace{5pt}\pcmn{结了很多的果子}\end{exemple}
\begin{exemple}\pjya{ʁmɯrtsɯ kɯ ɯ-mat lŋɤlŋɤt ko-tshoʁ}\hspace{5pt}\pcmn{黑泡儿结了很多果子}\end{exemple}
\begin{exemple}\pjya{ɯ-taʁ lŋɤlŋɤt ʑo pjɤ-lɟɣaʁ}\hspace{5pt}\pcmn{搭在上面,显得很大}\end{exemple}
\begin{exemple}\pjya{tɯ-ŋga lŋɤlŋɤt ʑo to-ɕɯɴqoʁ}\hspace{5pt}\pcmn{他挂了很多衣服}\end{exemple}
\begin{sous-entrée}{lŋɤnɤlŋɤt}{ⓔlŋɤlŋɤtⓝlŋɤnɤlŋɤt} 
\classe{idph.3} 
\begin{exemple}\pjya{lŋɤnɤlŋɤt ɲɯ-nɯndzɯlŋɯz}\hspace{5pt}\pcmn{他在打瞌睡,头一点一点的}\end{exemple}\end{sous-entrée}

\end{entrée}

\begin{entrée}{lo/\variante{loβ}}{₂}{ⓔloⓗ2} 
\classe{part} 
\begin{définition}\pfra{d'accord}\end{définition}
\begin{définition}\pcmn{吧}\end{définition}
\begin{exemple}\pjya{nɤ-ŋga tɤ-ŋge ɲɯ-mna lo}\hspace{5pt}\pcmn{你把衣服穿上好吗?}\end{exemple}\relationsémantique{参考}{\lien{ⓔlotɕi}{lotɕi}}\end{entrée}

\begin{entrée}{lo}{₁}{ⓔloⓗ1} 
\classe{vi} \paradigme{dir}{tɤ-}
\begin{définition}\pfra{avoir l'immunité contre}\end{définition}
\begin{définition}\pcmn{有免疫能力}\end{définition}
\begin{exemple}\pjya{aʑo tɤ-mbrɯm tɤ-lo-a}\end{exemple}
\begin{exemple}\pjya{aʑo tɤ-mbrɯm tɤ-kɯ-lo ŋu-a}\hspace{5pt}\pcmn{我对麻子有免疫能力}\end{exemple}\end{entrée}

\begin{entrée}{lochu}{}{ⓔlochu} 
\classe{adv} 
\begin{définition}\pfra{en amont}\end{définition}
\begin{définition}\pcmn{在上游}\end{définition}\end{entrée}

\begin{entrée}{lonba}{}{ⓔlonba} 
\classe{adv} 
\begin{définition}\pfra{tout}\end{définition}
\begin{définition}\pcmn{一切}\end{définition}
\begin{exemple}\pjya{lonbɯnba}\hspace{5pt}\pcmn{所有一切}\end{exemple}\étymologie{lon.pa}\end{entrée}

\begin{entrée}{loŋbutɕhi}{}{ⓔloŋbutɕhi} 
\classe{n} 
\begin{définition}\pfra{éléphant}\end{définition}
\begin{définition}\pcmn{大象}\end{définition}\étymologie{glaŋ.po.tɕʰe}\end{entrée}

\begin{entrée}{loŋloŋ}{}{ⓔloŋloŋ} 
\classe{idph.2} \sens{1}
\begin{définition}\pfra{imposant}\end{définition}
\begin{définition}\pcmn{高大;颜色比较黑}\end{définition}
\begin{exemple}\pjya{jla loŋloŋ ʑo ɲɯ-rɤʑi}\hspace{5pt}\pcmn{犏牛在那里又黑又高大}\end{exemple}
\begin{exemple}\pjya{si loŋloŋ ʑo ɲɯ-pa}\hspace{5pt}\pcmn{树很高大}\end{exemple}
\begin{exemple}\pjya{loŋloŋ ʑo ɲɯ-rŋgɯ}\hspace{5pt}\pcmn{他在那里睡觉(很高大的样子)}\end{exemple}
\begin{sous-entrée}{loŋnɤloŋ}{ⓔloŋloŋⓢ1ⓝloŋnɤloŋ} 
\classe{idph.3} 
\begin{exemple}\pjya{zdɯm loŋnɤloŋ ʑo jo-ɣi}\hspace{5pt}\pcmn{乌云从四方聚拢而来}\end{exemple}\end{sous-entrée}

\begin{sous-entrée}{loŋɯŋi}{ⓔloŋloŋⓢ1ⓝloŋɯŋi}
\begin{définition}\pfra{s'élever lentement}\end{définition}
\begin{définition}\pcmn{慢慢地升起}\end{définition}
\begin{exemple}\pjya{tɤ-khɯ loŋɯŋi ʑo to-ɬoʁ}\hspace{5pt}\pcmn{烟子慢慢地冒出来了}\end{exemple}\end{sous-entrée}

\begin{sous-entrée}{ɣɤloŋloŋ}{ⓔloŋloŋⓢ1ⓝɣɤloŋloŋ} 
\classe{vi} 
\begin{définition}\pfra{s'élever (fumée, nuage)}\end{définition}
\begin{définition}\pcmn{缭缭升起,缭绕上升(烟子、云)}\end{définition}
\begin{exemple}\pjya{zgo ɯ-taʁ zdɯm ɲɯ-ɣɤloŋloŋ}\hspace{5pt}\pcmn{山被云雾笼罩}\end{exemple}
\begin{exemple}\pjya{tɤ-khɯ ɲɯ-ɣɤloŋloŋ}\hspace{5pt}\pcmn{烟子寥寥升起}\end{exemple}\end{sous-entrée}

\end{entrée}

\begin{entrée}{loʁ}{}{ⓔloʁ} 
\classe{vs} \paradigme{dir}{nɯ-}
\begin{définition}\pfra{travailler, se déformer à cause de l'humidité}\end{définition}
\begin{définition}\pcmn{木板受潮而变形}\end{définition}
\begin{exemple}\pjya{tɤrɤmɕkho ɲo-loʁ}\hspace{5pt}\pcmn{地板变形了}\end{exemple}
\begin{exemple}\pjya{tʂɤm ɲɤ-loʁ}\hspace{5pt}\pcmn{板壁变形了}\end{exemple}
\begin{exemple}\pjya{kɯm ɲɤ-loʁ tɕe kɤ-nɤcɯpa ɲɯ-ɴqa}\hspace{5pt}\pcmn{门变形了,很难开关}\end{exemple}\end{entrée}

\begin{entrée}{loʁskɤr}{}{ⓔloʁskɤr} 
\classe{n} 
\begin{définition}\pfra{perles insérées dans les tresses}\end{définition}
\begin{définition}\pcmn{辫子上穿的一圈一圈的珠子}\end{définition}\end{entrée}

\begin{entrée}{lotɕi}{}{ⓔlotɕi} 
\classe{part} 
\begin{définition}\pfra{non}\end{définition}
\begin{définition}\pcmn{呢}\end{définition}
\begin{exemple}\pjya{nɤ-smɤn ko-tɯ-tshi-t lotɕi}\hspace{5pt}\pcmn{你不是喝了药的吗?}\end{exemple}
\begin{exemple}\pjya{ʑatsa tɯ-nɯɕe lotɕi}\hspace{5pt}\pcmn{你不是快要回去了吗?}\end{exemple}\relationsémantique{参考}{\lien{ⓔloⓗ2}{lo₂}}\end{entrée}

\begin{entrée}{lqɤnɤlqɤt}{}{ⓔlqɤnɤlqɤt} 
\classe{idph.2} 
\begin{définition}\pfra{lentement, en titubant}\end{définition}
\begin{définition}\pcmn{形容走路慢而摇晃的样子}\end{définition}
\begin{exemple}\pjya{tɤ-pɤtso nɯ lqɤnɤlqɤt tu-ŋke ɲɯ-cha}\hspace{5pt}\pcmn{小孩子能摇摇晃晃地走路}\end{exemple}\relationsémantique{参考}{\lien{ⓔsɤlqɤlqɤt}{sɤlqɤlqɤt}}\end{entrée}

\begin{entrée}{lʁa}{}{ⓔlʁa} 
\classe{n} 
\begin{définition}\pfra{sac en toile}\end{définition}
\begin{définition}\pcmn{麻袋}\end{définition}
\begin{exemple}\pjya{lʁa nɯ tasa thɯ-kɤ-βzu tɤ-fkɯm ŋu, tasa kɤ́-wɣ-phɯt tɕe khɤxtu zɯ pjɯ́-wɣ-z-nɯɣur tɕe ɲɯ́-wɣ-ta, tɤ-rom tɕe ɯ-rɣi nɯ pjɯ́-wɣ-kra tɕe ɯ-pɯ tú-wɣ-pa tɕe ɯ-ru tú-wɣ-xtɕɤr tɕe ɲɯ́-wɣ-ta, ftɕar tɕe tɯ-mɯ kɤ-lɤt tɕe, ɯ-thoʁ pjɯ́-wɣ-sɤʑɯrja tɕe kɤ-la ʑo tɕe chɯ́-wɣ-ʑɴɢu tɕe tɯ-zboʁ tɯ-zboʁ tú-wɣ-xtɕɤr, tɕe tɤ-zbaʁ tɕe lú-wɣ-pɣo, lú-wɣ-rɯm, tɕe tɤ-βri tú-wɣ-βzu, tɤ-βri nɯ kú-wɣ-sqa, ɯ-ŋgɯ thɤfkɤlɤɣi lú-wɣ-lɤt tɕe ɲɯ́-wɣ-sɯ-ɤla. thɯ́-wɣ-tɕɤt nɯ-afɕu tɕe ɲɯ́-wɣ-χtɕi, ɲɯ́-wɣ-sɯ-wɣrum ʑo tɕe ɲɯ́-wɣ-ɕkho, nɯ a-tɤ-zbaʁ tɕe pjɯ́-wɣ-xtsɯ ra, ɲɯ́-wɣ-ɣɤmpɯ ʑo tɕe kú-wɣ-sɤrɤt tɕe kɤ-taʁ kú-wɣ-thɯ. kɤ-taʁ nɯ ŋgɤrom tu, tɯ-taχte tu, tɯ-taχte nɯ kɯ kɯ-mna kɤ-sɯpa ŋu, ɯ-spa ɴqa, kɤ-taʁ ɴqa, tɕeri ngɯt, ŋgɤrom nɯ ɯ-spa mbat, kɤ-taʁ mbat, tɕeri kɤ-ntɕhoz tɕe mɤ-ngɯt, kɤ-taʁ thɯ-jɤɣ tɕe, li nɯ ɣɯ-χtɕi ra tɤ-zbaʁ tɕe ɣɯ-xtsɯ ra, pɯ́-wɣ-xtsɯ kóʁmɯz nɤ ɲɯ́-wɣ-qrɯ tɕe, tɕe chɯ́-wɣ-tʂɯβ, tɕe kɯ-ɤβʑɯrdu kɯ-rɲɟi tsa ɲɯ́-wɣ-βzu, ɯ-qa χchoʁe nɯ tɕu ɯ-jndɯz kú-wɣ-tshoʁ ra, ɯ-mŋu zɯ tɤ-fsɤri maʁ nɤ qase ɯ-sɤ-xtɕɤr kú-wɣ-tshoʁ ra, tɕe nɯ kóʁmɯz nɤ lʁa kɤ-ntɕhoz tu-βze ŋu. nɯnɯ kɯrɯ ra ji-tɯjpu ɯ-sɤ-rku ŋu, tɤ-fkɯm lʁa kɯ-maʁ ʁnɯ-tɯ-phɯ tu, tɯ-ndʐi kɯ thɯ-kɤ-βzu ci tu tɕe, nɯ zgrawa rmi, smɤɣ kɯ thɯ-kɤ-βzu ci tu tɕe, nɯnɯ zɟi rmi.}\hspace{5pt}\pcmn{麻袋是用大麻做成的口袋。大麻割了以后放在屋顶上,让它受霜。干了以后,把种子抖掉后收藏好,再把麻杆捆起来放好。到春天下雨的时候,把麻杆摆在地面上,让它浸泡,然后剥下麻皮,一把一把地捆在一起。晾干后先用手搓成细线,用吊干搓紧,再反搓成一绞一绞,然后再煮。里面要放比较多的灶灰,让它煮开一段时间才取出来。冷却后再洗白凉干,然后捶打到变柔和为止,才把线卸下来,再牵在牵杆上(准备织)。织布的方式有两种,单巴子和斜纹子。斜纹子是比较优质的,用的材料多一些,织起来难一些,但是比较结实。单巴子材料用得不多,织起来容易一些,但用起来不结实。织完后,又要洗,干了以后还要捶打,然后再裁下来,缝好,做成长方形的。在底部的左右两边要装饰上流苏,在口部要扎上麻绳或者皮绳用来封口。这样才能使用麻袋。那是我们藏民装粮食的口袋。除了麻袋,还有两种口袋,一种是用皮做成的,叫\lien{ⓔzgrawa}{zgrawa},另一种是用羊毛做成的,叫\lien{ⓔzɟi}{zɟi}。}\end{exemple}\end{entrée}

\begin{entrée}{lʁɤtɕɯ}{}{ⓔlʁɤtɕɯ} 
\classe{n} 
\begin{définition}\pfra{petit sac}\end{définition}
\begin{définition}\pcmn{小口袋}\end{définition}\end{entrée}

\begin{entrée}{lʁɯba}{}{ⓔlʁɯba} 
\classe{n} 
\begin{définition}\pfra{muet}\end{définition}
\begin{définition}\pcmn{哑巴}\end{définition}\étymologie{lkug.pa?glen.pa?}\end{entrée}

\begin{entrée}{ltɤβ}{}{ⓔltɤβ} 
\classe{vt} \paradigme{dir}{kɤ-}
\begin{définition}\pfra{plier}\end{définition}
\begin{définition}\pcmn{折起来;折叠}\end{définition}
\begin{exemple}\pjya{kɤ-ltaβ-a}\hspace{5pt}\pcmn{我折了}\end{exemple}
\begin{exemple}\pjya{nɤki tɯ-ŋga nɯ kɤ-ltɤβ}\hspace{5pt}\pcmn{你把那件衣服折一下}\end{exemple}\relationsémantique{同义词}{\lien{ⓔzdɤβ}{zdɤβ}}\étymologie{lteb}\end{entrée}

\begin{entrée}{ltɕaʁ}{}{ⓔltɕaʁ} 
\classe{vt} \paradigme{dir}{tɤ-}
\begin{définition}\pfra{frapper le beurre}\end{définition}
\begin{définition}\pcmn{用手拍打酥油(挤出奶水)}\end{définition}
\begin{exemple}\pjya{ta-mar tɤ-ltɕaʁ}\hspace{5pt}\pcmn{你拍打一下酥油吧}\end{exemple}\end{entrée}

\begin{entrée}{ltɕhaŋltɕhaŋ}{}{ⓔltɕhaŋltɕhaŋ} 
\classe{idph.2} 
\begin{définition}\pfra{long et pendant}\end{définition}
\begin{définition}\pcmn{形容长的物体悬吊着的样子}\end{définition}
\begin{exemple}\pjya{mbro ɯ-jme ltɕhaŋltɕhaŋ ʑo pa}\hspace{5pt}\pcmn{马的尾巴吊着很长}\end{exemple}\end{entrée}

\begin{entrée}{ltɕhɤltɕhɤt}{}{ⓔltɕhɤltɕhɤt} 
\classe{idph.2} 
\begin{définition}\pfra{être suspendu (épis de céréales, touffe de fils etc...)}\end{définition}
\begin{définition}\pcmn{形容一簇絮状的东西,穗子吊着的样子}\end{définition}
\begin{sous-entrée}{sɤltɕhɤltɕhɤt}{ⓔltɕhɤltɕhɤtⓝsɤltɕhɤltɕhɤt} 
\classe{vt} 
\begin{définition}\pfra{secouer légèrement (un objet long et fin)}\end{définition}
\begin{définition}\pcmn{轻轻滴抖动(软、细长的东西);轻轻地撒(水)}\end{définition}
\begin{exemple}\pjya{tɯ-ci ɲɯ-sɤltɕhɤltɕhɤt}\hspace{5pt}\pcmn{他在轻轻地撒水}\end{exemple}
\begin{exemple}\pjya{sɯjwaʁ ɲɯ-sɤltɕhɤltɕhɤt}\hspace{5pt}\pcmn{他在轻轻摇动树叶}\end{exemple}\relationsémantique{同义词}{\lien{ⓔʑdraŋʑdraŋⓝsɤʑdraŋlaŋ}{sɤʑdraŋlaŋ}}\relationsémantique{同义词}{\lien{ⓔɕtʂɯɣɕtʂɯɣⓝsɤɕtʂɯlɯɣ}{sɤɕtʂɯlɯɣ}}\end{sous-entrée}

\end{entrée}

\begin{entrée}{ltɕhɯɣltɕhɯɣ}{}{ⓔltɕhɯɣltɕhɯɣ} 
\classe{idph.2} 
\begin{définition}\pfra{long et fin, suspendu}\end{définition}
\begin{définition}\pcmn{形容细长,吊着的样子}\end{définition}
\begin{exemple}\pjya{ʑmbri ɯ-jwaʁ ltɕhɯɣltɕhɯɣ ɲɯ-pa}\hspace{5pt}\pcmn{杨树叶子在吊着}\end{exemple}
\begin{sous-entrée}{ltɕhɯɣnɤlɯɣ}{ⓔltɕhɯɣltɕhɯɣⓝltɕhɯɣnɤlɯɣ} 
\classe{idph.4} 
\begin{exemple}\pjya{tɤ-pɤtso kɯ ɯ-jaʁ tɤtar ltɕhɯɣnɤlɯɣ ɲɯ-ɤsɯ-stu}\hspace{5pt}\pcmn{小孩子在摇动棍子}\end{exemple}\end{sous-entrée}

\begin{sous-entrée}{sɤltɕhɯɣlɯɣ}{ⓔltɕhɯɣltɕhɯɣⓝsɤltɕhɯɣlɯɣ} 
\classe{vt} 
\begin{exemple}\pjya{jla kɯ ɯ-jme ɲɯ-sɤltɕhɯɣlɯɣ}\hspace{5pt}\pcmn{犏牛在摆动尾巴}\end{exemple}\end{sous-entrée}

\end{entrée}

\begin{entrée}{lthjɤlthjɤt}{}{ⓔlthjɤlthjɤt} 
\classe{idph.2} 
\begin{définition}\pfra{propre et bien repassé}\end{définition}
\begin{définition}\pcmn{形容又干净又柔软又平整样子}\end{définition}
\begin{exemple}\pjya{naŋʁɯ lthjɤlthjɤt ci ɲɯ-ŋu}\hspace{5pt}\pcmn{衬衣很柔软}\end{exemple}
\begin{exemple}\pjya{ɯ-ŋga ɯ-tɯ-ɕo kɯ lthjɤlthjɤt ʑo ɲɯ-pa}\hspace{5pt}\pcmn{他的衣服又干净又柔软}\end{exemple}
\begin{exemple}\pjya{tɕhemɤpɯ lthjɤlthjɤt to-nɯ-rɤmpɕɤr}\hspace{5pt}\pcmn{女孩子打扮得很漂亮}\end{exemple}\end{entrée}

\begin{entrée}{lthɯlthɯɣ}{}{ⓔlthɯlthɯɣ} 
\classe{idph.2} 
\begin{définition}\pfra{moelleux, mou}\end{définition}
\begin{définition}\pcmn{形容又平整又柔软的感觉}\end{définition}
\begin{exemple}\pjya{tshɤrtɯl lthɯlthɯɣ ɲɯ-ŋu}\hspace{5pt}\pcmn{羔羊皮衣里面是毛茸茸的}\end{exemple}
\begin{exemple}\pjya{tɯji pɯ-kɯ-jɤɣ ɣɯ sɤtɕha lthɯlthɯɣ ʑo pa}\hspace{5pt}\pcmn{下完种的地很松软}\end{exemple}\end{entrée}

\begin{entrée}{lthɯmlthɯm}{}{ⓔlthɯmlthɯm} 
\classe{idph.2} 
\begin{définition}\pfra{faible}\end{définition}
\begin{définition}\pcmn{形容没有精神,软弱的样子}\end{définition}
\begin{exemple}\pjya{ɲɯ-ngo rca ma, lthɯmlthɯm ʑo ɲɯ-rɤʑi}\hspace{5pt}\pcmn{他病了,在那里很软弱,没有精神的样子}\end{exemple}
\begin{sous-entrée}{lthɯmɯmi}{ⓔlthɯmlthɯmⓝlthɯmɯmi} 
\classe{idph.7} 
\begin{définition}\pfra{qui vient sans qu'on s'en rende compte (sommeil)}\end{définition}
\begin{définition}\pcmn{不知不觉地,产生了睡意(很舒服的感觉)}\end{définition}
\begin{exemple}\pjya{a-ʑɯβ lthɯmɯmi ʑo pɯ-ɣe}\hspace{5pt}\pcmn{我不知不觉地睡着了}\end{exemple}\relationsémantique{参考}{\lien{ⓔɬɯmɬɯm}{ɬɯmɬɯm}}\end{sous-entrée}

\end{entrée}

\begin{entrée}{ltshɤltshɤt}{}{ⓔltshɤltshɤt} 
\classe{idph.2} 
\begin{définition}\pfra{petit et faible}\end{définition}
\begin{définition}\pcmn{形容瘦而小的东西竖立着的样子}\end{définition}
\begin{sous-entrée}{ltshɤnɤltshɤt}{ⓔltshɤltshɤtⓝltshɤnɤltshɤt}
\begin{exemple}\pjya{tɤ-pɤtso ltshɤnɤltshɤt ʑo ɲɯ-ŋke}\hspace{5pt}\pcmn{小孩子走路显得又小又弱}\end{exemple}\relationsémantique{参考}{\lien{ⓔɣɤltshɤltshɤt}{ɣɤltshɤltshɤt}}\end{sous-entrée}

\end{entrée}

\begin{entrée}{lɯβ}{}{ⓔlɯβ} 
\classe{vs} \paradigme{dir}{thɯ-}\paradigme{dir}{tɤ-}
\begin{définition}\pfra{être sombre}\end{définition}
\begin{définition}\pcmn{天阴}\end{définition}
\begin{exemple}\pjya{tɯ-mɯ chɤ-lɯβ}\hspace{5pt}\pcmn{天变阴了}\end{exemple}
\begin{exemple}\pjya{tɯ-mɯ ɲɯ-lɯβ}\hspace{5pt}\pcmn{天很阴}\end{exemple}\relationsémantique{同义词}{\lien{ⓔqanɯ}{qanɯ}}\end{entrée}

\begin{entrée}{lɯɣ}{}{ⓔlɯɣ} 
\classe{vi} \sens{1}\paradigme{dir}{\_}
\begin{définition}\pfra{se détacher}\end{définition}
\begin{définition}\pcmn{解脱;滑}\end{définition}
\begin{exemple}\pjya{kutɕu kɤ-βraʁ-a ri, kɤ-lɯɣ}\hspace{5pt}\pcmn{我在这里拴了一下,但是滑了}\end{exemple}
\begin{exemple}\pjya{a@caidao pɯ-lɯɣ tɕe, a-jaʁ ta-xtsɯɣ}\hspace{5pt}\pcmn{我的菜刀滑了一下,切到了我的手}\end{exemple}
\begin{exemple}\pjya{khɯna nɯ-lɯɣ}\hspace{5pt}\pcmn{狗脱链了}\end{exemple}\sens{2}\paradigme{dir}{\_}
\begin{définition}\pfra{traverser}\end{définition}
\begin{définition}\pcmn{穿过}\end{définition}
\begin{exemple}\pjya{ɯʑo sɯŋgɯ kɤ-lɯɣ}\hspace{5pt}\pcmn{他穿过了森林}\end{exemple}\relationsémantique{同义词}{\lien{ⓔpjɤl}{pjɤl}}\sens{3}\paradigme{dir}{tɤ-}\paradigme{dir}{nɯ-}\paradigme{dir}{nɯ-}
\begin{définition}\pfra{se produire (incendie)}\end{définition}
\begin{définition}\pcmn{着(火)}\end{définition}
\begin{définition}\pfra{détacher}\end{définition}
\begin{définition}\pcmn{解开}\end{définition}
\begin{définition}\pfra{se détacher}\end{définition}
\begin{définition}\pcmn{解脱;逃脱}\end{définition}
\begin{exemple}\pjya{ɣndʑɤβ to-lɯɣ}\hspace{5pt}\pcmn{着火了}\end{exemple}
\begin{exemple}\pjya{na-sɯɣlɯɣ}\hspace{5pt}\pcmn{他(把线)解开了}\end{exemple}
\begin{exemple}\pjya{khɯna ɲɤ-ʑɣɤsɯɣlɯɣ}\hspace{5pt}\pcmn{狗脱链了}\end{exemple}
\begin{exemple}\pjya{tɯrme kú-wɣ-ja ri ɲɤ-ʑɣɤsɯɣlɯɣ}\hspace{5pt}\pcmn{他被关(进监狱),但是逃脱了}\end{exemple}\relationsémantique{参考}{\lien{ⓔɕlɯɣ}{ɕlɯɣ}}\relationsémantique{同义词}{\lien{ⓔsɯɕlɯɣ}{sɯɕlɯɣ}}
\begin{sous-entrée}{sɯɣlɯɣ}{ⓔlɯɣⓢ3ⓝsɯɣlɯɣ} 
\classe{vt} \end{sous-entrée}

\begin{sous-entrée}{ʑɣɤsɯɣlɯɣ}{ⓔlɯɣⓢ3ⓝʑɣɤsɯɣlɯɣ} 
\classe{vi}  
\grammaire{refl}
\grammaire{caus} \end{sous-entrée}

\end{entrée}

\begin{entrée}{lɯɣlu}{}{ⓔlɯɣlu} 
\classe{n} 
\begin{définition}\pfra{année du mouton}\end{définition}
\begin{définition}\pcmn{羊年}\end{définition}\étymologie{lug.lo}\end{entrée}

\begin{entrée}{lɯɣmbrɯm}{}{ⓔlɯɣmbrɯm} 
\classe{n} 
\begin{définition}\pfra{maladie du ventre}\end{définition}
\begin{définition}\pcmn{风丹}\end{définition}
\begin{exemple}\pjya{ɲɯ-rɤʑa pha tɯ-phoŋbu ʑo kɯ-ɤmɯrmɯrmbat ʑo tɤ-ndɤr ɲɯ-ɬoʁ tɕe nɯ lɯɣmbrɯm rmi. tɯ-phoŋbu nɯ-aci cho nɯ-mɯɕtaʁ tɕe ɲɯ-ɬoʁ ŋu}\hspace{5pt}\pcmn{全身痒,痘痘长得很密,叫作风丹。身体淋湿,发冷的情况下就会出现。}\end{exemple}\étymologie{lug.ⁿbrum}\end{entrée}

\begin{entrée}{lɯlu}{}{ⓔlɯlu} 
\classe{n} 
\begin{définition}\pfra{chat}\end{définition}
\begin{définition}\pcmn{猫}\end{définition}\relationsémantique{参考}{\lien{ⓔlɯlɤmu}{lɯlɤmu}}\relationsémantique{参考}{\lien{ⓔlɯlɤpɯ}{lɯlɤpɯ}}\end{entrée}

\begin{entrée}{lɯlɤmu}{}{ⓔlɯlɤmu} 
\classe{n} 
\begin{définition}\pfra{chatte}\end{définition}
\begin{définition}\pcmn{母猫}\end{définition}\end{entrée}

\begin{entrée}{lɯlɤpɯ}{}{ⓔlɯlɤpɯ} 
\classe{n} 
\begin{définition}\pfra{chaton}\end{définition}
\begin{définition}\pcmn{猫崽子}\end{définition}\relationsémantique{参考}{\lien{ⓔlɯlu}{lɯlu}}\end{entrée}

\begin{entrée}{lɯlɤrgɤn}{}{ⓔlɯlɤrgɤn} 
\classe{n} 
\begin{définition}\pfra{vieux chat}\end{définition}
\begin{définition}\pcmn{老猫}\end{définition}\relationsémantique{参考}{\lien{ⓔlɯlu}{lɯlu}}\end{entrée}

\begin{entrée}{lɯlukɯra}{}{ⓔlɯlukɯra} 
\classe{adv} 
\begin{définition}\pfra{avidité insatiable}\end{définition}
\begin{définition}\pcmn{贪得无厌}\end{définition}
\begin{exemple}\pjya{lɯlukɯra ma-tɤ-tɯ-βze}\hspace{5pt}\pcmn{你不要贪得无厌}\end{exemple}\end{entrée}

\begin{entrée}{lɯrlɯr}{}{ⓔlɯrlɯr} 
\classe{idph.2} 
\begin{définition}\pfra{petit objet rond (qui roule)}\end{définition}
\begin{définition}\pcmn{形容又小又圆的东西在滚(如皮球、洋芋、乒乓球等)的样子}\end{définition}
\begin{sous-entrée}{ɣɤlɯrlɯr}{ⓔlɯrlɯrⓝɣɤlɯrlɯr} 
\classe{vi} 
\begin{exemple}\pjya{staχpɯ ɲɯ-ɣɤlɯrlɯr chɯ-ndʐaβ ɲɯ-ŋu}\hspace{5pt}\pcmn{绿豆滚下去}\end{exemple}\end{sous-entrée}

\begin{sous-entrée}{sɤlɯrlɯr}{ⓔlɯrlɯrⓝsɤlɯrlɯr} 
\classe{vt} 
\begin{exemple}\pjya{tɤ-sɤlɯrlɯr-a thɯ-tʂaβ-a}\hspace{5pt}\pcmn{我令(球)滚下去了}\end{exemple}
\begin{exemple}\pjya{tɤ-pɤtso kɯ ɯ-kɯmtɕhɯ ɲɯ-sɤlɯrlɯr tha-tʂaβ}\hspace{5pt}\pcmn{小孩子把玩具滚了下去}\end{exemple}\end{sous-entrée}

\begin{sous-entrée}{lɯrɯri}{ⓔlɯrlɯrⓝlɯrɯri}
\begin{définition}\pfra{s'élever lentement (fumée, vapeur)}\end{définition}
\begin{définition}\pcmn{形容烟,蒸汽慢慢冒上来的样子}\end{définition}
\begin{exemple}\pjya{tɤ-khɯ lɯrɯri tu-ɬoʁ ɲɯ-ŋu}\hspace{5pt}\pcmn{烟慢慢地冒上来}\end{exemple}
\begin{exemple}\pjya{smi ɲɯ-ɤsɯ-βlɯ tɕe, tɤ-khɯ lɯrɯri ta-tɕɤt}\hspace{5pt}\pcmn{他烧火,令烟慢慢地冒上来}\end{exemple}\end{sous-entrée}

\end{entrée}

\begin{entrée}{lɯski}{}{ⓔlɯski} 
\classe{cnj} 
\begin{définition}\pfra{bien sûr}\end{définition}
\begin{définition}\pcmn{当然}\end{définition}\end{entrée}

\begin{entrée}{lɯtoʁ}{}{ⓔlɯtoʁ} 
\classe{n} 
\begin{définition}\pfra{récolte des plantes que l'on a semées et des plantes sauvage}\end{définition}
\begin{définition}\pcmn{庄稼和野草}\end{définition}\étymologie{lo.tog}\end{entrée}

\begin{entrée}{lɯxpa}{}{ⓔlɯxpa} 
\classe{n} 
\begin{définition}\pfra{berger (qui élève des moutons)}\end{définition}
\begin{définition}\pcmn{放羊的人}\end{définition}\étymologie{lug.pa}\end{entrée}

\begin{entrée}{lɯz}{}{ⓔlɯz} 
\classe{vi} 
\begin{définition}\pfra{rester}\end{définition}
\begin{définition}\pcmn{留下}\end{définition}
\begin{exemple}\pjya{nɤ-zda jɤ-anɯri-nɯ tɕe nɤʑo nɯ-lɯz}\hspace{5pt}\pcmn{你的伙伴回去了,你留下吧}\end{exemple}
\begin{exemple}\pjya{nɯ-lɯz-a}\hspace{5pt}\pcmn{我留下了}\end{exemple}\relationsémantique{参考}{\lien{ⓔsɯɣlɯz}{sɯɣlɯz}}\end{entrée}

\begin{entrée}{lɯzlɯz}{}{ⓔlɯzlɯz} 
\classe{idph.2} 
\begin{définition}\pfra{petit et immobile}\end{définition}
\begin{définition}\pcmn{形容很小的物体动也不动的样子}\end{définition}
\begin{exemple}\pjya{tɤ-pɤtso tɯ-sta zɯ lɯzlɯz ʑo ɲɯ-nɯ-rŋgɯ}\hspace{5pt}\pcmn{小孩子在床上动也不动地睡觉}\end{exemple}
\begin{sous-entrée}{lɯznɤlɯz}{ⓔlɯzlɯzⓝlɯznɤlɯz}
\begin{définition}\pfra{sans se presser}\end{définition}
\begin{définition}\pcmn{形容不慌不忙的的样子}\end{définition}
\begin{exemple}\pjya{lɯznɤlɯz ʑo ɲɯ-rɤma}\hspace{5pt}\pcmn{他不慌不忙地做事}\end{exemple}\end{sous-entrée}

\end{entrée}

\begin{entrée}{lwɤrlwɤr}{}{ⓔlwɤrlwɤr} 
\classe{idph.2} 
\begin{définition}\pfra{énorme}\end{définition}
\begin{définition}\pcmn{形容大块的样子}\end{définition}
\begin{exemple}\pjya{a-mthɯm kɯ-wxti lwɤrlwɤr ʑo na-βzu}\hspace{5pt}\pcmn{肉给我切了一大块(他很款待我)}\end{exemple}
\begin{sous-entrée}{lwɤrnɤlwɤr}{ⓔlwɤrlwɤrⓝlwɤrnɤlwɤr} 
\classe{idph.3} \end{sous-entrée}

\end{entrée}

\begin{entrée}{lwɤz}{}{ⓔlwɤz} 
\classe{vi} \paradigme{dir}{tɤ-}
\begin{définition}\pfra{retomber malade}\end{définition}
\begin{définition}\pcmn{犯病}\end{définition}
\begin{exemple}\pjya{a-kɯ-mŋɤm to-lwɤz}\hspace{5pt}\pcmn{我犯病了}\end{exemple}\end{entrée}

\begin{entrée}{lwoʁ}{}{ⓔlwoʁ} 
\classe{vt} \paradigme{dir}{thɯ-}\paradigme{dir}{pɯ-}
\begin{définition}\pfra{asperger}\end{définition}
\begin{définition}\pcmn{泼水}\end{définition}
\begin{exemple}\pjya{tɯ-ci pɯ-lwoʁ}\hspace{5pt}\pcmn{把水倒掉吧}\end{exemple}
\begin{exemple}\pjya{nɤki ɯ-ro nɯ thɯ-lwoʁ}\hspace{5pt}\pcmn{你把剩下的那个倒掉吧}\end{exemple}
\begin{exemple}\pjya{sɤlaŋphɤn ɯ-ŋgɯ tɯ-ci nɯ ra thɯ-lwoʁ-a}\hspace{5pt}\pcmn{我把盆子里的水倒掉了}\end{exemple}
\begin{exemple}\pjya{tɯ-mɯ ɲɯ-ɤsɯ-lwoʁ ʑo}\hspace{5pt}\pcmn{下了倾盆大雨}\end{exemple}\relationsémantique{参考}{\lien{ⓔciⓗ2}{ci}}\end{entrée}

\begin{entrée}{lwɯlwɯɣ}{}{ⓔlwɯlwɯɣ} 
\classe{idph.2} 
\begin{définition}\pfra{ébouriffé}\end{définition}
\begin{définition}\pcmn{形容凌乱而蓬松的样子}\end{définition}
\begin{exemple}\pjya{tɤtɕɯ nɯ ɯ-kɤrme lwɯlwɯɣ ʑo ɲɯ-pa tɕe pjɯ́-wɣ-qrɤz ɲɯ-ra}\hspace{5pt}\pcmn{那个男孩子的头发又长又乱、乱蓬蓬的,非得把它剃掉不可}\end{exemple}\end{entrée}

\begin{entrée}{lxɤβlxɤβ}{}{ⓔlxɤβlxɤβ} 
\classe{idph.2} 
\begin{définition}\pfra{épais (vêtements)}\end{définition}
\begin{définition}\pcmn{沉重;厚实(衣服)}\end{définition}
\begin{exemple}\pjya{tɯ-ŋga kɯ-jɯ-jaʁ ʑo lxɤβlxɤβ tɤ-ŋga-t-a}\hspace{5pt}\pcmn{我穿了很厚的衣服}\end{exemple}
\begin{sous-entrée}{lxɤβnɤlxɤβ}{ⓔlxɤβlxɤβⓝlxɤβnɤlxɤβ} 
\classe{idph.3} 
\begin{exemple}\pjya{lxɤβnɤlxɤβ ɲɯ-ŋke}\hspace{5pt}\pcmn{他穿了很重的衣服在走路}\end{exemple}\relationsémantique{参考}{\lien{ⓔlɣɤβlɣɤβ}{lɣɤβlɣɤβ}}\end{sous-entrée}

\end{entrée}

\begin{entrée}{lxɯlxi}{}{ⓔlxɯlxi} 
\classe{idph.2} 
\begin{définition}\pfra{épais, lourd}\end{définition}
\begin{définition}\pcmn{形容厚实,笨重的样子}\end{définition}
\begin{exemple}\pjya{kɯ-rʑi tsa ci lxɯlxi ɲɯ-ŋu}\hspace{5pt}\pcmn{很笨重}\end{exemple}
\begin{exemple}\pjya{tɯ-ŋga kɯ-jaʁ tsa ci lxɯlxi ɲɯ-ŋu}\hspace{5pt}\pcmn{衣服很厚实}\end{exemple}
\begin{exemple}\pjya{kɯ-khe ci lxɯlxi ɲɯ-tɯ-ŋu}\hspace{5pt}\pcmn{你有点笨}\end{exemple}\relationsémantique{参考}{\lien{ⓔlɣɤβlɣɤβ}{lɣɤβlɣɤβ}}\end{entrée}

\newpage\caractère{ɬ}

\begin{entrée}{ɬa}{}{ⓔɬa} 
\classe{n} 
\begin{définition}\pfra{bouddha}\end{définition}
\begin{définition}\pcmn{佛}\end{définition}\étymologie{lha}\end{entrée}

\begin{entrée}{ɬarɯɣ}{}{ⓔɬarɯɣ} 
\classe{n} 
\begin{définition}\pfra{réincarnation d'un dieu}\end{définition}
\begin{définition}\pcmn{神的化身}\end{définition}\étymologie{lha.rigs}\end{entrée}

\begin{entrée}{ɬasaŋga}{}{ⓔɬasaŋga} 
\classe{n} 
\begin{définition}\pfra{habits du Tibet central}\end{définition}
\begin{définition}\pcmn{西藏服装}\end{définition}\relationsémantique{参考}{\lien{ⓔtɯ-ŋga}{tɯ-ŋga}}\end{entrée}

\begin{entrée}{ɬaχpo}{}{ⓔɬaχpo} 
\classe{adv} 
\begin{définition}\pfra{exprime une impulsion soudaine que l'on essaie de réprimer}\end{définition}
\begin{définition}\pcmn{干脆……算了(一时冲动)}\end{définition}
\begin{exemple}\pjya{nɤʑo taʁndo maka mɯ́j-tɯ-tso tɕe, ɬaχpo ʑo mɯ-tu-ta-ʁndɯ}\hspace{5pt}\pcmn{你不听话,干脆打你一顿算了}\end{exemple}\end{entrée}

\begin{entrée}{ɬɤliaʁ}{}{ⓔɬɤliaʁ} 
\classe{n} 
\begin{définition}\pfra{nom d'un village de Sarndzu}\end{définition}
\begin{définition}\pcmn{沙尔宗的一个村}\end{définition}\end{entrée}

\begin{entrée}{ɬɤɬɤt}{}{ⓔɬɤɬɤt} 
\classe{idph.2} 
\begin{définition}\pfra{de bonne humeur}\end{définition}
\begin{définition}\pcmn{形容心情很舒畅的样子}\end{définition}
\begin{exemple}\pjya{aʑo a-sɯm ɬɤɬɤt ʑo ɲɯ-pa}\hspace{5pt}\pcmn{我心情很好}\end{exemple}
\begin{exemple}\pjya{aʑo a-sɯm ɬɤɬɤt ɲɯ-nɯ-ste-a ɕti}\hspace{5pt}\pcmn{我把心情放舒畅些}\end{exemple}\end{entrée}

\begin{entrée}{ɬɤndʐi}{}{ⓔɬɤndʐi} 
\classe{n} 
\begin{définition}\pfra{démon}\end{définition}
\begin{définition}\pcmn{鬼}\end{définition}\étymologie{lha.ⁿdre}\end{entrée}

\begin{entrée}{ɬɤndʐismi}{}{ⓔɬɤndʐismi} 
\classe{n} 
\begin{définition}\pfra{feu follet}\end{définition}
\begin{définition}\pcmn{磷火【鬼火】}\end{définition}\end{entrée}

\begin{entrée}{ɬɤndʐitɤlɤtshaʁ}{}{ⓔɬɤndʐitɤlɤtshaʁ} 
\classe{n} 
\begin{définition}\pfra{Delphinium sp.}\end{définition}
\begin{définition}\pcmn{翠雀花}\end{définition}
\begin{exemple}\pjya{ɬɤndʐitɤlɤtshaʁ nɯ si kɯ-xtɕi, rɯŋgu, tɯ-ji rkɯ, xɕaj kɯ-dɤn ɯ-rchɤβ tu-ɬoʁ ŋu, ɯ-ru cho ɯ-jwaʁ pɣi tsa arɯlɯŋkɤr, ɯ-mɯntoʁ nɯ kɯ-ɤrŋi tɕe kɯ-ɤɲaʁndzɯm ʑo ŋu. ɯ-ru xtshɯm, xɕaj sɤz kɯ-dɤn mɤ-mbro, ɯ-mɯntoʁ ɯ-tshɯɣa nɯ tɤlɤtshaʁ fse tɕe núndʐa nɯ ɬɤndʐitɤlɤtshaʁ rmi}\hspace{5pt}\pcmn{翠雀花生长在小树丛、草坪、地边、茂盛的草丛中,茎和叶子是淡蓝色的,带有一点灰色。花是深蓝色的。茎很细,比一般的草长不出多少,花的形状像滤牛奶的漏斗,所以叫做\lien{ⓔɬɤndʐitɤlɤtshaʁ}{ɬɤndʐitɤlɤtshaʁ}(鬼的漏斗)}\end{exemple}\end{entrée}

\begin{entrée}{ɬɤndʐitɤtsoʁ}{}{ⓔɬɤndʐitɤtsoʁ} 
\classe{n} 
\begin{définition}\pfra{une plante}\end{définition}
\begin{définition}\pcmn{植物的一种}\end{définition}
\begin{exemple}\pjya{ɬɤndʐi tɤtsoʁ nɯ sɯjno ci ŋu. ɯ-ku nɯ tɤtsoʁ cho naχtɕɯɣ, ɬɤndʐi tɤtsoʁ ɯ-ku wxti. ɯ-ru me, ɯ-jwaʁ nɯ pɣɤ-muj ɯ-tshɯɣa kɯ-fse ŋu, ɯ-ri kɯ-fse kɯ-ɣɯrni ju-kɯ-ɕe ci tu. ɯ-jwaʁ nɯ pjɤ-ɕkho kɯ-fse ŋu tɕe, ɯ-χcɤl ri ɯ-mɯntoʁ ɲɯ-βze ŋu. ɯ-mɯntoʁ tɯ-rdoʁ ma me, kɯ-qarne ŋu. ɯ-qa nɯ tɤtsoʁ cho naχtɕɯɣ, tɕeri ɬɤndʐi tɤtsoʁ ɣɯ ɯ-qa nɯ ɯ-rme ci tu. ɯ-rme nɯ ɲaʁ, mɤ-dɤn ri rɲɟi. ɬɤndʐi tɤtsoʁ nɯ tú-wɣ-ndza tɕe, tɯ-mdʑu ɲɯ-sɤzɯβzɯβ ŋu, chɯ́-wɣ-mqlaʁ tɕe, tɯ-rqo kɤ-sɯɣ kɯ-fse ɲɯ-sɤβze ŋu tɕe, kɤ-ndza mɤ-sna. tɤtsoʁ nɯ tú-wɣ-ndza tɕe, chi, tɯ-mdʑu ɯ-kɯ-sɤzɯβzɯβ me, tɕe nɯ mɯm.}\hspace{5pt}\pcmn{\lien{}{ɬɤndʐi tɤtsoʁ}是一种草,苗和人参果的一样,但大一些。没有茎,叶子像鸟的羽毛的形状,叶柄上好像有红线。叶子向四面展开,中间开花。只有一朵花,是黄色的。根也和人参果一样,但上面有毛,不多但是很长。吃了使舌头发麻,吞下去,使喉咙有很紧的感觉,所以不能吃。人参果吃起来是甜的,舌头没有麻的感觉,好吃。}\end{exemple}\end{entrée}

\begin{entrée}{ɬɤndʐithamaka}{}{ⓔɬɤndʐithamaka} 
\classe{n} 
\begin{définition}\pfra{vesse-de-loup}\end{définition}
\begin{définition}\pcmn{马勃}\end{définition}\relationsémantique{同义词}{\lien{ⓔsalaboŋboŋ}{salaboŋboŋ}}\end{entrée}

\begin{entrée}{ɬɤntshɤm}{}{ⓔɬɤntshɤm} 
\classe{n} 
\begin{définition}\pfra{nakṣatra anurādhās}\end{définition}
\begin{définition}\pcmn{房宿}\end{définition}
\begin{exemple}\pjya{ɬɤntshɤm cho sla ni nɯ-atɯɣ-ndʑi tɕe, sla nɯ ɬɤntshɤm ɯ-rqoʁ zɯ a-pɯ-ɕe tɕe ɣɯjpa taχpa pe kɤ-ti ŋu, sla nɯ ɬɤntshɤm ɣɯ ɯ-jaʁmɤχa a-pɯ-ɕe tɕe, taχpa nɤkɤro kɤ-ti ŋu, ɯ-thɤcu a-pɯ-ɕe tɕe, taχpa mɤ-pe kɤ-ti ŋu}\hspace{5pt}\pcmn{房宿和月亮相逢时,假如月亮到房宿的“手腕”以下,收成很好,假如月亮到房宿的“大拇指”和“食指”之间,收成中等,假如月亮再往下,收成不好。}\end{exemple}\étymologie{lha.mtsʰams}\end{entrée}

\begin{entrée}{ɬɤt}{}{ⓔɬɤt} 
\classe{vs} \paradigme{dir}{pɯ-}
\begin{définition}\pfra{vieillir, se dégrader}\end{définition}
\begin{définition}\pcmn{衰老}\end{définition}
\begin{exemple}\pjya{tɯrnda pjɤ-ɬɤt}\hspace{5pt}\pcmn{房子老化破损了}\end{exemple}\end{entrée}

\begin{entrée}{ɬoʁ}{₁}{ⓔɬoʁⓗ1} 
\classe{vs} 
\begin{définition}\pfra{devoir}\end{définition}
\begin{définition}\pcmn{必须}\end{définition}
\begin{exemple}\pjya{nɯ tu-ste-a ɲɯ-ɬoʁ}\hspace{5pt}\pcmn{我必须这样做}\end{exemple}\end{entrée}

\begin{entrée}{ɬoʁ}{₂}{ⓔɬoʁⓗ2} 
\classe{vi} \sens{1}
\begin{définition}\pfra{sortir, partir}\end{définition}
\begin{définition}\pcmn{出来;发生}\end{définition}
\begin{exemple}\pjya{tɤŋe tɤ-ɬoʁ}\hspace{5pt}\pcmn{太阳升起了}\end{exemple}
\begin{exemple}\pjya{@cai lɤ-ji-tɕi tɕe to-ɬoʁ}\hspace{5pt}\pcmn{我们俩种了菜就长出来了}\end{exemple}
\begin{exemple}\pjya{mbro ɯ-taʁ pɯ-ɬoʁ-a}\hspace{5pt}\pcmn{我下了马}\end{exemple}
\begin{exemple}\pjya{ɯ-kɯ-ɬoʁ mɯ́j-pe}\hspace{5pt}\pcmn{产量不高}\end{exemple}
\begin{exemple}\pjya{ɯ-re pjɤ-ɬoʁ}\hspace{5pt}\pcmn{他笑了一声}\end{exemple}\sens{2}
\begin{définition}\pfra{mettre bas (bovidé)}\end{définition}
\begin{définition}\pcmn{生崽子(牛类)}\end{définition}
\begin{exemple}\pjya{nɯŋa ɲo-ɬoʁ (=chɤ-rɤpɯ)}\hspace{5pt}\pcmn{奶牛生了崽子}\end{exemple}\end{entrée}

\begin{entrée}{ɬɯɣnɤɬɯɣ}{}{ⓔɬɯɣnɤɬɯɣ} 
\classe{idph.3} 
\begin{définition}\pfra{qui bouge}\end{définition}
\begin{définition}\pcmn{形容动物因为呼吸而发生的运动的样子}\end{définition}
\begin{exemple}\pjya{paʁ mɯ-pjɤ-si ma ɯ-xtu ɬɯɣnɤɬɯɣ ʑo ɲɯ-pa}\hspace{5pt}\pcmn{猪没有死,它肚子还在动(表示它在呼吸)}\end{exemple}\end{entrée}

\begin{entrée}{ɬɯmɬɯm}{}{ⓔɬɯmɬɯm} 
\classe{idph.2} \sens{1}
\begin{définition}\pfra{chaud}\end{définition}
\begin{définition}\pcmn{形容天气暖的感觉}\end{définition}
\begin{exemple}\pjya{jisŋi ɲɯ-mpja ɬɯmɬɯm ʑo}\hspace{5pt}\pcmn{今天天气很暖}\end{exemple}\sens{2}
\begin{définition}\pfra{sommeil agréable}\end{définition}
\begin{définition}\pcmn{形容舒服的睡眠}\end{définition}
\begin{exemple}\pjya{a-ʑɯβ ɬɯmɬɯm ʑo pɯ-ɣe}\hspace{5pt}\pcmn{我不知不觉地睡着了}\end{exemple}
\begin{sous-entrée}{ɬɯmɯmi}{ⓔɬɯmɬɯmⓢ2ⓝɬɯmɯmi} 
\classe{idph.7} 
\begin{définition}\pfra{qui vient sans qu'on s'en rende compte (sommeil)}\end{définition}
\begin{définition}\pcmn{不知不觉地,产生了睡意(很舒服的感觉)}\end{définition}
\begin{exemple}\pjya{a-ʑɯβ ɬɯmɯmi ʑo ɲɯ-ɣi}\hspace{5pt}\pcmn{不知不觉地就产生了睡意}\end{exemple}\relationsémantique{参考}{\lien{ⓔlthɯmlthɯm}{lthɯmlthɯm}}\relationsémantique{参考}{\lien{ⓔtɯ-ɬɯm}{tɯ-ɬɯm}}\end{sous-entrée}

\end{entrée}

\newpage\caractère{m}

\begin{entrée}{mu}{₂}{ⓔmuⓗ2} 
\classe{adv} 
\begin{définition}\pfra{pas du tout}\end{définition}
\begin{définition}\pcmn{根本没有}\end{définition}\end{entrée}

\begin{entrée}{mu}{₁}{ⓔmuⓗ1} 
\classe{vi} \paradigme{dir}{nɯ-}
\begin{définition}\pfra{avoir peur}\end{définition}
\begin{définition}\pcmn{害怕}\end{définition}
\begin{exemple}\pjya{aʑo pɯ-mu-a ma ɲɯ-sɤɣmu}\hspace{5pt}\pcmn{因为很可怕,我害怕了}\end{exemple}\relationsémantique{参考}{\lien{ⓔnɯɣmu}{nɯɣmu}}\relationsémantique{参考}{\lien{ⓔɕɯɣmu}{ɕɯɣmu}}\relationsémantique{参考}{\lien{ⓔsɤɣmu}{sɤɣmu}}\end{entrée}

\begin{entrée}{ma/\variante{mɯma}}{}{ⓔma} 
\classe{postp} 
\begin{définition}\pfra{à part}\end{définition}
\begin{définition}\pcmn{除了}\end{définition}
\begin{exemple}\pjya{nɯ ma kɯ-tu me}\hspace{5pt}\pcmn{没有其它的了}\end{exemple}\end{entrée}

\begin{entrée}{macatsɯt}{}{ⓔmacatsɯt} 
\classe{n} 
\begin{définition}\pfra{chaise en bambou}\end{définition}
\begin{définition}\pcmn{竹子编成的椅子}\end{définition}\end{entrée}

\begin{entrée}{mahi}{}{ⓔmahi} 
\classe{n} 
\begin{définition}\pfra{buffle}\end{définition}
\begin{définition}\pcmn{水牛}\end{définition}\end{entrée}

\begin{entrée}{maka}{}{ⓔmaka} 
\classe{adv} 
\begin{définition}\pfra{pas du tout}\end{définition}
\begin{définition}\pcmn{根本}\end{définition}
\begin{exemple}\pjya{maka ʑo pɯ-mto-t-a me / pɯ-mto-t-a maka me}\hspace{5pt}\pcmn{我什么也没有听到}\end{exemple}
\begin{exemple}\pjya{ta-tɯt maka kɯ-tu me}\hspace{5pt}\pcmn{他什么也没有说}\end{exemple}
\begin{exemple}\pjya{ɯ-kɤ-nɯfse maka ʑo me}\hspace{5pt}\pcmn{他谁也不认识}\end{exemple}\end{entrée}

\begin{entrée}{maldo}{}{ⓔmaldo} 
\classe{n} 
\begin{définition}\pfra{érable}\end{définition}
\begin{définition}\pcmn{枫树}\end{définition}
\begin{exemple}\pjya{maldo nɯ si wuma ʑo kɯ-mbro kɯ-wxti kɯ-jpum ci ŋu, ɯ-ru nɯ kɯ-pɣi tsa ci ŋu, ɯ-ru ɯ-taʁ nɯ ra ɯ-zbɤβ kɯ-fse tu, ɯ-rtaʁ dɤn, ɯ-jwaʁ wuma ʑo wxti, ɯ-βzɯr kɯmŋu tu, ɯ-si wuma ʑo ngɯt, mpɕɤr ma ɯ-rɯmu dɤn. ɯ-zrɤm nɯ rtazga ɯ-spa kɯ-pe ɲɯ-ŋu, ɯ-zbɤβ nɯ ra khɯtsa ɯ-spa tu-sɯ-βzu-nɯ ɲɯ-ŋgrɤl, ɯ-ru nɯ ŋgɤjpɤn chɯ-lɤt-nɯ tɕe, wuma ʑo kɯ-pe ɲɯ-ŋu, kha laχtɕha tɕhi kɯ-ra kɤ-βzu ɲɯ-sna.}\hspace{5pt}\pcmn{枫树是长得又高、又大、又粗的树种,树干带有一点灰色,树干上有树瘤,枝桠多,叶子很大,有五个角,木质很结实,因为有很多花纹所以很美。根是作马鞍的好材料,树瘤用来作木碗。树干可以锯成板子,质量很好,是制作各种家具的好材料。}\end{exemple}\end{entrée}

\begin{entrée}{maldzɯ}{}{ⓔmaldzɯ} 
\classe{n} 
\begin{définition}\pfra{une plante}\end{définition}
\begin{définition}\pcmn{植物的一种}\end{définition}
\begin{exemple}\pjya{maldzɯ nɯ ruŋgu kɯ-mbro kɯ-ɣɤndʐo tsa tu-kɯ-ɬoʁ sɯjno ci ŋu, ɯ-jwaʁ ɯ-tshɯɣa nɯ ra qarɣɤpɤt cho naχtɕɯɣ, ɯ-jwaʁ nɯ ra ɯ-rme tu, ɯ-ru kɯ-xtshɯ-xtshɯm ŋu, kɯ-qandʐi tɕe ɯ-taʁ ɯ-rme kɯ-tu ŋu, ɯ-mɯntoʁ li ɯ-fkɯm nɯ li ɯ-rme tu, tɕe pɯ-ɴɢaʁ tɕe ɯ-ŋgɯ ɯ-mɯntoʁ ɲɤ-nɯɬoʁ ŋu. ɯ-mɯntoʁ nɯ-nɯɬoʁ ɕɯmɯma tɕe, aʁrɯrʁu tɕe ʑɯrɯʑɤri pjɯ-ɤstɤko tɕe kɯ-ɣɯrni ŋu, ɯ-mɯntoʁ χsɯ-mpɕar ma me, . kɯ-rɲɟi tɕe kɯ-ɤmtɕoʁ ŋu. ɯ-rɣi me.}\hspace{5pt}\pcmn{\lien{ⓔmaldzɯ}{maldzɯ} 是生长在气候比较寒冷的高山上的一种草。叶子形状和鹿茸花的一样,叶子上有毛,茎很细,是乌色的,上面也有一点毛。花萼上也有毛,掉下了以后,里面就露出花来。花刚露出来时,是皱着的,逐渐伸展。花是红色的,只有三片花瓣,是长而尖的。没有种子。}\end{exemple}\end{entrée}

\begin{entrée}{maŋ}{}{ⓔmaŋ} 
\classe{vi} 
\begin{définition}\pfra{beaucoup}\end{définition}
\begin{définition}\pcmn{很多}\end{définition}
\begin{exemple}\pjya{ɯ-pjɤβlaʁ ɲɯ-maŋ}\hspace{5pt}\pcmn{他想得多}\end{exemple}\étymologie{maŋ}\end{entrée}

\begin{entrée}{maŋdi}{}{ⓔmaŋdi} 
\classe{vs} \paradigme{dir}{nɯ-}
\begin{définition}\pfra{être à l'ouest}\end{définition}
\begin{définition}\pcmn{在西方}\end{définition}\end{entrée}

\begin{entrée}{maŋe}{}{ⓔmaŋe} 
\classe{vi} 
\begin{définition}\pfra{ne pas avoir (sensoriel)}\end{définition}
\begin{définition}\pcmn{没有(亲验)}\end{définition}
\begin{exemple}\pjya{@shangge @xingqi @dianhua ɯ-kɯ-lɤt mataŋe}\hspace{5pt}\pcmn{你上个星期没有打电话来}\end{exemple}
\begin{exemple}\pjya{ɕɤxɕo kɤ-mtshɤm mataŋe}\hspace{5pt}\pcmn{最近没有你的消息}\end{exemple}
\begin{exemple}\pjya{pɯ-nɯ-tu pɯ-nɯ-me maŋe}\hspace{5pt}\pcmn{可有可无}\end{exemple}
\begin{exemple}\pjya{nɯ ma kɤ-pa maŋe}\hspace{5pt}\pcmn{没有其它办法}\end{exemple}\relationsémantique{参考}{\lien{ⓔɣɤʑu}{ɣɤʑu}}\forme{2s}{mataŋe}\end{entrée}

\begin{entrée}{maŋkɯ}{}{ⓔmaŋkɯ} 
\classe{vs} \paradigme{dir}{kɤ-}
\begin{définition}\pfra{être à l'est}\end{définition}
\begin{définition}\pcmn{在东方}\end{définition}\end{entrée}

\begin{entrée}{maŋlo}{}{ⓔmaŋlo} 
\classe{vs} \paradigme{dir}{lɤ-}\paradigme{dir}{lɤ-}
\begin{définition}\pfra{être en amont}\end{définition}
\begin{définition}\pcmn{在上游}\end{définition}
\begin{définition}\pfra{se mettre en amont}\end{définition}
\begin{définition}\pcmn{到上游的地方}\end{définition}
\begin{sous-entrée}{ʑɣɤmaŋlo}{ⓔmaŋloⓝʑɣɤmaŋlo} 
\classe{vi}  
\grammaire{refl} \end{sous-entrée}

\end{entrée}

\begin{entrée}{maŋpa}{}{ⓔmaŋpa} 
\classe{vs} \paradigme{dir}{pɯ-}
\begin{définition}\pfra{être en bas}\end{définition}
\begin{définition}\pcmn{在下面}\end{définition}\end{entrée}

\begin{entrée}{maŋtaʁ}{}{ⓔmaŋtaʁ} 
\classe{vs} \paradigme{dir}{tɤ-}
\begin{définition}\pfra{être en haut}\end{définition}
\begin{définition}\pcmn{在上面}\end{définition}\end{entrée}

\begin{entrée}{maŋthi}{}{ⓔmaŋthi} 
\classe{vs} \paradigme{dir}{thɯ-}
\begin{définition}\pfra{être en aval}\end{définition}
\begin{définition}\pcmn{在下游}\end{définition}\end{entrée}

\begin{entrée}{maqhu}{}{ⓔmaqhu} 
\classe{vs} \paradigme{dir}{nɯ-}\sens{1}
\begin{définition}\pfra{être tard}\end{définition}
\begin{définition}\pcmn{迟到}\end{définition}\sens{2}
\begin{définition}\pfra{être après}\end{définition}
\begin{définition}\pcmn{以后}\end{définition}\end{entrée}

\begin{entrée}{mar}{}{ⓔmar} 
\classe{vt} \paradigme{dir}{pɯ-}
\begin{définition}\pfra{enduire}\end{définition}
\begin{définition}\pcmn{涂;擦}\end{définition}
\begin{exemple}\pjya{ɯʑo kɯ tɯ-ndʐi na-mar}\hspace{5pt}\pcmn{他给皮子上油了}\end{exemple}
\begin{exemple}\pjya{nɤ-ɕnaβ aʁɤndɯndɤt ma-nɯ-tɯ-mar ma ɲɯ-sɤjloʁ}\hspace{5pt}\pcmn{你别到处擦鼻涕,很恶心}\end{exemple}
\begin{exemple}\pjya{khɤndzo ɣɯ-ta tɤ-mda tɕe, ɲɯ́-wɣ-mar ra}\hspace{5pt}\pcmn{用蒸笼的时候,要先擦一点油}\end{exemple}\relationsémantique{参考}{\lien{ⓔta-ʁɟazⓢ2ⓝta-ʁɟaz,mar}{ta-ʁɟaz,mar}}
\begin{sous-entrée}{ʑɣɤmar}{ⓔmarⓝʑɣɤmar} 
\classe{vi}  
\grammaire{refl} 
\begin{définition}\pfra{s'enduire}\end{définition}
\begin{définition}\pcmn{给自己涂}\end{définition}\end{sous-entrée}

\begin{sous-entrée}{amar}{ⓔmarⓝamar} 
\classe{vi}  
\grammaire{pass} 
\begin{définition}\pfra{être enduit}\end{définition}
\begin{définition}\pcmn{被涂在……}\end{définition}\end{sous-entrée}

\end{entrée}

\begin{entrée}{marɲaŋ}{}{ⓔmarɲaŋ} 
\classe{n} 
\begin{définition}\pfra{beurre rance}\end{définition}
\begin{définition}\pcmn{陈酥油}\end{définition}\end{entrée}

\begin{entrée}{marsɤr}{}{ⓔmarsɤr} 
\classe{n} 
\begin{définition}\pfra{beurre frais}\end{définition}
\begin{définition}\pcmn{新鲜酥油}\end{définition}\end{entrée}

\begin{entrée}{marwɤr}{}{ⓔmarwɤr} 
\classe{n} 
\begin{définition}\pfra{boîte où l'on met le beurre}\end{définition}
\begin{définition}\pcmn{酥油盒}\end{définition}\end{entrée}

\begin{entrée}{maʁ}{₂}{ⓔmaʁⓗ2} 
\classe{n} 
\begin{définition}\pfra{taille (chaussures)}\end{définition}
\begin{définition}\pcmn{码(鞋子)}\end{définition}
\begin{exemple}\pjya{nɤ-xtsa nɯ thɤstɯ-maʁ tu-tɯ-ŋge ŋu?}\hspace{5pt}\pcmn{你穿多少码的鞋子?}\end{exemple}\étymologie{fn:码}\end{entrée}

\begin{entrée}{maʁ}{₁}{ⓔmaʁⓗ1} 
\classe{vs} 
\begin{définition}\pfra{ne pas être}\end{définition}
\begin{définition}\pcmn{不是}\end{définition}\relationsémantique{反义词}{\lien{ⓔŋu}{ŋu}}\relationsémantique{参考}{\lien{ⓔrɯkɯmaʁ}{rɯkɯmaʁ}}\relationsémantique{参考}{\lien{ⓔnɯkɯmaʁ}{nɯkɯmaʁ}}\relationsémantique{参考}{\lien{ⓔnɤɣmaʁ}{nɤɣmaʁ}}\relationsémantique{参考}{\lien{ⓔkɯmaʁ}{kɯmaʁ}}\relationsémantique{参考}{\lien{ⓔznɤmaʁmaʁ}{znɤmaʁmaʁ}}\relationsémantique{参考}{\lien{ⓔɣɤmaʁ}{ɣɤmaʁ}}
\begin{sous-entrée}{nɯ maʁ nɤ}{ⓔmaʁⓗ1ⓝnɯ maʁ nɤ} 
\classe{cnj} 
\begin{définition}\pfra{ou bien, sinon}\end{définition}
\begin{définition}\pcmn{不然;要么……要么}\end{définition}\end{sous-entrée}

\begin{sous-entrée}{tɕhi maʁ nɤ}{ⓔmaʁⓗ1ⓝtɕhi maʁ nɤ} 
\classe{cnj} 
\begin{définition}\pfra{au moins}\end{définition}
\begin{définition}\pcmn{至少}\end{définition}\end{sous-entrée}

\begin{sous-entrée}{maʁ kɯ}{ⓔmaʁⓗ1ⓝmaʁ kɯ} 
\classe{cnj} 
\begin{définition}\pfra{non seulement}\end{définition}
\begin{définition}\pcmn{不但……也……}\end{définition}\end{sous-entrée}

\begin{sous-entrée}{tɕhi kɯ-fse ci kɯnɤ}{ⓔmaʁⓗ1ⓝtɕhi kɯ-fse ci kɯnɤ} 
\classe{cnj} 
\begin{définition}\pfra{au moins}\end{définition}
\begin{définition}\pcmn{至少;起码}\end{définition}\end{sous-entrée}

\begin{sous-entrée}{pjɯsɤɣmaʁ me}{ⓔmaʁⓗ1ⓝpjɯsɤɣmaʁ me}
\begin{définition}\pfra{il n'y a aucun doute}\end{définition}
\begin{définition}\pcmn{毫无疑问}\end{définition}
\begin{exemple}\pjya{kɯ-mɯrkɯ ɯʑo ɕti ma pjɯ-sɤɣ-maʁ me}\hspace{5pt}\pcmn{毫无疑问,他就是小偷}\end{exemple}\end{sous-entrée}

\end{entrée}

\begin{entrée}{masɤmdɤla}{}{ⓔmasɤmdɤla} 
\classe{adv} 
\begin{définition}\pfra{en avance}\end{définition}
\begin{définition}\pcmn{提前}\end{définition}
\begin{exemple}\pjya{nɤki tɤ-rɟit nɯ masɤmdɤla ʑo to-ŋke}\hspace{5pt}\pcmn{那个小孩子提前会走路了}\end{exemple}
\begin{exemple}\pjya{jisŋi masɤmdɤla ʑo tɤ-nɯsaχsɯ-j}\hspace{5pt}\pcmn{我们今天提前吃了午餐}\end{exemple}\relationsémantique{参考}{\lien{ⓔmda}{mda}}\end{entrée}

\begin{entrée}{masɤrɯrju}{}{ⓔmasɤrɯrju} 
\classe{adv} 
\begin{définition}\pfra{en cachette}\end{définition}
\begin{définition}\pcmn{悄悄的}\end{définition}\relationsémantique{参考}{\lien{ⓔarju}{arju}}\end{entrée}

\begin{entrée}{matɕi}{}{ⓔmatɕi} 
\classe{cnj} 
\begin{définition}\pfra{sinon, parce que}\end{définition}
\begin{définition}\pcmn{不然,因为}\end{définition}
\begin{exemple}\pjya{nɤ-ŋga nɯ-nɯ-tɕɤt matɕi ɲɯ-ɣɯtshɤdɯɣ nɤ}\hspace{5pt}\pcmn{你脱下衣服,不然很热}\end{exemple}
\begin{exemple}\pjya{mɤʑɯ nɤ-kɯ tsa kɤ-cit matɕi nɤ-ndi nɯ mɯ́j-xtɕhɯt nɤ}\hspace{5pt}\pcmn{你往左边站一点,不然你右边的那个人站不下(坐不下)}\end{exemple}\end{entrée}

\begin{entrée}{maχpɯn}{}{ⓔmaχpɯn} 
\classe{n} 
\begin{définition}\pfra{stratège, général,}\end{définition}
\begin{définition}\pcmn{军师;将军}\end{définition}
\begin{exemple}\pjya{maχpɯn lo-ndo}\hspace{5pt}\pcmn{他当了军师}\end{exemple}\étymologie{dmag.dpon}\end{entrée}

\begin{entrée}{maχtɕɯ}{}{ⓔmaχtɕɯ} 
\classe{intj} 
\begin{définition}\pfra{je te l'avais bien dit}\end{définition}
\begin{définition}\pcmn{本来应该这样}\end{définition}
\begin{exemple}\pjya{maχtɕɯ tɤ-tɯt-a nɯ mɤ-tɯ-ste kɯ}\hspace{5pt}\pcmn{你怎么没有照我说地去做呢?}\end{exemple}
\begin{exemple}\pjya{maχtɕɯ ma-jɤ-tɯ-ɕe tɤ-tɯt-a ri mɯ́j-tɯ-khɯ tɕe}\hspace{5pt}\pcmn{我本来叫你不要去,但是(你)没有听(果然出了问题)}\end{exemple}\end{entrée}

\begin{entrée}{mɤɕi}{}{ⓔmɤɕi} 
\classe{vs} \paradigme{dir}{thɯ-}
\begin{définition}\pfra{riche}\end{définition}
\begin{définition}\pcmn{富有}\end{définition}
\begin{exemple}\pjya{jiɕqha nɯ kɯ-fse kɯ-mɤɕi me}\hspace{5pt}\pcmn{没有人比他有钱}\end{exemple}
\begin{exemple}\pjya{mbroχpa thɯ-mɤɕi rɯʁgiwa, roŋwa thɯ-mɤɕi rɯkhɤrlɤn, kupa thɯ-mɤɕi rɯstɯnmɯ}\hspace{5pt}\pcmn{牧民富有了就请人念经,农民富有了就修房子,汉族富有了就结婚}\end{exemple}\end{entrée}

\begin{entrée}{mɤɕtʂa}{}{ⓔmɤɕtʂa} 
\classe{adv} \sens{1}
\begin{définition}\pfra{jusqu'à}\end{définition}
\begin{définition}\pcmn{一直到}\end{définition}
\begin{exemple}\pjya{lɤsɤr ɯ-qhu mɤɕtʂa a-ʁa me ɲɯ-ŋu}\hspace{5pt}\pcmn{一直到新年以后我都没有空}\end{exemple}
\begin{exemple}\pjya{nɯ mɤɕtʂa nɯ kɤ-ti mɯ-pɯ-mtsha-ma}\hspace{5pt}\pcmn{我从来没有听过(别人这样)说}\end{exemple}\sens{2}
\begin{définition}\pfra{sinon}\end{définition}
\begin{définition}\pcmn{不然}\end{définition}
\begin{exemple}\pjya{nɯ mɤɕtʂa aʑo mɤ-ɣi-a}\hspace{5pt}\pcmn{不然我是不会来的}\end{exemple}\end{entrée}

\begin{entrée}{mɤdɤmɲɤm}{}{ⓔmɤdɤmɲɤm} 
\classe{n} 
\begin{définition}\pfra{une espèce d'arbrisseau}\end{définition}
\begin{définition}\pcmn{灌木的一种}\end{définition}
\begin{exemple}\pjya{mɤdɤmɲɤm nɯ si kɯ-mbro tsa ci ŋu, ɯ-ru nɯ ra kɯ-pɣi ci ŋu, ɯ-jwaʁ ndɯβ ri dɤn, arŋi, ɯ-si nɯ mɤ-jpum, kɤ-ntɕhoz mɤ-sna ma ndoʁ tɕe mɤ-ngɯt, kɤ-nɯ-βlɯ ma mɤ-sna, ɯ-mɯntoʁ kɯ-ɤɣɯrnɯɕɯr ɲɯ-lɤt tɕe, ɯ-rtaʁ ɯ-kɤχcɤl zɯ kɯ-ndɯ-ndɯβ kɯ-dɯ-dɤn tɯtɯrca ku-ndzoʁ ŋu.}\hspace{5pt}\pcmn{\lien{ⓔmɤdɤmɲɤm}{mɤdɤmɲɤm}是一种长的比较高的树种,树干是灰色的,叶子小而多,是绿色的,树干不粗,不能使用因为脆,不结实。只能用来烧火。开淡红色的花,在树枝的顶端一朵朵地开在一起。}\end{exemple}\end{entrée}

\begin{entrée}{mɤ́ɣrɤz}{}{ⓔmɤ́ɣrɤz} 
\classe{cnj} 
\begin{définition}\pfra{de toute manière, en revanche}\end{définition}
\begin{définition}\pcmn{反而}\end{définition}\end{entrée}

\begin{entrée}{mɤku}{}{ⓔmɤku} 
\classe{vs} \paradigme{dir}{\_}\paradigme{dir}{\_}
\begin{définition}\pfra{être avant}\end{définition}
\begin{définition}\pcmn{以前;在前面}\end{définition}
\begin{définition}\pfra{faire avant}\end{définition}
\begin{définition}\pcmn{先做}\end{définition}
\begin{exemple}\pjya{lɤ-mɤku-a}\hspace{5pt}\pcmn{我在前面}\end{exemple}
\begin{exemple}\pjya{kɯ-nɯsaχsɯ ju-ɣi-j pɯ-ŋu tɕe aʑo ɲɯ-mɤku-a pɯ-ŋu}\hspace{5pt}\pcmn{我们来吃中午餐的时候,我走在前面}\end{exemple}
\begin{exemple}\pjya{a-tʂha ci pɯ-zmɤke pɯ-rke}\hspace{5pt}\pcmn{你先给我倒茶}\end{exemple}
\begin{exemple}\pjya{nɤʑo ɯ-ɲɯ́-tɯ-mbɣom nɤ nɤ-@chepiao kɤ-χtɯ tu-ta-zmɤku jɤɣ}\hspace{5pt}\pcmn{你如果急的话,我可以让你先买票}\end{exemple}\relationsémantique{参考}{\lien{ⓔtɯ-ku}{tɯ-ku}}
\begin{sous-entrée}{zmɤku}{ⓔmɤkuⓝzmɤku} 
\classe{vt} \end{sous-entrée}

\end{entrée}

\begin{entrée}{mɤkɯftshi}{}{ⓔmɤkɯftshi} 
\classe{adv} 
\begin{définition}\pfra{forcer}\end{définition}
\begin{définition}\pcmn{逼迫}\end{définition}
\begin{exemple}\pjya{mɤkɯftshi tú-wɣ-sɯ-ndza-a pɯ-ɕti}\hspace{5pt}\pcmn{他逼我吃}\end{exemple}
\begin{exemple}\pjya{mɯ-pɯ-kɯ-nɯ-cha kɯnɤ, mɤkɯftshi tú-wɣ-znɤma kɯ-ra ɕti}\hspace{5pt}\pcmn{虽然不能做,但是还是被迫做}\end{exemple}
\begin{exemple}\pjya{tɤ-fka-a ɕti ri, mɤkɯftshi ʑo tɤ-ndza-t-a pɯ-ra}\hspace{5pt}\pcmn{我饱了,但还是被迫吃了}\end{exemple}\relationsémantique{参考}{\lien{ⓔftshiⓢ2ⓝsɯftshi}{sɯftshi}}\end{entrée}

\begin{entrée}{mɤlɤn}{}{ⓔmɤlɤn} 
\classe{n} 
\begin{définition}\pfra{absolument}\end{définition}
\begin{définition}\pcmn{一定;必须}\end{définition}\end{entrée}

\begin{entrée}{mɤlmɤl}{}{ⓔmɤlmɤl} 
\classe{idph.2} 
\begin{définition}\pfra{très meuble (terre)}\end{définition}
\begin{définition}\pcmn{形容土地松软的样子}\end{définition}
\begin{exemple}\pjya{tɯji mɤlmɤl ʑo cho-sthɯt-nɯ}\hspace{5pt}\pcmn{他们下种完了,田地很松软}\end{exemple}
\begin{exemple}\pjya{tɯji pɯ-jɤɣ tɕe, tɯji ra mɤlmɤl ʑo ɲɯ-pa}\hspace{5pt}\pcmn{下种完了的时候,整个地面又松软又平整}\end{exemple}\end{entrée}

\begin{entrée}{mɤlɯm}{}{ⓔmɤlɯm} 
\classe{vs} \paradigme{dir}{tɤ-}\paradigme{dir}{thɯ-}
\begin{définition}\pfra{aux dimensions importantes}\end{définition}
\begin{définition}\pcmn{体积大}\end{définition}
\begin{exemple}\pjya{ɯʑo ʁnɯ-pɤrme chɤ-zɣɯt tɕe chɤ-mɯlɯm}\hspace{5pt}\pcmn{他到两岁,变大了}\end{exemple}\relationsémantique{参考}{\lien{ⓔtɯ-lɯm}{tɯ-lɯm}}\end{entrée}

\begin{entrée}{mɤmu}{}{ⓔmɤmu} 
\classe{vi}  
\grammaire{refl} \paradigme{dir}{lɤ-}\paradigme{dir}{tɤ-}\paradigme{dir}{tɤ-}
\begin{définition}\pfra{être la part la plus importante}\end{définition}
\begin{définition}\pcmn{是主要的;占多数}\end{définition}
\begin{définition}\pfra{considérer comme le plus important}\end{définition}
\begin{définition}\pcmn{认为是最重要的}\end{définition}
\begin{exemple}\pjya{kɯre kɤ-rɤma nɯ jiʑora kɤ-mɤmu ɬoʁ}\hspace{5pt}\pcmn{这个工作的重点部分要我们做}\end{exemple}
\begin{exemple}\pjya{jiʑo ji-tɯrme nɯra kɤ-mɤmu ɬoʁ}\hspace{5pt}\pcmn{要以我们的人为重点}\end{exemple}
\begin{exemple}\pjya{a-tɤ-rʑaʁ kɯ-mɤmu nɯ nɯre pɯ-ari ɕti}\hspace{5pt}\pcmn{我的时间主要花在那一方面}\end{exemple}
\begin{exemple}\pjya{tɯ-ci kɤ-tshi a-kɤ-mɤmu ra}\hspace{5pt}\pcmn{你主要还是喝水(不只要吃药)}\end{exemple}
\begin{exemple}\pjya{kutɕu jɤ-tɯ-ɣe tɕe, nɤ-kɤnɤma nɯ kɤ-zmɤmu ra}\hspace{5pt}\pcmn{你既然来到这里,要把工作看成是最重要}\end{exemple}
\begin{exemple}\pjya{ɯʑo kɯ ɯʑo ɯ-ma ntsɯ tu-znɤme ɲɯ-ɕti, tɯʑo tɯma ra kɤ-nɤma mɯ́j-ŋgrɯ}\hspace{5pt}\pcmn{我总是把自己的工作当做是最重要的事情,我们这边的事情就做不成}\end{exemple}
\begin{sous-entrée}{zmɤmu}{ⓔmɤmuⓝzmɤmu} 
\classe{vt} \end{sous-entrée}

\begin{sous-entrée}{ʑɣɤmɤmu}{ⓔmɤmuⓝʑɣɤmɤmu}\end{sous-entrée}

\begin{définition}\pfra{se proposer spontanément pour prendre en charge}\end{définition}
\begin{définition}\pcmn{自己占主要地位;自我推荐}\end{définition}
\begin{exemple}\pjya{jiɕqha nɯ-phe tɯ-rju kɤ-βzu ɲɯ-ra ri, aʑo tɤ-ʑɣɤmɤmu-a}\hspace{5pt}\pcmn{要跟他们说,主要出面的是我(主要的话是我说的)}\end{exemple}
\begin{exemple}\pjya{jiɕqha nɯ-phe tɯ-rju kɤ-βzu ɲɯ-ra ri, aʑo mɤ-ʑɣɤmɤmu-a}\hspace{5pt}\pcmn{要跟他们说,但是我不会主要出面(婉转的意思:那句话不好说)}\end{exemple}\end{entrée}

\begin{entrée}{mɤmbrɯmɤmbrɤt}{}{ⓔmɤmbrɯmɤmbrɤt} 
\classe{adv} 
\begin{définition}\pfra{par intermittence}\end{définition}
\begin{définition}\pcmn{断断续续}\end{définition}\relationsémantique{参考}{\lien{ⓔmbrɤt}{mbrɤt}}\end{entrée}

\begin{entrée}{mɤmbɯr}{}{ⓔmɤmbɯr} 
\classe{vs} \paradigme{dir}{\_}
\begin{définition}\pfra{saillant}\end{définition}
\begin{définition}\pcmn{凸}\end{définition}
\begin{exemple}\pjya{znde ɲɯ-mɤmbɯr tɕe, mbɯt ɲɯ-ŋu}\hspace{5pt}\pcmn{墙有个凸出的地方,快要塌下来了}\end{exemple}
\begin{exemple}\pjya{pjɤ-mɤmbɯr}\hspace{5pt}\pcmn{地上凸出来了}\end{exemple}\étymologie{ⁿbur}\end{entrée}

\begin{entrée}{mɤnɯɕaŋ}{}{ⓔmɤnɯɕaŋ} 
\classe{adv} 
\begin{définition}\pfra{je ne sais pas (expression toute faite, emprunt au situ)}\end{définition}
\begin{définition}\pcmn{我不知道(四土话借词)}\end{définition}\end{entrée}

\begin{entrée}{mɤŋgɯ}{}{ⓔmɤŋgɯ} 
\classe{vi} \paradigme{dir}{thɯ-}
\begin{définition}\pfra{être à l'intérieur}\end{définition}
\begin{définition}\pcmn{在里面}\end{définition}
\begin{définition}\pfra{porter à l'intérieur}\end{définition}
\begin{définition}\pcmn{穿在里面}\end{définition}
\begin{exemple}\pjya{ɯʑo tɯrme kɯ-mɤŋgɯ ɕti tɕe ɯ-rŋa tu.}\hspace{5pt}\pcmn{他是重要的人物,他面子很大}\end{exemple}
\begin{exemple}\pjya{stu kɯ-mɤŋgɯ}\hspace{5pt}\pcmn{最里层}\end{exemple}
\begin{exemple}\pjya{kɯki tɯ-ŋga ki chɯ́-wɣ-z-mɤŋgɯ ɲɯ-ra}\hspace{5pt}\pcmn{这件衣服要穿在里面}\end{exemple}
\begin{exemple}\pjya{ki tɯ-ŋga ki chɯ́-wɣ-z-mɤŋgɯ tɕe sɤscit}\hspace{5pt}\pcmn{这件衣服穿在里面舒服}\end{exemple}\relationsémantique{反义词}{\lien{ⓔmɤpɕi}{mɤpɕi}}\relationsémantique{参考}{\lien{ⓔɯ-ŋgɯ}{ɯ-ŋgɯ}}
\begin{sous-entrée}{zmɤŋgɯ}{ⓔmɤŋgɯⓝzmɤŋgɯ} 
\classe{vt} \end{sous-entrée}

\end{entrée}

\begin{entrée}{mɤpaχcɤl}{}{ⓔmɤpaχcɤl} 
\classe{vs} \paradigme{dir}{tɤ-}
\begin{définition}\pfra{être au centre}\end{définition}
\begin{définition}\pcmn{在中间}\end{définition}
\begin{exemple}\pjya{ɯʑo to-mɤpaχcɤl}\hspace{5pt}\pcmn{他站在中间了}\end{exemple}\relationsémantique{参考}{\lien{ⓔmɤχcɤl}{mɤχcɤl}}\end{entrée}

\begin{entrée}{mɤpɤrthɤβ}{}{ⓔmɤpɤrthɤβ} 
\classe{vi}  
\grammaire{refl} \paradigme{dir}{tɤ-}\paradigme{dir}{tɤ-}
\begin{définition}\pfra{être entre deux}\end{définition}
\begin{définition}\pcmn{在两个的中间}\end{définition}
\begin{exemple}\pjya{tɯrme χsɯm a-pɯ-tu-j, ɯ-χcɤl nɯ kɯ-mɤpɤrthɤβ}\hspace{5pt}\pcmn{我们有三个人,中间的那个在(我们俩)中间}\end{exemple}
\begin{exemple}\pjya{kɯ-mɤku nɯ tshɯraŋ ɲɯ-ŋu, kɯ-maqhu nɯ lɤβzaŋ ɲɯ-ŋu, waŋtɕin ɲɯ-mɤpɤrthɤβ}\hspace{5pt}\pcmn{前面的是次让,后面的是罗桑,王金在中间}\end{exemple}\relationsémantique{参考}{\lien{ⓔɯ-pɤrthɤβ}{ɯ-pɤrthɤβ}}
\begin{sous-entrée}{ʑɣɤmɤpɤrthɤβ}{ⓔmɤpɤrthɤβⓝʑɣɤmɤpɤrthɤβ} 
\classe{vi} \end{sous-entrée}

\begin{définition}\pfra{se mettre au milieu}\end{définition}
\begin{définition}\pcmn{走到中间}\end{définition}
\begin{exemple}\pjya{to-ʑɣɤmɤpɤrthɤβ}\hspace{5pt}\pcmn{他走到中间了}\end{exemple}\end{entrée}

\begin{entrée}{mɤpɕi}{}{ⓔmɤpɕi} 
\classe{vi}  
\grammaire{denom} 
\begin{définition}\pfra{se trouver à l'extérieur}\end{définition}
\begin{définition}\pcmn{在外面}\end{définition}
\begin{exemple}\pjya{stu kɯ-mɤpɕi}\hspace{5pt}\pcmn{最外层的}\end{exemple}\relationsémantique{参考}{\lien{ⓔɯ-pɕi}{ɯ-pɕi}}
\begin{sous-entrée}{zmɤpɕi}{ⓔmɤpɕiⓝzmɤpɕi} 
\classe{vt} \sens{1}\paradigme{dir}{thɯ-}
\begin{définition}\pfra{porter à l'extérieur}\end{définition}
\begin{définition}\pcmn{穿在外面}\end{définition}
\begin{exemple}\pjya{ki tɯ-ŋga ki chɯ́-wɣ-z-mɤpɕi tɕe mpɕɤr}\hspace{5pt}\pcmn{这件衣服穿在外面就美观}\end{exemple}\end{sous-entrée}

\sens{2}
\begin{définition}\pfra{considérer comme un étranger}\end{définition}
\begin{définition}\pcmn{当外人}\end{définition}
\begin{exemple}\pjya{nɯ́-wɣ-zmɤpɕi-a-nɯ}\hspace{5pt}\pcmn{他们把我当外人}\end{exemple}\relationsémantique{反义词}{\lien{ⓔmɤŋgɯ}{mɤŋgɯ}}\relationsémantique{参考}{\lien{ⓔɯ-pɕi}{ɯ-pɕi}}\end{entrée}

\begin{entrée}{mɤpɕoʁ}{}{ⓔmɤpɕoʁ} 
\classe{n} 
\begin{définition}\pfra{l'envers, l'autre côté}\end{définition}
\begin{définition}\pcmn{反面}\end{définition}\étymologie{ma.pʰʲogs}\end{entrée}

\begin{entrée}{mɤrdɤli}{}{ⓔmɤrdɤli} 
\classe{n} 
\begin{définition}\pfra{personne sans foi ni loi}\end{définition}
\begin{définition}\pcmn{无法无天的人}\end{définition}
\begin{exemple}\pjya{mɤrdɤli ɲɤ-ɕe qhe, nɯ ɯ-qhu kɤ-ndzɯmbra mɤ-nɤjtshɯ}\hspace{5pt}\pcmn{他变得无法无天,从此以后,再教育也没有用}\end{exemple}\end{entrée}

\begin{entrée}{mɤrdom}{}{ⓔmɤrdom} 
\classe{n} 
\begin{définition}\pfra{fléau}\end{définition}
\begin{définition}\pcmn{连枷}\end{définition}
\begin{exemple}\pjya{mɤrdom ɯ-mu}\hspace{5pt}\pcmn{连枷的把手}\end{exemple}
\begin{exemple}\pjya{mɤrdom-mɲa}\hspace{5pt}\pcmn{连枷打粮食的部分}\end{exemple}\end{entrée}

\begin{entrée}{mɤrnɤsɤŋo}{}{ⓔmɤrnɤsɤŋo} 
\classe{n} 
\begin{définition}\pfra{mal comprendre une parole}\end{définition}
\begin{définition}\pcmn{听错}\end{définition}
\begin{exemple}\pjya{ɯʑo kɯ ɲɯ-ti ri, mɤrnɤsɤŋo ɲɤ-βzu-t-a}\hspace{5pt}\pcmn{我听错了他讲的话}\end{exemple}\relationsémantique{参考}{\lien{ⓔtɯ-rna}{tɯ-rna}}\relationsémantique{参考}{\lien{}{sɤŋo₁}}\end{entrée}

\begin{entrée}{mɤro}{}{ⓔmɤro} 
\classe{n} 
\begin{définition}\pfra{endroit sur lequel on fait sécher la nourriture}\end{définition}
\begin{définition}\pcmn{粮架}\end{définition}
\begin{exemple}\pjya{mɤro nɯ ɕoŋtɕa mɤ-kɯ-jpum tsa pjɯ́-wɣ-phɯt tɕe, choʁe chɯ́-wɣ-βʑoʁ tɕe ɲɯ́-wɣ-sɤɕpɯɕpa tɕe kɯ-spoʁ χsɯm tú-wɣ-sɤʑɯrja, tɕe nɯ kɯ-spoʁ ɯ-ŋgɯ rorʁe nɯ mɤro ɯ-kɯ-spoʁ ɯ-ŋgɯ ɲɯ-ɕe kɯ-tɕhɯt ɯ-tshɤt ma mɤ-kɯ-jpum pjɯ-ŋu ra. tɕe mɤ-ro ɯ-thoʁ pjɯ́-wɣ-lɣa tɕe ɲɯ́-wɣ-sɤʑɯrja tɕe pjɯ́-wɣ-sɤtsa tɕe, ɯ-kɯ-spoʁ mɤro raŋri ɣɯ tú-wɣ-z-nɯstɯ-stu tɕe rorʁe ɲɯ́-wɣ-rʁe tɕe tɤ-rɤku kú-wɣ-sɤro ŋu. tɕe nɯ tɤ-rɤku tú-wɣ-rɤwum cho tú-wɣ-sɯɣ-rom kɤ-nɯmga ŋu.}\hspace{5pt}\pcmn{粮架就是把不太粗的木料砍下来,左右两边削下来,使它变扁,然后顺着木料的长度打三个洞,然后在洞里穿木棒,木棒的粗度要刚好配合洞的大小。在地面挖洞把粮架插在里面,这样排成一行,使粮架的每一个洞对端,然后穿木棒就把粮食架上去。粮架的作用是收拾粮食让它干。}\end{exemple}\end{entrée}

\begin{entrée}{mɤrom}{}{ⓔmɤrom} 
\classe{vs} \paradigme{dir}{thɯ-}\paradigme{dir}{tɤ-}
\begin{définition}\pfra{enfler}\end{définition}
\begin{définition}\pcmn{肿}\end{définition}
\begin{exemple}\pjya{to-mɤrom}\hspace{5pt}\pcmn{肿了}\end{exemple}
\begin{exemple}\pjya{a-mi thɯ-mɤrom}\hspace{5pt}\pcmn{我的脚肿了}\end{exemple}\end{entrée}

\begin{entrée}{mɤrpaʁ}{}{ⓔmɤrpaʁ} 
\classe{vt}  
\grammaire{denom} \paradigme{dir}{tɤ-}
\begin{définition}\pfra{porter à l’épaule}\end{définition}
\begin{définition}\pcmn{扛在肩上}\end{définition}
\begin{exemple}\pjya{tɤ-mɤrpaʁ-a}\hspace{5pt}\pcmn{我扛了}\end{exemple}
\begin{exemple}\pjya{ki laχtɕha ki tɤ-mɤrpaʁ}\hspace{5pt}\pcmn{你把这根东西扛在肩上}\end{exemple}\relationsémantique{同义词}{\lien{ⓔnɤrpaʁku}{nɤrpaʁku}}\relationsémantique{参考}{\lien{ⓔtɯ-rpaʁ}{tɯ-rpaʁ}}\end{entrée}

\begin{entrée}{mɤrtsaβ}{}{ⓔmɤrtsaβ} 
\classe{vs} \paradigme{dir}{nɯ-}\paradigme{dir}{pɯ-}\paradigme{dir}{pɯ-}
\begin{définition}\pfra{piquant}\end{définition}
\begin{définition}\pcmn{辣}\end{définition}
\begin{définition}\pfra{rendre piquant}\end{définition}
\begin{définition}\pcmn{把辣椒加多}\end{définition}
\begin{définition}\pfra{trouver trop piquant}\end{définition}
\begin{définition}\pcmn{觉得辣}\end{définition}
\begin{exemple}\pjya{ɲɯ-tɯ-mɤrtsaβ}\hspace{5pt}\pcmn{你脾气很泼辣}\end{exemple}
\begin{exemple}\pjya{pɯ-zmɤrtsaβ-a}\hspace{5pt}\pcmn{我把辣椒加多了}\end{exemple}
\begin{exemple}\pjya{a-jaʁ na-qhrɯt tɕe na-zmɤrtsaβ}\hspace{5pt}\pcmn{我被刮到手了,感觉辣乎乎的}\end{exemple}
\begin{exemple}\pjya{a-jaʁ mtshalu kɯ ka-mtsɯɣ tɕe na-zmɤrtsaβ}\hspace{5pt}\pcmn{我的手碰到荨麻就很痛}\end{exemple}
\begin{sous-entrée}{zmɤrtsaβ}{ⓔmɤrtsaβⓝzmɤrtsaβ} 
\classe{vt} \end{sous-entrée}

\begin{sous-entrée}{nɤmɤrtsaβ}{ⓔmɤrtsaβⓝnɤmɤrtsaβ} 
\classe{vt}  
\grammaire{trop} \end{sous-entrée}

\end{entrée}

\begin{entrée}{mɤrʑaβ}{}{ⓔmɤrʑaβ} 
\classe{vi}  
\grammaire{denom} \paradigme{dir}{nɯ-}
\begin{définition}\pfra{se marier (fille)}\end{définition}
\begin{définition}\pcmn{嫁人}\end{définition}
\begin{exemple}\pjya{ɯ-me nɯ-mɤrʑaβ}\hspace{5pt}\pcmn{他女儿结了婚}\end{exemple}\relationsémantique{参考}{\lien{ⓔtɤ-rʑaβ}{tɤ-rʑaβ}}\end{entrée}

\begin{entrée}{mɤsɲɯm}{}{ⓔmɤsɲɯm} 
\classe{vs} \paradigme{dir}{tɤ-}
\begin{définition}\pfra{aimer manger les tissus, les cordes (bovidé)}\end{définition}
\begin{définition}\pcmn{喜欢吃麻织品,毛织品,皮革(牛)}\end{définition}
\begin{exemple}\pjya{fsapaʁ ɲɯ-mɤsɲɯm}\hspace{5pt}\pcmn{牲畜爱吃纺织品}\end{exemple}\end{entrée}

\begin{entrée}{mɤstɤkɤmi}{}{ⓔmɤstɤkɤmi} 
\classe{adv} 
\begin{définition}\pfra{de façon incompréhensible}\end{définition}
\begin{définition}\pcmn{莫名其妙}\end{définition}
\begin{exemple}\pjya{tʂu mɤstɤkɤmi ʑo pjɤ-mbɯt}\hspace{5pt}\pcmn{路莫名其妙地塌下来了}\end{exemple}\end{entrée}

\begin{entrée}{mɤtɕɯ}{}{ⓔmɤtɕɯ} 
\classe{vi} \paradigme{dir}{jɤ-}
\begin{définition}\pfra{aller dans la maison de son épouse après le mariage}\end{définition}
\begin{définition}\pcmn{入赘(在别人家当女婿)}\end{définition}
\begin{exemple}\pjya{nɤʑo z-jɤ-tɯ-mɤtɕɯ ɕti}\hspace{5pt}\pcmn{你是去上门的}\end{exemple}
\begin{exemple}\pjya{nɤʑo ɣɯ-jɤ-tɯ-mɤtɕɯ ɕti}\hspace{5pt}\pcmn{你是来上门的}\end{exemple}\relationsémantique{参考}{\lien{ⓔtɤ-tɕɯ}{tɤ-tɕɯ}}\end{entrée}

\begin{entrée}{mɤtsamɤmu}{}{ⓔmɤtsamɤmu} 
\classe{n} 
\begin{définition}\pfra{nouvelle famille}\end{définition}
\begin{définition}\pcmn{新建立的家庭}\end{définition}\end{entrée}

\begin{entrée}{mɤtsomɤmu}{}{ⓔmɤtsomɤmu} 
\classe{n} 
\begin{définition}\pfra{conflit avec la belle-famille}\end{définition}
\begin{définition}\pcmn{婆媳关系不和(的家庭)}\end{définition}
\begin{exemple}\pjya{ʑara nɯ-mɤtsomɤmu ɲɯ-thɯ}\hspace{5pt}\pcmn{他们家的婆媳关系非常严重}\end{exemple}\end{entrée}

\begin{entrée}{mɤtɯmaʁri}{}{ⓔmɤtɯmaʁri} 
\classe{cnj} 
\begin{définition}\pfra{comme ... le dit}\end{définition}
\begin{définition}\pcmn{正如……所说的}\end{définition}
\begin{exemple}\pjya{nɤj mɤtɯmaʁri = nɤj mɤ-tɯ-ti maʁ ri}\hspace{5pt}\pcmn{正如你说的}\end{exemple}
\begin{exemple}\pjya{ɯʑo mɤtɯmaʁri = ɯʑo mɤ-ti maʁ ri}\hspace{5pt}\pcmn{正如他说的}\end{exemple}\relationsémantique{参考}{\lien{ⓔti}{ti}}\relationsémantique{参考}{\lien{ⓔmaʁⓗ1}{maʁ₁}}\end{entrée}

\begin{entrée}{mɤχcɤl}{}{ⓔmɤχcɤl} 
\classe{vi}  
\grammaire{denom} \paradigme{dir}{tɤ-}
\begin{définition}\pfra{être au milieu}\end{définition}
\begin{définition}\pcmn{在中间}\end{définition}
\begin{exemple}\pjya{nɤʑo jɤ-mɤku tɕe aj tu-mɤχcal-a (tu-mɯpɤχcal-a)}\hspace{5pt}\pcmn{你走前面吧,我走中间}\end{exemple}\relationsémantique{参考}{\lien{}{mɤpɤχcɤl}}\relationsémantique{参考}{\lien{ⓔɯ-χcɤl}{ɯ-χcɤl}}\end{entrée}

\begin{entrée}{mɤzɯr}{}{ⓔmɤzɯr} 
\classe{vi}  
\grammaire{denom} \paradigme{dir}{lɤ-}\paradigme{dir}{thɯ-}\paradigme{dir}{kɤ-}\paradigme{dir}{nɯ-}\sens{1}
\begin{définition}\pfra{être sur le côté}\end{définition}
\begin{définition}\pcmn{在旁边}\end{définition}
\begin{exemple}\pjya{pɯ-ŋke-j pɯ-ŋu tɕe, aʑo ɲɯ-mɤzɯr-a pɯ-ŋu (chɯ-mɤzɯr-a pɯ-ŋu)}\hspace{5pt}\pcmn{我们走的时候,我在边缘}\end{exemple}\sens{2}
\begin{définition}\pfra{reculé (endroit)}\end{définition}
\begin{définition}\pcmn{偏远;偏僻}\end{définition}\étymologie{zur}\end{entrée}

\begin{entrée}{mɤʑɯ}{}{ⓔmɤʑɯ} 
\classe{adv} 
\begin{définition}\pfra{encore}\end{définition}
\begin{définition}\pcmn{还有}\end{définition}
\begin{exemple}\pjya{jisŋi mɤʑɯ nɤ-kɤ-thu ɯ́-tu}\hspace{5pt}\pcmn{你今天还有问题吗?}\end{exemple}
\begin{exemple}\pjya{mɤʑɯ tu-ti-a ŋu nɤ?}\hspace{5pt}\pcmn{我再说一次吗?}\end{exemple}
\begin{exemple}\pjya{mɤʑɯ pɯ-pɯ-ŋu nɤ}\hspace{5pt}\pcmn{还有就是}\end{exemple}\relationsémantique{参考}{\lien{}{ʑɯ}}\end{entrée}

\begin{entrée}{mba}{}{ⓔmba} 
\classe{vs} \paradigme{dir}{nɯ-}\sens{1}
\begin{définition}\pfra{mince}\end{définition}
\begin{définition}\pcmn{薄}\end{définition}
\begin{exemple}\pjya{ɕoʁɕoʁ ɲɯ-mba}\hspace{5pt}\pcmn{纸很薄}\end{exemple}
\begin{exemple}\pjya{nɤ-ŋga ɲɯ-mba}\hspace{5pt}\pcmn{你的衣服很薄}\end{exemple}
\begin{exemple}\pjya{ɯ-sɯm ɲɯ-mba}\hspace{5pt}\pcmn{他心软}\end{exemple}\sens{2}
\begin{définition}\pfra{peu profond}\end{définition}
\begin{définition}\pcmn{浅}\end{définition}
\begin{sous-entrée}{ɣɤmba}{ⓔmbaⓢ2ⓝɣɤmba} 
\classe{vt} 
\begin{définition}\pfra{rendre fin}\end{définition}
\begin{définition}\pcmn{弄薄}\end{définition}\relationsémantique{反义词}{\lien{ⓔjaʁ}{jaʁ}}\end{sous-entrée}

\end{entrée}

\begin{entrée}{mbala}{}{ⓔmbala} 
\classe{n} 
\begin{définition}\pfra{bœuf}\end{définition}
\begin{définition}\pcmn{黄牛}\end{définition}\end{entrée}

\begin{entrée}{mbalalu}{}{ⓔmbalalu} 
\classe{n} 
\begin{définition}\pfra{année du bœuf}\end{définition}
\begin{définition}\pcmn{牛年}\end{définition}\end{entrée}

\begin{entrée}{mbalapɯ}{}{ⓔmbalapɯ} 
\classe{n} 
\begin{définition}\pfra{petit de vache}\end{définition}
\begin{définition}\pcmn{小黄牛}\end{définition}\end{entrée}

\begin{entrée}{mbanaʁxtsa}{}{ⓔmbanaʁxtsa} 
\classe{n} 
\begin{définition}\pfra{botte en cuir noir}\end{définition}
\begin{définition}\pcmn{黑皮鞋}\end{définition}\end{entrée}

\begin{entrée}{mbaqhu}{}{ⓔmbaqhu} 
\classe{n}  
\grammaire{n.lieu} 
\begin{définition}\pfra{l'un des hameaux de Gyutshapa}\end{définition}
\begin{définition}\pcmn{二茶村的大队之一}\end{définition}\end{entrée}

\begin{entrée}{mbarkhom}{}{ⓔmbarkhom} 
\classe{n}  
\grammaire{n.lieu} 
\begin{définition}\pfra{Mbarkham}\end{définition}
\begin{définition}\pcmn{马尔康}\end{définition}\end{entrée}

\begin{entrée}{mbarqhi}{}{ⓔmbarqhi} 
\classe{n} 
\begin{définition}\pfra{distance}\end{définition}
\begin{définition}\pcmn{距离}\end{définition}
\begin{exemple}\pjya{mbarqhi ɯ-ɲɯ́-βdi kɯ tú-wɣ-tshɤt ɲɯ-ra}\hspace{5pt}\pcmn{要试一下(话筒离嘴的)距离合不合适}\end{exemple}\relationsémantique{参考}{\lien{ⓔarqhi}{arqhi}}\relationsémantique{参考}{\lien{ⓔarmbat}{armbat}}\end{entrée}

\begin{entrée}{mbaʁ}{}{ⓔmbaʁ} 
\classe{vi} \paradigme{dir}{nɯ-}\paradigme{dir}{thɯ-}\sens{1}
\begin{définition}\pfra{se casser}\end{définition}
\begin{définition}\pcmn{破裂,裂开}\end{définition}
\begin{exemple}\pjya{rkɤsnom ɯ-srɯβ ɲɤ-mbaʁ}\hspace{5pt}\pcmn{裤子(的针脚)裂缝了}\end{exemple}
\begin{exemple}\pjya{tɯ-ŋga ɯ-srɯβ ɲɤ-mbaʁ}\hspace{5pt}\pcmn{衣服裂缝了}\end{exemple}
\begin{exemple}\pjya{ɟu chɤ-mbaʁ}\hspace{5pt}\pcmn{竹子裂开了}\end{exemple}
\begin{exemple}\pjya{ɕoŋtɕa ɲɤ-mbaʁ}\hspace{5pt}\pcmn{木料裂开了}\end{exemple}
\begin{exemple}\pjya{znde ɲɤ-mbaʁ}\hspace{5pt}\pcmn{墙裂开了}\end{exemple}\sens{2}
\begin{définition}\pfra{avoir une ouverture (habit)}\end{définition}
\begin{définition}\pcmn{叉开(衣服)}\end{définition}
\begin{exemple}\pjya{ɕɤntsɯt nɯ χchoʁe ʑo ɯ-ndo tu-kɯ-mbaʁ tu}\hspace{5pt}\pcmn{女式长衫的下面左右两边都是叉开的}\end{exemple}\end{entrée}

\begin{entrée}{mbaʁŋgu}{}{ⓔmbaʁŋgu} 
\classe{n} 
\begin{définition}\pfra{masque de danse}\end{définition}
\begin{définition}\pcmn{跳神时戴的面具}\end{définition}\étymologie{ⁿbag.mgo}\end{entrée}

\begin{entrée}{mbat}{}{ⓔmbat} 
\classe{vs} \paradigme{dir}{tɤ-}\sens{1}
\begin{définition}\pfra{léger (travail)}\end{définition}
\begin{définition}\pcmn{轻松}\end{définition}
\begin{exemple}\pjya{(kɤ-nɤtsoʁ) mɯ́j-mbat ma ɣɯ-lɣa ra}\hspace{5pt}\pcmn{找人参果不容易,因为要挖地}\end{exemple}
\begin{exemple}\pjya{kɤ-ɕe ɲɯ-mbat}\hspace{5pt}\pcmn{去很轻松}\end{exemple}\sens{2}
\begin{définition}\pfra{être presque fini}\end{définition}
\begin{définition}\pcmn{快要没有了}\end{définition}
\begin{exemple}\pjya{tɤŋe tɤ-mbat / ɲɤ-mbat}\hspace{5pt}\pcmn{太阳快要落山了}\end{exemple}
\begin{exemple}\pjya{nɤ-tɤ-rʑaʁ tɤ-mbat}\hspace{5pt}\pcmn{你快没有时间了}\end{exemple}
\begin{exemple}\pjya{kɤ-ndza to-mbat}\hspace{5pt}\pcmn{快没有食物了}\end{exemple}\sens{3}\paradigme{dir}{tɤ-}
\begin{définition}\pfra{bon marché}\end{définition}
\begin{définition}\pcmn{便宜}\end{définition}
\begin{définition}\pfra{baisser le prix}\end{définition}
\begin{définition}\pcmn{降价}\end{définition}
\begin{exemple}\pjya{ɯ-phɯ ɲɯ-mbat}\hspace{5pt}\pcmn{很便宜}\end{exemple}
\begin{exemple}\pjya{(ɯ-phɯ) tɤ-ɣɤmba-t-a}\hspace{5pt}\pcmn{我减价了}\end{exemple}\relationsémantique{参考}{\lien{ⓔstuⓗ1}{stu₁}}
\begin{sous-entrée}{ɣɤmbat}{ⓔmbatⓢ3ⓝɣɤmbat} 
\classe{vt}  
\grammaire{caus} \end{sous-entrée}

\end{entrée}

\begin{entrée}{mbɤβ}{₁}{ⓔmbɤβⓗ1} 
\classe{vi} \paradigme{dir}{kɤ-}\paradigme{dir}{kɤ-}
\begin{définition}\pfra{avoir fini de fermenter,avoir été distillé (alcool fort)}\end{définition}
\begin{définition}\pcmn{酿出来(酒、奶渣等)}\end{définition}
\begin{définition}\pfra{faire fermenter}\end{définition}
\begin{définition}\pcmn{把……酿出来}\end{définition}
\begin{exemple}\pjya{cha ko-mbɤβ}\hspace{5pt}\pcmn{酒酿出来了}\end{exemple}
\begin{exemple}\pjya{tɕhɯrwa wuma ɲɯ-mbɤβ}\hspace{5pt}\pcmn{奶渣酿出来}\end{exemple}
\begin{exemple}\pjya{cha kɤ-sɯɣmbɤβ mɯ́j-khɯ}\hspace{5pt}\pcmn{没有办法把酒酿出来}\end{exemple}\relationsémantique{同义词}{\lien{ⓔaβzu}{aβzu}}
\begin{sous-entrée}{sɯɣmbɤβ}{ⓔmbɤβⓗ1ⓝsɯɣmbɤβ} 
\classe{vt} \end{sous-entrée}

\end{entrée}

\begin{entrée}{mbɤβ}{₂}{ⓔmbɤβⓗ2} 
\classe{vi} \paradigme{dir}{pɯ-}\sens{1}
\begin{définition}\pfra{se calmer, se taire(d'une foule)}\end{définition}
\begin{définition}\pcmn{平静下来(喧哗的人)}\end{définition}
\begin{exemple}\pjya{mkhɤrmaŋ ra pjɤ-mbɤβ-nɯ}\hspace{5pt}\pcmn{老百姓平静下来了}\end{exemple}\sens{2}
\begin{définition}\pfra{se refroidir (eau bouillante)}\end{définition}
\begin{définition}\pcmn{冷却(开水)}\end{définition}
\begin{exemple}\pjya{tɯthɯ ɯ-ŋgɯ tɯ-ci ɲɯ-ɤla tɕe mbuz ɲɯ-ŋu, tɕe tɯ-ci kɯ-mɯɕtaʁ kɤ-lat-a tɕe pɯ-mbɤβ}\hspace{5pt}\pcmn{锅里的水开了,快要溢出来,所以我倒了一点冷水冷却一下}\end{exemple}\sens{3}
\begin{définition}\pfra{être bien peignée (chevelure, vers le bas)}\end{définition}
\begin{définition}\pcmn{(头发)整齐,梳理好了的}\end{définition}
\begin{exemple}\pjya{nɤ-ku pɯ-sɤɕɤt ma a-pɯ-mbɤβ tɕe ʁzraŋʁzraŋ a-mɤ-pɯ-pa}\hspace{5pt}\pcmn{梳一下头,头发要整齐,不要乱蓬蓬的}\end{exemple}\end{entrée}

\begin{entrée}{mbɤβ}{₃}{ⓔmbɤβⓗ3} 
\classe{vi} \paradigme{dir}{pɯ-}
\begin{définition}\pfra{camper}\end{définition}
\begin{définition}\pcmn{宿营}\end{définition}
\begin{exemple}\pjya{mbroχpa ra nɯre ri pjɤ-mbɤβ-nɯ}\hspace{5pt}\pcmn{草地人在那里宿营了}\end{exemple}\end{entrée}

\begin{entrée}{mbɤr}{₁}{ⓔmbɤrⓗ1} 
\classe{vs} \paradigme{dir}{pɯ-}
\begin{définition}\pfra{bas, petit}\end{définition}
\begin{définition}\pcmn{低;矮}\end{définition}
\begin{exemple}\pjya{nɤʑo ɲɯ-tɯ-mbro, aʑo ɲɯ-mbar-a}\hspace{5pt}\pcmn{你很高,我很矮}\end{exemple}
\begin{exemple}\pjya{tɤ-pɤtso kɯ-mbro ɣɤʑu, kɯ-mbɤr ɣɤʑu}\hspace{5pt}\pcmn{有的孩子长得高,有的长得矮}\end{exemple}
\begin{exemple}\pjya{ɯ-phoŋbu ɲɯ-mbɤr}\hspace{5pt}\pcmn{他身材很矮}\end{exemple}\relationsémantique{参考}{\lien{ⓔɣɤmbɤr}{ɣɤmbɤr}}\relationsémantique{反义词}{\lien{ⓔmbroⓗ1}{mbro₁}}\end{entrée}

\begin{entrée}{mbɤr}{₂}{ⓔmbɤrⓗ2} 
\classe{vi}  
\grammaire{acaus} \paradigme{dir}{pɯ-}
\begin{définition}\pfra{bien épousseté}\end{définition}
\begin{définition}\pcmn{拍得干净}\end{définition}\relationsémantique{参考}{\lien{ⓔsɤphɤr}{sɤphɤr}}\end{entrée}

\begin{entrée}{mbɤt}{}{ⓔmbɤt} 
\classe{interj} 
\begin{définition}\pfra{vite!}\end{définition}
\begin{définition}\pcmn{快点!}\end{définition}
\begin{exemple}\pjya{mbɤt, tɤ-mda!}\hspace{5pt}\pcmn{快点,时间到了}\end{exemple}\end{entrée}

\begin{entrée}{mbɤxɕɯβ}{}{ⓔmbɤxɕɯβ} 
\classe{n} 
\begin{définition}\pfra{une plante}\end{définition}
\begin{définition}\pcmn{植物的一种}\end{définition}
\begin{exemple}\pjya{mbɤxɕɯβ nɯ sɯjno kɯ-xtɕi ci ŋu, ɯ-ru ɣɯrni, ɯ-jwaʁ rʁom, rɕɯβrɕɯβ ʑo pa, ɯ-mɯntoʁ qarŋe, pjɯ́-wɣ-qlɯt tɕe ɯ-lu tu, paʁndza sna, tɯrme kɤ-ndza mɤ-sna.}\hspace{5pt}\pcmn{\lien{ⓔmbɤxɕɯβ}{mbɤxɕɯβ}是一种矮小的植物,茎是红色的,叶子像砂纸一样粗糙,花是黄色的,把茎折断时有乳汁,可以喂猪,人不能吃}\end{exemple}\end{entrée}

\begin{entrée}{mbe}{}{ⓔmbe} 
\classe{vs} \paradigme{dir}{nɯ-}
\begin{définition}\pfra{ancien}\end{définition}
\begin{définition}\pcmn{旧}\end{définition}
\begin{exemple}\pjya{tɯ-ŋga ɲɯ-mbe}\hspace{5pt}\pcmn{衣服是旧的}\end{exemple}\relationsémantique{反义词}{\lien{ⓔɕɤɣⓗ2}{ɕɤɣ}}\end{entrée}

\begin{entrée}{mbɣaʁ}{}{ⓔmbɣaʁ} 
\classe{vi}  
\grammaire{acaus} \paradigme{dir}{\_}
\begin{définition}\pfra{se retourner}\end{définition}
\begin{définition}\pcmn{翻身}\end{définition}
\begin{exemple}\pjya{@qiche cho-mbɣaʁ}\hspace{5pt}\pcmn{汽车翻车了}\end{exemple}
\begin{sous-entrée}{ɣɤmbɣaʁ}{ⓔmbɣaʁⓝɣɤmbɣaʁ} 
\classe{vs}  
\grammaire{facil} 
\begin{définition}\pfra{se retourner facilement (faire facilement un tonneau, d'une voiture)}\end{définition}
\begin{définition}\pcmn{容易翻车}\end{définition}
\begin{exemple}\pjya{ɲchɣaʁʑɤr kɯ-tu tʂu tɕe, @qiche kɤ-lɤt ɲɯ-sɤɣʑɯr ma ɲchɣaʁʑɤr nɯ mɯ́j-saχsɤl tɕe ɲɯ-ɣɤmbɣaʁ (chɯ-mbɣaʁ ɲɯ-mbat)}\hspace{5pt}\pcmn{路面结冰的时候,开车非常危险,因为结冰的道路不明显,容易翻车}\end{exemple}\relationsémantique{参考}{\lien{ⓔnɤmbɣaʁlaʁ}{nɤmbɣaʁlaʁ}}\relationsémantique{参考}{\lien{ⓔpɣaʁ}{pɣaʁ}}\relationsémantique{参考}{\lien{ⓔapɣaʁsci}{apɣaʁsci}}\end{sous-entrée}

\end{entrée}

\begin{entrée}{mbɣɤjroʁ}{}{ⓔmbɣɤjroʁ} 
\classe{n} 
\begin{définition}\pfra{sillon}\end{définition}
\begin{définition}\pcmn{犁沟}\end{définition}\end{entrée}

\begin{entrée}{mbɣɤru}{}{ⓔmbɣɤru} 
\classe{n} 
\begin{définition}\pfra{partie de la charrue}\end{définition}
\begin{définition}\pcmn{犁杆}\end{définition}
\begin{exemple}\pjya{mbɣopɤl ɯ-taʁ chɯ́-wɣ-tshoʁ tɕe ɯ-ɕnɤz nɯ stuxsi ɯ-taʁ lú-wɣ-βraʁ tɕe nɯ mbɣɤru rmi}\hspace{5pt}\pcmn{固定铧头的部分上面再装一根木杆,木杆的另一头拴在牛轭上,这根木杆叫犁干。}\end{exemple}\end{entrée}

\begin{entrée}{mbɣɤsroʁ}{}{ⓔmbɣɤsroʁ} 
\classe{n} 
\begin{définition}\pfra{partie de la charrue}\end{définition}
\begin{définition}\pcmn{犁的组成部分之一}\end{définition}
\begin{exemple}\pjya{mbɣɤsroʁ nɯ mbɣɤru cho mbɣopɤl ni ndʑi-kɯ-ɣɯthaʁ ŋu}\hspace{5pt}\pcmn{\lien{ⓔmbɣɤsroʁ}{mbɣɤsroʁ}是连接犁头和主干和的部分。}\end{exemple}\end{entrée}

\begin{entrée}{mbɣɤtɕɯkala}{}{ⓔmbɣɤtɕɯkala} 
\classe{n}  
\grammaire{n.lieu} 
\begin{définition}\pfra{montagne du village de Mangi}\end{définition}
\begin{définition}\pcmn{蒙岩村的一座山之一}\end{définition}\end{entrée}

\begin{entrée}{mbɣo}{}{ⓔmbɣo} 
\classe{n} 
\begin{définition}\pfra{charrue}\end{définition}
\begin{définition}\pcmn{犁头}\end{définition}\end{entrée}

\begin{entrée}{mbɣom}{}{ⓔmbɣom} 
\classe{vs} \paradigme{dir}{tɤ-}
\begin{définition}\pfra{occupé, pressé}\end{définition}
\begin{définition}\pcmn{忙}\end{définition}
\begin{exemple}\pjya{tɤ-mbɣom-a tɕe kɤ-ari-a}\hspace{5pt}\pcmn{我很急地去了}\end{exemple}
\begin{exemple}\pjya{tɤ-mbɣom ma tɯ-maqhu}\hspace{5pt}\pcmn{你快一点,不然你会迟到}\end{exemple}
\begin{exemple}\pjya{ma-tɤ-tɯ-mbɣom}\hspace{5pt}\pcmn{不用那么着急}\end{exemple}
\begin{exemple}\pjya{ɯ-ɲɯ́-tɯ-mbɣom?}\hspace{5pt}\pcmn{你很急吗?}\end{exemple}
\begin{exemple}\pjya{kɤ-mbɣom mɤ-ra / tɯ-mbɣom mɤ-ra}\hspace{5pt}\pcmn{不用着急}\end{exemple}\relationsémantique{参考}{\lien{ⓔɕɯmbɣom}{ɕɯmbɣom}}\relationsémantique{参考}{\lien{ⓔɣɤmbɣomru}{ɣɤmbɣomru}}\end{entrée}

\begin{entrée}{mbɣopɤl}{}{ⓔmbɣopɤl} 
\classe{n} 
\begin{définition}\pfra{partie de la charrue}\end{définition}
\begin{définition}\pcmn{用来固定铧头的部分}\end{définition}
\begin{exemple}\pjya{mbɣopɤl nɯ qraʁ ɯ-sɤ-tshoʁ nɯ ŋu, mbɣopɤl ɯ-pa qraʁ tú-wɣ-tshoʁ, mbɣɤru chɯ́-wɣ-tshoʁ tɕe kɤ-ntɕhoz tɯ-sna ɕti}\hspace{5pt}\pcmn{\lien{ⓔmbɣopɤl}{mbɣopɤl}是用来固定铧头的部分,\lien{ⓔmbɣopɤl}{mbɣopɤl}下端装上铧头,上端连上犁杆,就可以用了。}\end{exemple}\end{entrée}

\begin{entrée}{mbɣorna}{}{ⓔmbɣorna} 
\classe{n} 
\begin{définition}\pfra{partie de la charrue}\end{définition}
\begin{définition}\pcmn{犁把}\end{définition}
\begin{exemple}\pjya{mbɣorna nɯ mbɣopɤl ɯ-ku ɯ-taʁ ku-ndzoʁ tɕe kɯ-ɕlu ɣɯ ɯ-jaʁ sɤ-ndo ɯ-spa ŋu}\hspace{5pt}\pcmn{犁把是装在犁头上,供耕地人掌握方向的把手}\end{exemple}\end{entrée}

\begin{entrée}{mbɣɯrloʁ}{}{ⓔmbɣɯrloʁ} 
\classe{n} 
\begin{définition}\pfra{tonnerre}\end{définition}
\begin{définition}\pcmn{雷}\end{définition}
\begin{exemple}\pjya{mbɣɯrloʁ to-βzu}\hspace{5pt}\pcmn{打雷了}\end{exemple}\étymologie{ⁿbrug.glog}\end{entrée}

\begin{entrée}{mbi}{}{ⓔmbi} 
\classe{vt}  
\grammaire{secondatif} \paradigme{dir}{nɯ-}
\begin{définition}\pfra{donner}\end{définition}
\begin{définition}\pcmn{给}\end{définition}\paradigme{}{pɯ-}
\begin{exemple}\pjya{ɯʑo kɯ nɯ́-wɣ-mbi-a}\hspace{5pt}\pcmn{他给我了}\end{exemple}
\begin{sous-entrée}{ambi}{ⓔmbiⓝambi} 
\classe{vi}  
\grammaire{pass} 
\begin{définition}\pfra{être donné}\end{définition}
\begin{définition}\pcmn{给了}\end{définition}
\begin{exemple}\pjya{ɯʑo ambi ma jɯfɕɯr nɯ-mbi-t-a}\hspace{5pt}\pcmn{已经给了他,我昨天就给了}\end{exemple}
\begin{exemple}\pjya{tɤ-pɤtso ɯ-smɤn ambi}\hspace{5pt}\pcmn{小孩子的药已经给了}\end{exemple}
\begin{exemple}\pjya{ki tɤ-pɤtso saχsɯ ɣɯ ɯ-smɤn ambi, tɯrmɯ ɣɯ nɯ mɤ-ambi}\hspace{5pt}\pcmn{给了中午的药,没有给下午的(药)}\end{exemple}\end{sous-entrée}

\begin{sous-entrée}{ambɯmbi}{ⓔmbiⓝambɯmbi} 
\classe{vi}  
\grammaire{refl} 
\begin{définition}\pfra{se donner les uns les autres}\end{définition}
\begin{définition}\pcmn{互相送}\end{définition}
\begin{exemple}\pjya{ɕnɤto ɲɯ-ɤmbɯmbi-nɯ ŋgrɤl}\hspace{5pt}\pcmn{他们互相送鼻烟}\end{exemple}\end{sous-entrée}

\end{entrée}

\begin{entrée}{mbijtshi}{}{ⓔmbijtshi} 
\classe{vt}  
\grammaire{comp} \paradigme{dir}{nɯ-}
\begin{définition}\pfra{donner à boire et à manger}\end{définition}
\begin{définition}\pcmn{喂东西吃;喂东西喝}\end{définition}
\begin{exemple}\pjya{a-mu kɯ nɯ́-wɣ-mbijtshi-a}\hspace{5pt}\pcmn{我母亲给了我吃喝}\end{exemple}\relationsémantique{参考}{\lien{ⓔngɤjtshi}{ngɤjtshi}}\end{entrée}

\begin{entrée}{mbjiz}{}{ⓔmbjiz} 
\classe{vs} \paradigme{dir}{nɯ-}
\begin{définition}\pfra{s'effacer (couleur)}\end{définition}
\begin{définition}\pcmn{褪色}\end{définition}
\begin{exemple}\pjya{tɯ-nga ɲo-mbjiz}\hspace{5pt}\pcmn{衣服褪色了}\end{exemple}
\begin{exemple}\pjya{tɤ-scoz ɲo-mbjiz}\hspace{5pt}\pcmn{字褪色了}\end{exemple}
\begin{sous-entrée}{sɯmbjiz}{ⓔmbjizⓝsɯmbjiz} 
\classe{vt}  
\grammaire{caus} \end{sous-entrée}

\end{entrée}

\begin{entrée}{mbjom}{}{ⓔmbjom} 
\classe{vs} \paradigme{dir}{tɤ-}\paradigme{dir}{thɯ-}\paradigme{dir}{nɯ-}
\begin{définition}\pfra{rapide}\end{définition}
\begin{définition}\pcmn{快(跑的速度)}\end{définition}
\begin{définition}\pfra{s'efforcer à aller le plus vite possible}\end{définition}
\begin{définition}\pcmn{使自己跑得更快}\end{définition}
\begin{exemple}\pjya{kɯ-mbjom ci ɲɯ-ŋu}\hspace{5pt}\pcmn{他是个跑得快的人}\end{exemple}
\begin{exemple}\pjya{@qiche ɲɯ-mbjom}\hspace{5pt}\pcmn{汽车开得很快}\end{exemple}
\begin{exemple}\pjya{wo, nɤ-tɯ-mbjom !}\hspace{5pt}\pcmn{啊,你这么快就(回来了)!}\end{exemple}
\begin{exemple}\pjya{nɤ-mbro qale sthɯci a-nɯ-ʑɣɤɣɤmbjom}\hspace{5pt}\pcmn{你的马要令自己跑得像风一样快}\end{exemple}
\begin{sous-entrée}{ɣɤmbjom}{ⓔmbjomⓝɣɤmbjom} 
\classe{vt}  
\grammaire{caus} 
\begin{définition}\pfra{faire accélérer}\end{définition}
\begin{définition}\pcmn{加快}\end{définition}
\begin{exemple}\pjya{rdɤstaʁ kɤ-lɤt jú-wɣ-nɤxɕɤt tɕe kɤ-ɣɤmbjom khɯ}\hspace{5pt}\pcmn{用力扔石头可以令它飞出去得更快}\end{exemple}\end{sous-entrée}

\begin{sous-entrée}{ʑɣɤɣɤmbjom}{ⓔmbjomⓝʑɣɤɣɤmbjom} 
\classe{vi}  
\grammaire{refl}
\grammaire{caus} \end{sous-entrée}

\end{entrée}

\begin{entrée}{mblɯt}{}{ⓔmblɯt} 
\classe{vi.nh}  
\grammaire{acaus} \paradigme{dir}{nɯ-}\paradigme{dir}{thɯ-}
\begin{définition}\pfra{être détruit, disparaître}\end{définition}
\begin{définition}\pcmn{绝种;绝后}\end{définition}
\begin{exemple}\pjya{ji-paʁrɟit nɯ-mblɯt}\hspace{5pt}\pcmn{我们的猪绝种了}\end{exemple}\relationsémantique{参考}{\lien{ⓔplɯt}{plɯt}}\end{entrée}

\begin{entrée}{mboʁ}{}{ⓔmboʁ} 
\classe{n} 
\begin{définition}\pfra{tissu de lin rectangulaire}\end{définition}
\begin{définition}\pcmn{正方形的麻布}\end{définition}
\begin{exemple}\pjya{mboʁ nɯ ɯ-spa tasa ŋu, tasa lú-wɣ-pɣo tɕe lú-wɣ-rɯm tɕe tú-wɣ-rɯtɤβri tɕe kú-wɣ-sqa, kɤ́-wɣ-sqa tɕe ɯ-ŋgɯ thɤfkɤlɤɣi kú-wɣ-lɤt ra, nɯ mɤɕtʂa ku-smi mɤ-cha. kɤ-smi tɕe, ɲɯ́-wɣ-z-mɯɕtaʁ tɕe ɲɯ́-wɣ-χtɕi tɕe tɤ-zbaʁ tɕe kú-wɣ-sɤrɤt tɕe kɤ-taʁ kú-wɣ-thɯ ŋu. tɕe kɤ-taʁ thɯ-jɤɣ tɕe tɯ-rtɯthɯ ŋu tɕe ɣɯ-xtsɯ ra ɯ-xtsɯ pɯ-rtaʁ tɕe ɲɯ-mba ŋu ɲɯ-mpɕu ŋu tɕe chɯ́-wɣ-tʂɯβ tɕe, kɯ-ɤβʑɯrdɯ-rdu ʑo ɲɯ́-wɣ-βzu tɕe ɯ-βzɯr tɯ-ka nɯ tɕu ɯ-jndɯz cho ɯ-ltɕi kú-wɣ-tshoʁ tɕe kɤ-ntɕhoz tu-βze ŋu. tɕe ɯ-mboʁ nɯ kɯ-tɣa tɕe tɯ-fcaʁ ŋu, tɯ-mbri cho tɯ-ɲcɣa ɣɯ ɯ-fkɯm ŋu, tɯ-mthɤɣ ɲɯ́-wɣ-rtɤβ ŋu tɕe mɤ-saʁdɯɣ.}\hspace{5pt}\pcmn{\lien{ⓔmboʁ}{mboʁ} 的材料是大麻。要把麻捻成细线,搓紧,再反搓成一绞一绞的,然后煮了。煮的时候,要放大量的草木灰,不然煮不熟。煮熟了以后晾凉,然后洗干净。干了以后,把线卸下来堆在一旁,然后在牵杆上牵起来。织完了以后就成麻布,还要捶打。捶打好了以后,麻布就变薄了,变得很光滑,然后就把麻布缝成四方形的,在四个角做上流苏和彩色布条,这样就可以用了。\lien{ⓔmboʁ}{mboʁ}在收割的时候可以垫背或者裹绳子和镰刀,可以围在腰间,不碍事。}\end{exemple}\end{entrée}

\begin{entrée}{mboʁkhɯr}{}{ⓔmboʁkhɯr} 
\classe{n} 
\begin{définition}\pfra{paquet}\end{définition}
\begin{définition}\pcmn{包裹(用正方形的布)}\end{définition}\relationsémantique{参考}{\lien{ⓔrɯmboʁkhɯr}{rɯmboʁkhɯr}}\end{entrée}

\begin{entrée}{mboʁɲɟi}{}{ⓔmboʁɲɟi} 
\classe{adv} 
\begin{définition}\pfra{en milles morceaux}\end{définition}
\begin{définition}\pcmn{粉粹}\end{définition}
\begin{exemple}\pjya{mboʁɲɟi ʑo ɲɤ-ɕe}\hspace{5pt}\pcmn{粉粹了}\end{exemple}\relationsémantique{参考}{\lien{ⓔarɤmboʁɲɟi}{arɤmboʁɲɟi}}\end{entrée}

\begin{entrée}{mbraj}{}{ⓔmbraj} 
\classe{n} 
\begin{définition}\pfra{bouleau rouge}\end{définition}
\begin{définition}\pcmn{红桦树}\end{définition}
\begin{exemple}\pjya{mbraj nɯ sɤjku cho kɯ-naχtɕɯɣ ŋu ri, mbraj ɣɯ ɯ-rqhu nɯ kɯ-ɣɯrni ŋu, sɤjku ɣɯ ɯ-rqhu kɯ-wɣrum ŋu, ndʑi-jwaʁ ɲɯ-naχtɕɯɣ}\hspace{5pt}\pcmn{红桦树和白桦树相同,但红桦树的树皮是红色的,而白桦树是白色的,它们俩的叶子一样。}\end{exemple}\end{entrée}

\begin{entrée}{mbraʑɯm}{}{ⓔmbraʑɯm} 
\classe{n} 
\begin{définition}\pfra{une espèce de champignon}\end{définition}
\begin{définition}\pcmn{一种蘑菇}\end{définition}
\begin{exemple}\pjya{mbraʑɯm nɯ stɤmku kɯ-ɤmɯrmbɯrmbat ʑo tu-ɬoʁ, tɤjmɤɣ kɯ-ndɯ-ndɯβ ʑo ŋu, thɯ-kɤ-ɣɯri ʑo fse, kɯ-wɣrum ŋu, kɤ-ndza wuma mɯm, ftɕar tɕe tu-ɬoʁ ŋu}\hspace{5pt}\pcmn{\lien{ⓔmbraʑɯm}{mbraʑɯm}是长在草地上的一丛丛的小菌子,像是用线串起来的,呈白色,好吃,夏天生长。}\end{exemple}\end{entrée}

\begin{entrée}{mbrɤjqhɤt}{}{ⓔmbrɤjqhɤt} 
\classe{n} 
\begin{définition}\pfra{gentiane}\end{définition}
\begin{définition}\pcmn{秦艽}\end{définition}
\begin{exemple}\pjya{mbrɤjqhɤt nɯ sɯŋgɯ tu-kɯ-ɬoʁ sɯjno ci ŋu. smɤn kɤ-βzu ɲɯ-sna, ɯ-jwaʁ rɲɟi, jaʁ, mpɕu, ɯ-ru kɯ-nɤrko tsa ci ŋu, ɯ-ru ɯ-taʁ ɯ-jwaʁ nɯ tɯ-rtsɤɣ tɯ-rtsɤɣ lu-oʑɯrja tɕe, ɯ-ru cho ɯ-jwaʁ ni ndʑi-pɤrthɤβ ɯ-mɯntoʁ ɲɯ-lɤt ŋu, ɯ-mɯntoʁ wɣrum. ɯ-zrɤm nɯ wɣrum, ɯ-ru cho ɯ-zrɤm ɯ-khɤntshɤm ri ɯ-rme kɯ-rɲɟi kɯ-ngɯt ʑo tu. qartsɯ tɕe pjɯ-rom ɯ-fsaqhe tɕe pjɯ-ɬoʁ ŋu.}\hspace{5pt}\pcmn{秦艽是生长在森林里的草,可以入药。叶子长、厚、光滑。茎比较结实。叶子一节一节地长在茎上,叶子和茎的中间开花,花是白色的。根也是白色的,在根和茎交界处有又长又结实的毛。冬天枯萎,第二年又生长。}\end{exemple}\end{entrée}

\begin{entrée}{mbrɤmbrɯ}{}{ⓔmbrɤmbrɯ} 
\classe{n} 
\begin{définition}\pfra{légumineuse}\end{définition}
\begin{définition}\pcmn{豆类}\end{définition}\end{entrée}

\begin{entrée}{mbrɤndzoʁloʁ}{}{ⓔmbrɤndzoʁloʁ} 
\classe{n} 
\begin{définition}\pfra{auge}\end{définition}
\begin{définition}\pcmn{马槽}\end{définition}\end{entrée}

\begin{entrée}{mbrɤrɟɯɣ}{}{ⓔmbrɤrɟɯɣ} 
\classe{n} 
\begin{définition}\pfra{course de cheval}\end{définition}
\begin{définition}\pcmn{赛马}\end{définition}
\begin{exemple}\pjya{mbrɤrɟɯɣ βzu-j}\hspace{5pt}\pcmn{我们赛马}\end{exemple}\relationsémantique{参考}{\lien{ⓔnɯmbrɤrɟɯɣ}{nɯmbrɤrɟɯɣ}}\end{entrée}

\begin{entrée}{mbrɤsɤm}{}{ⓔmbrɤsɤm} 
\classe{n} 
\begin{définition}\pfra{vannerie en forme de cuve utilisée pour faire sécher les grains}\end{définition}
\begin{définition}\pcmn{放在房顶用来晒粮食,竹子编成的盆型的簸箕}\end{définition}\end{entrée}

\begin{entrée}{mbrɤsno}{}{ⓔmbrɤsno} 
\classe{n} 
\begin{définition}\pfra{selle}\end{définition}
\begin{définition}\pcmn{马鞍}\end{définition}\end{entrée}

\begin{entrée}{mbrɤstshi}{}{ⓔmbrɤstshi} 
\classe{n} 
\begin{définition}\pfra{gruau de riz}\end{définition}
\begin{définition}\pcmn{粥;稀饭}\end{définition}\end{entrée}

\begin{entrée}{mbrɤt}{}{ⓔmbrɤt} 
\classe{vi}  
\grammaire{acaus} \paradigme{dir}{nɯ-}
\begin{définition}\pfra{se casser, se couper (corde, fil)}\end{définition}
\begin{définition}\pcmn{断(线)}\end{définition}
\begin{exemple}\pjya{tɤ-ri ɲo-mbrɤt}\hspace{5pt}\pcmn{线断了}\end{exemple}
\begin{exemple}\pjya{tɯmbri ɲɤ-mbrɤt}\hspace{5pt}\pcmn{绳子断了}\end{exemple}
\begin{exemple}\pjya{jiɕqha nɯ-mbrɤt loβ !}\hspace{5pt}\pcmn{刚才电话断了!}\end{exemple}\relationsémantique{参考}{\lien{ⓔprɤt}{prɤt}}\end{entrée}

\begin{entrée}{mbrɤtaʁ}{}{ⓔmbrɤtaʁ} 
\classe{adv} 
\begin{définition}\pfra{à cheval}\end{définition}
\begin{définition}\pcmn{马背上}\end{définition}
\begin{exemple}\pjya{mbrɤtaʁ cha-a}\hspace{5pt}\pcmn{我会骑马}\end{exemple}\relationsémantique{参考}{\lien{ⓔmbroⓗ2}{mbro}}\relationsémantique{参考}{\lien{ⓔtaʁⓗ3}{taʁ₃}}\end{entrée}

\begin{entrée}{mbrɤz}{}{ⓔmbrɤz} 
\classe{n} 
\begin{définition}\pfra{riz}\end{définition}
\begin{définition}\pcmn{米}\end{définition}\étymologie{ⁿbras}\end{entrée}

\begin{entrée}{mbre}{}{ⓔmbre} 
\classe{vi} 
\begin{définition}\pfra{auspicieux (prédiction)}\end{définition}
\begin{définition}\pcmn{吉祥(预兆)}\end{définition}
\begin{exemple}\pjya{a-mphrɯmɯ ɲɯ-mbre}\hspace{5pt}\pcmn{算的卦很吉祥}\end{exemple}\end{entrée}

\begin{entrée}{mbri}{₁}{ⓔmbriⓗ1} 
\classe{vi} \paradigme{dir}{tɤ-}
\begin{définition}\pfra{fort (bruit), crier}\end{définition}
\begin{définition}\pcmn{响;叫}\end{définition}
\begin{exemple}\pjya{mbɣɯrloʁ ɲɯ-mbri}\hspace{5pt}\pcmn{打雷了}\end{exemple}
\begin{exemple}\pjya{pɣɤtɕɯ to-mbri}\hspace{5pt}\pcmn{鸟叫了}\end{exemple}
\begin{exemple}\pjya{ɯ-rmi ɲɤ-mbri (=to-caʁ)}\hspace{5pt}\pcmn{他出名了}\end{exemple}\relationsémantique{参考}{\lien{ⓔʑmbriⓗ1}{ʑmbri₁}}\end{entrée}

\begin{entrée}{mbri}{₂}{ⓔmbriⓗ2} 
\classe{vi}  
\grammaire{acaus} \paradigme{dir}{thɯ-}\paradigme{dir}{nɯ-}
\begin{définition}\pfra{se déchirer soudainement (habit)}\end{définition}
\begin{définition}\pcmn{烂掉(衣服)}\end{définition}
\begin{exemple}\pjya{tɯ-ŋga chɤ-mbri}\end{exemple}
\begin{exemple}\pjya{tɯ-ŋga ɲɤ-mbri}\hspace{5pt}\pcmn{衣服破了}\end{exemple}\relationsémantique{参考}{\lien{ⓔpriⓗ1}{pri₁}}\end{entrée}

\begin{entrée}{mbro}{₂}{ⓔmbroⓗ2} 
\classe{n} 
\begin{définition}\pfra{cheval}\end{définition}
\begin{définition}\pcmn{马}\end{définition}\relationsémantique{参考}{\lien{ⓔmbrɤtaʁ}{mbrɤtaʁ}}\relationsémantique{参考}{\lien{ⓔnɯmbrɤpɯ}{nɯmbrɤpɯ}}\end{entrée}

\begin{entrée}{mbro}{₁}{ⓔmbroⓗ1} 
\classe{vs} \paradigme{dir}{tɤ-}\paradigme{dir}{tɤ-}
\begin{définition}\pfra{haut}\end{définition}
\begin{définition}\pcmn{高}\end{définition}
\begin{définition}\pfra{augmenter}\end{définition}
\begin{définition}\pcmn{提高}\end{définition}
\begin{exemple}\pjya{ɯ-phoŋbu ɲɯ-mbro}\hspace{5pt}\pcmn{他身材很高}\end{exemple}
\begin{exemple}\pjya{aʑo staʁnɤ tɤ-mbro}\hspace{5pt}\pcmn{他比我高了}\end{exemple}
\begin{exemple}\pjya{nɤʑo jamar tɤ-mbro}\hspace{5pt}\pcmn{他长得跟你一样高了}\end{exemple}
\begin{exemple}\pjya{nɤ-@xuetang tu-ɣɤmbrɤm ɯ́-cha}\hspace{5pt}\pcmn{会不会使你的血糖升高?}\end{exemple}\relationsémantique{反义词}{\lien{ⓔmbɤrⓗ1}{mbɤr₁}}
\begin{sous-entrée}{ɣɤmbro}{ⓔmbroⓗ1ⓝɣɤmbro} 
\classe{vt} \end{sous-entrée}

\end{entrée}

\begin{entrée}{mbrolu}{}{ⓔmbrolu} 
\classe{n} 
\begin{définition}\pfra{année du cheval}\end{définition}
\begin{définition}\pcmn{马年}\end{définition}\end{entrée}

\begin{entrée}{mbrondza}{}{ⓔmbrondza} 
\classe{n} 
\begin{définition}\pfra{nourriture pour cheval}\end{définition}
\begin{définition}\pcmn{马料}\end{définition}\end{entrée}

\begin{entrée}{mbroqa}{}{ⓔmbroqa} 
\classe{n} 
\begin{définition}\pfra{sabot}\end{définition}
\begin{définition}\pcmn{马蹄,马的脚}\end{définition}\end{entrée}

\begin{entrée}{mbroʁkɕi}{}{ⓔmbroʁkɕi} 
\classe{n} 
\begin{définition}\pfra{chien tibétain}\end{définition}
\begin{définition}\pcmn{藏獒}\end{définition}\end{entrée}

\begin{entrée}{mbrosta}{}{ⓔmbrosta} 
\classe{n} 
\begin{définition}\pfra{écurie}\end{définition}
\begin{définition}\pcmn{马厩}\end{définition}\relationsémantique{同义词}{\lien{ⓔrtakhaŋ}{rtakhaŋ}}\end{entrée}

\begin{entrée}{mbroχpa}{}{ⓔmbroχpa} 
\classe{n} 
\begin{définition}\pfra{nomades}\end{définition}
\begin{définition}\pcmn{牧民}\end{définition}\étymologie{ⁿbrog.pa}\end{entrée}

\begin{entrée}{mbrozga}{}{ⓔmbrozga} 
\classe{n} 
\begin{définition}\pfra{selle}\end{définition}
\begin{définition}\pcmn{马鞍}\end{définition}\end{entrée}

\begin{entrée}{mbrɯɣlu}{}{ⓔmbrɯɣlu} 
\classe{n} 
\begin{définition}\pfra{année du dragon}\end{définition}
\begin{définition}\pcmn{龙年}\end{définition}\étymologie{ⁿbrug.lo}\end{entrée}

\begin{entrée}{mbrɯtɕɯ}{}{ⓔmbrɯtɕɯ} 
\classe{n} 
\begin{définition}\pfra{couteau}\end{définition}
\begin{définition}\pcmn{刀}\end{définition}
\begin{exemple}\pjya{mbrɯtɕɯ ɯ-ɕɣa}\hspace{5pt}\pcmn{刀刃}\end{exemple}
\begin{exemple}\pjya{mbrɯtɕɯ ɯ-pɯ}\hspace{5pt}\pcmn{小刀}\end{exemple}\end{entrée}

\begin{entrée}{mbɯlwa}{}{ⓔmbɯlwa} 
\classe{n} 
\begin{définition}\pfra{salaire d'un lama}\end{définition}
\begin{définition}\pcmn{和尚的工资}\end{définition}\étymologie{ⁿbul.ba}\end{entrée}

\begin{entrée}{mbɯmχtɤr}{}{ⓔmbɯmχtɤr} 
\classe{n} 
\begin{définition}\pfra{cent mille}\end{définition}
\begin{définition}\pcmn{100000}\end{définition}\étymologie{ⁿbum.tʰer}\end{entrée}

\begin{entrée}{mbɯrlɤn}{}{ⓔmbɯrlɤn} 
\classe{n} 
\begin{définition}\pfra{rabot}\end{définition}
\begin{définition}\pcmn{刨}\end{définition}\relationsémantique{参考}{\lien{ⓔnɯmbɯrlɤn}{nɯmbɯrlɤn}}\étymologie{ⁿbur.len}\end{entrée}

\begin{entrée}{mbɯrlɤnndoʁ}{}{ⓔmbɯrlɤnndoʁ} 
\classe{n} 
\begin{définition}\pfra{copeaux}\end{définition}
\begin{définition}\pcmn{刨花}\end{définition}\end{entrée}

\begin{entrée}{mbɯsɯt}{}{ⓔmbɯsɯt} 
\classe{n} 
\begin{définition}\pfra{râpeuse}\end{définition}
\begin{définition}\pcmn{丝丝(擦成)}\end{définition}
\begin{exemple}\pjya{lɤpɯɣ mbɯsɯt thɯ-lat-a}\hspace{5pt}\pcmn{我把萝卜擦成丝丝}\end{exemple}\end{entrée}

\begin{entrée}{mbɯt}{}{ⓔmbɯt} 
\classe{vi} \paradigme{dir}{thɯ-}\paradigme{dir}{pɯ-}
\begin{définition}\pfra{s’écrouler}\end{définition}
\begin{définition}\pcmn{塌毁下来;垮下来}\end{définition}
\begin{exemple}\pjya{ŋgɤm ki mbɯt ɲɯ-ŋu}\hspace{5pt}\pcmn{这个土坡快要塌下来}\end{exemple}
\begin{exemple}\pjya{tʂu ɲɯ-mbɯt pɯ-mto-t-a}\hspace{5pt}\pcmn{我看到路在塌方}\end{exemple}
\begin{exemple}\pjya{kha ɲɯ-mbɯt}\hspace{5pt}\pcmn{房子在垮}\end{exemple}
\begin{exemple}\pjya{tʂu cho-mbɯt}\hspace{5pt}\pcmn{路塌下来了}\end{exemple}\end{entrée}

\begin{entrée}{mbuz}{}{ⓔmbuz} 
\classe{vi.nh} \paradigme{dir}{tɤ-}\paradigme{dir}{thɯ-}\paradigme{dir}{tɤ-}\paradigme{dir}{thɯ-}
\begin{définition}\pfra{déborder}\end{définition}
\begin{définition}\pcmn{溢出来}\end{définition}
\begin{définition}\pfra{laisser ... déborder}\end{définition}
\begin{définition}\pcmn{令……溢出来}\end{définition}
\begin{exemple}\pjya{tʂha pjɤ-mbuz}\hspace{5pt}\pcmn{茶溢出来了}\end{exemple}
\begin{exemple}\pjya{tɯ-ci cho-mbuz}\hspace{5pt}\pcmn{水溢出来了}\end{exemple}
\begin{exemple}\pjya{tɤ-lu ɲɤ-nɯ-jmɯt-a tɕe tɤ-sɯɣmbuz-a}\hspace{5pt}\pcmn{我把(在煮)的牛奶忘了,就溢出来了}\end{exemple}
\begin{sous-entrée}{sɯɣmbɯz}{ⓔmbuzⓝsɯɣmbɯz} 
\classe{vt} \end{sous-entrée}

\end{entrée}

\begin{entrée}{mchɯn}{}{ⓔmchɯn} 
\classe{vt} \paradigme{dir}{tɤ-}
\begin{définition}\pfra{percevoir la vraie nature de quelqu'un (pouvoir de sprulsku)}\end{définition}
\begin{définition}\pcmn{活佛看穿别人的本性}\end{définition}
\begin{exemple}\pjya{sprɯskɯ kɯ kɯ-mchɯn ɕti}\hspace{5pt}\pcmn{活佛能看穿人的本性(\lien{ⓔkɯⓗ1}{kɯ}-是宾语泛指标记)}\end{exemple}\étymologie{mkʰʲen}\end{entrée}

\begin{entrée}{mchɯnba}{}{ⓔmchɯnba} 
\classe{n} 
\begin{définition}\pfra{don de prédiction}\end{définition}
\begin{définition}\pcmn{预知力}\end{définition}
\begin{exemple}\pjya{ɯ-mchɯnba ɣɤʑu}\hspace{5pt}\pcmn{他有预知力}\end{exemple}\étymologie{mkʰʲen.pa}\end{entrée}

\begin{entrée}{múcin}{}{ⓔmúcin} 
\classe{adv} 
\begin{définition}\pfra{pas du tout}\end{définition}
\begin{définition}\pcmn{根本没有}\end{définition}\end{entrée}

\begin{entrée}{mciphɯt}{}{ⓔmciphɯt} 
\classe{n} 
\begin{définition}\pfra{crachat}\end{définition}
\begin{définition}\pcmn{(吐的)口水}\end{définition}
\begin{exemple}\pjya{mciphɯt thɯ-lat-a}\hspace{5pt}\pcmn{我吐了口水}\end{exemple}\relationsémantique{参考}{\lien{ⓔsɯmciphɯt}{sɯmciphɯt}}\end{entrée}

\begin{entrée}{mu,cɯɣ}{}{ⓔmu,cɯɣ} 
\classe{vi}
\classe{vi}
\classe{vi} 
\begin{définition}\pfra{vivre constament dans la peur}\end{définition}
\begin{définition}\pcmn{提心吊胆,恐慌不安}\end{définition}
\begin{exemple}\pjya{kɤ-mu kɤ-cɯɣ kɤ-rɤʑi kɯ-ra ɲɯ-sɤdɯɣ}\hspace{5pt}\pcmn{整天提心吊胆的感觉不好受}\end{exemple}
\begin{exemple}\pjya{ɲɯ-mu ɲɯ-cɯɣ ʑo}\hspace{5pt}\pcmn{他提心吊胆}\end{exemple}
\begin{exemple}\pjya{ma-tɯ-mu ma-tɯ-cɯɣ ma mɤ-ʁdɯɣ nɤ}\hspace{5pt}\pcmn{你不要这样提心吊胆,不会有是的}\end{exemple}\relationsémantique{Component 1}{\lien{ⓔmuⓗ2}{mu}}\relationsémantique{Component 2}{\lien{}{cɯɣ}}\end{entrée}

\begin{entrée}{mcɯrɯβrɯβ}{}{ⓔmcɯrɯβrɯβ} 
\classe{n} 
\begin{définition}\pfra{personne qui bave tout le temps}\end{définition}
\begin{définition}\pcmn{总是流口水的人}\end{définition}
\begin{exemple}\pjya{nɤʑo mcɯrɯβrɯβ ki}\hspace{5pt}\pcmn{你这个爱流口水的家伙}\end{exemple}\relationsémantique{参考}{\lien{ⓔtɯ-mci}{tɯ-mci}}\relationsémantique{参考}{\lien{}{rɯβrɯβ}}\end{entrée}

\begin{entrée}{mda}{}{ⓔmda} 
\classe{vi.nh} \paradigme{dir}{tɤ-}
\begin{définition}\pfra{arriver au moment de}\end{définition}
\begin{définition}\pcmn{到时间}\end{définition}
\begin{exemple}\pjya{pɤjkhu mɯ́j-mda}\hspace{5pt}\pcmn{还没有到时间}\end{exemple}
\begin{exemple}\pjya{saχsɯ pɤjkhu mɯ́j-mda}\hspace{5pt}\pcmn{还没有到中午餐}\end{exemple}
\begin{exemple}\pjya{ʑa qanɯ ɲɯ-ŋu tɕe, kɤ-nɯɕe mda}\hspace{5pt}\pcmn{快天黑了,该回去了}\end{exemple}
\begin{exemple}\pjya{tɤŋe tɤ-anɯri tɕe kɤ-nɯɕe mda}\hspace{5pt}\pcmn{太阳落山了,该回去}\end{exemple}
\begin{exemple}\pjya{nɤj tɯ-nɯɕe mda}\hspace{5pt}\pcmn{你该回去了}\end{exemple}
\begin{exemple}\pjya{a-βra tɤ-mda}\hspace{5pt}\pcmn{轮到我了}\end{exemple}
\begin{exemple}\pjya{tɤ-rɤku kɯ-mda}\hspace{5pt}\pcmn{成熟的庄稼}\end{exemple}
\begin{exemple}\pjya{tɤ-rɤku kɤ-phɯt tɤ-mda}\hspace{5pt}\pcmn{收割的时间到了}\end{exemple}
\begin{exemple}\pjya{kɤ-nɯftɕaka tɤ-mda}\hspace{5pt}\pcmn{到了准备(晚餐)的时间}\end{exemple}
\begin{exemple}\pjya{nɤ-tɯ-ci (kɤ-rku) ɯ-ɲɯ́-mda?}\hspace{5pt}\pcmn{你杯子还有没有水?}\end{exemple}\relationsémantique{参考}{\lien{ⓔmasɤmdɤla}{masɤmdɤla}}\end{entrée}

\begin{entrée}{mdandzɯn}{}{ⓔmdandzɯn} 
\classe{n} 
\begin{définition}\pfra{grosse perle du chapelet}\end{définition}
\begin{définition}\pcmn{玛尼珠最大的珠子}\end{définition}
\begin{exemple}\pjya{mphruwa nɯ tɯ-mke pjɯ́-wɣ-nɯ-rʁe tɕe, ɯ-mdandzɯn nɯ tɯ-ʁɤri ɯ-stu ɲɯ́-wɣ-znɤtɯɣ ŋu}\hspace{5pt}\pcmn{脖子上戴玛尼珠的时候,大珠子要放在前面}\end{exemple}\end{entrée}

\begin{entrée}{mdarɯ}{}{ⓔmdarɯ} 
\classe{n} 
\begin{définition}\pfra{damaru}\end{définition}
\begin{définition}\pcmn{鼗}\end{définition}\étymologie{ɖa.ma.ru}\end{entrée}

\begin{entrée}{mdaʁʑɯɣ}{}{ⓔmdaʁʑɯɣ} 
\classe{n} 
\begin{définition}\pfra{arc et flèches}\end{définition}
\begin{définition}\pcmn{弓箭}\end{définition}\étymologie{mda.gʑu}\end{entrée}

\begin{entrée}{mdi}{}{ⓔmdi} 
\classe{vs} 
\begin{définition}\pfra{être au complet, être entièrement là}\end{définition}
\begin{définition}\pcmn{全部;齐全}\end{définition}
\begin{exemple}\pjya{laχtɕha ɲɯ-mdi}\hspace{5pt}\pcmn{东西齐全}\end{exemple}
\begin{exemple}\pjya{nɤ-laχtɕha kɯ-mdi ʑo tɤ-wum}\hspace{5pt}\pcmn{你收拾你所有的东西吧!}\end{exemple}\relationsémantique{参考}{\lien{ⓔmdoʁmdi}{mdoʁmdi}}\end{entrée}

\begin{entrée}{mdoʁmdi}{}{ⓔmdoʁmdi} 
\classe{vs}  
\grammaire{rdpl} \paradigme{dir}{tɤ-}
\begin{définition}\pfra{entier, complet (un objet)}\end{définition}
\begin{définition}\pcmn{完整(指一个物体,不能表示零散的东西聚齐)}\end{définition}
\begin{exemple}\pjya{tɯ-ndʐi ɲɯ-mdoʁmdi}\hspace{5pt}\pcmn{皮子是完整的}\end{exemple}\relationsémantique{参考}{\lien{ⓔmdi}{mdi}}\end{entrée}

\begin{entrée}{mdɯ}{}{ⓔmdɯ} 
\classe{vi} \paradigme{dir}{thɯ-}\sens{1}
\begin{définition}\pfra{vivre jusqu'à}\end{définition}
\begin{définition}\pcmn{到达(年龄)}\end{définition}
\begin{exemple}\pjya{ɯ-lɯz nɯ cho-mdɯ}\hspace{5pt}\pcmn{他年龄大了}\end{exemple}
\begin{exemple}\pjya{kɯrcɤsqi (pɤrme) cho-mdɯ}\hspace{5pt}\pcmn{他活到80岁}\end{exemple}\sens{2}
\begin{définition}\pfra{être assez long (fil)}\end{définition}
\begin{définition}\pcmn{够长(线;绳子)}\end{définition}
\begin{exemple}\pjya{tɤ-ri nɯ kú-wɣ-lɤt qhe ɲɯ-zri tɕe kuchɯ-rkɯ mɤɕtʂa ku-mdɯ ɲɯ-cha}\hspace{5pt}\pcmn{这根线很长,牵过去可以牵到那边的墙角}\end{exemple}\relationsémantique{同义词}{\lien{ⓔɕaβⓗ2}{ɕaβ₂}}\end{entrée}

\begin{entrée}{mdɯnri}{}{ⓔmdɯnri} 
\classe{n} 
\begin{définition}\pfra{cérémonie effectuée lorsqu'un membre de la famille part en voyage}\end{définition}
\begin{définition}\pcmn{家里有人出行的时候,为了保佑他安全顺利而念的经}\end{définition}\relationsémantique{同义词}{\lien{ⓔndʐɯnɬa}{ndʐɯnɬa}}\end{entrée}

\begin{entrée}{mdɯt}{}{ⓔmdɯt} 
\classe{vt} \paradigme{dir}{thɯ-}
\begin{définition}\pfra{décider fermement}\end{définition}
\begin{définition}\pcmn{下决心;一心想}\end{définition}
\begin{exemple}\pjya{a-ɣe kɤ-zrɤβzjoz kɯ-ra nɯ chɯ-mdɯt-a ʑo ɕti}\hspace{5pt}\pcmn{我一心想着让孙子继续读书}\end{exemple}
\begin{exemple}\pjya{aʑo a-phoŋbu kɤ-sɯɣndʐɯm nɯ chɯ-mdɯt-a ʑo ɕti}\hspace{5pt}\pcmn{我下了决心要锻炼身体}\end{exemple}
\begin{exemple}\pjya{aʑo kɯrɯ-skɤt kɤ-βzjoz nɯ chɯ-mdɯt-a ʑo ɕti}\hspace{5pt}\pcmn{我下了决心要学藏语}\end{exemple}\relationsémantique{参考}{\lien{ⓔmdɯ}{mdɯ}}\end{entrée}

\begin{entrée}{mdɯtpa}{}{ⓔmdɯtpa} 
\classe{n} 
\begin{définition}\pfra{nœud que fait un lama en attachant un bsrungs}\end{définition}
\begin{définition}\pcmn{喇嘛打护身线的结}\end{définition}\étymologie{mdud.pa}\end{entrée}

\begin{entrée}{mdzadi}{}{ⓔmdzadi} 
\classe{n} 
\begin{définition}\pfra{puce}\end{définition}
\begin{définition}\pcmn{跳蚤}\end{définition}
\begin{exemple}\pjya{mdzadi wuma kɯ-sɤndza ɲɯ-ŋu sɤɣdɯɣ ŋotɕu χtɕɯrɯpa ku-kɯ-rŋgɯ kɯnɤ tɯ-phe ju-zɣɯt ɕti}\hspace{5pt}\pcmn{跳蚤咬人很不舒服,你在哪里睡觉,它都会跟着你的}\end{exemple}\end{entrée}

\begin{entrée}{mdzadikɤdɤɣ}{}{ⓔmdzadikɤdɤɣ} 
\classe{n} 
\begin{définition}\pfra{espèce d'arbrisseau}\end{définition}
\begin{définition}\pcmn{灌木的一种}\end{définition}
\begin{exemple}\pjya{mdzadikɤdɤɣ nɯ si kɯ-mbɯ-mbɤr ci ŋu, ɯ-di wuma ʑo mnɤm, ɯ-ru kɯ-pɣi tsa ŋu, ɯ-jwaʁ kɯ-tɕɤr kɯ-rɲɟi tsa ŋu, ɯ-mɯntoʁ ɯ-tshɯɣa phaʁrzi kɯ-fse tɕe tɯ-pɕoʁ ci ma kɯ-me kɯ-ndɯ-ndɯβ kɯ-dɯ-dɤn kɯ-wɣrum ŋu}\hspace{5pt}\pcmn{\lien{ⓔmdzadikɤdɤɣ}{mdzadikɤdɤɣ}是矮小的树,味道很浓,树干是灰色的,叶子细长,花的形状像牙刷一样,只长在一面,长得小而密,是白色的。}\end{exemple}\end{entrée}

\begin{entrée}{mdzar}{}{ⓔmdzar} 
\classe{vt}  
\grammaire{caus}
\grammaire{caus} \paradigme{dir}{pɯ-}
\begin{définition}\pfra{égoutter, essorer}\end{définition}
\begin{définition}\pcmn{滗干}\end{définition}\relationsémantique{参考}{\lien{ⓔndzar}{ndzar}}\relationsémantique{同义词}{\lien{ⓔndzarⓝsɯɣndzar}{sɯɣndzar}}\end{entrée}

\begin{entrée}{mdzɤz}{}{ⓔmdzɤz} 
\classe{vs} \paradigme{dir}{tɤ-}
\begin{définition}\pfra{honorable}\end{définition}
\begin{définition}\pcmn{体面;好听}\end{définition}
\begin{exemple}\pjya{ɯ-sɯm pjɤ-sɤzdɯɣ ri, kɯ-mdzɤz to-βzu tɕe ɯ-rŋa ɲɤ-nɤre}\hspace{5pt}\pcmn{他虽然心里难受,他为了体面做出笑容}\end{exemple}
\begin{exemple}\pjya{nɯ tu-kɯ-ti tɕe ɲɯ-mdzɤz}\hspace{5pt}\pcmn{这样说好听一点}\end{exemple}
\begin{exemple}\pjya{nɤ-sɯm mɯ-pɯ-anɯri kɯnɤ, kɯ-mdzɤz tɤ-βze ma ma-tɤ-tɯ-nɯjʁo}\hspace{5pt}\pcmn{你即使心里不高兴,你要做得好看一些,不要骂人}\end{exemple}
\begin{sous-entrée}{znɤmdzɤz}{ⓔmdzɤzⓝznɤmdzɤz} 
\classe{vt} 
\begin{définition}\pfra{ne pas oser faire un reproche}\end{définition}
\begin{définition}\pcmn{打不开情面(否定式)}\end{définition}
\begin{exemple}\pjya{mɯ́j-znɤmdzɤz}\hspace{5pt}\pcmn{他不好意思说}\end{exemple}
\begin{exemple}\pjya{to-nɤma nɯ mɯ́j-nɤpe ri, mɯ́j-pe kɤ-ti nɯ mɯ́j-znɤmdzɤz}\hspace{5pt}\pcmn{他觉得他工作得不好,但是不好意思说}\end{exemple}\end{sous-entrée}

\étymologie{mdzes}\end{entrée}

\begin{entrée}{mdzukɤr}{}{ⓔmdzukɤr} 
\classe{n} 
\begin{définition}\pfra{mdzo de couleur blanche}\end{définition}
\begin{définition}\pcmn{白色的犏牛}\end{définition}\étymologie{mdzo.dkar}\end{entrée}

\begin{entrée}{mdzumɤr}{}{ⓔmdzumɤr} 
\classe{n} 
\begin{définition}\pfra{mdzo de couleur marron claire}\end{définition}
\begin{définition}\pcmn{褐色的犏牛}\end{définition}\étymologie{mdzo.dmar}\end{entrée}

\begin{entrée}{mdzoz}{}{ⓔmdzoz} 
\classe{vt} \paradigme{dir}{nɯ-}\paradigme{dir}{tɤ-}
\begin{définition}\pfra{interdire, éviter}\end{définition}
\begin{définition}\pcmn{忌讳;禁止;回避}\end{définition}
\begin{exemple}\pjya{nɤki @chabei nɯ tɤ-mdzoz, a-mɤ-pɯ-ɴɢrɯ}\hspace{5pt}\pcmn{小心茶杯,不要打破了}\end{exemple}
\begin{exemple}\pjya{ɯʑo kɯ na-mdzoz}\hspace{5pt}\pcmn{他避开了}\end{exemple}
\begin{exemple}\pjya{kɯki to-rɯkɯmaʁ ri, aʑo nɯ-mdzoz-a tɕe mɯ-jɤ-ari-a}\hspace{5pt}\pcmn{他们家里出了事故(死了人),我回避了他的葬礼,没有去}\end{exemple}
\begin{exemple}\pjya{jiɕqha nɯ pjɤ-si, aʑo nɯ-mdzoz-a}\hspace{5pt}\pcmn{那个人死了,我回避了他的葬礼}\end{exemple}
\begin{exemple}\pjya{a-ɕqhe ɣɤʑu tɕe, cha ɲɯ-mdzoz-a ɲɯ-ntshi}\hspace{5pt}\pcmn{我咳嗽,忌酒}\end{exemple}\relationsémantique{参考}{\lien{ⓔtɯmdzoz}{tɯmdzoz}}\étymologie{mdzaŋs}\end{entrée}

\begin{entrée}{mdzurgi}{}{ⓔmdzurgi} 
\classe{n}  
\grammaire{n.lieu} 
\begin{définition}\pfra{Mdzorge}\end{définition}
\begin{définition}\pcmn{若尔盖}\end{définition}\end{entrée}

\begin{entrée}{mdzusŋun}{}{ⓔmdzusŋun} 
\classe{n} 
\begin{définition}\pfra{yak hybride au pelage bariolé}\end{définition}
\begin{définition}\pcmn{杂色的犏牛(看起来是青色的)}\end{définition}\étymologie{mdzo.sŋon}\end{entrée}

\begin{entrée}{mdzɯt}{}{ⓔmdzɯt} 
\classe{vi} \sens{1}
\begin{définition}\pfra{décider}\end{définition}
\begin{définition}\pcmn{决定,做主}\end{définition}
\begin{exemple}\pjya{nɤʑo tɯ-mdzɯt}\hspace{5pt}\pcmn{你说了算}\end{exemple}
\begin{exemple}\pjya{aʑo mɤ-nɯ-mdzɯt-a}\hspace{5pt}\pcmn{我做不了主}\end{exemple}
\begin{exemple}\pjya{ji-kɯ-mdzɯt}\hspace{5pt}\pcmn{我们的领导}\end{exemple}\sens{2}
\begin{définition}\pfra{(pas) forcément}\end{définition}
\begin{définition}\pcmn{(不)一定}\end{définition}
\begin{exemple}\pjya{nɯ ʁo mɤ-mdzɯt}\hspace{5pt}\pcmn{那个倒不一定}\end{exemple}\relationsémantique{同义词}{\lien{ⓔrɤmdzɯt}{rɤmdzɯt}}\end{entrée}

\begin{entrée}{mdzuzga}{}{ⓔmdzuzga} 
\classe{n} 
\begin{définition}\pfra{attelage}\end{définition}
\begin{définition}\pcmn{驮架}\end{définition}\étymologie{mdzo.sga}\end{entrée}

\begin{entrée}{mdʑɤl}{}{ⓔmdʑɤl} 
\classe{vi} \paradigme{dir}{tɤ-}
\begin{définition}\pfra{faire un pèlerinage}\end{définition}
\begin{définition}\pcmn{朝圣}\end{définition}
\begin{exemple}\pjya{ɯʑo ɬasa ɕ-to-mdʑɤl}\hspace{5pt}\pcmn{他去拉萨朝圣了}\end{exemple}\étymologie{mdʑal}\end{entrée}

\begin{entrée}{mdʐɯɕɯɣ}{}{ⓔmdʐɯɕɯɣ} 
\classe{n} 
\begin{définition}\pfra{punaise}\end{définition}
\begin{définition}\pcmn{臭虫}\end{définition}\étymologie{ⁿdre.ɕig}\end{entrée}

\begin{entrée}{me}{₂}{ⓔmeⓗ2} 
\classe{postp} 
\begin{définition}\pfra{que ce soit ... ou bien ...}\end{définition}
\begin{définition}\pcmn{不管……都}\end{définition}
\begin{exemple}\pjya{aʑo me, nɤʑo me, ɯʑo me, kɤsɯfse ɕe-j ra}\hspace{5pt}\pcmn{不管是我、你还是他,我们都要去}\end{exemple}\end{entrée}

\begin{entrée}{me}{₁}{ⓔmeⓗ1} 
\classe{vi} \paradigme{dir}{nɯ-}
\begin{définition}\pfra{not exist}\end{définition}
\begin{définition}\pcmn{不存在;没有}\end{définition}
\begin{exemple}\pjya{aʑo mɤ-kɯ-pe ku-me-a}\end{exemple}
\begin{exemple}\pjya{aʑɯɣ mɤ-kɯ-pe ku-me}\hspace{5pt}\pcmn{我没有什么不好的}\end{exemple}
\begin{exemple}\pjya{ɕkrɤz tɤ-me tɕe nɯnɯ xɕaj ɲɯ-βzu-nɯ sna}\hspace{5pt}\pcmn{没有青冈树的时候可以用\lien{ⓔxɕajⓗ1}{xɕaj}来代替}\end{exemple}
\begin{exemple}\pjya{ɯʑo nɯ-me}\hspace{5pt}\pcmn{他没有了(过世了)}\end{exemple}\relationsémantique{反义词}{\lien{ⓔtu}{tu}}\relationsémantique{同义词}{\lien{ⓔmaŋe}{maŋe}}\relationsémantique{参考}{\lien{ⓔtɤ-rcaⓝtɤ-rca,me}{tɤ-rca,me}}\end{entrée}

\begin{entrée}{mgrɯn}{}{ⓔmgrɯn} 
\classe{vt}  
\grammaire{secondatif} \paradigme{dir}{nɯ-}
\begin{définition}\pfra{recevoir (un hôte), régaler (un hôte) avec}\end{définition}
\begin{définition}\pcmn{接待;款待}\end{définition}\paradigme{dir}{nɯ-}
\begin{définition}\pfra{recevoir des hôtes}\end{définition}
\begin{définition}\pcmn{接待客人}\end{définition}
\begin{exemple}\pjya{ɯʑo kɯ ɯ-zda na-mgrɯn tɕe, tʂha na-jtshi}\hspace{5pt}\pcmn{他接待了客人,给他茶喝}\end{exemple}
\begin{exemple}\pjya{aʑo ci ɲɯ-ta-mgrɯn}\hspace{5pt}\pcmn{我请你吧}\end{exemple}
\begin{sous-entrée}{sɤmgrɯn/\variante{rɤmgrɯn}}{ⓔmgrɯnⓝsɤmgrɯn} 
\classe{vi} \end{sous-entrée}

\étymologie{mgron}\end{entrée}

\begin{entrée}{mgɯ}{}{ⓔmgɯ} 
\classe{vt} \paradigme{dir}{nɯ-}
\begin{définition}\pfra{avoir confiance en, admirer}\end{définition}
\begin{définition}\pcmn{佩服;信任}\end{définition}
\begin{exemple}\pjya{ɯʑo kɯ ɲɯ-tɯ́-wɣ-mgɯ}\hspace{5pt}\pcmn{他很信任你}\end{exemple}
\begin{exemple}\pjya{nɤʑo nɤ-ma tɤ-tɯ-nɤma-t nɯ ra ɲɯ-ta-mgɯ}\hspace{5pt}\pcmn{我很相信你把工作做好了}\end{exemple}
\begin{sous-entrée}{ʑɣɤmgɯ}{ⓔmgɯⓝʑɣɤmgɯ} 
\classe{vi} 
\begin{définition}\pfra{avoir confiance en soi}\end{définition}
\begin{définition}\pcmn{自信}\end{définition}
\begin{exemple}\pjya{nɯsthɯci nɤ-kɤ-cha kɯ-tu ci tɯ-ŋu ɕi kɯma, nɤ-tɯ-ʑɣɤmgɯ nɯ!}\hspace{5pt}\pcmn{你是不是有那么大的本事,你这么自信}\end{exemple}\end{sous-entrée}

\end{entrée}

\begin{entrée}{mɢom}{}{ⓔmɢom} 
\classe{vt} \paradigme{dir}{tɤ-}
\begin{définition}\pfra{se mordre les lèvres (de rage)}\end{définition}
\begin{définition}\pcmn{咬牙切齿(用上唇咬着下唇的表情)}\end{définition}
\begin{exemple}\pjya{to-sɤmbrɯ tɕe ɯ-mtɕhi to-mɢom}\hspace{5pt}\pcmn{他生气了就做出咬牙切齿的表情}\end{exemple}
\begin{exemple}\pjya{nɤ-mtɕhi ma-tɤ-tɯ-mɢom}\hspace{5pt}\pcmn{你不要咬牙切齿}\end{exemple}\relationsémantique{参考}{\lien{ⓔtamɢom}{tamɢom}}\end{entrée}

\begin{entrée}{mi}{₂}{ⓔmiⓗ2} 
\classe{n} 
\begin{définition}\pfra{peuplier}\end{définition}
\begin{définition}\pcmn{杨树}\end{définition}
\begin{exemple}\pjya{mi nɯ si kɯ-mbro kɯ-wxti ci ŋu, tɯ-ci kɯ-tu zɯ tu-ɬoʁ ŋu, wuma ʑo ɣɤ-wxti, ɯ-jwaʁ tɯ-sni ɯ-tshɯɣa kɯ-fse ŋu, ɯ-jwaʁ mba, mpɕu, kɯ-wxti tsa ɲɯ-βze cha, ɯ-mnɯ ɣɯ jwaʁ jndʐɤz, thɯ-do tɕe ɯ-jwaʁ ɲɯ-ndɯβ ŋu. mi ɣɯ ɯ-βri nɯ kɯ-ɤrŋi ŋu, ɯ-rtaʁ dɤn, ɯ-si nɯ kɤ-ntɕhoz sna, tɕe mpɕɤr ri khro mɤ-ngɯt.}\hspace{5pt}\pcmn{白杨是高大的树种,生长在有水的地方,长得很快,叶子薄、光滑、长得比较大,树苗的叶子很大,树老叶子也就变小。树身是绿色的,枝桠多,木料可以使用,很美但不是很结实。}\end{exemple}\end{entrée}

\begin{entrée}{mi}{₁}{ⓔmiⓗ1} 
\classe{vs} \paradigme{dir}{pɯ-}\paradigme{dir}{nɯ-}\paradigme{dir}{thɯ-}
\begin{définition}\pfra{s’éteindre}\end{définition}
\begin{définition}\pcmn{灭}\end{définition}
\begin{exemple}\pjya{smi ɲɤ-mi}\hspace{5pt}\pcmn{火灭了}\end{exemple}
\begin{exemple}\pjya{tɤtʂu ɲɤ-mi}\hspace{5pt}\pcmn{灯灭了}\end{exemple}
\begin{exemple}\pjya{ɣndʑɤβ pjɤ-mi}\hspace{5pt}\pcmn{火灾灭了}\end{exemple}
\begin{exemple}\pjya{pɯlthi kɤ-nɯt pɯ-tsu tɕe ʑaʑa mɯ́j-mi}\hspace{5pt}\pcmn{引火线点燃以后久久不熄}\end{exemple}\relationsémantique{参考}{\lien{ⓔɣɤmi}{ɣɤmi}}\end{entrée}

\begin{entrée}{mɟa}{}{ⓔmɟa} 
\classe{vt} \sens{1}\paradigme{dir}{nɯ-}\paradigme{dir}{kɤ-}\paradigme{dir}{\_}
\begin{définition}\pfra{prendre}\end{définition}
\begin{définition}\pcmn{拿;捡}\end{définition}
\begin{exemple}\pjya{nɯ-mɟe}\hspace{5pt}\pcmn{你拿一下}\end{exemple}
\begin{exemple}\pjya{nɤkɤcu tɕhaχɯ kɤ-mɟe}\hspace{5pt}\pcmn{你把那边的茶壶拿过来}\end{exemple}\sens{2}
\begin{définition}\pfra{absorber}\end{définition}
\begin{définition}\pcmn{吸收}\end{définition}
\begin{exemple}\pjya{a-phoŋbu kɯ smɤn mɯ́j-mɟe tɕe kɤ-nɯsmɤn mɯ́j-khɯ}\hspace{5pt}\pcmn{我吸收不了药,没有办法把病治好}\end{exemple}\sens{3}
\begin{définition}\pfra{recevoir (des mains de quelqu'un)}\end{définition}
\begin{définition}\pcmn{接到(从别人的手中)}\end{définition}
\begin{exemple}\pjya{tɤ-scoz ci nɯ-mɟa-t-a}\hspace{5pt}\pcmn{我接到了一封信}\end{exemple}\sens{4}
\begin{définition}\pfra{ouvrir (couvercle)}\end{définition}
\begin{définition}\pcmn{揭开}\end{définition}
\begin{exemple}\pjya{tɤ-fkaβ tɤ-mɟa-t-a (=tɤ-pɣaʁ-a)}\hspace{5pt}\pcmn{我揭开了盖子}\end{exemple}
\begin{sous-entrée}{sɯmɟa}{ⓔmɟaⓢ4ⓝsɯmɟa} 
\classe{vt} 
\begin{définition}\pfra{allumer}\end{définition}
\begin{définition}\pcmn{点燃}\end{définition}
\begin{exemple}\pjya{smi tɤ-sɯmɟa-t-a}\hspace{5pt}\pcmn{我点燃了火}\end{exemple}\relationsémantique{参考}{\lien{ⓔnɯmɟa}{nɯmɟa}}\relationsémantique{参考}{\lien{ⓔamɟɤkho}{amɟɤkho}}\relationsémantique{参考}{\lien{ⓔtɯ-mɟaⓗ2}{tɯ-mɟa₂}}\end{sous-entrée}

\end{entrée}

\begin{entrée}{mɟoʁra}{}{ⓔmɟoʁra} 
\classe{n} 
\begin{définition}\pfra{corne où l'on met la poudre à fusil}\end{définition}
\begin{définition}\pcmn{火药角}\end{définition}\étymologie{mgʲogs.rwa}\end{entrée}

\begin{entrée}{mkɤɣɯr}{}{ⓔmkɤɣɯr} 
\classe{n} 
\begin{définition}\pfra{collier}\end{définition}
\begin{définition}\pcmn{项链}\end{définition}\end{entrée}

\begin{entrée}{mkhu}{₁₂}{ⓔmkhuⓗ1ⓗ2} 
\classe{vs}
\classe{vi} 
\begin{définition}\pfra{être nécessiteux}\end{définition}
\begin{définition}\pcmn{缺吃少穿}\end{définition}\paradigme{dir}{tɤ-}
\begin{définition}\pfra{réclamer}\end{définition}
\begin{définition}\pcmn{向别人要}\end{définition}
\begin{exemple}\pjya{jiɕqha nɯ kɯ-mkhu ci ɕti}\hspace{5pt}\pcmn{那个人缺吃少穿}\end{exemple}
\begin{exemple}\pjya{tɤ-mkhu}\hspace{5pt}\pcmn{他要了(别人给他东西)}\end{exemple}
\begin{exemple}\pjya{ɯ-pɕawtsɯ maŋe, rŋɯl ɲɯ-mkhu}\hspace{5pt}\pcmn{他没有钱,很缺钱}\end{exemple}
\begin{exemple}\pjya{tɯ-ŋga ɲɯ-mkhu}\hspace{5pt}\pcmn{他缺衣服}\end{exemple}
\begin{exemple}\pjya{laχtɕha ci pjɤ-tu tɕe ɲɤ-nɯsŋom tɕe to-mkhu}\hspace{5pt}\pcmn{因为想得到那个东西,他向别人要了}\end{exemple}
\begin{sous-entrée}{mkhu}{ⓔmkhuⓗ1ⓝmkhu}\end{sous-entrée}

\begin{sous-entrée}{znɤmkhɯmkhu}{ⓔmkhuⓗ1ⓗ2ⓝznɤmkhɯmkhu} 
\classe{vs} 
\begin{définition}\pfra{réclamer des choses}\end{définition}
\begin{définition}\pcmn{向别人要东西}\end{définition}
\begin{exemple}\pjya{a-ɕki ɲɯ-znɤmkhɯmkhu}\hspace{5pt}\pcmn{他向我要了东西}\end{exemple}\end{sous-entrée}

\étymologie{mkʰo}\end{entrée}

\begin{entrée}{mkhɤrmaŋ}{}{ⓔmkhɤrmaŋ} 
\classe{n} 
\begin{définition}\pfra{peuple}\end{définition}
\begin{définition}\pcmn{百姓}\end{définition}\étymologie{ⁿkʰor.dmaŋs}\end{entrée}

\begin{entrée}{mkhɤrŋa}{}{ⓔmkhɤrŋa} 
\classe{n} 
\begin{définition}\pfra{gong}\end{définition}
\begin{définition}\pcmn{锣}\end{définition}\étymologie{ⁿkʰar.rŋa}\end{entrée}

\begin{entrée}{mkhɤscoʁ}{}{ⓔmkhɤscoʁ} 
\classe{n} 
\begin{définition}\pfra{masque}\end{définition}
\begin{définition}\pcmn{口罩}\end{définition}\end{entrée}

\begin{entrée}{mkhɤz}{}{ⓔmkhɤz} 
\classe{vs} \paradigme{dir}{tɤ-}\paradigme{dir}{thɯ-}
\begin{définition}\pfra{être bon à}\end{définition}
\begin{définition}\pcmn{擅长}\end{définition}
\begin{exemple}\pjya{ɕoŋβzu ɲɯ-mkhɤz}\hspace{5pt}\pcmn{木匠很厉害}\end{exemple}
\begin{exemple}\pjya{kɤ-rɤrɤt ɲɯ-mkhɤz}\hspace{5pt}\pcmn{他很擅长写字}\end{exemple}\étymologie{mkʰas}\end{entrée}

\begin{entrée}{mkhoŋ}{}{ⓔmkhoŋ} 
\classe{n} 
\begin{définition}\pfra{abri à bestiaux (dans les pâturages)}\end{définition}
\begin{définition}\pcmn{(牧场上的)牛棚}\end{définition}\end{entrée}

\begin{entrée}{mkhroŋ}{}{ⓔmkhroŋ} 
\classe{vi} \paradigme{dir}{tɤ-}
\begin{définition}\pfra{se réincarner}\end{définition}
\begin{définition}\pcmn{投胎;转世}\end{définition}
\begin{exemple}\pjya{sprɯskɯ to-mkhroŋ}\hspace{5pt}\pcmn{活佛转世了}\end{exemple}\étymologie{ⁿkʰruŋ}\end{entrée}

\begin{entrée}{mkhrɯmkhaŋ}{}{ⓔmkhrɯmkhaŋ} 
\classe{n} 
\begin{définition}\pfra{prison}\end{définition}
\begin{définition}\pcmn{监狱}\end{définition}\étymologie{kʰrims.kʰaŋ}\end{entrée}

\begin{entrée}{mkhrɯn}{}{ⓔmkhrɯn} 
\classe{vs} \paradigme{dir}{tɤ-}
\begin{définition}\pfra{avare}\end{définition}
\begin{définition}\pcmn{吝啬}\end{définition}
\begin{exemple}\pjya{ɲɯ-znɤje tɕe ɲɯ-mkhrɯn}\hspace{5pt}\pcmn{他很不舍得,很吝啬}\end{exemple}\étymologie{ⁿkʰren}\end{entrée}

\begin{entrée}{mkhɯrlu}{}{ⓔmkhɯrlu} 
\classe{n} 
\begin{définition}\pfra{roue}\end{définition}
\begin{définition}\pcmn{轮子}\end{définition}\étymologie{ⁿkʰor.lo}\end{entrée}

\begin{entrée}{mkhɯrlɤmnɯ}{}{ⓔmkhɯrlɤmnɯ} 
\classe{n} 
\begin{définition}\pfra{perceuse}\end{définition}
\begin{définition}\pcmn{钻子}\end{définition}\étymologie{ⁿkʰor.lo}\end{entrée}

\begin{entrée}{mkɯm}{}{ⓔmkɯm} 
\classe{vi} \paradigme{dir}{thɯ-}\paradigme{dir}{lɤ-}
\begin{définition}\pfra{avoir la tête tournée dans une certaine direction dans son lit}\end{définition}
\begin{définition}\pcmn{头朝着(哪个方向)睡}\end{définition}
\begin{exemple}\pjya{soz tɤ-tɕɯ nɯ rɤru tɤkha tɕe, tɕheme nɯ ɣɯ ɯ-jme pɕoʁ nɯ tɕu ntsɯ chɯ-mkɯm pjɤ-ŋu}\hspace{5pt}\pcmn{早上,那个男孩子起来的时候,发现自己的头总是朝着他妻子的脚(独角鬼13)}\end{exemple}\relationsémantique{参考}{\lien{ⓔtɤ-mkɯm}{tɤ-mkɯm}}\relationsémantique{参考}{\lien{ⓔnɤmkɯm}{nɤmkɯm}}\end{entrée}

\begin{entrée}{mkɯt}{}{ⓔmkɯt} 
\classe{vi} \paradigme{dir}{tɤ-}\paradigme{dir}{pɯ-}
\begin{définition}\pfra{avaler de travers (avec du xanthoxyle)}\end{définition}
\begin{définition}\pcmn{呛到(吃花椒的时候)}\end{définition}
\begin{exemple}\pjya{to-mkɯt-a}\hspace{5pt}\pcmn{我呛到了}\end{exemple}
\begin{sous-entrée}{sɯmkɯt}{ⓔmkɯtⓝsɯmkɯt} 
\classe{vt}  
\grammaire{caus} \end{sous-entrée}

\sens{1}
\begin{définition}\pfra{faire que quelqu'un s'étrangle, avale de travers}\end{définition}
\begin{définition}\pcmn{呛到}\end{définition}
\begin{exemple}\pjya{@cai ɯ-ŋgɯ tɕɣom ɣɤʑu tɕe pɯ́-wɣ-sɯmkɯt-a}\hspace{5pt}\pcmn{菜里面有花椒,把我呛到了}\end{exemple}\sens{2}
\begin{définition}\pfra{faire en sorte que quelqu'un n'arrive pas à répondre}\end{définition}
\begin{définition}\pcmn{使人答不上来(开玩笑的时候)}\end{définition}
\begin{exemple}\pjya{jiɕqha nɯ kɯ khɤβdɤr ɲɯ-ɤsɯ-βzu tɕe, ɯ-sci tɤ-βzu-t-a (=ɯ-ntsi tɤ-βzu-t-a) tɕe, pɯ-sɯmkɯ-t-a}\hspace{5pt}\pcmn{这个人在讲笑话,我也说了一句,让他答不上来了}\end{exemple}\end{entrée}

\begin{entrée}{mnu}{}{ⓔmnu} 
\classe{vs} \paradigme{dir}{nɯ-}\sens{1}
\begin{définition}\pfra{lisse}\end{définition}
\begin{définition}\pcmn{光滑}\end{définition}\sens{2}
\begin{définition}\pfra{doux}\end{définition}
\begin{définition}\pcmn{柔软}\end{définition}
\begin{exemple}\pjya{tɯ-ŋga ɲɯ-mnu}\hspace{5pt}\pcmn{衣服很柔软}\end{exemple}
\begin{exemple}\pjya{cha ɲɯ-mnu}\hspace{5pt}\pcmn{酒的浓度低}\end{exemple}\relationsémantique{参考}{\lien{ⓔmnumne}{mnumne}}\relationsémantique{参考}{\lien{ⓔmnule}{mnule}}\relationsémantique{反义词}{\lien{ⓔrʁom}{rʁom}}\relationsémantique{反义词}{\lien{ⓔtsaβ}{tsaβ}}\end{entrée}

\begin{entrée}{mna}{}{ⓔmna} 
\classe{vi} \paradigme{dir}{tɤ-}\sens{1}
\begin{définition}\pfra{meilleur}\end{définition}
\begin{définition}\pcmn{优秀}\end{définition}\sens{2}
\begin{définition}\pfra{guérir}\end{définition}
\begin{définition}\pcmn{痊愈;康复}\end{définition}
\begin{exemple}\pjya{a-ɕqhe to-mna}\hspace{5pt}\pcmn{我的咳嗽好了}\end{exemple}
\begin{exemple}\pjya{nɤ-ɕqhe ɯ-ɲɯ́-mna?}\hspace{5pt}\pcmn{你的咳嗽好了没有?}\end{exemple}
\begin{exemple}\pjya{smɤn tu-ndze-a tɕe, a-ɕqhe a-tɤ-mna}\hspace{5pt}\pcmn{我吃药,希望我的咳嗽会好}\end{exemple}
\begin{exemple}\pjya{(smɤnba kɯ tɤ́-wɣ-nɯsman-a) mna ɕi mɤ-mna mɤxsi}\hspace{5pt}\pcmn{医生给我治了病,不知道有没有效}\end{exemple}
\begin{exemple}\pjya{a-tɤ-mna tsa tɕe, tɕe nɯɣi}\hspace{5pt}\pcmn{(他母亲)好一些,他就会回来}\end{exemple}\relationsémantique{参考}{\lien{ⓔɣɤmna}{ɣɤmna}}\relationsémantique{反义词}{\lien{ⓔʑɤn}{ʑɤn}}\end{entrée}

\begin{entrée}{mnɤm}{}{ⓔmnɤm} 
\classe{vi} \paradigme{dir}{nɯ-}
\begin{définition}\pfra{avoir une odeur}\end{définition}
\begin{définition}\pcmn{发出气味}\end{définition}
\begin{exemple}\pjya{cha ɯ-di ɲɯ-mnɤm}\hspace{5pt}\pcmn{酒有味道}\end{exemple}
\begin{exemple}\pjya{ɯ-dɯχɯn ɲɯ-mnɤm}\hspace{5pt}\pcmn{香味很浓}\end{exemple}
\begin{exemple}\pjya{nɤki kɤ-ndza nɯ ɲɯ-mɲɤt tɕe ɯ-di ɲɯ-mnɤm}\hspace{5pt}\pcmn{那个食物坏了,有臭味了}\end{exemple}\relationsémantique{参考}{\lien{ⓔnɤmnɤm}{nɤmnɤm}}\relationsémantique{参考}{\lien{ⓔɕɯmnɤm}{ɕɯmnɤm}}\end{entrée}

\begin{entrée}{mnɤt}{}{ⓔmnɤt} 
\classe{vt} \paradigme{dir}{\_}
\begin{définition}\pfra{refaire}\end{définition}
\begin{définition}\pcmn{重新做}\end{définition}
\begin{exemple}\pjya{kɤ-nɤma tú-wɣ-sɤpe tɕe kɤ-mnɤt mɤra}\hspace{5pt}\pcmn{如果工作做好了的话,不需要重新做}\end{exemple}
\begin{exemple}\pjya{kɤ-nɤma tɤ-mnɤt ma mɯ́jpe}\hspace{5pt}\pcmn{你把工作重新做一遍,因为没有做好}\end{exemple}
\begin{exemple}\pjya{kɤ-rɤrɤt pɯ-mnɤt}\hspace{5pt}\pcmn{你重新写一遍}\end{exemple}
\begin{exemple}\pjya{kɤ-ti tɤ-mnɤt}\hspace{5pt}\pcmn{你重新讲一遍}\end{exemple}
\begin{exemple}\pjya{kɤ-ɕe ci jɤ-mnɤt ɲɯ-ntshi}\hspace{5pt}\pcmn{需要重新去一趟}\end{exemple}
\begin{exemple}\pjya{kɤ-nɯrɤɣo ci thɯ-mnɤt ɲɯ-ntshi}\hspace{5pt}\pcmn{要重新唱一遍}\end{exemple}\end{entrée}

\begin{entrée}{mnule}{}{ⓔmnule} 
\classe{vs} 
\begin{définition}\pfra{doux et chaux}\end{définition}
\begin{définition}\pcmn{又柔软又光滑又暖和}\end{définition}\relationsémantique{参考}{\lien{ⓔmnumne}{mnumne}}\relationsémantique{参考}{\lien{ⓔmnu}{mnu}}\end{entrée}

\begin{entrée}{mnumne}{}{ⓔmnumne} 
\classe{vs}  
\grammaire{rdpl} 
\begin{définition}\pfra{doux et chaux}\end{définition}
\begin{définition}\pcmn{又柔软又光滑又暖和}\end{définition}
\begin{exemple}\pjya{kɯ-mnumne ci ɲɯ-ŋu}\hspace{5pt}\pcmn{它又柔软又光滑}\end{exemple}\relationsémantique{参考}{\lien{ⓔmnu}{mnu}}\relationsémantique{参考}{\lien{ⓔmnule}{mnule}}\end{entrée}

\begin{entrée}{mɲaqrɯ}{}{ⓔmɲaqrɯ} 
\classe{n} 
\begin{définition}\pfra{regard}\end{définition}
\begin{définition}\pcmn{瞪眼}\end{définition}
\begin{exemple}\pjya{ɯʑo kɯ a-ɕki mɲaqrɯ ci to-lɤt}\hspace{5pt}\pcmn{他瞪了我一眼}\end{exemple}\relationsémantique{参考}{\lien{ⓔnɯmɲaqrɯ}{nɯmɲaqrɯ}}\end{entrée}

\begin{entrée}{mɲaʁlaχtɕhɯ}{}{ⓔmɲaʁlaχtɕhɯ} 
\classe{n} 
\begin{définition}\pfra{gaspillage}\end{définition}
\begin{définition}\pcmn{浪费}\end{définition}
\begin{exemple}\pjya{mɲaʁlaχtɕhɯ ma-nɯ-tɯ-sɯxɕe}\hspace{5pt}\pcmn{你不要浪费}\end{exemple}
\begin{exemple}\pjya{kɯki tɯ-mgo tɤ-βzu-t-a ri, koŋla mɯ-ta-ndza-nɯ tɕe mɲaʁlaχtɕhɯ nɯ-ari}\hspace{5pt}\pcmn{我做的饭他们没有吃完,被浪费了}\end{exemple}\end{entrée}

\begin{entrée}{mɲaʁmtsaʁ}{}{ⓔmɲaʁmtsaʁ} 
\classe{n} 
\begin{définition}\pfra{criquet}\end{définition}
\begin{définition}\pcmn{蝗虫}\end{définition}\end{entrée}

\begin{entrée}{mɲaʁtɕhɯβ}{}{ⓔmɲaʁtɕhɯβ} 
\classe{n} 
\begin{définition}\pfra{clin d'œil}\end{définition}
\begin{définition}\pcmn{眨眼}\end{définition}
\begin{exemple}\pjya{mɲaʁtɕhɯβ ci pɯ-lat-a}\hspace{5pt}\pcmn{我眨了眼}\end{exemple}\end{entrée}

\begin{entrée}{mɲɤm}{}{ⓔmɲɤm} 
\classe{n} 
\begin{définition}\pfra{espèce d'arbre}\end{définition}
\begin{définition}\pcmn{【野白杨】}\end{définition}
\begin{exemple}\pjya{mɲɤm nɯ sɤtɕha kɯ-mbɤr tsa zgoku tu-ɬoʁ ŋu, ɯ-jwaʁ nɯ mi ɣɯ cho naχtɕɯɣ ri xtɕi cho mba, ɯ-ru kɯ-pɣi ŋu, wuma ʑo ɣɤ-wxti, ɯ-si nɯ mɯ-tɤ-nɤkhɯ mɤɕtʂa mɤ-ngɯt}\hspace{5pt}\pcmn{野白杨生长在半山上,叶子和白杨一样,但比较小和薄,树干是灰色的,长得很快,木质要经过烟熏才结实。}\end{exemple}\end{entrée}

\begin{entrée}{mɲɤt}{}{ⓔmɲɤt} 
\classe{vs} \paradigme{dir}{nɯ-}
\begin{définition}\pfra{se détériorer (nourriture), faible, maigre (personne)}\end{définition}
\begin{définition}\pcmn{食物变味;人虚弱、瘦}\end{définition}
\begin{exemple}\pjya{tɤ-mthɯm ɲɤ-mɲɤt}\hspace{5pt}\pcmn{肉变味了}\end{exemple}
\begin{exemple}\pjya{nɤʑo ɲɤ-tɯ-mɲɤt}\hspace{5pt}\pcmn{你瘦了}\end{exemple}
\begin{sous-entrée}{ɣɤmɲɤt}{ⓔmɲɤtⓝɣɤmɲɤt} 
\classe{vs} 
\begin{définition}\pfra{qui se détériore facilement}\end{définition}
\begin{définition}\pcmn{容易变味}\end{définition}\end{sous-entrée}

\end{entrée}

\begin{entrée}{mɲi}{}{ⓔmɲi} 
\classe{vs}
\classe{vt} \paradigme{dir}{pɯ-}\paradigme{dir}{pɯ-}
\begin{définition}\pfra{peu}\end{définition}
\begin{définition}\pcmn{少}\end{définition}
\begin{définition}\pfra{donner une part trop petite}\end{définition}
\begin{définition}\pcmn{分得少}\end{définition}
\begin{exemple}\pjya{tɯkro pa-βzu tɕe, tsuku ɣɯ pa-ɣɤdɤn, tsuku ɣɯ pa-ɣɤmɲi}\hspace{5pt}\pcmn{他分东西的时候,有些人分得多,有些人分得少}\end{exemple}\relationsémantique{同义词}{\lien{ⓔrkɯn}{rkɯn}}
\begin{sous-entrée}{ɣɤmɲi}{ⓔmɲiⓝɣɤmɲi} 
\classe{vt} \end{sous-entrée}

\begin{sous-entrée}{sɤdɤmɲi}{ⓔmɲiⓝsɤdɤmɲi}\end{sous-entrée}

\begin{définition}\pfra{partager de façon injuste}\end{définition}
\begin{définition}\pcmn{分得不均匀}\end{définition}
\begin{exemple}\pjya{kɯ-rɤkro tɤ-kɯ-ŋu tɕe kɤ-sɤdɤmɲi mɤ-pe}\hspace{5pt}\pcmn{分东西的时候,不要分得不均匀}\end{exemple}\end{entrée}

\begin{entrée}{mɲo}{₁}{ⓔmɲoⓗ1} 
\classe{vt} \paradigme{dir}{tɤ-}\paradigme{dir}{tɤ-}\paradigme{dir}{tɤ-}
\begin{définition}\pfra{préparer}\end{définition}
\begin{définition}\pcmn{准备}\end{définition}
\begin{définition}\pfra{préparer}\end{définition}
\begin{définition}\pcmn{准备}\end{définition}
\begin{définition}\pfra{se préparer}\end{définition}
\begin{définition}\pcmn{准备}\end{définition}
\begin{exemple}\pjya{laχtɕha tɤ-mɲɤm}\hspace{5pt}\pcmn{你准备东西吧}\end{exemple}
\begin{exemple}\pjya{ɕoŋtɕa tɤ-mɲɤm}\hspace{5pt}\pcmn{你准备木料吧}\end{exemple}
\begin{exemple}\pjya{kɤ-ndza tɤ-mɲɤm}\hspace{5pt}\pcmn{你准备食物吧}\end{exemple}
\begin{exemple}\pjya{nɤ-ŋga tɤ-mɲo-t-a}\hspace{5pt}\pcmn{我给你准备了衣服}\end{exemple}
\begin{exemple}\pjya{nɤ-tʂha tɤ-mɲo-t-a}\hspace{5pt}\pcmn{我给你准备早饭}\end{exemple}
\begin{exemple}\pjya{nɤ-kɤ-nɤma tɤ-mɲo-t-a}\hspace{5pt}\pcmn{我准备了你的工作}\end{exemple}
\begin{exemple}\pjya{kɯ-ɕe tɤ-ʑɣɤmɲo-a}\hspace{5pt}\pcmn{我准备去}\end{exemple}\relationsémantique{参考}{\lien{ⓔsɯmɲo}{sɯmɲo}}\relationsémantique{同义词}{\lien{ⓔnɯftɕaka}{nɯftɕaka}}\relationsémantique{同义词}{\lien{ⓔrɤŋgat}{rɤŋgat}}
\begin{sous-entrée}{rɤmɲo}{ⓔmɲoⓗ1ⓝrɤmɲo} 
\classe{vi}  
\grammaire{apass} \end{sous-entrée}

\begin{sous-entrée}{ʑɣɤmɲo}{ⓔmɲoⓗ1ⓝʑɣɤmɲo} 
\classe{vi}  
\grammaire{refl} \end{sous-entrée}

\begin{sous-entrée}{ɯ-mɲoz}{ⓔmɲoⓗ1ⓝɯ-mɲoz}
\begin{exemple}\pjya{pjɯ-ɲɟo ɕɯŋgɯ tɕe ɯ-mɲoz tú-wɣ-βzu ra}\hspace{5pt}\pcmn{在损失发生之前我做好准备了}\end{exemple}
\begin{exemple}\pjya{mɤ-mbɯt ɯ-jtsi, mɤ-ɲɟo ɯ-mɲoz}\hspace{5pt}\pcmn{倒塌之前顶柱子,损失到来以前要预防}\end{exemple}\end{sous-entrée}

\end{entrée}

\begin{entrée}{mɲo}{₂}{ⓔmɲoⓗ2} 
\classe{vs} 
\begin{définition}\pfra{être prévisible}\end{définition}
\begin{définition}\pcmn{在预料之中}\end{définition}
\begin{exemple}\pjya{nɤ-laχtɕha tɯ-nɯβde ɲɯ-mɲo ɕti ma aʁɤndɯndɤt ʑo ɕɯ-tɯ-khɤt ɲɯ-ra}\hspace{5pt}\pcmn{你总是到处走,肯定会丢东西}\end{exemple}
\begin{exemple}\pjya{ɯʑo ndʐaβ ɲɯ-mɲo ɕti ma tɤ-ŋke tɕe tʂu pjɯ-ru ɲɯ-maʁ}\hspace{5pt}\pcmn{他走路的时候不看路,肯定会摔跤}\end{exemple}
\begin{sous-entrée}{amɲɯmɲo}{ⓔmɲoⓗ2ⓝamɲɯmɲo} 
\classe{vs} 
\begin{définition}\pfra{comme prévu}\end{définition}
\begin{définition}\pcmn{预料之中}\end{définition}
\begin{exemple}\pjya{ɯʑo kɯ jɯfɕɯr "ɣi-a" ɲɯ-ti tɕe, jɯxɕo tɕe amɲɯmɲo ʑo jo-ɣi}\hspace{5pt}\pcmn{他昨天说要来,今天早上果然就来了}\end{exemple}\relationsémantique{参考}{\lien{ⓔmɲoⓗ1}{mɲo₁}}\end{sous-entrée}

\end{entrée}

\begin{entrée}{mɲoχpɣa}{}{ⓔmɲoχpɣa} 
\classe{n} 
\begin{définition}\pfra{décorations (sur les pains à la vapeur)}\end{définition}
\begin{définition}\pcmn{包子上的花纹}\end{définition}\end{entrée}

\begin{entrée}{mɲɯka}{}{ⓔmɲɯka} 
\classe{n} 
\begin{définition}\pfra{humiliation}\end{définition}
\begin{définition}\pcmn{耻辱}\end{définition}
\begin{exemple}\pjya{ɯ-mɲɯka pjɤ-ɬoʁ}\hspace{5pt}\pcmn{他受耻辱了}\end{exemple}\relationsémantique{参考}{\lien{ⓔnɯmɲɯka}{nɯmɲɯka}}\end{entrée}

\begin{entrée}{mɲɯmaŋ}{}{ⓔmɲɯmaŋ} 
\classe{n} 
\begin{définition}\pfra{tout le monde}\end{définition}
\begin{définition}\pcmn{群众}\end{définition}\étymologie{mi.dmaŋs}\end{entrée}

\begin{entrée}{mɲɯrgɤt}{}{ⓔmɲɯrgɤt} 
\classe{n} 
\begin{définition}\pfra{yéti}\end{définition}
\begin{définition}\pcmn{野人}\end{définition}\étymologie{mi.rgod}\end{entrée}

\begin{entrée}{mɲɯrɟɤβ}{}{ⓔmɲɯrɟɤβ} 
\classe{n} 
\begin{définition}\pfra{laic}\end{définition}
\begin{définition}\pcmn{凡人,非宗教徒}\end{définition}\end{entrée}

\begin{entrée}{mɲɯrɟit}{}{ⓔmɲɯrɟit} 
\classe{n} 
\begin{définition}\pfra{progéniture}\end{définition}
\begin{définition}\pcmn{孩子}\end{définition}\étymologie{mi}\end{entrée}

\begin{entrée}{mɲɯrɯri}{}{ⓔmɲɯrɯri} 
\classe{n} 
\begin{définition}\pfra{chaque personne}\end{définition}
\begin{définition}\pcmn{每个人}\end{définition}
\begin{exemple}\pjya{mɲɯrɯri nɯ kɯ nɯ-kɤ-sɯso nɯ to-nɯ-ti-nɯ}\hspace{5pt}\pcmn{每个人讲了自己的想法}\end{exemple}
\begin{exemple}\pjya{mɲɯrɯri nɯ ɯ-tɯ-rju ɯ-tshɯɣa mɯ́j-naχtɕɯɣ}\hspace{5pt}\pcmn{每个人讲话的方式都不一样}\end{exemple}\étymologie{mi.re.re}\end{entrée}

\begin{entrée}{mɲɯʁʑi}{}{ⓔmɲɯʁʑi} 
\classe{n} 
\begin{définition}\pfra{humeur}\end{définition}
\begin{définition}\pcmn{脾气,态度}\end{définition}
\begin{exemple}\pjya{ɯ-mɲɯʁʑi βdi}\hspace{5pt}\pcmn{他脾气好}\end{exemple}\relationsémantique{参考}{\lien{ⓔnɯmɲɯʁʑi}{nɯmɲɯʁʑi}}\étymologie{mi.gʑi}\end{entrée}

\begin{entrée}{mɲɯtɕhɤz}{}{ⓔmɲɯtɕhɤz} 
\classe{n} 
\begin{définition}\pfra{humeur}\end{définition}
\begin{définition}\pcmn{脾气}\end{définition}
\begin{exemple}\pjya{ɯ-mɲɯtɕhɤz tu}\hspace{5pt}\pcmn{他有他的脾气}\end{exemple}
\begin{exemple}\pjya{nɤ-mɲɯtɕhɤz ɲɯ-βdi (ɲɯ-tɯ-nɯmɲɯtɕhɤz)}\hspace{5pt}\pcmn{你脾气很好}\end{exemple}\relationsémantique{参考}{\lien{ⓔmɲɯʁʑi}{mɲɯʁʑi}}\end{entrée}

\begin{entrée}{mŋaʁnɤmŋaʁ}{}{ⓔmŋaʁnɤmŋaʁ} 
\classe{idph.3} 
\begin{définition}\pfra{ouvrant la bouche avec difficulté, à l'article de la mort}\end{définition}
\begin{définition}\pcmn{形容嘴巴一开一张,有气无力的样子}\end{définition}
\begin{exemple}\pjya{ɯ-kɯ-mŋɤm ɯ-tɯ-thɯ kɯ ɯ-ɣmɤr mŋaʁnɤmŋaʁ tu-ste ma mɯ́j-cha}\hspace{5pt}\pcmn{他病得很严重,嘴巴只能慢慢地一开一张(没有力气把话说出来)}\end{exemple}\end{entrée}

\begin{entrée}{mŋɤm}{}{ⓔmŋɤm} 
\classe{vs} \paradigme{dir}{tɤ-}
\begin{définition}\pfra{avoir mal}\end{définition}
\begin{définition}\pcmn{痛}\end{définition}
\begin{exemple}\pjya{a-xtu ɲɯ-mŋɤm}\hspace{5pt}\pcmn{我肚子很痛}\end{exemple}
\begin{exemple}\pjya{ndʑi-kɯ-mŋɤm ɯβrɤ-pɯ-tu?}\hspace{5pt}\pcmn{你们俩没有生病吧?}\end{exemple}
\begin{sous-entrée}{nɤmŋɤm}{ⓔmŋɤmⓝnɤmŋɤm} 
\classe{vt}  
\grammaire{trop} 
\begin{définition}\pfra{avoir mal à}\end{définition}
\begin{définition}\pcmn{感觉到痛}\end{définition}
\begin{exemple}\pjya{ɯ-xtu ɲɯ-nɤmŋɤm}\hspace{5pt}\pcmn{他感到肚子痛}\end{exemple}\relationsémantique{参考}{\lien{ⓔɕɯmŋɤm}{ɕɯmŋɤm}}\end{sous-entrée}

\end{entrée}

\begin{entrée}{mŋɤrmŋɤr}{}{ⓔmŋɤrmŋɤr} 
\classe{idph.2} 
\begin{définition}\pfra{regardant en allongeant le cou}\end{définition}
\begin{définition}\pcmn{形容伸着脖子到处张望的样子}\end{définition}
\begin{exemple}\pjya{khɤxtu ri ɲɯ-ndzur tɕe mŋɤrmŋɤr ʑo pjɯ-nɤrɯra ɲɯ-ŋu}\hspace{5pt}\pcmn{他在房背上伸着脖子到处张望}\end{exemple}\end{entrée}

\begin{entrée}{mŋulɤn}{}{ⓔmŋulɤn} 
\classe{n} 
\begin{définition}\pfra{semelle}\end{définition}
\begin{définition}\pcmn{鞋底}\end{définition}\end{entrée}

\begin{entrée}{mŋurɯm}{}{ⓔmŋurɯm} 
\classe{n} 
\begin{définition}\pfra{ouverture qui peut se refermer en tirant sur un fil (pantalon, sac)}\end{définition}
\begin{définition}\pcmn{(可以收拢的)口(口袋、裤子)}\end{définition}
\begin{exemple}\pjya{lʁɤtɕɯ mŋurɯm kɤ-βzu-t-a}\hspace{5pt}\pcmn{我把口袋的口收拢了。}\end{exemple}\end{entrée}

\begin{entrée}{mŋɯn}{}{ⓔmŋɯn} 
\classe{vs} \paradigme{dir}{pɯ-}\sens{1}
\begin{définition}\pfra{content (d'avoir obtenu quelque chose)}\end{définition}
\begin{définition}\pcmn{(得到了某个东西)很满意}\end{définition}
\begin{exemple}\pjya{laχtɕha nɯ-kɯ-mbi-a nɯ pɯ-mŋɯn}\hspace{5pt}\pcmn{你给了我的东西,我很满意(我很需要这个东西)}\end{exemple}
\begin{exemple}\pjya{kɯ-mŋɯn maŋe}\hspace{5pt}\pcmn{不需要(这个东西)}\end{exemple}
\begin{exemple}\pjya{wo ki pɯ-mŋɯn rcanɯ wuma pɯ-pe}\hspace{5pt}\pcmn{很需要(这个东西),很好}\end{exemple}
\begin{exemple}\pjya{nɯ́-wɣ-mbi-a tɕe pɯ-mŋɯn}\hspace{5pt}\pcmn{他给了我,我很满意}\end{exemple}
\begin{exemple}\pjya{kɯ-mŋɯn ʑo ta-ndza}\hspace{5pt}\pcmn{他很痛快地吃了}\end{exemple}\sens{2}
\begin{définition}\pfra{rassuré}\end{définition}
\begin{définition}\pcmn{放心;心里踏实}\end{définition}
\begin{exemple}\pjya{kɤ-nɯna mɯ́j-mŋɯn}\hspace{5pt}\pcmn{(工作没有做好就)无法安心地休息}\end{exemple}\relationsémantique{同义词}{\lien{}{tʂɯn}}\relationsémantique{参考}{\lien{ⓔnɤmŋɯn}{nɤmŋɯn}}\end{entrée}

\begin{entrée}{mochi}{}{ⓔmochi} 
\classe{n} 
\begin{définition}\pfra{chien}\end{définition}
\begin{définition}\pcmn{狗}\end{définition}\étymologie{kʰʲi}\end{entrée}

\begin{entrée}{moɣɤz}{}{ⓔmoɣɤz} 
\classe{n} 
\begin{définition}\pfra{habit en laine féminin}\end{définition}
\begin{définition}\pcmn{女子的毛织品衣服}\end{définition}\end{entrée}

\begin{entrée}{moli}{}{ⓔmoli} 
\classe{n} 
\begin{définition}\pfra{chatte}\end{définition}
\begin{définition}\pcmn{母猫}\end{définition}\end{entrée}

\begin{entrée}{moʁ}{}{ⓔmoʁ} 
\classe{vt} \paradigme{dir}{tɤ-}\paradigme{dir}{tɤ-}
\begin{définition}\pfra{manger de la tsampa}\end{définition}
\begin{définition}\pcmn{吃粉状的食物;吃干糌粑}\end{définition}
\begin{définition}\pfra{donner de la tsampa à manger}\end{définition}
\begin{définition}\pcmn{给……吃干糌粑}\end{définition}
\begin{exemple}\pjya{tɯ-ɣndʑɤr to-moʁ}\hspace{5pt}\pcmn{他吃了干糌粑}\end{exemple}
\begin{exemple}\pjya{aʑo tɯ-ɣndʑɤr tɤ-moʁ-a}\hspace{5pt}\pcmn{我吃了干糌粑}\end{exemple}
\begin{exemple}\pjya{nɤ-ɣndʑɤr ci tu-kɯ-sɯɣmoʁ-a}\hspace{5pt}\pcmn{给我吃一点糌粑}\end{exemple}
\begin{sous-entrée}{sɯɣmoʁ/\variante{sɯmoʁ}}{ⓔmoʁⓝsɯɣmoʁ} 
\classe{vt} \end{sous-entrée}

\end{entrée}

\begin{entrée}{moʁmoʁ}{}{ⓔmoʁmoʁ} 
\classe{idph.2} 
\begin{définition}\pfra{très fine (poudre)}\end{définition}
\begin{définition}\pcmn{形容粉末极细,或者土壤软而看起来肥沃的样子}\end{définition}
\begin{exemple}\pjya{sɯβɣi ɲɯ-ndɯβ moʁmoʁ ʑo}\hspace{5pt}\pcmn{木屑是细的}\end{exemple}
\begin{exemple}\pjya{βɣa kɯ tɯ-ɣndʑɤr kɯ-ndɯβ ʑo moʁmoʁ chɤ-tɕɤt (chɤ-ɣɤndɯβ ʑo moʁmoʁ)}\hspace{5pt}\pcmn{水磨把糌粑磨得很细}\end{exemple}\end{entrée}

\begin{entrée}{mpɤβmpɤβ}{}{ⓔmpɤβmpɤβ} 
\classe{idph.2} 
\begin{définition}\pfra{épais et mou}\end{définition}
\begin{définition}\pcmn{形容泡沫或蘑菇厚而软的样子}\end{définition}\relationsémantique{同义词}{\lien{ⓔmphɤβmphɤβ}{mphɤβmphɤβ}}\end{entrée}

\begin{entrée}{mpɕu}{}{ⓔmpɕu} 
\classe{vs} \paradigme{dir}{nɯ-}\paradigme{dir}{nɯ-}
\begin{définition}\pfra{lisse}\end{définition}
\begin{définition}\pcmn{光滑}\end{définition}
\begin{définition}\pfra{rendre lisse}\end{définition}
\begin{définition}\pcmn{弄光滑}\end{définition}
\begin{exemple}\pjya{laʁdɯn ɯ-jɯ ɲɯ-mpɕu}\hspace{5pt}\pcmn{工具的把子是光滑的}\end{exemple}
\begin{exemple}\pjya{tɤ-jtsi ɲɯ-mpɕu}\hspace{5pt}\pcmn{柱子是光滑的}\end{exemple}
\begin{exemple}\pjya{tɯ-ci kɯ tɕhɯrdu ɲo-ɣɤmpɕu ʑo}\hspace{5pt}\pcmn{水把卵石磨光滑了}\end{exemple}
\begin{exemple}\pjya{ɲɯ́-wɣ-ɣɤmpɕu tɕe kɤ-ŋga sɤscit}\hspace{5pt}\pcmn{把衣服弄软了穿起来舒服}\end{exemple}
\begin{sous-entrée}{ɣɤmpɕu}{ⓔmpɕuⓝɣɤmpɕu} 
\classe{vt} \end{sous-entrée}

\end{entrée}

\begin{entrée}{mpɕa}{}{ⓔmpɕa} 
\classe{vt} \paradigme{dir}{tɤ-}
\begin{définition}\pfra{reprocher, ne pas pouvoir pardonner}\end{définition}
\begin{définition}\pcmn{责怪}\end{définition}
\begin{exemple}\pjya{nɤʑɯɣ nɯ mɯ-tɤ-nɤma-t-a tɕe ma-tɤ-kɯ-mpɕa-a a ma}\hspace{5pt}\pcmn{你的那个我没有做,不要责怪我}\end{exemple}\étymologie{ⁿpʰʲa}\end{entrée}

\begin{entrée}{mpɕɤr}{}{ⓔmpɕɤr} 
\classe{vi} \paradigme{dir}{thɯ-}
\begin{définition}\pfra{beau}\end{définition}
\begin{définition}\pcmn{美,好看,好听}\end{définition}\relationsémantique{参考}{\lien{ⓔɯ-kɯmpɕɤr}{ɯ-kɯmpɕɤr}}
\begin{sous-entrée}{ɣɤmpɕɤr}{ⓔmpɕɤrⓝɣɤmpɕɤr} 
\classe{vt}  
\grammaire{caus} 
\begin{définition}\pfra{rendre beau}\end{définition}
\begin{définition}\pcmn{令……变漂亮}\end{définition}
\begin{exemple}\pjya{tɯpɤr pɯ-lat-a tɕe, mɯ-nɯ-ta-ɣɤmpɕɤr ɯ́-ŋu?}\hspace{5pt}\pcmn{我拍照片的时候,把你拍得不好看吗?}\end{exemple}\relationsémantique{参考}{\lien{ⓔnɤmpɕɤr}{nɤmpɕɤr}}\relationsémantique{参考}{\lien{ⓔrɤmpɕɤr}{rɤmpɕɤr}}\end{sous-entrée}

\étymologie{mtɕʰor}\end{entrée}

\begin{entrée}{mpɕɯmɤr}{}{ⓔmpɕɯmɤr} 
\classe{n} 
\begin{définition}\pfra{célébration}\end{définition}
\begin{définition}\pcmn{庆祝}\end{définition}
\begin{exemple}\pjya{ɯ-mpɕɯmɤr tɤ-βzu-t-a}\hspace{5pt}\pcmn{我祝贺了他}\end{exemple}\end{entrée}

\begin{entrée}{mphɤβmphɤβ}{}{ⓔmphɤβmphɤβ} 
\classe{idph.2} 
\begin{définition}\pfra{épais et mou}\end{définition}
\begin{définition}\pcmn{形容泡沫或蘑菇厚而软的样子}\end{définition}\relationsémantique{同义词}{\lien{ⓔmpɤβmpɤβ}{mpɤβmpɤβ}}\end{entrée}

\begin{entrée}{mphɣaʁmphɣaʁ}{}{ⓔmphɣaʁmphɣaʁ} 
\classe{idph.2} 
\begin{définition}\pfra{très serré}\end{définition}
\begin{définition}\pcmn{形容紧的样子}\end{définition}
\begin{exemple}\pjya{ɯ-mthɤɣ mphɣaʁmphɣaʁ ʑo ko-xtɕɤr}\hspace{5pt}\pcmn{他把腰带系得很紧}\end{exemple}\end{entrée}

\begin{entrée}{mphrɤt}{}{ⓔmphrɤt} 
\classe{vs} \paradigme{dir}{tɤ-}
\begin{définition}\pfra{bien fermé, sans interstice}\end{définition}
\begin{définition}\pcmn{关紧,密封,没有缝隙}\end{définition}
\begin{exemple}\pjya{rgɤm ɲɯ-mphrɤt}\hspace{5pt}\pcmn{盒子关得很密封}\end{exemple}
\begin{exemple}\pjya{khɯɣɲɟɯ mɯ́j-mphrɤt tɕe a-lɤ-mphrɤt ɲɯ-ra}\hspace{5pt}\pcmn{窗子没有关紧,要关紧}\end{exemple}\sens{1}\paradigme{dir}{tɤ-}
\begin{définition}\pfra{mettre ensemble (des pièces) de façon parfaitement bien agencée}\end{définition}
\begin{définition}\pcmn{使……变得严密}\end{définition}
\begin{exemple}\pjya{kɤ-sprɤt to-sɯ-mphrɤt}\hspace{5pt}\pcmn{他(把零件)组合得很严密}\end{exemple}
\begin{sous-entrée}{sɯmphrɤt}{ⓔmphrɤtⓢ1ⓝsɯmphrɤt} 
\classe{vt}  
\grammaire{caus} \end{sous-entrée}

\sens{2}\paradigme{dir}{\_}
\begin{définition}\pfra{bien fermer}\end{définition}
\begin{définition}\pcmn{关紧}\end{définition}
\begin{exemple}\pjya{aʑo khɯɣɲɟɯ kɤ-sɯmphrat-a}\hspace{5pt}\pcmn{我关了窗子}\end{exemple}
\begin{exemple}\pjya{kɯm mɯ-chɤ-sɯmphrat-a tɕe ɯ-pɕi tɤ-zgra nɯ ɲɯ-sɤmtshɤm}\hspace{5pt}\pcmn{我没有把门关紧听到外面的声音}\end{exemple}\étymologie{ⁿpʰrod}\end{entrée}

\begin{entrée}{mphrɯɣ}{}{ⓔmphrɯɣ} 
\classe{n} 
\begin{définition}\pfra{habit tibétain en laine}\end{définition}
\begin{définition}\pcmn{氆氇}\end{définition}\étymologie{pʰrug}\end{entrée}

\begin{entrée}{mphrɯmdɯt}{}{ⓔmphrɯmdɯt} 
\classe{n} 
\begin{définition}\pfra{groupe de neuf nœuds (sur un khatag ou avec un fil normal)}\end{définition}
\begin{définition}\pcmn{在哈达上打九个结}\end{définition}\étymologie{ⁿpʰreŋ.mdud}\end{entrée}

\begin{entrée}{mphrɯmɯ}{}{ⓔmphrɯmɯ} 
\classe{n} 
\begin{définition}\pfra{prédiction}\end{définition}
\begin{définition}\pcmn{看相;算命}\end{définition}
\begin{exemple}\pjya{βlama kɯ a-mphrɯmɯ pa-ru}\hspace{5pt}\pcmn{喇嘛给我算了命}\end{exemple}\relationsémantique{参考}{\lien{ⓔrɯmphrɯmɯ}{rɯmphrɯmɯ}}\end{entrée}

\begin{entrée}{mphruwa}{}{ⓔmphruwa} 
\classe{n} 
\begin{définition}\pfra{chapelet}\end{définition}
\begin{définition}\pcmn{玛尼珠子}\end{définition}\end{entrée}

\begin{entrée}{mphɯl}{}{ⓔmphɯl} 
\classe{vi} \paradigme{dir}{nɯ-}
\begin{définition}\pfra{se multiplier (animaux)}\end{définition}
\begin{définition}\pcmn{繁殖(动物)}\end{définition}
\begin{exemple}\pjya{paʁ ɲɤ-mphɯl}\hspace{5pt}\pcmn{猪繁殖了}\end{exemple}\étymologie{ⁿpʰel}\end{entrée}

\begin{entrée}{mphɯli}{}{ⓔmphɯli} 
\classe{n} 
\begin{définition}\pfra{fibres de sésame}\end{définition}
\begin{définition}\pcmn{芝麻皮(用来制火绒、麻布的纬线)}\end{définition}\end{entrée}

\begin{entrée}{mphɯr}{}{ⓔmphɯr} 
\classe{vt}  
\grammaire{refl}
\grammaire{refl}
\grammaire{caus} \paradigme{dir}{kɤ-}\paradigme{dir}{tɤ-}\paradigme{dir}{tɤ-}\paradigme{dir}{kɤ-}\paradigme{dir}{kɤ-}
\begin{définition}\pfra{envelopper}\end{définition}
\begin{définition}\pcmn{包}\end{définition}
\begin{définition}\pfra{préparer des pains ou des raviolis}\end{définition}
\begin{définition}\pcmn{包包子;包饺子}\end{définition}
\begin{définition}\pfra{conserver qqch en l'enveloppant dans toutes sortes de choses}\end{définition}
\begin{définition}\pcmn{为了保存某个东西,把它一直包来包去}\end{définition}
\begin{exemple}\pjya{kɯ-chi ɕoʁɕoʁ ɯ-ŋgɯ tɤ-mphɯr-a}\hspace{5pt}\pcmn{我把这颗糖包在纸里了}\end{exemple}
\begin{exemple}\pjya{ki tɯ-ŋga ki tɤ-mphɯr}\hspace{5pt}\pcmn{你把衣服折起来}\end{exemple}
\begin{exemple}\pjya{smɤn tɤ-mphɯr-a}\hspace{5pt}\pcmn{我包了药}\end{exemple}
\begin{exemple}\pjya{ɕkɤbɯ kɤ-mphɯr-a}\hspace{5pt}\pcmn{我包了韭菜包子}\end{exemple}
\begin{exemple}\pjya{kɤ-rɤmphɯr-a}\hspace{5pt}\pcmn{我包了包子}\end{exemple}
\begin{exemple}\pjya{ki laχtɕha ki a-mɤ-nɯ-me nɯ-sɯso-t-a tɕe, pɯ-nɤmphoʁmphɯr-a ntsɯ ɕti}\hspace{5pt}\pcmn{我怕把这个东西丢失了,所以一直把它包在一个地方}\end{exemple}
\begin{sous-entrée}{rɤmphɯr}{ⓔmphɯrⓝrɤmphɯr} 
\classe{vi} \end{sous-entrée}

\begin{sous-entrée}{nɤmphoʁmphɯr}{ⓔmphɯrⓝnɤmphoʁmphɯr} 
\classe{vt} \end{sous-entrée}

\begin{sous-entrée}{ʑɣɤmphɯr}{ⓔmphɯrⓝʑɣɤmphɯr} 
\classe{vi} \end{sous-entrée}

\paradigme{dir}{tɤ-}
\begin{définition}\pfra{s'envelopper dans}\end{définition}
\begin{définition}\pcmn{把自己裹在…里}\end{définition}
\begin{sous-entrée}{ʑɣɤsɯmphɯr}{ⓔmphɯrⓝʑɣɤsɯmphɯr}\end{sous-entrée}

\begin{définition}\pfra{s'envelopper dans}\end{définition}
\begin{définition}\pcmn{把自己裹在…里}\end{définition}
\begin{sous-entrée}{amphɯmphɯr}{ⓔmphɯrⓝamphɯmphɯr} 
\classe{vs} 
\begin{définition}\pfra{envelopper de plusieurs couches}\end{définition}
\begin{définition}\pcmn{一层一层地裹着}\end{définition}\end{sous-entrée}

\end{entrée}

\begin{entrée}{mphɯrpa}{}{ⓔmphɯrpa} 
\classe{n} 
\begin{définition}\pfra{bâton pour frapper les chiens}\end{définition}
\begin{définition}\pcmn{打狗棒}\end{définition}\end{entrée}

\begin{entrée}{mpja}{}{ⓔmpja} 
\classe{vs} \paradigme{dir}{thɯ-}
\begin{définition}\pfra{chaud}\end{définition}
\begin{définition}\pcmn{热}\end{définition}
\begin{exemple}\pjya{tɯ-ŋga ɲɯ-mpja}\hspace{5pt}\pcmn{衣服很热}\end{exemple}
\begin{exemple}\pjya{kɯ-mpja tɤ-ndze}\hspace{5pt}\pcmn{你吃热的吧}\end{exemple}
\begin{exemple}\pjya{nɤ-ŋga ra kɯ-mpja tɤ-ŋge}\hspace{5pt}\pcmn{衣服穿暖一些}\end{exemple}\relationsémantique{参考}{\lien{ⓔɣɤmpja}{ɣɤmpja}}\end{entrée}

\begin{entrée}{mpɯ}{}{ⓔmpɯ} 
\classe{vs} \paradigme{dir}{kɤ-}\sens{1}
\begin{définition}\pfra{être mou}\end{définition}
\begin{définition}\pcmn{柔软}\end{définition}\sens{2}\paradigme{dir}{kɤ-}\paradigme{dir}{nɯ-}
\begin{définition}\pfra{être tendre}\end{définition}
\begin{définition}\pcmn{嫩}\end{définition}
\begin{exemple}\pjya{tɯ-pa ɲɯ-mpɯ}\hspace{5pt}\pcmn{坐的地方很软}\end{exemple}
\begin{exemple}\pjya{sɤtɕha ko-mpɯ}\hspace{5pt}\pcmn{地变软了}\end{exemple}
\begin{exemple}\pjya{@yangyu nɯnɯ chɯ́-wɣ-pu ɕɯŋgɯ tɕe rko ri, thɯ-smi tɕe ɲɯ-mpɯ ŋu loβ}\hspace{5pt}\pcmn{洋芋煨熟之前很硬,熟了之后就是软的}\end{exemple}
\begin{sous-entrée}{nɤmpɯ}{ⓔmpɯⓢ2ⓝnɤmpɯ} 
\classe{vt}  
\grammaire{trop} 
\begin{définition}\pfra{trouver mou}\end{définition}
\begin{définition}\pcmn{觉得软}\end{définition}\end{sous-entrée}

\begin{sous-entrée}{ɣɤmpɯ}{ⓔmpɯⓢ2ⓝɣɤmpɯ} 
\classe{vt}  
\grammaire{caus} \end{sous-entrée}

\end{entrée}

\begin{entrée}{mpɯmnu}{}{ⓔmpɯmnu} 
\classe{vs}  
\grammaire{comp} \paradigme{dir}{nɯ-}
\begin{définition}\pfra{mou}\end{définition}
\begin{définition}\pcmn{柔软}\end{définition}
\begin{exemple}\pjya{kɯki tɯ-nga ki ɲɯ-mpɯmnu tɕe, kɤ-ŋga ɲɯ-sɤscit}\hspace{5pt}\pcmn{这件衣服很软,穿起来很舒服}\end{exemple}\end{entrée}

\begin{entrée}{mqlaʁ}{}{ⓔmqlaʁ} 
\classe{vt} \paradigme{dir}{thɯ-}
\begin{définition}\pfra{avaler}\end{définition}
\begin{définition}\pcmn{吞}\end{définition}
\begin{exemple}\pjya{tɯ-mgo thɯ-mqlaʁ-a}\hspace{5pt}\pcmn{我把饭吞了}\end{exemple}
\begin{exemple}\pjya{tɯ-ci thɯ-mqlaʁ-a}\hspace{5pt}\pcmn{我咽了水}\end{exemple}
\begin{exemple}\pjya{a-mci thɯ-mqlaʁ-a}\hspace{5pt}\pcmn{我吞了口水}\end{exemple}
\begin{exemple}\pjya{smɤn thɯ-mqlaʁ-a}\hspace{5pt}\pcmn{我吞了药}\end{exemple}\end{entrée}

\begin{entrée}{mtɕhaʁnɤmtɕhaʁ}{}{ⓔmtɕhaʁnɤmtɕhaʁ} 
\classe{idph.3} 
\begin{définition}\pfra{bruit que l'on fait lorsqu'on évalue le goût d'un aliment après l'avoir avalé}\end{définition}
\begin{définition}\pcmn{细嚼慢咽的声音}\end{définition}
\begin{exemple}\pjya{kɤ-rɯndzɤtshi ɯ-ʁjiz mtɕhaʁnɤmtɕhaʁ mɯ́j-ɣi}\hspace{5pt}\pcmn{他细嚼慢咽发出声音,不想吃饭}\end{exemple}
\begin{sous-entrée}{sɤmtɕhaʁmtɕhaʁ}{ⓔmtɕhaʁnɤmtɕhaʁⓝsɤmtɕhaʁmtɕhaʁ} 
\classe{vt} 
\begin{exemple}\pjya{ɲɯ-sɤmtɕhaʁmtɕhaʁ ndzɤtshi ɯ-ʁjiz mɯ́j-ɣi}\hspace{5pt}\pcmn{他细嚼慢咽发出声音,不想吃饭}\end{exemple}\end{sous-entrée}

\end{entrée}

\begin{entrée}{mtɕhɤnmbrɯ}{}{ⓔmtɕhɤnmbrɯ} 
\classe{n} 
\begin{définition}\pfra{grains destinés aux monastères}\end{définition}
\begin{définition}\pcmn{供神的粮食}\end{définition}
\begin{exemple}\pjya{rgɯnba smi pɯ́-wɣ-βlɯ tɕe, (fsaŋ lɤ́-wɣ-ta tɕe) tɤ-khɯ tu-tɕɤt nɯ ɣɯ ɯ-ŋgɯ kú-wɣ-lɤt cho rgɯnba lú-wɣ-sɤri ŋu}\hspace{5pt}\pcmn{在寺庙里求烟仔,就把(粮食)用柏树枝熏一下供神}\end{exemple}\relationsémantique{同义词}{\lien{ⓔkɤsɤri}{kɤsɤri}}\étymologie{mtɕʰod.ⁿbru}\end{entrée}

\begin{entrée}{mtɕhɤnmi}{}{ⓔmtɕhɤnmi} 
\classe{n} 
\begin{définition}\pfra{lampe à beurre allumée}\end{définition}
\begin{définition}\pcmn{点过的酥油灯}\end{définition}\étymologie{mtɕʰod.me}\end{entrée}

\begin{entrée}{mtɕhɤtkho}{}{ⓔmtɕhɤtkho} 
\classe{n} 
\begin{définition}\pfra{endroit où l'on fait des fumigations rituelles}\end{définition}
\begin{définition}\pcmn{烧香的地方;经堂}\end{définition}\étymologie{mtɕʰod.kʰaŋ}\end{entrée}

\begin{entrée}{mtɕhi}{}{ⓔmtɕhi} 
\classe{n} 
\begin{définition}\pfra{argousier}\end{définition}
\begin{définition}\pcmn{沙棘}\end{définition}
\begin{exemple}\pjya{mtɕhi nɯ zgoku aʁɤndɯndɤt tu-ɬoʁ cha, ɯ-ru mɤ-astu, kɤ-jʁɯ-jʁu ŋu, ɯ-rtaʁ dɤn, ɯ-mdzu wuma ʑo mtɕoʁ, jpum. ɯ-ru nɯ kɯ-ɲaʁ ŋu. ɯ-rqhu jaʁ, ɯ-si nɯ kɤ-ɣɯŋgɯŋgɯ kɯ-fse ŋu, ɯ-rtaʁ ɯ-rqhu nɯ mpɕu, ɯ-si nɯ kɤ-βlɯ kɯnɤ mɤ-sna, ɯ-jwaʁ nɯ kɯ-pɣi tsa tɕe kɯ-ndɯ-ndɯβ, pɣɤmuj ɯ-tshɯɣa fse. ɯ-mɯntoʁ kɤ-mto me, ɯ-mat kɤ-tshoʁ ɕɯmɯma tɕe, ldʑaŋsɤr ŋu, thɯ-tɯt wuma ʑo qarŋe, ɯ-tɯ-tɕur saχaʁ ri mɤ-sɤndɤɣ.}\hspace{5pt}\pcmn{沙棘在高山上到处生长,树干不直,弯弯曲曲的,枝桠多,刺又锋利又粗。树干是黑色,树皮很厚。树木是由一圈一圈的年轮组成的,枝桠的树皮是光滑的。木质连烧火都不好。叶子是灰色的,很小,形状像羽毛。看不见花,果实才开始结时是浅绿色,成熟后是黄色。吃起来很酸,但没有毒性。}\end{exemple}\end{entrée}

\begin{entrée}{mtɕhinaʁ}{}{ⓔmtɕhinaʁ} 
\classe{n} 
\begin{définition}\pfra{espèce de chien dont la bouche est noire et le corps rouge}\end{définition}
\begin{définition}\pcmn{黑嘴巴的狗}\end{définition}\end{entrée}

\begin{entrée}{mtɕho}{}{ⓔmtɕho} 
\classe{vt} \paradigme{dir}{pɯ-}
\begin{définition}\pfra{fixer}\end{définition}
\begin{définition}\pcmn{固定}\end{définition}
\begin{exemple}\pjya{pjɯ-mtɕham-a}\hspace{5pt}\pcmn{我固定}\end{exemple}
\begin{exemple}\pjya{rɟɤɕi thɯ-rku-t-a tɕe, pɯ-mtɕho-t-a tɕe nɯ-sɤsɯɣ-a.}\hspace{5pt}\pcmn{我套了楦头,把它固定了并弄紧了}\end{exemple}
\begin{exemple}\pjya{qaʁ ɯ-jɯ thɯ-tshoʁ-a tɕe thɯ-mtɕho-t-a}\hspace{5pt}\pcmn{我把锄头装上了把子并把它固定了}\end{exemple}\relationsémantique{同义词}{\lien{ⓔrɤsta}{rɤsta}}\end{entrée}

\begin{entrée}{mtɕhortɯn}{}{ⓔmtɕhortɯn} 
\classe{n} 
\begin{définition}\pfra{stûpa}\end{définition}
\begin{définition}\pcmn{塔}\end{définition}\étymologie{mtɕʰod.rten}\end{entrée}

\begin{entrée}{mtɕhostɤt}{}{ⓔmtɕhostɤt} 
\classe{vt} 
\begin{définition}\pfra{louer}\end{définition}
\begin{définition}\pcmn{赞美;奉承}\end{définition}
\begin{exemple}\pjya{jisŋi skɤrma kɯ-sna tu-ta-mtɕhostɤt-nɯ ŋu tɕe}\hspace{5pt}\pcmn{在今天这个吉利的日子,我赞颂你们(山神)}\end{exemple}\end{entrée}

\begin{entrée}{mtɕhot}{}{ⓔmtɕhot} 
\classe{interj} 
\begin{définition}\pfra{mot de prière}\end{définition}
\begin{définition}\pcmn{求神仙保佑}\end{définition}
\begin{sous-entrée}{ɯ-mtɕhot}{ⓔmtɕhotⓝɯ-mtɕhot} 
\classe{np} 
\begin{définition}\pfra{morceau de pain jeté lors d'un souhait}\end{définition}
\begin{définition}\pcmn{请求神仙保佑的时候扔上去的一块馍馍}\end{définition}\end{sous-entrée}

\étymologie{mtɕʰod}\end{entrée}

\begin{entrée}{mtɕhɯβ}{}{ⓔmtɕhɯβ} 
\classe{vi} \sens{1}\paradigme{dir}{nɯ-}\paradigme{dir}{kɤ-}
\begin{définition}\pfra{mouiller, s’infiltrer}\end{définition}
\begin{définition}\pcmn{浸入;渗入}\end{définition}
\begin{exemple}\pjya{tɯ-ŋga ɯ-ŋgɯ tɯ-ci ɲɤ-ɕe tɕe ɲɤ-mtɕhɯβ}\hspace{5pt}\pcmn{水渗入到衣服里,浸湿了衣服}\end{exemple}
\begin{exemple}\pjya{a-ŋga ɲɤ-k-ɤci-ci tɕe tɯ-ci kɯ ɲɤ-mtɕhɯβ}\hspace{5pt}\pcmn{我的衣服湿了,浸泡了水}\end{exemple}
\begin{exemple}\pjya{si (ɕoŋtɕa) ɯ-ŋgɯ tɯ-ci ɲɤ-mtɕhɯβ}\hspace{5pt}\pcmn{水渗透到木头里}\end{exemple}
\begin{exemple}\pjya{stoʁ ko-mtɕhɯβ}\hspace{5pt}\pcmn{胡豆(被水)泡了}\end{exemple}\sens{2}\paradigme{dir}{kɤ-}
\begin{définition}\pfra{ajouter}\end{définition}
\begin{définition}\pcmn{添加}\end{définition}
\begin{exemple}\pjya{kɤ-ndza mɯ́j-nɤrtaʁ tɕe kɤ-mtɕhɯβ ɲɯ-ɬoʁ}\hspace{5pt}\pcmn{他觉得食物不够,要加一点}\end{exemple}
\begin{exemple}\pjya{aʑo kɤ-ndza mɯ́j-nɤrtaʁ-a tɕe ku-kɯ-mtɕhɯβ-a ɲɯ-ntshi}\hspace{5pt}\pcmn{我觉得食物不够,你要给我添一点}\end{exemple}
\begin{sous-entrée}{sɤmtɕhɯβ}{ⓔmtɕhɯβⓢ2ⓝsɤmtɕhɯβ} 
\classe{vi}  
\grammaire{apass} 
\begin{définition}\pfra{ajouter}\end{définition}
\begin{définition}\pcmn{给别人添加一点}\end{définition}\end{sous-entrée}

\end{entrée}

\begin{entrée}{mtɕhɯtsaʁ}{}{ⓔmtɕhɯtsaʁ} 
\classe{n} 
\begin{définition}\pfra{une maladie (boutons purulents sur la bouche, fièvre)}\end{définition}
\begin{définition}\pcmn{嘴唇上生疮,化脓,发烧}\end{définition}
\begin{exemple}\pjya{ɯ-mtɕhɯtsaʁ ɲɤ-ɬoʁ}\hspace{5pt}\pcmn{他嘴上生了疮}\end{exemple}\relationsémantique{参考}{\lien{ⓔnɯmtɕhɯtsaʁ}{nɯmtɕhɯtsaʁ}}\end{entrée}

\begin{entrée}{mtɕoʁ}{}{ⓔmtɕoʁ} 
\classe{vs} \paradigme{dir}{tɤ-}\paradigme{dir}{tɤ-}
\begin{définition}\pfra{aiguisé}\end{définition}
\begin{définition}\pcmn{锋利}\end{définition}
\begin{définition}\pfra{aiguiser}\end{définition}
\begin{définition}\pcmn{使锋利}\end{définition}
\begin{exemple}\pjya{mbrɯtɕɯ ɲɯ-mtɕoʁ}\hspace{5pt}\pcmn{刀很锋利}\end{exemple}
\begin{exemple}\pjya{tɯmnɯ ɲɯ-mtɕoʁ}\hspace{5pt}\pcmn{锥子很锋利}\end{exemple}
\begin{exemple}\pjya{rɟaŋsoʁ ɲɯ-mtɕoʁ}\hspace{5pt}\pcmn{锯子很锋利}\end{exemple}
\begin{exemple}\pjya{mɤ-kɯ-mtɕoʁ}\hspace{5pt}\pcmn{钝}\end{exemple}
\begin{exemple}\pjya{mbrɯtɕɯ cho-fse tɕe to-ɣɤmtɕoʁ}\hspace{5pt}\pcmn{把刀磨得很锋利了}\end{exemple}\relationsémantique{参考}{\lien{ⓔamtɕoʁ}{amtɕoʁ}}
\begin{sous-entrée}{ɣɤmtɕoʁ}{ⓔmtɕoʁⓝɣɤmtɕoʁ} 
\classe{vt}  
\grammaire{caus} \end{sous-entrée}

\end{entrée}

\begin{entrée}{mtɕɯr}{}{ⓔmtɕɯr} 
\classe{vi} \paradigme{dir}{\_}
\begin{définition}\pfra{tourner}\end{définition}
\begin{définition}\pcmn{转动}\end{définition}
\begin{exemple}\pjya{aʑo kɤ-mtɕɯr-a}\hspace{5pt}\pcmn{我转过了(转身)}\end{exemple}
\begin{exemple}\pjya{a-kɤrnoʁ ɲɯ-mtɕɯr}\hspace{5pt}\pcmn{我头晕}\end{exemple}
\begin{exemple}\pjya{mkhɯrlu ɲɯ-mtɕɯr}\hspace{5pt}\pcmn{车在转动}\end{exemple}
\begin{exemple}\pjya{zdɯm ɲɯ-mtɕɯr}\hspace{5pt}\pcmn{云在转动}\end{exemple}
\begin{exemple}\pjya{tɯ-mɯ mɤ-mtɕɯr zdɯm mtɕɯr}\hspace{5pt}\pcmn{人世间的事变化无常}\end{exemple}\relationsémantique{参考}{\lien{ⓔnɤmtɕɯrlu}{nɤmtɕɯrlu}}\relationsémantique{参考}{\lien{ⓔsɯmtɕɯr}{sɯmtɕɯr}}\end{entrée}

\begin{entrée}{mthu}{}{ⓔmthu} 
\classe{vs} \paradigme{dir}{thɯ-}\sens{1}
\begin{définition}\pfra{qui a confiance en lui}\end{définition}
\begin{définition}\pcmn{有自信(性情)}\end{définition}
\begin{exemple}\pjya{nɤ-sɯm ɯ-tɯ-mthu nɯ}\hspace{5pt}\pcmn{你自以为是}\end{exemple}
\begin{exemple}\pjya{ɯ-snoŋwa ɲɯ-mthu}\hspace{5pt}\pcmn{他很有自信,觉得自高自大}\end{exemple}\sens{2}
\begin{définition}\pfra{qui a de la graisse}\end{définition}
\begin{définition}\pcmn{膘情好;强壮}\end{définition}
\begin{exemple}\pjya{ɯ-ɲɤm ɲɯ-pe tɕe ɲɯ-mthu}\hspace{5pt}\pcmn{(牛)膘情很好,很强壮}\end{exemple}
\begin{exemple}\pjya{ɯ-ɲɤm mɯ́j-pe tɕe mɯ́j-mthu, ndʐaβ ɲɯ-ŋu}\hspace{5pt}\pcmn{牛膘情不好,不强壮,快要倒了}\end{exemple}
\begin{exemple}\pjya{nɯŋɤdo nɯ ɲɯ-mthu, mɯ́j-mthu}\hspace{5pt}\pcmn{老母牛膘情很好,膘情不好}\end{exemple}\sens{4}
\begin{définition}\pfra{trop haut (coup de fusil)}\end{définition}
\begin{définition}\pcmn{打枪瞄高}\end{définition}\relationsémantique{反义词}{\lien{ⓔʁma}{ʁma}}\relationsémantique{参考}{\lien{ⓔnɤmthu}{nɤmthu}}\étymologie{mtʰo}\end{entrée}

\begin{entrée}{mthama}{}{ⓔmthama} 
\classe{n} \sens{1}
\begin{définition}\pfra{le dernier}\end{définition}
\begin{définition}\pcmn{最后;最终}\end{définition}
\begin{exemple}\pjya{tɕetha ɯ-qhu mthama tɕe kɯ-pe ci βze}\hspace{5pt}\pcmn{最终会有好结果}\end{exemple}\sens{2}
\begin{définition}\pfra{le plus mauvais}\end{définition}
\begin{définition}\pcmn{最低级}\end{définition}
\begin{exemple}\pjya{nɤki laχtɕha nɯ ʁo stu mɤ-kɯ-pe mthama ʑo nɯ ɲɯ-ɕti}\hspace{5pt}\pcmn{这个东西是最差的}\end{exemple}\étymologie{mtʰa.ma}\end{entrée}

\begin{entrée}{mthɤrpɯ}{}{ⓔmthɤrpɯ} 
\classe{n} 
\begin{définition}\pfra{décoration en argent qui pend de la ceinture}\end{définition}
\begin{définition}\pcmn{垂吊在腰带上的银装饰品}\end{définition}\end{entrée}

\begin{entrée}{mthuri}{}{ⓔmthuri} 
\classe{n} 
\begin{définition}\pfra{feu mon père}\end{définition}
\begin{définition}\pcmn{已故的父亲}\end{définition}
\begin{exemple}\pjya{a-wa mthuri}\hspace{5pt}\pcmn{我已故的父亲}\end{exemple}\étymologie{mtʰo.ris}\end{entrée}

\begin{entrée}{mthɯ}{}{ⓔmthɯ} 
\classe{n} 
\begin{définition}\pfra{mantra}\end{définition}
\begin{définition}\pcmn{咒经}\end{définition}
\begin{exemple}\pjya{mthɯ cho-lɤt}\hspace{5pt}\pcmn{他念了咒经}\end{exemple}
\begin{exemple}\pjya{ɯ-mthɯ thɯ-lat-a}\hspace{5pt}\pcmn{我对他念了咒语}\end{exemple}\relationsémantique{参考}{\lien{ⓔnɯmthɯⓗ2}{nɯmthɯ₂}}\étymologie{mtʰu}\end{entrée}

\begin{entrée}{mthɯmɤr}{}{ⓔmthɯmɤr} 
\classe{n} 
\begin{définition}\pfra{sceau}\end{définition}
\begin{définition}\pcmn{印章}\end{définition}
\begin{exemple}\pjya{mthɯmɤr pɯ-ta-t-a}\hspace{5pt}\pcmn{我盖了个章}\end{exemple}\relationsémantique{同义词}{\lien{ⓔthotsi}{thotsi}}\end{entrée}

\begin{entrée}{mthɯrnda}{}{ⓔmthɯrnda} 
\classe{n} 
\begin{définition}\pfra{rênes}\end{définition}
\begin{définition}\pcmn{缰绳}\end{définition}\étymologie{mtʰur.mda}\end{entrée}

\begin{entrée}{mthɯt}{}{ⓔmthɯt} 
\classe{vt} \paradigme{dir}{pɯ-}\paradigme{dir}{thɯ-}
\begin{définition}\pfra{relier, rallonger}\end{définition}
\begin{définition}\pcmn{连接起来(补充一部分)}\end{définition}
\begin{exemple}\pjya{ki tɯmbri ki mɯ́j-rtaʁ tɕe kɤ-mthɯt ɲɯ-ɬoʁ}\hspace{5pt}\pcmn{这条绳子不够,需要把它连接起来}\end{exemple}
\begin{exemple}\pjya{tɤ-ri nɯ kɯ mɯ́j-ɕaβ tɕe kɤ-mthɯt-a}\hspace{5pt}\pcmn{绳子不够长,我就接了一段}\end{exemple}
\begin{exemple}\pjya{ɯ-ŋga ɯ-ndo pɯ-mthɯt-a}\hspace{5pt}\pcmn{我补了他的衣角}\end{exemple}
\begin{exemple}\pjya{tɤ-pɤloʁ ɯ-ku thɯ-mthɯt-a}\hspace{5pt}\pcmn{我补了袖子}\end{exemple}\relationsémantique{同义词}{\lien{ⓔsɤlɤɣɯ}{sɤlɤɣɯ}}\étymologie{mtʰud}\end{entrée}

\begin{entrée}{mthɯxtɕɤr}{}{ⓔmthɯxtɕɤr} 
\classe{n} 
\begin{définition}\pfra{ceinture (large, tissée à la main)}\end{définition}
\begin{définition}\pcmn{腰带}\end{définition}\relationsémantique{参考}{\lien{ⓔtɯ-mthɤɣ}{tɯ-mthɤɣ}}\relationsémantique{参考}{\lien{ⓔxtɕɤr}{xtɕɤr}}\end{entrée}

\begin{entrée}{mti/\variante{mtɯj}}{}{ⓔmti} 
\classe{n} 
\begin{définition}\pfra{turquoise}\end{définition}
\begin{définition}\pcmn{碧玉;绿松石}\end{définition}\end{entrée}

\begin{entrée}{mto}{₂}{ⓔmtoⓗ2} 
\classe{vt}
\classe{vs} \paradigme{dir}{pɯ-}\paradigme{dir}{tɤ-}
\begin{définition}\pfra{voir}\end{définition}
\begin{définition}\pcmn{看见}\end{définition}
\begin{exemple}\pjya{lɯlu kɯ aʑo ʁnɯz ʑo ka-ndo pɯ-mto-t-a}\hspace{5pt}\pcmn{我看见过猫抓了两只(小鸡)}\end{exemple}
\begin{sous-entrée}{mto}{ⓔmtoⓝmto}\end{sous-entrée}

\sens{1}
\begin{définition}\pfra{être capable de voir}\end{définition}
\begin{définition}\pcmn{看得见}\end{définition}
\begin{exemple}\pjya{ɯ-mɲaʁ ɲɯ-mto ʂɯŋʂɯŋ ʑo}\hspace{5pt}\pcmn{他眼睛看得很清楚(视力很好)}\end{exemple}
\begin{exemple}\pjya{ɯ-mɲaʁ χchoʁe ni to-mto}\hspace{5pt}\pcmn{他的双眼复明了}\end{exemple}\sens{2}\paradigme{dir}{pɯ-}\paradigme{dir}{pɯ-}\paradigme{dir}{pɯ-}\paradigme{dir}{tɤ-}
\begin{définition}\pfra{être fiable (prédiction)}\end{définition}
\begin{définition}\pcmn{灵(算卦)}\end{définition}
\begin{définition}\pfra{montrer, laisser voir}\end{définition}
\begin{définition}\pcmn{让人看见;给人看}\end{définition}
\begin{définition}\pfra{se voir}\end{définition}
\begin{définition}\pcmn{看到自己}\end{définition}
\begin{définition}\pfra{se faire voir}\end{définition}
\begin{définition}\pcmn{让别人看到自己}\end{définition}
\begin{définition}\pfra{rendre la vue (à un aveugle)}\end{définition}
\begin{définition}\pcmn{令(盲人)复明}\end{définition}
\begin{exemple}\pjya{ɯ-mphrɯmɯ wuma ʑo mto}\hspace{5pt}\pcmn{他算的卦很灵}\end{exemple}
\begin{exemple}\pjya{tɯ-ci ɯ-ŋgɯ ɕ-pɯ-ru-a ri, pjɯ-ntɕhar-a ɲɯ-ŋu tɕe pɯ-ʑɣɤmto-a}\hspace{5pt}\pcmn{我往水里看了一下,里面有我的倒影,我看到我自己了}\end{exemple}
\begin{exemple}\pjya{ma-pɯ-tɯ-ʑɣɤsɯmto}\hspace{5pt}\pcmn{你不要让人看见你}\end{exemple}
\begin{exemple}\pjya{ɯʑo mɯ-to-rɯndzaŋspa tɕe pjɤ-ʑɣɤsɯmto}\hspace{5pt}\pcmn{他不小心让人看见了}\end{exemple}\relationsémantique{参考}{\lien{ⓔamɯmto}{amɯmto}}\relationsémantique{参考}{\lien{ⓔsɤmto}{sɤmto}}
\begin{sous-entrée}{sɯmto}{ⓔmtoⓗ2ⓢ2ⓝsɯmto} 
\classe{vt}  
\grammaire{caus} \end{sous-entrée}

\begin{sous-entrée}{ʑɣɤmto}{ⓔmtoⓗ2ⓢ2ⓝʑɣɤmto} 
\classe{vi}  
\grammaire{refl} \end{sous-entrée}

\begin{sous-entrée}{ʑɣɤsɯmto}{ⓔmtoⓗ2ⓢ2ⓝʑɣɤsɯmto} 
\classe{vi}  
\grammaire{refl}
\grammaire{caus} \end{sous-entrée}

\begin{sous-entrée}{amto}{ⓔmtoⓗ2ⓢ2ⓝamto} 
\classe{vi} 
\begin{définition}\pfra{être visible}\end{définition}
\begin{définition}\pcmn{看得见}\end{définition}
\begin{exemple}\pjya{khɯɣɲɟɯ ju-kɯ-ru tɕe, qhaqhu nɯra tɯrme nɯra amto}\hspace{5pt}\pcmn{往窗子外面看的话,看得见房子后面的人}\end{exemple}\end{sous-entrée}

\begin{sous-entrée}{ɣɤmto}{ⓔmtoⓗ2ⓢ2ⓝɣɤmto} 
\classe{vt} \end{sous-entrée}

\end{entrée}

\begin{entrée}{mtsaʁ}{}{ⓔmtsaʁ} 
\classe{vi} \paradigme{dir}{\_}
\begin{définition}\pfra{sauter}\end{définition}
\begin{définition}\pcmn{跳}\end{définition}
\begin{exemple}\pjya{@lanqiu mɯ́j-mtsaʁ, ɯ-@qi mɯ́j-rtaʁ}\hspace{5pt}\pcmn{篮球跳不了,气不够}\end{exemple}
\begin{exemple}\pjya{aʑo pjɯ-kɯ-ɣɤrat-a-nɯ mɤ-ra ma aʑo pjɯ-nɯ-mtsaʁ-a jɤɣ}\hspace{5pt}\pcmn{你们不需要把我扔下去,我自己跳}\end{exemple}
\begin{exemple}\pjya{ɯ-rtsa ɲɯ-mtsaʁ, ɯ-sni ɲɯ-mtsaʁ}\hspace{5pt}\pcmn{他的脉搏在跳动}\end{exemple}\relationsémantique{同义词}{\lien{ⓔnɯmdar}{nɯmdar}}\relationsémantique{参考}{\lien{ⓔnɯmbrɯmtsaʁ}{nɯmbrɯmtsaʁ}}\end{entrée}

\begin{entrée}{mtshalu}{}{ⓔmtshalu} 
\classe{n} 
\begin{définition}\pfra{ortie}\end{définition}
\begin{définition}\pcmn{荨麻【和麻】}\end{définition}
\begin{exemple}\pjya{mtshalu kɯ kɤ́-wɣ-mtsɯɣ-a}\hspace{5pt}\pcmn{被荨麻蛰到了}\end{exemple}
\begin{exemple}\pjya{mtshalu nɯ sɯjno ci ŋu, kha ɯ-rkɯ kɯ-ɤrmbat tsa zndɤrchɤβ ɯ-ŋgɯ nɯ ra dɤn, mtshalu ɯ-ru ɯ-taʁ ɯ-jwaʁ ɯ-taʁ nɯ ra ɯ-rme dɤn tɕeri ɯ-rme sɤmtsɯɣ, ɯ-rme tɯ-ɕa ɲɯ-ɤtɯɣ tɕe ɕɯmŋɤm tɤ-ndɤr ʑo tu-tɕɤt ɕti, ɯ-mat tu-βze ɕɯŋgɯ kɤ-ndza sna. tɯ-xpa tɯ-xpa tu-ɬoʁ ŋu.}\hspace{5pt}\pcmn{和麻是一种植物,生长在房子周边的墙缝里,和麻的茎和叶子上都长有细毛但这些细毛会蛰人,细毛碰到皮肤时,导致疼痛,皮肤上生痘痘,结果之前可以吃,每年都会发芽。}\end{exemple}\relationsémantique{参考}{\lien{ⓔmtshalɤɲaʁ}{mtshalɤɲaʁ}}\relationsémantique{参考}{\lien{ⓔmtshalɤɣrum}{mtshalɤɣrum}}\relationsémantique{参考}{\lien{ⓔqarmamtshalu}{qarmamtshalu}}\relationsémantique{参考}{\lien{ⓔnɯmtshalu}{nɯmtshalu}}\end{entrée}

\begin{entrée}{mtshalɤɣrum}{}{ⓔmtshalɤɣrum} 
\classe{n} 
\begin{définition}\pfra{espèce d'ortie}\end{définition}
\begin{définition}\pcmn{荨麻的一种}\end{définition}\end{entrée}

\begin{entrée}{mtshalɤɲaʁ}{}{ⓔmtshalɤɲaʁ} 
\classe{n} 
\begin{définition}\pfra{ortie}\end{définition}
\begin{définition}\pcmn{荨麻的一种}\end{définition}\end{entrée}

\begin{entrée}{mtshɤm}{}{ⓔmtshɤm} 
\classe{vt} \paradigme{dir}{pɯ-}\sens{1}
\begin{définition}\pfra{entendre}\end{définition}
\begin{définition}\pcmn{听见}\end{définition}
\begin{exemple}\pjya{ɯ-pɯ́-tɯ-mtshɤm}\hspace{5pt}\pcmn{你听见了吗?}\end{exemple}\sens{2}
\begin{définition}\pfra{sentir}\end{définition}
\begin{définition}\pcmn{闻到}\end{définition}
\begin{exemple}\pjya{tɤ-khɯ ɯ-di pjɯ-tɯ-mtshɤm tɕe tú-wɣ-sɤɕqhe-a ŋu}\hspace{5pt}\pcmn{我一闻到烟味都会咳嗽}\end{exemple}
\begin{exemple}\pjya{ɯ-di ɲɯ-mtsham-a}\hspace{5pt}\pcmn{我闻到它的气味了}\end{exemple}\sens{3}
\begin{définition}\pfra{ressentir}\end{définition}
\begin{définition}\pcmn{感觉到}\end{définition}
\begin{exemple}\pjya{tɤ-mpja ɲɯ-mtsham-a}\hspace{5pt}\pcmn{我感觉到热}\end{exemple}
\begin{exemple}\pjya{tɤ-ŋɤm ɲɯ-mtsham-a}\hspace{5pt}\pcmn{我感觉到痛}\end{exemple}
\begin{exemple}\pjya{tɤkhe tshɤfkri mɤ-mtshɤm}\hspace{5pt}\pcmn{傻子感觉不到汤里加了很多盐}\end{exemple}
\begin{exemple}\pjya{tɤkhe tshɤdɯɣ mɤ-mtshɤm}\hspace{5pt}\pcmn{傻子感觉不到热}\end{exemple}\relationsémantique{参考}{\lien{ⓔsɤmtshɤm}{sɤmtshɤm}}\relationsémantique{参考}{\lien{ⓔamɯmtshɤm}{amɯmtshɤm}}\end{entrée}

\begin{entrée}{mtshɤmdi}{}{ⓔmtshɤmdi} 
\classe{n}  
\grammaire{n.lieu} 
\begin{définition}\pfra{nom commun à plusieurs champs à Kamnyu}\end{définition}
\begin{définition}\pcmn{干木鸟村几块田地的统称}\end{définition}\end{entrée}

\begin{entrée}{mtshɤmŋu}{}{ⓔmtshɤmŋu} 
\classe{n} 
\begin{définition}\pfra{plage}\end{définition}
\begin{définition}\pcmn{海边}\end{définition}\relationsémantique{参考}{\lien{}{mtshu}}\end{entrée}

\begin{entrée}{mtshɤri}{}{ⓔmtshɤri} 
\classe{adv} 
\begin{définition}\pfra{étrange}\end{définition}
\begin{définition}\pcmn{奇怪}\end{définition}
\begin{exemple}\pjya{mtshɤri ɲɯ-sɯsam-a ma a-zda ra mɯ́j-nɤndʐo-nɯ ri aʑo ɲɯ-nɤndʐo-a}\hspace{5pt}\pcmn{我觉得奇怪,其他人不冷,我却觉得很冷}\end{exemple}
\begin{exemple}\pjya{nɯ tɕhi pɯ-ŋu kɯma mɯ́j-sɯʁjit-a tɕe mtshɤri ɲɯ-sɯsam-a}\hspace{5pt}\pcmn{很奇怪,我根本就想不起到底是什么回事}\end{exemple}\relationsémantique{参考}{\lien{ⓔsɤmtshɤr}{sɤmtshɤr}}\étymologie{mtshar}\end{entrée}

\begin{entrée}{mtshɤt}{}{ⓔmtshɤt} 
\classe{vs} \paradigme{dir}{tɤ-}
\begin{définition}\pfra{rempli}\end{définition}
\begin{définition}\pcmn{满(再也不能装了)}\end{définition}
\begin{exemple}\pjya{tɯ-ci to-mtshɤt}\hspace{5pt}\pcmn{满了水}\end{exemple}
\begin{exemple}\pjya{lʁa to-mtshɤt}\hspace{5pt}\pcmn{口袋满了}\end{exemple}
\begin{exemple}\pjya{khɯtsa to-mtshɤt}\hspace{5pt}\pcmn{碗满了}\end{exemple}
\begin{exemple}\pjya{zɯm to-mtshɤt}\hspace{5pt}\pcmn{桶满了}\end{exemple}\relationsémantique{同义词}{\lien{ⓔfkaⓗ1}{fka₁}}\relationsémantique{参考}{\lien{ⓔsɯmtshɤt}{sɯmtshɤt}}\end{entrée}

\begin{entrée}{mtshi}{}{ⓔmtshi} 
\classe{vt} \paradigme{dir}{tɤ-}\paradigme{dir}{\_}\paradigme{dir}{tɤ-}
\begin{définition}\pfra{conduire}\end{définition}
\begin{définition}\pcmn{带路;牵}\end{définition}
\begin{définition}\pfra{mener la marche}\end{définition}
\begin{définition}\pcmn{领头}\end{définition}
\begin{exemple}\pjya{jla tɤ-mtshi-t-a}\hspace{5pt}\pcmn{我牵了犏牛}\end{exemple}
\begin{exemple}\pjya{mbro tɤ-mtshi-t-a}\hspace{5pt}\pcmn{我牵了马}\end{exemple}
\begin{exemple}\pjya{tɕhaʁla zɯ tɯmbri ci kɤ-mtshi-t-a (kɤ-lat-a)}\hspace{5pt}\pcmn{我在院子里拉了一根绳子(晒衣服)}\end{exemple}
\begin{exemple}\pjya{aʑo ju-sɤmtshi-a jɤɣ}\hspace{5pt}\pcmn{我可以带路}\end{exemple}
\begin{sous-entrée}{sɤmtshi}{ⓔmtshiⓝsɤmtshi} 
\classe{vi} \end{sous-entrée}

\end{entrée}

\begin{entrée}{mtshukha}{}{ⓔmtshukha} 
\classe{n} 
\begin{définition}\pfra{bord du lac}\end{définition}
\begin{définition}\pcmn{海岸;湖边}\end{définition}\end{entrée}

\begin{entrée}{mtshoŋ}{}{ⓔmtshoŋ} 
\classe{vs} \paradigme{dir}{tɤ-}
\begin{définition}\pfra{être complet}\end{définition}
\begin{définition}\pcmn{齐备;齐全}\end{définition}\paradigme{dir}{tɤ-}
\begin{définition}\pfra{préparer complètement}\end{définition}
\begin{définition}\pcmn{准备齐全}\end{définition}
\begin{exemple}\pjya{laχtɕha ɯ-kɯ-χtɯ jɤ-ari-a ri, kɯ-mtshoŋ ʑo ɣɤʑu}\hspace{5pt}\pcmn{我去买东西,需要的东西全部都有}\end{exemple}
\begin{exemple}\pjya{@xinhua @shudian ɯ-ŋgɯ zɯ laχtɕha kɤ-χtɯ kɯ-mtshoŋ ʑo ɣɤʑu}\hspace{5pt}\pcmn{在新华书店里,需要的东西全部都有}\end{exemple}
\begin{exemple}\pjya{ɯ-ftɕaka ɲɯ-mtshoŋ}\hspace{5pt}\pcmn{都准备齐全}\end{exemple}
\begin{exemple}\pjya{a-ftɕaka tɤ-mtshoŋ}\hspace{5pt}\pcmn{我什么都准备齐全了}\end{exemple}
\begin{exemple}\pjya{ɯʑo ɯ-tɕha ɲɯ-mtshoŋ, mɤ-kɯ-rtaʁ maŋe}\hspace{5pt}\pcmn{他条件齐全,没有什么欠缺的}\end{exemple}
\begin{exemple}\pjya{a-laχtɕha tɤ-sɯmtshoŋ-a}\hspace{5pt}\pcmn{我把需要的东西都准备好了}\end{exemple}
\begin{sous-entrée}{sɯmtshoŋ}{ⓔmtshoŋⓝsɯmtshoŋ} 
\classe{vt} \end{sous-entrée}

\étymologie{mtsʰuŋs}\end{entrée}

\begin{entrée}{mtshoʁlaŋ}{}{ⓔmtshoʁlaŋ} 
\classe{n} 
\begin{définition}\pfra{animal mythique vivant dans la mer (hippopotame)}\end{définition}
\begin{définition}\pcmn{海象(河马)}\end{définition}\étymologie{mtsʰo.glaŋ}\end{entrée}

\begin{entrée}{mtshoʁzaŋ}{}{ⓔmtshoʁzaŋ} 
\classe{n} 
\begin{définition}\pfra{la plus grosses des casseroles en cuivre}\end{définition}
\begin{définition}\pcmn{最大的红铜锅子}\end{définition}
\begin{exemple}\pjya{mtshoʁzaŋ nɯ tɯthɯ ɯ-ŋgɯz stu ʑo kɯ-wxti ŋu, tɯ-ci kɯɕnɯz zɯm tɕhɯt tu-kɯ-ti pjɤ-ŋu tɕe, tɯ-zɯm nɯ kɯtʂɤsqi kɯɕnɤsqi tɯ-rpa kɯ-tɕhɯt tu. ɯ-spa nɯ zaŋ ŋu, ɯ-mŋu kɯ-jɯ-jom ŋu, ɯ-mŋu tɕhi kɯ-jom nɯ ɯ-phoŋbu kɯnɤ jpum, ɯ-mthɤɣ ri ɯ-xtɕɤr kɯ-fse ci ku-ɕe ŋu tɕe nɯ ɯ-mthɯxtɕɤr tu-ti-nɯ ŋgrɤl. ɯ-mthɯxtɕɤr cho ɯ-mŋu ɯ-pɤrthɤβ nɯ ra thɯci tsuku ʑo ɯ-χpi ku-oʑɯrja ŋu, qaɟy ɯ-χpi, pɣa ɯ-χpi, lɤntsa, ʁjaŋtʂoŋ, rɟanaʁ tɕaʁri, kɯɕnom nɯ ra tu. mtshoʁzaŋ nɯ kɯɕɯŋgɯ rgɯnba kɯ-wxti tɯrme kɯ-dɤn ɣɯ nɯ-tʂha sɤ-ta, tɯ-tshi ɯ-sɤ-βzu nɯ ra pjɤ-ŋu.}\hspace{5pt}\pcmn{\lien{ⓔmtshoʁzaŋ}{mtshoʁzaŋ}是锅里面最大的一种,据说能装七桶水,每一桶有六七十斤重。是用红铜铸成的。口很宽,口有多宽锅身就有多粗,在锅身中间箍着一根薄铜条,人们说是锅子的腰带。腰带和口之间排列着各种各样的图案,有鱼形花纹、鸟形花纹以及各种佛教图纹。过去,在大寺庙和人多的时候用来熬茶,煮粥。}\end{exemple}\étymologie{mtsʰog.zaŋs}\end{entrée}

\begin{entrée}{mtshɯβ}{}{ⓔmtshɯβ} 
\classe{vi} \paradigme{dir}{nɯ-}\paradigme{dir}{nɯ-}
\begin{définition}\pfra{se noyer}\end{définition}
\begin{définition}\pcmn{溺死}\end{définition}
\begin{définition}\pfra{noyer}\end{définition}
\begin{définition}\pcmn{令…溺死}\end{définition}
\begin{exemple}\pjya{ɲɤ-mtshɯβ}\hspace{5pt}\pcmn{他溺死了}\end{exemple}
\begin{exemple}\pjya{tɯ-ci kɯ ɲɤ́-wɣ-sɯmtshɯβ tɕe pjɤ-si}\hspace{5pt}\pcmn{他被水溺死了}\end{exemple}
\begin{sous-entrée}{sɯmtshɯβ}{ⓔmtshɯβⓝsɯmtshɯβ} 
\classe{vt} \end{sous-entrée}

\end{entrée}

\begin{entrée}{mtshɯntɕha}{}{ⓔmtshɯntɕha} 
\classe{n} 
\begin{définition}\pfra{arme}\end{définition}
\begin{définition}\pcmn{武器}\end{définition}\étymologie{mtsʰon.tɕʰa}\end{entrée}

\begin{entrée}{mtshɯzwɤr}{}{ⓔmtshɯzwɤr} 
\classe{n} 
\begin{définition}\pfra{lac}\end{définition}
\begin{définition}\pcmn{湖}\end{définition}\end{entrée}

\begin{entrée}{mtsɯɣ}{}{ⓔmtsɯɣ} 
\classe{vt} \paradigme{dir}{kɤ-}\paradigme{dir}{tɤ-}
\begin{définition}\pfra{mordre}\end{définition}
\begin{définition}\pcmn{咬}\end{définition}
\begin{définition}\pfra{mordre les gens}\end{définition}
\begin{définition}\pcmn{咬人;蜇人}\end{définition}
\begin{exemple}\pjya{khɯna kɯ ndʐuwa ka-mtsɯɣ}\hspace{5pt}\pcmn{狗咬了客人}\end{exemple}
\begin{exemple}\pjya{lɯlu kɯ βʑɯ ka-mtsɯɣ}\hspace{5pt}\pcmn{猫咬了老鼠}\end{exemple}
\begin{exemple}\pjya{ɣʑo kɯ kɤ́-wɣ-mtsɯɣ-a}\hspace{5pt}\pcmn{蜜蜂蛰了我}\end{exemple}
\begin{exemple}\pjya{qapri ɲɯ-sɤmtsɯɣ}\hspace{5pt}\pcmn{蛇咬人}\end{exemple}
\begin{exemple}\pjya{βɣɤrtshi ɲɯ-sɤmtsɯɣ}\hspace{5pt}\pcmn{蚊子咬人}\end{exemple}
\begin{exemple}\pjya{khɯna ɲɯ-sɤmtsɯɣ}\hspace{5pt}\pcmn{狗咬人}\end{exemple}
\begin{sous-entrée}{sɤmtsɯɣ}{ⓔmtsɯɣⓝsɤmtsɯɣ} 
\classe{vs}  
\grammaire{apass} \end{sous-entrée}

\end{entrée}

\begin{entrée}{mtsɯr}{}{ⓔmtsɯr} 
\classe{vi}
\classe{vs}  
\grammaire{facil} \sens{1}\paradigme{dir}{nɯ-}
\begin{définition}\pfra{avoir faim}\end{définition}
\begin{définition}\pcmn{饿}\end{définition}
\begin{exemple}\pjya{saχsɯ tɤ-ndze ma tɯrmɯ tɯ-mtsɯr}\hspace{5pt}\pcmn{你吃中午饭,不然下午就会饿}\end{exemple}
\begin{exemple}\pjya{tɤ-nɯsaχsɯ ma tɤ-pɤri ɕɯŋgɯ tɯ-mtsɯr}\hspace{5pt}\pcmn{你吃中午饭,不然在晚餐之前就会饿)}\end{exemple}
\begin{exemple}\pjya{aʑo ɲɯ-mtsɯr-a}\hspace{5pt}\pcmn{我很饿}\end{exemple}
\begin{exemple}\pjya{ɯʑo ɲɯ-mtsɯr}\hspace{5pt}\pcmn{他很饿}\end{exemple}
\begin{exemple}\pjya{nɤʑo ɲɯ-tɯ-mtsɯr ɯ-ŋu}\hspace{5pt}\pcmn{你饿不饿}\end{exemple}
\begin{exemple}\pjya{kɤ-mtsɯr ɲɯ-sɤɣdɯɣ}\hspace{5pt}\pcmn{挨饿很辛苦}\end{exemple}\sens{2}\paradigme{dir}{thɯ-}
\begin{définition}\pfra{avoir très faim}\end{définition}
\begin{définition}\pcmn{饿得厉害}\end{définition}\relationsémantique{参考}{\lien{ⓔfsɯr}{fsɯr}}\relationsémantique{参考}{\lien{ⓔsɤmtsɯr}{sɤmtsɯr}}
\begin{sous-entrée}{ɣɤmtsɯr}{ⓔmtsɯrⓢ2ⓝɣɤmtsɯr}\end{sous-entrée}

\paradigme{dir}{nɯ-}
\begin{définition}\pfra{avoir faim facilement}\end{définition}
\begin{définition}\pcmn{容易饿}\end{définition}
\begin{définition}\pfra{avoir faim}\end{définition}
\begin{définition}\pcmn{令……挨饿}\end{définition}
\begin{exemple}\pjya{nɯ-ta-sɯmtsɯr}\hspace{5pt}\pcmn{我让你饿了(没有及时给你饭吃)}\end{exemple}
\begin{sous-entrée}{sɯmtsɯr}{ⓔmtsɯrⓝsɯmtsɯr} 
\classe{vt} \end{sous-entrée}

\end{entrée}

\begin{entrée}{mtsɯrɕpaʁ}{}{ⓔmtsɯrɕpaʁ} 
\classe{vi}  
\grammaire{comp} \paradigme{dir}{nɯ-}
\begin{définition}\pfra{avoir soif et faim}\end{définition}
\begin{définition}\pcmn{又饿又渴}\end{définition}
\begin{exemple}\pjya{ɯʑo ʁo mtsɯrɕpaʁ ntsɯ ɕti kɯ}\hspace{5pt}\pcmn{他一直又饿又渴的样子}\end{exemple}\end{entrée}

\begin{entrée}{mtʂɤkhoz}{}{ⓔmtʂɤkhoz} 
\classe{n} 
\begin{définition}\pfra{bavette}\end{définition}
\begin{définition}\pcmn{口水巾}\end{définition}\end{entrée}

\begin{entrée}{mɯβʑi}{}{ⓔmɯβʑi} 
\classe{n} 
\begin{définition}\pfra{nakṣatra hasta}\end{définition}
\begin{définition}\pcmn{轸宿}\end{définition}\étymologie{me.bʑi}\end{entrée}

\begin{entrée}{mɯɕtaʁ}{}{ⓔmɯɕtaʁ} 
\classe{vs} \paradigme{dir}{thɯ-}
\begin{définition}\pfra{froid}\end{définition}
\begin{définition}\pcmn{冷}\end{définition}
\begin{exemple}\pjya{tɤjpa ɲɯ-mɯɕtaʁ}\hspace{5pt}\pcmn{雪很冷}\end{exemple}
\begin{exemple}\pjya{qale ɲɯ-mɯɕtaʁ}\hspace{5pt}\pcmn{风很冷}\end{exemple}
\begin{exemple}\pjya{rɯŋgu ɲɯ-mɯɕtaʁ}\hspace{5pt}\pcmn{牧场很冷}\end{exemple}
\begin{exemple}\pjya{ɯ-ku ra pjɤ-mɯɕtaʁ ʑo}\hspace{5pt}\pcmn{他被吓到了}\end{exemple}
\begin{sous-entrée}{zmɯɕtaʁ}{ⓔmɯɕtaʁⓝzmɯɕtaʁ} 
\classe{vt} 
\begin{définition}\pfra{rendre froid}\end{définition}
\begin{définition}\pcmn{使变冷}\end{définition}\end{sous-entrée}

\end{entrée}

\begin{entrée}{mɯɣ}{}{ⓔmɯɣ}\relationsémantique{参考}{\lien{ⓔɕmɯɣ}{ɕmɯɣ}}\end{entrée}

\begin{entrée}{mɯɣmɯɣ}{}{ⓔmɯɣmɯɣ} 
\classe{idph.2} 
\begin{définition}\pfra{souriant}\end{définition}
\begin{définition}\pcmn{形容笑嘻嘻的模样}\end{définition}
\begin{exemple}\pjya{mɯɣmɯɣ ɲɯ-nɤre}\hspace{5pt}\pcmn{他笑嘻嘻的}\end{exemple}\end{entrée}

\begin{entrée}{mɯjphɤt}{}{ⓔmɯjphɤt} 
\classe{vi} \paradigme{dir}{lɤ-}\paradigme{dir}{pɯ-}
\begin{définition}\pfra{vomir}\end{définition}
\begin{définition}\pcmn{呕吐}\end{définition}
\begin{exemple}\pjya{@yunche ɲo-βzu tɕe lo-mɯjphɤt}\hspace{5pt}\pcmn{他晕车就吐了}\end{exemple}\relationsémantique{同义词}{\lien{ⓔqioʁ}{qioʁ}}\end{entrée}

\begin{entrée}{mɯjrɯ}{}{ⓔmɯjrɯ} 
\classe{vi} \paradigme{dir}{nɯ-}
\begin{définition}\pfra{bien élevé}\end{définition}
\begin{définition}\pcmn{受到良好的教育,孝顺}\end{définition}
\begin{définition}\pfra{bien élever}\end{définition}
\begin{définition}\pcmn{教育好}\end{définition}
\begin{exemple}\pjya{ɯ-rɟit ra ɲɯ-mɯjrɯ-nɯ}\hspace{5pt}\pcmn{他的孩子接受过很好的教育}\end{exemple}
\begin{exemple}\pjya{tɤ-pɤtso ɲɯ-tɯ-zmɯjri}\hspace{5pt}\pcmn{你把孩子教育得很好}\end{exemple}
\begin{sous-entrée}{zmɯjrɯ}{ⓔmɯjrɯⓝzmɯjrɯ} 
\classe{vt} \end{sous-entrée}

\end{entrée}

\begin{entrée}{mɯm}{}{ⓔmɯm} 
\classe{vs} \paradigme{dir}{tɤ-}\paradigme{dir}{tɤ-}\paradigme{dir}{tɤ-}
\begin{définition}\pfra{bon (goût)}\end{définition}
\begin{définition}\pcmn{香(味道)、好吃}\end{définition}
\begin{définition}\pfra{rendre bon à manger}\end{définition}
\begin{définition}\pcmn{使好吃}\end{définition}
\begin{définition}\pfra{trouver bon à manger}\end{définition}
\begin{définition}\pcmn{觉得好吃}\end{définition}
\begin{exemple}\pjya{tɤ-mthɯm ɲɯ-mɯm}\hspace{5pt}\pcmn{肉好吃}\end{exemple}
\begin{exemple}\pjya{kɤ-ndzɤtshi ɲɯ-mɯm}\hspace{5pt}\pcmn{食物好吃}\end{exemple}
\begin{exemple}\pjya{tɯmgo zmɤrɤβ to-ɣɤmɯm}\hspace{5pt}\pcmn{他把菜弄得很好吃}\end{exemple}
\begin{exemple}\pjya{to-nɤmɯm}\hspace{5pt}\pcmn{他觉得好吃了(原来觉得不好吃)}\end{exemple}
\begin{sous-entrée}{ɣɤmɯm}{ⓔmɯmⓝɣɤmɯm} 
\classe{vt}  
\grammaire{caus} \end{sous-entrée}

\begin{sous-entrée}{nɤmɯm}{ⓔmɯmⓝnɤmɯm} 
\classe{vt}  
\grammaire{trop} \end{sous-entrée}

\end{entrée}

\begin{entrée}{mɯma}{}{ⓔmɯma} 
\classe{postp} 
\begin{définition}\pfra{à part}\end{définition}
\begin{définition}\pcmn{除了}\end{définition}
\begin{exemple}\pjya{kɤ-ndza mɯma a-jtɯ}\hspace{5pt}\pcmn{除了食物,什么都能积攒}\end{exemple}\end{entrée}

\begin{entrée}{mɯmta}{}{ⓔmɯmta} 
\classe{vs} \paradigme{dir}{thɯ-}
\begin{définition}\pfra{parler dans son sommeil, noctambule}\end{définition}
\begin{définition}\pcmn{说梦话,患梦游症}\end{définition}
\begin{exemple}\pjya{kɯ-mɯmta ci ɲɯ-ŋu}\hspace{5pt}\pcmn{他是一个说梦话的人}\end{exemple}
\begin{exemple}\pjya{cho-mɯmta}\hspace{5pt}\pcmn{他说梦话了}\end{exemple}
\begin{exemple}\pjya{ɲɯ-tɯ-mɯmta}\hspace{5pt}\pcmn{你在说梦话}\end{exemple}\end{entrée}

\begin{entrée}{mɯmtsrɯɣ}{}{ⓔmɯmtsrɯɣ} 
\classe{vt} \paradigme{dir}{tɤ-}\paradigme{dir}{pɯ-}
\begin{définition}\pfra{aspirer à la paille}\end{définition}
\begin{définition}\pcmn{吸吮(用吸管)}\end{définition}
\begin{exemple}\pjya{ɯʑo kɯ chɤmdɤru pa-mɯmtsrɯɣ}\hspace{5pt}\pcmn{他吸了杂酒}\end{exemple}\end{entrée}

\begin{entrée}{mɯndʐamɯχtɕɯɣ}{}{ⓔmɯndʐamɯχtɕɯɣ} 
\classe{n} 
\begin{définition}\pfra{toutes sortes}\end{définition}
\begin{définition}\pcmn{各种各样}\end{définition}\étymologie{mi.ⁿdra.mi.gtɕig}\end{entrée}

\begin{entrée}{mɯnmu}{}{ⓔmɯnmu} 
\classe{vi} \paradigme{dir}{tɤ-}\paradigme{dir}{nɯ-}\paradigme{dir}{tɤ-}
\begin{définition}\pfra{bouger}\end{définition}
\begin{définition}\pcmn{动}\end{définition}
\begin{définition}\pfra{faire bouger}\end{définition}
\begin{définition}\pcmn{使移动}\end{définition}
\begin{exemple}\pjya{ɯʑo ɲɯ-mɯnmu}\hspace{5pt}\pcmn{他在动}\end{exemple}
\begin{exemple}\pjya{fsapaʁ ɲɯ-mɯnmu}\hspace{5pt}\pcmn{牲畜在动}\end{exemple}
\begin{exemple}\pjya{rŋgɯ tɤ-zmɯnmu-t-a}\hspace{5pt}\pcmn{我把大石包移动了}\end{exemple}
\begin{exemple}\pjya{ɯ-jaʁ kɤ-zmɯnmu ɯ-ɲɯ́-khɯ?}\hspace{5pt}\pcmn{他能不能移动他的手?(受了伤的人)}\end{exemple}\relationsémantique{参考}{\lien{ⓔnmu}{nmu}}
\begin{sous-entrée}{zmɯnmu}{ⓔmɯnmuⓝzmɯnmu} 
\classe{vt}  
\grammaire{caus} \end{sous-entrée}

\end{entrée}

\begin{entrée}{mɯntoʁ}{}{ⓔmɯntoʁ} 
\classe{n} 
\begin{définition}\pfra{fleur}\end{définition}
\begin{définition}\pcmn{花}\end{définition}
\begin{exemple}\pjya{mɯntoʁ ɯ-spa}\hspace{5pt}\pcmn{蓓蕾}\end{exemple}
\begin{exemple}\pjya{stɤmku mɯntoʁ ɲɤ-lɤt}\hspace{5pt}\pcmn{平原上的花开了}\end{exemple}
\begin{exemple}\pjya{thaχtsa ɯ-χcɤl nɯ ɯ-mɯntoʁ rmi}\hspace{5pt}\pcmn{花带中间的(图像)叫\lien{ⓔmɯntoʁ}{mɯntoʁ}}\end{exemple}\relationsémantique{参考}{\lien{ⓔrɯmɯntoʁ}{rɯmɯntoʁ}}\relationsémantique{参考}{\lien{ⓔarɯmɯntoʁ}{arɯmɯntoʁ}}\étymologie{me.tog}\end{entrée}

\begin{entrée}{mɯntoʁ sɤrtɕɯn}{}{ⓔmɯntoʁ sɤrtɕɯn} 
\classe{n} 
\begin{définition}\pfra{fleur jaune}\end{définition}
\begin{définition}\pcmn{金色的黄花}\end{définition}\étymologie{me.tog.gser.tɕan}\end{entrée}

\begin{entrée}{mɯŋi}{}{ⓔmɯŋi} 
\classe{n}  
\grammaire{n.lieu} 
\begin{définition}\pfra{Mangi (village de Gdongbrgyad)}\end{définition}
\begin{définition}\pcmn{蒙岩村}\end{définition}\end{entrée}

\begin{entrée}{mɯɴɢɯ}{}{ⓔmɯɴɢɯ} 
\classe{n} 
\begin{définition}\pfra{Ligularia fischeria}\end{définition}
\begin{définition}\pcmn{山紫菀}\end{définition}
\begin{exemple}\pjya{mɯɴɢɯ nɯ sɯjno ci ŋu. tɯ-ci ɯ-rkɯ tu-ɬoʁ rga. ɯ-jwaʁ ɯ-ru tu, ɯ-jwaʁ kɯ-ɤrtɯ-rtɯm kɯ-wxtɯ-wxti ŋu, ɯ-jwaʁ ɯ-qhuʁɤri kɯ-mpɕɯ-mpɕu ŋu, ɯ-χcɤl ɯ-spjɯŋ tu-ɬoʁ tɕe, nɯ ɯ-kɤχcɤl ri ɲɯ-rɯmɯntoʁ. ɯ-mɯntoʁ kɯ-qarŋɯ-rŋe kɯ-mpɕɯ-mpɕɤr ŋu, wuma ʑo nɤmbju, dɤn. ɯ-ru jpum. ɯ-ŋgɯ kɯ-so ŋu, tɤ-jko ɯ-cu kú-wɣ-nɯ-lɤt sna. fsapaʁ ndza pe. tɯ-ci kɯ-me ra tɕe tu-ɬoʁ mɤ-cha.}\hspace{5pt}\pcmn{山紫菀是一种植物,一般生长在水边。叶子有茎,又大又圆。叶子前后两面都是光滑的。中间的茎生长时,在顶开花。花是大黄色的,很美,有光泽,长得很多。茎很粗,是空心的。可以放进酸菜里吃,也可以喂牲畜。没有水的地方长不出来。}\end{exemple}\end{entrée}

\begin{entrée}{mɯrkuj}{}{ⓔmɯrkuj} 
\classe{n} 
\begin{définition}\pfra{espèce d'herbe}\end{définition}
\begin{définition}\pcmn{草的一种}\end{définition}\relationsémantique{同义词}{\lien{ⓔzgri}{zgri}}\end{entrée}

\begin{entrée}{mɯrkɯ}{}{ⓔmɯrkɯ} 
\classe{vl} \paradigme{dir}{tɤ-}
\begin{définition}\pfra{voler}\end{définition}
\begin{définition}\pcmn{偷}\end{définition}
\begin{exemple}\pjya{ɯʑo kɯ ta-mɯrkɯ ŋu}\hspace{5pt}\pcmn{是他偷的}\end{exemple}
\begin{sous-entrée}{kɯmɯrkɯ}{ⓔmɯrkɯⓝkɯmɯrkɯ} 
\classe{n} 
\begin{définition}\pfra{voleur}\end{définition}
\begin{définition}\pcmn{小偷}\end{définition}\end{sous-entrée}

\end{entrée}

\begin{entrée}{mɯrkɯrku}{}{ⓔmɯrkɯrku} 
\classe{n} 
\begin{définition}\pfra{tous les soirs}\end{définition}
\begin{définition}\pcmn{每天晚上}\end{définition}\end{entrée}

\begin{entrée}{mɯrmɯmbju}{}{ⓔmɯrmɯmbju} 
\classe{n} 
\begin{définition}\pfra{hirondelle}\end{définition}
\begin{définition}\pcmn{燕子}\end{définition}
\begin{exemple}\pjya{mɯrmɯmbju nɯ pɣa kɯ-xtɕi tsa ci ŋu, ɯ-βri ɲaʁ, ɯ-xtɤpa wɣrum, ɯ-jme artaʁ, tsɯntu fse, ɯ-ku xtɕi, jɤɣɤt ɯ-pa ri tɤ-rcoʁ kɯ kha tu-nɯ-βze ŋu, kɯ-dɯ-dɤn tɯtɯrca ku-rɤʑi-nɯ ŋu. tɯ-mɯ lɤt tɤ-kha tɕɤkɯ-ki ʑo ɲɯ-nɯqambɯmbjom ŋu tɕe, nɯ tɕu tɕe kɤ-mto dɤn ma nɯ mɤɕtʂa kɤ-mto rkɯn.}\hspace{5pt}\pcmn{燕子是一种比较小的鸟,身子黑,腹部白,尾巴是分叉的,像剪刀一样。在走缘下用稀泥打窝,成群地生活在一起。快要下雨的时候,飞得很低,所以就见的多一些,在其它时候见的少一些。}\end{exemple}\end{entrée}

\begin{entrée}{mɯrmɯr}{}{ⓔmɯrmɯr} 
\classe{idph.2} 
\begin{définition}\pfra{très fin (poudre)}\end{définition}
\begin{définition}\pcmn{形容磨得很细的样子}\end{définition}
\begin{exemple}\pjya{tɯsqar ɲɯ-ndɯβ mɯrmɯr ʑo}\hspace{5pt}\pcmn{糌粑磨得很细}\end{exemple}\end{entrée}

\begin{entrée}{mɯrnɤmɯr}{}{ⓔmɯrnɤmɯr} 
\classe{idph.2} 
\begin{définition}\pfra{bouchée après bouchée}\end{définition}
\begin{définition}\pcmn{形容吃东西一口接着一口的样子}\end{définition}
\begin{exemple}\pjya{rtɕhɯʁjɯ nɯ kɯ tɯrtɕhi ɯ-jwaʁ mɯrnɤmɯr tu-ndze ŋu}\hspace{5pt}\pcmn{毛虫把酸酸草的叶子一口接着一口地吃得很快}\end{exemple}\end{entrée}

\begin{entrée}{mɯrʁɯz}{}{ⓔmɯrʁɯz} 
\classe{vt} \paradigme{dir}{pɯ-}
\begin{définition}\pfra{griffer}\end{définition}
\begin{définition}\pcmn{抓}\end{définition}
\begin{exemple}\pjya{lɯlu kɯ pɯ́-wɣ-mɯrʁɯz-a tɕe a-jaʁ pjɤ-qraʁ}\hspace{5pt}\pcmn{猫把我抓了一下,抓破了我的手}\end{exemple}
\begin{exemple}\pjya{ma-pɯ-tɯ-mɯrʁɯz ma tɯ-ɕɯmŋɤm}\hspace{5pt}\pcmn{你别抓他,你令他很痛}\end{exemple}
\begin{sous-entrée}{sɤmɯrʁɯz}{ⓔmɯrʁɯzⓝsɤmɯrʁɯz} 
\classe{vi} 
\begin{définition}\pfra{griffer les gens}\end{définition}
\begin{définition}\pcmn{抓人}\end{définition}
\begin{exemple}\pjya{ma-tɯ-sɤmɯrʁɯz ma lɯlu ʑo ɲɯ-tɯ-fse}\hspace{5pt}\pcmn{你别抓人,你像一只猫!(教育小孩子时)}\end{exemple}\relationsémantique{参考}{\lien{ⓔtɯ-mɯrʁɯz}{tɯ-mɯrʁɯz}}\end{sous-entrée}

\end{entrée}

\begin{entrée}{mɯrtsɯɣ}{}{ⓔmɯrtsɯɣ} 
\classe{vt} \paradigme{dir}{nɯ-}\paradigme{dir}{nɯ-}
\begin{définition}\pfra{pincer (pour faire mal)}\end{définition}
\begin{définition}\pcmn{捏(惩罚人的方式)}\end{définition}
\begin{définition}\pfra{pincer les gens}\end{définition}
\begin{définition}\pcmn{捏别人}\end{définition}
\begin{exemple}\pjya{nɯ́-wɣ-mɯrtsɯɣ-a}\hspace{5pt}\pcmn{他捏了我}\end{exemple}
\begin{exemple}\pjya{tɤ-pɤtso taʁndo mɯ-tɤ-tso tɕe, ɲɯ́-wɣ-mɯrtsɯɣ tɕe phɤn}\hspace{5pt}\pcmn{小孩子不听话的时候捏一下就会好}\end{exemple}
\begin{exemple}\pjya{ki tɯrme kɤ-sɤmɯrtsɯɣ rga}\hspace{5pt}\pcmn{这个人喜欢捏别人}\end{exemple}\relationsémantique{参考}{\lien{ⓔtɯmɯrtsɯɣ}{tɯmɯrtsɯɣ}}
\begin{sous-entrée}{sɤmɯrtsɯɣ}{ⓔmɯrtsɯɣⓝsɤmɯrtsɯɣ} 
\classe{vi} \end{sous-entrée}

\end{entrée}

\begin{entrée}{mɯrʑa}{}{ⓔmɯrʑa} 
\classe{n}  
\grammaire{n.lieu} 
\begin{définition}\pfra{Merja (village de Gdongbrgyad)}\end{définition}
\begin{définition}\pcmn{木尔渣村}\end{définition}\end{entrée}

\begin{entrée}{mɯsta}{}{ⓔmɯsta} 
\classe{vs} 
\begin{définition}\pfra{ancien}\end{définition}
\begin{définition}\pcmn{古老}\end{définition}
\begin{exemple}\pjya{wuma kɯ-mɯsta ci ɲɯ-ŋu}\hspace{5pt}\pcmn{非常古老}\end{exemple}\end{entrée}

\begin{entrée}{mɯsti}{}{ⓔmɯsti} 
\classe{vs} \paradigme{dir}{tɤ-}
\begin{définition}\pfra{seul}\end{définition}
\begin{définition}\pcmn{孤单}\end{définition}
\begin{exemple}\pjya{jiɕqha nɯ ɯʑosti ɲɯ-ŋu, kɯ-mɯsti ci ɲɯ-ŋu}\hspace{5pt}\pcmn{他一个人,是个孤单的人}\end{exemple}\relationsémantique{参考}{\lien{ⓔɯʑo-sti}{ɯʑo-sti}}\end{entrée}

\begin{entrée}{mɯtɕhɯmɯrɯz}{}{ⓔmɯtɕhɯmɯrɯz} 
\classe{adv} 
\begin{définition}\pfra{toute sortes de}\end{définition}
\begin{définition}\pcmn{各种各样}\end{définition}
\begin{exemple}\pjya{tɯ-ŋga mɯtɕhɯmɯrɯz ɲɯ-xcat ʑo}\hspace{5pt}\pcmn{有各种各样的衣服}\end{exemple}\étymologie{mi.tɕʰi.mi.rigs}\end{entrée}

\begin{entrée}{mɯxte}{}{ⓔmɯxte} 
\classe{vs} 
\begin{définition}\pfra{être la majorité}\end{définition}
\begin{définition}\pcmn{占多数}\end{définition}
\begin{exemple}\pjya{kɯ-mɯxte ɯʑo kɯ ja-nɯtsɯm ɕti}\hspace{5pt}\pcmn{他把大部分东西带回家了}\end{exemple}
\begin{exemple}\pjya{kɯ-mɯxte aʑo tɤ-nɤma-t-a}\hspace{5pt}\pcmn{大多数都是我做的}\end{exemple}
\begin{exemple}\pjya{a-tɤ-rʑaʁ kɯ-mɯxte thɯ-arɕo ɕti}\hspace{5pt}\pcmn{我的时间大部分都过完了}\end{exemple}
\begin{exemple}\pjya{jiʑo kutɕu kɯrɯ tɯrme mɯxte-j}\hspace{5pt}\pcmn{我们藏族在这里占多数的人口}\end{exemple}\end{entrée}

\begin{entrée}{mɯχtɤn}{}{ⓔmɯχtɤn} 
\classe{vs} \paradigme{dir}{tɤ-}
\begin{définition}\pfra{dans une situation stable}\end{définition}
\begin{définition}\pcmn{踏实,稳定}\end{définition}
\begin{exemple}\pjya{aʑɯɣ tɤ-mɯχtɤn}\hspace{5pt}\pcmn{我的情况稳定了(不可能再有变动)}\end{exemple}
\begin{exemple}\pjya{a-z-rɤʑi tɤ-mɯχtɤn}\hspace{5pt}\pcmn{我住的地方固定了}\end{exemple}\étymologie{gtan}\end{entrée}

\begin{entrée}{mɯzi}{}{ⓔmɯzi} 
\classe{n} 
\begin{définition}\pfra{poudre}\end{définition}
\begin{définition}\pcmn{火药}\end{définition}\étymologie{mu.zi}\end{entrée}

\newpage\caractère{n}

\begin{entrée}{naŋɕa}{}{ⓔnaŋɕa} 
\classe{n} 
\begin{définition}\pfra{doublure}\end{définition}
\begin{définition}\pcmn{夹层衣服的内层部分}\end{définition}\relationsémantique{反义词}{\lien{ⓔɯ-ʁjoʁ}{ɯ-ʁjoʁ}}\end{entrée}

\begin{entrée}{naŋma}{}{ⓔnaŋma} 
\classe{n} 
\begin{définition}\pfra{partie intérieure des habits tibétains}\end{définition}
\begin{définition}\pcmn{衣服里子(藏装)}\end{définition}\étymologie{naŋ.ma}\end{entrée}

\begin{entrée}{naŋrzoŋ}{}{ⓔnaŋrzoŋ} 
\classe{n} 
\begin{définition}\pfra{rénovation (habitation)}\end{définition}
\begin{définition}\pcmn{装修}\end{définition}
\begin{exemple}\pjya{jiʑo naŋrzoŋ ku-osɯ-βzu-j}\hspace{5pt}\pcmn{我在装修房子}\end{exemple}\relationsémantique{参考}{\lien{ⓔkhɤrlɤn}{khɤrlɤn}}\end{entrée}

\begin{entrée}{naŋʁɯ}{}{ⓔnaŋʁɯ} 
\classe{n} 
\begin{définition}\pfra{chemise}\end{définition}
\begin{définition}\pcmn{衬衣}\end{définition}\end{entrée}

\begin{entrée}{naŋtɕɯ}{}{ⓔnaŋtɕɯ} 
\classe{n} 
\begin{définition}\pfra{organes internes}\end{définition}
\begin{définition}\pcmn{内脏}\end{définition}\end{entrée}

\begin{entrée}{naʁa}{}{ⓔnaʁa} 
\classe{vi} \paradigme{dir}{pɯ-}
\begin{définition}\pfra{travailler pour un salaire}\end{définition}
\begin{définition}\pcmn{打工}\end{définition}
\begin{exemple}\pjya{kɯ-naʁa jo-ɕe}\hspace{5pt}\pcmn{他去打工了}\end{exemple}\relationsémantique{同义词}{\lien{ⓔnɯŋgra}{nɯŋgra}}\relationsémantique{参考}{\lien{ⓔta-ʁa}{ta-ʁa}}\end{entrée}

\begin{entrée}{naʁdɤz}{}{ⓔnaʁdɤz} 
\classe{vt} \paradigme{dir}{tɤ-}\paradigme{dir}{tɤ-}
\begin{définition}\pfra{détester}\end{définition}
\begin{définition}\pcmn{讨厌;排挤;排斥}\end{définition}
\begin{définition}\pfra{détester les gens}\end{définition}
\begin{définition}\pcmn{排斥别人}\end{définition}
\begin{exemple}\pjya{ɲɯ́-wɣ-naʁdaz-a}\hspace{5pt}\pcmn{他讨厌我、想排挤我}\end{exemple}
\begin{exemple}\pjya{kɤnɤma mɤ-kɯ-cha nɯra ɲɯ-naʁdɤz}\hspace{5pt}\pcmn{他排斥不会做工的那些人}\end{exemple}
\begin{exemple}\pjya{ɯʑo ɲɯ-sɤnaʁdɤz, ci nɯ jo-nɯɕe pjɤ-ra}\hspace{5pt}\pcmn{他排斥别人,有人(因为他)必须离开}\end{exemple}\relationsémantique{参考}{\lien{ⓔɯ-ʁdɤz}{ɯ-ʁdɤz}}
\begin{sous-entrée}{sɤnaʁdɤz}{ⓔnaʁdɤzⓝsɤnaʁdɤz} 
\classe{vi}  
\grammaire{apass} \end{sous-entrée}

\end{entrée}

\begin{entrée}{naʁdɯɣ}{}{ⓔnaʁdɯɣ} 
\classe{vt}  
\grammaire{trop} \paradigme{dir}{tɤ-}\sens{1}
\begin{définition}\pfra{chicaner; être dérangé par}\end{définition}
\begin{définition}\pcmn{计较;介意}\end{définition}
\begin{exemple}\pjya{ta-ma tú-wɣ-znɤma jɤɣ ma mɤ-naʁdɯɣ}\hspace{5pt}\pcmn{可以让他做事,他不会介意的}\end{exemple}
\begin{exemple}\pjya{tɯ́-wɣ-naʁdɯɣ}\hspace{5pt}\pcmn{他会跟你计较}\end{exemple}
\begin{exemple}\pjya{ɯʑo ɯ-ŋga tú-wɣ-ŋga jɤɣ ma mɤ-naʁdɯɣ}\hspace{5pt}\pcmn{可以穿他的衣服,他不会介意的}\end{exemple}
\begin{exemple}\pjya{ndzɤtshi mɤ-kɯ-mɯm jarma kɯnɤ, naʁdɯɣ}\hspace{5pt}\pcmn{菜的味道他都要计较}\end{exemple}
\begin{exemple}\pjya{ki kɯ-xtɕi jamar ʑo mɤ-naʁdɯɣ}\hspace{5pt}\pcmn{这么小的事情,他不会计较}\end{exemple}
\begin{exemple}\pjya{mɤ-naʁdɯɣ-a}\hspace{5pt}\pcmn{我不在乎}\end{exemple}\sens{2}
\begin{définition}\pfra{dédaigner}\end{définition}
\begin{définition}\pcmn{嫌弃}\end{définition}\relationsémantique{参考}{\lien{ⓔʁdɯɣⓗ1}{ʁdɯɣ₁}}\end{entrée}

\begin{entrée}{naʁju}{}{ⓔnaʁju} 
\classe{vt} \paradigme{dir}{nɯ-}\paradigme{dir}{thɯ-}\paradigme{dir}{tɤ-}\paradigme{dir}{kɤ-}\paradigme{dir}{lɤ-}
\begin{définition}\pfra{curer (les dents, un trou etc)}\end{définition}
\begin{définition}\pcmn{剔(牙齿)、掏(洞等)}\end{définition}
\begin{exemple}\pjya{a-ɕɣa nɯ-naʁju-t-a}\hspace{5pt}\pcmn{我剔了牙齿}\end{exemple}
\begin{exemple}\pjya{zndɤrchɤβ thɯ-naʁju-t-a}\hspace{5pt}\pcmn{我掏了缝隙}\end{exemple}
\begin{exemple}\pjya{kɯspoʁ pjɤ-sti tɕe tɤ-naʁju-t-a}\hspace{5pt}\pcmn{洞塞了,我把它捅了一下}\end{exemple}
\begin{exemple}\pjya{a-rna kɤ-naʁju-t-a}\hspace{5pt}\pcmn{我掏了耳朵}\end{exemple}
\begin{exemple}\pjya{a-ɕna lɤ-naʁju-t-a}\hspace{5pt}\pcmn{我抠了鼻子}\end{exemple}\end{entrée}

\begin{entrée}{naʁjɯβ}{}{ⓔnaʁjɯβ} 
\classe{vt} \paradigme{dir}{tɤ-}
\begin{définition}\pfra{se cacher derrière}\end{définition}
\begin{définition}\pcmn{躲在……后}\end{définition}
\begin{exemple}\pjya{tu-kɯ-naʁjɯβ-a tɕe a-pɯ-tɯ́-wɣ-mto}\hspace{5pt}\pcmn{你躲在我身后,不让他们看见你}\end{exemple}\relationsémantique{同义词}{\lien{ⓔnaʁrɯm}{naʁrɯm}}\relationsémantique{参考}{\lien{ⓔta-ʁjɯβ}{ta-ʁjɯβ}}\relationsémantique{同义词}{\lien{ⓔanbaʁⓝsɤnbaʁ}{sɤnbaʁ}}\relationsémantique{参考}{\lien{ⓔsaʁjɯβ}{saʁjɯβ}}
\begin{sous-entrée}{znaʁjɯβ}{ⓔnaʁjɯβⓝznaʁjɯβ} 
\classe{vt}  
\grammaire{caus} \end{sous-entrée}

\end{entrée}

\begin{entrée}{naʁlo}{}{ⓔnaʁlo} 
\classe{n} 
\begin{définition}\pfra{casserole en fer}\end{définition}
\begin{définition}\pcmn{生铁锅}\end{définition}\end{entrée}

\begin{entrée}{naʁnɤt}{}{ⓔnaʁnɤt}\relationsémantique{参考}{\lien{ⓔʁnɤt}{ʁnɤt}}\end{entrée}

\begin{entrée}{naʁŋu}{}{ⓔnaʁŋu} 
\classe{n} 
\begin{définition}\pfra{seconde période du mois}\end{définition}
\begin{définition}\pcmn{下半月}\end{définition}\étymologie{nag.ŋo}\end{entrée}

\begin{entrée}{naʁre}{}{ⓔnaʁre} 
\classe{vt} 
\begin{définition}\pfra{respecter et craindre}\end{définition}
\begin{définition}\pcmn{敬重,敬畏}\end{définition}
\begin{exemple}\pjya{ɯ-sloχpɯn ɲɯ-naʁre}\hspace{5pt}\pcmn{他敬重他的老师}\end{exemple}\relationsémantique{参考}{\lien{ⓔsaʁre}{saʁre}}\relationsémantique{参考}{\lien{}{ɯ-rʁe}}\end{entrée}

\begin{entrée}{naʁri}{}{ⓔnaʁri} 
\classe{vs}  
\grammaire{denom} \paradigme{dir}{kɤ-}
\begin{définition}\pfra{hourdé de graisse}\end{définition}
\begin{définition}\pcmn{沾满油渍}\end{définition}
\begin{exemple}\pjya{tɯ-ŋga ɲɯ-naʁri tɕe, kɤ-ŋga mɯ́j-sɯsam-a}\hspace{5pt}\pcmn{衣服上沾满油渍,我不想穿}\end{exemple}\relationsémantique{参考}{\lien{ⓔta-ʁri}{ta-ʁri}}\end{entrée}

\begin{entrée}{naʁrɯm}{}{ⓔnaʁrɯm} 
\classe{vt} 
\begin{définition}\pfra{se cacher en se plaçant derrière}\end{définition}
\begin{définition}\pcmn{躲在……后面}\end{définition}
\begin{exemple}\pjya{tu-ta-naʁrɯm tɕe a-mɤ-pɯ́-wɣ-mto-a}\hspace{5pt}\pcmn{我要躲在你后面,不让别人看见我}\end{exemple}\relationsémantique{同义词}{\lien{ⓔnaʁjɯβ}{naʁjɯβ}}\relationsémantique{参考}{\lien{ⓔsaʁrɯm}{saʁrɯm}}\end{entrée}

\begin{entrée}{naʁzi}{}{ⓔnaʁzi} 
\classe{vt}  
\grammaire{trop} \paradigme{dir}{tɤ-}
\begin{définition}\pfra{avoir besoin de}\end{définition}
\begin{définition}\pcmn{需要用}\end{définition}
\begin{exemple}\pjya{ɲɯ-naʁzi}\hspace{5pt}\pcmn{他需要}\end{exemple}
\begin{exemple}\pjya{tɯrme kɯ-dɤn tsa naʁzi-a}\hspace{5pt}\pcmn{我需要多一点人}\end{exemple}
\begin{exemple}\pjya{a-tɤ́-wɣ-qur-a naʁzi-a}\hspace{5pt}\pcmn{我需要他帮我}\end{exemple}
\begin{exemple}\pjya{@cai ɕɯ-kɤ-χtɯ naʁzia}\hspace{5pt}\pcmn{我需要去买菜}\end{exemple}
\begin{exemple}\pjya{to-naʁzi}\hspace{5pt}\pcmn{他以前不需要,现在需要了}\end{exemple}
\begin{exemple}\pjya{nɤj nɤ-kɤ-naʁzi ɯ-tu ?}\hspace{5pt}\pcmn{有没有什么需要的?}\end{exemple}
\begin{exemple}\pjya{khɯtsa ɯ-tɯ-naʁzi ?}\hspace{5pt}\pcmn{你需不需要碗?}\end{exemple}
\begin{exemple}\pjya{kɤ-ʁndɯ ɯ-tɯ-naʁzi ?}\hspace{5pt}\pcmn{你想挨打是吧?(教育小孩子)}\end{exemple}
\begin{exemple}\pjya{mɤ-ta-naʁzi}\hspace{5pt}\pcmn{我不需要你}\end{exemple}
\begin{exemple}\pjya{nɤʑo nɤ-kɤ-nɤʁzi ɯ-ɣɤʑu nɤ, aʑo a-ɕki a-ɣɯ-jɤ-tɯ-re}\hspace{5pt}\pcmn{如果你有需要的话,你就在我这里来拿}\end{exemple}\relationsémantique{参考}{\lien{ⓔʁzi}{ʁzi}}\end{entrée}

\begin{entrée}{natɕhɯ}{}{ⓔnatɕhɯ} 
\classe{n} 
\begin{définition}\pfra{marais}\end{définition}
\begin{définition}\pcmn{沼泽(草坪上)}\end{définition}\étymologie{na.tɕʰu}\end{entrée}

\begin{entrée}{naχaʁ}{}{ⓔnaχaʁ} 
\classe{vt} 
\begin{définition}\pfra{être surpris par}\end{définition}
\begin{définition}\pcmn{对……感到惊奇}\end{définition}
\begin{exemple}\pjya{aʑo ɲɯ-naχaʁ-a ɕti}\hspace{5pt}\pcmn{我对这件事感到惊奇}\end{exemple}\relationsémantique{同义词}{\lien{ⓔnɤmtshɤr}{nɤmtshɤr}}\relationsémantique{参考}{\lien{ⓔsaχaʁ}{saχaʁ}}\end{entrée}

\begin{entrée}{naχɕɯn}{}{ⓔnaχɕɯn} 
\classe{vt} 
\begin{définition}\pfra{trouver propre}\end{définition}
\begin{définition}\pcmn{觉得干净}\end{définition}\relationsémantique{同义词}{\lien{ⓔnaχtso}{naχtso}}\relationsémantique{参考}{\lien{ⓔsaχɕɯn}{saχɕɯn}}\end{entrée}

\begin{entrée}{naχkɯ}{}{ⓔnaχkɯ} 
\classe{n} 
\begin{définition}\pfra{thé sans lait}\end{définition}
\begin{définition}\pcmn{黑茶(不加牛奶)}\end{définition}\end{entrée}

\begin{entrée}{naχpjɤt}{}{ⓔnaχpjɤt}\relationsémantique{参考}{\lien{ⓔχpjɤt}{χpjɤt}}\end{entrée}

\begin{entrée}{naχsoz}{}{ⓔnaχsoz} 
\classe{vs} \paradigme{dir}{tɤ-}
\begin{définition}\pfra{frais}\end{définition}
\begin{définition}\pcmn{新鲜}\end{définition}\paradigme{dir}{tɤ-}
\begin{définition}\pfra{se remettre d'aplomb}\end{définition}
\begin{définition}\pcmn{使自己提起精神}\end{définition}
\begin{exemple}\pjya{tɯβli ɲɯ-naχsoz}\hspace{5pt}\pcmn{苗子很新鲜(很有活力)}\end{exemple}
\begin{exemple}\pjya{jiɕqha nɯ kɯ-naχsoz ci ɲɯ-ŋu}\hspace{5pt}\pcmn{他是一个(面貌)很精神的人}\end{exemple}
\begin{exemple}\pjya{tʂha kɤ-tshi tɕe tɤ-ʑɣɤnaχsoz}\hspace{5pt}\pcmn{喝点茶,提起精神}\end{exemple}
\begin{sous-entrée}{sɤnaχsoz}{ⓔnaχsozⓝsɤnaχsoz} 
\classe{vs} 
\begin{définition}\pfra{vivifiant}\end{définition}
\begin{définition}\pcmn{使人清醒}\end{définition}
\begin{exemple}\pjya{qale a-pɯ-mɯɕtaʁ tɕe, wuma ʑo ɲɯ-sɤnaχsoz}\hspace{5pt}\pcmn{风冷的时候就叫人清醒}\end{exemple}\end{sous-entrée}

\begin{sous-entrée}{znaχsoz}{ⓔnaχsozⓝznaχsoz} 
\classe{vt} 
\begin{définition}\pfra{vivifier, remettre d'alpomb, réveiller}\end{définition}
\begin{définition}\pcmn{使……清醒}\end{définition}
\begin{exemple}\pjya{tʂha kú-wɣ-tshi tɕe, ɲɯ-kɯ-znaχsoz}\hspace{5pt}\pcmn{喝了茶就觉得清醒}\end{exemple}\end{sous-entrée}

\begin{sous-entrée}{ʑɣɤnaχsoz/\variante{ʑɣɤznaχsoz}}{ⓔnaχsozⓝʑɣɤnaχsoz} 
\classe{vi} \end{sous-entrée}

\étymologie{gsos}\end{entrée}

\begin{entrée}{naχtɕɯɣ}{}{ⓔnaχtɕɯɣ} 
\classe{vs} \paradigme{dir}{tɤ-}
\begin{définition}\pfra{semblable}\end{définition}
\begin{définition}\pcmn{一样}\end{définition}\paradigme{dir}{tɤ-}
\begin{exemple}\pjya{naχtɕɯɣ ɕti}\hspace{5pt}\pcmn{无所谓,都一样}\end{exemple}
\begin{exemple}\pjya{mɤ-kɯ-naχtɕɯɣ tu thaŋ nɯ-sɯso-t-a}\hspace{5pt}\pcmn{我想(两种说法的意思)可能不一样}\end{exemple}
\begin{exemple}\pjya{ɯʑo cho tɕi-tshɯɣa naχtɕɯɣ-tɕi}\hspace{5pt}\pcmn{我跟他长得很像}\end{exemple}
\begin{sous-entrée}{znaχtɕɯɣ}{ⓔnaχtɕɯɣⓝznaχtɕɯɣ} 
\classe{vt}  
\grammaire{caus} \end{sous-entrée}

\sens{1}
\begin{définition}\pfra{faire pareil, rendre semblable}\end{définition}
\begin{définition}\pcmn{使一样}\end{définition}
\begin{exemple}\pjya{ndʑi-mbɯlwa ta-znaχtɕɯɣ}\hspace{5pt}\pcmn{他给了他们俩一样的工资}\end{exemple}
\begin{exemple}\pjya{ɕkat ta-znaχtɕɯɣ}\hspace{5pt}\pcmn{他把驮子做成一样}\end{exemple}\sens{2}
\begin{définition}\pfra{trouver semblable}\end{définition}
\begin{définition}\pcmn{觉得一样}\end{définition}\étymologie{gtɕig}\end{entrée}

\begin{entrée}{naχthɤβ}{}{ⓔnaχthɤβ} 
\classe{vt} \paradigme{dir}{kɤ-}
\begin{définition}\pfra{en profiter pour}\end{définition}
\begin{définition}\pcmn{趁机会}\end{définition}
\begin{exemple}\pjya{aʁa tu ʑo kú-wɣ-naχthɤβ ra}\hspace{5pt}\pcmn{要趁我有空的时候}\end{exemple}
\begin{exemple}\pjya{aʑo mɤ-rɤʑia ʑo ko-naχthɤβ}\hspace{5pt}\pcmn{他趁我不在的时候}\end{exemple}
\begin{exemple}\pjya{ko-naχthɤβ tɕe kɯ-chi to-ndza}\hspace{5pt}\pcmn{他趁了这个机会吃了糖}\end{exemple}
\begin{exemple}\pjya{ɯʑo mɯ́j-rɤʑi ʑo kɤ-naχthaβ-a tɕe nɯ-nɯɣe-a ma nɯ maʁ nɤ kɤ-nɯɣi mɯ́j-nɤle}\hspace{5pt}\pcmn{我趁了他不在的时候回来,不然的话他不让我回来}\end{exemple}\end{entrée}

\begin{entrée}{naχti}{}{ⓔnaχti} 
\classe{vt} \paradigme{dir}{kɤ-}
\begin{définition}\pfra{devenir ami}\end{définition}
\begin{définition}\pcmn{结为伴侣}\end{définition}
\begin{exemple}\pjya{kɯ-mɤku ɯ-rʑaβ nɯ ɲɤ-βde tɕe kɯ-maqhu kɯmaʁ ci ko-naχti}\hspace{5pt}\pcmn{他跟原来的妻子离了婚,跟另外一个结为伴侣了}\end{exemple}\relationsémantique{参考}{\lien{ⓔtɯ-χti}{tɯ-χti}}\relationsémantique{参考}{\lien{ⓔsaχti}{saχti}}\end{entrée}

\begin{entrée}{naχto}{}{ⓔnaχto} 
\classe{vt} \paradigme{dir}{\_}
\begin{définition}\pfra{regarder fixement}\end{définition}
\begin{définition}\pcmn{盯}\end{définition}
\begin{exemple}\pjya{kɤ-naχto-t-a}\hspace{5pt}\pcmn{我盯着他看了}\end{exemple}
\begin{sous-entrée}{anaχtɯχto}{ⓔnaχtoⓝanaχtɯχto} 
\classe{vi}  
\grammaire{recip} 
\begin{définition}\pfra{se regarder fixement les uns uns les autres}\end{définition}
\begin{définition}\pcmn{互相盯着}\end{définition}\end{sous-entrée}

\end{entrée}

\begin{entrée}{naχtso}{}{ⓔnaχtso}\relationsémantique{参考}{\lien{ⓔχtso}{χtso}}\end{entrée}

\begin{entrée}{nɤboʁboʁ}{}{ⓔnɤboʁboʁ} 
\classe{vt} \paradigme{dir}{kɤ-}
\begin{définition}\pfra{s'attrouper autour de}\end{définition}
\begin{définition}\pcmn{簇拥}\end{définition}
\begin{exemple}\pjya{tɯrme ci a-pɯ-ndʐaβ tɕe kɯmaʁ tɯrme ra ku-nɤboʁboʁ-nɯ tɕe tu-qur-nɯ ŋgrɤl}\hspace{5pt}\pcmn{当有人摔跤的时候,其他人会拥上来帮他}\end{exemple}\relationsémantique{同义词}{\lien{ⓔnɤɣɯβɣɯβ}{nɤɣɯβɣɯβ}}\end{entrée}

\begin{entrée}{nɤβdɤle}{}{ⓔnɤβdɤle}\relationsémantique{参考}{\lien{ⓔβde}{βde}}\end{entrée}

\begin{entrée}{nɤβdi}{}{ⓔnɤβdi} 
\classe{vi} \paradigme{dir}{kɤ-}
\begin{définition}\pfra{sois en bonne santé}\end{définition}
\begin{définition}\pcmn{祝你平安}\end{définition}
\begin{exemple}\pjya{aj nɯɕe-a ŋu, kɤ-nɤβdi}\hspace{5pt}\pcmn{我回去了,祝你平安}\end{exemple}
\begin{exemple}\pjya{aʑo nɯ mɤɕtʂa pɯ-rɤʑi-a, kɤ-nɤβdi}\hspace{5pt}\pcmn{我坐了这么多,现在要走了,祝你平安}\end{exemple}
\begin{exemple}\pjya{kɤ-nɤβdi je !}\hspace{5pt}\pcmn{祝你平安(离别的人出发时对留住的人说的)}\end{exemple}
\begin{exemple}\pjya{kɤ-nɤβdi-ndʑi}\hspace{5pt}\pcmn{祝你们俩平安}\end{exemple}\end{entrée}

\begin{entrée}{nɤβɟu}{}{ⓔnɤβɟu} 
\classe{vt} \paradigme{dir}{pɯ-}
\begin{définition}\pfra{se servir de ... comme d'un matelas}\end{définition}
\begin{définition}\pcmn{垫着坐}\end{définition}
\begin{exemple}\pjya{@bandeng pɯ-nɤβɟe}\hspace{5pt}\pcmn{坐在板凳上吧}\end{exemple}
\begin{exemple}\pjya{tɤ-βɟu pɯ-nɤβɟe}\hspace{5pt}\pcmn{用垫子垫着坐吧}\end{exemple}
\begin{exemple}\pjya{ɯ-thoʁ ɲɯ-ɤci tɕe, a-ŋga pɯ-nɯ-nɤβɟu-t-a}\hspace{5pt}\pcmn{因为地上很湿,所以我垫了衣服坐}\end{exemple}\relationsémantique{参考}{\lien{ⓔtɤ-βɟu}{tɤ-βɟu}}\end{entrée}

\begin{entrée}{nɤβɟɯβɟi}{}{ⓔnɤβɟɯβɟi} 
\classe{vt}  
\grammaire{n.orient} \paradigme{dir}{tɤ-}
\begin{définition}\pfra{poursuivre dans tous les sens}\end{définition}
\begin{définition}\pcmn{追来追去}\end{définition}
\begin{exemple}\pjya{aj nɤj tu-ta-nɯβɟɯβji}\hspace{5pt}\pcmn{我把你追来追去}\end{exemple}
\begin{exemple}\pjya{nɤʑo kɯ aʑo tu-kɯnɤβɟɯβɟi-a}\hspace{5pt}\pcmn{你把我追来追去}\end{exemple}
\begin{exemple}\pjya{khɯna kɯ tshɤt ta-nɤβɟɯβɟi}\hspace{5pt}\pcmn{狗把山羊追得到处跑了}\end{exemple}\relationsémantique{参考}{\lien{ⓔβɟiⓗ1}{βɟi₁}}\end{entrée}

\begin{entrée}{nɤβrɯβraʁ}{}{ⓔnɤβrɯβraʁ}\relationsémantique{参考}{\lien{ⓔβraʁ}{βraʁ}}\end{entrée}

\begin{entrée}{nɤβzɯβzu}{}{ⓔnɤβzɯβzu}\relationsémantique{参考}{\lien{ⓔβzuⓗ1}{βzu₁}}\end{entrée}

\begin{entrée}{nɤcu}{}{ⓔnɤcu}\relationsémantique{参考}{\lien{ⓔacu}{acu}}\end{entrée}

\begin{entrée}{nɤcɯpa}{}{ⓔnɤcɯpa} 
\classe{vt} \paradigme{dir}{tɤ-}
\begin{définition}\pfra{fermer et ouvrir}\end{définition}
\begin{définition}\pcmn{开和关}\end{définition}\relationsémantique{参考}{\lien{ⓔcɯⓗ3}{cɯ}}\relationsémantique{参考}{\lien{ⓔpaⓗ1}{pa₁}}\end{entrée}

\begin{entrée}{nɤɕu}{}{ⓔnɤɕu} 
\classe{vi} \paradigme{dir}{kɤ-}
\begin{définition}\pfra{se protéger du soleil}\end{définition}
\begin{définition}\pcmn{避暑}\end{définition}
\begin{exemple}\pjya{si ɯ-pa kɤ-nɤɕu-a}\hspace{5pt}\pcmn{我在树下面避暑了}\end{exemple}\relationsémantique{参考}{\lien{ⓔɣɤɕu}{ɣɤɕu}}\end{entrée}

\begin{entrée}{nɤɕarlar}{}{ⓔnɤɕarlar} 
\classe{vt}  
\grammaire{n.orient} \paradigme{dir}{nɯ-}
\begin{définition}\pfra{chercher partout}\end{définition}
\begin{définition}\pcmn{到处找}\end{définition}\relationsémantique{参考}{\lien{ⓔɕar}{ɕar}}\relationsémantique{同义词}{\lien{ⓔnɤɕɯɕar}{nɤɕɯɕar}}\relationsémantique{参考}{\lien{ⓔɕar}{ɕar}}\end{entrée}

\begin{entrée}{nɤɕejɣi/\variante{nɤɕejɣɯjɣi}}{}{ⓔnɤɕejɣi} 
\classe{vi} \paradigme{dir}{\_}
\begin{définition}\pfra{aller et venir}\end{définition}
\begin{définition}\pcmn{来回;来来往往(次数多)}\end{définition}
\begin{exemple}\pjya{kɯm a-pɯ-ɲɟɯ, khɯɣɲɟɯ a-pɯ-ɲɟɯ tɕe, kha ɯ-ŋgɯ qale nɯ ju-nɤɕejɣi ɲɯ-cha, tɕe qale ɯ-mbe ju-ɕe, qale kɯ-ɕɤɣ ju-ɣi ɲɯ-cha tɕe ɲɯ-sɤscit}\hspace{5pt}\pcmn{如果门和窗子是开着的,风可以在屋里流动,旧的空气流出,新鲜空气流进来,这样就显得舒服一些}\end{exemple}\relationsémantique{参考}{\lien{ⓔɕe}{ɕe}}\relationsémantique{参考}{\lien{ⓔɣi}{ɣi}}\end{entrée}

\begin{entrée}{nɤɕkhɯɕkho}{}{ⓔnɤɕkhɯɕkho} 
\classe{vi}  
\grammaire{n.orient} \paradigme{dir}{nɯ-}
\begin{définition}\pfra{faire sécher pendant plusieurs jours en retournant régulièrement}\end{définition}
\begin{définition}\pcmn{晒几天,翻来覆去地晒}\end{définition}
\begin{exemple}\pjya{stoʁ staχpɯ ɲɤ-k-ɤci tɕe, jisŋi na-nɤɕkhɯɕkho}\hspace{5pt}\pcmn{胡豆和豌豆湿了,所以他今天把它们晒干了}\end{exemple}
\begin{exemple}\pjya{tɯ-ŋga mɯ-tɤ-kɯ-zbaʁ ri, jisŋi na-nɤɕkhɯɕkho}\hspace{5pt}\pcmn{衣服没有干,所以他今天晒干了}\end{exemple}\end{entrée}

\begin{entrée}{nɤɕqa}{}{ⓔnɤɕqa} 
\classe{vt} \paradigme{dir}{nɯ-}\paradigme{dir}{nɯ-}
\begin{définition}\pfra{supporter}\end{définition}
\begin{définition}\pcmn{忍耐}\end{définition}
\begin{définition}\pfra{être capable de supporter}\end{définition}
\begin{définition}\pcmn{忍得了}\end{définition}
\begin{exemple}\pjya{jɯfɕɯr a-xtu tɤ-mŋɤm tɕe nɯ-nɤɕqa-t-a}\hspace{5pt}\pcmn{昨天我肚子疼起来了,但是我还是忍了}\end{exemple}
\begin{exemple}\pjya{ɯ-sŋɯro kɤ-lɤt ɲɤ-nɤɕqa}\hspace{5pt}\pcmn{他屏住呼吸了}\end{exemple}
\begin{exemple}\pjya{tɯ-rju kɯ-ŋɤn ɲɯ-ti tɕe, na-nɤɕqa}\hspace{5pt}\pcmn{他说了很难听的话,但是他还是忍了}\end{exemple}
\begin{exemple}\pjya{ɯ-mi ka-ɣle ri, ɲɯ-znɤɕqe}\hspace{5pt}\pcmn{他崴了脚,但是还是忍住了}\end{exemple}
\begin{exemple}\pjya{mɯ́j-znɤɕqe-a}\hspace{5pt}\pcmn{我忍不住(我受不了)}\end{exemple}\relationsémantique{参考}{\lien{ⓔsɤɕqa}{sɤɕqa}}
\begin{sous-entrée}{znɤɕqa}{ⓔnɤɕqaⓝznɤɕqa} 
\classe{vt} \end{sous-entrée}

\end{entrée}

\begin{entrée}{nɤɕqraʁ}{}{ⓔnɤɕqraʁ}\relationsémantique{参考}{\lien{ⓔɕqraʁ}{ɕqraʁ}}\end{entrée}

\begin{entrée}{nɤɕthɯɕthɯz}{}{ⓔnɤɕthɯɕthɯz}\relationsémantique{参考}{\lien{ⓔɕthɯz}{ɕthɯz}}\end{entrée}

\begin{entrée}{nɤɕtʂaʁli}{}{ⓔnɤɕtʂaʁli} 
\classe{vt} \paradigme{dir}{nɯ-}
\begin{définition}\pfra{torturer}\end{définition}
\begin{définition}\pcmn{折磨}\end{définition}\end{entrée}

\begin{entrée}{nɤɕɯɕar}{}{ⓔnɤɕɯɕar} 
\classe{vt}  
\grammaire{n.orient} \paradigme{dir}{nɯ-}
\begin{définition}\pfra{chercher partout}\end{définition}
\begin{définition}\pcmn{到处找}\end{définition}
\begin{exemple}\pjya{nɯ-nɤɕɯɕar-a}\hspace{5pt}\pcmn{我到处找了}\end{exemple}\relationsémantique{同义词}{\lien{ⓔnɤɕarlar}{nɤɕarlar}}\relationsémantique{参考}{\lien{ⓔɕar}{ɕar}}\end{entrée}

\begin{entrée}{nɤɕɯɕe}{}{ⓔnɤɕɯɕe} 
\classe{vi}  
\grammaire{n.orient} \paradigme{dir}{\_}
\begin{définition}\pfra{aller partout}\end{définition}
\begin{définition}\pcmn{到处走,出远门}\end{définition}
\begin{exemple}\pjya{a-mɤ-jɤ-nɤɕɯɕe}\hspace{5pt}\pcmn{不要让它到处走(一般说牲畜)}\end{exemple}\relationsémantique{参考}{\lien{ⓔɕe}{ɕe}}\end{entrée}

\begin{entrée}{nɤɕɯɕi}{}{ⓔnɤɕɯɕi} 
\classe{vt} \paradigme{dir}{\_}
\begin{définition}\pfra{traîner par terre}\end{définition}
\begin{définition}\pcmn{在地上拖}\end{définition}
\begin{exemple}\pjya{ɕoŋtɕa kɤ-fkur mɯ́j-sɤcha tɕe thɯ-nɤɕɯɕi-t-a ɕti}\hspace{5pt}\pcmn{木料背不动,所以我拖了}\end{exemple}\relationsémantique{同义词}{\lien{ⓔnɤkhɯkhrɯt}{nɤkhɯkhrɯt}}\end{entrée}

\begin{entrée}{nɤdɤn}{}{ⓔnɤdɤn}\relationsémantique{参考}{\lien{ⓔdɤn}{dɤn}}\end{entrée}

\begin{entrée}{nɤfcaʁ}{}{ⓔnɤfcaʁ} 
\classe{vt} \paradigme{dir}{tɤ-}
\begin{définition}\pfra{se servir (d'un tissu= pour protéger son dos lorsque l'on porte des charges sur le dos}\end{définition}
\begin{définition}\pcmn{当作背垫}\end{définition}
\begin{exemple}\pjya{mboʁ tɤ-nɤfcaʁ-a}\hspace{5pt}\pcmn{我把正方形布料当作背垫了}\end{exemple}\relationsémantique{参考}{\lien{ⓔtɯfcaʁ}{tɯfcaʁ}}\end{entrée}

\begin{entrée}{nɤfɕɯfɕɤt}{}{ⓔnɤfɕɯfɕɤt}\relationsémantique{参考}{\lien{ⓔfɕɤtⓗ1}{fɕɤt₁}}\end{entrée}

\begin{entrée}{nɤfkɯfkur}{}{ⓔnɤfkɯfkur}\relationsémantique{参考}{\lien{ⓔfkur}{fkur}}\end{entrée}

\begin{entrée}{nɤfse}{}{ⓔnɤfse} 
\classe{vt}  
\grammaire{trop} \paradigme{dir}{pɯ-}
\begin{définition}\pfra{trouver semblable}\end{définition}
\begin{définition}\pcmn{觉得相似;觉得好像}\end{définition}
\begin{exemple}\pjya{jiɕqha nɯ ɲɯ-nɤfse-a}\hspace{5pt}\pcmn{我觉得这个人很像他}\end{exemple}
\begin{exemple}\pjya{jiɕqha nɯnɯ xiangbolin ŋu thaŋ ri, ɲɯ-nɤfse-a ri, ŋu maʁ mɤxsi}\hspace{5pt}\pcmn{我觉得这个人很像向柏霖,但不清楚是不是他}\end{exemple}
\begin{exemple}\pjya{nɤʑo ndɤre a-mu ɲɯ-ta-nɤfse}\hspace{5pt}\pcmn{我觉得你很像我的母亲}\end{exemple}
\begin{exemple}\pjya{jɯfɕɯr pɯ-kɯ-fse nɯ ra a-jmŋo ʑo ɲɯ-nɤfse-a}\hspace{5pt}\pcmn{昨天发生的事情我觉得像一场梦一样}\end{exemple}\relationsémantique{参考}{\lien{ⓔfseⓗ1}{fse₁}}\end{entrée}

\begin{entrée}{nɤfsɯfse}{}{ⓔnɤfsɯfse} 
\classe{vi} \paradigme{dir}{tɤ-}\sens{1}
\begin{définition}\pfra{être présomptueux}\end{définition}
\begin{définition}\pcmn{自以为是;装模作样}\end{définition}
\begin{exemple}\pjya{nɤ-kɤ-nɤfsɯfse ra mɤ-ra, nɤ-stu tɤ-fse}\hspace{5pt}\pcmn{你不要装模作样,要规矩一点}\end{exemple}
\begin{exemple}\pjya{nɯ ɕɯŋgɯ kɯ-fse mɯ-to-nɤfsɯfse tɕe tham tɕe ɲɯ-pe}\hspace{5pt}\pcmn{他没有以前那样自以为是了,现在好了}\end{exemple}\sens{2}
\begin{définition}\pfra{faire n'importe quoi}\end{définition}
\begin{définition}\pcmn{乱来}\end{définition}
\begin{exemple}\pjya{ma-tɯ-nɤfsɯfse}\hspace{5pt}\pcmn{你不要乱来}\end{exemple}\end{entrée}

\begin{entrée}{nɤfsɯr}{}{ⓔnɤfsɯr} 
\classe{vt}  
\grammaire{denom} \paradigme{dir}{tɤ-}
\begin{définition}\pfra{se servir comme cible}\end{définition}
\begin{définition}\pcmn{当靶子}\end{définition}
\begin{exemple}\pjya{tɤ-nɤfsɯr-a}\hspace{5pt}\pcmn{我把它当靶子了}\end{exemple}
\begin{exemple}\pjya{rdɤstaʁ tɤ-lat-a tɕe tɤ-nɤfsɯr-a}\hspace{5pt}\pcmn{我扔了一块石头,当做靶子}\end{exemple}
\begin{exemple}\pjya{tɤfsɯr ʑ-lɤ-ta-t-a tɕe tɤ-nɤfsɯr-a}\hspace{5pt}\pcmn{我把靶子放在那里,当做靶子}\end{exemple}
\begin{exemple}\pjya{ʁmaʁmi ra kɯ χɕɤlphoŋ ɲɯ-ɤz-nɤfsɯr-nɯ}\hspace{5pt}\pcmn{士兵们把瓶子当靶子(练习打枪)}\end{exemple}\relationsémantique{参考}{\lien{ⓔtɤfsɯr}{tɤfsɯr}}\end{entrée}

\begin{entrée}{nɤgɯr}{}{ⓔnɤgɯr} 
\classe{vt} \sens{1}
\begin{définition}\pfra{faire la plus grande partie de}\end{définition}
\begin{définition}\pcmn{大多数都是……}\end{définition}
\begin{exemple}\pjya{tɯ-ɣli kɤ-nɯzʁe nɤj pɯ-tɯ-nɤgɯr}\hspace{5pt}\pcmn{肥料大多数都是你背的}\end{exemple}\sens{2}
\begin{définition}\pfra{accepter de son plein gré (des critiques)}\end{définition}
\begin{définition}\pcmn{心服口服地接受批评}\end{définition}
\begin{exemple}\pjya{mɯ́j-nɤgɯr-a}\hspace{5pt}\pcmn{我不服气}\end{exemple}\end{entrée}

\begin{entrée}{nɤɣa}{}{ⓔnɤɣa} 
\classe{vi} \sens{1}
\begin{définition}\pfra{ne pas avoir de vertige}\end{définition}
\begin{définition}\pcmn{不畏高}\end{définition}
\begin{exemple}\pjya{praʁ ɯ-taʁ ɲɯ-nɤɣa}\hspace{5pt}\pcmn{他在悬崖上不畏高}\end{exemple}
\begin{exemple}\pjya{kɯ-mbro kɤ-ɕe ɲɯ-cha, ɲɯ-nɤɣa}\hspace{5pt}\pcmn{他可以去很高的地方,他不畏高}\end{exemple}
\begin{exemple}\pjya{ndzom ɯ-taʁ ɲɯ-nɤɣa}\hspace{5pt}\pcmn{他在桥上不畏高}\end{exemple}
\begin{exemple}\pjya{mɯ́j-nɤɣa tɕe, kɯ-mbro kɤ-ɕe mɯ́j-nɤz}\hspace{5pt}\pcmn{他畏高,不敢去到高地}\end{exemple}\relationsémantique{反义词}{\lien{ⓔnɤjmbɣom}{nɤjmbɣom}}\sens{2}
\begin{définition}\pfra{être complètement visible}\end{définition}
\begin{définition}\pcmn{完全看得到;明显}\end{définition}
\begin{exemple}\pjya{tɤɣal ɲɯ-rɤʑi tɕe ɲɯ-nɤɣa}\hspace{5pt}\pcmn{完全看得到他}\end{exemple}\relationsémantique{同义词}{\lien{ⓔsɤmto}{sɤmto}}\end{entrée}

\begin{entrée}{nɤɣɤɕe}{}{ⓔnɤɣɤɕe}\relationsémantique{参考}{\lien{ⓔɣɤɕe}{ɣɤɕe}}\end{entrée}

\begin{entrée}{nɤɣɤzri}{}{ⓔnɤɣɤzri}\relationsémantique{参考}{\lien{ⓔzri}{zri}}\end{entrée}

\begin{entrée}{nɤɣɟaj}{}{ⓔnɤɣɟaj} 
\classe{vt} \paradigme{dir}{tɤ-}
\begin{définition}\pfra{forcer, soulever avec un levier}\end{définition}
\begin{définition}\pcmn{撬开}\end{définition}
\begin{exemple}\pjya{rŋgɯ tɤ-nɤɣɟaj-a}\hspace{5pt}\pcmn{我把石包撬开了}\end{exemple}
\begin{exemple}\pjya{kɯm kɤ-cɯ mɯ́j-khɯ tɕe tɤ-nɤɣɟaj-i}\hspace{5pt}\pcmn{门打不开了,我们就把它撬开了}\end{exemple}\relationsémantique{参考}{\lien{ⓔtɤɣɟajⓗ1}{tɤɣɟaj}}\end{entrée}

\begin{entrée}{nɤɣlɤɣle}{}{ⓔnɤɣlɤɣle}\relationsémantique{参考}{\lien{ⓔɣle}{ɣle}}\end{entrée}

\begin{entrée}{nɤɣmaʁ}{}{ⓔnɤɣmaʁ} 
\classe{vt}  
\grammaire{trop} \paradigme{dir}{nɯ-}
\begin{définition}\pfra{considérer comme injuste, regretter}\end{définition}
\begin{définition}\pcmn{觉得不对;后悔}\end{définition}
\begin{exemple}\pjya{jiɕqha tɤ-kɯ-nɤmqe-a tɕe nɯ-nɤɣmaʁ-a}\hspace{5pt}\pcmn{你刚才骂我,我觉得是不对的}\end{exemple}
\begin{exemple}\pjya{tɤ-ta-nɤmqe tɕe nɯ-nɤɣmaʁ-a}\hspace{5pt}\pcmn{我骂了你,现在后悔了}\end{exemple}\relationsémantique{参考}{\lien{ⓔmaʁⓗ1}{maʁ₁}}\end{entrée}

\begin{entrée}{nɤɣmɤr}{}{ⓔnɤɣmɤr} 
\classe{vt}  
\grammaire{denom} \paradigme{dir}{tɤ-}
\begin{définition}\pfra{tenir dans la bouche}\end{définition}
\begin{définition}\pcmn{含在嘴里}\end{définition}
\begin{exemple}\pjya{aʑo kɯ-chi tɤ-nɤɣmar-a}\hspace{5pt}\pcmn{我把糖含在嘴里了}\end{exemple}\relationsémantique{参考}{\lien{ⓔtɯ-ɣmɤr}{tɯ-ɣmɤr}}\end{entrée}

\begin{entrée}{nɤɣmbat}{}{ⓔnɤɣmbat} 
\classe{vi}  
\grammaire{trop} \sens{1}\paradigme{dir}{pɯ-}
\begin{définition}\pfra{finir facilement}\end{définition}
\begin{définition}\pcmn{很轻松地做完}\end{définition}
\begin{exemple}\pjya{laχtɕha kɤ-nɯzʁe-t-a, pɯ-nɤɣmbat-a}\hspace{5pt}\pcmn{我搬东西,很轻松地搬完了}\end{exemple}
\begin{exemple}\pjya{jisŋi @kaoshi tɤ-βzu-t-a, pɯ-nɤɣmbat-a}\hspace{5pt}\pcmn{今天的考试很轻松就做完了}\end{exemple}\sens{2}\paradigme{dir}{tɤ-}
\begin{définition}\pfra{être presque fini}\end{définition}
\begin{définition}\pcmn{快没有了,只剩下一点}\end{définition}
\begin{exemple}\pjya{tɤ-fkɯm ɯ-ŋgɯ kɤ-rku nɯ to-nɤɣmbat}\hspace{5pt}\pcmn{装在口袋里的东西快没有了}\end{exemple}\relationsémantique{参考}{\lien{ⓔmbat}{mbat}}\end{entrée}

\begin{entrée}{nɤɣmbɤβ}{}{ⓔnɤɣmbɤβ} 
\classe{vt} \paradigme{dir}{pɯ-}\paradigme{dir}{tɤ-}
\begin{définition}\pfra{être disposé à écouter}\end{définition}
\begin{définition}\pcmn{服从;愿意听}\end{définition}
\begin{exemple}\pjya{aj tu-ti-a nɯ maka mɯ́j-kɯ-nɤɣmbaβ-a}\hspace{5pt}\pcmn{我说的话你怎么都愿意听}\end{exemple}
\begin{exemple}\pjya{tɤ-ndzɯmbra-t-a tɕe, ɲɯ-nɤɣmbɤβ}\hspace{5pt}\pcmn{我教育了他,他就听了}\end{exemple}\end{entrée}

\begin{entrée}{nɤɣro}{}{ⓔnɤɣro}\relationsémantique{参考}{\lien{ⓔaɣro}{aɣro}}\end{entrée}

\begin{entrée}{nɤɣrɯ}{}{ⓔnɤɣrɯ} 
\classe{vt} \paradigme{dir}{tɤ-}
\begin{définition}\pfra{avoir besoin de}\end{définition}
\begin{définition}\pcmn{需要}\end{définition}
\begin{exemple}\pjya{tɕhi kɯ-fse tɯ-nɤɣri}\hspace{5pt}\pcmn{你需要怎么样的东西?}\end{exemple}
\begin{exemple}\pjya{nɤʑo ɲɯ-tɯ-nɤɣri ndʐa aj chɯ-fɕi-a ŋu}\hspace{5pt}\pcmn{我会根据你需要的东西打铁的}\end{exemple}
\begin{exemple}\pjya{ki ɲɯ-nɤɣri-a, ki mɯ́j-nɤɣri-a}\hspace{5pt}\pcmn{我需要这个,不需要那个}\end{exemple}
\begin{exemple}\pjya{mɤ-kɤ-nɤɣrɯ nɯ aʑɯɣ mɯ́j-ra}\hspace{5pt}\pcmn{不需要的东西我不要}\end{exemple}\end{entrée}

\begin{entrée}{nɤɣɯβɣɯβ}{}{ⓔnɤɣɯβɣɯβ} 
\classe{vt} \paradigme{dir}{kɤ-}
\begin{définition}\pfra{s'attrouper}\end{définition}
\begin{définition}\pcmn{簇拥,围拢}\end{définition}
\begin{exemple}\pjya{tɤ-mthɯm ci a-pɯ-tu tɕe, khɯna ra kɯ ku-nɤɣɯβɣɯβ-nɯ ʑo ŋu}\hspace{5pt}\pcmn{当有一块肉在那里的时候,狗就蜂拥而来}\end{exemple}\relationsémantique{同义词}{\lien{ⓔnɤboʁboʁ}{nɤboʁboʁ}}\end{entrée}

\begin{entrée}{nɤɣʑa}{}{ⓔnɤɣʑa} 
\classe{vt} \paradigme{dir}{tɤ-}
\begin{définition}\pfra{récolter}\end{définition}
\begin{définition}\pcmn{拔出来;收割(大麻)}\end{définition}
\begin{exemple}\pjya{tasa tɤ-nɤɣʑa-t-a}\hspace{5pt}\pcmn{我把大麻收割了(选出不能结种子的大麻)}\end{exemple}\relationsémantique{参考}{\lien{}{tasɤɣʑa}}\end{entrée}

\begin{entrée}{nɤj}{}{ⓔnɤj} 
\classe{pro} 
\begin{définition}\pfra{toi}\end{définition}
\begin{définition}\pcmn{你}\end{définition}\relationsémantique{参考}{\lien{ⓔnɤʑo}{nɤʑo}}\end{entrée}

\begin{entrée}{nɤja}{}{ⓔnɤja} 
\classe{vi} \paradigme{dir}{pɯ-}
\begin{définition}\pfra{être dommage}\end{définition}
\begin{définition}\pcmn{可惜}\end{définition}
\begin{exemple}\pjya{jiɕqha laχtɕha pjɤ-ɴɢrɯ, pɯ-nɤja}\hspace{5pt}\pcmn{这个东西破了,很可惜}\end{exemple}\relationsémantique{参考}{\lien{ⓔznɤja}{znɤja}}\end{entrée}

\begin{entrée}{nɤjaʁ}{₁}{ⓔnɤjaʁⓗ1} 
\classe{vt}  
\grammaire{denom} \paradigme{dir}{kɤ-}
\begin{définition}\pfra{toucher}\end{définition}
\begin{définition}\pcmn{抚摸}\end{définition}
\begin{exemple}\pjya{ɯʑo kɯ kɤ́-wɣ-nɤjaʁ-a}\hspace{5pt}\pcmn{他抚摸了我}\end{exemple}\relationsémantique{同义词}{\lien{ⓔnɤmɤle}{nɤmɤle}}\relationsémantique{同义词}{\lien{ⓔnɤmɯma}{nɤmɯma}}\relationsémantique{参考}{\lien{ⓔtɯ-jaʁ}{tɯ-jaʁ}}\end{entrée}

\begin{entrée}{nɤjaʁ}{₂}{ⓔnɤjaʁⓗ2} 
\classe{vt}  
\grammaire{trop} 
\begin{définition}\pfra{trouver trop épais}\end{définition}
\begin{définition}\pcmn{觉得太厚}\end{définition}
\begin{exemple}\pjya{a-ŋga ɲɯ-nɤjaʁ-a}\hspace{5pt}\pcmn{我觉得衣服太厚}\end{exemple}\relationsémantique{参考}{\lien{ⓔjaʁ}{jaʁ}}\end{entrée}

\begin{entrée}{nɤjɤrjɤr}{}{ⓔnɤjɤrjɤr}\relationsémantique{参考}{\lien{ⓔjɤrjɤr}{jɤrjɤr}}\end{entrée}

\begin{entrée}{nɤjɣɯjɣɤt}{}{ⓔnɤjɣɯjɣɤt} 
\classe{vi}  
\grammaire{n.orient} \paradigme{dir}{\_}
\begin{définition}\pfra{aller et revenir}\end{définition}
\begin{définition}\pcmn{走了又转回来}\end{définition}
\begin{exemple}\pjya{aki jiɕqha pɯ-ari-a li tɤ-jɣat-a tɕe, pɯ-nɤjɣɯjɣat-a ntsɯ}\hspace{5pt}\pcmn{我往下去了,又往上转来,我反复去了几次又回来了}\end{exemple}\relationsémantique{参考}{\lien{ⓔjɣɤt}{jɣɤt}}\end{entrée}

\begin{entrée}{nɤjkɯz}{}{ⓔnɤjkɯz} 
\classe{vt}  
\grammaire{denom} \paradigme{dir}{tɤ-}\paradigme{dir}{tɤ-}
\begin{définition}\pfra{faire quelque chose en cachette}\end{définition}
\begin{définition}\pcmn{瞒着}\end{définition}
\begin{définition}\pfra{faire quelque chose en cachette des autres}\end{définition}
\begin{définition}\pcmn{瞒着别人}\end{définition}
\begin{exemple}\pjya{laχtɕha tɤ-nɤjkɯz-a tɕe kɤ-ɣɯt-a}\hspace{5pt}\pcmn{我偷偷地把东西带过来了}\end{exemple}
\begin{exemple}\pjya{aʑo tɤ-kɯ-nɤjkɯz-a tɕe a-laχtɕha jɤ-tɯ-tsɯm}\hspace{5pt}\pcmn{你瞒着我把我的东西带来了}\end{exemple}
\begin{exemple}\pjya{laχtɕha to-mɯrkɯ tɤ́-wɣ-nɤjkɯz-a}\hspace{5pt}\pcmn{他偷了东西,没有让我发现}\end{exemple}
\begin{exemple}\pjya{tɯrme tɤ-nɤjkɯz-a}\hspace{5pt}\pcmn{我瞒着人家做了}\end{exemple}\relationsémantique{参考}{\lien{ⓔtɤjkɯz}{tɤjkɯz}}
\begin{sous-entrée}{sɤnɤjkɯz}{ⓔnɤjkɯzⓝsɤnɤjkɯz} 
\classe{vi}  
\grammaire{apass} \end{sous-entrée}

\end{entrée}

\begin{entrée}{nɤjlɤβsqa}{}{ⓔnɤjlɤβsqa} 
\classe{vt}  
\grammaire{incorp} \paradigme{dir}{kɤ-}
\begin{définition}\pfra{cuire}\end{définition}
\begin{définition}\pcmn{炖}\end{définition}
\begin{exemple}\pjya{tɯ-pu kɤ-nɤjlɤβsqa-t-a}\hspace{5pt}\pcmn{我炖血肠}\end{exemple}\relationsémantique{参考}{\lien{ⓔtɤjlɤβ}{tɤjlɤβ}}\relationsémantique{参考}{\lien{ⓔsqa}{sqa}}\end{entrée}

\begin{entrée}{nɤjmɤɣ}{}{ⓔnɤjmɤɣ} 
\classe{vi}  
\grammaire{denom} \paradigme{dir}{pɯ-}
\begin{définition}\pfra{chercher des champignons}\end{définition}
\begin{définition}\pcmn{找菌子}\end{définition}
\begin{exemple}\pjya{jisŋi kɯ-nɤjmɤɣ jɤ-ari-a}\hspace{5pt}\pcmn{我今天去找菌子了}\end{exemple}
\begin{exemple}\pjya{jisŋi aj pɯ-nɤjmaɣ-a}\hspace{5pt}\pcmn{我今天找菌子了}\end{exemple}
\begin{exemple}\pjya{aʑo ɕɯ-nɤjmaɣ-a ŋu, nɤʑo ɯ-tɯ́-ɣi?}\hspace{5pt}\pcmn{我要去找菌子,你来不来?}\end{exemple}
\begin{exemple}\pjya{ɯʑo pɯ-nɤjmɤɣ, aʑo kɤ-nɤjo-t-a}\hspace{5pt}\pcmn{他找了菌子,我等了他}\end{exemple}\relationsémantique{参考}{\lien{ⓔtɤjmɤɣ}{tɤjmɤɣ}}\end{entrée}

\begin{entrée}{nɤjmbɣom}{}{ⓔnɤjmbɣom} 
\classe{vs} \paradigme{dir}{tɤ-}
\begin{définition}\pfra{avoir le vertige}\end{définition}
\begin{définition}\pcmn{畏高}\end{définition}
\begin{exemple}\pjya{praʁku tɕe ɲɯ-nɤjmbɣom}\hspace{5pt}\pcmn{他在悬崖上畏高}\end{exemple}\relationsémantique{反义词}{\lien{ⓔnɤɣa}{nɤɣa}}\end{entrée}

\begin{entrée}{nɤjmbrɯma}{}{ⓔnɤjmbrɯma} 
\classe{n} 
\begin{définition}\pfra{bol}\end{définition}
\begin{définition}\pcmn{瓷碗,上面有青稞穗子的印}\end{définition}\étymologie{nas.ⁿbru.ma}\end{entrée}

\begin{entrée}{nɤjmŋozdɯɣ}{}{ⓔnɤjmŋozdɯɣ} 
\classe{vi} \paradigme{dir}{pɯ-}
\begin{définition}\pfra{beaucoup rêver, avoir un cauchemard}\end{définition}
\begin{définition}\pcmn{做很多梦,做噩梦}\end{définition}
\begin{exemple}\pjya{jɯfɕɯɕɤr ɯβrɤ-pɯ-tɯ-nɤjmŋozdɯɣ?}\hspace{5pt}\pcmn{你昨天没有做噩梦吧}\end{exemple}
\begin{exemple}\pjya{pɯ-nɤjmŋozdɯɣ-a ʑo ti thɯ́-wɣ-sɯsta-a}\hspace{5pt}\pcmn{他把我从梦中叫醒了}\end{exemple}\relationsémantique{参考}{\lien{ⓔtɤjmŋozdɯɣ}{tɤjmŋozdɯɣ}}\end{entrée}

\begin{entrée}{nɤjndɤt}{}{ⓔnɤjndɤt} 
\classe{vt} \sens{1}
\begin{définition}\pfra{trouver mignon, trouver adorable}\end{définition}
\begin{définition}\pcmn{觉得可爱}\end{définition}\sens{2}\paradigme{dire}{kɤ-}
\begin{définition}\pfra{taquiner}\end{définition}
\begin{définition}\pcmn{逗着玩}\end{définition}\relationsémantique{参考}{\lien{ⓔsɤjndɤt}{sɤjndɤt}}\relationsémantique{同义词}{\lien{ⓔsɤjndɤtⓝnɤsɤjndɤt}{nɤsɤjndɤt}}\end{entrée}

\begin{entrée}{nɤjno}{}{ⓔnɤjno} 
\classe{vi} \paradigme{dir}{\_}
\begin{définition}\pfra{galoper et ruer (cheval)}\end{définition}
\begin{définition}\pcmn{一边奔跑一边跳(马)}\end{définition}
\begin{exemple}\pjya{mbro jɤ-nɤjno ʑo jɤ-anɯri}\hspace{5pt}\pcmn{马奔跑了}\end{exemple}\relationsémantique{参考}{\lien{ⓔno}{no}}\end{entrée}

\begin{entrée}{nɤjo}{}{ⓔnɤjo} 
\classe{vt}
\classe{vt}  
\grammaire{refl}
\grammaire{caus} \paradigme{dir}{pɯ-}\paradigme{dir}{kɤ-}\sens{1}
\begin{définition}\pfra{attendre}\end{définition}
\begin{définition}\pcmn{等候}\end{définition}
\begin{exemple}\pjya{pɯ-nɤjo-t-a, kɤ-nɤjo-t-a}\hspace{5pt}\pcmn{我等了他}\end{exemple}
\begin{exemple}\pjya{pɯ-kɯ-nɤjo-a ɯ́-ŋu}\hspace{5pt}\pcmn{你有没有等我?}\end{exemple}
\begin{exemple}\pjya{tɕe nɯ ɣɯ-nɤjo ɲɯ-ra}\hspace{5pt}\pcmn{那件事情,需要等}\end{exemple}
\begin{exemple}\pjya{li tɯrme @dianhua kɤ-nɤjo-t-a pɯ-ra}\hspace{5pt}\pcmn{我又接了人家的电话}\end{exemple}
\begin{exemple}\pjya{pɯ-kɯ-nɤjo-a je!}\hspace{5pt}\pcmn{等一下我!}\end{exemple}\sens{2}
\begin{définition}\pfra{prendre (de l'eau, des grains etc)}\end{définition}
\begin{définition}\pcmn{接(液体、颗粒)}\end{définition}\sens{3}\paradigme{dir}{tɤ-}
\begin{définition}\pfra{passer par, faire l'expérience de}\end{définition}
\begin{définition}\pcmn{经历}\end{définition}
\begin{exemple}\pjya{ɯʑo kɯ kɯ-sɤscit ko-nɤjo ri, li kɯ-sɤɣdɯɣ kɯnɤ pjɤ-rɲo}\hspace{5pt}\pcmn{他经历过快乐的日子,也体验过困难的日子}\end{exemple}\relationsémantique{同义词}{\lien{ⓔrɲo}{rɲo}}
\begin{sous-entrée}{znɤjo}{ⓔnɤjoⓢ3ⓝznɤjo}\end{sous-entrée}

\sens{1}
\begin{définition}\pfra{prendre (de l'eau) avec}\end{définition}
\begin{définition}\pcmn{用……来接(液体)}\end{définition}
\begin{exemple}\pjya{tɤ-lu pɯ-tɕat-a tɕe, βʑɤzu kɯ tɤ-znɤjo-t-a}\hspace{5pt}\pcmn{我挤奶的时候用奶桶接住}\end{exemple}
\begin{exemple}\pjya{ki tɯthɯ kɯ tú-wɣ-znɤjo}\hspace{5pt}\pcmn{用这个锅子接水}\end{exemple}\sens{2}\paradigme{dir}{kɤ-}\paradigme{dir}{kɤ-}
\begin{définition}\pfra{ne pas pouvoir attendre}\end{définition}
\begin{définition}\pcmn{等不及(用否定式)}\end{définition}
\begin{définition}\pfra{attendre des gens}\end{définition}
\begin{définition}\pcmn{等别人}\end{définition}
\begin{exemple}\pjya{aʑo ɲɯ-ɕpaʁ-a, ju-zɣɯt mɯ́j-znɤjam-a tɕe tɯcɯrqɯ kɤ-tshi-t-a}\hspace{5pt}\pcmn{我很渴,等不及他的到来就喝了冷水(有人会送热水来)}\end{exemple}
\begin{exemple}\pjya{ma-kɤ-tɯ-sɤznɤjo}\hspace{5pt}\pcmn{你不要等别人了}\end{exemple}
\begin{sous-entrée}{sɤznɤjo/\variante{sɤnɤjo}}{ⓔnɤjoⓢ2ⓝsɤznɤjo} 
\classe{vi} \end{sous-entrée}

\begin{sous-entrée}{ʑɣɤznɤjo}{ⓔnɤjoⓢ2ⓝʑɣɤznɤjo} 
\classe{vi} \end{sous-entrée}

\begin{définition}\pfra{faire attendre les autres}\end{définition}
\begin{définition}\pcmn{令别人等}\end{définition}
\begin{exemple}\pjya{ma-kɤ-tɯ-ʑɣɤ-z-nɤjo}\hspace{5pt}\pcmn{你让别人等你}\end{exemple}
\begin{sous-entrée}{anɤjɯjo}{ⓔnɤjoⓝanɤjɯjo} 
\classe{vi} 
\begin{définition}\pfra{s'attendre l'un l'autre}\end{définition}
\begin{définition}\pcmn{互相等待}\end{définition}
\begin{exemple}\pjya{kɯ-mbɣom kɯ-dal anɤjɯjo}\hspace{5pt}\pcmn{急躁的和缓慢的到最后会一起到达}\end{exemple}\end{sous-entrée}

\end{entrée}

\begin{entrée}{nɤjoʁjoʁ}{}{ⓔnɤjoʁjoʁ} 
\classe{vt} \paradigme{dir}{tɤ-}
\begin{définition}\pfra{flatter}\end{définition}
\begin{définition}\pcmn{吹捧;奉承}\end{définition}\relationsémantique{参考}{\lien{ⓔjoʁ}{joʁ}}\end{entrée}

\begin{entrée}{nɤjtshɯ}{}{ⓔnɤjtshɯ} 
\classe{vs} \sens{1}\paradigme{dir}{kɤ-}
\begin{définition}\pfra{pouvoir être utilisé}\end{définition}
\begin{définition}\pcmn{可以用……(不用再找其它的)}\end{définition}
\begin{exemple}\pjya{a-ŋga jɤ-nɯ-ɣɯt-a tɕe, nɯ ma kɤ-χtɯ mɯ́j-ra ma ɲɯ-nɤjtshɯ}\hspace{5pt}\pcmn{我带了自己的衣服来,不用再买了,穿这件就行了}\end{exemple}
\begin{exemple}\pjya{nɤ-z-rɤrɤt ɯ-thɯ-arɕo nɤ, ki ɯ-taʁ pɯ-rɤt tɕe a-kɤ-nɤjtshɯ tɕe nɯ ma kɤ-χtɯ a-mɤ-pɯ-ra}\hspace{5pt}\pcmn{如果你纸用完了,写在这个上面来代替,不用再买了}\end{exemple}\sens{2}\paradigme{dir}{pɯ-}
\begin{définition}\pfra{être utile}\end{définition}
\begin{définition}\pcmn{有用}\end{définition}
\begin{exemple}\pjya{nɤj ʑa jɤ-tɯ-ari mɯ-pjɤ-nɤjtshɯ}\hspace{5pt}\pcmn{我早去没有用}\end{exemple}\sens{3}
\begin{définition}\pfra{être capable de se charger de}\end{définition}
\begin{définition}\pcmn{可以胜任}\end{définition}
\begin{exemple}\pjya{nɤʑo kha kɤ-nɤma tɯ-nɤjtshɯ}\hspace{5pt}\pcmn{家里的事情,你可以胜任(就不用我做了)}\end{exemple}\end{entrée}

\begin{entrée}{nɤjwaʁ}{}{ⓔnɤjwaʁ} 
\classe{vt}  
\grammaire{denom} \paradigme{dir}{thɯ-}
\begin{définition}\pfra{arracher les feuilles}\end{définition}
\begin{définition}\pcmn{扯叶子}\end{définition}
\begin{exemple}\pjya{ki si ki ɯ-rtaʁ thɯ-nɤjwaʁ-a}\hspace{5pt}\pcmn{我把这棵树上的树枝扯下了叶子}\end{exemple}
\begin{exemple}\pjya{@wosun thɯ-nɤjwaʁ}\hspace{5pt}\pcmn{你拔莴笋的叶子吧}\end{exemple}
\begin{exemple}\pjya{ɯʑo kɯ @wosun tha-nɤjwaʁ}\hspace{5pt}\pcmn{他把莴笋的叶子拔了}\end{exemple}\relationsémantique{参考}{\lien{ⓔtɤ-jwaʁ}{tɤ-jwaʁ}}\end{entrée}

\begin{entrée}{nɤɟɤβlɤβ}{}{ⓔnɤɟɤβlɤβ} 
\classe{vt} \paradigme{dir}{tɤ-}
\begin{définition}\pfra{chercher à plaire}\end{définition}
\begin{définition}\pcmn{讨好,拉拢}\end{définition}
\begin{exemple}\pjya{ma-tɤ-kɯ-nɤɟɤβlaβ-a ma aʑo mɤ-khɯ-a}\hspace{5pt}\pcmn{你要讨好我,我不会答应的}\end{exemple}\end{entrée}

\begin{entrée}{nɤkɤphɤr}{}{ⓔnɤkɤphɤr} 
\classe{vt}  
\grammaire{incorp} \paradigme{dir}{pɯ-}\paradigme{dir}{kɤ-}
\begin{définition}\pfra{frapper avec le fléau sur les épis}\end{définition}
\begin{définition}\pcmn{对着青稞穗打连枷}\end{définition}
\begin{exemple}\pjya{ɯʑo kɯ tɤɕi pa-nɤkɤphɤr}\hspace{5pt}\pcmn{他用连枷打了青稞穗}\end{exemple}\relationsémantique{参考}{\lien{ⓔsɤphɤr}{sɤphɤr}}\relationsémantique{参考}{\lien{ⓔtɯ-ku}{tɯ-ku}}\end{entrée}

\begin{entrée}{nɤkɤro}{}{ⓔnɤkɤro} 
\classe{vs} \paradigme{dir}{tɤ-}\paradigme{dir}{tɤ-}
\begin{définition}\pfra{acceptable}\end{définition}
\begin{définition}\pcmn{还可以}\end{définition}
\begin{définition}\pfra{trouver acceptable}\end{définition}
\begin{définition}\pcmn{觉得还可以}\end{définition}
\begin{exemple}\pjya{ɲɯ-nɤkɤro, zdɯxthɯɣ ɲɯ-ŋu}\hspace{5pt}\pcmn{还可以,勉强可以}\end{exemple}
\begin{exemple}\pjya{mɯ́j-mŋɤm, ɲɯ-nɤkɤro ɲɯ-khɯ}\hspace{5pt}\pcmn{不痛,还可以}\end{exemple}
\begin{exemple}\pjya{ɯ-tɯ-rɤt nɯ to-znɤkɤro ɕti}\hspace{5pt}\pcmn{他觉得他写得还可以}\end{exemple}
\begin{sous-entrée}{kɯ-nɤkɤro}{ⓔnɤkɤroⓝkɯ-nɤkɤro}
\begin{définition}\pfra{un certain temps}\end{définition}
\begin{définition}\pcmn{好一段时间}\end{définition}
\begin{exemple}\pjya{kɯ-nɤkɤro ʑo pɯ-ta-nɤjo}\hspace{5pt}\pcmn{我等了你好一阵子}\end{exemple}\end{sous-entrée}

\begin{sous-entrée}{znɤkɤro}{ⓔnɤkɤroⓝznɤkɤro} 
\classe{vt}  
\grammaire{caus} \end{sous-entrée}

\end{entrée}

\begin{entrée}{nɤkɤtɕhɯ}{}{ⓔnɤkɤtɕhɯ} 
\classe{vt}  
\grammaire{incorp} \paradigme{dir}{kɤ-}\paradigme{dir}{tɤ-}
\begin{définition}\pfra{donner un coup de tête}\end{définition}
\begin{définition}\pcmn{用头顶}\end{définition}
\begin{exemple}\pjya{kɤ́-wɣ-nɤkɤtɕhɯ-a}\hspace{5pt}\pcmn{他用头顶了我}\end{exemple}\relationsémantique{参考}{\lien{ⓔtɯ-ku}{tɯ-ku}}\relationsémantique{参考}{\lien{ⓔtɕhɯ}{tɕhɯ}}\relationsémantique{参考}{\lien{ⓔkɤtɕhɯ}{kɤtɕhɯ}}\end{entrée}

\begin{entrée}{nɤkhu}{}{ⓔnɤkhu} 
\classe{vt}  
\grammaire{appl} \paradigme{dir}{nɯ-}\paradigme{dir}{kɤ-}
\begin{définition}\pfra{inviter}\end{définition}
\begin{définition}\pcmn{请客}\end{définition}
\begin{exemple}\pjya{tɯlu ra kɯ aʑo nɯ́-wɣ-nɤkhu-a-nɯ}\hspace{5pt}\pcmn{德鲁那家人请了我}\end{exemple}
\begin{exemple}\pjya{aʑo tɯlu ra kɤ-nɤkhu-t-a-nɯ}\hspace{5pt}\pcmn{我请了德鲁那家人}\end{exemple}
\begin{exemple}\pjya{aʑo nɯ-kha ɯ-kɯ-nɤkhu jɤ-ari-a}\hspace{5pt}\pcmn{我去请他们了}\end{exemple}
\begin{exemple}\pjya{nɯ-kha nɯtɕu kɤ-nɤkhu jɤ-ari-a}\hspace{5pt}\pcmn{我去他们家坐客}\end{exemple}
\begin{sous-entrée}{sɤnɤkhu}{ⓔnɤkhuⓝsɤnɤkhu} 
\classe{vi}  
\grammaire{apass} 
\begin{définition}\pfra{inviter des gens}\end{définition}
\begin{définition}\pcmn{请客}\end{définition}\relationsémantique{参考}{\lien{ⓔakhu}{akhu}}\end{sous-entrée}

\end{entrée}

\begin{entrée}{nɤkhar}{}{ⓔnɤkhar} 
\classe{vt} \paradigme{dir}{pɯ-}\paradigme{dir}{tɤ-}
\begin{définition}\pfra{entourer}\end{définition}
\begin{définition}\pcmn{包围;围着坐}\end{définition}
\begin{exemple}\pjya{@zhuozi pɯ-nɤkhar-i}\hspace{5pt}\pcmn{我们把桌子围起来了}\end{exemple}
\begin{exemple}\pjya{@zhuozi pɯ-nɤkhar-i tɕe tɤ-rɯndzɤtshi-j}\hspace{5pt}\pcmn{我们围着桌子坐了就吃饭了}\end{exemple}
\begin{exemple}\pjya{kɤndza pɯ-nɤkhar-i}\hspace{5pt}\pcmn{我们围着食物坐了}\end{exemple}
\begin{exemple}\pjya{paʁ kɤ-nɤkhar-i}\hspace{5pt}\pcmn{我们把猪围住了}\end{exemple}\end{entrée}

\begin{entrée}{nɤkhɤzŋga}{}{ⓔnɤkhɤzŋga} 
\classe{vt}  
\grammaire{appl} \paradigme{dir}{nɯ-}
\begin{définition}\pfra{appeler}\end{définition}
\begin{définition}\pcmn{喊;叫}\end{définition}
\begin{exemple}\pjya{nɯ-nɤkhɤzŋga-t-a}\hspace{5pt}\pcmn{我喊了他一声}\end{exemple}
\begin{exemple}\pjya{aʑo kɯ tɯrme ɲɯ-nɤkhɤzŋge-a}\hspace{5pt}\pcmn{我在叫人}\end{exemple}\relationsémantique{参考}{\lien{ⓔakhu}{akhu}}
\begin{sous-entrée}{anɤkhɤzŋgɯzŋga}{ⓔnɤkhɤzŋgaⓝanɤkhɤzŋgɯzŋga} 
\classe{vi}  
\grammaire{recip} 
\begin{définition}\pfra{s'appeler les uns les autres}\end{définition}
\begin{définition}\pcmn{互相喊叫}\end{définition}\end{sous-entrée}

\begin{sous-entrée}{znɤkhɤzŋga}{ⓔnɤkhɤzŋgaⓝznɤkhɤzŋga} 
\classe{vt}  
\grammaire{caus} 
\begin{définition}\pfra{faire appeler}\end{définition}
\begin{définition}\pcmn{请……喊叫}\end{définition}\end{sous-entrée}

\end{entrée}

\begin{entrée}{nɤkhe}{}{ⓔnɤkhe} 
\classe{vt} \paradigme{dir}{tɤ-}\paradigme{dir}{kɤ-}
\begin{définition}\pfra{maltraiter}\end{définition}
\begin{définition}\pcmn{欺负}\end{définition}
\begin{exemple}\pjya{jiɕqha kɯ nɯ ɲɯ-ti tú-wɣ-nɤkhe-a}\hspace{5pt}\pcmn{这个人说了那一句话,把我欺负了}\end{exemple}
\begin{exemple}\pjya{ɯʑo kɯ tú-wɣ-nɤkhe-a ɲɯ-ŋu}\hspace{5pt}\pcmn{他欺负我}\end{exemple}
\begin{exemple}\pjya{ma-tɤ-kɯ-nɤkhe-a}\hspace{5pt}\pcmn{你不要欺负我}\end{exemple}
\begin{exemple}\pjya{ko-nɤkhe}\hspace{5pt}\pcmn{他强奸了她}\end{exemple}
\begin{sous-entrée}{sɤnɤkhe}{ⓔnɤkheⓝsɤnɤkhe} 
\classe{vi} 
\begin{définition}\pfra{maltraiter les autres}\end{définition}
\begin{définition}\pcmn{欺负人}\end{définition}
\begin{exemple}\pjya{jiɕqha tɯrme kɯ-sɤnɤkhe ci ɲɯ-ŋu}\hspace{5pt}\pcmn{那个人总是欺负别人的}\end{exemple}\end{sous-entrée}

\end{entrée}

\begin{entrée}{nɤkhɯ}{}{ⓔnɤkhɯ} 
\classe{vi}  
\grammaire{denom} \paradigme{dir}{tɤ-}\paradigme{dir}{tɤ-}
\begin{définition}\pfra{être enfumé}\end{définition}
\begin{définition}\pcmn{被熏到(人)}\end{définition}
\begin{définition}\pfra{enfumer}\end{définition}
\begin{définition}\pcmn{熏到}\end{définition}
\begin{exemple}\pjya{ɲɯ-nɤkhɯ-a}\hspace{5pt}\pcmn{我被熏到}\end{exemple}
\begin{exemple}\pjya{smi pjɯ́-wɣ-βlɯ tɕe, tu-kɯ-znɤkhɯ ɲɯ-ŋu}\hspace{5pt}\pcmn{火烧了,(烟)把人熏到}\end{exemple}\relationsémantique{参考}{\lien{ⓔtɤ-khɯ}{tɤ-khɯ}}\relationsémantique{参考}{\lien{ⓔɣɤkhɯⓗ1}{ɣɤkhɯ₁}}\relationsémantique{参考}{\lien{ⓔsɤkhɯ}{sɤkhɯ}}
\begin{sous-entrée}{znɤkhɯ}{ⓔnɤkhɯⓝznɤkhɯ} 
\classe{vt}  
\grammaire{caus} \end{sous-entrée}

\end{entrée}

\begin{entrée}{nɤkhɯkhrɯt}{}{ⓔnɤkhɯkhrɯt} 
\classe{vt} \paradigme{dir}{\_}
\begin{définition}\pfra{traîner}\end{définition}
\begin{définition}\pcmn{拖}\end{définition}
\begin{exemple}\pjya{thɯ-nɤkhɯkhrɯt-a}\hspace{5pt}\pcmn{我拖了}\end{exemple}
\begin{exemple}\pjya{ɕoŋtɕa tha-nɤkhɯkhrɯt}\hspace{5pt}\pcmn{他拖了木料}\end{exemple}
\begin{exemple}\pjya{laχtɕha tha-nɤkhɯkhrɯt}\hspace{5pt}\pcmn{他拖了东西}\end{exemple}
\begin{exemple}\pjya{nɤki nɯ kɤ-fkɯr mɯ-mɤ-ɲɯ-tɯ-cha nɤ, nɯ-nɤkhɯkhrɯt}\hspace{5pt}\pcmn{如果你不能背,你就拖吧}\end{exemple}\relationsémantique{同义词}{\lien{ⓔnɤɕɯɕi}{nɤɕɯɕi}}\end{entrée}

\begin{entrée}{nɤkɯka}{}{ⓔnɤkɯka} 
\classe{vt} \paradigme{dir}{nɯ-}
\begin{définition}\pfra{mâcher}\end{définition}
\begin{définition}\pcmn{咀嚼}\end{définition}
\begin{exemple}\pjya{nɯ-nɤkɯka-t-a}\hspace{5pt}\pcmn{我咀嚼了}\end{exemple}
\begin{exemple}\pjya{stoʁ nɯ mɤ-kɤ-nɤkɯka kɤ-mqlaʁ mɤ-khɯ}\hspace{5pt}\pcmn{胡豆要先咀嚼才可以吞}\end{exemple}\end{entrée}

\begin{entrée}{nɤkɯkro}{}{ⓔnɤkɯkro}\relationsémantique{参考}{\lien{ⓔkro}{kro}}\end{entrée}

\begin{entrée}{nɤkɯt}{}{ⓔnɤkɯt} 
\classe{vi} \paradigme{dir}{\_}
\begin{définition}\pfra{être acculé}\end{définition}
\begin{définition}\pcmn{没有退路}\end{définition}
\begin{sous-entrée}{znɤkɯt}{ⓔnɤkɯtⓝznɤkɯt} 
\classe{vt} 
\begin{définition}\pfra{acculer}\end{définition}
\begin{définition}\pcmn{逼上绝路}\end{définition}
\begin{exemple}\pjya{lɤ-kɯ-znɤkɯt-a}\hspace{5pt}\pcmn{你把我逼上绝路了}\end{exemple}\end{sous-entrée}

\end{entrée}

\begin{entrée}{nɤlu}{}{ⓔnɤlu} 
\classe{vt}  
\grammaire{denom} \paradigme{dir}{pɯ-}
\begin{définition}\pfra{traire}\end{définition}
\begin{définition}\pcmn{挤奶}\end{définition}
\begin{exemple}\pjya{nɯŋa pɯ-tɯ-nɤlu-t}\hspace{5pt}\pcmn{你挤了(奶牛的)奶}\end{exemple}
\begin{exemple}\pjya{pɯ-nɤle (= tɤ-lu pɯ-tɕɤt)}\hspace{5pt}\pcmn{你挤奶吧}\end{exemple}\relationsémantique{参考}{\lien{ⓔtɤ-lu}{tɤ-lu}}\end{entrée}

\begin{entrée}{nɤla}{}{ⓔnɤla} 
\classe{vt} \paradigme{dir}{tɤ-}
\begin{définition}\pfra{acquiescer, être d’accord}\end{définition}
\begin{définition}\pcmn{同意;允许;答应}\end{définition}
\begin{exemple}\pjya{tɤ-nɤla-t-a}\hspace{5pt}\pcmn{我答应了}\end{exemple}
\begin{exemple}\pjya{aʑo a-kɤ-ti nɯ ta-nɤla}\hspace{5pt}\pcmn{他答应了我所说的话}\end{exemple}
\begin{exemple}\pjya{aʑo a-kɤ-ti nɯ tɕhindʐa mɤ-tɯ-nɤle}\hspace{5pt}\pcmn{你怎么不答应我说的话?}\end{exemple}
\begin{exemple}\pjya{mɯ́j-nɤle}\hspace{5pt}\pcmn{他不答应,他拒绝}\end{exemple}
\begin{exemple}\pjya{mɯ-ɕɯ-tɯ-nɤle nɯ-sɯso-t-a}\hspace{5pt}\pcmn{我怕你不答应}\end{exemple}\end{entrée}

\begin{entrée}{nɤldaʁldaʁ}{}{ⓔnɤldaʁldaʁ} 
\classe{vt} \paradigme{dir}{nɯ-}
\begin{définition}\pfra{accueillir chaleureusement}\end{définition}
\begin{définition}\pcmn{热情款待}\end{définition}\end{entrée}

\begin{entrée}{nɤliɤliɤt}{}{ⓔnɤliɤliɤt} 
\classe{vt} \paradigme{dir}{tɤ-}
\begin{définition}\pfra{faire la fête à qqn (chien)}\end{définition}
\begin{définition}\pcmn{摆尾巴、热烈地扑上(狗)}\end{définition}
\begin{exemple}\pjya{khɯna kɯ tɤ́-wɣ-nɤliɤliɤt-a}\hspace{5pt}\pcmn{狗朝我摇尾巴}\end{exemple}\end{entrée}

\begin{entrée}{nɤma}{}{ⓔnɤma} 
\classe{vt}  
\grammaire{denom} \paradigme{dir}{tɤ-}
\begin{définition}\pfra{travailler}\end{définition}
\begin{définition}\pcmn{干活}\end{définition}
\begin{exemple}\pjya{tɤ-nɤma-t-a}\hspace{5pt}\pcmn{我干活了}\end{exemple}
\begin{exemple}\pjya{pɤjkhu mɯ-tɤ-nɤma-t-a}\hspace{5pt}\pcmn{我还没有干活}\end{exemple}
\begin{exemple}\pjya{a-sɯm mɤ-kɯ-ɕe khro ʑo nɤme-a ra}\hspace{5pt}\pcmn{我要做许多不想做的工作}\end{exemple}
\begin{exemple}\pjya{tɕhi ku-tɯ-nɤme?}\hspace{5pt}\pcmn{你在做什么?}\end{exemple}
\begin{exemple}\pjya{spɯ-spe-a nɯ tɤ-nɤma-t-a}\hspace{5pt}\pcmn{我会做的都做了}\end{exemple}
\begin{sous-entrée}{anɤma}{ⓔnɤmaⓝanɤma} 
\classe{vi} 
\begin{définition}\pfra{être fait}\end{définition}
\begin{définition}\pcmn{做好(事情)}\end{définition}
\begin{exemple}\pjya{pɤjkhu mɤ-anɤma}\hspace{5pt}\pcmn{事情还没有做好}\end{exemple}\end{sous-entrée}

\begin{sous-entrée}{znɤma}{ⓔnɤmaⓝznɤma} 
\grammaire{habil} 
\begin{définition}\pfra{pouvoir faire}\end{définition}
\begin{définition}\pcmn{能做;做得了}\end{définition}
\begin{exemple}\pjya{aʑo nɯ thamtɕɤt mɯ́j-znɤme-a}\hspace{5pt}\pcmn{我做不了那么多事情}\end{exemple}\end{sous-entrée}

\begin{sous-entrée}{nɯɣɯnɤma}{ⓔnɤmaⓝnɯɣɯnɤma} 
\classe{vs} 
\begin{définition}\pfra{facile à faire}\end{définition}
\begin{définition}\pcmn{容易办}\end{définition}\relationsémantique{参考}{\lien{ⓔta-ma}{ta-ma}}\relationsémantique{参考}{\lien{ⓔrɤma}{rɤma}}\end{sous-entrée}

\end{entrée}

\begin{entrée}{nɤmar}{}{ⓔnɤmar} 
\classe{vs} \paradigme{dir}{tɤ-}
\begin{définition}\pfra{gras (surface)}\end{définition}
\begin{définition}\pcmn{油油的}\end{définition}
\begin{exemple}\pjya{ɯ-jaʁ ɲɯ-nɤmar}\hspace{5pt}\pcmn{他的手很油}\end{exemple}\relationsémantique{参考}{\lien{ⓔta-mar}{ta-mar}}\end{entrée}

\begin{entrée}{nɤmɤla}{}{ⓔnɤmɤla}\relationsémantique{参考}{\lien{ⓔnɤmɤle}{nɤmɤle}}\end{entrée}

\begin{entrée}{nɤmɤle/\variante{nɤmɤla}}{}{ⓔnɤmɤle} 
\classe{vt} \sens{1}\paradigme{dir}{pɯ-}
\begin{définition}\pfra{toucher}\end{définition}
\begin{définition}\pcmn{摸;弄}\end{définition}
\begin{exemple}\pjya{pɯ-nɤmɤle-t-a}\hspace{5pt}\pcmn{我摸了}\end{exemple}
\begin{exemple}\pjya{kɯki laχtɕha tɯ-nɤmɤle mɤ-ra}\hspace{5pt}\pcmn{你不要摸这个东西}\end{exemple}
\begin{exemple}\pjya{ki ɯ-phɯ ɲɯ-wxti tɕe, ma-pɯ-tɯ-ɣɤrɤt ma a-mɤ-pɯ-ɴɢrɯ ma ma-tɯ-nɤmɤle}\hspace{5pt}\pcmn{这个东西很贵重,你不要让它摔下来,不要让它破了,不要乱摸}\end{exemple}\sens{2}\paradigme{dir}{tɤ-}
\begin{définition}\pfra{faire une tâche}\end{définition}
\begin{définition}\pcmn{做一个工作(完成整个制作过程)}\end{définition}
\begin{exemple}\pjya{ɯʑo kɯ tɤ-lu tu-nɤmɤle ŋgrɤl}\hspace{5pt}\pcmn{是他平时负责牛奶的工作}\end{exemple}
\begin{exemple}\pjya{ɯʑo kɯ kha tu-nɤmɤle ŋgrɤl}\hspace{5pt}\pcmn{是她平时做家务}\end{exemple}\end{entrée}

\begin{entrée}{nɤmbɤβ}{}{ⓔnɤmbɤβ} 
\classe{vs} \paradigme{dir}{tɤ-}
\begin{définition}\pfra{se gonfler (corps)}\end{définition}
\begin{définition}\pcmn{胀(身体)}\end{définition}
\begin{exemple}\pjya{ɯ-xtu nɤmbɤβ}\hspace{5pt}\pcmn{(他喝了这么多水),肚子(可能)会胀的}\end{exemple}
\begin{exemple}\pjya{pjɤ-si tɕe to-nɤmbɤβ xtaŋxtaŋ ʑo}\hspace{5pt}\pcmn{死了(身体)就浮肿了}\end{exemple}
\begin{exemple}\pjya{nɯɣmbɤβ}\end{exemple}\relationsémantique{参考}{\lien{ⓔtɯ-ɣmbɤβ}{tɯ-ɣmbɤβ}}\end{entrée}

\begin{entrée}{nɤmbɤndɤr}{}{ⓔnɤmbɤndɤr} 
\classe{vt} 
\begin{définition}\pfra{aider à marcher, s'occuper de (malade, persone saoûle)}\end{définition}
\begin{définition}\pcmn{搀扶和照顾(病人,喝醉的人)}\end{définition}
\begin{exemple}\pjya{ɯʑo lo-βzi tɕe tɤ-nɤmbɤndɤr-i pɯ-ra}\hspace{5pt}\pcmn{他醉了,我们只好搀扶他}\end{exemple}\end{entrée}

\begin{entrée}{nɤmbɣaʁlaʁ}{}{ⓔnɤmbɣaʁlaʁ} 
\classe{vi}  
\grammaire{n.orient} \paradigme{dir}{\_}
\begin{définition}\pfra{se tourner et se retourner (à droite puis à gauche)}\end{définition}
\begin{définition}\pcmn{(左右)打滚,辗转}\end{définition}
\begin{exemple}\pjya{mbro ɲɯ-nɤmbɣaʁlaʁ}\hspace{5pt}\pcmn{马在打滚}\end{exemple}
\begin{exemple}\pjya{laʁtɕha ɯ-thoʁ nɯ fse ɲɯ-nɤmbɣaʁlaʁ}\hspace{5pt}\pcmn{东西在地上乱七八糟}\end{exemple}
\begin{exemple}\pjya{ma-nɯ-tɯ-nɤmbɣaʁlaʁ}\hspace{5pt}\pcmn{你不要打滚}\end{exemple}\relationsémantique{参考}{\lien{ⓔmbɣaʁ}{mbɣaʁ}}\end{entrée}

\begin{entrée}{nɤmbju}{}{ⓔnɤmbju} 
\classe{vs} \paradigme{dir}{tɤ-}\sens{1}
\begin{définition}\pfra{éblouissant}\end{définition}
\begin{définition}\pcmn{耀眼}\end{définition}
\begin{exemple}\pjya{tɤŋe ɲɯ-nɤmbju}\hspace{5pt}\pcmn{太阳很耀眼}\end{exemple}\sens{2}
\begin{définition}\pfra{brillant}\end{définition}
\begin{définition}\pcmn{闪光,发光}\end{définition}
\begin{exemple}\pjya{tɤtʂu nɤmbju}\hspace{5pt}\pcmn{灯发光}\end{exemple}
\begin{exemple}\pjya{ɕɤr tɕe sɯtɤpɯz nɤmbju}\hspace{5pt}\pcmn{晚上朽木会发光}\end{exemple}
\begin{sous-entrée}{znɤmbju}{ⓔnɤmbjuⓢ2ⓝznɤmbju} 
\classe{vt} 
\begin{définition}\pfra{éblouir}\end{définition}
\begin{définition}\pcmn{令……眼花}\end{définition}
\begin{exemple}\pjya{tɤŋe kɯ a-mɲaʁ ɲɯ-znɤmbju}\hspace{5pt}\pcmn{太阳令我眼花}\end{exemple}\end{sous-entrée}

\end{entrée}

\begin{entrée}{nɤmbrɯ}{}{ⓔnɤmbrɯ} 
\classe{vt} \paradigme{dir}{tɤ-}
\begin{définition}\pfra{s'énerver contre}\end{définition}
\begin{définition}\pcmn{生……的气}\end{définition}
\begin{exemple}\pjya{ma-tɤ-kɯ-nɤmbrɯ-a}\hspace{5pt}\pcmn{你不要生我的气}\end{exemple}\relationsémantique{参考}{\lien{ⓔsɤmbrɯ}{sɤmbrɯ}}\end{entrée}

\begin{entrée}{nɤmbrɯm}{}{ⓔnɤmbrɯm} 
\classe{vi} \paradigme{dir}{tɤ-}
\begin{définition}\pfra{attraper la variole}\end{définition}
\begin{définition}\pcmn{得麻子}\end{définition}
\begin{exemple}\pjya{tɯ-mɲɯtsi tɯ-ɣjɤn a-tɤ-kɯ-nɤmbrɯm qhe, tɕe ɯ-qhu mɯ́j-kɯ-nɤmbrɯm tɕe tu-kɯ-lo ɲɯ-ŋu.}\hspace{5pt}\pcmn{一辈子得麻子就终身不再得这种病,有免疫能力}\end{exemple}\end{entrée}

\begin{entrée}{nɤmbrɯmtɕɤz}{}{ⓔnɤmbrɯmtɕɤz} 
\classe{vs} 
\begin{définition}\pfra{grêlé}\end{définition}
\begin{définition}\pcmn{麻子}\end{définition}
\begin{exemple}\pjya{jiɕqha nɯ ɯ-rŋa ɲɯ-nɤmbrɯmtɕɤz}\hspace{5pt}\pcmn{那个人的满脸是麻子}\end{exemple}\étymologie{ⁿbrum}\end{entrée}

\begin{entrée}{nɤmda}{}{ⓔnɤmda} 
\classe{vt}  
\grammaire{trop} \paradigme{dir}{tɤ-}\paradigme{dir}{pɯ-}
\begin{définition}\pfra{sentir que le moment est arrivé}\end{définition}
\begin{définition}\pcmn{觉得时间到了}\end{définition}
\begin{exemple}\pjya{saχsɯ tɤ-nɤmda-t-a}\hspace{5pt}\pcmn{我觉得中午餐的时间到了(肚子很饿)}\end{exemple}
\begin{exemple}\pjya{saχsɯ jisŋi mɯ-pɯ-nɤmda-t-a}\hspace{5pt}\pcmn{今天我没有发觉中午餐的时间到了}\end{exemple}
\begin{exemple}\pjya{kɤ-nɯɕe tɤ-nɤmda-t-a}\hspace{5pt}\pcmn{我觉得回去的时间到了}\end{exemple}
\begin{exemple}\pjya{kɤ-nɯɕe ɲɯ-nɤmde-a}\hspace{5pt}\pcmn{我现在觉得回去的时间到了}\end{exemple}
\begin{exemple}\pjya{aʑo ɲɯ-ɕe-a ra ri pɤjkhu mɯ́j-nɤmde-a}\hspace{5pt}\pcmn{我要回去,但是时间还没有到}\end{exemple}\relationsémantique{参考}{\lien{ⓔmda}{mda}}\end{entrée}

\begin{entrée}{nɤmdɯmdar}{}{ⓔnɤmdɯmdar}\relationsémantique{参考}{\lien{ⓔnɯmdar}{nɯmdar}}\end{entrée}

\begin{entrée}{nɤmdzɯ}{}{ⓔnɤmdzɯ} 
\classe{vt}  
\grammaire{appl} \paradigme{dir}{kɤ-}
\begin{définition}\pfra{garder}\end{définition}
\begin{définition}\pcmn{看管;看守}\end{définition}
\begin{exemple}\pjya{ki smi ki kɤ-nɤmdzi}\hspace{5pt}\pcmn{请你注意这个火}\end{exemple}
\begin{exemple}\pjya{ki tɤ-pɤtso kɤ-nɤmdzi a-mɤ-pɯ-ndʐaβ}\hspace{5pt}\pcmn{你看着这个孩子,不要让他摔跤}\end{exemple}\relationsémantique{参考}{\lien{ⓔamdzɯ}{amdzɯ}}\relationsémantique{同义词}{\lien{ⓔrɯru}{rɯru}}\end{entrée}

\begin{entrée}{nɤme}{}{ⓔnɤme} 
\classe{vt} \paradigme{dir}{tɤ-}
\begin{définition}\pfra{adopter (une fille)}\end{définition}
\begin{définition}\pcmn{领养(女儿)}\end{définition}
\begin{exemple}\pjya{tɯrme ɯ-me to-nɤme}\hspace{5pt}\pcmn{他们领养别人的女儿}\end{exemple}\relationsémantique{同义词}{\lien{ⓔnɤtɕɯ}{nɤtɕɯ}}\relationsémantique{参考}{\lien{ⓔtɯ-me}{tɯ-me}}\end{entrée}

\begin{entrée}{nɤmuj}{}{ⓔnɤmuj} 
\classe{vt} \paradigme{dir}{thɯ-}
\begin{définition}\pfra{enlever les plumes}\end{définition}
\begin{définition}\pcmn{拔羽毛}\end{définition}
\begin{exemple}\pjya{pɣa thɯ-nɤmuj-a}\hspace{5pt}\pcmn{我拔了鸡的羽毛}\end{exemple}\relationsémantique{参考}{\lien{ⓔtɤ-muj}{tɤ-muj}}\end{entrée}

\begin{entrée}{nɤmujmaj}{}{ⓔnɤmujmaj} 
\classe{vi}  
\grammaire{denom} \paradigme{dir}{thɯ-}
\begin{définition}\pfra{élaguer}\end{définition}
\begin{définition}\pcmn{修剪(树枝)}\end{définition}\relationsémantique{参考}{\lien{ⓔɯ-mujmaj}{ɯ-mujmaj}}\end{entrée}

\begin{entrée}{nɤmkha}{}{ⓔnɤmkha} 
\classe{n} 
\begin{définition}\pfra{ciel}\end{définition}
\begin{définition}\pcmn{天}\end{définition}\relationsémantique{同义词}{\lien{ⓔtɯ-mɯ}{tɯ-mɯ}}\étymologie{nam.mkʰa}\end{entrée}

\begin{entrée}{nɤmkɯm}{}{ⓔnɤmkɯm} 
\classe{vt}  
\grammaire{denom} \paradigme{dir}{thɯ-}\paradigme{dir}{kɤ-}
\begin{définition}\pfra{utiliser ... comme un oreiller}\end{définition}
\begin{définition}\pcmn{枕(躺着的时候把头放在东西上)}\end{définition}
\begin{exemple}\pjya{tɯ-ŋga nɯ ka-nɤmkɯm}\hspace{5pt}\pcmn{他枕了衣服睡}\end{exemple}\end{entrée}

\begin{entrée}{nɤmnɤm}{}{ⓔnɤmnɤm} 
\classe{vt}  
\grammaire{trop} \paradigme{dir}{tɤ-}
\begin{définition}\pfra{sentir}\end{définition}
\begin{définition}\pcmn{闻嗅(故意)}\end{définition}
\begin{exemple}\pjya{khɯna kɯ ta-nɤmnɤm}\hspace{5pt}\pcmn{狗闻了一下}\end{exemple}
\begin{exemple}\pjya{ki tɤ-nɤmnɤm}\hspace{5pt}\pcmn{你闻一下这个}\end{exemple}
\begin{exemple}\pjya{tɤ-mthɯm aj tɤ-nɤmnam-a}\hspace{5pt}\pcmn{我闻了一下肉}\end{exemple}
\begin{exemple}\pjya{tɤ-mthɯm ɯ-ɲɤ́-ɣɤdi kɯ tɤ-nɤmnam-a}\hspace{5pt}\pcmn{我闻了一下肉有没有变味}\end{exemple}
\begin{exemple}\pjya{ɯβrɤ-ɲɯ-ɣɤdi tɤ-nɤmnam-a}\hspace{5pt}\pcmn{我闻了一下有没有变味}\end{exemple}\relationsémantique{参考}{\lien{ⓔmnɤm}{mnɤm}}\relationsémantique{参考}{\lien{ⓔɕɯmnɤm}{ɕɯmnɤm}}\étymologie{mnam}\end{entrée}

\begin{entrée}{nɤmɲo}{}{ⓔnɤmɲo} 
\classe{vl} \paradigme{dir}{kɤ-}
\begin{définition}\pfra{regarder}\end{définition}
\begin{définition}\pcmn{观看}\end{définition}
\begin{exemple}\pjya{@dianshi ma-kɤ-tɯ-nɤmɲɤm}\hspace{5pt}\pcmn{你不要看电视}\end{exemple}\relationsémantique{参考}{\lien{ⓔnɤmɲole}{nɤmɲole}}\relationsémantique{参考}{\lien{ⓔsɤmɲo}{sɤmɲo}}\end{entrée}

\begin{entrée}{nɤmɲole}{}{ⓔnɤmɲole} 
\classe{vi} \paradigme{dir}{nɯ-}
\begin{définition}\pfra{regarder le paysage}\end{définition}
\begin{définition}\pcmn{观看风景;到处观看}\end{définition}
\begin{exemple}\pjya{@Chengdu kɯra nɯ-nɤmɲole}\hspace{5pt}\pcmn{你观看成都(的风景)吧}\end{exemple}
\begin{exemple}\pjya{kɯra aʁɤndɯndɤt nɯ-nɤmɲole}\hspace{5pt}\pcmn{到处观看风景吧!}\end{exemple}
\begin{exemple}\pjya{nɯ-nɤmɲole-a}\hspace{5pt}\pcmn{我观看了}\end{exemple}
\begin{sous-entrée}{znɤmɲole}{ⓔnɤmɲoleⓝznɤmɲole} 
\classe{vt} 
\begin{définition}\pfra{faire regarder le paysage}\end{définition}
\begin{définition}\pcmn{带……到处观看}\end{définition}\relationsémantique{参考}{\lien{ⓔnɤmɲo}{nɤmɲo}}\end{sous-entrée}

\end{entrée}

\begin{entrée}{nɤmŋɤm}{}{ⓔnɤmŋɤm}\relationsémantique{参考}{\lien{ⓔmŋɤm}{mŋɤm}}\end{entrée}

\begin{entrée}{nɤmŋɯn}{}{ⓔnɤmŋɯn} 
\classe{vt}  
\grammaire{trop}
\grammaire{trop} \paradigme{dir}{pɯ-}
\begin{définition}\pfra{être reconnaissant}\end{définition}
\begin{définition}\pcmn{感激;满意}\end{définition}
\begin{exemple}\pjya{pɯ-ta-nɤmŋɯn}\hspace{5pt}\pcmn{我令你满意了}\end{exemple}
\begin{exemple}\pjya{jiɕqha nɯ kɯ laχtɕha ci nɯ́-wɣ-mbi-a rcanɯ pɯ-nɤmŋɯn-a}\hspace{5pt}\pcmn{我很感激这个人给我这个东西}\end{exemple}
\begin{exemple}\pjya{jiɕqha nɯ kɯ cha ci nɯ́-wɣ-jtshi-a rcanɯ ci pɯ-nɤmŋɯn-a}\hspace{5pt}\pcmn{我很感激这个人给我喝酒}\end{exemple}\relationsémantique{参考}{\lien{ⓔmŋɯn}{mŋɯn}}\relationsémantique{同义词}{\lien{ⓔnɤxpe}{nɤxpe}}\end{entrée}

\begin{entrée}{nɤmpɕɤr}{}{ⓔnɤmpɕɤr} 
\classe{vt}  
\grammaire{trop} \paradigme{dir}{nɯ-}
\begin{définition}\pfra{trouver beau}\end{définition}
\begin{définition}\pcmn{觉得漂亮}\end{définition}
\begin{exemple}\pjya{nɤʑo ɲɯ-ta-nɤmpɕɤr}\hspace{5pt}\pcmn{我觉得你很漂亮}\end{exemple}
\begin{exemple}\pjya{jiɕqha tɕheme nɯ ɲɯ-nɤmpɕar-a}\hspace{5pt}\pcmn{我觉得这个姑娘长得很漂亮}\end{exemple}
\begin{exemple}\pjya{jiɕqha tɯ-ŋga nɯ ɲɯ-nɤmpɕar-a}\hspace{5pt}\pcmn{我觉得这件衣服很漂亮}\end{exemple}
\begin{sous-entrée}{sɤnɤmpɕɤr}{ⓔnɤmpɕɤrⓝsɤnɤmpɕɤr} 
\classe{vi}  
\grammaire{apass} 
\begin{définition}\pfra{trouver beau (les gens)}\end{définition}
\begin{définition}\pcmn{觉得别人漂亮}\end{définition}
\begin{exemple}\pjya{nɤki tɤtɕɯpɯ nɯ kɯ-sɤ-nɤmpɕɤr ci ŋu}\hspace{5pt}\pcmn{那个男孩子觉得(所有的姑娘都)很漂亮}\end{exemple}\end{sous-entrée}

\begin{sous-entrée}{ʑɣɤnɤmpɕɤr}{ⓔnɤmpɕɤrⓝʑɣɤnɤmpɕɤr} 
\classe{vi}  
\grammaire{refl}
\grammaire{trop} 
\begin{définition}\pfra{se trouver beau}\end{définition}
\begin{définition}\pcmn{觉得自己漂亮}\end{définition}
\begin{exemple}\pjya{tɕhemɤpɯ ra kɯ ``aʑo mpɕar-a" ntsɯ sɯso-nɯ ɕti nɤ, ɲɯ-ʑɣɤ-nɤmpɕɤrnɯ ɕti}\hspace{5pt}\pcmn{姑娘们总觉得自己长得很漂亮}\end{exemple}\end{sous-entrée}

\begin{sous-entrée}{anɤmpɕɯpɕɤr}{ⓔnɤmpɕɤrⓝanɤmpɕɯpɕɤr} 
\classe{vi} 
\begin{définition}\pfra{se trouver beau les uns les autres}\end{définition}
\begin{définition}\pcmn{互相觉得漂亮}\end{définition}
\begin{exemple}\pjya{ɲɯ-ɤnɤmpɕɯmpɕɤr-ndʑi}\hspace{5pt}\pcmn{他们俩互相觉得漂亮}\end{exemple}\relationsémantique{参考}{\lien{ⓔmpɕɤr}{mpɕɤr}}\end{sous-entrée}

\end{entrée}

\begin{entrée}{nɤmphoʁmphɯr}{}{ⓔnɤmphoʁmphɯr}\relationsémantique{参考}{\lien{ⓔmphɯr}{mphɯr}}\end{entrée}

\begin{entrée}{nɤmphruʑa}{}{ⓔnɤmphruʑa} 
\classe{vt}  
\grammaire{incorp} \paradigme{dir}{tɤ-}\paradigme{dir}{\_}
\begin{définition}\pfra{faire l'un après l'autre}\end{définition}
\begin{définition}\pcmn{一个挨着一个地做}\end{définition}
\begin{exemple}\pjya{kɯki ta-ma ki rcanɯ tɤ-nɤmphruʑa-t-a ʑo}\hspace{5pt}\pcmn{我把任务一个接着一个的做了}\end{exemple}
\begin{exemple}\pjya{@guazi tɯ-tɤ-fkɯm nɯ tɤ-nɤmphruʑa-t-a tɕe tɤ-ndza-t-a}\hspace{5pt}\pcmn{我把那一袋的瓜子一个个地吃了}\end{exemple}\relationsémantique{参考}{\lien{ⓔʑaⓗ1}{ʑa₁}}\relationsémantique{参考}{\lien{ⓔɯ-mphru}{ɯ-mphru}}\relationsémantique{参考}{\lien{ⓔsɤʑa}{sɤʑa}}\end{entrée}

\begin{entrée}{nɤmpɯ}{}{ⓔnɤmpɯ}\relationsémantique{参考}{\lien{ⓔmpɯ}{mpɯ}}\end{entrée}

\begin{entrée}{nɤmqe}{}{ⓔnɤmqe} 
\classe{vt} \paradigme{dir}{tɤ-}
\begin{définition}\pfra{insulter, gronder}\end{définition}
\begin{définition}\pcmn{骂}\end{définition}
\begin{exemple}\pjya{tɤ-nɤmqe-t-a}\hspace{5pt}\pcmn{我骂了他}\end{exemple}
\begin{exemple}\pjya{ta-nɤmqe}\hspace{5pt}\pcmn{他骂了他}\end{exemple}
\begin{exemple}\pjya{jiɕqha tɤ-mu nɯ kɯ ɯ-rɟit ta-nɤmqe}\hspace{5pt}\pcmn{那个母亲骂了他的儿子}\end{exemple}
\begin{exemple}\pjya{ma-tɤ-kɯ-nɤmqe-a}\hspace{5pt}\pcmn{你不要骂我}\end{exemple}
\begin{exemple}\pjya{tɤ-ta-nɤmqe}\hspace{5pt}\pcmn{我骂了你}\end{exemple}\relationsémantique{同义词}{\lien{ⓔnɯjʁo}{nɯjʁo}}
\begin{sous-entrée}{ʑɣɤnɤmqe}{ⓔnɤmqeⓝʑɣɤnɤmqe} 
\classe{vi}  
\grammaire{refl} 
\begin{définition}\pfra{se faire insulter}\end{définition}
\begin{définition}\pcmn{招人骂}\end{définition}\end{sous-entrée}

\end{entrée}

\begin{entrée}{nɤmtɕɯrlu/\variante{nɤmtɕɯrlɯr}}{}{ⓔnɤmtɕɯrlu} 
\classe{vi}  
\grammaire{n.orient} \paradigme{dir}{thɯ-}
\begin{définition}\pfra{tourner en rond}\end{définition}
\begin{définition}\pcmn{转来转去}\end{définition}
\begin{exemple}\pjya{nɯtɕu ma-tɯ-nɤmtɕɯrlɯr ntsɯ}\hspace{5pt}\pcmn{你别总是在那里转来转去}\end{exemple}
\begin{exemple}\pjya{pɯ-nɤmtɕɯrlɯr-a ntsɯ}\hspace{5pt}\pcmn{我总是(在那里)转来转去(过去时)}\end{exemple}
\begin{exemple}\pjya{nɤʑo sɲikuku kha ɯ-ŋgɯ ntsɯ chɯ-tɯ-nɤmtɕhɯrlu}\hspace{5pt}\pcmn{你总是每天在家里转来转去(乌鸦之言)}\end{exemple}\relationsémantique{参考}{\lien{ⓔmtɕɯr}{mtɕɯr}}\end{entrée}

\begin{entrée}{nɤmthu}{}{ⓔnɤmthu} 
\classe{vt} \paradigme{dir}{nɯ-}
\begin{définition}\pfra{envier}\end{définition}
\begin{définition}\pcmn{羡慕}\end{définition}
\begin{exemple}\pjya{ɯ-scɯʁzɯɣ ɲɯ-βdi, nɯ-nɤmthu-t-a}\hspace{5pt}\pcmn{她相貌美观,我很羡慕}\end{exemple}
\begin{exemple}\pjya{ɲɯ-ta-nɤmthu}\hspace{5pt}\pcmn{我很羡慕你}\end{exemple}\relationsémantique{同义词}{\lien{ⓔnɤsma}{nɤsma}}\relationsémantique{参考}{\lien{ⓔmthu}{mthu}}\end{entrée}

\begin{entrée}{nɤmthɯn}{}{ⓔnɤmthɯn} 
\classe{vt}  
\grammaire{trop} \paradigme{dir}{nɯ-}
\begin{définition}\pfra{aimer}\end{définition}
\begin{définition}\pcmn{爱;喜欢(异性)}\end{définition}
\begin{exemple}\pjya{jiɕqha tɕheme nɯ ɲɤ-nɤmthɯn ɲɯ-ŋu}\hspace{5pt}\pcmn{他喜欢了这个女人}\end{exemple}
\begin{exemple}\pjya{jiɕqha nɯ ɯ-nmaʁ ɣɤʑu ri, li ɲɤ-nɤmthɯn}\hspace{5pt}\pcmn{那个(女人)虽然有丈夫,但是爱上了另外一个}\end{exemple}
\begin{exemple}\pjya{aj ɲɯ-ta-nɤmthɯn! hehe, nɯ ma-tɯ-ti!}\hspace{5pt}\pcmn{我喜欢你!(女孩子的答复)你别那样说!}\end{exemple}\relationsémantique{同义词}{\lien{ⓔnɤntshi}{nɤntshi}}\relationsémantique{参考}{\lien{ⓔamthɯn}{amthɯn}}\end{entrée}

\begin{entrée}{nɤmtshɤr}{}{ⓔnɤmtshɤr} 
\classe{vt}  
\grammaire{trop} \paradigme{dir}{nɯ-}
\begin{définition}\pfra{trouver étrange}\end{définition}
\begin{définition}\pcmn{觉得奇怪;感到惊讶}\end{définition}\paradigme{dir}{nɯ-}
\begin{définition}\pfra{trouver bizarre quelque chose à propos de soi}\end{définition}
\begin{définition}\pcmn{觉得自己的事情奇怪}\end{définition}
\begin{exemple}\pjya{jiɕqha nɯ tɤ-aʑɯʑu-ndʑi ri, ``mɤ-cha" nɯ-sɯso-t-a ri, pɯ-cha rcanɯ nɯ-nɤmtshar-a}\hspace{5pt}\pcmn{在角力的时候,我以为他不行,但他居然赢了,我感到很惊讶}\end{exemple}
\begin{exemple}\pjya{``mɤ-tɯ-cha" nɯ-sɯso-t-a ri, pɯ-tɯ-cha rcanɯ, nɯ-ta-nɤmtshɤr}\hspace{5pt}\pcmn{我以为你不行,但是你居然成功了,(你令)我感到很惊讶}\end{exemple}
\begin{exemple}\pjya{jiʑora ji-skɤt ɯ-ɲɯ́-nɤmtshɤr-nɯ?}\hspace{5pt}\pcmn{他们觉得我们的语言奇怪吗?}\end{exemple}
\begin{exemple}\pjya{ɲɯ-ʑɣɤnɤmtshar-a ɕti ma pɯ-nɯʑɯβ-a ɕti ri, nɯ kɯnɤ mɤ-kɯ-mbrɤt ʑo ɲɯ-ɤχom-a}\hspace{5pt}\pcmn{我觉得很奇怪,虽然我睡了,还是不停地打哈欠}\end{exemple}\relationsémantique{同义词}{\lien{ⓔnaχaʁ}{naχaʁ}}\relationsémantique{参考}{\lien{ⓔsɤmtshɤr}{sɤmtshɤr}}
\begin{sous-entrée}{ʑɣɤnɤmtshɤr}{ⓔnɤmtshɤrⓝʑɣɤnɤmtshɤr} 
\classe{vi}  
\grammaire{refl}
\grammaire{trop} \end{sous-entrée}

\étymologie{mtsʰar}\end{entrée}

\begin{entrée}{nɤmtsioʁ}{}{ⓔnɤmtsioʁ} 
\classe{vt}  
\grammaire{denom}
\grammaire{互相啄} \paradigme{dir}{tɤ-}\paradigme{dir}{tɤ-}\paradigme{dir}{tɤ-}
\begin{définition}\pfra{donner un coup de bec}\end{définition}
\begin{définition}\pcmn{啄}\end{définition}
\begin{définition}\pfra{donner des coups de bec aux gens}\end{définition}
\begin{définition}\pcmn{啄人}\end{définition}
\begin{définition}\pfra{se donner des coups de bec les uns aux autres}\end{définition}
\begin{exemple}\pjya{kumpɣa kɯ tɤ́-wɣ-nɤmtsioʁ-a}\hspace{5pt}\pcmn{我被鸡啄了一口}\end{exemple}
\begin{exemple}\pjya{ta-nɤmtsioʁ}\hspace{5pt}\pcmn{鸡啄了一口}\end{exemple}
\begin{exemple}\pjya{kumpɣa ɲɯ-sɤnɤmtsioʁ}\hspace{5pt}\pcmn{鸡老爱啄人}\end{exemple}\relationsémantique{参考}{\lien{ⓔɯ-mtsioʁ}{ɯ-mtsioʁ}}
\begin{sous-entrée}{sɤnɤmtsioʁ}{ⓔnɤmtsioʁⓝsɤnɤmtsioʁ} 
\classe{vi}  
\grammaire{apass} \end{sous-entrée}

\begin{sous-entrée}{anɤmtsɯmtsioʁ}{ⓔnɤmtsioʁⓝanɤmtsɯmtsioʁ} 
\classe{vi}  
\grammaire{recip} \end{sous-entrée}

\begin{exemple}\pjya{to-k-ɤnɤmtsɯmtsioʁ-nɯ-ci}\hspace{5pt}\pcmn{(公鸡)互相啄了起来}\end{exemple}\end{entrée}

\begin{entrée}{nɤmɯm}{}{ⓔnɤmɯm}\relationsémantique{参考}{\lien{ⓔmɯm}{mɯm}}\end{entrée}

\begin{entrée}{nɤmɯma}{}{ⓔnɤmɯma} 
\classe{vt} \paradigme{dir}{nɯ-}\sens{1}
\begin{définition}\pfra{caresser}\end{définition}
\begin{définition}\pcmn{抚摸}\end{définition}
\begin{exemple}\pjya{nɯ-nɤmɯma-t-a, kɤ-nɤmɯma-t-a}\hspace{5pt}\pcmn{我摸了}\end{exemple}
\begin{exemple}\pjya{@zhuozi nɯ-nɤmɯma-t-a ri ɲɯ-mpɕu}\hspace{5pt}\pcmn{我摸了一下桌子,很光滑}\end{exemple}\sens{2}
\begin{définition}\pfra{tâtonner, chercher à tâtons}\end{définition}
\begin{définition}\pcmn{(暗中)摸索、探索}\end{définition}\relationsémantique{参考}{\lien{ⓔznɤmɯma}{znɤmɯma}}\end{entrée}

\begin{entrée}{nɤnɤm}{}{ⓔnɤnɤm} 
\classe{vs} 
\begin{définition}\pfra{être amplement suffisant}\end{définition}
\begin{définition}\pcmn{绰绰有余}\end{définition}
\begin{exemple}\pjya{a-kɤ-ndza kɤ-ŋga pɯ-nɤnɤm}\hspace{5pt}\pcmn{我以前吃穿都充裕}\end{exemple}\end{entrée}

\begin{entrée}{nɤnbaʁ}{}{ⓔnɤnbaʁ}\relationsémantique{参考}{\lien{ⓔanbaʁ}{anbaʁ}}\end{entrée}

\begin{entrée}{nɤnbɯnbaʁ}{}{ⓔnɤnbɯnbaʁ}\relationsémantique{参考}{\lien{ⓔanbaʁ}{anbaʁ}}\end{entrée}

\begin{entrée}{nɤndɤɣ}{}{ⓔnɤndɤɣ} 
\classe{vi} \paradigme{dir}{pɯ-}\paradigme{dir}{pɯ-}
\begin{définition}\pfra{être empoisonné}\end{définition}
\begin{définition}\pcmn{中毒}\end{définition}
\begin{définition}\pfra{empoisonner}\end{définition}
\begin{définition}\pcmn{使人中毒}\end{définition}
\begin{exemple}\pjya{ɯ-tshɯɣa a-mɤ-tɤ-βdi tɕe pjɯ-kɯ-znɤndɤɣ ŋu}\hspace{5pt}\pcmn{如果弄得不好就会中毒(例如,蘑菇加工的方法)}\end{exemple}
\begin{exemple}\pjya{ma-tɤ-tɯ-ndze ma tɯ́-wɣ-znɤndɤɣ}\hspace{5pt}\pcmn{不要吃,会中毒}\end{exemple}\relationsémantique{参考}{\lien{ⓔsɤndɤɣ}{sɤndɤɣ}}
\begin{sous-entrée}{znɤndɤɣ}{ⓔnɤndɤɣⓝznɤndɤɣ} 
\classe{vt} \end{sous-entrée}

\end{entrée}

\begin{entrée}{nɤndɤɣri}{}{ⓔnɤndɤɣri} 
\classe{vi}  
\grammaire{denom} \paradigme{dir}{pɯ-}\paradigme{dir}{pɯ-}
\begin{définition}\pfra{avoir un enfant illégitime (femme)}\end{définition}
\begin{définition}\pcmn{生私生子}\end{définition}
\begin{définition}\pfra{avoir un enfant illégitime (homme)}\end{définition}
\begin{définition}\pcmn{令(一个女子)生私生子}\end{définition}
\begin{exemple}\pjya{kɯmaʁ tɤ-tɕɯ nɯ kɯ pa-znɤndɤɣri}\hspace{5pt}\pcmn{她跟别的男人有了私生子}\end{exemple}\relationsémantique{参考}{\lien{ⓔtɤndɤɣri}{tɤndɤɣri}}
\begin{sous-entrée}{znɤndɤɣri}{ⓔnɤndɤɣriⓝznɤndɤɣri} 
\classe{vt}  
\grammaire{caus} \end{sous-entrée}

\end{entrée}

\begin{entrée}{nɤndɤr}{}{ⓔnɤndɤr} 
\classe{vi} \paradigme{dir}{tɤ-}
\begin{définition}\pfra{vibrer}\end{définition}
\begin{définition}\pcmn{振动}\end{définition}\paradigme{dir}{tɤ-}
\begin{définition}\pfra{faire vibrer}\end{définition}
\begin{définition}\pcmn{(使)振动}\end{définition}
\begin{exemple}\pjya{nɯ pɯ-atɤr tɕe tɤ-nɤndɤr}\hspace{5pt}\pcmn{掉下去就震动了}\end{exemple}
\begin{exemple}\pjya{ta-znɤndɤr}\hspace{5pt}\pcmn{他(让那个东西)震动了一下}\end{exemple}
\begin{exemple}\pjya{tɤ-znɤndar-a}\hspace{5pt}\pcmn{我震动了一下}\end{exemple}
\begin{exemple}\pjya{kutɕu tɤŋkhɯt tɤ-lat-a tɕe, @luyinji tɤ-znɤndar-a}\hspace{5pt}\pcmn{我在这里拍了一拳,把录音机震动了一下}\end{exemple}
\begin{sous-entrée}{znɤndɤr}{ⓔnɤndɤrⓝznɤndɤr} 
\classe{vt}  
\grammaire{caus} \end{sous-entrée}

\étymologie{ⁿdar}\end{entrée}

\begin{entrée}{nɤndɯndɤt}{}{ⓔnɤndɯndɤt} 
\classe{vi} \paradigme{dir}{\_}
\begin{définition}\pfra{aller n'importe où}\end{définition}
\begin{définition}\pcmn{到处走}\end{définition}
\begin{exemple}\pjya{ma-ɕɯ-tɯ-nɤndɯndɤt}\hspace{5pt}\pcmn{你不要到处乱走}\end{exemple}\relationsémantique{参考}{\lien{ⓔŋotɕuŋondɤt}{ŋotɕuŋondɤt}}\relationsémantique{参考}{\lien{ⓔaʁɤndɯndɤt}{aʁɤndɯndɤt}}
\begin{sous-entrée}{znɤndɯndɤt}{ⓔnɤndɯndɤtⓝznɤndɯndɤt} 
\classe{vt}  
\grammaire{caus} 
\begin{définition}\pfra{laisser aller n'importe oæ}\end{définition}\end{sous-entrée}

\end{entrée}

\begin{entrée}{nɤndɯndo}{}{ⓔnɤndɯndo} 
\classe{vt}  
\grammaire{n.orient} \paradigme{dir}{tɤ-}
\begin{définition}\pfra{emmener partout}\end{définition}
\begin{définition}\pcmn{随身带着;带来带去}\end{définition}
\begin{exemple}\pjya{laχtɕha nɯ kɤ-nɤndɯndo ntsɯ ɲɯ-ra}\hspace{5pt}\pcmn{这个东西要随身带上}\end{exemple}
\begin{exemple}\pjya{@zixingche ɲɯ-ɤz-nɤndɯndo}\hspace{5pt}\pcmn{他去哪里都带自行车}\end{exemple}\relationsémantique{参考}{\lien{ⓔndo}{ndo}}\end{entrée}

\begin{entrée}{nɤndɯt}{}{ⓔnɤndɯt} 
\classe{vt} \paradigme{dir}{tɤ-}
\begin{définition}\pfra{se disputer}\end{définition}
\begin{définition}\pcmn{争论;争吵}\end{définition}
\begin{exemple}\pjya{jiɕqha nɯ cho tɤ-anɯmqaj-tɕi tɕe tɤ-nɤndɯt-tɕi}\hspace{5pt}\pcmn{我跟这个人吵架了,争吵了}\end{exemple}
\begin{exemple}\pjya{laχtɕha tɤ-nɤndɯt-tɕi}\hspace{5pt}\pcmn{我们俩争了那个东西}\end{exemple}
\begin{exemple}\pjya{ma-tɤ-tɯ-nɤndɯt-nɯ}\hspace{5pt}\pcmn{你们不要再争了}\end{exemple}
\begin{exemple}\pjya{mɤ-nɤndɯt-tɕi}\hspace{5pt}\pcmn{我们俩不再争了}\end{exemple}\end{entrée}

\begin{entrée}{nɤndza}{}{ⓔnɤndza} 
\classe{vi} \paradigme{dir}{kɤ-}
\begin{définition}\pfra{attraper la lèpre}\end{définition}
\begin{définition}\pcmn{患上麻疯病}\end{définition}
\begin{exemple}\pjya{kɤ-kɯ-nɤndza}\hspace{5pt}\pcmn{麻风病患者(骂人的话)}\end{exemple}
\begin{exemple}\pjya{kɤ-kɯ-nɤndza nɯ tɯŋgo stu kɯ-ŋɤn ŋu}\hspace{5pt}\pcmn{麻风病是最严重的病}\end{exemple}\end{entrée}

\begin{entrée}{nɤndzɤβ}{}{ⓔnɤndzɤβ}\relationsémantique{参考}{\lien{ⓔndzɤβ}{ndzɤβ}}\end{entrée}

\begin{entrée}{nɤndzɯt}{}{ⓔnɤndzɯt}\relationsémantique{参考}{\lien{ⓔandzɯt}{andzɯt}}\end{entrée}

\begin{entrée}{nɤndʑaʁlaʁ}{}{ⓔnɤndʑaʁlaʁ} 
\classe{vi}  
\grammaire{n.orient} \paradigme{dir}{pɯ-}
\begin{définition}\pfra{nager}\end{définition}
\begin{définition}\pcmn{游泳;漂浮;游来游去}\end{définition}
\begin{exemple}\pjya{aj pɯ-nɤndʑaʁlaʁ-a}\hspace{5pt}\pcmn{我游来游去了}\end{exemple}\relationsémantique{参考}{\lien{ⓔndʑaʁ}{ndʑaʁ}}\end{entrée}

\begin{entrée}{nɤndʑe}{}{ⓔnɤndʑe} 
\classe{vi} \paradigme{dir}{pɯ-}
\begin{définition}\pfra{profiter de quelque chose}\end{définition}
\begin{définition}\pcmn{占便宜}\end{définition}
\begin{exemple}\pjya{pɯ-tɯ-nɤndʑe}\hspace{5pt}\pcmn{你占了便宜}\end{exemple}
\begin{exemple}\pjya{tɯtsɣe tɤ-βzu-tɕi, aʑo pɯ-nɯzɟɯ-a, nɤʑo pɯ-tɯ-nɤndʑe}\hspace{5pt}\pcmn{我们俩做了生意,我吃亏了,你占了便宜}\end{exemple}
\begin{exemple}\pjya{ʑɴɢɯloʁ pɯ-nɯkro-tɕi, nɤʑo pɯ-tɯ-nɤndʑe, aʑo pɯ-nɯzɟɯ-a}\hspace{5pt}\pcmn{我们俩分核桃,你占了便宜,我吃亏了}\end{exemple}\relationsémantique{反义词}{\lien{ⓔnɯzɟɯ}{nɯzɟɯ}}\end{entrée}

\begin{entrée}{nɤndʑɣi}{}{ⓔnɤndʑɣi} 
\classe{vt} \paradigme{dir}{thɯ-}\paradigme{dir}{pɯ-}
\begin{définition}\pfra{avoir de la morve au nez}\end{définition}
\begin{définition}\pcmn{吊着鼻涕}\end{définition}
\begin{exemple}\pjya{ɯʑo kɯ ɯ-ɕnaβ pjɯ-nɤndʑɣi ŋgrɤl}\hspace{5pt}\pcmn{他平时吊着鼻涕}\end{exemple}\end{entrée}

\begin{entrée}{nɤndʐaβlaβ}{}{ⓔnɤndʐaβlaβ} 
\classe{vi}  
\grammaire{n.orient} \paradigme{dir}{\_}
\begin{définition}\pfra{rouler}\end{définition}
\begin{définition}\pcmn{滚来滚去}\end{définition}
\begin{exemple}\pjya{ɲɯ-saʁdɤt tɕe pɯ-nɤndʐaβlaβ-a}\hspace{5pt}\pcmn{地很滑,我摔跤了很多次(爬起来又摔跤了几次)}\end{exemple}
\begin{exemple}\pjya{mkhɯrlu aʁɤndɯndɤt ɲɯ-nɤndʐaβlaβ}\hspace{5pt}\pcmn{轮子到处滚动}\end{exemple}\relationsémantique{参考}{\lien{ⓔndʐaβ}{ndʐaβ}}\end{entrée}

\begin{entrée}{nɤndʐɤrqɯ}{}{ⓔnɤndʐɤrqɯ} 
\classe{vs} \paradigme{dir}{tɤ-}
\begin{définition}\pfra{avoir froid, être frileux}\end{définition}
\begin{définition}\pcmn{觉得冷,怕冷}\end{définition}
\begin{exemple}\pjya{kɯre ku-nɤndʐɤrqɯ-a}\hspace{5pt}\pcmn{我在这里觉得冷}\end{exemple}
\begin{exemple}\pjya{qartsɯ tɕe kɯ-nɤndʐɤrqɯ ntsɯ ŋgrɤl}\hspace{5pt}\pcmn{冬天的时候人总是觉得冷}\end{exemple}
\begin{exemple}\pjya{ɣɯjpa qartsɯ pɯ-ɣɤndʐo tɕe aʑo pɯ-nɤndʐɤrqɯ-a ntsɯ}\hspace{5pt}\pcmn{今年冬天我一直觉得很冷}\end{exemple}\relationsémantique{参考}{\lien{ⓔtɤndʐo}{tɤndʐo}}\end{entrée}

\begin{entrée}{nɤndʐi}{}{ⓔnɤndʐi} 
\classe{vt} \paradigme{dir}{pɯ-}
\begin{définition}\pfra{enlever la peau}\end{définition}
\begin{définition}\pcmn{剥皮}\end{définition}
\begin{exemple}\pjya{paʁ-ku pɯ-nɤndʐi-t-a}\hspace{5pt}\pcmn{我剥了猪头的皮}\end{exemple}\relationsémantique{参考}{\lien{ⓔtɯ-ndʐi}{tɯ-ndʐi}}\end{entrée}

\begin{entrée}{nɤndʐo}{}{ⓔnɤndʐo} 
\classe{vi}  
\grammaire{refl} \paradigme{dir}{nɯ-}\paradigme{dir}{nɯ-}
\begin{définition}\pfra{prendre froid}\end{définition}
\begin{définition}\pcmn{着凉}\end{définition}
\begin{exemple}\pjya{nɯ-nɤndʐo-a}\hspace{5pt}\pcmn{我着凉了}\end{exemple}
\begin{exemple}\pjya{a-ŋga ɲɯ-mba tɕe nɯ-nɤndʐo-a}\hspace{5pt}\pcmn{我的衣服很薄,我着凉了}\end{exemple}
\begin{sous-entrée}{ʑɣɤnɤndʐo}{ⓔnɤndʐoⓝʑɣɤnɤndʐo} 
\classe{vi} \end{sous-entrée}

\begin{définition}\pfra{attraper froid}\end{définition}
\begin{définition}\pcmn{令自己受凉}\end{définition}
\begin{exemple}\pjya{nɯ-ʑɣɤnɤndʐo-a}\hspace{5pt}\pcmn{我令自己受凉了}\end{exemple}
\begin{exemple}\pjya{ma-nɯ-tɯ-ʑɣɤ-nɤndʐo}\hspace{5pt}\pcmn{你不要令自己受凉}\end{exemple}\relationsémantique{参考}{\lien{ⓔtɤndʐo}{tɤndʐo}}\end{entrée}

\begin{entrée}{nɤngɯt}{}{ⓔnɤngɯt} 
\classe{vt} \paradigme{dir}{tɤ-}\paradigme{dir}{tɤ-}
\begin{définition}\pfra{posséder en commun, partager}\end{définition}
\begin{définition}\pcmn{共同拥有}\end{définition}
\begin{définition}\pfra{partager avec}\end{définition}
\begin{définition}\pcmn{跟……分、一起用}\end{définition}
\begin{exemple}\pjya{@cai tɤ-nɤngɯt-tɕi}\hspace{5pt}\pcmn{我们俩一起吃了一碗菜}\end{exemple}
\begin{exemple}\pjya{ki @cidian ki nɤngɯt-tɕi}\hspace{5pt}\pcmn{我们一起拥有这本词典}\end{exemple}
\begin{exemple}\pjya{tɤ-nɤngɯt-a}\hspace{5pt}\pcmn{我也有了一份}\end{exemple}\relationsémantique{参考}{\lien{ⓔtɤngɯt}{tɤngɯt}}\relationsémantique{参考}{\lien{ⓔangɯt}{angɯt}}
\begin{sous-entrée}{znɤngɯt}{ⓔnɤngɯtⓝznɤngɯt} 
\classe{vt} \end{sous-entrée}

\end{entrée}

\begin{entrée}{nɤntshɣɤz}{}{ⓔnɤntshɣɤz} 
\classe{vt} \paradigme{dir}{nɯ-}
\begin{définition}\pfra{heurter, se cogner contre}\end{définition}
\begin{définition}\pcmn{撞到(无意)}\end{définition}
\begin{exemple}\pjya{nɯ́-wɣ-nɤntshɣaz-a}\hspace{5pt}\pcmn{他撞到我了}\end{exemple}
\begin{exemple}\pjya{@qiche ɲɯ-dɤn tɕe ɲɯ-kɯ-nɤntshɣɤz ɲɯ-ŋu}\hspace{5pt}\pcmn{汽车很多,会撞到人}\end{exemple}
\begin{sous-entrée}{sɤnɤntshɣɤz}{ⓔnɤntshɣɤzⓝsɤnɤntshɣɤz} 
\classe{vi}  
\grammaire{apass} 
\begin{définition}\pfra{heurter les gens}\end{définition}
\begin{définition}\pcmn{撞到人}\end{définition}\end{sous-entrée}

\end{entrée}

\begin{entrée}{nɤntshi}{}{ⓔnɤntshi} 
\classe{vt}  
\grammaire{trop} \paradigme{dir}{nɯ-}
\begin{définition}\pfra{aimer}\end{définition}
\begin{définition}\pcmn{爱;喜欢(异性)}\end{définition}
\begin{exemple}\pjya{ɲɯ-nɤntshi-a tɕe aʑo ɕɯ-the-a ra}\hspace{5pt}\pcmn{我喜欢她,我要去向她求婚}\end{exemple}
\begin{exemple}\pjya{jiɕqha tɕheme nɯ ɲɯ-nɤntshi-a}\hspace{5pt}\pcmn{我喜欢这个女子}\end{exemple}\relationsémantique{同义词}{\lien{ⓔnɤmthɯn}{nɤmthɯn}}\relationsémantique{参考}{\lien{ⓔntshiⓗ2}{ntshi₂}}
\begin{sous-entrée}{anɤntshɯntshi}{ⓔnɤntshiⓝanɤntshɯntshi} 
\classe{vi}  
\grammaire{appl}
\grammaire{recip} 
\begin{définition}\pfra{s'aimer l'un l'autre}\end{définition}
\begin{définition}\pcmn{相爱}\end{définition}
\begin{exemple}\pjya{ɲɯ-ɤnɤntshɯntshi-ndʑi}\hspace{5pt}\pcmn{他们俩相爱}\end{exemple}\relationsémantique{同义词}{\lien{ⓔanɯrgɯrga}{anɯrgɯrga}}\end{sous-entrée}

\end{entrée}

\begin{entrée}{nɤɲchɯɲchɤm}{}{ⓔnɤɲchɯɲchɤm} 
\classe{vi} \paradigme{dir}{thɯ-}
\begin{définition}\pfra{rôder}\end{définition}
\begin{définition}\pcmn{无聊地到处流浪;到处走动}\end{définition}
\begin{exemple}\pjya{kɯ-nɤɲchɯɲchɤm jɤ-ari-a}\hspace{5pt}\pcmn{我去到处流浪}\end{exemple}\end{entrée}

\begin{entrée}{nɤɲi}{}{ⓔnɤɲi} 
\classe{vt} \paradigme{dir}{tɤ-}
\begin{définition}\pfra{se tenir avec}\end{définition}
\begin{définition}\pcmn{拄着……,仗着……}\end{définition}
\begin{exemple}\pjya{laʁjɯɣ ci tɤ-nɤɲi tɕe mɤ-tɯ-ndʐaβ}\hspace{5pt}\pcmn{你拄着拐杖就不会摔倒}\end{exemple}\relationsémantique{参考}{\lien{ⓔtɤɲi}{tɤɲi}}\end{entrée}

\begin{entrée}{nɤɲɟɯɲɟu}{}{ⓔnɤɲɟɯɲɟu} 
\classe{vt} \paradigme{dir}{tɤ-}\paradigme{dir}{tɤ-}
\begin{définition}\pfra{appâter}\end{définition}
\begin{définition}\pcmn{引过来;吸引}\end{définition}
\begin{définition}\pfra{attirer}\end{définition}
\begin{définition}\pcmn{用东西引过来}\end{définition}
\begin{exemple}\pjya{jla tɤ-nɤɲɟɯɲɟu-t-a}\hspace{5pt}\pcmn{我把犏牛引过来了}\end{exemple}
\begin{exemple}\pjya{daltsɯtsa tɤ-tɯt-a tɕe tɤ-nɤɲɟɯɲɟu-t-a}\hspace{5pt}\pcmn{我慢慢地说,把他骗过来了}\end{exemple}
\begin{exemple}\pjya{jla tsha kɯ tɤ-znɤɲɟɯɲɟu-t-a}\hspace{5pt}\pcmn{我用盐巴把犏牛引过来了}\end{exemple}
\begin{sous-entrée}{znɤɲɟɯɲɟu}{ⓔnɤɲɟɯɲɟuⓝznɤɲɟɯɲɟu} 
\classe{vt}  
\grammaire{caus} \end{sous-entrée}

\end{entrée}

\begin{entrée}{nɤŋɤβ}{}{ⓔnɤŋɤβ}\relationsémantique{参考}{\lien{ⓔsɤŋɤβ}{sɤŋɤβ}}\end{entrée}

\begin{entrée}{nɤŋgɤr}{}{ⓔnɤŋgɤr}\relationsémantique{参考}{\lien{ⓔŋgɤr}{ŋgɤr}}\end{entrée}

\begin{entrée}{nɤŋgiolo}{}{ⓔnɤŋgiolo}\relationsémantique{参考}{\lien{ⓔŋgio}{ŋgio}}\end{entrée}

\begin{entrée}{nɤŋgɯ}{}{ⓔnɤŋgɯ} 
\classe{vt} \paradigme{dir}{nɯ-}\paradigme{dir}{nɯ-}
\begin{définition}\pfra{emprunter}\end{définition}
\begin{définition}\pcmn{向别人借(不能归还原物)}\end{définition}
\begin{définition}\pfra{prêter}\end{définition}
\begin{définition}\pcmn{借给别人(不能归还原物)}\end{définition}
\begin{exemple}\pjya{aʑo mɯ-ɲɯ-ɤro-a tɕe, nɤ-phe nɯ-nɤŋgɯ-t-a}\end{exemple}
\begin{exemple}\pjya{aʑɯɣ maŋe tɕe, nɤj nɤ-phe nɯ-nɤŋgɯ-t-a}\hspace{5pt}\pcmn{我没有,所以向你借了}\end{exemple}
\begin{exemple}\pjya{a-pɯ-tɯ-ɤro tɕe nɤ-phe kɤ-nɤŋgɯ khɯ, a-mɤ-pɯ-tɯ-ɤro tɕe, nɤ-phe kɤ-nɤŋgɯ mɤ-khɯ}\hspace{5pt}\pcmn{你有的话可以向你借,没有的话就不能借}\end{exemple}
\begin{exemple}\pjya{(kɤndza, rŋɯl) nɯ-znɤŋgɯ-t-a}\hspace{5pt}\pcmn{我借给他了(食物、钱)}\end{exemple}\relationsémantique{参考}{\lien{ⓔrŋo}{rŋo}}\relationsémantique{参考}{\lien{ⓔtɤŋgɯ}{tɤŋgɯ}}\relationsémantique{参考}{\lien{ⓔɕɯrŋo}{ɕɯrŋo}}
\begin{sous-entrée}{znɤŋgɯ}{ⓔnɤŋgɯⓝznɤŋgɯ} 
\classe{vt} \end{sous-entrée}

\end{entrée}

\begin{entrée}{nɤŋgɯŋga}{}{ⓔnɤŋgɯŋga}\relationsémantique{参考}{\lien{ⓔŋga}{ŋga}}\end{entrée}

\begin{entrée}{nɤŋka}{}{ⓔnɤŋka} 
\classe{vt}  
\grammaire{denom} \paradigme{dir}{nɯ-}
\begin{définition}\pfra{ronger}\end{définition}
\begin{définition}\pcmn{嚼}\end{définition}
\begin{exemple}\pjya{kɯchi nɯ-nɤŋka-t-a}\hspace{5pt}\pcmn{我嚼了糖}\end{exemple}
\begin{exemple}\pjya{kɯchi na-nɤŋka}\hspace{5pt}\pcmn{他嚼了糖}\end{exemple}
\begin{exemple}\pjya{smɤn nɯ-nɤŋka-t-a}\hspace{5pt}\pcmn{我嚼了药}\end{exemple}
\begin{exemple}\pjya{ki ɯ-mdʑu ɲɤ-nɤŋka tɕe si ɲɯ-ŋu}\hspace{5pt}\pcmn{(这头牛)已经把舌头露出来了,快要死了}\end{exemple}\relationsémantique{参考}{\lien{ⓔtɯ-ŋka}{tɯ-ŋka}}\end{entrée}

\begin{entrée}{nɤŋkhɯt}{}{ⓔnɤŋkhɯt} 
\classe{vt}  
\grammaire{denom} \paradigme{dir}{tɤ-}
\begin{définition}\pfra{donner un coup de poing}\end{définition}
\begin{définition}\pcmn{用拳头捶打}\end{définition}\relationsémantique{参考}{\lien{ⓔtɤŋkhɯt}{tɤŋkhɯt}}\end{entrée}

\begin{entrée}{nɤŋkɯŋke}{}{ⓔnɤŋkɯŋke} 
\classe{vi}  
\grammaire{n.orient} \paradigme{dir}{\_}
\begin{définition}\pfra{passer}\end{définition}
\begin{définition}\pcmn{走来走去}\end{définition}
\begin{exemple}\pjya{aki @bazi ɯ-ŋgɯ ɲɯ-nɤŋkɯŋke}\hspace{5pt}\pcmn{他在下面的坝子里走来走去}\end{exemple}
\begin{exemple}\pjya{tshɯrɟɯn ɕ-tu-nɤŋkɯŋke-a ŋu ri, ɕɤxɕo kɯkɯra mɤʑɯ tʂu, kɤntɕhaʁ ra ku-oz-ɣɤβdi-nɯ tɕe, sɤŋke khro maŋe wo}\hspace{5pt}\pcmn{我虽然经常去散步,这几天他们又在修路,路不好走}\end{exemple}\relationsémantique{参考}{\lien{ⓔŋke}{ŋke}}\end{entrée}

\begin{entrée}{nɤŋɯr}{}{ⓔnɤŋɯr} 
\classe{vt} \paradigme{dir}{nɯ-}
\begin{définition}\pfra{respecter}\end{définition}
\begin{définition}\pcmn{尊重}\end{définition}
\begin{exemple}\pjya{ɲɯ́-wɣ-nɤŋɯr-a}\hspace{5pt}\pcmn{他尊重我}\end{exemple}\relationsémantique{同义词}{\lien{ⓔnaʁre}{naʁre}}
\begin{sous-entrée}{sɤŋɯr}{ⓔnɤŋɯrⓝsɤŋɯr} 
\classe{vs} 
\begin{définition}\pfra{être respecté}\end{définition}
\begin{définition}\pcmn{得到尊重}\end{définition}\relationsémantique{同义词}{\lien{ⓔsaʁre}{saʁre}}\end{sous-entrée}

\end{entrée}

\begin{entrée}{nɤɴɢiɤt}{}{ⓔnɤɴɢiɤt}\relationsémantique{参考}{\lien{ⓔɴɢiɤt}{ɴɢiɤt}}\end{entrée}

\begin{entrée}{nɤɴqa}{}{ⓔnɤɴqa} 
\classe{vt}  
\grammaire{trop} \paradigme{dir}{nɯ-}
\begin{définition}\pfra{trouver le travail dur}\end{définition}
\begin{définition}\pcmn{觉得辛苦}\end{définition}
\begin{exemple}\pjya{jiɕqha tɤton ɲɯ-nɤɴqe}\hspace{5pt}\pcmn{他在上坡。觉得很辛苦}\end{exemple}
\begin{exemple}\pjya{tɤton na-nɤɴqa}\hspace{5pt}\pcmn{他在上坡。觉得很辛苦了}\end{exemple}
\begin{exemple}\pjya{ɯ-phɯ ɲɯ-nɤɴqe}\hspace{5pt}\pcmn{他觉得贵}\end{exemple}
\begin{exemple}\pjya{@yingyu kɤ-βzjoz ɲɯ-nɤɴqe}\hspace{5pt}\pcmn{他觉得学英语很难}\end{exemple}\relationsémantique{参考}{\lien{ⓔɴqa}{ɴqa}}\end{entrée}

\begin{entrée}{nɤɴqhi}{}{ⓔnɤɴqhi}\relationsémantique{参考}{\lien{ⓔɴqhi}{ɴqhi}}\end{entrée}

\begin{entrée}{nɤɴqi}{}{ⓔnɤɴqi} 
\classe{vt} \paradigme{dir}{nɯ-}
\begin{définition}\pfra{avoir la flemme de, ne pas avoir envie de faire...}\end{définition}
\begin{définition}\pcmn{懒得……;怕麻烦}\end{définition}
\begin{exemple}\pjya{ɯʑo kɯ ɲɯ-nɤɴqi tɕe ɯ-zda jo-sɯxɕe}\hspace{5pt}\pcmn{他因为怕麻烦叫别人去了}\end{exemple}
\begin{exemple}\pjya{ta-ma kɤ-nɤma nɯ-nɤɴqi-t-a}\hspace{5pt}\pcmn{我变得很懒,不想劳动}\end{exemple}\relationsémantique{参考}{\lien{ⓔnɯpɤɴqi}{nɯpɤɴqi}}\end{entrée}

\begin{entrée}{nɤpɤβdaʁ/\variante{nɤpɤdaʁ}}{}{ⓔnɤpɤβdaʁ} 
\classe{vt} \paradigme{dir}{tɤ-}
\begin{définition}\pfra{s'occuper de (enfant)}\end{définition}
\begin{définition}\pcmn{抚养;照顾(孩子)}\end{définition}
\begin{exemple}\pjya{tɤ-pɤtso tɤ-nɤpɤβdaʁ-a}\hspace{5pt}\pcmn{我抚养了孩子}\end{exemple}\relationsémantique{同义词}{\lien{ⓔsɤβlo}{sɤβlo}}\end{entrée}

\begin{entrée}{nɤpɤmbat}{}{ⓔnɤpɤmbat} 
\classe{vi}  
\grammaire{comp} \paradigme{dir}{tɤ-}
\begin{définition}\pfra{faire n'importe comment}\end{définition}
\begin{définition}\pcmn{做得很粗糙;敷衍了事}\end{définition}
\begin{exemple}\pjya{tɤ-nɤpɤmba-t-a}\hspace{5pt}\pcmn{我将就做了}\end{exemple}
\begin{exemple}\pjya{to-nɤpɤmbat}\hspace{5pt}\pcmn{他将就做了}\end{exemple}
\begin{exemple}\pjya{aj jisŋi tɯ-ŋga ci thɯ-tʂɯβ-a ri, tɤ-nɤpɤmba-t-a}\hspace{5pt}\pcmn{我今天缝了一件衣服,缝得很粗糙}\end{exemple}
\begin{exemple}\pjya{ta-ma kɤ-nɤpɤmbat mɤ-ra}\hspace{5pt}\pcmn{工作不要敷衍了事}\end{exemple}\relationsémantique{参考}{\lien{ⓔpaⓗ1}{pa₁}}\relationsémantique{参考}{\lien{ⓔmbat}{mbat}}\end{entrée}

\begin{entrée}{nɤpɤru}{}{ⓔnɤpɤru} 
\classe{vt} \paradigme{dir}{tɤ-}
\begin{définition}\pfra{garder}\end{définition}
\begin{définition}\pcmn{保管;管理}\end{définition}
\begin{exemple}\pjya{aʑo jiɕqha laχtɕha nɯ tɤ-nɤpɤru-t-a}\hspace{5pt}\pcmn{我把这个东西保管(好)了}\end{exemple}
\begin{exemple}\pjya{tɤ-rɤku lɤ́-wɣ-ji tɕe ɣɯ-nɤpɤru ra}\hspace{5pt}\pcmn{种庄稼的时候一定要有人管理}\end{exemple}\relationsémantique{参考}{\lien{ⓔruⓗ1}{ru₁}}\end{entrée}

\begin{entrée}{nɤpɤri}{}{ⓔnɤpɤri} 
\classe{vi}  
\grammaire{denom} \paradigme{dir}{tɤ-}\paradigme{dir}{tɤ-}
\begin{définition}\pfra{diner}\end{définition}
\begin{définition}\pcmn{吃晚饭}\end{définition}
\begin{définition}\pfra{donner un diner}\end{définition}
\begin{définition}\pcmn{给……吃晚餐}\end{définition}
\begin{exemple}\pjya{tɤ-nɤpɤri-a}\hspace{5pt}\pcmn{我已经吃了晚餐}\end{exemple}
\begin{exemple}\pjya{jɯɣmɯr kɤ-nɯ-rma tɕe, tɯ-rca nɤpɤri-j}\hspace{5pt}\pcmn{你今天晚上留宿,我们一起吃晚餐}\end{exemple}\relationsémantique{参考}{\lien{ⓔtɤ-pɤri}{tɤ-pɤri}}
\begin{sous-entrée}{znɤpɤri}{ⓔnɤpɤriⓝznɤpɤri} 
\classe{vt} \end{sous-entrée}

\end{entrée}

\begin{entrée}{nɤpe}{}{ⓔnɤpe}\relationsémantique{参考}{\lien{ⓔpe}{pe}}\end{entrée}

\begin{entrée}{nɤphɤtphɤt}{}{ⓔnɤphɤtphɤt} 
\classe{vt} \paradigme{dir}{kɤ-}\paradigme{dir}{kɤ-}
\begin{définition}\pfra{tapoter}\end{définition}
\begin{définition}\pcmn{(轻轻地)拍打}\end{définition}
\begin{définition}\pfra{tapoter avec}\end{définition}
\begin{définition}\pcmn{用……(轻轻地)拍打}\end{définition}
\begin{exemple}\pjya{tɤ-pɤtso kɤ-nɤphɤtphat-a kɤ-znɯʑɯβ-a}\hspace{5pt}\pcmn{我拍打小孩子让他睡着了}\end{exemple}
\begin{sous-entrée}{znɤphɤtphɤt}{ⓔnɤphɤtphɤtⓝznɤphɤtphɤt} 
\classe{vt} \end{sous-entrée}

\end{entrée}

\begin{entrée}{nɤphɯphɣo}{}{ⓔnɤphɯphɣo} 
\classe{vi}  
\grammaire{n.orient} \paradigme{dir}{\_}
\begin{définition}\pfra{fuir dans tous les sens}\end{définition}
\begin{définition}\pcmn{逃来逃去;到处乱跑}\end{définition}
\begin{exemple}\pjya{aʑo pɯ-nɤphɯphɣo-a ntsɯ}\hspace{5pt}\pcmn{我到处乱跑了}\end{exemple}\relationsémantique{参考}{\lien{ⓔphɣo}{phɣo}}\end{entrée}

\begin{entrée}{nɤphɯphɯ}{}{ⓔnɤphɯphɯ} 
\classe{vi} \paradigme{dir}{thɯ-}
\begin{définition}\pfra{mendier}\end{définition}
\begin{définition}\pcmn{乞讨【讨口】}\end{définition}
\begin{exemple}\pjya{ɯʑo thɯ-nɤphɯphɯ}\hspace{5pt}\pcmn{他乞讨了}\end{exemple}
\begin{sous-entrée}{kɯ-nɤphɯphɯ}{ⓔnɤphɯphɯⓝkɯ-nɤphɯphɯ}
\begin{définition}\pfra{mendiant}\end{définition}
\begin{définition}\pcmn{乞丐}\end{définition}\end{sous-entrée}

\end{entrée}

\begin{entrée}{nɤphɯxtsɯ}{}{ⓔnɤphɯxtsɯ} 
\classe{vi}  
\grammaire{incorp} \paradigme{dir}{thɯ-}\paradigme{dir}{pɯ-}
\begin{définition}\pfra{écraser les mottes de terre}\end{définition}
\begin{définition}\pcmn{打土巴}\end{définition}
\begin{exemple}\pjya{aʑo pɯ-nɤphɯxtsɯ-a}\hspace{5pt}\pcmn{我打了土巴}\end{exemple}\relationsémantique{参考}{\lien{ⓔtɤphɯxtsɯ}{tɤphɯxtsɯ}}\end{entrée}

\begin{entrée}{nɤpra}{}{ⓔnɤpra} 
\classe{vi}  
\grammaire{denom} \paradigme{dir}{tɤ-}
\begin{définition}\pfra{être envoyé}\end{définition}
\begin{définition}\pcmn{被人致使/派去}\end{définition}
\begin{exemple}\pjya{aʑo tɤ-nɤpra-a pɯ-ɕti wo}\hspace{5pt}\pcmn{我是被人派来的}\end{exemple}\relationsémantique{参考}{\lien{ⓔtɤpra}{tɤpra}}\relationsémantique{参考}{\lien{ⓔɣɤxpra}{ɣɤxpra}}\end{entrée}

\begin{entrée}{nɤprɯ}{}{ⓔnɤprɯ} 
\classe{vi}  
\grammaire{denom} \paradigme{dir}{kɤ-}
\begin{définition}\pfra{se protéger de la pluie}\end{définition}
\begin{définition}\pcmn{遮雨}\end{définition}
\begin{exemple}\pjya{@san ɯ-pa nɤprɯ-tɕi}\hspace{5pt}\pcmn{我们俩用伞遮雨}\end{exemple}
\begin{exemple}\pjya{praχpa nɤprɯ-tɕi}\hspace{5pt}\pcmn{我们俩在悬崖下遮雨}\end{exemple}
\begin{exemple}\pjya{jɤɣɤt ɯ-pa kɤ-nɤprɯ-a}\hspace{5pt}\pcmn{我在走缘下遮雨了}\end{exemple}
\begin{exemple}\pjya{tɯ-mɯ ɲɯ-ɤsɯ-lɤt tɕe, kɤ-nɤprɯ-a}\hspace{5pt}\pcmn{下雨的时候,我躲雨了}\end{exemple}\relationsémantique{参考}{\lien{ⓔtɤprɯ}{tɤprɯ}}\end{entrée}

\begin{entrée}{nɤpɯpa}{}{ⓔnɤpɯpa} 
\classe{vt} \paradigme{dir}{tɤ-}
\begin{définition}\pfra{s'occuper de}\end{définition}
\begin{définition}\pcmn{照顾}\end{définition}\relationsémantique{同义词}{\lien{ⓔnɯβdaʁ}{nɯβdaʁ}}\end{entrée}

\begin{entrée}{nɤpɯprɤt}{}{ⓔnɤpɯprɤt}\relationsémantique{参考}{\lien{ⓔprɤt}{prɤt}}\end{entrée}

\begin{entrée}{nɤqa}{}{ⓔnɤqa} 
\classe{vt}  
\grammaire{denom} \paradigme{dir}{thɯ-}
\begin{définition}\pfra{arracher à la racine}\end{définition}
\begin{définition}\pcmn{拔根}\end{définition}
\begin{exemple}\pjya{thɯ-nɤqa-t-a}\hspace{5pt}\pcmn{我拔了根}\end{exemple}
\begin{exemple}\pjya{ki sɯjno thɯ-nɤqa-t-a}\hspace{5pt}\pcmn{我拔了这个草}\end{exemple}\relationsémantique{参考}{\lien{ⓔtɯ-qa}{tɯ-qa}}\end{entrée}

\begin{entrée}{nɤqadrɤt}{}{ⓔnɤqadrɤt} 
\classe{vt}  
\grammaire{incorp} \paradigme{dir}{thɯ-}
\begin{définition}\pfra{griffer partout}\end{définition}
\begin{définition}\pcmn{到处乱抓(鸟)}\end{définition}
\begin{exemple}\pjya{kumpɣa kɯ ɯ-ndza cho qajɯ ɲɯ-ɕar tɕe, sɤtɕha aʁɤndɯndɤt ʑo chɯ-nɤqadrɤt ŋu}\hspace{5pt}\pcmn{鸡在找食物和虫子的时候,会到到处乱抓地面}\end{exemple}\relationsémantique{参考}{\lien{ⓔadrɤt}{adrɤt}}\relationsémantique{参考}{\lien{ⓔtɯ-qa}{tɯ-qa}}\end{entrée}

\begin{entrée}{nɤqɤrqɤr}{}{ⓔnɤqɤrqɤr}\relationsémantique{参考}{\lien{ⓔqɤr}{qɤr}}\end{entrée}

\begin{entrée}{nɤqɤtsa}{}{ⓔnɤqɤtsa} 
\classe{vt}  
\grammaire{trop} \paradigme{dir}{tɤ-}
\begin{définition}\pfra{adéquat}\end{définition}
\begin{définition}\pcmn{合适}\end{définition}
\begin{exemple}\pjya{kɯki rɟɯma ki nɯtɕu tú-wɣ-rku ɲɯ-jɤɣ ma ɲɯ-nɤqɤtse}\hspace{5pt}\pcmn{这个螺丝可以装在那里,因为刚合适}\end{exemple}
\begin{sous-entrée}{anɤqɤtsɯtsa}{ⓔnɤqɤtsaⓝanɤqɤtsɯtsa} 
\classe{vi}  
\grammaire{recip} 
\begin{définition}\pfra{adéquat}\end{définition}
\begin{définition}\pcmn{合适;相配}\end{définition}
\begin{exemple}\pjya{ɯ-fkɯm cho ɯ-ŋgɯ nɯ ɲɯ-ɤnɤqɤtsɯtsa-ndʑi}\hspace{5pt}\pcmn{口袋和内容刚刚合适}\end{exemple}\relationsémantique{参考}{\lien{ⓔaqɤtsa}{aqɤtsa}}\end{sous-entrée}

\end{entrée}

\begin{entrée}{nɤqɤʑa}{}{ⓔnɤqɤʑa} 
\classe{vt}  
\grammaire{incorp} \paradigme{dir}{\_}
\begin{définition}\pfra{faire complètement du début à la fin}\end{définition}
\begin{définition}\pcmn{从头到尾地做}\end{définition}
\begin{exemple}\pjya{kha tɤ-nɤqɤʑa-t-a ʑo tɤ-βzu-t-a ɕti}\end{exemple}
\begin{exemple}\pjya{aj kha tɤ-nɤqɤʑa-t-a ɕti}\hspace{5pt}\pcmn{这个房子是我一手修建的}\end{exemple}
\begin{exemple}\pjya{tɤ-pɤtso kɤ-sɯxɕɤt pɯ-nɤqɤʑa-t-a pɯ-ra}\hspace{5pt}\pcmn{我只好从头到尾教了这个小孩子}\end{exemple}\relationsémantique{参考}{\lien{ⓔtɯ-qa}{tɯ-qa}}\relationsémantique{参考}{\lien{ⓔʑaⓗ1}{ʑa₁}}\end{entrée}

\begin{entrée}{nɤqharu}{}{ⓔnɤqharu} 
\classe{vi}  
\grammaire{incorp} \paradigme{dir}{\_}
\begin{définition}\pfra{se retourner}\end{définition}
\begin{définition}\pcmn{转身;回头}\end{définition}
\begin{exemple}\pjya{a-qhu chu lɤ-tɯ-ɣe tɕe thɯ-nɤqharu-a}\hspace{5pt}\pcmn{你从我后面来了,我回头看了一下}\end{exemple}\relationsémantique{参考}{\lien{ⓔɯ-qhu}{ɯ-qhu}}\relationsémantique{参考}{\lien{ⓔruⓗ1}{ru₁}}\relationsémantique{参考}{\lien{ⓔqharu}{qharu}}\end{entrée}

\begin{entrée}{nɤqhɤβde}{}{ⓔnɤqhɤβde} 
\classe{vt}  
\grammaire{incorp} \paradigme{dir}{nɯ-}\sens{1}
\begin{définition}\pfra{reporter à plus tard}\end{définition}
\begin{définition}\pcmn{拖到以后}\end{définition}
\begin{exemple}\pjya{nɤ-kɤnɤma ra ɲɯ-tɯ-nɤqhɤβde ntsɯ ɲɯ-ŋu}\hspace{5pt}\pcmn{你总是把事情拖到以后干}\end{exemple}\sens{2}
\begin{définition}\pfra{négliger}\end{définition}
\begin{définition}\pcmn{不管}\end{définition}
\begin{exemple}\pjya{tɤ-pɤtso kɤ-nɤqhɤβde mɤ-khɯ}\hspace{5pt}\pcmn{不能忽视小孩子}\end{exemple}\relationsémantique{参考}{\lien{ⓔɯ-qhu}{ɯ-qhu}}\relationsémantique{参考}{\lien{ⓔβde}{βde}}\end{entrée}

\begin{entrée}{nɤqhɤŋga}{}{ⓔnɤqhɤŋga} 
\classe{vt}  
\grammaire{incorp} \paradigme{dir}{thɯ-}\paradigme{dir}{kɤ-}
\begin{définition}\pfra{se mettre un habit sur les épaules}\end{définition}
\begin{définition}\pcmn{披衣}\end{définition}
\begin{exemple}\pjya{a-ŋga thɯ-nɤqhɤŋga-t-a}\hspace{5pt}\pcmn{我披上了衣服}\end{exemple}\relationsémantique{参考}{\lien{ⓔɯ-qhu}{ɯ-qhu}}\relationsémantique{参考}{\lien{ⓔŋga}{ŋga}}\end{entrée}

\begin{entrée}{nɤqhɤwɯr}{}{ⓔnɤqhɤwɯr} 
\classe{vt}  
\grammaire{denom} \paradigme{dir}{thɯ-}\paradigme{dir}{tɤ-}
\begin{définition}\pfra{se mettre un habit sur les épaules pour se protéger de la pluie}\end{définition}
\begin{définition}\pcmn{披衣挡雨}\end{définition}
\begin{exemple}\pjya{@pugai thɯ-nɤqhɤwɯr-a}\hspace{5pt}\pcmn{我披上了铺盖(被子)挡雨}\end{exemple}
\begin{exemple}\pjya{tɯ-ŋga thɯ-nɤqhɤwɯr-a}\hspace{5pt}\pcmn{我披上了衣服}\end{exemple}
\begin{exemple}\pjya{ɯʑo kɯ tɯ-ŋga tha-nɤqhɤwɯr}\hspace{5pt}\pcmn{他披上了衣服}\end{exemple}\relationsémantique{参考}{\lien{ⓔtɯwɯr}{tɯwɯr}}\end{entrée}

\begin{entrée}{nɤqhrɯmbɤβ}{}{ⓔnɤqhrɯmbɤβ} 
\classe{vi} \paradigme{dir}{tɤ-}
\begin{définition}\pfra{roter}\end{définition}
\begin{définition}\pcmn{打饱嗝}\end{définition}
\begin{exemple}\pjya{tɤ-nɤqhrɯmbɤβ}\hspace{5pt}\pcmn{他打嗝了}\end{exemple}\relationsémantique{参考}{\lien{ⓔtɯ-qhrɯmbɤβ}{tɯ-qhrɯmbɤβ}}\end{entrée}

\begin{entrée}{nɤru}{}{ⓔnɤru} 
\classe{vi} \paradigme{dir}{kɤ-}
\begin{définition}\pfra{voler de la nourriture (animaux)}\end{définition}
\begin{définition}\pcmn{偷吃粮食(牲畜)}\end{définition}
\begin{exemple}\pjya{tshɤt ɲɯ-nɤru}\hspace{5pt}\pcmn{山羊在偷吃}\end{exemple}
\begin{exemple}\pjya{nɯŋa ɲɯ-nɤru}\hspace{5pt}\pcmn{牛在偷吃}\end{exemple}
\begin{exemple}\pjya{paʁ ɲɯ-nɤru}\hspace{5pt}\pcmn{猪在偷吃}\end{exemple}
\begin{exemple}\pjya{tshɤt ko-nɤru}\hspace{5pt}\pcmn{山羊偷吃了}\end{exemple}\end{entrée}

\begin{entrée}{nɤrɤɟaʁ}{}{ⓔnɤrɤɟaʁ} 
\classe{vi} \paradigme{dir}{thɯ-}
\begin{définition}\pfra{échanger des plaisanteries}\end{définition}
\begin{définition}\pcmn{说说笑笑}\end{définition}\relationsémantique{参考}{\lien{ⓔtɤre tɤɟaʁ}{tɤre tɤɟaʁ}}\end{entrée}

\begin{entrée}{nɤrɕu}{}{ⓔnɤrɕu} 
\classe{vi} \paradigme{dir}{nɯ-}\paradigme{dir}{nɯ-}
\begin{définition}\pfra{égratigner}\end{définition}
\begin{définition}\pcmn{皮肤擦破}\end{définition}
\begin{exemple}\pjya{a-jaʁ ɲo-nɤrɕu}\hspace{5pt}\pcmn{我的手擦破了皮}\end{exemple}
\begin{exemple}\pjya{a-jaʁ nɯ-tɯ-znɤrɕu-t}\hspace{5pt}\pcmn{你擦破了我的手}\end{exemple}
\begin{sous-entrée}{znɤrɕu}{ⓔnɤrɕuⓝznɤrɕu} 
\classe{vt}  
\grammaire{caus} \end{sous-entrée}

\end{entrée}

\begin{entrée}{nɤrɕɤmŋɤm}{}{ⓔnɤrɕɤmŋɤm} 
\classe{vt}  
\grammaire{incorp} \paradigme{dir}{nɯ-}
\begin{définition}\pfra{chérir}\end{définition}
\begin{définition}\pcmn{疼爱}\end{définition}
\begin{exemple}\pjya{a-ʁi ɲɯ-nɤrɕɤmŋam-a}\hspace{5pt}\pcmn{我很疼爱我的弟弟}\end{exemple}
\begin{exemple}\pjya{ɯ-rɕa,mŋɤm}\end{exemple}\end{entrée}

\begin{entrée}{nɤre}{₁₂}{ⓔnɤreⓗ1ⓗ2} 
\classe{vi}
\classe{vt} \paradigme{dir}{nɯ-}\sens{1}
\begin{définition}\pfra{rire}\end{définition}
\begin{définition}\pcmn{笑}\end{définition}
\begin{exemple}\pjya{nɯ-kɯ-nɤre-a}\hspace{5pt}\pcmn{你笑了我}\end{exemple}
\begin{exemple}\pjya{nɯ-nɤre-a}\hspace{5pt}\pcmn{我笑了}\end{exemple}\sens{2}\paradigme{dir}{nɯ-}
\begin{définition}\pfra{se réouvrir (blessure)}\end{définition}
\begin{définition}\pcmn{重复开裂(伤口)}\end{définition}
\begin{définition}\pfra{se moquer de}\end{définition}
\begin{définition}\pcmn{笑}\end{définition}
\begin{exemple}\pjya{ɯ-tɯ-ɣmaz ɲɤ-nɤre}\hspace{5pt}\pcmn{他的伤口复裂了}\end{exemple}
\begin{exemple}\pjya{nɯ-tɯ-nɤre-t}\hspace{5pt}\pcmn{你笑了他}\end{exemple}
\begin{exemple}\pjya{nɯ-nɤre-t-a}\hspace{5pt}\pcmn{我笑了他}\end{exemple}
\begin{sous-entrée}{nɤre}{ⓔnɤreⓗ1ⓢ2ⓝnɤre}\end{sous-entrée}

\begin{sous-entrée}{znɤre}{ⓔnɤreⓗ1ⓢ2ⓝznɤre} 
\classe{vt} 
\begin{définition}\pfra{faire rire}\end{définition}
\begin{définition}\pcmn{笑令人笑}\end{définition}\end{sous-entrée}

\begin{sous-entrée}{sɤnɤre}{ⓔnɤreⓗ1ⓢ2ⓝsɤnɤre} 
\classe{vi}  
\grammaire{apass} 
\begin{définition}\pfra{se moquer des gens}\end{définition}
\begin{définition}\pcmn{取笑人}\end{définition}\end{sous-entrée}

\begin{sous-entrée}{anɤrɯre}{ⓔnɤreⓗ1ⓢ2ⓝanɤrɯre} 
\classe{vi} 
\begin{définition}\pfra{se moquer les uns des autres}\end{définition}
\begin{définition}\pcmn{互相取笑}\end{définition}
\begin{exemple}\pjya{ɲɯ-ɤnɤrɯre-ndʑi}\hspace{5pt}\pcmn{他们俩互相取笑}\end{exemple}\end{sous-entrée}

\end{entrée}

\begin{entrée}{nɤrgɤŋɯ}{}{ⓔnɤrgɤŋɯ} 
\classe{vi} \paradigme{dir}{nɯ-}
\begin{définition}\pfra{pleurer de bonheur}\end{définition}
\begin{définition}\pcmn{高兴得哭}\end{définition}
\begin{exemple}\pjya{aʑo jɤ-azɣɯt-a tɕe, a-mu ɲɯ-nɤrgɤŋɯ}\hspace{5pt}\pcmn{我到了的时候,我目前高兴得哭了出来}\end{exemple}\étymologie{rga.ŋu}\end{entrée}

\begin{entrée}{nɤrɣɤma}{}{ⓔnɤrɣɤma} 
\classe{vi} \paradigme{dir}{tɤ-}
\begin{définition}\pfra{implorer la pluie}\end{définition}
\begin{définition}\pcmn{求雨}\end{définition}
\begin{exemple}\pjya{aʑo ɕ-tɤ-nɤrɣɤma-a}\hspace{5pt}\pcmn{我去求雨了}\end{exemple}\end{entrée}

\begin{entrée}{nɤrɟɯrɟɯɣ}{}{ⓔnɤrɟɯrɟɯɣ} 
\classe{vi}  
\grammaire{n.orient} \paradigme{dir}{\_}
\begin{définition}\pfra{courir dans tous les sens}\end{définition}
\begin{définition}\pcmn{跑来跑去}\end{définition}
\begin{exemple}\pjya{aj tɤ-nɤrɟɯrɟɯɣ-a}\hspace{5pt}\pcmn{我跑来跑去了}\end{exemple}
\begin{exemple}\pjya{ɯ-kɯ-rtoʁ tɤ-nɤrɟɯrɟɯɣ-a}\hspace{5pt}\pcmn{我到处跑去看他了}\end{exemple}\relationsémantique{参考}{\lien{ⓔrɟɯɣⓗ1}{rɟɯɣ₁}}\end{entrée}

\begin{entrée}{nɤrkɤja}{}{ⓔnɤrkɤja} 
\classe{vt} \paradigme{dir}{\_}
\begin{définition}\pfra{s'occuper des animaux}\end{définition}
\begin{définition}\pcmn{管理牲畜(赶、放牧)}\end{définition}
\begin{exemple}\pjya{fsapaʁ ku-nɤrkɤje-a}\hspace{5pt}\pcmn{我在管理牲畜}\end{exemple}\end{entrée}

\begin{entrée}{nɤrkhɯrkhɯβ/\variante{\_nɤrkhɯβrkhɯβ}}{}{ⓔnɤrkhɯrkhɯβ} 
\classe{vt}  
\grammaire{deidph} \sens{1}\paradigme{dir}{kɤ-}\paradigme{dir}{nɯ-}
\begin{définition}\pfra{frapper à la porte}\end{définition}
\begin{définition}\pcmn{敲门}\end{définition}
\begin{exemple}\pjya{kɯm kɤ-nɤrkhɯrkhɯβ-a}\hspace{5pt}\pcmn{我敲了门}\end{exemple}
\begin{exemple}\pjya{na-nɤrkhɯrkhɯβ}\hspace{5pt}\pcmn{他敲了门}\end{exemple}
\begin{exemple}\pjya{kɯm ɲɤ-nɤrkhɯrkhɯβ}\hspace{5pt}\pcmn{他敲了门}\end{exemple}\sens{2}\paradigme{dir}{lɤ-}\paradigme{dir}{tɤ-}
\begin{définition}\pfra{frapper en faisant du bruit}\end{définition}
\begin{définition}\pcmn{敲响}\end{définition}
\begin{exemple}\pjya{ɯʑo kɯ kɯm la-nɤrkhɯβrkhɯβ tɕe ɯ-zgra ta-ʑmbri tɕe pɯ-mtsham-a}\hspace{5pt}\pcmn{他敲了门,发出了声音我就听到了}\end{exemple}\relationsémantique{参考}{\lien{ⓔrkhɯβrkhɯβ}{rkhɯβrkhɯβ}}
\begin{sous-entrée}{sɤrkhɯβrkhɯβ}{ⓔnɤrkhɯrkhɯβⓢ2ⓝsɤrkhɯβrkhɯβ} 
\classe{vt} \end{sous-entrée}

\end{entrée}

\begin{entrée}{nɤrko}{₁}{ⓔnɤrkoⓗ1} 
\classe{vt}  
\grammaire{trop}
\grammaire{caus}
\grammaire{habil}
\grammaire{refl} \paradigme{dir}{tɤ-}\paradigme{dir}{tɤ-}
\begin{définition}\pfra{tenir, supporter}\end{définition}
\begin{définition}\pcmn{坚持}\end{définition}
\begin{exemple}\pjya{tɤ-nɤrko-t-a}\hspace{5pt}\pcmn{我坚持了}\end{exemple}
\begin{exemple}\pjya{aj ɲɯ-ngo-a ri tɤ-nɤrko-t-a}\hspace{5pt}\pcmn{我病了但是坚持了}\end{exemple}
\begin{exemple}\pjya{nɤ-ndzɤtshi tɤ-nɤrkɤm}\hspace{5pt}\pcmn{你坚持吃吧}\end{exemple}
\begin{exemple}\pjya{nɤ-ndzɤtshi tɤ-nɤrkɤm je tɕe ʑa a-tɤ-tɯ-mna}\hspace{5pt}\pcmn{你坚持吃,希望早日康复}\end{exemple}
\begin{exemple}\pjya{mɤʑɯ laʁnɤ-rʑaʁ pɯ-ri tɕe tɤ-nɤrkɤm je}\hspace{5pt}\pcmn{只剩下几天,一定要坚持到底!}\end{exemple}
\begin{exemple}\pjya{nɤʑo nɯ sthɯci kɯ-ɤrqhi ju-tɯ-ɣi ɕti tɕe, aʑo ku-rɤʑe-a ma mɯ́j-ra tɕe, tu-nɤrkam-a}\hspace{5pt}\pcmn{你从那么远的地方来,我这里也没有事做,我会坚持的}\end{exemple}
\begin{exemple}\pjya{kɤ-nɤrko me}\hspace{5pt}\pcmn{只有那么点能力,已经尽力了}\end{exemple}
\begin{exemple}\pjya{tɤ-nɤrko-t-a tɕe tɤ-ndza-t-a pɯ-ra (= tɤrkoz ʑo tɤ-ndza-t-a pɯ-ra)}\hspace{5pt}\pcmn{我被迫吃了}\end{exemple}
\begin{sous-entrée}{znɤrko}{ⓔnɤrkoⓗ1ⓝznɤrko} 
\classe{vt} \end{sous-entrée}

\sens{1}
\begin{définition}\pfra{forcer}\end{définition}
\begin{définition}\pcmn{强迫}\end{définition}
\begin{exemple}\pjya{kɤ-ndza ta-znɤrko}\hspace{5pt}\pcmn{他强迫他吃了}\end{exemple}\sens{2}\paradigme{dir}{tɤ-}
\begin{définition}\pfra{pouvoir tenir, supporter}\end{définition}
\begin{définition}\pcmn{能坚持}\end{définition}
\begin{exemple}\pjya{ɯ-kɯ-mŋɤm ɲɯ-thɯ ri, wuma ʑo ɲɯ-znɤrkɤm}\hspace{5pt}\pcmn{他虽然病得很严重,但是还是坚持得住}\end{exemple}
\begin{exemple}\pjya{kɤ-rɤma wuma ɲɯ-znɤrkɤm}\hspace{5pt}\pcmn{他坚持工作}\end{exemple}
\begin{sous-entrée}{ʑɣɤnɤrko}{ⓔnɤrkoⓗ1ⓝʑɣɤnɤrko} 
\classe{vi} \end{sous-entrée}

\begin{définition}\pfra{se forcer}\end{définition}
\begin{définition}\pcmn{强迫自己}\end{définition}
\begin{exemple}\pjya{kɤ-ndza a-tɤ-tɯ-ʑɣɤnɤrko}\hspace{5pt}\pcmn{你强迫自己吃}\end{exemple}\relationsémantique{参考}{\lien{ⓔrko}{rko}}\relationsémantique{参考}{\lien{ⓔanɤrkɯrko}{anɤrkɯrko}}\end{entrée}

\begin{entrée}{nɤrko/\variante{nɤrkɯrko}}{₂}{ⓔnɤrkoⓗ2} 
\classe{vs} 
\begin{définition}\pfra{résistant}\end{définition}
\begin{définition}\pcmn{结实;硬;不容易变形}\end{définition}\relationsémantique{参考}{\lien{ⓔrko}{rko}}\end{entrée}

\begin{entrée}{nɤrkɯn}{}{ⓔnɤrkɯn}\relationsémantique{参考}{\lien{ⓔrkɯn}{rkɯn}}\end{entrée}

\begin{entrée}{nɤrkɯrku}{}{ⓔnɤrkɯrku} 
\classe{vt}  
\grammaire{n.orient} \paradigme{dir}{pɯ-}
\begin{définition}\pfra{verser à tout le monde}\end{définition}
\begin{définition}\pcmn{给所有人倒茶;酒等}\end{définition}
\begin{exemple}\pjya{tʂha pɯ-nɤrkɯrku-t-a}\hspace{5pt}\pcmn{我给大家倒了茶}\end{exemple}
\begin{exemple}\pjya{tʂha pɯ-nɤrkɯrke}\hspace{5pt}\pcmn{你给大家倒茶吧}\end{exemple}\relationsémantique{参考}{\lien{ⓔrku}{rku}}\end{entrée}

\begin{entrée}{nɤrme}{}{ⓔnɤrme} 
\classe{vt} \paradigme{dir}{thɯ-}\sens{1}
\begin{définition}\pfra{enlever les poils}\end{définition}
\begin{définition}\pcmn{拔毛}\end{définition}
\begin{exemple}\pjya{qaʑo thɯ-nɤrme-t-a}\hspace{5pt}\pcmn{我剪了羊毛}\end{exemple}
\begin{exemple}\pjya{pɣa thɯ-nɤrme-t-a}\hspace{5pt}\pcmn{我拔了鸡的毛}\end{exemple}\sens{2}
\begin{définition}\pfra{enlever les mauvaises herbes du sol}\end{définition}
\begin{définition}\pcmn{把地面长出来的杂草铲掉}\end{définition}\relationsémantique{参考}{\lien{ⓔtɤ-rme}{tɤ-rme}}\end{entrée}

\begin{entrée}{nɤrmi}{}{ⓔnɤrmi} 
\classe{vt} \paradigme{dir}{tɤ-}\sens{1}
\begin{définition}\pfra{dire le nom de}\end{définition}
\begin{définition}\pcmn{叫……的名字}\end{définition}
\begin{exemple}\pjya{nɤ-rca jɤ-kɯ-ɣe nɯ ɕɯ ŋu nɯ tɤ-nɤrmi ɲɯ-ra. (ɯ-rmi tɤ-βze ɲɯ-ra)}\hspace{5pt}\pcmn{跟你一起来的那个人,你说一下他叫什么名字}\end{exemple}
\begin{exemple}\pjya{tɯrme nɯ-kɯ-si nɯ ma-tɤ-tɯ-nɤrmi.}\hspace{5pt}\pcmn{已经过世了的人,不要叫出他的名字}\end{exemple}\sens{2}
\begin{définition}\pfra{accepter un nom}\end{définition}
\begin{définition}\pcmn{承认自己的名字}\end{définition}
\begin{exemple}\pjya{nɯŋa nɯ kɯ ɯ-rmi mɯ́j-nɤrmi}\hspace{5pt}\pcmn{那头牛不接受它的名字}\end{exemple}\relationsémantique{参考}{\lien{ⓔtɤ-rmi}{tɤ-rmi}}\end{entrée}

\begin{entrée}{nɤrnoʁ}{}{ⓔnɤrnoʁ} 
\classe{vt} \paradigme{dir}{nɯ-}
\begin{définition}\pfra{réfléchir}\end{définition}
\begin{définition}\pcmn{动脑筋}\end{définition}
\begin{exemple}\pjya{koŋla ɲɯ́-wɣ-nɤrnoʁ ɲɯ-ra}\hspace{5pt}\pcmn{要真的动脑筋}\end{exemple}
\begin{exemple}\pjya{tɯ-rju nɯ ɲɯ́-wɣ-nɤrnoʁ ɯ-jɯja ɲɯ-ɲɯ-ɤpɤɴqa ɲɯ-ɕti}\hspace{5pt}\pcmn{我们的语言分析得越深入就越复杂}\end{exemple}\relationsémantique{参考}{\lien{ⓔtɯ-rnoʁ}{tɯ-rnoʁ}}\end{entrée}

\begin{entrée}{nɤrŋi}{}{ⓔnɤrŋi} 
\classe{n} 
\begin{définition}\pfra{bébé}\end{définition}
\begin{définition}\pcmn{婴儿}\end{définition}\end{entrée}

\begin{entrée}{nɤro}{}{ⓔnɤro} 
\classe{vt} \paradigme{dir}{kɤ-}
\begin{définition}\pfra{utiliser pour la première fois}\end{définition}
\begin{définition}\pcmn{第一次用;开始吃}\end{définition}
\begin{exemple}\pjya{kɯki kɤ-ndza ki aʑo ku-nɤram-a ŋu}\hspace{5pt}\pcmn{我开始吃这顿餐}\end{exemple}
\begin{exemple}\pjya{ki qajɣi ki aj kɤ-nɤro-t-a}\hspace{5pt}\pcmn{我馍馍开始吃馍馍了}\end{exemple}\end{entrée}

\begin{entrée}{nɤrpaʁ}{}{ⓔnɤrpaʁ} 
\classe{vt}  
\grammaire{denom} \paradigme{dir}{tɤ-}\sens{1}
\begin{définition}\pfra{porter à l’épaule}\end{définition}
\begin{définition}\pcmn{扛}\end{définition}
\begin{exemple}\pjya{tɤ-nɤrpaʁ-a}\hspace{5pt}\pcmn{我扛了}\end{exemple}\sens{2}
\begin{définition}\pfra{s'entendre bien avec}\end{définition}
\begin{définition}\pcmn{和别人投合}\end{définition}
\begin{exemple}\pjya{nɯnɯ smɤnba nɯ kɯ tɤ́-wɣ-nɯsman-a tɕe ɲɯ-nɤrpaʁ-a}\hspace{5pt}\pcmn{那个医生给我治病治得非常好}\end{exemple}
\begin{exemple}\pjya{kɤ-nɤma ɲɯ-nɤrpaʁ ma kɤ-nɯ-rɤʑi mɯ́j-nɤrpaʁ}\hspace{5pt}\pcmn{他适合工作,不是适合闲着}\end{exemple}\relationsémantique{参考}{\lien{ⓔtɯ-rpaʁ}{tɯ-rpaʁ}}\relationsémantique{参考}{\lien{ⓔmɤrpaʁ}{mɤrpaʁ}}\relationsémantique{参考}{\lien{ⓔnɤrpaʁku}{nɤrpaʁku}}\relationsémantique{参考}{\lien{ⓔanɤrpɯrpaʁ}{anɤrpɯrpaʁ}}\end{entrée}

\begin{entrée}{nɤrpaʁku}{}{ⓔnɤrpaʁku} 
\classe{vt}  
\grammaire{denom} \paradigme{dir}{tɤ-}
\begin{définition}\pfra{porter à l'épaule}\end{définition}
\begin{définition}\pcmn{扛}\end{définition}
\begin{exemple}\pjya{tɤrɤm tɤ-nɤrpaʁku-t-a}\hspace{5pt}\pcmn{我扛了木板}\end{exemple}\relationsémantique{同义词}{\lien{ⓔmɤrpaʁ}{mɤrpaʁ}}\relationsémantique{参考}{\lien{ⓔtɯ-rpaʁ}{tɯ-rpaʁ}}\end{entrée}

\begin{entrée}{nɤrpɯrpu}{}{ⓔnɤrpɯrpu} 
\classe{vt}  
\grammaire{n.orient} \paradigme{dir}{\_}
\begin{définition}\pfra{se cogner partout}\end{définition}
\begin{définition}\pcmn{撞来撞去;到处乱撞}\end{définition}
\begin{exemple}\pjya{laχtɕha ma-nɯ-tɯ-nɤrpɯrpe}\hspace{5pt}\pcmn{你不要乱撞东西}\end{exemple}\relationsémantique{参考}{\lien{ⓔrpu}{rpu}}\end{entrée}

\begin{entrée}{nɤrqaʁ}{}{ⓔnɤrqaʁ} 
\classe{vt} \paradigme{dir}{nɯ-}
\begin{définition}\pfra{enlever la peau (navet)}\end{définition}
\begin{définition}\pcmn{剥皮(圆根)}\end{définition}
\begin{exemple}\pjya{na-nɤrqaʁ}\hspace{5pt}\pcmn{他剥了皮}\end{exemple}
\begin{exemple}\pjya{rasti nɯ-nɤrqaʁ}\hspace{5pt}\pcmn{你剥圆根的皮吧}\end{exemple}
\begin{exemple}\pjya{rasti nɯ-nɤrqaʁ-a}\hspace{5pt}\pcmn{我剥了圆根的皮}\end{exemple}\end{entrée}

\begin{entrée}{nɤrqhu}{}{ⓔnɤrqhu} 
\classe{vt}  
\grammaire{denom} \paradigme{dir}{pɯ-}\paradigme{dir}{thɯ-}
\begin{définition}\pfra{éplucher, décortiquer}\end{définition}
\begin{définition}\pcmn{用手剥}\end{définition}
\begin{exemple}\pjya{pejka pa-nɤrqhu}\hspace{5pt}\pcmn{他把白瓜剥了}\end{exemple}
\begin{exemple}\pjya{si pa-nɤrqhu}\hspace{5pt}\pcmn{他把树皮剥了}\end{exemple}
\begin{exemple}\pjya{ʑɴɢɯloʁ pa-nɤrqhu}\hspace{5pt}\pcmn{他把核桃剥了壳}\end{exemple}
\begin{exemple}\pjya{ʑɴɢɯloʁ pɯ-nɤrqhe}\hspace{5pt}\pcmn{你把核桃壳剥一下}\end{exemple}\relationsémantique{参考}{\lien{ⓔtɤ-rqhu}{tɤ-rqhu}}\end{entrée}

\begin{entrée}{nɤrʁaʁ}{}{ⓔnɤrʁaʁ} 
\classe{vt} \paradigme{dir}{tɤ-}
\begin{définition}\pfra{chasser}\end{définition}
\begin{définition}\pcmn{打猎}\end{définition}
\begin{exemple}\pjya{pɣɤtɕɯ tɤ-nɤrʁaʁ}\hspace{5pt}\pcmn{你打鸟吧}\end{exemple}
\begin{exemple}\pjya{qarma ɕ-tɤ-nɤrʁaʁ}\hspace{5pt}\pcmn{你去打马鸡吧}\end{exemple}
\begin{exemple}\pjya{pɣɤtɕɯ ɲɯ-ɤz-nɤrʁaʁ}\hspace{5pt}\pcmn{他在猎鸟}\end{exemple}
\begin{exemple}\pjya{tɕɤkɯ tsɯʁot ci ɣɤʑu tɕe ta-nɤrʁaʁ}\hspace{5pt}\pcmn{那边有一只野鸡,他去打了}\end{exemple}
\begin{exemple}\pjya{βʑɯ wuma ɲɯ-nɤrʁaʁ}\hspace{5pt}\pcmn{(猫)捉很多老鼠}\end{exemple}\relationsémantique{参考}{\lien{ⓔɣɤrʁaʁ}{ɣɤrʁaʁ}}\end{entrée}

\begin{entrée}{nɤrtaʁ}{₁}{ⓔnɤrtaʁⓗ1} 
\classe{vt}  
\grammaire{denom} \paradigme{dir}{thɯ-}\paradigme{dir}{pɯ-}
\begin{définition}\pfra{élaguer}\end{définition}
\begin{définition}\pcmn{砍树的枝桠}\end{définition}
\begin{exemple}\pjya{tɯrgi thɯ-nɤrtaʁ-a}\hspace{5pt}\pcmn{我砍了杉树的树叶}\end{exemple}
\begin{exemple}\pjya{si pɯ-nɤrtaʁ}\hspace{5pt}\pcmn{你砍树枝吧}\end{exemple}\relationsémantique{参考}{\lien{ⓔtɤ-rtaʁ}{tɤ-rtaʁ}}\end{entrée}

\begin{entrée}{nɤrtaʁ}{₂}{ⓔnɤrtaʁⓗ2} 
\classe{vt}  
\grammaire{trop} \paradigme{dir}{pɯ-}\paradigme{dir}{tɤ-}
\begin{définition}\pfra{trouver suffisant}\end{définition}
\begin{définition}\pcmn{觉得够}\end{définition}
\begin{exemple}\pjya{pɯ-nɤrtaʁ-a}\hspace{5pt}\pcmn{我觉得足够}\end{exemple}
\begin{exemple}\pjya{ɲɯ-nɤrtaʁ}\hspace{5pt}\pcmn{他觉得足够}\end{exemple}
\begin{exemple}\pjya{laχtɕha nɯ-kɯ-mbi-a pɯ-nɤrtaʁ-a}\hspace{5pt}\pcmn{你给的东西已经够了}\end{exemple}
\begin{exemple}\pjya{kɤ-nɤrtaʁ-nɯ pjɤ-me}\hspace{5pt}\pcmn{他们贪得无厌}\end{exemple}\relationsémantique{参考}{\lien{ⓔrtaʁ}{rtaʁ}}\end{entrée}

\begin{entrée}{nɤrte}{}{ⓔnɤrte} 
\classe{vt} \paradigme{dir}{tɤ-}
\begin{définition}\pfra{porter un chapeau}\end{définition}
\begin{définition}\pcmn{戴帽子}\end{définition}
\begin{exemple}\pjya{tɤ-rte tu-nɤrte-a ŋu}\hspace{5pt}\pcmn{我戴帽子}\end{exemple}\relationsémantique{参考}{\lien{ⓔtɤ-rte}{tɤ-rte}}\end{entrée}

\begin{entrée}{nɤrtoχpjɤt/\variante{\_nɤrtɤχpjɤt}}{}{ⓔnɤrtoχpjɤt} 
\classe{vt}  
\grammaire{comp} \paradigme{dir}{kɤ-}
\begin{définition}\pfra{observer}\end{définition}
\begin{définition}\pcmn{仔细观察;打量}\end{définition}
\begin{exemple}\pjya{ʑara kɯ tɕiʑo tu-rɤma-tɕi nɯ tú-wɣ-nɤrtoχpjɤt-tɕi ɲɯ-ŋu}\hspace{5pt}\pcmn{他们在观察我们做事}\end{exemple}
\begin{exemple}\pjya{jiɕqha nɯ kɯ kú-wɣ-nɤrtoχpjat-a ɲɯ-ŋu}\hspace{5pt}\pcmn{那个人在观察我}\end{exemple}
\begin{exemple}\pjya{nɤʑo kɯ kú-wɣ-nɤrtɤχpjat-a ɲɯ-ŋu}\hspace{5pt}\pcmn{你在观察我}\end{exemple}
\begin{exemple}\pjya{aj kɤ-ta-nɤrtɤχpjɤt}\hspace{5pt}\pcmn{我观察了你}\end{exemple}\relationsémantique{参考}{\lien{ⓔrtoʁ}{rtoʁ}}\relationsémantique{参考}{\lien{ⓔχpjɤt}{χpjɤt}}\end{entrée}

\begin{entrée}{nɤrtɯrtoʁ}{}{ⓔnɤrtɯrtoʁ}\relationsémantique{参考}{\lien{ⓔrtoʁ}{rtoʁ}}\end{entrée}

\begin{entrée}{nɤrɯra}{}{ⓔnɤrɯra} 
\classe{vi} \paradigme{dir}{\_}\sens{1}
\begin{définition}\pfra{regarder dans tous les sens, observer}\end{définition}
\begin{définition}\pcmn{四处张望;观察}\end{définition}
\begin{exemple}\pjya{tɯrme ɯ-ɣɤʑu nɯ-nɤrɯra}\hspace{5pt}\pcmn{你看一下有没有人}\end{exemple}\sens{2}
\begin{définition}\pfra{faire attention à}\end{définition}
\begin{définition}\pcmn{注意}\end{définition}
\begin{exemple}\pjya{tshɤt ra pɯ-nɤrɯra}\hspace{5pt}\pcmn{你看着山羊}\end{exemple}
\begin{exemple}\pjya{tɤ-pɤtso nɯ-nɤrɯra je}\hspace{5pt}\pcmn{你看着小孩子}\end{exemple}
\begin{exemple}\pjya{nɯ-nɤrɯra je ma a-mɤ-tɤ-ta-xtsɯɣ}\hspace{5pt}\pcmn{你小心不要被我打到}\end{exemple}\end{entrée}

\begin{entrée}{nɤrwa}{}{ⓔnɤrwa} 
\classe{n} 
\begin{définition}\pfra{pâturage}\end{définition}
\begin{définition}\pcmn{牧场}\end{définition}\étymologie{nor.ba}\end{entrée}

\begin{entrée}{nɤrwɯ}{}{ⓔnɤrwɯ} 
\classe{n} 
\begin{définition}\pfra{trésor}\end{définition}
\begin{définition}\pcmn{宝贝}\end{définition}\étymologie{nor.bu}\end{entrée}

\begin{entrée}{nɤrʑaβ}{}{ⓔnɤrʑaβ} 
\classe{vt}  
\grammaire{denom} \paradigme{dir}{kɤ-}
\begin{définition}\pfra{se marier (garçon)}\end{définition}
\begin{définition}\pcmn{娶妻子}\end{définition}
\begin{exemple}\pjya{tɯrme ɯ-rʑaβ ka-nɤrʑaβ}\hspace{5pt}\pcmn{他娶了人家的妻子}\end{exemple}\relationsémantique{参考}{\lien{ⓔtɤ-rʑaβ}{tɤ-rʑaβ}}\relationsémantique{参考}{\lien{ⓔmɤrʑaβ}{mɤrʑaβ}}\end{entrée}

\begin{entrée}{nɤrʑaʁ}{}{ⓔnɤrʑaʁ} 
\classe{vi}  
\grammaire{denom} \paradigme{dir}{tɤ-}\paradigme{dir}{\_}
\begin{définition}\pfra{rester longtemps}\end{définition}
\begin{définition}\pcmn{待很久}\end{définition}
\begin{exemple}\pjya{alo thɯ-ɣe-a tɕe, jɤxtshi aj mbarkhom thɯ-nɤrʑaʁ-a}\hspace{5pt}\pcmn{这一次,我在马尔康待了很久}\end{exemple}
\begin{exemple}\pjya{japa pɯ-nɤrʑaʁ-a, ɣɯjpa ʑatsa ju-ɕe-a ra}\hspace{5pt}\pcmn{我去年待了很长时间,今年要早点回去}\end{exemple}
\begin{exemple}\pjya{fsapaʁ thɯ-sci mɤ-kɯ-nɤrʑaʁ tɕe, tu-ŋke ɕti}\hspace{5pt}\pcmn{牲畜出生不久就会走路}\end{exemple}\relationsémantique{参考}{\lien{}{tɤ-rʑaʁ₁}}\end{entrée}

\begin{entrée}{nɤʁarphɤβ}{}{ⓔnɤʁarphɤβ} 
\classe{vt}  
\grammaire{incorp} \paradigme{dir}{pɯ-}
\begin{définition}\pfra{frapper avec ses ailes}\end{définition}
\begin{définition}\pcmn{用翅膀拍打}\end{définition}
\begin{exemple}\pjya{qaliaʁ kɯ paʁtsa pjɤ-nɤʁarphɤβ}\hspace{5pt}\pcmn{老鹰用翅膀拍打了小猪}\end{exemple}\relationsémantique{参考}{\lien{ⓔʁarphɤβ}{ʁarphɤβ}}\end{entrée}

\begin{entrée}{nɤʁaʁ}{}{ⓔnɤʁaʁ} 
\classe{vi} \sens{1}\paradigme{dir}{nɯ-}
\begin{définition}\pfra{faire la fête, s'amuser}\end{définition}
\begin{définition}\pcmn{休息}\end{définition}\sens{2}\paradigme{dir}{pɯ-}
\begin{définition}\pfra{se prélasser au soleil}\end{définition}
\begin{définition}\pcmn{晒太阳}\end{définition}
\begin{exemple}\pjya{ɯʑo pɯ-nɤʁaʁ}\hspace{5pt}\pcmn{他晒了太阳}\end{exemple}
\begin{exemple}\pjya{jisŋi tɤŋe ɲɯ-wxti, pɯ-nɤʁaʁ-a}\hspace{5pt}\pcmn{今天太阳很大,我晒了太阳}\end{exemple}\relationsémantique{参考}{\lien{ⓔtɤʁaʁ}{tɤʁaʁ}}\relationsémantique{参考}{\lien{ⓔnɤʁɯmʁaʁ}{nɤʁɯmʁaʁ}}\end{entrée}

\begin{entrée}{nɤʁdɤn}{}{ⓔnɤʁdɤn} 
\classe{vt}  
\grammaire{denom} \sens{1}\paradigme{dir}{pɯ-}\paradigme{dir}{tɤ-}
\begin{définition}\pfra{placer sous}\end{définition}
\begin{définition}\pcmn{垫}\end{définition}\sens{2}\paradigme{dir}{kɤ-}\paradigme{dir}{nɯ-}
\begin{définition}\pfra{inviter}\end{définition}
\begin{définition}\pcmn{邀请(敬语)}\end{définition}\paradigme{dir}{pɯ-}\paradigme{dir}{tɤ-}
\begin{définition}\pfra{placer quelque chose sous}\end{définition}
\begin{définition}\pcmn{垫}\end{définition}
\begin{exemple}\pjya{smɤnmi mitoʁ kuɕana ɕ-ku-nɤʁdɤn tɕe, tɕetha aʑo a-kɯ-mŋɤm phɤn ɕti}\hspace{5pt}\pcmn{如果他请到古夏纳的话,我的病马上就会好}\end{exemple}
\begin{exemple}\pjya{@bandeng ɯ-pa tɤ-znɤʁdɤn}\hspace{5pt}\pcmn{你在板凳的下面垫一块东西(保护地板)}\end{exemple}
\begin{exemple}\pjya{si kɯ pɯ-znɤʁdɤn}\hspace{5pt}\pcmn{你用木块垫一下}\end{exemple}
\begin{exemple}\pjya{@dianshi ɯ-ʁdɤn ci ɲɯ-ra, tɕe si pɯ-znɤʁdan-a}\hspace{5pt}\pcmn{电视需要东西垫着,我用木头垫了一些}\end{exemple}
\begin{exemple}\pjya{tɤ-mkɯm kɯ tɤ-znɤʁdɤn}\hspace{5pt}\pcmn{你用枕头垫一下吧}\end{exemple}
\begin{exemple}\pjya{jɯfɕɯr tɤ-mkɯm kutɕu pɯ-tɯ-ta-t tɕe, kɯki tɤ-tɯ-znɤʁdɤn}\hspace{5pt}\pcmn{你昨天放了枕头,用来垫这个(话筒)}\end{exemple}
\begin{exemple}\pjya{khɯtsa ɯ-pa nɯ tɕu ɕico kɤ-znɤʁdɤn mɤ-sna}\hspace{5pt}\pcmn{在碗下面不能用塑料垫着}\end{exemple}\relationsémantique{参考}{\lien{ⓔtɤ-ʁdɤn}{tɤ-ʁdɤn}}
\begin{sous-entrée}{znɤʁdɤn}{ⓔnɤʁdɤnⓝznɤʁdɤn}\end{sous-entrée}

\étymologie{gdan}\end{entrée}

\begin{entrée}{nɤʁndɯʁndɯ}{}{ⓔnɤʁndɯʁndɯ} 
\classe{vt}  
\grammaire{n.orient} \paradigme{dir}{tɤ-}
\begin{définition}\pfra{battre à n'importe quelle occasion}\end{définition}
\begin{définition}\pcmn{打来打去;随便乱打}\end{définition}
\begin{exemple}\pjya{tshɤt qaʑo ra tɤ-nɤʁndɯʁndɯ-t-a}\hspace{5pt}\pcmn{我打了山羊和绵羊}\end{exemple}\end{entrée}

\begin{entrée}{nɤʁnoŋ}{}{ⓔnɤʁnoŋ} 
\classe{vi} \paradigme{dir}{nɯ-}
\begin{définition}\pfra{trouver dommage, regretter}\end{définition}
\begin{définition}\pcmn{觉得可惜;后悔}\end{définition}
\begin{exemple}\pjya{@beibei pɯ-qrɯ-t-a tɕe ɲɯ-nɤʁnoŋ-a}\hspace{5pt}\pcmn{我打破了杯子,觉得很可惜}\end{exemple}
\begin{exemple}\pjya{@beibei pa-qrɯ tɕe ɲɯ-nɤʁnoŋ}\hspace{5pt}\pcmn{他打破了杯子,觉得很可惜}\end{exemple}
\begin{exemple}\pjya{khɯtsa pɯ-kɤ-qrɯ nɯ kɤ-nɤʁnoŋ me}\hspace{5pt}\pcmn{这个碗打破了,没有关系}\end{exemple}\étymologie{gnoŋ}\end{entrée}

\begin{entrée}{nɤʁombi}{}{ⓔnɤʁombi} 
\classe{vt} \paradigme{dir}{nɯ-}
\begin{définition}\pfra{perdre l'espoir}\end{définition}
\begin{définition}\pcmn{灰心;对……没有希望}\end{définition}
\begin{exemple}\pjya{jiɕqha kɤ-nɤma nɯ wuma ʑo ɲɯ-ɴqa tɕe, nɯ-nɤʁombi-t-a ɕti}\hspace{5pt}\pcmn{这件事情做起来很难,我对它没有希望了}\end{exemple}\relationsémantique{参考}{\lien{ⓔtɯ-ʁo,mbi}{tɯ-ʁo,mbi}}\relationsémantique{参考}{\lien{ⓔsɤʁombi}{sɤʁombi}}\end{entrée}

\begin{entrée}{nɤʁɯmʁaʁ}{}{ⓔnɤʁɯmʁaʁ} 
\classe{vi} 
\begin{définition}\pfra{s'amuser à droite et à gauche}\end{définition}
\begin{définition}\pcmn{到处玩耍}\end{définition}\relationsémantique{参考}{\lien{ⓔnɤʁaʁ}{nɤʁaʁ}}\end{entrée}

\begin{entrée}{nɤsaʁdɯɣ}{}{ⓔnɤsaʁdɯɣ} 
\classe{vt}  
\grammaire{trop} 
\begin{définition}\pfra{trouver désagréable}\end{définition}
\begin{définition}\pcmn{觉得讨厌}\end{définition}
\begin{exemple}\pjya{nɤ-kɤcu si nɯ ɲɯ-nɤsaʁdɯɣ-a}\hspace{5pt}\pcmn{你那边的树,我觉得很麻烦}\end{exemple}
\begin{exemple}\pjya{nɤ-kɤcu laχtɕha nɯ ɲɯ-nɤsaʁdɯɣ-a}\hspace{5pt}\pcmn{你那边的东西,我觉得很麻烦}\end{exemple}
\begin{exemple}\pjya{aʑo kutɕu ku-rɤʑit-a ɣɯ-nɤsaʁdɯɣ-a ɲɯ-sɯsam-a}\hspace{5pt}\pcmn{我想他觉得我在这里很碍事}\end{exemple}\relationsémantique{参考}{\lien{ⓔʁdɯɣⓗ1}{ʁdɯɣ₁}}\end{entrée}

\begin{entrée}{nɤsɤɕke}{}{ⓔnɤsɤɕke}\relationsémantique{参考}{\lien{ⓔsɤɕkeⓗ1}{sɤɕke₁}}\end{entrée}

\begin{entrée}{nɤsɤɣ}{}{ⓔnɤsɤɣ} 
\classe{vt} \paradigme{dir}{tɤ-}\paradigme{dir}{tɤ-}
\begin{définition}\pfra{être jaloux}\end{définition}
\begin{définition}\pcmn{吃醋}\end{définition}
\begin{définition}\pfra{être jaloux}\end{définition}
\begin{définition}\pcmn{吃醋}\end{définition}
\begin{exemple}\pjya{jiɕqha kɯ tú-wɣ-nɤsaɣ-a ɲɯ-ŋu}\hspace{5pt}\pcmn{他吃我的醋}\end{exemple}
\begin{exemple}\pjya{jiɕqha nɯ tɤ-nɤsaɣ-a}\hspace{5pt}\pcmn{我吃他的醋}\end{exemple}
\begin{exemple}\pjya{ɲɯ-sɤnɤsɤɣ}\hspace{5pt}\pcmn{他在吃醋}\end{exemple}
\begin{exemple}\pjya{tu-kɯ-nɤsɤɣ a-pɯ-ŋu tɕe, tɕe ɯ-rʑaβ ɲɯ́-wɣ-nɤmthɯn ndʐa ɲɯ-ŋu}\hspace{5pt}\pcmn{别人吃自己的醋,是因为自己喜欢上别人的妻子}\end{exemple}
\begin{sous-entrée}{sɤnɤsɤɣ}{ⓔnɤsɤɣⓝsɤnɤsɤɣ} 
\classe{vi}  
\grammaire{apass} \end{sous-entrée}

\end{entrée}

\begin{entrée}{nɤsɤɣdɯɣ}{}{ⓔnɤsɤɣdɯɣ} 
\classe{vt}  
\grammaire{trop} \paradigme{dir}{nɯ-}
\begin{définition}\pfra{se sentir mal, s'ennuyer}\end{définition}
\begin{définition}\pcmn{觉得不舒服;觉得无聊}\end{définition}
\begin{exemple}\pjya{na-nɤsɤɣdɯɣ}\hspace{5pt}\pcmn{他觉得不舒服}\end{exemple}
\begin{exemple}\pjya{tɯrme kɯ tɯ-rju mɤ-kɯ-mpɕɤr ta-tɯt tɕe, ɯʑo kɯ na-nɤsɤɣdɯɣ}\hspace{5pt}\pcmn{有人说了不好听的话,他就觉得不舒服}\end{exemple}
\begin{exemple}\pjya{aʑo nɤ-phe ku-rɤʑit-a ɲɯ-tɯ-nɤsɤɣdɯɣ}\hspace{5pt}\pcmn{我住在你这里,你觉得很麻烦}\end{exemple}\relationsémantique{参考}{\lien{ⓔdɯɣⓝsɤɣdɯɣ}{sɤɣdɯɣ}}\relationsémantique{参考}{\lien{ⓔdɯɣ}{dɯɣ}}\end{entrée}

\begin{entrée}{nɤsɤjloʁ}{}{ⓔnɤsɤjloʁ}\relationsémantique{参考}{\lien{ⓔsɤjloʁ}{sɤjloʁ}}\end{entrée}

\begin{entrée}{nɤsɤre}{}{ⓔnɤsɤre}\relationsémantique{参考}{\lien{ⓔsɤre}{sɤre}}\end{entrée}

\begin{entrée}{nɤsɤscit}{}{ⓔnɤsɤscit} 
\classe{vt}  
\grammaire{trop} \paradigme{dir}{pɯ-}
\begin{définition}\pfra{trouver agréable}\end{définition}
\begin{définition}\pcmn{觉得舒服}\end{définition}
\begin{exemple}\pjya{wo, ki sɤtɕha ki ɲɯ-nɤsɤscit-a rcanɯ}\hspace{5pt}\pcmn{我觉得这个地方很舒服}\end{exemple}
\begin{exemple}\pjya{ki tɯ-ŋga ki ɲɯ-nɤsɤscit-a}\hspace{5pt}\pcmn{我觉得这件衣服穿起来很舒服}\end{exemple}
\begin{exemple}\pjya{jɯfɕɯr pɯ-nɤʁaʁ-i, pɯ-nɤsɤscit-a}\hspace{5pt}\pcmn{我们昨天晒太阳,我觉得很开心}\end{exemple}\relationsémantique{参考}{\lien{ⓔscit}{scit}}\relationsémantique{参考}{\lien{ⓔsɤscit}{sɤscit}}\end{entrée}

\begin{entrée}{nɤscɤlɤt}{}{ⓔnɤscɤlɤt} 
\classe{vt}  
\grammaire{comp} 
\begin{définition}\pfra{aller chercher et ramener}\end{définition}
\begin{définition}\pcmn{接送}\end{définition}
\begin{exemple}\pjya{nɤʑo chɤ-tɯ-wxti tɕe nɯ ma nɤ-kɯ-nɤscɤlɤt mɤ-ra}\hspace{5pt}\pcmn{你长大了,不再需要人接送}\end{exemple}\relationsémantique{参考}{\lien{ⓔsco}{sco}}\relationsémantique{参考}{\lien{ⓔlɤtⓗ1}{lɤt₁}}\end{entrée}

\begin{entrée}{nɤscɤr}{}{ⓔnɤscɤr} 
\classe{vi} \paradigme{dir}{pɯ-}\paradigme{dir}{pɯ-}
\begin{définition}\pfra{être saisi de frayeur}\end{définition}
\begin{définition}\pcmn{受惊}\end{définition}
\begin{définition}\pfra{effrayer}\end{définition}
\begin{définition}\pcmn{惊吓}\end{définition}
\begin{exemple}\pjya{pɯ-nɤscar-a}\hspace{5pt}\pcmn{我吓了一跳}\end{exemple}
\begin{exemple}\pjya{@laba tha-ʑmbri tɕe pɯ́-wɣ-znɤscar-a}\hspace{5pt}\pcmn{他吹了喇叭,把我吓了一跳}\end{exemple}
\begin{exemple}\pjya{a-mgɯr zɯ taχphe ta-lɤt tɕe pɯ́-wɣ-znɤscar-a}\hspace{5pt}\pcmn{他用手掌拍了我的背部,把我吓了一跳}\end{exemple}\relationsémantique{参考}{\lien{}{tɤ-scɤr}}
\begin{sous-entrée}{znɤscɤr}{ⓔnɤscɤrⓝznɤscɤr} 
\classe{vt}  
\grammaire{caus} \end{sous-entrée}

\end{entrée}

\begin{entrée}{nɤsci}{}{ⓔnɤsci} 
\classe{vt} \paradigme{dir}{tɤ-}\paradigme{dir}{nɯ-}\paradigme{dir}{thɯ-}
\begin{définition}\pfra{changer}\end{définition}
\begin{définition}\pcmn{换}\end{définition}
\begin{exemple}\pjya{tɯthɯ ɯ-ŋgɯ tɯ-ci ka-nɤsci}\hspace{5pt}\pcmn{他换了锅子里的水}\end{exemple}
\begin{exemple}\pjya{a-ŋga tɤ-nɤsci-t-a}\hspace{5pt}\pcmn{我换了衣服}\end{exemple}
\begin{exemple}\pjya{ɯ-ŋga ta-nɤsci}\hspace{5pt}\pcmn{他换了衣服}\end{exemple}
\begin{exemple}\pjya{a-ŋga ɲɤ-ci tɕe, tɤ-nɤsci-t-a}\hspace{5pt}\pcmn{我的衣服湿了,我来换}\end{exemple}
\begin{exemple}\pjya{a-ma tɤ-nɤsci-t-a}\hspace{5pt}\pcmn{我换了工作}\end{exemple}
\begin{exemple}\pjya{a-sɤtɕha nɯ-nɤsci-t-a}\hspace{5pt}\pcmn{我换了住的地方}\end{exemple}
\begin{exemple}\pjya{laʁdɯn ɯ-jɯ thɯ-nɤsci-t-a}\hspace{5pt}\pcmn{我换了工具的把子}\end{exemple}\relationsémantique{同义词}{\lien{ⓔsɤndu}{sɤndu}}\end{entrée}

\begin{entrée}{nɤscɯɕa}{}{ⓔnɤscɯɕa} 
\classe{vt} \paradigme{dir}{pɯ-}
\begin{définition}\pfra{tanner}\end{définition}
\begin{définition}\pcmn{把皮子刮干净(鞣制)}\end{définition}
\begin{exemple}\pjya{pɯ-nɤscɯɕa-t-a}\hspace{5pt}\pcmn{我刮干净了}\end{exemple}
\begin{exemple}\pjya{kɯ-mɤku, tɯ-ndʐi nɯ pjɯ́-wɣ-nɤscɯɕa tɕe nɯ kóʁmɯz nɤ ta-mar kú-wɣ-mar tɕe z-ɲɯ́-wɣ-χtsɤβ ra}\hspace{5pt}\pcmn{首先要把皮子刮干净,然后涂一层油,然后揉}\end{exemple}\end{entrée}

\begin{entrée}{nɤscɯscɤt}{}{ⓔnɤscɯscɤt}\relationsémantique{参考}{\lien{ⓔscɤt}{scɤt}}\end{entrée}

\begin{entrée}{nɤsma}{}{ⓔnɤsma} 
\classe{vt} \paradigme{dir}{pɯ-}\paradigme{dir}{nɯ-}
\begin{définition}\pfra{admirer}\end{définition}
\begin{définition}\pcmn{羡慕}\end{définition}
\begin{exemple}\pjya{a-zda ra nɯ ɲɯ-ɤro-nɯ tɕe ɲɯ-nɤsme-a-nɯ}\hspace{5pt}\pcmn{他们拥有那个东西,我很羡慕他们}\end{exemple}
\begin{exemple}\pjya{jɯlco ra kɯ wuma ʑo pjɤ́-wɣ-nɤsma-ndʑi}\hspace{5pt}\pcmn{村民们很羡慕他们}\end{exemple}
\begin{exemple}\pjya{kɯ-mɤɕi cho kɯ-ɤchɯcha nɯra mɤ-nɤsme-a}\hspace{5pt}\pcmn{我不羡慕有钱和才能的人}\end{exemple}
\begin{sous-entrée}{sɤnɤsma}{ⓔnɤsmaⓝsɤnɤsma} 
\classe{vi}  
\grammaire{apass} 
\begin{définition}\pfra{admirer les autres}\end{définition}
\begin{définition}\pcmn{羡慕别人}\end{définition}\end{sous-entrée}

\begin{sous-entrée}{sɤsma}{ⓔnɤsmaⓝsɤsma} 
\classe{vs}  
\grammaire{deexp} 
\begin{définition}\pfra{admirable}\end{définition}
\begin{définition}\pcmn{值得羡慕}\end{définition}
\begin{exemple}\pjya{ɯʑo kɯ-sɤsma ci ŋu}\hspace{5pt}\pcmn{他是个值得羡慕的人}\end{exemple}\end{sous-entrée}

\end{entrée}

\begin{entrée}{nɤsna}{}{ⓔnɤsna} 
\classe{vt}  
\grammaire{trop} \paradigme{dir}{tɤ-}
\begin{définition}\pfra{vouloir avoir}\end{définition}
\begin{définition}\pcmn{想要}\end{définition}
\begin{exemple}\pjya{ɲɯ-nɤsne-a}\hspace{5pt}\pcmn{我想要}\end{exemple}
\begin{exemple}\pjya{ɯ-phɯ ɲɯ-wxti tɕe, ma-tɤ-tɯ-nɤsne}\hspace{5pt}\pcmn{这个东西很贵,你不能要(我给不起)}\end{exemple}
\begin{exemple}\pjya{tɤ-mthɯm mɯ-nɯ-kɯ-ɣɤdi ra nɤsna-j ma nɯ-kɯ-ɣɤdi ra mɤ-nɤsna-j}\hspace{5pt}\pcmn{我想要没变味的肉,不想要变味了的肉}\end{exemple}\relationsémantique{参考}{\lien{ⓔsna}{sna}}\end{entrée}

\begin{entrée}{nɤsnɯndo}{}{ⓔnɤsnɯndo} 
\classe{vt}  
\grammaire{incorp} \paradigme{dir}{kɤ-}
\begin{définition}\pfra{en vouloir à, garder rancune envers}\end{définition}
\begin{définition}\pcmn{对……怀恨在心}\end{définition}
\begin{exemple}\pjya{kɤ-nɤsnɯndo-t-a}\hspace{5pt}\pcmn{我对你怀恨在心}\end{exemple}
\begin{exemple}\pjya{kɤ-kɯ-nɤsnɯndo-a}\hspace{5pt}\pcmn{你对我怀恨在心}\end{exemple}\relationsémantique{参考}{\lien{ⓔtɯ-sni}{tɯ-sni}}\relationsémantique{参考}{\lien{ⓔndo}{ndo}}\end{entrée}

\begin{entrée}{nɤsɲɯsɲu}{}{ⓔnɤsɲɯsɲu}\relationsémantique{参考}{\lien{ⓔsɲu}{sɲu}}\end{entrée}

\begin{entrée}{nɤsŋɯt}{}{ⓔnɤsŋɯt} 
\classe{vt} \paradigme{dir}{tɤ-}\paradigme{dir}{lɤ-}
\begin{définition}\pfra{ronger, tenir avec ses dent}\end{définition}
\begin{définition}\pcmn{啃;用牙齿咬住}\end{définition}
\begin{exemple}\pjya{@gangbi ma-tɤ-tɯ-nɤsŋɯt}\hspace{5pt}\pcmn{你不要啃你的钢笔}\end{exemple}
\begin{exemple}\pjya{βʑɯ kɯ @dianxian to-nɤsŋɯt}\hspace{5pt}\pcmn{老鼠把电线啃了}\end{exemple}
\begin{exemple}\pjya{tɯ-ŋga to-nɤsŋɯt}\hspace{5pt}\pcmn{啃了衣服}\end{exemple}
\begin{exemple}\pjya{tɤ-ri to-nɤsŋɯt}\hspace{5pt}\pcmn{啃了线}\end{exemple}
\begin{exemple}\pjya{@zhuozi to-nɤsŋɯt}\hspace{5pt}\pcmn{啃了桌子}\end{exemple}
\begin{exemple}\pjya{nɯ-jaʁmu lu-nɤsŋɯt-nɯ ŋgrɤl}\hspace{5pt}\pcmn{(婴儿)爱嘬大拇指}\end{exemple}\relationsémantique{参考}{\lien{ⓔtɤ-sŋɯt}{tɤ-sŋɯt}}\end{entrée}

\begin{entrée}{nɤso}{}{ⓔnɤso} 
\classe{vt} \paradigme{dir}{nɯ-}
\begin{définition}\pfra{manquer}\end{définition}
\begin{définition}\pcmn{想念很久}\end{définition}
\begin{exemple}\pjya{tɤjko nɯ-nɤso-t-a}\hspace{5pt}\pcmn{我想念酸菜很久了}\end{exemple}\relationsémantique{同义词}{\lien{ⓔnɯɣbɯɣ}{nɯɣbɯɣ}}\end{entrée}

\begin{entrée}{nɤsphjarlar}{}{ⓔnɤsphjarlar} 
\classe{vt} \paradigme{dir}{tɤ-}
\begin{définition}\pfra{étendre}\end{définition}
\begin{définition}\pcmn{展开(衣服、布料)}\end{définition}
\begin{exemple}\pjya{tɯ-ŋga tɤ-nɤsphjarlar-a ma mɯ́j-zbaʁ}\hspace{5pt}\pcmn{我把衣服晾开了,因为还没有干}\end{exemple}\relationsémantique{参考}{\lien{ⓔsphjar}{sphjar}}\end{entrée}

\begin{entrée}{nɤstu}{}{ⓔnɤstu} 
\classe{vt} \paradigme{dir}{nɯ-}
\begin{définition}\pfra{croire (quelqu'un)}\end{définition}
\begin{définition}\pcmn{相信(人)}\end{définition}
\begin{exemple}\pjya{aʑo ɲɯ-ta-nɤstu}\hspace{5pt}\pcmn{我相信你}\end{exemple}
\begin{exemple}\pjya{jiɕqha nɯ rɯkhramba tɕe, ma-nɯ-tɯ-nɤste}\hspace{5pt}\pcmn{这个人在说谎,你不要相信他}\end{exemple}
\begin{exemple}\pjya{ɯʑo kɯ ta-tɯt nɯ ma-nɯ-tɯ-nɤste}\hspace{5pt}\pcmn{你不要相信他刚才讲的话}\end{exemple}
\begin{sous-entrée}{ʑɣɤnɤstu}{ⓔnɤstuⓝʑɣɤnɤstu} 
\classe{vi}  
\grammaire{refl} 
\begin{exemple}\pjya{ɯ-zda mɯ́j-nɤste ma ɯʑo ɲɯ-ʑɣɤ-nɤstu}\hspace{5pt}\pcmn{他不相信别人,只相信自己}\end{exemple}
\begin{exemple}\pjya{mɯ́j-ʑɣɤnɤstu}\hspace{5pt}\pcmn{他没有自信}\end{exemple}\end{sous-entrée}

\begin{sous-entrée}{sɤnɤstu}{ⓔnɤstuⓝsɤnɤstu} 
\classe{vi}  
\grammaire{apass} 
\begin{définition}\pfra{avoir confiance en les gens}\end{définition}
\begin{définition}\pcmn{相信别人}\end{définition}\relationsémantique{参考}{\lien{ⓔstuⓗ2}{stu₂}}\end{sous-entrée}

\end{entrée}

\begin{entrée}{nɤsta}{}{ⓔnɤsta} 
\classe{vt} \paradigme{dir}{nɯ-}\paradigme{dir}{kɤ-}
\begin{définition}\pfra{s'habituer}\end{définition}
\begin{définition}\pcmn{习惯(环境)}\end{définition}
\begin{exemple}\pjya{ki sɤtɕha kɤ-nɤsta-t-a}\hspace{5pt}\pcmn{我习惯了这个地方}\end{exemple}
\begin{exemple}\pjya{nɯʑo nɯ-kha aʑo nɯ-nɤsta-t-a}\hspace{5pt}\pcmn{我习惯了你们的家}\end{exemple}\end{entrée}

\begin{entrée}{nɤstɤr}{}{ⓔnɤstɤr} 
\classe{vt} \paradigme{dir}{\_}
\begin{définition}\pfra{tirer d'un seul coup}\end{définition}
\begin{définition}\pcmn{突然一拉}\end{définition}
\begin{exemple}\pjya{nɤ-kɤcu nɯ nɯ-nɤstɤr}\hspace{5pt}\pcmn{你突然拉了你那边的那个东西}\end{exemple}
\begin{exemple}\pjya{akɤcu laχtɕha nɯ-nɤstar-a}\hspace{5pt}\pcmn{我突然拉了我这边的东西}\end{exemple}
\begin{exemple}\pjya{kɤ-nɤstar-a ʑo kɤ-ɣɤrat-a}\hspace{5pt}\pcmn{我拿了然后扔了(手机)}\end{exemple}
\begin{exemple}\pjya{qandʑɣi kɯ a-mthɯm tha-nɤstɤr ʑo tha-nɯtsɯm}\hspace{5pt}\pcmn{鹰把我的肉突然地抢走了}\end{exemple}\end{entrée}

\begin{entrée}{nɤstumbat}{}{ⓔnɤstumbat}\relationsémantique{参考}{\lien{ⓔstuⓗ1}{stu₁}}\end{entrée}

\begin{entrée}{nɤstɯstu}{}{ⓔnɤstɯstu} 
\classe{vt} 
\begin{définition}\pfra{causer des ennuis à}\end{définition}
\begin{définition}\pcmn{找……麻烦}\end{définition}\end{entrée}

\begin{entrée}{nɤtu/\variante{atu}}{}{ⓔnɤtu} 
\classe{adv} 
\begin{définition}\pfra{sur les trois pierres du foyer}\end{définition}
\begin{définition}\pcmn{在三脚架上}\end{définition}
\begin{exemple}\pjya{nɤtu tɯthɯ ɯ-ŋgɯ lɤ-lɤt}\hspace{5pt}\pcmn{倒在锅子上(在三脚架)}\end{exemple}\end{entrée}

\begin{entrée}{nɤtar}{}{ⓔnɤtar} 
\classe{vt} \paradigme{dir}{kɤ-}\paradigme{dir}{nɯ-}
\begin{définition}\pfra{battre avec un bâton}\end{définition}
\begin{définition}\pcmn{用木棍打}\end{définition}
\begin{exemple}\pjya{nɯŋa kɤ-nɤtar}\hspace{5pt}\pcmn{你用木棍打一下牛吧}\end{exemple}
\begin{exemple}\pjya{nɤ-stu tɤ-fse ma tha ci ta-nɤtar}\hspace{5pt}\pcmn{你要守规矩,不然我就会打你一下。}\end{exemple}\relationsémantique{参考}{\lien{ⓔtɤtar}{tɤtar}}\end{entrée}

\begin{entrée}{nɤtɕɯ}{}{ⓔnɤtɕɯ} 
\classe{vt} \paradigme{dir}{tɤ-}
\begin{définition}\pfra{adopter (un garçon)}\end{définition}
\begin{définition}\pcmn{领养(男孩子)}\end{définition}\relationsémantique{同义词}{\lien{ⓔnɤme}{nɤme}}\relationsémantique{参考}{\lien{ⓔtɤ-tɕɯ}{tɤ-tɕɯ}}\end{entrée}

\begin{entrée}{nɤthɤβ}{}{ⓔnɤthɤβ} 
\classe{vi} \paradigme{dir}{kɤ-}
\begin{définition}\pfra{entourer des deux côtés}\end{définition}
\begin{définition}\pcmn{从两面围起来、挡着}\end{définition}
\begin{exemple}\pjya{kɤ-nɤthɤβ-tɕi}\hspace{5pt}\pcmn{我们俩围起来了}\end{exemple}
\begin{exemple}\pjya{nɤthɤβ-tɕi tɕe scɤt-tɕi}\hspace{5pt}\pcmn{我们俩先站在两边,再搬过去(这个东西很重,需要两个人一起抬)}\end{exemple}\end{entrée}

\begin{entrée}{nɤthɯthu}{}{ⓔnɤthɯthu} 
\classe{vt}  
\grammaire{n.orient} \paradigme{dir}{nɯ-}
\begin{définition}\pfra{demander partout}\end{définition}
\begin{définition}\pcmn{到处去问}\end{définition}
\begin{exemple}\pjya{aʑo nɯ-nɤthɯthu-t-a}\hspace{5pt}\pcmn{我到处问了}\end{exemple}
\begin{exemple}\pjya{tʂu ɲɯ-ɤz-nɤthɯthu}\hspace{5pt}\pcmn{他到处问路}\end{exemple}\relationsémantique{参考}{\lien{ⓔthuⓗ1}{thu₁}}\end{entrée}

\begin{entrée}{nɤtsa}{}{ⓔnɤtsa} 
\classe{n} 
\begin{définition}\pfra{maladie}\end{définition}
\begin{définition}\pcmn{病痛}\end{définition}
\begin{exemple}\pjya{nɤ-nɤtsa ɲɯ-ɣɤrŋa}\hspace{5pt}\pcmn{你有病痛的可能}\end{exemple}\end{entrée}

\begin{entrée}{nɤtsa}{}{ⓔnɤtsa} 
\classe{vs} \paradigme{dir}{tɤ-}
\begin{définition}\pfra{adapté, convenable}\end{définition}
\begin{définition}\pcmn{合适}\end{définition}
\begin{exemple}\pjya{``a-pi" tu-kɯ-ti kɯ ɲɯ-nɤtsa lo}\hspace{5pt}\pcmn{说“哥哥”比较合适(编故事的时候,豹子对马说话时用“哥哥”比较合适,因为马比较高大)}\end{exemple}
\begin{exemple}\pjya{ki kowa ki mɯ́j-nɤtsa}\hspace{5pt}\pcmn{这个办法不合适}\end{exemple}
\begin{exemple}\pjya{nɤki tɤ-rte nɯ tu-tɯ-nɤrte ɲɯ-tɯ-nɤtsa}\hspace{5pt}\pcmn{你适合戴这顶帽子}\end{exemple}\relationsémantique{参考}{\lien{ⓔnɯtsa}{nɯtsa}}\end{entrée}

\begin{entrée}{nɤtsɤtkhɯ}{}{ⓔnɤtsɤtkhɯ} 
\classe{vs} 
\begin{définition}\pfra{être obéissant}\end{définition}
\begin{définition}\pcmn{听指挥}\end{définition}
\begin{exemple}\pjya{ki tɯrme ki mɤ-kɯ-nɤtsɤtkhɯ ci ɲɯ-ŋu}\hspace{5pt}\pcmn{这个人不听话}\end{exemple}
\begin{exemple}\pjya{a-mdʑu ki mɯ́j-nɤtsɤtkhɯ}\hspace{5pt}\pcmn{我这个音发不出来}\end{exemple}\end{entrée}

\begin{entrée}{nɤtshɤxtshɯ}{}{ⓔnɤtshɤxtshɯ} 
\classe{vt} \paradigme{dir}{tɤ-}
\begin{définition}\pfra{inciter, pousser}\end{définition}
\begin{définition}\pcmn{催}\end{définition}
\begin{exemple}\pjya{mɯ́j-mbɣom tɕe tɤ-nɤtshɤxtshɯ-t-a}\hspace{5pt}\pcmn{他很不积极,所以我就催了他一下}\end{exemple}
\begin{exemple}\pjya{ma-tɤ-kɯ-nɤtshɤxtshɯ-a ma nɯ ma mɯ́j-cha-a}\hspace{5pt}\pcmn{你不要催我}\end{exemple}\end{entrée}

\begin{entrée}{nɤtshɯtshɤt}{}{ⓔnɤtshɯtshɤt} 
\classe{vt}  
\grammaire{n.orient} \paradigme{dir}{tɤ-}\paradigme{dir}{kɤ-}
\begin{définition}\pfra{tenter de déterminer}\end{définition}
\begin{définition}\pcmn{试探}\end{définition}
\begin{exemple}\pjya{ŋu ɕi maʁ, khramba tu-βze ŋu maʁ nɯ tɤ-nɤtshɯtshat-a}\hspace{5pt}\pcmn{我试探了一下他是不是在说谎}\end{exemple}\relationsémantique{参考}{\lien{ⓔtshɤtⓗ1}{tshɤt₁}}\end{entrée}

\begin{entrée}{nɤtsoʁ}{}{ⓔnɤtsoʁ} 
\classe{vi}  
\grammaire{denom} \paradigme{dir}{lɤ-}
\begin{définition}\pfra{ramasser des gromas}\end{définition}
\begin{définition}\pcmn{挖人参果}\end{définition}
\begin{exemple}\pjya{lɤ-nɤtsoʁ}\hspace{5pt}\pcmn{他挖了人参果}\end{exemple}
\begin{exemple}\pjya{pɯ-nɤtsoʁ-tɕi}\hspace{5pt}\pcmn{我们俩挖了人参果}\end{exemple}\relationsémantique{参考}{\lien{ⓔtɤtsoʁ}{tɤtsoʁ}}\end{entrée}

\begin{entrée}{nɤtsɯ}{}{ⓔnɤtsɯ} 
\classe{vt} \paradigme{dir}{kɤ-}\paradigme{dir}{kɤ-}
\begin{définition}\pfra{cacher, garder le secret}\end{définition}
\begin{définition}\pcmn{保密;藏起来}\end{définition}
\begin{définition}\pfra{dissimuler (ses actions)}\end{définition}
\begin{définition}\pcmn{隐瞒(自己的行为)}\end{définition}
\begin{exemple}\pjya{aʑo kɯnɤ kɤ-nɤtsɯ-t-a}\hspace{5pt}\pcmn{我也保密了}\end{exemple}
\begin{exemple}\pjya{kɤ-tɯ-nɤtsɯ-t}\hspace{5pt}\pcmn{你保密了}\end{exemple}
\begin{exemple}\pjya{jiɕqha khɯtsa pɯ-tɯ-qrɯ-t nɯ kɤ-nɤtsi}\hspace{5pt}\pcmn{你把碗打破了,你不要跟人家说}\end{exemple}
\begin{exemple}\pjya{kɯmaʁ tɤ́-wɣ-nɤma qhe kɤ-ʑɣɤnɤtsɯ mɤ-phɤn, tɯrme ra kɯ ciz qhe sɯz-nɯ ɕti.}\hspace{5pt}\pcmn{自己做了坏事的时候隐瞒是没有用的,因为人们始终是会知道的}\end{exemple}\relationsémantique{参考}{\lien{ⓔanɤtsɯtsɯ}{anɤtsɯtsɯ}}\relationsémantique{参考}{\lien{}{ɯ-tsɯ}}
\begin{sous-entrée}{ʑɣɤnɤtsɯ}{ⓔnɤtsɯⓝʑɣɤnɤtsɯ} 
\classe{vi} \end{sous-entrée}

\end{entrée}

\begin{entrée}{nɤtsɯmɣɯt}{}{ⓔnɤtsɯmɣɯt} 
\classe{vt}  
\grammaire{comp} \paradigme{dir}{tɤ-}
\begin{définition}\pfra{déplacer les objets dans tous les sens}\end{définition}
\begin{définition}\pcmn{把东西拿过来拿过去}\end{définition}
\begin{exemple}\pjya{aʑo tɤ-nɤtsɯmɣɯt-a}\hspace{5pt}\pcmn{我拿过去拿过来了}\end{exemple}
\begin{exemple}\pjya{tɤ-scoz kɯ-nɤtsɯmɣɯt}\hspace{5pt}\pcmn{邮递员}\end{exemple}
\begin{exemple}\pjya{kɤ-nɤtsɯmɣɯt mɤ-cha-a}\hspace{5pt}\pcmn{(我又不是邮递员),我不能把东西拿过去拿过来}\end{exemple}
\begin{exemple}\pjya{tɯ-rju ɲɯ-ɤz-nɤtsɯmɣɯt}\hspace{5pt}\pcmn{他在传播谣言}\end{exemple}\relationsémantique{参考}{\lien{ⓔtsɯm}{tsɯm}}\relationsémantique{参考}{\lien{ⓔɣɯt}{ɣɯt}}\end{entrée}

\begin{entrée}{nɤtsɯtsɯm}{}{ⓔnɤtsɯtsɯm}\relationsémantique{参考}{\lien{ⓔtsɯm}{tsɯm}}\end{entrée}

\begin{entrée}{nɤtʂa}{}{ⓔnɤtʂa} 
\classe{vs} \paradigme{dir}{tɤ-}\paradigme{dir}{\_}\paradigme{dir}{\_}
\begin{définition}\pfra{serré à fond (le couvercle d'une boîte)}\end{définition}
\begin{définition}\pcmn{盖子盖得很紧}\end{définition}
\begin{exemple}\pjya{ɯ-sti ɲɯ-nɤtʂa}\hspace{5pt}\pcmn{塞得很紧}\end{exemple}
\begin{exemple}\pjya{ɯ-ŋgɯ nɯ-znɤtʂa-t-a}\hspace{5pt}\pcmn{我在里面装得很紧}\end{exemple}
\begin{sous-entrée}{znɤtʂa}{ⓔnɤtʂaⓝznɤtʂa} 
\classe{vt}  
\grammaire{caus} \end{sous-entrée}

\end{entrée}

\begin{entrée}{nɤtʂaβlaβ}{}{ⓔnɤtʂaβlaβ} 
\classe{vt}  
\grammaire{n.orient} \paradigme{dir}{\_}
\begin{définition}\pfra{faire rouler dans tous les sens}\end{définition}
\begin{définition}\pcmn{使滚来滚去}\end{définition}
\begin{exemple}\pjya{@lanqiu nɯ-nɤtʂaβlaβ-a}\hspace{5pt}\pcmn{我让篮球滚来滚去了}\end{exemple}\relationsémantique{参考}{\lien{ⓔtʂaβ}{tʂaβ}}\end{entrée}

\begin{entrée}{nɤtʂaŋ}{}{ⓔnɤtʂaŋ}\relationsémantique{参考}{\lien{ⓔtʂaŋ}{tʂaŋ}}\end{entrée}

\begin{entrée}{nɤtʂɤtshi}{}{ⓔnɤtʂɤtshi} 
\classe{vt}  
\grammaire{incorp} \paradigme{dir}{\_}
\begin{définition}\pfra{bloquer le chemin}\end{définition}
\begin{définition}\pcmn{挡路}\end{définition}
\begin{exemple}\pjya{ɯʑo kɯ tɤ́-wɣ-nɤtʂɤtshi-a}\hspace{5pt}\pcmn{他挡了我的路}\end{exemple}\relationsémantique{参考}{\lien{ⓔtshiⓗ3}{tshi₃}}\relationsémantique{参考}{\lien{ⓔtʂu}{tʂu}}\end{entrée}

\begin{entrée}{nɤtɯɣ}{}{ⓔnɤtɯɣ}\relationsémantique{参考}{\lien{ⓔatɯɣ}{atɯɣ}}\end{entrée}

\begin{entrée}{nɤtɯta}{}{ⓔnɤtɯta}\relationsémantique{参考}{\lien{ⓔta}{ta}}\end{entrée}

\begin{entrée}{nɤtɯti}{}{ⓔnɤtɯti} 
\classe{vt}  
\grammaire{n.orient} \paradigme{dir}{tɤ-}\paradigme{dir}{thɯ-}\paradigme{past stem}{nɤtɯtɯt}
\begin{définition}\pfra{dire à tout le monde}\end{définition}
\begin{définition}\pcmn{到处说}\end{définition}
\begin{exemple}\pjya{tɤ-nɤtɯtɯt-a}\hspace{5pt}\pcmn{我到处说了}\end{exemple}
\begin{exemple}\pjya{@kaihui ɲɯ-ra tɤ-nɤtɯti}\hspace{5pt}\pcmn{你告诉大家需要开会}\end{exemple}
\begin{exemple}\pjya{nɤʑo tɤ-tɯ-nɤtɯtɯt ɯ́-tu?}\hspace{5pt}\pcmn{你到处说了没有}\end{exemple}\relationsémantique{参考}{\lien{ⓔti}{ti}}\end{entrée}

\begin{entrée}{nɤwu}{}{ⓔnɤwu} 
\classe{vt} \paradigme{dir}{nɯ-}
\begin{définition}\pfra{pleurer pour}\end{définition}
\begin{définition}\pcmn{为……而哭}\end{définition}
\begin{exemple}\pjya{ɲɯ-kɯ-nɤwu-a mɤ-ra}\hspace{5pt}\pcmn{你不用为我哭}\end{exemple}
\begin{exemple}\pjya{nɯ kɤ-nɤwu ci me nɤ}\hspace{5pt}\pcmn{那没有什么好哭的}\end{exemple}\relationsémantique{参考}{\lien{}{tɤ-wu}}\relationsémantique{参考}{\lien{ⓔɣɤwu}{ɣɤwu}}\end{entrée}

\begin{entrée}{nɤwɤt}{}{ⓔnɤwɤt} 
\classe{vs} \paradigme{dir}{nɯ-}
\begin{définition}\pfra{qui s'ouvre vers l'extérieur en forme de cloche inversée}\end{définition}
\begin{définition}\pcmn{向外开着的形状(形成圆锥形)}\end{définition}
\begin{exemple}\pjya{@beibei ɯ-mŋu ɲɯ-nɤwɤt ɲɯ-ŋu}\hspace{5pt}\pcmn{杯子的口子向外开着}\end{exemple}\relationsémantique{同义词}{\lien{ⓔɣɤrɣɤr}{ɣɤrɣɤr}}\end{entrée}

\begin{entrée}{nɤwxti}{}{ⓔnɤwxti} 
\classe{vt}  
\grammaire{trop} \paradigme{dir}{tɤ-}
\begin{définition}\pfra{trouver trop grand}\end{définition}
\begin{définition}\pcmn{觉得太大}\end{définition}
\begin{exemple}\pjya{ɯ-xtsa ɲɯ-nɤwxti}\hspace{5pt}\pcmn{他觉得鞋子太大了}\end{exemple}
\begin{exemple}\pjya{tɯ-xtsa tɤ-χtɯ-t-a ɲɯ-nɤwxti-a}\hspace{5pt}\pcmn{我买了鞋子,但是觉得太大了}\end{exemple}
\begin{exemple}\pjya{tɯ-ŋga tɤ-χtɯ-t-a nɯ ɲɯ-nɤwxti-a}\hspace{5pt}\pcmn{我觉得衣服买太大了}\end{exemple}\relationsémantique{参考}{\lien{ⓔwxti}{wxti}}\end{entrée}

\begin{entrée}{nɤxchi}{}{ⓔnɤxchi}\relationsémantique{参考}{\lien{ⓔchi}{chi}}\end{entrée}

\begin{entrée}{nɤxɕɤt}{}{ⓔnɤxɕɤt} 
\classe{vt}  
\grammaire{denom} \paradigme{dir}{tɤ-}\paradigme{dir}{thɯ-}\paradigme{dir}{pɯ-}\sens{1}
\begin{définition}\pfra{faire un gros effort}\end{définition}
\begin{définition}\pcmn{用力}\end{définition}\sens{2}
\begin{définition}\pfra{faire qqch avec force}\end{définition}
\begin{définition}\pcmn{使劲}\end{définition}
\begin{exemple}\pjya{ki rdɤstaʁ ɲɯ-rʑi ri, tɤ-nɤxɕat-a tɕe tɤ-mɟa-t-a}\hspace{5pt}\pcmn{这块石头很重,我很用力捡起来了}\end{exemple}
\begin{exemple}\pjya{pɯ-nɤxɕɤt tɕe pɯ-ɣɤrɤt}\hspace{5pt}\pcmn{你用力扔吧}\end{exemple}
\begin{exemple}\pjya{rɤɣo tha-nɤxɕɤt}\hspace{5pt}\pcmn{他唱歌唱得很大声}\end{exemple}\relationsémantique{参考}{\lien{ⓔtɯ-xɕɤt}{tɯ-xɕɤt}}\end{entrée}

\begin{entrée}{nɤxpe}{}{ⓔnɤxpe} 
\classe{vt}  
\grammaire{trop} \paradigme{dir}{pɯ-}
\begin{définition}\pfra{trouver bien}\end{définition}
\begin{définition}\pcmn{觉得很好}\end{définition}
\begin{exemple}\pjya{pɯ-nɤxpe-t-a}\hspace{5pt}\pcmn{我觉得很好了}\end{exemple}\relationsémantique{参考}{\lien{ⓔpe}{pe}}\end{entrée}

\begin{entrée}{nɤxtɕɤβ}{}{ⓔnɤxtɕɤβ}\relationsémantique{参考}{\lien{ⓔɣɤxtɕɤβ}{ɣɤxtɕɤβ}}\end{entrée}

\begin{entrée}{nɤxtɕhɯβ}{}{ⓔnɤxtɕhɯβ} 
\classe{vt} \paradigme{dir}{nɯ-}\sens{1}
\begin{définition}\pfra{s'appuyer sur, dépendre de}\end{définition}
\begin{définition}\pcmn{依靠}\end{définition}
\begin{exemple}\pjya{kɤ-nɤma nɯ, tɯʑo tu-kɯ-nɯ-stu ɲɯ́-wɣ-nɤma ra ma tɯ-zda kɤ-nɤxtɕhɯβ mɤ-pe}\hspace{5pt}\pcmn{在工作方面要自己努力,不要依靠别人}\end{exemple}\sens{2}
\begin{définition}\pfra{profiter de}\end{définition}
\begin{définition}\pcmn{趁着}\end{définition}
\begin{exemple}\pjya{ɯʑo ju-kɯ-ɕe nɯ nɯ-nɤxtɕhɯβ-a tɕe ɯ-rca jɤ-ari-a pɯ-ŋu}\hspace{5pt}\pcmn{我趁着他去那边跟他一起去}\end{exemple}\relationsémantique{参考}{\lien{ⓔɯ-tɕhɯβ}{ɯ-tɕhɯβ}}\end{entrée}

\begin{entrée}{nɤxtɕi}{}{ⓔnɤxtɕi} 
\classe{vt}  
\grammaire{trop} \paradigme{dir}{tɤ-}
\begin{définition}\pfra{trouver trop petit}\end{définition}
\begin{définition}\pcmn{觉得太小}\end{définition}
\begin{exemple}\pjya{aʑo ɲɯ-nɤxtɕi-a}\hspace{5pt}\pcmn{我觉得太小}\end{exemple}
\begin{exemple}\pjya{nɤ-mi ɲɯ-wxti, nɤ-xtsa ɲɯ-tɯ-nɤxtɕi}\hspace{5pt}\pcmn{你脚很大,鞋子太小了}\end{exemple}
\begin{exemple}\pjya{kɯki ɯ-spa mɯ́j-rtaʁ, ɲɯ-nɤxtɕi-a}\hspace{5pt}\pcmn{材料不够,我觉得太小}\end{exemple}\relationsémantique{参考}{\lien{ⓔxtɕi}{xtɕi}}\end{entrée}

\begin{entrée}{nɤxtɕur}{}{ⓔnɤxtɕur}\relationsémantique{参考}{\lien{ⓔtɕurⓗ1}{tɕur₁}}\end{entrée}

\begin{entrée}{nɤxthɯ}{}{ⓔnɤxthɯ} 
\classe{vt}  
\grammaire{trop} \paradigme{dir}{nɯ-}
\begin{définition}\pfra{trouver grave}\end{définition}
\begin{définition}\pcmn{觉得严重}\end{définition}
\begin{exemple}\pjya{jɤxtshi tɕhomba aʑo pɯ-nɤxthɯ-t-a}\hspace{5pt}\pcmn{这一次感冒,我觉得很严重了}\end{exemple}
\begin{exemple}\pjya{jɤxtshi tɕhomba, ɲɯ-tɯ-nɤxthi}\hspace{5pt}\pcmn{这一次感冒,你觉得很严重}\end{exemple}
\begin{exemple}\pjya{ɯ-kɯ-mŋɤm ɲɯ-nɤxthi}\hspace{5pt}\pcmn{他觉得很严重}\end{exemple}\relationsémantique{参考}{\lien{ⓔthɯⓗ2}{thɯ₂}}\end{entrée}

\begin{entrée}{nɤxtsa}{}{ⓔnɤxtsa} 
\classe{vt} \paradigme{dir}{tɤ-}
\begin{définition}\pfra{forcer}\end{définition}
\begin{définition}\pcmn{撬}\end{définition}
\begin{exemple}\pjya{sɤcɯ tɤ-nɤxtsa-t-a}\hspace{5pt}\pcmn{我把锁撬开了}\end{exemple}
\begin{exemple}\pjya{kɯm tɤ-nɤxtsa-t-a}\hspace{5pt}\pcmn{我把门撬开了}\end{exemple}\end{entrée}

\begin{entrée}{nɤxtʂɯ}{}{ⓔnɤxtʂɯ} 
\classe{vt} \paradigme{dir}{kɤ-}\paradigme{dir}{nɯ-}\paradigme{dir}{lɤ-}\paradigme{dir}{thɯ-}
\begin{définition}\pfra{apporter en passant}\end{définition}
\begin{définition}\pcmn{顺便带}\end{définition}
\begin{exemple}\pjya{aʑo ɲɯ-nɤxtʂi-a}\hspace{5pt}\pcmn{我顺便带过去}\end{exemple}
\begin{exemple}\pjya{nɤʑo nɯ-nɤxtʂi}\hspace{5pt}\pcmn{你顺便带过来吧}\end{exemple}
\begin{exemple}\pjya{tɕɤndi jiʑo ji-laχtɕha ata tɕe kɤ-nɤxtʂi}\hspace{5pt}\pcmn{我们的东西在那边,你顺便带过来吧}\end{exemple}
\begin{exemple}\pjya{nɤʑɯɣ kɤ-nɤxtʂɯ-t-a}\hspace{5pt}\pcmn{我顺便带给你了}\end{exemple}\relationsémantique{同义词}{\lien{ⓔnɯpjaχpa}{nɯpjaχpa}}\end{entrée}

\begin{entrée}{nɤxtʂɯn}{}{ⓔnɤxtʂɯn} 
\classe{vt}  
\grammaire{trop} \paradigme{dir}{nɯ-}
\begin{définition}\pfra{remercier, avoir de la reconnaissance pour}\end{définition}
\begin{définition}\pcmn{感激}\end{définition}
\begin{exemple}\pjya{nɯ-nɤxtʂɯn-a}\hspace{5pt}\pcmn{我很感激了}\end{exemple}
\begin{exemple}\pjya{ɲɯ-nɤxtʂɯn}\hspace{5pt}\pcmn{他很感激}\end{exemple}
\begin{exemple}\pjya{laχtɕha nɯ-kɯ-mbi-a pɯ-nɤxtʂɯn-a}\hspace{5pt}\pcmn{我很感激你把东西给我了}\end{exemple}
\begin{exemple}\pjya{aʑo nɯ-mbi-t-a tɕe ɲɯ-nɤxtʂɯn}\hspace{5pt}\pcmn{我给了他,他很感激}\end{exemple}\relationsémantique{参考}{\lien{ⓔtɯ-tʂɯn}{tɯ-tʂɯn}}\étymologie{drin}\end{entrée}

\begin{entrée}{nɤxtɯt}{}{ⓔnɤxtɯt} 
\classe{vt}  
\grammaire{trop} \paradigme{dir}{tɤ-}
\begin{définition}\pfra{trouver trop court}\end{définition}
\begin{définition}\pcmn{觉得太短}\end{définition}
\begin{exemple}\pjya{a-ŋga ɲɯ-nɤxtɯt-a}\hspace{5pt}\pcmn{我觉得衣服太小}\end{exemple}\relationsémantique{参考}{\lien{ⓔxtɯtⓗ2}{xtɯt}}\end{entrée}

\begin{entrée}{nɤχɤmthi}{}{ⓔnɤχɤmthi} 
\classe{vi} \paradigme{dir}{thɯ-}\paradigme{dir}{thɯ-}
\begin{définition}\pfra{être bouche bée}\end{définition}
\begin{définition}\pcmn{目瞪口呆}\end{définition}
\begin{définition}\pfra{rendre bouche bée}\end{définition}
\begin{définition}\pcmn{令人目瞪口呆}\end{définition}
\begin{exemple}\pjya{thɯ-nɤχɤmthi-a}\hspace{5pt}\pcmn{我目瞪口呆了}\end{exemple}
\begin{exemple}\pjya{chɤ́-wɣ-znɤχɤmthi ʑo}\hspace{5pt}\pcmn{令他目瞪口呆}\end{exemple}\relationsémantique{同义词}{\lien{ⓔnɯcaχto}{nɯcaχto}}\relationsémantique{同义词}{\lien{ⓔnɯcaχtoⓝznɯcaχto}{znɯcaχto}}
\begin{sous-entrée}{znɤχɤmthi}{ⓔnɤχɤmthiⓝznɤχɤmthi} 
\classe{vt} \end{sous-entrée}

\end{entrée}

\begin{entrée}{nɤχcɤl}{}{ⓔnɤχcɤl} 
\classe{vi}  
\grammaire{denom} \paradigme{dir}{\_}
\begin{définition}\pfra{être au milieu}\end{définition}
\begin{définition}\pcmn{到中间}\end{définition}
\begin{exemple}\pjya{tɯ-ci kɤ-nɤχcɤl (tɯ-ci ɯ-ŋgɯ kɤ-ndʑaʁ tɕe kɤ-nɤχcɤl)}\hspace{5pt}\pcmn{他到水的中间了}\end{exemple}
\begin{exemple}\pjya{ndzom ɯ-taʁ kɤ-nɤχcɤl}\hspace{5pt}\pcmn{他到桥的中间}\end{exemple}
\begin{exemple}\pjya{ki kɤ-nɤχcal-a}\hspace{5pt}\pcmn{我到中间了(例如,我看书看到一半)}\end{exemple}
\begin{exemple}\pjya{tɯ-ci kɤ-nɤχcal-a}\hspace{5pt}\pcmn{我到了河的中间(游过去)}\end{exemple}
\begin{exemple}\pjya{tʂu jɤ-nɤχcal-a ʑo}\hspace{5pt}\pcmn{我走了一半的路}\end{exemple}\relationsémantique{参考}{\lien{ⓔɯ-χcɤl}{ɯ-χcɤl}}\étymologie{dkʲil}\end{entrée}

\begin{entrée}{nɤχphe}{}{ⓔnɤχphe} 
\classe{vt}  
\grammaire{denom} \paradigme{dir}{tɤ-}\paradigme{dir}{pɯ-}
\begin{définition}\pfra{frapper}\end{définition}
\begin{définition}\pcmn{拍(用手掌)}\end{définition}
\begin{exemple}\pjya{aʑo a-xtu tɤ-nɤχphe-t-a}\hspace{5pt}\pcmn{我拍了我的肚子}\end{exemple}
\begin{exemple}\pjya{ɯʑo kɯ ɯ-xtu ta-nɤχphe}\hspace{5pt}\pcmn{他拍了自己的肚子}\end{exemple}
\begin{exemple}\pjya{tɕoχtsi pɯ-nɤχphe-t-a (=tɕoχtsi ɯ-taʁ taχphe pɯ-lat-a)}\hspace{5pt}\pcmn{我拍了桌子}\end{exemple}\relationsémantique{参考}{\lien{ⓔtaχphe}{taχphe}}\end{entrée}

\begin{entrée}{nɤz}{}{ⓔnɤz} 
\classe{vi} \paradigme{dir}{tɤ-}\paradigme{dir}{nɯ-}
\begin{définition}\pfra{oser}\end{définition}
\begin{définition}\pcmn{敢}\end{définition}
\begin{définition}\pfra{faire oser}\end{définition}
\begin{définition}\pcmn{令……敢}\end{définition}
\begin{exemple}\pjya{ɲɯ-nɤz}\hspace{5pt}\pcmn{他敢}\end{exemple}
\begin{exemple}\pjya{aʑo nɯ tu-ti-a naz-a}\hspace{5pt}\pcmn{我敢说这个}\end{exemple}
\begin{exemple}\pjya{aj ku-ɕe-a naz-a}\hspace{5pt}\pcmn{我敢去}\end{exemple}
\begin{exemple}\pjya{ki ɯ-ʁɤri mɯ-pɯ-naz-a, tham to-naz-a}\hspace{5pt}\pcmn{我以前不敢,现在敢了}\end{exemple}
\begin{exemple}\pjya{a-sɯm kɯ-kɯ-fse nɯ kɤ-ti mɤ-naz-a}\hspace{5pt}\pcmn{我不敢说我的心里话}\end{exemple}
\begin{exemple}\pjya{ɯʑo kɯ nɯ ɲɯ-ti tɕe kɤ-ɕe mɯ-nɯ́-wɣ-sɯɣnaz-a}\hspace{5pt}\pcmn{他讲的话令我不敢去了}\end{exemple}
\begin{sous-entrée}{sɤnɤz}{ⓔnɤzⓝsɤnɤz} 
\classe{vs}  
\grammaire{deexp} 
\begin{exemple}\pjya{ki khɯna kɯ-ɤtɯɣ mɯ́j-sɤnɤz ma ku-kɯ-mtsɯɣ ɲɯ-ɕti}\hspace{5pt}\pcmn{不敢碰到这条狗,因为会咬人}\end{exemple}
\begin{exemple}\pjya{tɤ-pɤtso kɤ-sɯsɤβlo mɯ́j-sɤnɤz}\hspace{5pt}\pcmn{不敢让他看小孩子}\end{exemple}\end{sous-entrée}

\begin{sous-entrée}{sɯɣnɤz}{ⓔnɤzⓝsɯɣnɤz} 
\classe{vt} \end{sous-entrée}

\end{entrée}

\begin{entrée}{nɤzda}{}{ⓔnɤzda} 
\classe{vt}  
\grammaire{denom} \paradigme{dir}{tɤ-}
\begin{définition}\pfra{inviter quelqu'un à se joindre à son groupe}\end{définition}
\begin{définition}\pcmn{请别人加入自己的队伍}\end{définition}
\begin{exemple}\pjya{aʑo tɤ-nɤzda-t-a}\hspace{5pt}\pcmn{我请他加入了}\end{exemple}
\begin{exemple}\pjya{ɯʑo kɯ tɤ́-wɣ-nɤzda-a}\hspace{5pt}\pcmn{他请我加入了}\end{exemple}
\begin{exemple}\pjya{jisŋi tɤjmɤɣ ɯ-kɯ-ɕar aj ɕe-a ŋu, tu-ta-nɤzda ɯ-tɯ-ɣi?}\hspace{5pt}\pcmn{今天我去找蘑菇,请你一起去好吗?}\end{exemple}
\begin{exemple}\pjya{kɯ-ɣɯɕoŋtɕa aj ɕe-a ŋu, tu-ta-nɤzda ɯ-tɯ-ɣi?}\hspace{5pt}\pcmn{我去砍木头,请你一起去好吗?}\end{exemple}
\begin{exemple}\pjya{nɤʑo si ɕɯ-tɯ-phɯt ɲɯ-ŋu, tu-kɯ-nɤzda-a tɕe aj kɯnɤ ɣi-a}\hspace{5pt}\pcmn{你要去砍树,可以结伴一起去}\end{exemple}
\begin{exemple}\pjya{tu-kɯ-nɤzda-a ɯ́-jɤɣ}\hspace{5pt}\pcmn{我能不能跟你一起去}\end{exemple}
\begin{exemple}\pjya{tu-ta-nɤzda ɕe-tɕi}\hspace{5pt}\pcmn{我们俩一起去}\end{exemple}\relationsémantique{参考}{\lien{ⓔtɯ-zda}{tɯ-zda}}\relationsémantique{参考}{\lien{ⓔɣɤzda}{ɣɤzda}}\relationsémantique{参考}{\lien{ⓔrɤzda}{rɤzda}}\relationsémantique{参考}{\lien{ⓔsɤzda}{sɤzda}}\end{entrée}

\begin{entrée}{nɤzɲɟoʁ}{}{ⓔnɤzɲɟoʁ} 
\classe{vt} \paradigme{dir}{tɤ-}
\begin{définition}\pfra{frapper, fouetter}\end{définition}
\begin{définition}\pcmn{抽打(用鞭子或者木棒)}\end{définition}
\begin{exemple}\pjya{nɤ-stu tɤ-fse ma ta-nɤzɲɟoʁ}\hspace{5pt}\pcmn{你小心一点,不然我会抽打你的}\end{exemple}\relationsémantique{参考}{\lien{ⓔtɤzɲɟoʁ}{tɤzɲɟoʁ}}\end{entrée}

\begin{entrée}{nɤzraʁ}{}{ⓔnɤzraʁ} 
\classe{vi}  
\grammaire{denom} \paradigme{dir}{nɯ-}\paradigme{dir}{nɯ-}
\begin{définition}\pfra{avoir honte, être gêné, se sentir embarrassé}\end{définition}
\begin{définition}\pcmn{害羞}\end{définition}
\begin{définition}\pfra{embarrasser}\end{définition}
\begin{définition}\pcmn{令……不好意思}\end{définition}
\begin{exemple}\pjya{tɕheme kɤ-rqoʁ mɯ́j-nɤzraʁ}\hspace{5pt}\pcmn{他抱女孩子不感到害羞}\end{exemple}
\begin{exemple}\pjya{khramba tu-βze ɲɯ-ŋu, mɯ́j-nɤzraʁ}\hspace{5pt}\pcmn{他说谎不感到害羞}\end{exemple}
\begin{exemple}\pjya{jiɕqha nɯ kɯ thɯci ɲɯ-ti, aʑo nɯ-nɤzraʁ-a}\hspace{5pt}\pcmn{这个人说了一些什么话,我感到很害羞}\end{exemple}
\begin{exemple}\pjya{nɯ ma-tɤ-tɯ-ti ma tɯ-znɤzraʁ}\hspace{5pt}\pcmn{你不要说这些话,让他不好意思(吃)}\end{exemple}\relationsémantique{参考}{\lien{ⓔtɤzraʁ}{tɤzraʁ}}
\begin{sous-entrée}{sɤzraʁ}{ⓔnɤzraʁⓝsɤzraʁ} 
\classe{vs} 
\begin{définition}\pfra{honteux}\end{définition}
\begin{définition}\pcmn{可耻}\end{définition}
\begin{exemple}\pjya{kɤ-mɯrkɯ ndɤre ɲɯ-sɤzraʁ}\hspace{5pt}\pcmn{偷东西是可耻的行为}\end{exemple}\end{sous-entrée}

\begin{sous-entrée}{znɤzraʁ}{ⓔnɤzraʁⓝznɤzraʁ} 
\classe{vt} \end{sous-entrée}

\end{entrée}

\begin{entrée}{nɤzri}{}{ⓔnɤzri} 
\classe{vt}  
\grammaire{trop} \paradigme{dir}{tɤ-}
\begin{définition}\pfra{trouver trop long}\end{définition}
\begin{définition}\pcmn{觉得太长}\end{définition}
\begin{exemple}\pjya{a-ŋga tɤ-χtɯ-t-a nɯ ɲɯ-nɤzri-a}\hspace{5pt}\pcmn{我觉得衣服买太长了}\end{exemple}\relationsémantique{参考}{\lien{ⓔzri}{zri}}\end{entrée}

\begin{entrée}{nɤʑɤmŋɤn/\variante{nɯʑɤmŋɤn}}{}{ⓔnɤʑɤmŋɤn} 
\classe{vt} \paradigme{dir}{tɤ-}
\begin{définition}\pfra{envier}\end{définition}
\begin{définition}\pcmn{妒忌}\end{définition}\paradigme{dir}{tɤ-}
\begin{définition}\pfra{envier les gens}\end{définition}
\begin{définition}\pcmn{妒忌别人}\end{définition}
\begin{exemple}\pjya{aʑo laχtɕha tɤ-nɯ-χtɯ-t-a tɕe, ɲɯ-kɯ-nɯʑɤmŋan-a}\hspace{5pt}\pcmn{我给自己买了东西,你在妒忌我}\end{exemple}
\begin{exemple}\pjya{ma-tɤ-tɯ-sɤnɤʑɤmŋɤn}\hspace{5pt}\pcmn{你不要妒忌别人}\end{exemple}
\begin{sous-entrée}{sɤnɯʑɤmŋɤn}{ⓔnɤʑɤmŋɤnⓝsɤnɯʑɤmŋɤn} 
\classe{vi} \end{sous-entrée}

\begin{sous-entrée}{anɤʑɤmŋɯmŋɤn}{ⓔnɤʑɤmŋɤnⓝanɤʑɤmŋɯmŋɤn} 
\classe{vi} 
\begin{définition}\pfra{s'envier les uns les autres}\end{définition}
\begin{définition}\pcmn{互相妒忌}\end{définition}\end{sous-entrée}

\étymologie{ʑe.ŋan}\end{entrée}

\begin{entrée}{nɤʑo}{}{ⓔnɤʑo} 
\classe{pro} 
\begin{définition}\pfra{toi}\end{définition}
\begin{définition}\pcmn{你}\end{définition}\relationsémantique{参考}{\lien{ⓔnɤj}{nɤj}}\end{entrée}

\begin{entrée}{nɤʑru}{}{ⓔnɤʑru}\relationsémantique{参考}{\lien{ⓔʑru}{ʑru}}\end{entrée}

\begin{entrée}{nɤʑri}{}{ⓔnɤʑri} 
\classe{vi}  
\grammaire{denom} \paradigme{dir}{nɯ-}
\begin{définition}\pfra{être mouillé par la rosée}\end{définition}
\begin{définition}\pcmn{沾上露水}\end{définition}
\begin{exemple}\pjya{tɯ-mɯ pjɤ-lɤt tɕe aj nɯ-nɤʑri-a}\hspace{5pt}\pcmn{下雨了,我沾上了露水}\end{exemple}\relationsémantique{参考}{\lien{ⓔtɤʑri}{tɤʑri}}\end{entrée}

\begin{entrée}{nɤʑɯloʁ}{}{ⓔnɤʑɯloʁ} 
\classe{vi}  
\grammaire{incorp} \paradigme{dir}{nɯ-}\paradigme{dir}{pɯ-}
\begin{définition}\pfra{avoir la nausée}\end{définition}
\begin{définition}\pcmn{感到恶心;觉得恶心}\end{définition}
\begin{exemple}\pjya{pɯ-nɤʑɯloʁ-a}\hspace{5pt}\pcmn{我感到恶心}\end{exemple}
\begin{exemple}\pjya{tɤ-pɤtso kɯ ɯ-qe na-lɤt tɕe pɯ-nɤʑɯloʁ-a}\hspace{5pt}\pcmn{小孩子屙了屎,我觉得很恶心}\end{exemple}
\begin{exemple}\pjya{khɯna lo-qioʁ tɕe pɯ-nɤʑɯloʁ-a}\hspace{5pt}\pcmn{狗在那里吐了,我觉得很恶心}\end{exemple}\relationsémantique{参考}{\lien{ⓔsɤʑɯloʁ}{sɤʑɯloʁ}}\relationsémantique{参考}{\lien{ⓔtɯ-ʑi,loʁ}{tɯ-ʑi,loʁ}}\end{entrée}

\begin{entrée}{nɤʑɯn}{}{ⓔnɤʑɯn} 
\classe{vs} 
\begin{définition}\pfra{en pente}\end{définition}
\begin{définition}\pcmn{陡;斜度高}\end{définition}
\begin{exemple}\pjya{tʂu ki pjɯ-nɤʑɯn ɲɯ-ŋu, tɕe mɯ́j-nɯɣɯŋke ri, mɯ́j-ʁdɯɣ ma ɲɯ-nɯtɕhɯrɟɯɣ tɕe ɲɯ-ɣɤzbaʁ}\hspace{5pt}\pcmn{这条路很陡,虽然不好走,但是水流得很快(下雨的时候不会积水,路面就很快干)}\end{exemple}\relationsémantique{参考}{\lien{ⓔɣɤʑɯn}{ɣɤʑɯn}}\relationsémantique{参考}{\lien{ⓔɣɤʑɯn}{ɣɤʑɯn}}\end{entrée}

\begin{entrée}{nɤʑɯʑu}{}{ⓔnɤʑɯʑu}\relationsémantique{参考}{\lien{ⓔaʑɯʑu}{aʑɯʑu}}\end{entrée}

\begin{entrée}{nbraʁ}{}{ⓔnbraʁ} 
\classe{vt} \paradigme{dir}{lɤ-}
\begin{définition}\pfra{rendre la terre plus meuble}\end{définition}
\begin{définition}\pcmn{锄(麦,青稞)}\end{définition}
\begin{exemple}\pjya{tɤɕi tɤ-nbraʁ-a}\hspace{5pt}\pcmn{我锄了地种青稞}\end{exemple}
\begin{exemple}\pjya{tɯ-ji ɯ-qhu kɯ-ɤrmbat tɕe, kɤ-nbraʁ tu-mda ŋu}\hspace{5pt}\pcmn{收割之后不久,就是松土的时候}\end{exemple}\end{entrée}

\begin{entrée}{ndɤɣ}{}{ⓔndɤɣ} 
\classe{vs} \paradigme{dir}{pɯ-}\paradigme{dir}{thɯ-}\paradigme{dir}{nɯ-}
\begin{définition}\pfra{avoir trempé (assez)}\end{définition}
\begin{définition}\pcmn{泡好}\end{définition}
\begin{définition}\pfra{tremper}\end{définition}
\begin{définition}\pcmn{浸泡}\end{définition}
\begin{exemple}\pjya{tɤjlu ɲɤ-ndɤɣ}\hspace{5pt}\pcmn{面粉泡好了}\end{exemple}
\begin{exemple}\pjya{nɤ-ŋga pɯ-ɣɤle a-nɯ-ndɤɣ tɕe kɤ-χtɕi mbat}\hspace{5pt}\pcmn{把衣服泡好,这样就容易洗干净}\end{exemple}
\begin{exemple}\pjya{paʁtshi chɤ-sɯɣndɤɣ}\hspace{5pt}\pcmn{他把猪食浸泡了}\end{exemple}
\begin{exemple}\pjya{tɯ-ŋga ko-ɴqhi tɕe pjɯ́-wɣ-sɯɣndɤɣ ɲɯ-ɬoʁ}\hspace{5pt}\pcmn{衣服脏了就要浸泡}\end{exemple}
\begin{sous-entrée}{sɯɣndɤɣ}{ⓔndɤɣⓝsɯɣndɤɣ} 
\classe{vt}  
\grammaire{caus} \end{sous-entrée}

\end{entrée}

\begin{entrée}{ndɤre}{}{ⓔndɤre} 
\classe{cnj} 
\begin{définition}\pfra{par contre}\end{définition}
\begin{définition}\pcmn{反而}\end{définition}\end{entrée}

\begin{entrée}{ndɤrmbjom}{}{ⓔndɤrmbjom} 
\classe{vs} 
\begin{définition}\pfra{aux mouvements rapides}\end{définition}
\begin{définition}\pcmn{勤快,动作伶俐}\end{définition}\relationsémantique{同义词}{\lien{ⓔɣɤphɯɕlaʁ}{ɣɤphɯɕlaʁ}}\relationsémantique{参考}{\lien{ⓔmbjom}{mbjom}}
\begin{sous-entrée}{sɯndɤrmbjom}{ⓔndɤrmbjomⓝsɯndɤrmbjom} 
\classe{vs} 
\begin{définition}\pfra{rendre rapide}\end{définition}
\begin{définition}\pcmn{令……动作伶俐}\end{définition}
\begin{exemple}\pjya{tɤ-pɤtso ɲɯ́-wɣ-sɯxtɕɤt tɕe kɤ-sɯndɤrmbjom tu}\hspace{5pt}\pcmn{对孩子教导好了的话,就可以令他变得勤快}\end{exemple}\end{sous-entrée}

\end{entrée}

\begin{entrée}{ndɤrndɤr}{}{ⓔndɤrndɤr} 
\classe{idph.2} 
\begin{définition}\pfra{grand, imposant}\end{définition}
\begin{définition}\pcmn{强壮}\end{définition}
\begin{exemple}\pjya{jla ci rcanɯ kɯ-wxti ndɤrndɤr ci ɲɯ-ŋu}\hspace{5pt}\pcmn{犏牛很强壮}\end{exemple}
\begin{exemple}\pjya{ɯʑo ɯ-phoŋbu ndɤrndɤr ʑo ɲɯ-pa}\hspace{5pt}\pcmn{他身材很高大}\end{exemple}
\begin{sous-entrée}{mɤlɤndɤr}{ⓔndɤrndɤrⓝmɤlɤndɤr}
\begin{définition}\pfra{imposant}\end{définition}
\begin{définition}\pcmn{高大}\end{définition}
\begin{exemple}\pjya{kha rcanɯ mɤlɤndɤr ci ɲɯ-ŋu}\hspace{5pt}\pcmn{房子很高大}\end{exemple}\end{sous-entrée}

\end{entrée}

\begin{entrée}{ndɣɤndɣɤt}{}{ⓔndɣɤndɣɤt} 
\classe{idph.2} 
\begin{définition}\pfra{(bois) entassé, très haut}\end{définition}
\begin{définition}\pcmn{形容柴堆得很高的样子}\end{définition}
\begin{exemple}\pjya{sɯpɣo ndɣɤndɣɤt ʑo pjɤ-ta}\hspace{5pt}\pcmn{柴垛子堆得很高}\end{exemple}\relationsémantique{参考}{\lien{ⓔɣɤndɣɤndɣɤt}{ɣɤndɣɤndɣɤt}}\end{entrée}

\begin{entrée}{ndɣɤt}{}{ⓔndɣɤt} 
\classe{idph.1} 
\begin{définition}\pfra{sursaut de frayeur}\end{définition}
\begin{définition}\pcmn{(吓了)一跳}\end{définition}
\begin{exemple}\pjya{ndɣɤt ʑo pjɤ́-wɣ-znɤscɤr}\hspace{5pt}\pcmn{把他吓了一跳}\end{exemple}\end{entrée}

\begin{entrée}{ndjɤndjɤt}{}{ⓔndjɤndjɤt} 
\classe{idph.2} 
\begin{définition}\pfra{imposant et gracieux}\end{définition}
\begin{définition}\pcmn{亭亭玉立}\end{définition}
\begin{exemple}\pjya{tɕhemɤpɯ ci ndjɤndjɤt ɲɯ-ŋu}\hspace{5pt}\pcmn{女孩子亭亭玉立}\end{exemple}
\begin{exemple}\pjya{tɕheme nɯ ɯ-phoŋbu ɯ-tɯ-mpɕɤr kɯ ndjɤndjɤt ʑo ɲɯ-pa}\hspace{5pt}\pcmn{这个女子(身材)亭亭玉立}\end{exemple}\end{entrée}

\begin{entrée}{ndo}{}{ⓔndo} 
\classe{vt}
\classe{np}
\classe{vt}
\classe{np}
\classe{vt}  
\grammaire{recip}
\grammaire{caus}
\grammaire{refl} \sens{1}\paradigme{dir}{\_}
\begin{définition}\pfra{tenir, prendre}\end{définition}
\begin{définition}\pcmn{拿}\end{définition}
\begin{exemple}\pjya{mbrɯtɕɯ ɲɯ-tɯ-ɤsɯ-ndo}\hspace{5pt}\pcmn{你在拿刀}\end{exemple}
\begin{exemple}\pjya{nɤ-mtɕhi kɤ-ndɤm}\hspace{5pt}\pcmn{你闭嘴}\end{exemple}
\begin{exemple}\pjya{nɤ-jaʁ kɤ-ndɤm}\hspace{5pt}\pcmn{不要插手}\end{exemple}\sens{2}\paradigme{dir}{kɤ-}
\begin{définition}\pfra{attraper}\end{définition}
\begin{définition}\pcmn{抓到;捕到}\end{définition}
\begin{exemple}\pjya{ɣzɯ ɣɯ ɯ-pɯ ci ko-ndo tɕe jo-ɣɯt}\hspace{5pt}\pcmn{(我们的邻居)抓到了猴崽子并把它带来了}\end{exemple}\sens{3}\paradigme{dir}{tɤ-}
\begin{définition}\pfra{devenir, prendre une charge}\end{définition}
\begin{définition}\pcmn{当上}\end{définition}
\begin{exemple}\pjya{rɟɤlpu to-ndo}\hspace{5pt}\pcmn{他当上国王}\end{exemple}\sens{4}\paradigme{dir}{tɤ-}
\begin{définition}\pfra{tomber enceinte (animal femelle)}\end{définition}
\begin{définition}\pcmn{怀胎(动物)}\end{définition}
\begin{exemple}\pjya{paʁ ɣɯ ɯ-pɯ to-ndo}\hspace{5pt}\pcmn{母猪怀了胎了}\end{exemple}\sens{5}\paradigme{dir}{kɤ-}
\begin{définition}\pfra{attraper (une maladie grave)}\end{définition}
\begin{définition}\pcmn{得病(致命的病)}\end{définition}
\begin{exemple}\pjya{ɯʑo kɯ @aizheng ko-ndo}\hspace{5pt}\pcmn{他得了癌症}\end{exemple}\sens{6}\paradigme{dir}{kɤ-}
\begin{définition}\pfra{garder (chemin)}\end{définition}
\begin{définition}\pcmn{守(路)}\end{définition}
\begin{exemple}\pjya{aʑo kɯre tʂu ku-osɯ-ndo-a}\hspace{5pt}\pcmn{我正在守路}\end{exemple}\sens{7}\paradigme{dir}{kɤ-}
\begin{définition}\pfra{entrer dans ses ... ans}\end{définition}
\begin{définition}\pcmn{进入……岁}\end{définition}
\begin{exemple}\pjya{kɯβdesqamnɯz ko-ndo}\hspace{5pt}\pcmn{他开始进入42岁了}\end{exemple}\relationsémantique{参考}{\lien{ⓔnɤndɯndo}{nɤndɯndo}}\relationsémantique{参考}{\lien{ⓔando}{ando}}\relationsémantique{参考}{\lien{ⓔnɤsnɯndo}{nɤsnɯndo}}\relationsémantique{参考}{\lien{ⓔrtsawaⓝɯ-rtsawa,ndo}{ɯ-rtsawa,ndo}}\relationsémantique{Component 1}{\lien{ⓔtɯ-phoŋbu}{tɯ-phoŋbu}}\relationsémantique{Component 2}{\lien{ⓔndo}{ndo}}
\begin{sous-entrée}{tɯ-phoŋbu,ndo}{ⓔndoⓢ7ⓝtɯ-phoŋbu,ndo}\end{sous-entrée}

\begin{définition}\pfra{trembler de froid}\end{définition}
\begin{définition}\pcmn{(冷得)发抖(无法控制)}\end{définition}
\begin{exemple}\pjya{a-phoŋbu kɤ-ndo ʑo mɯ́j-khɯ}\hspace{5pt}\pcmn{我冷得发抖}\end{exemple}\relationsémantique{Component 1}{\lien{ⓔɯ-mdoʁ}{ɯ-mdoʁ}}\relationsémantique{Component 2}{\lien{ⓔndo}{ndo}}
\begin{sous-entrée}{ɯ-mdoʁ,ndo}{ⓔndoⓝɯ-mdoʁ,ndo}\end{sous-entrée}

\paradigme{dir}{\_}\paradigme{dir}{nɯ-}
\begin{définition}\pfra{avoir (couleur)}\end{définition}
\begin{définition}\pcmn{有(颜色)}\end{définition}
\begin{définition}\pfra{prendre avec}\end{définition}
\begin{définition}\pcmn{用……拿、用……固定、使带走、配种}\end{définition}
\begin{exemple}\pjya{tɤtshoʁ kɯ kɤ-sɯ-ndo-t-a}\hspace{5pt}\pcmn{我用钉子固定了}\end{exemple}
\begin{exemple}\pjya{tɤ-ri kɯ (tɯ-xtsa, tɯ-ŋga) kɤ-sɯ-ndo-t-a}\hspace{5pt}\pcmn{我用线固定了}\end{exemple}
\begin{sous-entrée}{sɯndo}{ⓔndoⓝsɯndo} 
\classe{vt}  
\grammaire{caus} \end{sous-entrée}

\begin{sous-entrée}{nɯɣɯndo}{ⓔndoⓝnɯɣɯndo} 
\classe{vi} 
\begin{définition}\pfra{facile à prendre, facile à attraper}\end{définition}
\begin{définition}\pcmn{容易抓住}\end{définition}\end{sous-entrée}

\begin{sous-entrée}{andɯndo}{ⓔndoⓝandɯndo} 
\classe{vi} \end{sous-entrée}

\paradigme{dir}{nɯ-}\paradigme{dir}{kɤ-}
\begin{définition}\pfra{s'attacher}\end{définition}
\begin{définition}\pcmn{粘在一起}\end{définition}
\begin{définition}\pfra{clouer, coller ensemble}\end{définition}
\begin{définition}\pcmn{钉在一起;粘在一起}\end{définition}
\begin{exemple}\pjya{qraʁ ɯ-kɯ-spoʁ ci tu tɕe, ɕɤmtshoʁ pjɯ́-wɣ-no tɕe mbɣo cho qraʁ ni ɲɯ́-wɣ-sɤndɯndo ra}\hspace{5pt}\pcmn{铧头中间有个洞,在那里钉个钉子,把铧头钉在犁头上}\end{exemple}
\begin{sous-entrée}{sɤndɯndo}{ⓔndoⓝsɤndɯndo} 
\classe{vt} \end{sous-entrée}

\begin{sous-entrée}{ʑɣɤsɯndo}{ⓔndoⓝʑɣɤsɯndo} 
\classe{vi} \end{sous-entrée}

\sens{1}
\begin{définition}\pfra{se faire attraper}\end{définition}
\begin{définition}\pcmn{被人抓}\end{définition}\sens{2}
\begin{définition}\pfra{se contrôler}\end{définition}
\begin{définition}\pcmn{自我控制}\end{définition}
\begin{exemple}\pjya{ɯʑo kɤ-ʑɣɤsɯndo mɤ-kɯ-cha ci ɲɯ-ŋu}\hspace{5pt}\pcmn{他是不能控制自己的人}\end{exemple}\end{entrée}

\begin{entrée}{ndom}{}{ⓔndom} 
\classe{vi.nh}  
\grammaire{acaus} \paradigme{dir}{nɯ-}
\begin{définition}\pfra{horizontal}\end{définition}
\begin{définition}\pcmn{横}\end{définition}
\begin{exemple}\pjya{tʂu ri ɕoŋtɕa ci pjɤ-ndom}\hspace{5pt}\pcmn{路中间横放着一条木柴}\end{exemple}
\begin{exemple}\pjya{pjɤ-ndʐaβ tɕe pjɤ-ndom}\hspace{5pt}\pcmn{他跌倒了,躺在地上了}\end{exemple}\relationsémantique{参考}{\lien{ⓔxthom}{xthom}}\end{entrée}

\begin{entrée}{ndoʁ}{}{ⓔndoʁ} 
\classe{vs} \paradigme{dir}{nɯ-}\paradigme{dir}{tɤ-}
\begin{définition}\pfra{croquant, craquant (fruit, branche d'arbre)}\end{définition}
\begin{définition}\pcmn{脆(水果、树枝)}\end{définition}
\begin{exemple}\pjya{si ɲɯ-ndoʁ}\hspace{5pt}\pcmn{木头是脆的}\end{exemple}
\begin{exemple}\pjya{ʑmbri ɯ-rtaʁ a-nɯ-lni tɕe, tɕe mɤ-ndoʁ tɕe kɤ-qlɯt mɤ-khɯ}\hspace{5pt}\pcmn{杨柳枝晒干了就不脆,不容易折}\end{exemple}\end{entrée}

\begin{entrée}{ndɯ}{₁}{ⓔndɯⓗ1} 
\classe{vi}  
\grammaire{acaus} \paradigme{dir}{\_}\sens{1}
\begin{définition}\pfra{être construit, être praticable (chemin, pont)}\end{définition}
\begin{définition}\pcmn{开通(桥、路)}\end{définition}
\begin{exemple}\pjya{tʂu ko-ndɯ (lo-ndɯ, ɲɤ-ndɯ, pjɤ-ndɯ)}\hspace{5pt}\pcmn{路通了}\end{exemple}
\begin{exemple}\pjya{ndzom ko-ndɯ}\hspace{5pt}\pcmn{桥通了}\end{exemple}
\begin{exemple}\pjya{kɯki ɯ-stu ki tʂu pɯ-me ri, pjɤ-ndɯ}\hspace{5pt}\pcmn{以前没有路,现在就通了}\end{exemple}
\begin{exemple}\pjya{kukɯtɕu tʂu pjɤ-nɯ-ndɯndɯ ɕti}\hspace{5pt}\pcmn{以前这里路是通的}\end{exemple}
\begin{exemple}\pjya{tɕhi to-ndɯ}\hspace{5pt}\pcmn{有梯子}\end{exemple}
\begin{exemple}\pjya{ɯ-jroʁ ko-ndɯ}\hspace{5pt}\pcmn{他留了痕迹}\end{exemple}\sens{2}
\begin{définition}\pfra{apparaître (arc-en-ciel)}\end{définition}
\begin{définition}\pcmn{出现(彩虹)}\end{définition}
\begin{exemple}\pjya{ndʑa pjɤ-ndɯ}\hspace{5pt}\pcmn{出现了彩虹}\end{exemple}\relationsémantique{参考}{\lien{ⓔthɯⓗ1}{thɯ₁}}\end{entrée}

\begin{entrée}{ndɯ}{₂}{ⓔndɯⓗ2} 
\classe{vi} \paradigme{dir}{tɤ-}
\begin{définition}\pfra{s'accumuler}\end{définition}
\begin{définition}\pcmn{积累}\end{définition}
\begin{exemple}\pjya{ji-kɯ-rtoʁ pɯ-dɤn tɕɤn, laχtɕha khro to-ndɯ}\hspace{5pt}\pcmn{来看我们的人很多,所以收了很多东西}\end{exemple}
\begin{exemple}\pjya{jisŋi tɯtsɣe nɯ khro to-ndɯ}\hspace{5pt}\pcmn{我们今天生意(很好),赚到很多}\end{exemple}
\begin{exemple}\pjya{tɯmpɕar nɤ tɯmpɕar ntsɯ na-kho-nɯ tɕe, tham tɕe sqamŋu-mpɕar jamar to-ndɯ}\hspace{5pt}\pcmn{(虽然)他们一块一块地给,(但是)现在筹到十五块了}\end{exemple}
\begin{exemple}\pjya{wo nɯ sthɯci kɯ-rkɯn nɤ, ndɯ ci me loβ}\hspace{5pt}\pcmn{(他们给的)那么少,积攒不到很多}\end{exemple}
\begin{exemple}\pjya{tɕhaʁra kɯ-rɯru nɯ kɯ kɯmŋu toŋtsi ku-wum ɲɯ-ɕti ri, tɯrmɯkha tɕe ʁnɯ-ri χsɯ-ri tu-ndɯ ɲɯ-ɕti / ku-ojtɯ ɕti}\hspace{5pt}\pcmn{看厕所的人虽然每个人只拿五毛钱,到了下午就能积攒到两三百块钱}\end{exemple}\relationsémantique{参考}{\lien{ⓔajtɯ}{ajtɯ}}\end{entrée}

\begin{entrée}{ndɯβ}{}{ⓔndɯβ} 
\classe{vs} \paradigme{dir}{nɯ-}
\begin{définition}\pfra{fine (poudre)}\end{définition}
\begin{définition}\pcmn{细(粉状)}\end{définition}
\begin{exemple}\pjya{mɯ́j-ndʐɤz kɯ ɲɯ-ndɯβ}\hspace{5pt}\pcmn{不粗,很细}\end{exemple}\relationsémantique{反义词}{\lien{ⓔjndʐɤz}{jndʐɤz}}\relationsémantique{参考}{\lien{ⓔɣɤndɯβ}{ɣɤndɯβ}}\end{entrée}

\begin{entrée}{ndɯchu}{}{ⓔndɯchu} 
\classe{adv} 
\begin{définition}\pfra{à l'ouest}\end{définition}
\begin{définition}\pcmn{在西部}\end{définition}\relationsémantique{参考}{\lien{ⓔɯ-ndɤcu}{ɯ-ndɤcu}}\end{entrée}

\begin{entrée}{ndɯɣsa}{}{ⓔndɯɣsa} 
\classe{n} 
\begin{définition}\pfra{lieu de résidence}\end{définition}
\begin{définition}\pcmn{住处(敬语)}\end{définition}\étymologie{ɴdug.sa}\end{entrée}

\begin{entrée}{ndɯl}{₁}{ⓔndɯlⓗ1} 
\classe{vi} \paradigme{dir}{pɯ-}\paradigme{dir}{pɯ-}
\begin{définition}\pfra{être apprivoisé}\end{définition}
\begin{définition}\pcmn{被驯服}\end{définition}
\begin{définition}\pfra{apprivoiser}\end{définition}
\begin{définition}\pcmn{驯服}\end{définition}
\begin{exemple}\pjya{kɯki jla ki kɤ-ndɯl wuma ʑo pɯ-mbat}\hspace{5pt}\pcmn{这头犏牛很好驯服}\end{exemple}
\begin{exemple}\pjya{mbro nɯ pɯ-sɯɣndɯl-a}\hspace{5pt}\pcmn{我驯服了这匹马}\end{exemple}
\begin{sous-entrée}{sɯɣndɯl}{ⓔndɯlⓗ1ⓝsɯɣndɯl} 
\classe{vt} \end{sous-entrée}

\begin{sous-entrée}{ɣɤndɯl}{ⓔndɯlⓗ1ⓝɣɤndɯl} 
\classe{vi}  
\grammaire{facil} 
\begin{définition}\pfra{facile à apprivoiser}\end{définition}
\begin{définition}\pcmn{容易驯服}\end{définition}\relationsémantique{同义词}{\lien{ⓔftɯlⓗ1ⓝnɯɣɯftɯl}{nɯɣɯftɯl}}\end{sous-entrée}

\end{entrée}

\begin{entrée}{ndɯl}{₂}{ⓔndɯlⓗ2} 
\classe{vt} \paradigme{dir}{thɯ-}\paradigme{dir}{nɯ-}
\begin{définition}\pfra{pulvériser, mettre en poudre}\end{définition}
\begin{définition}\pcmn{磨细}\end{définition}
\begin{exemple}\pjya{ɕnɤto thɯ-ndɯl-a}\hspace{5pt}\pcmn{我把鼻烟磨细了}\end{exemple}\end{entrée}

\begin{entrée}{ndɯn}{}{ⓔndɯn} 
\classe{vt} \paradigme{dir}{pɯ-}\paradigme{dir}{kɤ-}
\begin{définition}\pfra{lire à haute voix}\end{définition}
\begin{définition}\pcmn{读}\end{définition}\paradigme{dir}{pɯ-}
\begin{définition}\pfra{lire à haute voix}\end{définition}
\begin{définition}\pcmn{诵经;背书}\end{définition}
\begin{exemple}\pjya{tɕe nɤ-χpi pɯ-ndɯn jɤɣ}\hspace{5pt}\pcmn{好吧,你念你的故事吧}\end{exemple}
\begin{exemple}\pjya{ɯ-mphru pɯ-ʑe tɕe pɯ-ndɯn}\hspace{5pt}\pcmn{从头开始念吧}\end{exemple}
\begin{exemple}\pjya{χpɯn ra ɲɯ-rɤndɯn-nɯ}\hspace{5pt}\pcmn{和尚们在诵经}\end{exemple}
\begin{sous-entrée}{rɤndɯn}{ⓔndɯnⓝrɤndɯn} 
\classe{vi}  
\grammaire{apass} \end{sous-entrée}

\étymologie{ⁿdon}\end{entrée}

\begin{entrée}{ndɯχu/\variante{dɯχu}}{}{ⓔndɯχu} 
\classe{n} 
\begin{définition}\pfra{lis}\end{définition}
\begin{définition}\pcmn{百合}\end{définition}
\begin{exemple}\pjya{ndɯχu nɯ sɯjno ci ŋu, sɯŋgɯ cho praʁ ɯ-rchɤβ ra kɤ-ɬoʁ rga, ɯ-qa nɯ ɕkɤtɯm kɯ-fse ɲɯ-βze ŋu, ɯ-ru nɯ kɯ-rɲɟi tsa ʑo tu-fse cha, ɯ-rtaʁ ɲɯ-ɬoʁ mɤ-cha. ɯ-jwaʁ nɯ kɯ-xtɕi tsa kɯ-ɤmtɕoʁ tsa ŋu. ɯ-mɯntoʁ nɯ wuma ʑo mpɕɤr, ʁmɤrsɤr ŋu tɕe, ɲɯ-ʁaʁ tɕe, ɯ-mɯntoʁ nɯ pjɯ-nɯqhɤɴɢaʁ ɯ-qhu chu pɕoʁ tu-ŋgɤɣ ŋu, tɕe ɯ-taʁ kɯ-ɲaʁ kɯ-ɤkhra tu.}\hspace{5pt}\pcmn{百合是一种植物,一般生长在灌木丛中和岩石上,根部像大蒜的根一样,茎细长,不分叉。叶子小而尖。花很美,金黄色。开花时,花瓣向后面卷起来,上面有黑点。}\end{exemple}\end{entrée}

\begin{entrée}{ndzu}{}{ⓔndzu} 
\classe{vi}  
\grammaire{caus} \paradigme{dir}{tɤ-}\paradigme{dir}{tɤ-}
\begin{définition}\pfra{être prêt}\end{définition}
\begin{définition}\pcmn{准备好了,正要出发}\end{définition}
\begin{exemple}\pjya{kɤ-ɕe pɤjkhu mɤ-ndzu-a}\hspace{5pt}\pcmn{我还没有准备出发}\end{exemple}
\begin{exemple}\pjya{kɤ-ɕe tɤ-ndzu-a}\hspace{5pt}\pcmn{我准备出发了}\end{exemple}
\begin{exemple}\pjya{ʑa a-tɤ-ndzu ɲɯ-ra tɕe tɤ-ɕɯmbɣom}\hspace{5pt}\pcmn{你要早一点启程的话,就让他快点}\end{exemple}
\begin{sous-entrée}{sɯɣndzu}{ⓔndzuⓝsɯɣndzu} 
\classe{vt} \end{sous-entrée}

\begin{définition}\pfra{préparer au départ}\end{définition}
\begin{définition}\pcmn{让……启程}\end{définition}\end{entrée}

\begin{entrée}{ndza}{}{ⓔndza} 
\classe{vt}  
\grammaire{habil} \paradigme{dir}{tɤ-}\paradigme{dir}{thɯ-}\sens{1}
\begin{définition}\pfra{manger}\end{définition}
\begin{définition}\pcmn{吃}\end{définition}
\begin{définition}\pfra{tɕhi tɤ-tɯ-ndza-t?}\end{définition}
\begin{définition}\pcmn{你吃了什么?}\end{définition}\sens{2}\paradigme{dir}{tɤ-}
\begin{définition}\pfra{mâcher}\end{définition}
\begin{définition}\pcmn{咀嚼}\end{définition}
\begin{exemple}\pjya{βʑɯ kɯ tɯ-ŋga to-ndza}\hspace{5pt}\pcmn{老鼠把衣服咬破了}\end{exemple}
\begin{sous-entrée}{sɯndza}{ⓔndzaⓢ2ⓝsɯndza} 
\classe{vt} \end{sous-entrée}

\sens{1}
\begin{définition}\pfra{manger avec, faire manger}\end{définition}
\begin{définition}\pcmn{用……吃}\end{définition}
\begin{exemple}\pjya{ɯ-ʁe nɯ kɯ tu-sɯndze ŋu}\hspace{5pt}\pcmn{她用左(手)吃饭(因为右手受伤了)}\end{exemple}\sens{2}
\begin{définition}\pfra{faire manger}\end{définition}
\begin{définition}\pcmn{使……吃}\end{définition}\sens{3}
\begin{définition}\pfra{pouvoir manger}\end{définition}
\begin{définition}\pcmn{吃得下}\end{définition}
\begin{exemple}\pjya{nɯ thamtɕɤt mɯ́j-sɯndze-a (mɯ́j-sɯ-ɕkɯt-a)}\hspace{5pt}\pcmn{我吃不下那么多}\end{exemple}
\begin{sous-entrée}{nɯɣɯndza}{ⓔndzaⓝnɯɣɯndza} 
\classe{vs} 
\begin{définition}\pfra{agréable à manger}\end{définition}
\begin{définition}\pcmn{吃着顺口}\end{définition}\end{sous-entrée}

\begin{sous-entrée}{sɤndza}{ⓔndzaⓝsɤndza} 
\classe{vs}  
\grammaire{apass} 
\begin{définition}\pfra{piquer}\end{définition}
\begin{définition}\pcmn{刺人}\end{définition}
\begin{exemple}\pjya{mɤ-sɤndza ma ɯ-mdzu me}\hspace{5pt}\pcmn{没有刺所以刺不到人}\end{exemple}\relationsémantique{参考}{\lien{ⓔandzɯndza}{andzɯndza}}\relationsémantique{参考}{\lien{ⓔndzaⓢ2ⓝsɯndza}{sɯndza}}\end{sous-entrée}

\end{entrée}

\begin{entrée}{ndzamthaŋ}{}{ⓔndzamthaŋ} 
\classe{n}  
\grammaire{n.lieu} 
\begin{définition}\pfra{Ndzamthang}\end{définition}
\begin{définition}\pcmn{壤塘}\end{définition}\end{entrée}

\begin{entrée}{ndzar}{}{ⓔndzar} 
\classe{vi} \paradigme{dir}{pɯ-}
\begin{définition}\pfra{s'égoutter complètement}\end{définition}
\begin{définition}\pcmn{滤干;滗干}\end{définition}
\begin{exemple}\pjya{tɯ-mɯ pjɤ-ndzar}\hspace{5pt}\pcmn{雨下得不会再下}\end{exemple}
\begin{exemple}\pjya{a-ŋga ɯ-taʁ tɯ-ci mɯ-pɯ-ndzar mɤɕtʂa pɯ-ndzur-a}\hspace{5pt}\pcmn{一直站着,等到衣服上的水滴干了}\end{exemple}
\begin{exemple}\pjya{tɯ-ŋga nɯ-χtɕi-t-a tɕe a-pɯ-ndzar tɕe chɯ́-wɣ-ɕkho jɤɣ}\hspace{5pt}\pcmn{我洗了衣服,现在要滴干了才能晒}\end{exemple}
\begin{exemple}\pjya{tɯ-ŋga nɯ-χtɕi-t-a tɕe a-pɯ-ndzar tɕe tɕetha ʑatsa zbaʁ}\hspace{5pt}\pcmn{我洗了衣服,现在要滴干了就会很快变干}\end{exemple}
\begin{exemple}\pjya{@cai tɤ́-wɣ-χtɕi tɕe @shaoji ɯ-ŋgɯ kɤ-rku ra ma kɤ-ndzar mɤ-cha}\hspace{5pt}\pcmn{洗了菜以后要放在簸箕里,不然就不能变干}\end{exemple}
\begin{exemple}\pjya{nɤ-ŋga nɯ-tɯ-χtɕi tɕe a-tɤ-tɯ-ɕɯɴqoʁ tɕe, tɕe kɤ-ndzar cha}\hspace{5pt}\pcmn{洗了衣服以后一定要把它挂起来才能变干}\end{exemple}\relationsémantique{参考}{\lien{ⓔkɯndzarmɯ}{kɯndzarmɯ}}\relationsémantique{参考}{\lien{ⓔmdzar}{mdzar}}
\begin{sous-entrée}{sɯɣndzar}{ⓔndzarⓝsɯɣndzar} 
\classe{vt}  
\grammaire{caus} 
\begin{définition}\pfra{laisser décanter, s'égoutter}\end{définition}
\begin{définition}\pcmn{把……滤干}\end{définition}
\begin{exemple}\pjya{mbrɤz tɤ́-wɣ-χtɕi tɕe, kɤ-sɯɣndzar ra}\hspace{5pt}\pcmn{淘了米就要把它滤干}\end{exemple}\end{sous-entrée}

\end{entrée}

\begin{entrée}{ndzaʁlaŋ}{}{ⓔndzaʁlaŋ} 
\classe{n} 
\begin{définition}\pfra{le monde}\end{définition}
\begin{définition}\pcmn{世界}\end{définition}\étymologie{ⁿdzam.gliŋ}\end{entrée}

\begin{entrée}{ndzɤβ}{}{ⓔndzɤβ} 
\classe{vs} \paradigme{dir}{nɯ-}
\begin{définition}\pfra{collant, épais (gruau)}\end{définition}
\begin{définition}\pcmn{稠(粥)、黏(比较干)}\end{définition}
\begin{exemple}\pjya{tɯtshi ɲɯ-ndzɤβ}\hspace{5pt}\pcmn{粥很稠}\end{exemple}
\begin{exemple}\pjya{tɕhɯβroʁ kɯ-ndzɤβ}\hspace{5pt}\pcmn{稠糌粑}\end{exemple}
\begin{exemple}\pjya{smar chɤ-ndzɤβ}\hspace{5pt}\pcmn{河里的泥沙变多了}\end{exemple}
\begin{sous-entrée}{nɤndzɤβ}{ⓔndzɤβⓝnɤndzɤβ} 
\classe{vt}  
\grammaire{trop} 
\begin{définition}\pfra{trouver épais}\end{définition}
\begin{définition}\pcmn{觉得很稠(水分少)}\end{définition}\relationsémantique{反义词}{\lien{ⓔŋgri}{ŋgri}}\end{sous-entrée}

\end{entrée}

\begin{entrée}{ndzɤβrta}{}{ⓔndzɤβrta} 
\classe{n} 
\begin{définition}\pfra{grosse perle du rosaire, utilisée pour compter les tours}\end{définition}
\begin{définition}\pcmn{计数用的珠子(玛尼珠)}\end{définition}\end{entrée}

\begin{entrée}{ndzɤpri}{}{ⓔndzɤpri} 
\classe{n} 
\begin{définition}\pfra{ours brun}\end{définition}
\begin{définition}\pcmn{马熊}\end{définition}\relationsémantique{参考}{\lien{ⓔpriⓗ2}{pri₂}}\end{entrée}

\begin{entrée}{ndzɤqhɤjɯ}{}{ⓔndzɤqhɤjɯ} 
\classe{n} 
\begin{définition}\pfra{(fait) de manger dans son coin}\end{définition}
\begin{définition}\pcmn{偷吃}\end{définition}
\begin{exemple}\pjya{ndzɤqhɤjɯ to-βzu}\hspace{5pt}\pcmn{他偷吃了很多}\end{exemple}\relationsémantique{参考}{\lien{ⓔrɯndzɤqhɤjɯ}{rɯndzɤqhɤjɯ}}\end{entrée}

\begin{entrée}{ndzɤrndzɤr}{}{ⓔndzɤrndzɤr} 
\classe{idph.2} 
\begin{définition}\pfra{seul}\end{définition}
\begin{définition}\pcmn{形容独自一人的样子}\end{définition}
\begin{exemple}\pjya{tɯrme tɯ-rdoʁ ndzɤrndzɤr}\hspace{5pt}\pcmn{单独一个人}\end{exemple}\end{entrée}

\begin{entrée}{ndzɤt}{}{ⓔndzɤt} 
\classe{vi} \paradigme{dir}{thɯ-}\paradigme{dir}{tɤ-}
\begin{définition}\pfra{grandir}\end{définition}
\begin{définition}\pcmn{长大}\end{définition}
\begin{exemple}\pjya{stoʁ ɕɤxɕo tɕe to-ndzɤt}\hspace{5pt}\pcmn{胡豆这几天长大}\end{exemple}
\begin{exemple}\pjya{tɤ-pɤtso cho-ndzɤt}\hspace{5pt}\pcmn{孩子长大了}\end{exemple}\end{entrée}

\begin{entrée}{ndzɤtshi}{₁}{ⓔndzɤtshiⓗ1} 
\classe{n} 
\begin{définition}\pfra{plat}\end{définition}
\begin{définition}\pcmn{饭菜}\end{définition}\relationsémantique{参考}{\lien{ⓔrɯndzɤtshi}{rɯndzɤtshi}}\end{entrée}

\begin{entrée}{ndzɤtshi}{₂}{ⓔndzɤtshiⓗ2} 
\classe{vt}  
\grammaire{comp} \paradigme{dir}{tɤ-}
\begin{définition}\pfra{manger et boire}\end{définition}
\begin{définition}\pcmn{吃喝}\end{définition}
\begin{exemple}\pjya{ɯʑo kɯ tɯ-mgo ra to-ndzɤtshi}\hspace{5pt}\pcmn{他把饭吃了}\end{exemple}\end{entrée}

\begin{entrée}{ndzi}{}{ⓔndzi} 
\classe{vs} 
\begin{définition}\pfra{enroué}\end{définition}
\begin{définition}\pcmn{嗓子哑了}\end{définition}
\begin{exemple}\pjya{a-rqo ko-ndzi}\hspace{5pt}\pcmn{我嗓子哑了}\end{exemple}
\begin{sous-entrée}{sɯɣndzi}{ⓔndziⓝsɯɣndzi} 
\classe{vt}  
\grammaire{caus} 
\begin{définition}\pfra{rendre enroué}\end{définition}
\begin{définition}\pcmn{令……嗓子哑}\end{définition}\end{sous-entrée}

\end{entrée}

\begin{entrée}{ndziaʁ}{}{ⓔndziaʁ} 
\classe{vs} \sens{1}\paradigme{dir}{thɯ-}
\begin{définition}\pfra{solide (nœud)}\end{définition}
\begin{définition}\pcmn{紧(结)}\end{définition}
\begin{définition}\pcmn{这个结打得很紧,很难解开}\end{définition}
\begin{exemple}\pjya{kɯki tɤ-mtɯ chɤ-ndziaʁ tɕe kɤ-rla ɲɯ-ɴqa}\end{exemple}\sens{2}
\begin{définition}\pfra{foncé}\end{définition}
\begin{définition}\pcmn{深(颜色)}\end{définition}
\begin{exemple}\pjya{ɯ-mdoʁ ɲɯ-ndziaʁ}\hspace{5pt}\pcmn{颜色很深}\end{exemple}\sens{3}\paradigme{dir}{thɯ-}
\begin{définition}\pfra{complet (temps)}\end{définition}
\begin{définition}\pcmn{满、到期(时间)}\end{définition}
\begin{exemple}\pjya{tɯ-xpa ɲɤ-ndziaʁ}\hspace{5pt}\pcmn{满了一周年}\end{exemple}
\begin{exemple}\pjya{tɤ-mtɯ thɯ-ɣɤndziaʁ tɕe a-mɤ-nɯ-nɯ-ɬoʁ}\hspace{5pt}\pcmn{你把结打得紧些,不要让它散开}\end{exemple}
\begin{sous-entrée}{ɣɤndziaʁ}{ⓔndziaʁⓢ3ⓝɣɤndziaʁ} 
\classe{vt} \end{sous-entrée}

\end{entrée}

\begin{entrée}{ndzom}{₁}{ⓔndzomⓗ1} 
\classe{n} 
\begin{définition}\pfra{pont}\end{définition}
\begin{définition}\pcmn{桥}\end{définition}\relationsémantique{参考}{\lien{ⓔndzomⓗ2}{ndzom₂}}\end{entrée}

\begin{entrée}{ndzom}{₂}{ⓔndzomⓗ2} 
\classe{vi}  
\grammaire{denom} \paradigme{dir}{kɤ-}
\begin{définition}\pfra{couvrir le fleuve (glace)}\end{définition}
\begin{définition}\pcmn{覆盖河流(冰层)}\end{définition}
\begin{exemple}\pjya{tɯ-ci ɯ-taʁ tɤjpɣom ko-ndzom}\hspace{5pt}\pcmn{水上结成了一层冰}\end{exemple}\relationsémantique{参考}{\lien{ⓔndzomⓗ1}{ndzom₁}}\end{entrée}

\begin{entrée}{ndzoʁ}{}{ⓔndzoʁ} 
\classe{vi}  
\grammaire{acaus} \paradigme{dir}{kɤ-}
\begin{définition}\pfra{porter (fruits; partie d'un objet)}\end{définition}
\begin{définition}\pcmn{结果子,带有}\end{définition}
\begin{exemple}\pjya{ɯ-mat ko-ndzoʁ}\hspace{5pt}\pcmn{结果了}\end{exemple}
\begin{exemple}\pjya{ftɕar ko-ndzoʁ}\hspace{5pt}\pcmn{春天到了}\end{exemple}
\begin{exemple}\pjya{qartsɯ ko-ndzoʁ}\hspace{5pt}\pcmn{冬天到了}\end{exemple}
\begin{exemple}\pjya{jisŋi @qihao ko-ndzoʁ}\hspace{5pt}\pcmn{今天已经是七号了}\end{exemple}
\begin{exemple}\pjya{zdɯm zgo ɯ-taʁ pjɤ-ndzoʁ tɕe mɯ-to-ka}\hspace{5pt}\pcmn{云贴在山尖上,还没有离开地面}\end{exemple}\relationsémantique{参考}{\lien{ⓔtshoʁ}{tshoʁ}}\end{entrée}

\begin{entrée}{ndzoʁtɣa}{}{ⓔndzoʁtɣa} 
\classe{n} 
\begin{définition}\pfra{empan (pouce et index)}\end{définition}
\begin{définition}\pcmn{一拃(大拇指和食指之间的距离)}\end{définition}\end{entrée}

\begin{entrée}{ndzur}{}{ⓔndzur} 
\classe{vi} \paradigme{dir}{tɤ-}
\begin{définition}\pfra{être debout}\end{définition}
\begin{définition}\pcmn{站}\end{définition}
\begin{exemple}\pjya{aʑo tɤ-ndzur-a}\hspace{5pt}\pcmn{我站起来了}\end{exemple}
\begin{exemple}\pjya{nɤʑo to-tɯ-ndzur}\hspace{5pt}\pcmn{你站起来了}\end{exemple}\relationsémantique{参考}{\lien{ⓔsɯɣndzur}{sɯɣndzur}}\end{entrée}

\begin{entrée}{ndzri}{}{ⓔndzri} 
\classe{vt} \paradigme{dir}{tɤ-}\paradigme{dir}{pɯ-}\paradigme{dir}{\_}\paradigme{dir}{lɤ-}
\begin{définition}\pfra{tordre}\end{définition}
\begin{définition}\pcmn{拧}\end{définition}
\begin{définition}\pfra{être tordu, enroulé (fil, corde)}\end{définition}
\begin{définition}\pcmn{揪着(绳子)}\end{définition}
\begin{exemple}\pjya{aʑo tɯ-ŋga nɯ-χtɕi-t-a tɕe pɯ-ndzri-t-a tɕe tɯ-ci pɯ-tɕat-a}\hspace{5pt}\pcmn{我把衣服洗了,然后把水拧出来了}\end{exemple}
\begin{exemple}\pjya{tɯ-ŋga kɤ-χtɕi nɯ-jɤɣ tɕe tú-wɣ-ndzri ra, tɕe ɯ-ci a-pɯ-nɯ-ɬoʁ ra}\hspace{5pt}\pcmn{洗完了衣服就要拧,这样水就会出来}\end{exemple}
\begin{exemple}\pjya{a-ɕa ma-tɤ-tɯ-ndzri ma ɲɯ-mŋɤm}\hspace{5pt}\pcmn{你不要拧我的肉,很痛!}\end{exemple}
\begin{exemple}\pjya{rɟɯma kɤ-ndzri tɕe a-tɤ-ɤsɯɣ}\hspace{5pt}\pcmn{你拧一下螺丝就会紧}\end{exemple}
\begin{sous-entrée}{andzɯndzri}{ⓔndzriⓝandzɯndzri} 
\classe{vs} \end{sous-entrée}

\end{entrée}

\begin{entrée}{ndzrɯ}{}{ⓔndzrɯ} 
\classe{n} 
\begin{définition}\pfra{poinçon}\end{définition}
\begin{définition}\pcmn{凿子}\end{définition}
\begin{exemple}\pjya{ndzrɯ pɯ-lat-a}\hspace{5pt}\pcmn{我用了凿子}\end{exemple}\end{entrée}

\begin{entrée}{ndzɯ}{}{ⓔndzɯ} 
\classe{vt} \paradigme{dir}{tɤ-}\paradigme{dir}{tɤ-}
\begin{définition}\pfra{éduquer}\end{définition}
\begin{définition}\pcmn{教育}\end{définition}
\begin{définition}\pfra{se réconforter soi-même}\end{définition}
\begin{définition}\pcmn{宽慰自己}\end{définition}
\begin{exemple}\pjya{tɤ-pɤtso kɤ-ndzɯ ra}\hspace{5pt}\pcmn{一定要教育小孩子}\end{exemple}
\begin{exemple}\pjya{a-mu kɯ tɤ́-wɣ-ndzɯ-a}\hspace{5pt}\pcmn{我母亲教育了我}\end{exemple}
\begin{exemple}\pjya{tɯʑo kɤ-nɯ-ʑɣɤndzɯ a-pɯ-kɯ-cha ra}\hspace{5pt}\pcmn{自己要会宽慰自己}\end{exemple}\relationsémantique{参考}{\lien{ⓔndzɯmbra}{ndzɯmbra}}\relationsémantique{参考}{\lien{ⓔsindzɯ}{sindzɯ}}
\begin{sous-entrée}{tɯ-sɯm,sɯndzɯ}{ⓔndzɯⓝtɯ-sɯm,sɯndzɯ} 
\classe{vt} 
\begin{définition}\pfra{réconforter}\end{définition}
\begin{définition}\pcmn{调解心态}\end{définition}
\begin{exemple}\pjya{ɯʑo kɯ ɯ-sɯm ɲɯ-nɯ-sɯndzi}\hspace{5pt}\pcmn{他在调解自己的心态}\end{exemple}\end{sous-entrée}

\begin{sous-entrée}{ʑɣɤndzɯ}{ⓔndzɯⓝʑɣɤndzɯ} 
\classe{vi} \end{sous-entrée}

\end{entrée}

\begin{entrée}{ndzɯɣ}{}{ⓔndzɯɣ} 
\classe{vs} 
\begin{définition}\pfra{soigneux}\end{définition}
\begin{définition}\pcmn{谨慎,做事很有条理}\end{définition}
\begin{exemple}\pjya{nɤki tɯrme nɯ tɕhi tɤ-nɯ-mɯ-ma wuma ʑo kɯ-ndzɯɣ ci ŋu}\hspace{5pt}\pcmn{这个人无论做什么事都非常谨慎}\end{exemple}
\begin{sous-entrée}{kɯndzɯɣ}{ⓔndzɯɣⓝkɯndzɯɣ}
\begin{définition}\pfra{on dirait que}\end{définition}
\begin{définition}\pcmn{看起来,好像是}\end{définition}
\begin{exemple}\pjya{ɯʑo kɯ ɕoŋβzu ɲɯ-βze kɯndzɯɣ ŋu (=kɤti ŋu)}\hspace{5pt}\pcmn{他看起来是在做木工的}\end{exemple}\end{sous-entrée}

\end{entrée}

\begin{entrée}{ndzɯmbra}{}{ⓔndzɯmbra} 
\classe{vt} \paradigme{dir}{tɤ-}
\begin{définition}\pfra{éduquer}\end{définition}
\begin{définition}\pcmn{教育}\end{définition}
\begin{exemple}\pjya{a-mu kɯ tɤ́-wɣ-ndzɯmbra-a}\hspace{5pt}\pcmn{我母亲教育了我}\end{exemple}\relationsémantique{参考}{\lien{ⓔndzɯ}{ndzɯ}}\end{entrée}

\begin{entrée}{ndzɯŋɯ}{}{ⓔndzɯŋɯ} 
\classe{n} 
\begin{définition}\pfra{récipient en terre}\end{définition}
\begin{définition}\pcmn{泥巴捏成的罐子}\end{définition}\end{entrée}

\begin{entrée}{ndzɯpe}{}{ⓔndzɯpe} 
\classe{n} 
\begin{définition}\pfra{s'asseoir par terre avec les deux jambes l'une sur l'autre en travers (la manière dont les femmes doivent s'asseoir lorsqu'elles n'ont pas de travail à faire)}\end{définition}
\begin{définition}\pcmn{双腿斜在一边坐着(妇女坐的姿势)}\end{définition}
\begin{exemple}\pjya{ndzɯpe nɯ-βzu-t-a}\hspace{5pt}\pcmn{我坐了}\end{exemple}\relationsémantique{参考}{\lien{ⓔsɯndzɯpe}{sɯndzɯpe}}\end{entrée}

\begin{entrée}{ndzɯr}{}{ⓔndzɯr} 
\classe{vs} 
\begin{définition}\pfra{être au complet}\end{définition}
\begin{définition}\pcmn{齐全}\end{définition}
\begin{exemple}\pjya{tɯrme jo-ndzɯr-nɯ}\hspace{5pt}\pcmn{人们齐全了}\end{exemple}\relationsémantique{同义词}{\lien{ⓔtshoz}{tshoz}}
\begin{sous-entrée}{sɯɣndzɯr}{ⓔndzɯrⓝsɯɣndzɯr} 
\classe{vt}  
\grammaire{caus} 
\begin{définition}\pfra{préparer au complet}\end{définition}
\begin{définition}\pcmn{准备齐全}\end{définition}\relationsémantique{同义词}{\lien{ⓔsɯxtshoz}{sɯxtshoz}}\end{sous-entrée}

\end{entrée}

\begin{entrée}{ndzɯrnaʁ}{}{ⓔndzɯrnaʁ} 
\classe{n} 
\begin{définition}\pfra{guêpe}\end{définition}
\begin{définition}\pcmn{马蜂}\end{définition}
\begin{exemple}\pjya{ndzɯrnaʁ nɯ ɯʑo mɯ́j-wxti ri, ku-kɯ-mtsɯɣ tɕe wuma ɲɯ-ɤɣɯtɯɣ tɕe ʁʑɯnɯ kɯ tɤŋkhɯt tɤ-kɤ-lɤt fse tu-kɯ-ti ɲɯ-ŋu ma kɤ-kɯ-mtsɯɣ tɕe pjɯ-kɯ-tʂaβ ɲɯ-ŋgrɤl, laʁnɯ-sŋi ku-kɯ-ɕɯ-rŋgɯ ɲɯ-ngrɤl, tɕe ɯʑo sɯku ri ku-ndzoʁ tɕe ɯ-kho kɯ-wxtɯ-wxti tu-nɯ-βze ɲɯ-cha. ɯ-kho nɯ kɯ-ɤrtɯ-rtɯm ɲɯ-ŋu. ɯ-spa nɯ tɕhi ŋu ma ku-nɤmɯma tɕe kɯ-mnɯ-mnu kɯ-mpɯ-mpɯ ci ɲɯ-ŋu, ɯ-ŋgɯ tɯ-mɯ cho qale ku-ɕe mɯ́j-cha, ɯ-kɯ-spoʁ pa pjɯ-ru ɲɯ-ŋu tɕe ɯʑo ɯ-kɯm ɲɯ-ŋu. kú-wɣ-nɯrca tɕe ɲɯ-kɯ-mtsɯɣ ma nɯ maʁ nɤ mɯ́j-kɯ-mtsɯɣ.}\hspace{5pt}\pcmn{马蜂虽小,蜇人时毒性特别大,据说同被年轻人打了一拳一样难受,会晕倒,一两天不能起床。它栖息在树上,可以做很大的窝,不知是什么材料作成的,摸起来很柔和、很软,风雨透不进。洞口朝下,是蜂窝的门。有人骚扰时就会蜇,不然不会轻易蜇人。}\end{exemple}\end{entrée}

\begin{entrée}{ndʑu}{}{ⓔndʑu} 
\classe{n} 
\begin{définition}\pfra{baguettes}\end{définition}
\begin{définition}\pcmn{筷子}\end{définition}
\begin{exemple}\pjya{ndʑu kɯ tɤ-sɯ-mɟa-t-a}\hspace{5pt}\pcmn{我用筷子夹了}\end{exemple}\relationsémantique{参考}{\lien{ⓔandʑɯndʑu}{andʑɯndʑu}}\end{entrée}

\begin{entrée}{ndʑa}{}{ⓔndʑa} 
\classe{n} 
\begin{définition}\pfra{arc-en-ciel}\end{définition}
\begin{définition}\pcmn{彩虹}\end{définition}
\begin{exemple}\pjya{thaχtsa ɯ-rkɯ khatoʁ lu-kɯ-ɕe nɯ ɯ-ndʑa rmi}\hspace{5pt}\pcmn{花带边的彩色竖条叫\lien{}{ɯ-ndʑa}}\end{exemple}\étymologie{ndʑa}\end{entrée}

\begin{entrée}{ndʑaʁ}{}{ⓔndʑaʁ} 
\classe{vi} \paradigme{dir}{\_}\sens{1}
\begin{définition}\pfra{flotter, nager}\end{définition}
\begin{définition}\pcmn{漂浮;游泳}\end{définition}
\begin{exemple}\pjya{kɤ-ndʑaʁ ndɤre sɤɣmu tɕe aj mɤ-cha-a}\hspace{5pt}\pcmn{游泳很恐怖,我不敢}\end{exemple}\sens{2}
\begin{définition}\pfra{traverser la rivière à gué}\end{définition}
\begin{définition}\pcmn{涉过去}\end{définition}\end{entrée}

\begin{entrée}{ndʑɤβ}{}{ⓔndʑɤβ} 
\classe{vi}  
\grammaire{acaus} \paradigme{dir}{tɤ-}\paradigme{dir}{pɯ-}
\begin{définition}\pfra{brûler}\end{définition}
\begin{définition}\pcmn{燃烧}\end{définition}
\begin{exemple}\pjya{ɕoʁɕoʁ to-ndʑɤβ}\hspace{5pt}\pcmn{纸燃起来了}\end{exemple}
\begin{exemple}\pjya{xɕaj a-tɤ-ndʑɤβ tɕe pe ma ɯ-fsaqhe tɕe tɕe nɯ ɯ-stu nɯ kɯ-tshu tu-ɬoʁ ŋu, tɕe fsapaʁ ra rga-nɯ}\hspace{5pt}\pcmn{草烧了是一件好事,因为第二年那个地方就会长出茂盛的草,牲畜们喜欢}\end{exemple}
\begin{exemple}\pjya{tɯ-ŋga pjɤ-ndʑɤβ}\hspace{5pt}\pcmn{衣服烧了}\end{exemple}
\begin{exemple}\pjya{kha pjɤ-ndʑɤβ}\hspace{5pt}\pcmn{房子烧了}\end{exemple}\relationsémantique{参考}{\lien{ⓔtɕɤβ}{tɕɤβ}}\relationsémantique{参考}{\lien{ⓔɣndʑɤβ}{ɣndʑɤβ}}\end{entrée}

\begin{entrée}{ndʑɤm}{}{ⓔndʑɤm} 
\classe{vs} \paradigme{dir}{thɯ-}\sens{1}
\begin{définition}\pfra{chaud, tiède}\end{définition}
\begin{définition}\pcmn{温暖}\end{définition}
\begin{exemple}\pjya{kɯ-ndʑɤm tɤ-ndze}\hspace{5pt}\pcmn{趁热吃!}\end{exemple}\sens{2}
\begin{définition}\pfra{tendre (personne)}\end{définition}
\begin{définition}\pcmn{温柔}\end{définition}\relationsémantique{参考}{\lien{ⓔɣɤndʑɤm}{ɣɤndʑɤm}}\étymologie{ⁿdʑam}\end{entrée}

\begin{entrée}{ndʑɤrndʑɤr}{}{ⓔndʑɤrndʑɤr} 
\classe{idph.2} 
\begin{définition}\pfra{très fin}\end{définition}
\begin{définition}\pcmn{很细}\end{définition}
\begin{exemple}\pjya{kɯ-xtshɯ-xtshɯm ndʑɤrndʑɤr ɲɯ-ŋu}\hspace{5pt}\pcmn{非常细}\end{exemple}
\begin{exemple}\pjya{tɤ-ri kɯ-fse ci ndʑɤrndʑɤr pjɤ-ri}\end{exemple}\end{entrée}

\begin{entrée}{ndʑɤrtaʁ}{}{ⓔndʑɤrtaʁ} 
\classe{n} 
\begin{définition}\pfra{fourchette de bois}\end{définition}
\begin{définition}\pcmn{木叉子}\end{définition}\relationsémantique{参考}{\lien{ⓔtɤ-rtaʁ}{tɤ-rtaʁ}}\relationsémantique{参考}{\lien{ⓔndʑu}{ndʑu}}\end{entrée}

\begin{entrée}{ndʑɣaʁ}{}{ⓔndʑɣaʁ} 
\classe{vi}  
\grammaire{acaus} \paradigme{dir}{tɤ-}\paradigme{dir}{thɯ-}\paradigme{dir}{pɯ-}
\begin{définition}\pfra{s'extraire (spontanément)}\end{définition}
\begin{définition}\pcmn{(自动)挤出来}\end{définition}
\begin{exemple}\pjya{tɤ-ndʑɣaʁ}\hspace{5pt}\pcmn{(脓包)挤出来了}\end{exemple}
\begin{exemple}\pjya{ɯ-tɯcɯste chɤ-ndʑɣaʁ}\hspace{5pt}\pcmn{她羊水破了}\end{exemple}\relationsémantique{参考}{\lien{ⓔtɕɣaʁ}{tɕɣaʁ}}\end{entrée}

\begin{entrée}{ndʑɣɤrndʑɣɤr}{}{ⓔndʑɣɤrndʑɣɤr} 
\classe{idph.2} 
\begin{définition}\pfra{long (à propos des dents)}\end{définition}
\begin{définition}\pcmn{形容牙齿长}\end{définition}
\begin{exemple}\pjya{ɯ-ɕɣa ɯ-tɯ-zri kɯ ndʑɣɤrndʑɣɤr ʑo ɲɯ-pa}\hspace{5pt}\pcmn{他的牙齿很长}\end{exemple}\end{entrée}

\begin{entrée}{ndʑiŋgri}{}{ⓔndʑiŋgri} 
\classe{n} 
\begin{définition}\pfra{Sambucus sp.}\end{définition}
\begin{définition}\pcmn{血满草【臭草】}\end{définition}
\begin{exemple}\pjya{ndʑiŋgri ɯ-di wuma ʑo sɤjloʁ, sɯjno kɯ-wxti tsa ci ŋu, ɯ-jwaʁ kɯ-tɕɤr tɕe kɯ-rɲɟi tsa ŋu, ɯ-βzɯr nɯ ra ɯ-ɕɣa kɯ-fse kɯ-xtɕɯ-xtɕi tu, ɯ-mɯntoʁ ɯ-tshɯɣa nɯ san tsa fse, ɯ-mdoʁ kɯ-wɣrum ŋu, ɯ-mat thɯ-tɯt tɕe, kɯ-ɣɯrni ŋu, wuma ʑo qiaβ.}\hspace{5pt}\pcmn{臭草是一种发出臭味的大植物,叶子窄而长,叶边有小齿,花的形状像伞,花是白色的,果实成熟后是红色的,很苦。}\end{exemple}\end{entrée}

\begin{entrée}{ndʑiʑo}{}{ⓔndʑiʑo} 
\classe{pro} 
\begin{définition}\pfra{vous deux}\end{définition}
\begin{définition}\pcmn{你们俩}\end{définition}\end{entrée}

\begin{entrée}{ndʑrɯ}{}{ⓔndʑrɯ} 
\classe{n} 
\begin{définition}\pfra{lente}\end{définition}
\begin{définition}\pcmn{虮子}\end{définition}\relationsémantique{参考}{\lien{ⓔzrɯɣ}{zrɯɣ}}\end{entrée}

\begin{entrée}{ndʑrɯɕɤt}{}{ⓔndʑrɯɕɤt} 
\classe{n} 
\begin{définition}\pfra{peigne à lentes}\end{définition}
\begin{définition}\pcmn{捉虱子用的梳子}\end{définition}\end{entrée}

\begin{entrée}{ndʑɯ}{}{ⓔndʑɯ} 
\classe{vt} \paradigme{dir}{kɤ-}
\begin{définition}\pfra{accuser}\end{définition}
\begin{définition}\pcmn{告状;投诉}\end{définition}
\begin{exemple}\pjya{khɯtsa pɯ-tɯ-qrɯt, aʑo mɤ-ta-ndʑɯ}\hspace{5pt}\pcmn{你把碗打破了,我不会告你的状}\end{exemple}
\begin{sous-entrée}{sɤndʑɯ}{ⓔndʑɯⓝsɤndʑɯ} 
\classe{vi}  
\grammaire{apass} 
\begin{définition}\pfra{accuser (des gens)}\end{définition}
\begin{définition}\pcmn{告状}\end{définition}\end{sous-entrée}

\étymologie{ʑu}\end{entrée}

\begin{entrée}{ndʑɯɣ}{}{ⓔndʑɯɣ} 
\classe{vi} \paradigme{dir}{nɯ-}
\begin{définition}\pfra{être détruit}\end{définition}
\begin{définition}\pcmn{灭亡}\end{définition}
\begin{exemple}\pjya{skalpa ɲɤ-ndʑɯɣ}\hspace{5pt}\pcmn{世界灭亡了;时代变了}\end{exemple}\étymologie{ⁿdʑig}\end{entrée}

\begin{entrée}{ndʑɯnɯ}{}{ⓔndʑɯnɯ} 
\classe{n} 
\begin{définition}\pfra{Angelica sp.}\end{définition}
\begin{définition}\pcmn{当归}\end{définition}
\begin{exemple}\pjya{ndʑɯnɯ nɯ sɯjno ci ŋu. sɯŋgɯ tu-kɯ-ɬoʁ ci tu, lu-kɤ-nɯ-ji ci tu tɕe lonba ʑo kɯ-naχtɕɯɣ ɕti, ɯ-jwaʁ dɤn, ndɯβ, ɯ-χcɤl ɯ-ru tu-ɬoʁ tɕe, ɯ-jwaʁ cho aɣɯmdoʁ, ɯ-ru ɯ-kɤχcɤl tɕe, ɯ-mɯntoʁ ɲɯ-lɤt ŋu, ɯ-mɯntoʁ wɣrum, ɯ-mat dɤn. ɯ-di χɕɤβ, ɯ-qa nɯ kú-wɣ-sqa tɕe tú-wɣ-ndza tɕe smɤn kɯ-ʑru ci ɲɯ-ŋu. ɯ-jwaʁ kɤ-ndza sna.}\hspace{5pt}\pcmn{当归是一种草,有的长在森林了,也有自己种的,都完全一样。叶子多、细小,中间长茎,茎和叶子颜色相同。茎的顶端开花,花是白色的,果实多,香味浓。根煮来吃是一种名贵的药。叶子也可以吃。}\end{exemple}\end{entrée}

\begin{entrée}{ndʑɯrpɯt}{}{ⓔndʑɯrpɯt} 
\classe{vs} \paradigme{dir}{thɯ-}\paradigme{dir}{tɤ-}
\begin{définition}\pfra{être engourdi}\end{définition}
\begin{définition}\pcmn{麻木}\end{définition}
\begin{exemple}\pjya{a-mi ɲɯ-ndʑɯrpɯt}\hspace{5pt}\pcmn{我的脚是麻木的}\end{exemple}\relationsémantique{参考}{\lien{ⓔɣɤzoŋzoŋ}{ɣɤzoŋzoŋ}}\end{entrée}

\begin{entrée}{ndʑɯrwɯz}{}{ⓔndʑɯrwɯz} 
\classe{n} 
\begin{définition}\pfra{Sonchus brachyotus}\end{définition}
\begin{définition}\pcmn{苣荬菜【苦苦草】}\end{définition}
\begin{exemple}\pjya{ndʑɯrwɯz nɯ sɯjno ci ŋu, ɯ-jwaʁ nɯ kɯ-tɕɤr kɯ-rɲɟi tsa ŋu, kɯ-ɤɣrɤɣrum tsa ŋu, ɯ-spjɯŋ tu-ɬoʁ tɕe, kɯ-zri tsa tɯ-ɟom jarma tu-βze cha. ɯ-ru ɯ-taʁ ɯ-jwaʁ tu-oʑɯrja ŋu. ɯ-kɤχcɤl tɕe ɲɯ-rɯmɯntoʁ ŋu. ɯ-mɯntoʁ kɯ-qarŋe ŋu. ɯ-jwaʁ ɯ-βzɯr nɯ ra ɯ-mdzu kɯ-fse kɯ-xtɕɯ-xtɕi tu. pɯ́-wɣ-qlɯt tɕe, ɯ-lu tu, qiaβ. fsapaʁ ndza ma mɤ-sna.}\hspace{5pt}\pcmn{苦苦草是一种植物,叶子窄而长,带有白色,茎长出来时,可以长到一米来高。叶子排列在茎上。茎的顶上开花。花是黄色的。叶子边缘有小刺。折断时有乳汁,很苦,只能喂牲畜。}\end{exemple}\end{entrée}

\begin{entrée}{ndʐa}{}{ⓔndʐa} 
\classe{n} 
\begin{définition}\pfra{cause}\end{définition}
\begin{définition}\pcmn{原因}\end{définition}
\begin{exemple}\pjya{aʑɯɣ ndʐa ŋu ɕi, nɤʑɯɣ ndʐa ŋu, aj mɯ́j-tso-a}\hspace{5pt}\pcmn{不知道是你的原因,还是我的原因}\end{exemple}\end{entrée}

\begin{entrée}{ndʐaβ}{}{ⓔndʐaβ} 
\classe{vi}  
\grammaire{acaus} \paradigme{dir}{pɯ-}\paradigme{dir}{thɯ-}\paradigme{dir}{\_}\sens{1}
\begin{définition}\pfra{tomber à la renverse}\end{définition}
\begin{définition}\pcmn{摔倒}\end{définition}
\begin{exemple}\pjya{ɯʑo pjɤ-ndʐaβ}\hspace{5pt}\pcmn{他摔倒了}\end{exemple}
\begin{exemple}\pjya{pɯ-ndʐaβ-a}\hspace{5pt}\pcmn{我摔倒了}\end{exemple}\sens{2}
\begin{définition}\pfra{dégringoler, rouler}\end{définition}
\begin{définition}\pcmn{滚下}\end{définition}
\begin{exemple}\pjya{tʂu mɯ́j-sɤɣa tɕe pjɤ-ndʐaβ}\hspace{5pt}\pcmn{路不安全,他滚下去了}\end{exemple}
\begin{exemple}\pjya{nɯŋa alo pjɤ-ndʐaβ}\hspace{5pt}\pcmn{奶牛在那里滚下去了}\end{exemple}\sens{3}
\begin{définition}\pfra{s'effondrer (arbre, mur)}\end{définition}
\begin{définition}\pcmn{倒塌}\end{définition}
\begin{exemple}\pjya{znde cho-ndʐaβ}\hspace{5pt}\pcmn{墙倒下了}\end{exemple}\relationsémantique{参考}{\lien{ⓔtʂaβ}{tʂaβ}}\relationsémantique{参考}{\lien{ⓔsɤndʐaβ}{sɤndʐaβ}}\end{entrée}

\begin{entrée}{ndʐi}{}{ⓔndʐi} 
\classe{vs}  
\grammaire{acaus} \paradigme{dir}{pɯ-}
\begin{définition}\pfra{fondre}\end{définition}
\begin{définition}\pcmn{融化}\end{définition}
\begin{exemple}\pjya{tɯ-ci ɯ-rkɯ tɤjpɣom ra pjɤ-ndʐi}\hspace{5pt}\pcmn{河边的冰融掉了}\end{exemple}
\begin{exemple}\pjya{zgoku tɤjpa ra pjɤ-ndʐi}\hspace{5pt}\pcmn{山上的雪融掉了}\end{exemple}
\begin{exemple}\pjya{ta-mar pɯ-ndʐi}\hspace{5pt}\pcmn{酥油融了}\end{exemple}
\begin{sous-entrée}{sɯɣndʐi}{ⓔndʐiⓝsɯɣndʐi} 
\classe{vt} 
\begin{définition}\pfra{faire fondre}\end{définition}
\begin{définition}\pcmn{使…融化}\end{définition}\relationsémantique{参考}{\lien{ⓔftʂi}{ftʂi}}\end{sous-entrée}

\end{entrée}

\begin{entrée}{ndʐoʁ}{}{ⓔndʐoʁ} 
\classe{vi} \paradigme{dir}{nɯ-}
\begin{définition}\pfra{être effrayé (animal, surtout cheval)}\end{définition}
\begin{définition}\pcmn{受惊(动物,特别是马)到处乱跑乱跳}\end{définition}\paradigme{dir}{nɯ-}
\begin{définition}\pfra{effrayer}\end{définition}
\begin{définition}\pcmn{令(动物)受惊}\end{définition}
\begin{exemple}\pjya{mbro ɲɤ-ndʐoʁ}\hspace{5pt}\pcmn{马受惊了}\end{exemple}
\begin{exemple}\pjya{khɯna kɯ tshɤt na-sɯɣndʐoʁ}\hspace{5pt}\pcmn{狗令山羊受惊了}\end{exemple}
\begin{sous-entrée}{sɯɣndʐoʁ}{ⓔndʐoʁⓝsɯɣndʐoʁ} 
\classe{vt} \end{sous-entrée}

\étymologie{ⁿdrog}\end{entrée}

\begin{entrée}{ndʐɯɣlɤm}{}{ⓔndʐɯɣlɤm} 
\classe{n} 
\begin{définition}\pfra{règle}\end{définition}
\begin{définition}\pcmn{规矩,政策}\end{définition}\end{entrée}

\begin{entrée}{ndʐɯm}{}{ⓔndʐɯm} 
\classe{vs} \paradigme{dir}{tɤ-}\sens{1}
\begin{définition}\pfra{rapide}\end{définition}
\begin{définition}\pcmn{快}\end{définition}\sens{2}
\begin{définition}\pfra{courant (parole)}\end{définition}
\begin{définition}\pcmn{流利(语言)}\end{définition}
\begin{exemple}\pjya{@piqiu ɲɯ-ndʐɯm}\hspace{5pt}\pcmn{皮球转动得很快}\end{exemple}
\begin{exemple}\pjya{mkhɯrlu ɲɯ-ndʐɯm}\hspace{5pt}\pcmn{轮子转动得很快}\end{exemple}
\begin{exemple}\pjya{qale ɲɯ-ndʐɯm}\hspace{5pt}\pcmn{风不停地吹}\end{exemple}
\begin{exemple}\pjya{ɯ-tɯ-ɤre ɲɯ-ndʐɯm}\hspace{5pt}\pcmn{他笑个不停}\end{exemple}
\begin{exemple}\pjya{ɯ-ɕmi ɲɯ-ndʐɯm}\hspace{5pt}\pcmn{他滔滔不绝地讲}\end{exemple}
\begin{exemple}\pjya{kɤ-ti a-kɤ-ndʐɯm ɲɯ-sɯsam-a}\hspace{5pt}\pcmn{我希望能讲得流利一点}\end{exemple}
\begin{sous-entrée}{sɯɣndʐɯm}{ⓔndʐɯmⓢ2ⓝsɯɣndʐɯm} 
\classe{vt}  
\grammaire{caus} 
\begin{définition}\pfra{faire des exercices}\end{définition}
\begin{définition}\pcmn{锻炼}\end{définition}
\begin{exemple}\pjya{a-phoŋbu ɲɯ-sɯɣndʐɯm-a ra ɲɯ-sɯsam-a}\hspace{5pt}\pcmn{我觉得自己要锻炼身体}\end{exemple}
\begin{exemple}\pjya{kɯrɯ skɤt kɤ-ti tu-sɯɣndʐɯm-a ra ɲɯ-sɯsam-a}\hspace{5pt}\pcmn{我想讲藏语讲的流利一点}\end{exemple}\end{sous-entrée}

\end{entrée}

\begin{entrée}{ndʐɯnbu}{}{ⓔndʐɯnbu} 
\classe{n} 
\begin{définition}\pfra{hôte}\end{définition}
\begin{définition}\pcmn{远方的客人}\end{définition}
\begin{exemple}\pjya{ndʐɯnbu ɕe-a}\hspace{5pt}\pcmn{我要出差}\end{exemple}\relationsémantique{参考}{\lien{ⓔnɯndʐɯnbu}{nɯndʐɯnbu}}\étymologie{mgron.po}\end{entrée}

\begin{entrée}{ndʐɯnɬa}{}{ⓔndʐɯnɬa} 
\classe{n} 
\begin{définition}\pfra{cérémonie effectuée lorsqu'un membre de la famille part en voyage}\end{définition}
\begin{définition}\pcmn{家里有人出行的时候,为了保佑他安全顺利而念的经}\end{définition}\relationsémantique{同义词}{\lien{ⓔmdɯnri}{mdɯnri}}\end{entrée}

\begin{entrée}{ndʐɯnphɤrscoʁ}{}{ⓔndʐɯnphɤrscoʁ} 
\classe{n} 
\begin{définition}\pfra{louche en cuivre}\end{définition}
\begin{définition}\pcmn{红铜勺子}\end{définition}
\begin{exemple}\pjya{ndʐɯnphɤrscoʁ nɯ scoʁ ɯ-jɯ kɯ-zri tsa ci ŋu, ɯ-jɯ nɯ ɕom ŋu, ɯ-pɤl nɯ ɯ-pa nɯ ra zan ŋu, ɯ-taʁ tɕe raʁ ku-ɕe ŋu, ɯ-mŋu nɯ li zaŋ ŋu, tɕe kɯɕɯŋgɯ tʂha ɯ-z-nɤrkɯku pjɤ-ŋu, tɯ-jno sɤ-rku mɤ-sna ma zaŋ cho raʁ ni ʁnaʁna ʑo kɯ-nɤmar nɯ-atɕaʁ tɕe ʁja ɲɯ-ɬoʁ ŋu tɕe, ʁja nɯ kɯ tɯrme tu-kɯ-ɕɯngo ŋgrɤl.}\hspace{5pt}\pcmn{红铜勺子是把比较长的勺子,把是铁作成的,勺子头外层部分是红铜,里层是黄铜,勺沿也是红铜。过去是专门用来舀茶的器具,不能用来舀菜,因为黄铜和红铜沾上油会生锈,锈会导致人生病。}\end{exemple}\end{entrée}

\begin{entrée}{ndʐɯz}{}{ⓔndʐɯz} 
\classe{vs} \sens{1}
\begin{définition}\pfra{bien s'entendre}\end{définition}
\begin{définition}\pcmn{谈得来}\end{définition}
\begin{exemple}\pjya{tɕiʑo tɕi-khɤcɤl ndʐɯz ŋgrɤl}\hspace{5pt}\pcmn{我们俩很谈得来}\end{exemple}\sens{2}
\begin{définition}\pfra{bien suivre le rythme (danse)}\end{définition}
\begin{définition}\pcmn{跟着节奏跳(舞)}\end{définition}
\begin{exemple}\pjya{ɯʑo kɤ-sɤmtshi ɲɯ-mkhɤz tɕe tɯrɟaʁ ɲɯ-ndʐɯz}\hspace{5pt}\pcmn{他很会领舞,所以舞跳得很整齐}\end{exemple}\étymologie{ⁿdris}\end{entrée}

\begin{entrée}{ndʐuwa}{}{ⓔndʐuwa} 
\classe{n} 
\begin{définition}\pfra{hôte}\end{définition}
\begin{définition}\pcmn{客人}\end{définition}
\begin{exemple}\pjya{jisŋi a-ndʐuwa ɣɤʑu wo}\hspace{5pt}\pcmn{我今天有我客人}\end{exemple}\end{entrée}

\begin{entrée}{nétɕi}{}{ⓔnétɕi} 
\classe{part} 
\begin{définition}\pfra{non ?}\end{définition}
\begin{définition}\pcmn{吧}\end{définition}
\begin{exemple}\pjya{nɤ-smɤn mɯ-ko-tɯ-tshi-t nétɕi}\hspace{5pt}\pcmn{你没有喝药吧?}\end{exemple}
\begin{exemple}\pjya{tɤ-mbɣom tu-ti-a ŋu nétɕi}\hspace{5pt}\pcmn{我不是叫你快点吗?}\end{exemple}
\begin{exemple}\pjya{japa kɯnɤ nɤ-ɴɢar ɯ-ŋgɯ tɤ-se pɯ-tu nétɕi}\hspace{5pt}\pcmn{去年你的痰也有痰对吧}\end{exemple}\end{entrée}

\begin{entrée}{ngu}{}{ⓔngu} 
\classe{vt} \sens{1}\paradigme{dir}{pɯ-}
\begin{définition}\pfra{nourrir, donner à manger (aux animaux)}\end{définition}
\begin{définition}\pcmn{喂(动物)}\end{définition}
\begin{exemple}\pjya{paʁ pɯ-ngu-t-a}\hspace{5pt}\pcmn{我喂了猪}\end{exemple}
\begin{exemple}\pjya{fsapaʁ ra pɯ-ngu-t-a}\hspace{5pt}\pcmn{我喂了牲畜}\end{exemple}\sens{2}\paradigme{dir}{nɯ-}
\begin{définition}\pfra{donner la bouchée}\end{définition}
\begin{définition}\pcmn{亲口喂}\end{définition}
\begin{exemple}\pjya{pɣɤmu kɯ ɯ-pɯ na-ngu}\hspace{5pt}\pcmn{母鸡喂了小鸡}\end{exemple}
\begin{exemple}\pjya{ɯ-pɯ ɲɯ-nge ɲɯ-ŋu}\hspace{5pt}\pcmn{(母鸡)在喂小鸡}\end{exemple}\relationsémantique{同义词}{\lien{ⓔχsu}{χsu}}\relationsémantique{同义词}{\lien{ⓔɕpɯt}{ɕpɯt}}\end{entrée}

\begin{entrée}{ngɤjtshi}{}{ⓔngɤjtshi} 
\classe{vt}  
\grammaire{comp} \paradigme{dir}{nɯ-}
\begin{définition}\pfra{nourrir}\end{définition}
\begin{définition}\pcmn{喂(给别人吃喝)}\end{définition}
\begin{exemple}\pjya{tɤwɯ nɯ kɯ ɲó-wɣ-ngɤjtshi}\hspace{5pt}\pcmn{老年人喂了他们}\end{exemple}\relationsémantique{参考}{\lien{ⓔngu}{ngu}}\relationsémantique{参考}{\lien{ⓔjtshi}{jtshi}}\relationsémantique{参考}{\lien{ⓔmbijtshi}{mbijtshi}}\end{entrée}

\begin{entrée}{nge}{}{ⓔnge} 
\classe{vs} 
\begin{définition}\pfra{très solide}\end{définition}
\begin{définition}\pcmn{非常结实}\end{définition}
\begin{exemple}\pjya{laχtɕha tɤ́-wɣ-χtɯ tɕe, kɯ-ngɯ-nge tsa tú-wɣ-nɯ-qɤr ra, tɕe tɤ́-wɣ-ntɕhoz khro mdɯ}\hspace{5pt}\pcmn{买东西的的时候,要挑选结实的,这样就可以用得长久}\end{exemple}\relationsémantique{参考}{\lien{ⓔngɯnge}{ngɯnge}}\end{entrée}

\begin{entrée}{ngo}{}{ⓔngo} 
\classe{vi} \paradigme{dir}{tɤ-}
\begin{définition}\pfra{tomber malade}\end{définition}
\begin{définition}\pcmn{生病}\end{définition}
\begin{exemple}\pjya{ɲɯ-nɯtɕhomba-a tɕe tɤ-ngo-a}\hspace{5pt}\pcmn{我感冒了,生病了}\end{exemple}
\begin{exemple}\pjya{kɤ-ngo ɲɯ-nɯɣme-a}\hspace{5pt}\pcmn{我很怕生病}\end{exemple}\relationsémantique{参考}{\lien{ⓔɣɤtsɤngo}{ɣɤtsɤngo}}\relationsémantique{参考}{\lien{ⓔɕɯngo}{ɕɯngo}}\relationsémantique{参考}{\lien{ⓔtɯ-ŋgo}{tɯ-ŋgo}}\end{entrée}

\begin{entrée}{ngrɯβ}{}{ⓔngrɯβ}\paradigme{dir}{pɯ-}
\begin{définition}\pfra{accomplir}\end{définition}
\begin{définition}\pcmn{成功;完成}\end{définition}
\begin{exemple}\pjya{nɤ-kɤ-nɯsmɯlɤm nɯ a-pɯ-ngrɯβ}\hspace{5pt}\pcmn{希望你祈求的事情会成功!}\end{exemple}
\begin{sous-entrée}{sɯngrɯβ}{ⓔngrɯβⓝsɯngrɯβ} 
\classe{vt} 
\begin{définition}\pfra{réaliser}\end{définition}
\begin{définition}\pcmn{使……成功}\end{définition}\end{sous-entrée}

\étymologie{ⁿgrub}\end{entrée}

\begin{entrée}{ngɯnge}{}{ⓔngɯnge} 
\classe{vs} 
\begin{définition}\pfra{résistant}\end{définition}
\begin{définition}\pcmn{结实}\end{définition}\relationsémantique{参考}{\lien{ⓔngɯt}{ngɯt}}\relationsémantique{参考}{\lien{ⓔnge}{nge}}\end{entrée}

\begin{entrée}{ngɯt}{}{ⓔngɯt} 
\classe{vs} \paradigme{dir}{tɤ-}\paradigme{dir}{tɤ-}
\begin{définition}\pfra{solide}\end{définition}
\begin{définition}\pcmn{结实}\end{définition}
\begin{définition}\pfra{rendre solide}\end{définition}
\begin{définition}\pcmn{加固}\end{définition}
\begin{exemple}\pjya{kɯki ɲɯ-ngɯt}\hspace{5pt}\pcmn{这个东西很结实}\end{exemple}
\begin{exemple}\pjya{tɯmbri ɲɯ-ngɯt}\hspace{5pt}\pcmn{绳子很结实}\end{exemple}
\begin{exemple}\pjya{tɯ-ŋga ɲɯ-ngɯt}\hspace{5pt}\pcmn{衣服很结实}\end{exemple}\relationsémantique{参考}{\lien{ⓔngɯnge}{ngɯnge}}\relationsémantique{参考}{\lien{ⓔnge}{nge}}
\begin{sous-entrée}{ɣɤngɯt}{ⓔngɯtⓝɣɤngɯt} 
\classe{vt} \end{sous-entrée}

\begin{sous-entrée}{zɣɤngɯt}{ⓔngɯtⓝzɣɤngɯt} 
\classe{vt} 
\begin{définition}\pfra{rendre solide avec ...}\end{définition}
\begin{définition}\pcmn{用……加固}\end{définition}
\begin{exemple}\pjya{tɯ-ŋga ɯ-tɯ-tʂɯβ kɤntɕhɯ kɤ-lat-a tɕe kɤ-zɣɤngɯt-a}\hspace{5pt}\pcmn{我把衣服多缝了几道,使它更结实了}\end{exemple}\end{sous-entrée}

\end{entrée}

\begin{entrée}{ni}{}{ⓔni} 
\classe{det} 
\begin{définition}\pfra{duel}\end{définition}
\begin{définition}\pcmn{双数}\end{définition}\end{entrée}

\begin{entrée}{nmu}{}{ⓔnmu} 
\classe{vi} \paradigme{dir}{nɯ-}
\begin{définition}\pfra{trembler (tremblement de terre)}\end{définition}
\begin{définition}\pcmn{震动(地震)}\end{définition}
\begin{exemple}\pjya{waɟɯ ɲɤ-nmu}\hspace{5pt}\pcmn{发生了地震}\end{exemple}\relationsémantique{参考}{\lien{ⓔmɯnmu}{mɯnmu}}
\begin{sous-entrée}{ɣɤnmu}{ⓔnmuⓝɣɤnmu} 
\classe{vs}  
\grammaire{facil} 
\begin{définition}\pfra{se produire souvent (tremblement de terre)}\end{définition}
\begin{définition}\pcmn{容易有地震}\end{définition}
\begin{exemple}\pjya{iɕqha sɤtɕha nɯ waɟɯ ɲɯ-ɣɤnmu}\hspace{5pt}\pcmn{这个地方容易有地震}\end{exemple}\end{sous-entrée}

\end{entrée}

\begin{entrée}{núndʐa}{}{ⓔnúndʐa} 
\classe{adv} 
\begin{définition}\pfra{pour cette raison}\end{définition}
\begin{définition}\pcmn{因此}\end{définition}\end{entrée}

\begin{entrée}{nŋo}{}{ⓔnŋo} 
\classe{vi} \paradigme{dir}{pɯ-}
\begin{définition}\pfra{échouer, perdre}\end{définition}
\begin{définition}\pcmn{败;输}\end{définition}
\begin{exemple}\pjya{nɤʑo pɯ-tɯ-nŋo}\hspace{5pt}\pcmn{你输了}\end{exemple}
\begin{exemple}\pjya{nɤ-ɕki pɯ-nŋo-a}\hspace{5pt}\pcmn{我输给你了}\end{exemple}\relationsémantique{参考}{\lien{ⓔɕɯnŋo}{ɕɯnŋo}}\end{entrée}

\begin{entrée}{no}{}{ⓔno} 
\classe{vt} \sens{1}\paradigme{dir}{\_}
\begin{définition}\pfra{mener (animaux), chasser}\end{définition}
\begin{définition}\pcmn{赶;驱逐}\end{définition}
\begin{définition}\pfra{clouer}\end{définition}
\begin{définition}\pcmn{钉(钉子 )}\end{définition}
\begin{exemple}\pjya{fsapaʁ kɤ-nɤm}\hspace{5pt}\pcmn{你赶牲畜吧}\end{exemple}\sens{2}
\begin{exemple}\pjya{ɕɤmtshoʁ kɤ-no-t-a}\hspace{5pt}\pcmn{我钉了钉子}\end{exemple}
\begin{exemple}\pjya{ɕɤmtshoʁ kɤ-nɤm}\hspace{5pt}\pcmn{你钉钉子吧}\end{exemple}
\begin{exemple}\pjya{tɤtshoʁ pɯ-nɤm}\hspace{5pt}\pcmn{你钉木钉吧}\end{exemple}\relationsémantique{参考}{\lien{ⓔnɯno}{nɯno}}\end{entrée}

\begin{entrée}{noŋstɤn}{}{ⓔnoŋstɤn} 
\classe{n} 
\begin{définition}\pfra{tapis de selle}\end{définition}
\begin{définition}\pcmn{鞍垫}\end{définition}\étymologie{naŋ.stan}\end{entrée}

\begin{entrée}{nor}{}{ⓔnor} 
\classe{vi} \paradigme{dir}{nɯ-}
\begin{définition}\pfra{se tromper}\end{définition}
\begin{définition}\pcmn{弄错(指不严重的错误)}\end{définition}
\begin{exemple}\pjya{ɲɤ-nor}\hspace{5pt}\pcmn{他弄错了}\end{exemple}\relationsémantique{同义词}{\lien{ⓔnɯkɯmaʁ}{nɯkɯmaʁ}}\étymologie{nor}\end{entrée}

\begin{entrée}{nóʁmɯz}{}{ⓔnóʁmɯz} 
\classe{adv} 
\begin{définition}\pfra{alors seulement}\end{définition}
\begin{définition}\pcmn{那才}\end{définition}\relationsémantique{参考}{\lien{ⓔkóʁmɯz}{kóʁmɯz}}\end{entrée}

\begin{entrée}{ntaβ}{}{ⓔntaβ} 
\classe{vs} \paradigme{dir}{kɤ-}\paradigme{dir}{nɯ-}
\begin{définition}\pfra{stable}\end{définition}
\begin{définition}\pcmn{稳当}\end{définition}
\begin{définition}\pfra{laisser là}\end{définition}
\begin{définition}\pcmn{放在那里}\end{définition}
\begin{exemple}\pjya{ɯ-mdzɯ ko-ntaβ ma kɯmaʁ rɯsɯso mɯ-ɲɤ-ra}\hspace{5pt}\pcmn{他安心了,不需要再考虑其它事情}\end{exemple}
\begin{exemple}\pjya{ɯ-sɯm ko-ntaβ}\hspace{5pt}\pcmn{他放心了}\end{exemple}
\begin{exemple}\pjya{ɯ-ʑɯβ ko-ntaβ}\hspace{5pt}\pcmn{他睡得很熟}\end{exemple}
\begin{exemple}\pjya{kɤ-ndza mɯ-mɤ-ɲɯ-tɯ-ɕkɯt nɤ, nɯ-ɕɯ-ntaβ jɤɣ}\hspace{5pt}\pcmn{你如果吃不完的话,可以放在那里}\end{exemple}
\begin{sous-entrée}{ɕɯntaβ}{ⓔntaβⓝɕɯntaβ} 
\classe{vt}  
\grammaire{caus} \end{sous-entrée}

\begin{sous-entrée}{ʑɣɤɕɯntaβ}{ⓔntaβⓝʑɣɤɕɯntaβ} 
\classe{vi}  
\grammaire{refl}
\grammaire{caus} 
\begin{définition}\pfra{rester sans rien faire}\end{définition}
\begin{définition}\pcmn{该做的不想做;不想动}\end{définition}
\begin{exemple}\pjya{ɲɯ-ʑɣɤ-ɕɯntaβ}\hspace{5pt}\pcmn{他安静下来了}\end{exemple}\relationsémantique{参考}{\lien{ⓔɣɤntaβ}{ɣɤntaβ}}\end{sous-entrée}

\end{entrée}

\begin{entrée}{ntɕha}{}{ⓔntɕha} 
\classe{vt} \sens{1}\paradigme{dir}{pɯ-}
\begin{définition}\pfra{tuer (animal)}\end{définition}
\begin{définition}\pcmn{宰(动物)}\end{définition}\sens{2}\paradigme{dir}{nɯ-}\paradigme{dir}{pɯ-}
\begin{définition}\pfra{découper en morceaux (animal)}\end{définition}
\begin{définition}\pcmn{剥皮;切分(动物)}\end{définition}
\begin{définition}\pfra{tuer des animaux}\end{définition}
\begin{définition}\pcmn{屠宰}\end{définition}
\begin{exemple}\pjya{tshɤt pjɤ-si tɕe nɯ-ntɕhe}\hspace{5pt}\pcmn{山羊死了,切分了它}\end{exemple}
\begin{exemple}\pjya{ji-tshɤt pjɤ-si tɕe na-ntɕha}\hspace{5pt}\pcmn{我们的山羊死了,他就把它切分了}\end{exemple}
\begin{sous-entrée}{rɤntɕha}{ⓔntɕhaⓢ2ⓝrɤntɕha} 
\classe{vi} \end{sous-entrée}

\end{entrée}

\begin{entrée}{ntɕhɤr}{}{ⓔntɕhɤr} 
\classe{vi} \paradigme{dir}{kɤ-}
\begin{définition}\pfra{éclairer}\end{définition}
\begin{définition}\pcmn{映照}\end{définition}
\begin{exemple}\pjya{@dianying ko-ntɕhɤr}\hspace{5pt}\pcmn{电影上映了}\end{exemple}
\begin{exemple}\pjya{tɤŋe ɯ-ɣot ko-ntɕhɤr}\hspace{5pt}\pcmn{太阳光照下来了}\end{exemple}
\begin{exemple}\pjya{χɕɤlzgoŋ ɯ-ŋgɯ ko-ntɕhar-a}\hspace{5pt}\pcmn{我在镜子里}\end{exemple}
\begin{exemple}\pjya{ɯ-jmŋo ɯ-ŋgɯ sɯŋgi ko-ntɕhɤr}\hspace{5pt}\pcmn{他梦见了狮子}\end{exemple}
\begin{exemple}\pjya{χɕɤlzgoŋ ɯ-ŋgɯ kɤ-ntɕhɤr-tɕi nɯ ra pɯ-mto-t-a}\hspace{5pt}\pcmn{我在镜子里看见了我们俩的照影}\end{exemple}
\begin{exemple}\pjya{jɯfɕɯɕɤr a-ʑɯβ ɯ-mɤ-tɯ-ɣi kɯ ʑa ɯ-mɤ-fsoʁ ma kɤ-sɯso ʑo ɲɯ-ntɕhɤr}\hspace{5pt}\pcmn{我昨天晚上睡不着,在我想象中,多么希望早点天亮}\end{exemple}\étymologie{ⁿtɕʰar}\end{entrée}

\begin{entrée}{ntɕhɣaʁ}{}{ⓔntɕhɣaʁ} 
\classe{vi} \paradigme{dir}{kɤ-}\paradigme{dir}{tɤ-}\paradigme{dir}{kɤ-}
\begin{définition}\pfra{éclabousser}\end{définition}
\begin{définition}\pcmn{溅起来}\end{définition}
\begin{définition}\pfra{éclabousser}\end{définition}
\begin{définition}\pcmn{使溅起来}\end{définition}
\begin{exemple}\pjya{tɯ-ci kɤ-ntɕhɣaʁ}\hspace{5pt}\pcmn{水溅起来了}\end{exemple}
\begin{exemple}\pjya{tɯ-ci a-taʁ tɤ-ntɕhɣaʁ}\hspace{5pt}\pcmn{水溅到我身上}\end{exemple}
\begin{exemple}\pjya{tɤrcoʁ kɤ-ntɕhɣaʁ}\hspace{5pt}\pcmn{稀泥溅起来了}\end{exemple}
\begin{exemple}\pjya{tɤ-rɯndzaŋspa tsa, tɯ-ci nɯ ma-kɤ-tɯ-sɯntɕhɣaʁ}\hspace{5pt}\pcmn{小心一点,不要让水溅起来}\end{exemple}
\begin{sous-entrée}{sɯntɕhɣaʁ}{ⓔntɕhɣaʁⓝsɯntɕhɣaʁ} 
\classe{vt} \end{sous-entrée}

\end{entrée}

\begin{entrée}{ntɕhomŋga}{}{ⓔntɕhomŋga} 
\classe{n} 
\begin{définition}\pfra{habits de danse}\end{définition}
\begin{définition}\pcmn{跳神时穿的服装}\end{définition}\end{entrée}

\begin{entrée}{ntɕhoz}{}{ⓔntɕhoz} 
\classe{vt} \paradigme{dir}{tɤ-}\paradigme{dir}{kɤ-}
\begin{définition}\pfra{utiliser}\end{définition}
\begin{définition}\pcmn{使用}\end{définition}
\begin{exemple}\pjya{khɯtsa kɤ-ntɕhoz-a}\hspace{5pt}\pcmn{我用了碗}\end{exemple}
\begin{exemple}\pjya{nɤʑo kɤ-ntɕhoz ɯ-tɯ-spe?}\hspace{5pt}\pcmn{你会不会用?}\end{exemple}
\begin{exemple}\pjya{``ɯɟɤm" nɯ tɯ-rju ɯ-ŋgɯ tɕe kɤ-ntɕhoz tɕhi tú-wɣ-stu ŋu"}\hspace{5pt}\pcmn{怎么在句子中用\lien{}{ɯɟɤm}这个词}\end{exemple}
\begin{exemple}\pjya{nɯnɯ ŋotɕu tɤ-tɯ-nɯ-tɕhɯ-ntɕhoz khɯ}\hspace{5pt}\pcmn{(这一句话)你在什么环境都可以用}\end{exemple}
\begin{exemple}\pjya{kɤ-ntɕhoz tɕhi ɕɯ-ste ɲɯ-ŋu?}\hspace{5pt}\pcmn{这有什么用呢?}\end{exemple}
\begin{sous-entrée}{nɯɣɯntɕhoz}{ⓔntɕhozⓝnɯɣɯntɕhoz} 
\classe{vs}  
\grammaire{facil} 
\begin{définition}\pfra{facile à utiliser}\end{définition}
\begin{définition}\pcmn{容易用}\end{définition}
\begin{exemple}\pjya{ki qaʁ ki wuma ɲɯ-nɯɣɯntɕhoz}\hspace{5pt}\pcmn{这个锄头很好用}\end{exemple}\end{sous-entrée}

\end{entrée}

\begin{entrée}{ntɕhɯɣ}{}{ⓔntɕhɯɣ} 
\classe{vi} \paradigme{dir}{tɤ-}
\begin{définition}\pfra{s'abîmer}\end{définition}
\begin{définition}\pcmn{损坏}\end{définition}
\begin{exemple}\pjya{a-ɕɣa kɤ-ɴɢrɯ nɯ mɯ-pɯ-ɴɢrɯ ri, ɯ-qa to-ntɕhɯɣ tɕe ɲɯ-mŋɤm}\hspace{5pt}\pcmn{我的牙齿没有裂,但是牙龈损坏了,很痛}\end{exemple}
\begin{exemple}\pjya{mkhɯrlu to-ntɕhɯɣ tɕe kɤ-lɤt mɯ́jsna}\hspace{5pt}\pcmn{车损坏了,不能再开了}\end{exemple}
\begin{exemple}\pjya{nɤ-wa nɯ mɯ́j-tɯ-ntɕhɯɣ}\hspace{5pt}\pcmn{你不亏是你父亲的儿子}\end{exemple}\end{entrée}

\begin{entrée}{nthar}{}{ⓔnthar} 
\classe{vt} \paradigme{dir}{nɯ-}
\begin{définition}\pfra{rouler la pâte}\end{définition}
\begin{définition}\pcmn{擀面}\end{définition}
\begin{exemple}\pjya{ɯʑo kɯ pɤjpe na-nthar}\hspace{5pt}\pcmn{他擀了面}\end{exemple}\end{entrée}

\begin{entrée}{nthɤβ}{}{ⓔnthɤβ} 
\classe{vt} \paradigme{dir}{nɯ-}\paradigme{dir}{kɤ-}\paradigme{dir}{nɯ-}\paradigme{dir}{kɤ-}
\begin{définition}\pfra{serrer, coincer}\end{définition}
\begin{définition}\pcmn{夹住;夹到}\end{définition}
\begin{définition}\pfra{faire se coincer}\end{définition}
\begin{définition}\pcmn{使夹住}\end{définition}
\begin{exemple}\pjya{ki tɤ-ri ɲɤ-nthɤβ}\hspace{5pt}\pcmn{这根线夹到了}\end{exemple}
\begin{exemple}\pjya{a-jaʁ na-nthɤβ}\hspace{5pt}\pcmn{他夹到了我的手}\end{exemple}
\begin{exemple}\pjya{kɯm kɤ-pa-t-a, a-jaʁ na-nthɤβ}\hspace{5pt}\pcmn{我关了门,夹到了我的手}\end{exemple}
\begin{exemple}\pjya{a-jaʁ kɤ-nɯ-sɯnthaβ-a}\hspace{5pt}\pcmn{我自己夹到了自己的手}\end{exemple}
\begin{sous-entrée}{sɯnthɤβ}{ⓔnthɤβⓝsɯnthɤβ} 
\classe{vt}  
\grammaire{caus} \end{sous-entrée}

\end{entrée}

\begin{entrée}{nthor}{}{ⓔnthor} 
\classe{vi} \paradigme{dir}{\_}
\begin{définition}\pfra{rôder}\end{définition}
\begin{définition}\pcmn{流浪}\end{définition}
\begin{exemple}\pjya{mbro ɯ-zda maŋe tɕe ɲɤ-nthor}\hspace{5pt}\pcmn{马离了群就流浪}\end{exemple}\étymologie{ⁿtʰor}\end{entrée}

\begin{entrée}{ntoʁntoʁ}{}{ⓔntoʁntoʁ} 
\classe{idph.2} 
\begin{définition}\pfra{petit, rond et dur}\end{définition}
\begin{définition}\pcmn{形容小、圆而硬的样子}\end{définition}
\begin{exemple}\pjya{mɯntoʁ ɯ-tɯ-mpɕɤr kɯ ntoʁntoʁ ʑo ɲɯ-pa}\hspace{5pt}\pcmn{花又小又圆又漂亮}\end{exemple}\end{entrée}

\begin{entrée}{ntsɣe}{}{ⓔntsɣe} 
\classe{vt} \paradigme{dir}{nɯ-}
\begin{définition}\pfra{vendre}\end{définition}
\begin{définition}\pcmn{卖}\end{définition}
\begin{exemple}\pjya{@luyinji kɤ-ntsɣe ɯ-spa ɯ-ɲɯ́-ŋu}\hspace{5pt}\pcmn{有没有录音机卖}\end{exemple}
\begin{exemple}\pjya{ʑɴɢɯloʁ nɯ-ntsɣe-t-a}\hspace{5pt}\pcmn{我卖了核桃}\end{exemple}
\begin{sous-entrée}{ʑɣɤntsɣe}{ⓔntsɣeⓝʑɣɤntsɣe} 
\classe{vi}  
\grammaire{refl} 
\begin{définition}\pfra{se trahir}\end{définition}
\begin{définition}\pcmn{出卖自己}\end{définition}
\begin{exemple}\pjya{nɤʑo ɲɯ-tɯ-nɯ-ʑɣɤntsɣe ʑo ɲɯ-ɕti}\hspace{5pt}\pcmn{你自己出卖自己}\end{exemple}\relationsémantique{参考}{\lien{ⓔtɯtsɣe}{tɯtsɣe}}\relationsémantique{参考}{\lien{ⓔrɤtsɣe}{rɤtsɣe}}\relationsémantique{参考}{\lien{ⓔraχtɯtsɣe}{raχtɯtsɣe}}\end{sous-entrée}

\end{entrée}

\begin{entrée}{ntshɤβ}{}{ⓔntshɤβ} 
\classe{vs} \paradigme{dir}{tɤ-}
\begin{définition}\pfra{être affolé}\end{définition}
\begin{définition}\pcmn{慌张}\end{définition}
\begin{exemple}\pjya{ɯʑo ɲɤ-mu tɕe to-ntshɤβ}\hspace{5pt}\pcmn{他受到惊吓就慌张起来了}\end{exemple}\paradigme{dir}{tɤ-}
\begin{définition}\pfra{être affolé}\end{définition}
\begin{définition}\pcmn{慌张}\end{définition}
\begin{définition}\pfra{paniquer, se mettre dans tous ses états}\end{définition}
\begin{définition}\pcmn{慌张}\end{définition}
\begin{exemple}\pjya{ma-tɯ-ntshɤβ-rlu-ndʑi}\hspace{5pt}\pcmn{你们俩不要慌张}\end{exemple}
\begin{exemple}\pjya{tɤ-ntshɤβ-tɤ-rlu-a ʑo tɤ-rɤŋgat-a ɕti tɕe a-sɤcɯ kɤ-ndo ɲɤ-nɯjmɯt-a}\hspace{5pt}\pcmn{我出发的时候慌张到忘了带钥匙}\end{exemple}
\begin{sous-entrée}{sɯntshɤβ}{ⓔntshɤβⓝsɯntshɤβ} 
\classe{vt} 
\begin{définition}\pfra{affolé, rendre}\end{définition}
\begin{définition}\pcmn{令……紧张、慌张}\end{définition}
\begin{exemple}\pjya{ma-kɯ-sɯntshaβ-a ma tha a-laχtɕha kɯ-sɯjmɯt-a}\hspace{5pt}\pcmn{你不要令我紧张,你会令我忘记带东西}\end{exemple}\relationsémantique{同义词}{\lien{ⓔɕɯmbɣom}{ɕɯmbɣom}}\end{sous-entrée}

\begin{sous-entrée}{ʑɣɤsɯntshɤβ}{ⓔntshɤβⓝʑɣɤsɯntshɤβ} 
\classe{vi} \end{sous-entrée}

\begin{sous-entrée}{ntshɤβ,rlu}{ⓔntshɤβⓝntshɤβ,rlu} 
\classe{vs}
\classe{vs} \paradigme{dir}{tɤ-}\relationsémantique{Component 1}{\lien{ⓔntshɤβ}{ntshɤβ}}\relationsémantique{Component 2}{\lien{}{rlu}}\end{sous-entrée}

\étymologie{ⁿtsʰab}\end{entrée}

\begin{entrée}{ntshɤβ,rlu}{}{ⓔntshɤβ,rlu}\relationsémantique{参考}{\lien{ⓔntshɤβ}{ntshɤβ}}\end{entrée}

\begin{entrée}{ntshɤr}{}{ⓔntshɤr} 
\classe{vi} \paradigme{dir}{nɯ-}
\begin{définition}\pfra{hennir}\end{définition}
\begin{définition}\pcmn{叫(马叫)}\end{définition}
\begin{exemple}\pjya{mbro ɲɯ-ntshɤr}\hspace{5pt}\pcmn{马在嘶叫}\end{exemple}
\begin{exemple}\pjya{mbro nɯ-ntshɤr}\hspace{5pt}\pcmn{马嘶叫了}\end{exemple}\end{entrée}

\begin{entrée}{ntshi}{₁}{ⓔntshiⓗ1} 
\classe{vt} \paradigme{dir}{nɯ-}
\begin{définition}\pfra{sélectionner}\end{définition}
\begin{définition}\pcmn{挑选;拣}\end{définition}
\begin{exemple}\pjya{rasti nɯ-ntshi}\hspace{5pt}\pcmn{你选一下圆根}\end{exemple}
\begin{exemple}\pjya{aʑo nɯ-ntshi-t-a}\hspace{5pt}\pcmn{我选了}\end{exemple}\end{entrée}

\begin{entrée}{ntshi}{₂}{ⓔntshiⓗ2} 
\classe{vs} 
\begin{définition}\pfra{mieux valoir que, devoir}\end{définition}
\begin{définition}\pcmn{只好}\end{définition}\end{entrée}

\begin{entrée}{ntshoʁ}{}{ⓔntshoʁ} 
\classe{vi} \paradigme{dir}{pɯ-}
\begin{définition}\pfra{réciter des soutras en groupe}\end{définition}
\begin{définition}\pcmn{念经}\end{définition}
\begin{exemple}\pjya{χpɯn ra ɲɯ-ntshoʁ-nɯ}\hspace{5pt}\pcmn{和尚们在念经}\end{exemple}\étymologie{ⁿtsʰog}\end{entrée}

\begin{entrée}{ntsɯ}{}{ⓔntsɯ} 
\classe{adv} \sens{1}
\begin{définition}\pfra{tout le temps}\end{définition}
\begin{définition}\pcmn{总是}\end{définition}\sens{2}
\begin{définition}\pfra{à chaque ...}\end{définition}
\begin{définition}\pcmn{每一...}\end{définition}\end{entrée}

\begin{entrée}{ntʂu}{}{ⓔntʂu} 
\classe{vl} \paradigme{dir}{lɤ-}
\begin{définition}\pfra{sarcler}\end{définition}
\begin{définition}\pcmn{薅锄;锄草}\end{définition}
\begin{exemple}\pjya{la-ntʂu}\hspace{5pt}\pcmn{他锄了草}\end{exemple}
\begin{exemple}\pjya{lɤ-ntʂu-t-a}\hspace{5pt}\pcmn{我锄了草}\end{exemple}
\begin{exemple}\pjya{lɤ-ntʂu-j}\hspace{5pt}\pcmn{我们锄了草}\end{exemple}
\begin{exemple}\pjya{tɤɕi ɯ-ŋgɯ lɤ-ntʂu-a}\hspace{5pt}\pcmn{我锄了青稞}\end{exemple}
\begin{exemple}\pjya{tɤɕi lɤ-ntʂu-t-a}\hspace{5pt}\pcmn{我锄了青稞}\end{exemple}\relationsémantique{参考}{\lien{ⓔtʂu}{tʂu}}\end{entrée}

\begin{entrée}{nɯ}{}{ⓔnɯ} 
\classe{dem} 
\begin{définition}\pfra{celà}\end{définition}
\begin{définition}\pcmn{那个}\end{définition}\end{entrée}

\begin{entrée}{nɯbabɯ}{}{ⓔnɯbabɯ} 
\classe{vi} \paradigme{dir}{\_}
\begin{définition}\pfra{ramasser du cassis}\end{définition}
\begin{définition}\pcmn{捡黑茶藨子}\end{définition}\relationsémantique{参考}{\lien{ⓔbabɯ}{babɯ}}\end{entrée}

\begin{entrée}{nɯbɤβ}{}{ⓔnɯbɤβ}\relationsémantique{参考}{\lien{ⓔbɤbɤβ}{bɤbɤβ}}\end{entrée}

\begin{entrée}{nɯβdaʁ}{}{ⓔnɯβdaʁ} 
\classe{vt}  
\grammaire{denom} \paradigme{dir}{tɤ-}\sens{1}
\begin{définition}\pfra{surveiller}\end{définition}
\begin{définition}\pcmn{看管(孩子、东西等)}\end{définition}
\begin{exemple}\pjya{laχtɕha tɤ-nɯβdaʁ}\hspace{5pt}\pcmn{你把东西看好}\end{exemple}\sens{2}
\begin{définition}\pfra{contrôler}\end{définition}
\begin{définition}\pcmn{控制;管理}\end{définition}\sens{3}\paradigme{dir}{nɯ-}
\begin{définition}\pfra{prendre la responsabilité}\end{définition}
\begin{définition}\pcmn{承担;被冤枉}\end{définition}
\begin{exemple}\pjya{ɯʑo kɯ ta-nɤma pɯ-ɕti ri, aʑo kɯ nɯ-nɯβdaʁ-a pɯ-ra}\hspace{5pt}\pcmn{本来是他做的事情,最后让我承担了}\end{exemple}
\begin{sous-entrée}{znɯβdaʁ}{ⓔnɯβdaʁⓝznɯβdaʁ} 
\classe{vt} \paradigme{dir}{nɯ-}
\begin{définition}\pfra{faire porter la responsabilité à}\end{définition}
\begin{définition}\pcmn{让……承担}\end{définition}
\begin{exemple}\pjya{mɤ-kɯ-tʂaŋ ɲɯ́-wɣ-znɯβdaʁ-a-nɯ ɲɯ-ŋu}\hspace{5pt}\pcmn{他们冤枉我(要我承担不公平的事情)}\end{exemple}
\begin{exemple}\pjya{ki laχtɕha kɯra nɤʑo tɤ-kɤ-znɯβdaʁ tɕe ɯ-pɯ tɤ-kɤ-pa ɲɯ-ŋu}\hspace{5pt}\pcmn{这些东西是为你的名义贮存的}\end{exemple}\relationsémantique{同义词}{\lien{ⓔnɤpɯpa}{nɤpɯpa}}\end{sous-entrée}

\étymologie{bdag}\end{entrée}

\begin{entrée}{nɯβdaχpu}{}{ⓔnɯβdaχpu} 
\classe{vi} \paradigme{dir}{lɤ-}\sens{1}
\begin{définition}\pfra{s'accaparer}\end{définition}
\begin{définition}\pcmn{归为己有}\end{définition}
\begin{exemple}\pjya{kɯki nɤʑɯɣ ɕti tɕe, aʑo lu-nɯβdaχpu-a mɤ-pe}\hspace{5pt}\pcmn{这是你的东西,我不应该归为己有}\end{exemple}\sens{2}
\begin{définition}\pfra{se prendre pour le maître de maison}\end{définition}
\begin{définition}\pcmn{喧宾夺主}\end{définition}
\begin{exemple}\pjya{ki tɯrme ɯ-kha ɲɯ-ɕti tɕe, ma-lɤ-tɯ-nɯβdaχpu ma βdaχpu ri ɲɯ-tɯ-maʁ}\hspace{5pt}\pcmn{这是别人的家,你不要喧宾夺主,你又不是主人}\end{exemple}\relationsémantique{参考}{\lien{ⓔβdaχpuⓗ1}{βdaχpu}}\end{entrée}

\begin{entrée}{nɯβde}{}{ⓔnɯβde}\relationsémantique{参考}{\lien{ⓔβde}{βde}}\end{entrée}

\begin{entrée}{nɯβɣɤmu}{}{ⓔnɯβɣɤmu} 
\classe{vi}  
\grammaire{denom} \paradigme{dir}{pɯ-}
\begin{définition}\pfra{surveiller la meule}\end{définition}
\begin{définition}\pcmn{看守水磨}\end{définition}
\begin{exemple}\pjya{pɯ-nɯβɣɤmu-a}\hspace{5pt}\pcmn{我守过水磨}\end{exemple}\relationsémantique{参考}{\lien{ⓔβɣɤmu}{βɣɤmu}}\end{entrée}

\begin{entrée}{nɯβɣe}{}{ⓔnɯβɣe} 
\classe{vi} \paradigme{dir}{pɯ-}
\begin{définition}\pfra{perdre un membre de sa famille}\end{définition}
\begin{définition}\pcmn{失去亲人}\end{définition}
\begin{exemple}\pjya{tɤ-rɟit pɯ-kɯ-nɯβɣe}\hspace{5pt}\pcmn{孤儿}\end{exemple}
\begin{exemple}\pjya{tɤ-pɤtso pjɤ-nɯβɣe}\hspace{5pt}\pcmn{小孩子变成了孤儿}\end{exemple}\end{entrée}

\begin{entrée}{nɯβɣɯz}{}{ⓔnɯβɣɯz} 
\classe{vi} \paradigme{dir}{tɤ-}
\begin{définition}\pfra{chasser les blaireaux}\end{définition}
\begin{définition}\pcmn{抓獾}\end{définition}
\begin{exemple}\pjya{ɕ-tu-nɯβɣɯz-nɯ tɕe, βɣɯz nɯ pjɯ-sat-nɯ ŋgrɤl}\hspace{5pt}\pcmn{他们抓獾并杀獾}\end{exemple}\relationsémantique{参考}{\lien{ⓔβɣɯz}{βɣɯz}}\end{entrée}

\begin{entrée}{nɯβlu}{}{ⓔnɯβlu} 
\classe{vt}
\classe{vi}  
\grammaire{denom} \paradigme{dir}{pɯ-}\paradigme{dir}{pɯ-}
\begin{définition}\pfra{tromper}\end{définition}
\begin{définition}\pcmn{欺骗}\end{définition}
\begin{définition}\pfra{se laisser tromper}\end{définition}
\begin{définition}\pcmn{被骗}\end{définition}
\begin{exemple}\pjya{tɤ-pɤtso pɯ-nɯβlu-t-a}\hspace{5pt}\pcmn{我骗了小孩子}\end{exemple}
\begin{exemple}\pjya{jiɕqha nɯ kɯ pɯ́-wɣ-nɯβlu-a}\hspace{5pt}\pcmn{这个人把我骗了}\end{exemple}\relationsémantique{参考}{\lien{ⓔɯ-βlu}{ɯ-βlu}}
\begin{sous-entrée}{sɤnɯβlu}{ⓔnɯβluⓝsɤnɯβlu} 
\classe{vi}  
\grammaire{apass} 
\begin{définition}\pfra{tromper les gens}\end{définition}
\begin{définition}\pcmn{骗人}\end{définition}\end{sous-entrée}

\begin{sous-entrée}{ʑɣɤnɯβlu}{ⓔnɯβluⓝʑɣɤnɯβlu}\end{sous-entrée}

\end{entrée}

\begin{entrée}{nɯβlɤmtɕhɤt}{}{ⓔnɯβlɤmtɕhɤt} 
\classe{vt} \paradigme{dir}{tɤ-}
\begin{définition}\pfra{réciter des soutras}\end{définition}
\begin{définition}\pcmn{念经}\end{définition}
\begin{exemple}\pjya{nɤʑo tu-ta-nɯβlɤmtɕhɤt ɯ́-jɤɣ?}\hspace{5pt}\pcmn{请你为我们念经行吗?}\end{exemple}\relationsémantique{参考}{\lien{ⓔβlɤmtɕhɤt}{βlɤmtɕhɤt}}\end{entrée}

\begin{entrée}{nɯβlɯz}{}{ⓔnɯβlɯz} 
\classe{vt} \paradigme{dir}{pɯ-}
\begin{définition}\pfra{réciter par cœur, faire sans modèle}\end{définition}
\begin{définition}\pcmn{背诵;不用模型地做}\end{définition}
\begin{exemple}\pjya{kɤ-rɤt pɯ-nɯβlɯz-a ɕti}\hspace{5pt}\pcmn{我没有样板也画出来了}\end{exemple}\relationsémantique{参考}{\lien{ⓔtɯ-βlɯz}{tɯ-βlɯz}}\end{entrée}

\begin{entrée}{nɯβra}{}{ⓔnɯβra} 
\classe{vt} \paradigme{dir}{tɤ-}\sens{1}
\begin{définition}\pfra{fournir}\end{définition}
\begin{définition}\pcmn{提供}\end{définition}
\begin{exemple}\pjya{nɤ-ŋga tɤ-nɯβra-t-a}\hspace{5pt}\pcmn{我给你提供了衣服}\end{exemple}\sens{2}
\begin{définition}\pfra{qui a pour usage de ...}\end{définition}
\begin{définition}\pcmn{有……的功能}\end{définition}\end{entrée}

\begin{entrée}{nɯβraʁ}{}{ⓔnɯβraʁ}\relationsémantique{参考}{\lien{ⓔβraʁ}{βraʁ}}\end{entrée}

\begin{entrée}{nɯβzaŋsa}{}{ⓔnɯβzaŋsa} 
\classe{vt}  
\grammaire{denom} \paradigme{dir}{tɤ-}
\begin{définition}\pfra{devenir ami}\end{définition}
\begin{définition}\pcmn{交朋友}\end{définition}
\begin{exemple}\pjya{tɤ-nɯβzaŋsa-t-a}\hspace{5pt}\pcmn{我跟他交了朋友}\end{exemple}
\begin{exemple}\pjya{tɤ́-wɣ-nɯβzaŋsa-a}\hspace{5pt}\pcmn{他跟我交了朋友}\end{exemple}
\begin{exemple}\pjya{tɯrme mɤ-kɯ-frtɤn nɯ a-mɤ-tɤ-tɯ-nɯβzaŋse}\hspace{5pt}\pcmn{你不要跟不可靠的人交朋友}\end{exemple}\relationsémantique{同义词}{\lien{ⓔnɯɣɯfsu}{nɯɣɯfsu}}\relationsémantique{参考}{\lien{ⓔβzaŋsa}{βzaŋsa}}\end{entrée}

\begin{entrée}{nɯβʑit}{}{ⓔnɯβʑit} 
\classe{vt} \sens{1}\paradigme{dir}{nɯ-}
\begin{définition}\pfra{faire diminuer}\end{définition}
\begin{définition}\pcmn{减一部分;减一段}\end{définition}
\begin{exemple}\pjya{laχtɕha ɲɤ-nɯβʑit}\hspace{5pt}\pcmn{他减了一些东西}\end{exemple}
\begin{exemple}\pjya{ji-kɤndza ɲɤ-nɯβʑit}\hspace{5pt}\pcmn{他减了我们的食物}\end{exemple}
\begin{exemple}\pjya{pɕawtsɯ ɲɤ-nɯβʑit}\hspace{5pt}\pcmn{我减了钱(贪污了)}\end{exemple}
\begin{exemple}\pjya{nɤ-fkur pjɤ-nɯβʑit}\hspace{5pt}\pcmn{他减少了你的负担}\end{exemple}\sens{2}\paradigme{dir}{pɯ-}
\begin{définition}\pfra{raccourcir}\end{définition}
\begin{définition}\pcmn{弄短一点}\end{définition}
\begin{exemple}\pjya{ɕoŋtɕa ɲɯ-zri tɕe pjɤ-nɯβʑit}\hspace{5pt}\pcmn{木料太长,他弄短了一些}\end{exemple}\end{entrée}

\begin{entrée}{nɯcaχto}{}{ⓔnɯcaχto} 
\classe{vi} \paradigme{dir}{thɯ-}\paradigme{dir}{thɯ-}
\begin{définition}\pfra{être bouche bée}\end{définition}
\begin{définition}\pcmn{目瞪口呆}\end{définition}
\begin{définition}\pfra{rendre bouche bée}\end{définition}
\begin{définition}\pcmn{令人目瞪口呆}\end{définition}
\begin{exemple}\pjya{thɯ-nɯcaχto-a}\hspace{5pt}\pcmn{我目瞪口呆了}\end{exemple}
\begin{exemple}\pjya{chɤ́-wɣ-znɯcaχto ʑo}\hspace{5pt}\pcmn{令他目瞪口呆}\end{exemple}\relationsémantique{同义词}{\lien{ⓔnɤχɤmthi}{nɤχɤmthi}}\relationsémantique{同义词}{\lien{ⓔznɤχɤmthi}{znɤχɤmthi}}
\begin{sous-entrée}{znɯcaχto}{ⓔnɯcaχtoⓝznɯcaχto} 
\classe{vt} \end{sous-entrée}

\end{entrée}

\begin{entrée}{nɯcɤɕna}{}{ⓔnɯcɤɕna} 
\classe{vi} \paradigme{dir}{pɯ-}\paradigme{dir}{\_}
\begin{définition}\pfra{ramasser du rumex japonicus}\end{définition}
\begin{définition}\pcmn{采集山菠菜}\end{définition}
\begin{exemple}\pjya{ɕ-pɯ-nɯcɤɕna}\hspace{5pt}\pcmn{你去采集山菠菜吧}\end{exemple}\relationsémantique{参考}{\lien{ⓔcɤɕna}{cɤɕna}}\end{entrée}

\begin{entrée}{nɯcha}{}{ⓔnɯcha} 
\classe{vs}  
\grammaire{denom} \paradigme{dir}{lɤ-}
\begin{définition}\pfra{être saoul}\end{définition}
\begin{définition}\pcmn{喝醉}\end{définition}\relationsémantique{同义词}{\lien{ⓔβzi}{βzi}}\relationsémantique{参考}{\lien{ⓔchaⓗ2}{cha₂}}\end{entrée}

\begin{entrée}{nɯchɤmda}{}{ⓔnɯchɤmda} 
\classe{vt}  
\grammaire{denom} \paradigme{dir}{}\paradigme{thɯ-}{}
\begin{définition}\pfra{boire de l'alcool à la paille}\end{définition}
\begin{définition}\pcmn{喝干干酒}\end{définition}
\begin{exemple}\pjya{tɯ́-wɣ-nɯchɤmda ɕti}\hspace{5pt}\pcmn{(妖精)会在你背后插吸管喝你的血}\end{exemple}\relationsémantique{参考}{\lien{ⓔchɤmda}{chɤmda}}\end{entrée}

\begin{entrée}{nɯchɤrga}{}{ⓔnɯchɤrga} 
\classe{vs}  
\grammaire{incorp} 
\begin{définition}\pfra{aimer boire de l'alcool}\end{définition}
\begin{définition}\pcmn{喜欢喝酒}\end{définition}
\begin{exemple}\pjya{ki kɯ-nɯchɤrga ci ŋu}\hspace{5pt}\pcmn{他是酒鬼}\end{exemple}\relationsémantique{参考}{\lien{ⓔchaⓗ2}{cha₂}}\relationsémantique{参考}{\lien{}{rga₂}}\end{entrée}

\begin{entrée}{nɯchɯβ}{}{ⓔnɯchɯβ} 
\classe{vt} \paradigme{dir}{tɤ-}
\begin{définition}\pfra{s'empiffrer}\end{définition}
\begin{définition}\pcmn{大口大口地吃}\end{définition}
\begin{exemple}\pjya{tɤ-nɯchɯβ-a ʑo tɤ-ndza-t-a}\hspace{5pt}\pcmn{我大口大口地吃了}\end{exemple}\relationsémantique{同义词}{\lien{ⓔnɯkhɯɣ}{nɯkhɯɣ}}\relationsémantique{同义词}{\lien{ⓔnɯlŋɤβ}{nɯlŋɤβ}}\end{entrée}

\begin{entrée}{nɯchɯra}{}{ⓔnɯchɯra} 
\classe{vi} \paradigme{dir}{pɯ-}
\begin{définition}\pfra{monter la garde}\end{définition}
\begin{définition}\pcmn{站岗;守卫}\end{définition}
\begin{exemple}\pjya{ɯʑo ku-nɯchɯra}\hspace{5pt}\pcmn{他在站岗}\end{exemple}\relationsémantique{同义词}{\lien{ⓔnɯsuwa}{nɯsuwa}}\end{entrée}

\begin{entrée}{nɯci}{}{ⓔnɯci} 
\classe{vi}  
\grammaire{denom} \paradigme{dir}{pɯ-}
\begin{définition}\pfra{boire sans se servir de ses mains}\end{définition}
\begin{définition}\pcmn{直接用嘴巴喝地下的水(不用手)}\end{définition}
\begin{exemple}\pjya{tɯ-ci ɯ-ŋgɯ pɯ-nɯci}\hspace{5pt}\pcmn{它喝了河流的水}\end{exemple}\relationsémantique{参考}{\lien{ⓔtɯ-ci}{tɯ-ci}}\end{entrée}

\begin{entrée}{nɯco}{}{ⓔnɯco} 
\classe{vt} \paradigme{dir}{\_}\paradigme{dir}{nɯ-}\paradigme{dir}{pɯ-}
\begin{définition}\pfra{suivre}\end{définition}
\begin{définition}\pcmn{跟踪;顺着走}\end{définition}
\begin{définition}\pfra{suivre les instructions de}\end{définition}
\begin{définition}\pcmn{依照……的说法}\end{définition}
\begin{exemple}\pjya{zgo nɯnɯ lɤ-nɯco-t-a}\hspace{5pt}\pcmn{我沿着这个山顶走了}\end{exemple}
\begin{exemple}\pjya{kɯki khri ki nɯ-nɯco-t-a ma nɯ ma mɯ́j-cha-a}\hspace{5pt}\pcmn{我只好沿着床边走,其它还不行(病人说的话)}\end{exemple}
\begin{exemple}\pjya{tʂu maŋe tɕe, tɯ-ci lɤ-nɯco-t-a tɕe lɤ-ari-a}\hspace{5pt}\pcmn{因为没有路,我顺着水流去了}\end{exemple}
\begin{exemple}\pjya{a-tɕɯ kɯ znde ku-nɯcɤm ŋu (ɲɯ-ɤz-nɯco)}\hspace{5pt}\pcmn{我儿子扶着墙慢慢走}\end{exemple}
\begin{exemple}\pjya{ɯʑo kɯ kɤ-ɕe ɲɯ-sɯsɤm qhe, tɕe nɯ-znɯɲco-t-a}\hspace{5pt}\pcmn{他想去那里,我就依他了}\end{exemple}\relationsémantique{参考}{\lien{ⓔnɯɴqhu}{nɯɴqhu}}
\begin{sous-entrée}{znɯɲco}{ⓔnɯcoⓝznɯɲco} 
\classe{vt} \end{sous-entrée}

\end{entrée}

\begin{entrée}{nɯcɯnthaʁ}{}{ⓔnɯcɯnthaʁ} 
\classe{vt} \sens{1}\paradigme{dir}{pɯ-}
\begin{définition}\pfra{hacher de la viande}\end{définition}
\begin{définition}\pcmn{剁肉}\end{définition}
\begin{exemple}\pjya{jiɕqha tɤ-mthɯm pɯ-nɯcɯnthaʁ}\hspace{5pt}\pcmn{你剁一下肉吧}\end{exemple}\sens{2}\paradigme{dir}{thɯ-}
\begin{définition}\pfra{hacher de l'ail}\end{définition}
\begin{définition}\pcmn{剁大蒜}\end{définition}\end{entrée}

\begin{entrée}{nɯɕu}{}{ⓔnɯɕu} 
\classe{vi}  
\grammaire{denom} \paradigme{dir}{pɯ-}
\begin{définition}\pfra{jouer aux cartes}\end{définition}
\begin{définition}\pcmn{打牌}\end{définition}
\begin{exemple}\pjya{jiʑo pɯ-nɯɕu-j}\hspace{5pt}\pcmn{我们打牌了}\end{exemple}\relationsémantique{参考}{\lien{ⓔɕu}{ɕu}}\end{entrée}

\begin{entrée}{nɯɕɤɣ}{}{ⓔnɯɕɤɣ} 
\classe{vi} \paradigme{dir}{pɯ-}
\begin{définition}\pfra{couper des genévriers (pour faire des fumigations)}\end{définition}
\begin{définition}\pcmn{砍柏树枝桠}\end{définition}
\begin{exemple}\pjya{ɕ-pɯ-nɯɕaɣ-a}\hspace{5pt}\pcmn{我砍了柏树枝桠}\end{exemple}\end{entrée}

\begin{entrée}{nɯɕɤlɤmbɯmbjom}{}{ⓔnɯɕɤlɤmbɯmbjom} 
\classe{vi} \paradigme{dir}{\_}
\begin{définition}\pfra{faire une course}\end{définition}
\begin{définition}\pcmn{赛跑}\end{définition}
\begin{exemple}\pjya{ɲɯ-nɯɕɤlɤmbɯmbjom-tɕi a-pɯ-ŋu tɕe, nɤʑo mɤ-tɯ-cha}\hspace{5pt}\pcmn{如果我们赛跑的话,你肯定不行}\end{exemple}\relationsémantique{同义词}{\lien{ⓔnɯsaχɕɯβ}{nɯsaχɕɯβ}}\end{entrée}

\begin{entrée}{nɯɕɤmɯɣdɯ}{}{ⓔnɯɕɤmɯɣdɯ} 
\classe{vt}  
\grammaire{denom} \paradigme{dir}{tɤ-}
\begin{définition}\pfra{tirer au fusil}\end{définition}
\begin{définition}\pcmn{射枪}\end{définition}
\begin{exemple}\pjya{tɤfsɯr ra tɤ-nɯɕɤmɯɣdɯ-t-a}\hspace{5pt}\pcmn{我用枪射了靶子}\end{exemple}\relationsémantique{参考}{\lien{ⓔɕɤmɯɣdɯ}{ɕɤmɯɣdɯ}}\end{entrée}

\begin{entrée}{nɯɕɤrɤz}{}{ⓔnɯɕɤrɤz} 
\classe{vi} 
\begin{définition}\pfra{tenir de, ressembler (à ses parents)}\end{définition}
\begin{définition}\pcmn{遗传;有点像(父母)}\end{définition}
\begin{exemple}\pjya{ɯ-wa ɲɯ-nɯɕɤrɤz}\hspace{5pt}\pcmn{他有点像父亲}\end{exemple}
\begin{exemple}\pjya{ɯ-phoŋbu ɯ-wa ɲɯ-nɯɕɤrɤz}\hspace{5pt}\pcmn{他的身体有点像他父亲的}\end{exemple}\end{entrée}

\begin{entrée}{nɯɕe}{}{ⓔnɯɕe} 
\classe{vi}  
\grammaire{vert} \paradigme{dir}{\_}\paradigme{past stem}{anɯri}
\begin{définition}\pfra{rentrer chez soi}\end{définition}
\begin{définition}\pcmn{回家}\end{définition}
\begin{exemple}\pjya{nɤʑo tɯ-nɯɕe ɕi, nɤ-mu ɯ-ɕki ?}\hspace{5pt}\pcmn{你回不回家,你母亲那边?}\end{exemple}\relationsémantique{参考}{\lien{ⓔɕe}{ɕe}}\end{entrée}

\begin{entrée}{nɯɕɣɤthɯt}{}{ⓔnɯɕɣɤthɯt} 
\classe{vt} \paradigme{dir}{thɯ-}
\begin{définition}\pfra{réparer (une lame, un soc de charrue)}\end{définition}
\begin{définition}\pcmn{补(铧头、斧头等)}\end{définition}\relationsémantique{参考}{\lien{ⓔtɯ-ɕɣa}{tɯ-ɕɣa}}\relationsémantique{参考}{\lien{ⓔmthɯt}{mthɯt}}\end{entrée}

\begin{entrée}{nɯɕkat}{}{ⓔnɯɕkat} 
\classe{vi}  
\grammaire{denom} 
\begin{définition}\pfra{transporter à dos d'animal}\end{définition}
\begin{définition}\pcmn{驮东西}\end{définition}
\begin{exemple}\pjya{aʑo kɯ-nɯɕkat ŋu-a}\hspace{5pt}\pcmn{我是驮东西的人}\end{exemple}\relationsémantique{参考}{\lien{ⓔɣɯɕkat}{ɣɯɕkat}}\end{entrée}

\begin{entrée}{nɯɕkrɤɣ}{}{ⓔnɯɕkrɤɣ} 
\classe{vt}  
\grammaire{deidph} \paradigme{dir}{\_}
\begin{définition}\pfra{renverser avec force son adversaire}\end{définition}
\begin{définition}\pcmn{力气很大,很轻松地把对方摔下去}\end{définition}
\begin{exemple}\pjya{tɤ-aʑɯʑu-ndʑi tɕe, ɯ-zda pa-nɯɕkrɤɣ ʑo pa-tʂaβ}\hspace{5pt}\pcmn{在角力的时候,他很轻松地把对方摔下去了}\end{exemple}\relationsémantique{参考}{\lien{ⓔɕkrɤɣɕkrɤɣ}{ɕkrɤɣɕkrɤɣ}}\relationsémantique{参考}{\lien{ⓔnɯʑgrɤɣ}{nɯʑgrɤɣ}}\end{entrée}

\begin{entrée}{nɯɕmɯrga}{}{ⓔnɯɕmɯrga} 
\classe{vs}  
\grammaire{incorp} 
\begin{définition}\pfra{bavard}\end{définition}
\begin{définition}\pcmn{爱说话}\end{définition}
\begin{exemple}\pjya{ɯʑo wuma kɯ-nɯɕmɯrga ci ɲɯ-ŋu}\end{exemple}\relationsémantique{参考}{\lien{ⓔrɯɕmi}{rɯɕmi}}\relationsémantique{参考}{\lien{ⓔrɯɕmɯχtɤm}{rɯɕmɯχtɤm}}\relationsémantique{参考}{\lien{ⓔrgaⓗ1ⓝrga}{rga}}\end{entrée}

\begin{entrée}{nɯɕpɯz}{}{ⓔnɯɕpɯz} 
\classe{vt} \paradigme{dir}{tɤ-}
\begin{définition}\pfra{se déguiser, imiter}\end{définition}
\begin{définition}\pcmn{打扮;模仿}\end{définition}
\begin{exemple}\pjya{staʁthɤr kɯ ɯ-pi ta-nɯɕpɯz}\hspace{5pt}\pcmn{斯达塔尔学了他的哥哥}\end{exemple}
\begin{exemple}\pjya{tɤ-pɤtso ra kɯ ʁmaʁmi ɲɯ-ɤz-nɯɕpɯz-nɯ}\hspace{5pt}\pcmn{孩子们在打扮成士兵}\end{exemple}
\begin{exemple}\pjya{tɤ-pɤtso ra kɯ ɕɤmɯɣdɯ ɯ-kɯ-lɤt ɲɯ-ɤz-nɯɕpɯz-nɯ}\hspace{5pt}\pcmn{孩子们装作在打枪}\end{exemple}
\begin{exemple}\pjya{tɤ́-wɣ-nɯɕpɯz-a}\hspace{5pt}\pcmn{他模仿了我的模样}\end{exemple}\relationsémantique{参考}{\lien{ⓔɯ-ɕpɯz}{ɯ-ɕpɯz}}\end{entrée}

\begin{entrée}{nɯɕqhu}{}{ⓔnɯɕqhu} 
\classe{vt} \sens{1}\paradigme{dir}{\_}
\begin{définition}\pfra{tourner le dos à}\end{définition}
\begin{définition}\pcmn{转身背向别人}\end{définition}
\begin{exemple}\pjya{khɯɣɲɟɯ ku-nɯɕqhe-a ŋu}\hspace{5pt}\pcmn{我背向窗子}\end{exemple}\sens{2}\paradigme{dir}{tɤ-}
\begin{définition}\pfra{trahir, revenir sur sa parole}\end{définition}
\begin{définition}\pcmn{背叛;违背约定;违反约定}\end{définition}
\begin{exemple}\pjya{nɤʑo tu-kɯ-nɯɕqhu-a ɲɯ-ŋu}\hspace{5pt}\pcmn{你背叛我}\end{exemple}\relationsémantique{同义词}{\lien{ⓔnɯɣɤtɕa}{nɯɣɤtɕa}}\relationsémantique{反义词}{\lien{ⓔnɯʁɤri}{nɯʁɤri}}\relationsémantique{参考}{\lien{ⓔɯ-qhu}{ɯ-qhu}}\relationsémantique{参考}{\lien{ⓔznɯɕqhɯɕqhu}{znɯɕqhɯɕqhu}}
\begin{sous-entrée}{anɯɕqhɯɕqhu}{ⓔnɯɕqhuⓢ2ⓝanɯɕqhɯɕqhu} 
\classe{vi}  
\grammaire{recip} \sens{1}
\begin{définition}\pfra{être opposé, être le contraire (parole)}\end{définition}
\begin{définition}\pcmn{相反(话)}\end{définition}
\begin{exemple}\pjya{ndʑiʑo ndʑi-kɤ-ti ɲɯ-ɤnɯɕqhɯɕqhu}\hspace{5pt}\pcmn{他说的话跟你说的话是相反的}\end{exemple}\end{sous-entrée}

\sens{2}
\begin{définition}\pfra{revenir chacun sur sa parole}\end{définition}
\begin{définition}\pcmn{互相违背(约定的事情)}\end{définition}
\begin{exemple}\pjya{tɯkrɤz tɤ-βzu-tɕi ŋu tɕe, mɤ-anɯɕqhɯɕqhu-tɕi ra nɤ!}\hspace{5pt}\pcmn{我们商量好了,不要违背约定}\end{exemple}\end{entrée}

\begin{entrée}{nɯɕtar}{}{ⓔnɯɕtar} 
\classe{vi} \paradigme{dir}{pɯ-}\paradigme{dir}{pɯ-}
\begin{définition}\pfra{avoir une leçon}\end{définition}
\begin{définition}\pcmn{受教训}\end{définition}
\begin{définition}\pfra{donner une leçon}\end{définition}
\begin{définition}\pcmn{教训}\end{définition}
\begin{exemple}\pjya{aʑo pɯ-nɯɕtar-a}\hspace{5pt}\pcmn{我受过那个教训}\end{exemple}
\begin{exemple}\pjya{a-pɯ-tɯ-nɯɕtar ɲɯ-ra wo!}\hspace{5pt}\pcmn{你应该吸取教训!}\end{exemple}
\begin{exemple}\pjya{mɯ-ɲɤ-stu-nɯ ma pjɤ-nɯɕtar-nɯ}\hspace{5pt}\pcmn{他们吸取了教训,再也不相信他了}\end{exemple}
\begin{exemple}\pjya{aʑo kutɕu kɤ-ɣi pɯ-nɯɕtar-a ma ɲɯ-ɤrqhi}\hspace{5pt}\pcmn{我来到这里很辛苦,因为很远}\end{exemple}
\begin{sous-entrée}{znɯɕtar}{ⓔnɯɕtarⓝznɯɕtar} 
\classe{vt} \end{sous-entrée}

\begin{sous-entrée}{sɤnɯɕtar/\variante{sɤɕtar}}{ⓔnɯɕtarⓝsɤnɯɕtar} 
\classe{vs} 
\begin{définition}\pfra{qui donne une leçon}\end{définition}
\begin{définition}\pcmn{令人受教训}\end{définition}\end{sous-entrée}

\end{entrée}

\begin{entrée}{nɯɕɯβɟɤlɯlu}{}{ⓔnɯɕɯβɟɤlɯlu} 
\classe{vi} 
\begin{définition}\pfra{à qui mieux mieux}\end{définition}
\begin{définition}\pcmn{争先恐后}\end{définition}
\begin{exemple}\pjya{laχtɕha kɤ-χtɯ ɲɯ-pe tɕe, to-nɯɕɯβɟɤlɯlu-nɯ ʑo to-χtɯ-nɯ}\hspace{5pt}\pcmn{买的东西很好,所以他们争着买}\end{exemple}
\begin{exemple}\pjya{nɤki nɯ kɯ-fse ɯ-qhu tɕe kɤ-χtɯ tu ɕti tɕe, kɤ-nɯɕɯβɟɤlɯlu mɤ-ra wo}\hspace{5pt}\pcmn{那个东西以后还有的买,不必争}\end{exemple}\end{entrée}

\begin{entrée}{nɯɕɯlu}{}{ⓔnɯɕɯlu} 
\classe{vs} 
\begin{définition}\pfra{qui peut être traite pendant longtemps (vache)}\end{définition}
\begin{définition}\pcmn{挤奶期长(的奶牛)}\end{définition}
\begin{exemple}\pjya{ki nɯŋa ki kɯ-nɯɕɯlu ci ŋu}\hspace{5pt}\pcmn{这个奶牛的挤奶期比较长}\end{exemple}\end{entrée}

\begin{entrée}{nɯɕɯrɲɟo}{}{ⓔnɯɕɯrɲɟo} 
\classe{vi} \paradigme{dir}{pɯ-}
\begin{définition}\pfra{rougir (feuilles d'arbre en automne)}\end{définition}
\begin{définition}\pcmn{秋天叶子变色}\end{définition}
\begin{exemple}\pjya{si pjɤ-nɯɕɯrɲɟo}\hspace{5pt}\pcmn{树的叶子变红了}\end{exemple}\end{entrée}

\begin{entrée}{nɯɕɯʁjɯ}{}{ⓔnɯɕɯʁjɯ} 
\classe{vi} \paradigme{dir}{nɯ-}
\begin{définition}\pfra{faire le mort}\end{définition}
\begin{définition}\pcmn{装死}\end{définition}
\begin{exemple}\pjya{ma-nɯ-tɯ-nɯɕɯʁjɯ kɯ nɯ-rɤma}\hspace{5pt}\pcmn{你不要装自己不会做,要劳动!}\end{exemple}\relationsémantique{同义词}{\lien{ⓔraʁjɯ}{raʁjɯ}}\end{entrée}

\begin{entrée}{nɯdu}{}{ⓔnɯdu} 
\classe{vi} \paradigme{dir}{pɯ-}
\begin{définition}\pfra{tirer à la courte paille}\end{définition}
\begin{définition}\pcmn{抽签}\end{définition}
\begin{exemple}\pjya{ɕɯ-nɯdu-j}\hspace{5pt}\pcmn{我们来抽签}\end{exemple}\relationsémantique{参考}{\lien{ⓔɯ-du}{ɯ-du}}\end{entrée}

\begin{entrée}{nɯdrɯβ}{}{ⓔnɯdrɯβ} 
\classe{vt} 
\begin{définition}\pfra{encorner à de nombreuses reprises}\end{définition}
\begin{définition}\pcmn{一次又一次地顶}\end{définition}
\begin{exemple}\pjya{srɯnmɯ nɯ to-nɯdrɯβ ʑo to-tɕhɯ}\hspace{5pt}\pcmn{水牛把妖精一次又一次地顶了}\end{exemple}\relationsémantique{参考}{\lien{ⓔdrɯβ}{drɯβ}}\end{entrée}

\begin{entrée}{nɯfɕi}{}{ⓔnɯfɕi} 
\classe{vt} \paradigme{dir}{nɯ-}
\begin{définition}\pfra{s'inquiéter, ne pas vouloir faire (un travail)}\end{définition}
\begin{définition}\pcmn{怕麻烦;担心}\end{définition}
\begin{exemple}\pjya{nɯ kɤ-nɤma aʑo ɲɯ-nɯfɕi-a ɕti}\hspace{5pt}\pcmn{我不想做这个这个工作,很怕麻烦}\end{exemple}
\begin{exemple}\pjya{kɯki ɯ-tɯ-dɤn ɲɯ-tɕhom tɕe aj ɲɯ-nɯfɕi-a}\hspace{5pt}\pcmn{太多工作,我怕麻烦}\end{exemple}
\begin{exemple}\pjya{zgoku ɲɯ-mbro tɕe kɤ-ɕe ɲɯ-nɯfɕi}\hspace{5pt}\pcmn{山很高,他想去,怕麻烦}\end{exemple}
\begin{exemple}\pjya{khro tɤ-nɤma-t-a tɕe, mɤʑɯ kɤ-nɤma ɲɯ-nɯfɕi-a ɕti}\hspace{5pt}\pcmn{我已经工作很多了,不想再做了}\end{exemple}\end{entrée}

\begin{entrée}{nɯfkurzʁe}{}{ⓔnɯfkurzʁe} 
\classe{vi} \paradigme{dir}{\_}
\begin{définition}\pfra{transporter des charges sur le dos}\end{définition}
\begin{définition}\pcmn{背东西}\end{définition}
\begin{exemple}\pjya{jisŋi ɕ-pɯ-nɯfkurzʁe-j tɕe wuma ʑo pɯ-ɴqa}\hspace{5pt}\pcmn{我们今天背了很多东西,很辛苦}\end{exemple}\relationsémantique{参考}{\lien{ⓔnɯzʁe}{nɯzʁe}}\relationsémantique{参考}{\lien{ⓔfkur}{fkur}}\relationsémantique{参考}{\lien{ⓔfkurzʁe}{fkurzʁe}}\end{entrée}

\begin{entrée}{nɯfse}{₂}{ⓔnɯfseⓗ2} 
\classe{adv} \sens{1}
\begin{définition}\pfra{comme cela, sans but particulier}\end{définition}
\begin{définition}\pcmn{就这样;笼统地;没有目标地;随便}\end{définition}
\begin{exemple}\pjya{nɯfse ɕ-tu-nɤŋkɯŋke-a ŋu}\hspace{5pt}\pcmn{我(没有目标地)逛街}\end{exemple}\sens{2}
\begin{définition}\pfra{malgré tout}\end{définition}
\begin{définition}\pcmn{无论怎么样都……,不顾一切}\end{définition}
\begin{exemple}\pjya{ɲɯ-mɯtɕaʁ ri, nɯfse ʑo pɯ-rɤʑit-a pɯ-ta-nɤjo.}\hspace{5pt}\pcmn{虽然很冷,我还是在那里等了你}\end{exemple}\relationsémantique{参考}{\lien{ⓔfseⓗ1}{fse₁}}\end{entrée}

\begin{entrée}{nɯfse}{₁}{ⓔnɯfseⓗ1} 
\classe{vt} \paradigme{dir}{kɤ-}
\begin{définition}\pfra{reconnaître, être familier}\end{définition}
\begin{définition}\pcmn{认得;熟悉(人物)}\end{définition}
\begin{exemple}\pjya{ɯ-ɲɯ́-kɯ-nɯfse-a?}\hspace{5pt}\pcmn{你认得我吗?}\end{exemple}
\begin{exemple}\pjya{ɯʑo kɤ-nɯfse-t-a}\hspace{5pt}\pcmn{我认得他了}\end{exemple}
\begin{exemple}\pjya{ki tɯrme ɲɯ-nɯfse-a}\hspace{5pt}\pcmn{我认识这个人}\end{exemple}
\begin{exemple}\pjya{nɤ-kɤ-nɯfse kɯ-me nɯ sɤzdɯɣ}\hspace{5pt}\pcmn{没有你认识的人,很难受}\end{exemple}\end{entrée}

\begin{entrée}{nɯfsosɲɯm}{}{ⓔnɯfsosɲɯm} 
\classe{vi}  
\grammaire{denom} 
\begin{définition}\pfra{demander l'aumône (moines)}\end{définition}
\begin{définition}\pcmn{讨布施}\end{définition}
\begin{exemple}\pjya{kɯ-nɯfsosɲɯm jɤ-ari-a}\hspace{5pt}\pcmn{我去讨布施了f}\end{exemple}\relationsémantique{参考}{\lien{ⓔfsosɲɯm}{fsosɲɯm}}\end{entrée}

\begin{entrée}{nɯftɕaka}{}{ⓔnɯftɕaka} 
\classe{vt}  
\grammaire{denom} \paradigme{dir}{tɤ-}
\begin{définition}\pfra{préparer}\end{définition}
\begin{définition}\pcmn{准备;收拾}\end{définition}\relationsémantique{同义词}{\lien{}{mɲo}}\relationsémantique{参考}{\lien{ⓔrɯftɕaka}{rɯftɕaka}}\relationsémantique{参考}{\lien{ⓔftɕaka}{ftɕaka}}\end{entrée}

\begin{entrée}{nɯftsaʁ}{}{ⓔnɯftsaʁ} 
\classe{vi}  
\grammaire{denom} \paradigme{dir}{pɯ-}
\begin{définition}\pfra{couler goutte à goutte}\end{définition}
\begin{définition}\pcmn{滴水}\end{définition}
\begin{exemple}\pjya{@kongtiao ɯ-ŋgɯ tɯ-ci ɲɯ-nɯftsaʁ}\hspace{5pt}\pcmn{空调在滴水}\end{exemple}
\begin{exemple}\pjya{khɤxtu ɲɯ-nɯftsaʁ}\hspace{5pt}\pcmn{房背在滴水}\end{exemple}
\begin{exemple}\pjya{@ganggang ɲɯ-spoʁ rcáma, tɯ-ci ɲɯ-nɯftsaʁ}\hspace{5pt}\pcmn{也许是杯子漏了,因为在滴水}\end{exemple}
\begin{exemple}\pjya{tɯ-ftsaʁ tʂhɤtnɤtʂhɤt ʑo ɲɯ-nɯftsaʁ}\hspace{5pt}\pcmn{一滴一滴地漏水}\end{exemple}\relationsémantique{参考}{\lien{ⓔtɯftsaʁ}{tɯftsaʁ}}\end{entrée}

\begin{entrée}{nɯgrɤl}{}{ⓔnɯgrɤl} 
\classe{vi}  
\grammaire{denom} \paradigme{dir}{kɤ-}
\begin{définition}\pfra{être en rang}\end{définition}
\begin{définition}\pcmn{平排}\end{définition}
\begin{exemple}\pjya{ʁmaʁmi ra ko-nɯgrɤl-nɯ}\hspace{5pt}\pcmn{士兵排成队伍了}\end{exemple}\relationsémantique{参考}{\lien{ⓔɯ-grɤl}{ɯ-grɤl}}\end{entrée}

\begin{entrée}{nɯɣɤja}{}{ⓔnɯɣɤja} 
\classe{vl} \paradigme{dir}{tɤ-}
\begin{définition}\pfra{tenir tête à}\end{définition}
\begin{définition}\pcmn{反抗;对抗;顶嘴}\end{définition}
\begin{exemple}\pjya{nɤ-mu nɤ-wa ma-tɤ-tɯ-nɯɣɤje}\hspace{5pt}\pcmn{你不要跟你父母顶嘴}\end{exemple}\relationsémantique{同义词}{\lien{ⓔnɯkhɤja}{nɯkhɤja}}\end{entrée}

\begin{entrée}{nɯɣɤtɕa}{}{ⓔnɯɣɤtɕa} 
\classe{vt} \paradigme{dir}{pɯ-}
\begin{définition}\pfra{revenir sur sa parole}\end{définition}
\begin{définition}\pcmn{违反约定}\end{définition}
\begin{exemple}\pjya{jɯfɕɯr tɯkrɤz kɯ-βdɯ-βdi tɤ-nɯ-βzu-tɕi ɕti ri, nɤʑo pjɤ-kɯ-nɯɣɤtɕa-a.}\hspace{5pt}\pcmn{我们昨天商量得好好的,但是你违反了约定}\end{exemple}\relationsémantique{同义词}{\lien{ⓔnɯɕqhu}{nɯɕqhu}}\relationsémantique{参考}{\lien{ⓔɣɤtɕa}{ɣɤtɕa}}\end{entrée}

\begin{entrée}{nɯɣbɯɣ}{}{ⓔnɯɣbɯɣ} 
\classe{vt}  
\grammaire{appl} \paradigme{dir}{nɯ-}
\begin{définition}\pfra{penser à}\end{définition}
\begin{définition}\pcmn{想念}\end{définition}
\begin{exemple}\pjya{mbarkhom ɲɯ-nɯɣbɯɣ-a}\hspace{5pt}\pcmn{我想念马尔康}\end{exemple}
\begin{exemple}\pjya{@faguo ɲɯ-nɯɣbɯɣ-a}\hspace{5pt}\pcmn{我想念法国}\end{exemple}
\begin{exemple}\pjya{ɲɯ-ta-nɯɣbɯɣ-nɯ}\hspace{5pt}\pcmn{我想念你们}\end{exemple}
\begin{sous-entrée}{anɯɣbɯɣbɯɣ}{ⓔnɯɣbɯɣⓝanɯɣbɯɣbɯɣ} 
\classe{vi} 
\begin{définition}\pfra{se manquer les uns aux autres}\end{définition}
\begin{définition}\pcmn{互相思念}\end{définition}
\begin{exemple}\pjya{ɲɯ-ɤnɯɣbɯɣbɯɣ-ndʑi}\hspace{5pt}\pcmn{他们俩互相思念}\end{exemple}\relationsémantique{参考}{\lien{ⓔbɯɣ}{bɯɣ}}\end{sous-entrée}

\end{entrée}

\begin{entrée}{nɯɣe}{}{ⓔnɯɣe} 
\classe{part} 
\begin{définition}\pfra{n'est ce pas ?}\end{définition}
\begin{définition}\pcmn{陈述自己的感觉,征求别人的看法“是吗”(只能和程度动词化名词合用)}\end{définition}
\begin{exemple}\pjya{ɯ-tɯ-pe nɯɣe}\hspace{5pt}\pcmn{很好是吗}\end{exemple}
\begin{exemple}\pjya{jisŋi tɯ-mɯ ɯ-tɯ-jɯm nɯɣe}\hspace{5pt}\pcmn{今天天气很好是吗}\end{exemple}\end{entrée}

\begin{entrée}{nɯɣi}{}{ⓔnɯɣi} 
\classe{vi}  
\grammaire{vert} \paradigme{dir}{\_}\paradigme{past stem}{nɯɣe}
\begin{définition}\pfra{revenir}\end{définition}
\begin{définition}\pcmn{回来}\end{définition}
\begin{exemple}\pjya{@shiwuhao ri lɤ-ari tɕe, pɤjkhu mɯ-thɯ-nɯɣe}\hspace{5pt}\pcmn{他十五号就去了,还没有回来}\end{exemple}
\begin{exemple}\pjya{fso tɕe chɯ-nɯɣi ɲɯ-khɯ khi}\hspace{5pt}\pcmn{他说明天就可以回来}\end{exemple}
\begin{exemple}\pjya{ʑatsa lu-nɯɣi ɕti}\hspace{5pt}\pcmn{他很快回家}\end{exemple}\relationsémantique{参考}{\lien{ⓔɣi}{ɣi}}\end{entrée}

\begin{entrée}{nɯɣɟɯ}{}{ⓔnɯɣɟɯ} 
\classe{vi} \paradigme{dir}{nɯ-}\paradigme{dir}{nɯ-}
\begin{définition}\pfra{mourir de faim}\end{définition}
\begin{définition}\pcmn{饿死}\end{définition}
\begin{définition}\pfra{faire mourir de faim}\end{définition}
\begin{définition}\pcmn{令……饿死}\end{définition}
\begin{exemple}\pjya{ɲɤ-nɯɣɟɯ}\hspace{5pt}\pcmn{他饿死了}\end{exemple}
\begin{exemple}\pjya{fsapaʁ ɲɤ-nɯɣɟɯ}\hspace{5pt}\pcmn{牲畜饿死了}\end{exemple}\relationsémantique{参考}{\lien{}{ɣɯ}}
\begin{sous-entrée}{znɯɣɟɯ}{ⓔnɯɣɟɯⓝznɯɣɟɯ} 
\classe{vt} \end{sous-entrée}

\end{entrée}

\begin{entrée}{nɯɣlɯɣli}{}{ⓔnɯɣlɯɣli} 
\classe{vi} \paradigme{dir}{kɤ-}
\begin{définition}\pfra{se frayer un chemin (dans un endroit bondé de personnes)}\end{définition}
\begin{définition}\pcmn{(从人群中)挤出来/过去}\end{définition}
\begin{exemple}\pjya{kɤ-nɯɣlɯɣli-a ʑo kɤ-nɯɬoʁ-a}\hspace{5pt}\pcmn{我(从人群中)挤出来了}\end{exemple}\end{entrée}

\begin{entrée}{nɯɣmu}{}{ⓔnɯɣmu} 
\classe{vt} \paradigme{dir}{nɯ-}
\begin{définition}\pfra{avoir peur de}\end{définition}
\begin{définition}\pcmn{害怕}\end{définition}
\begin{exemple}\pjya{khu ɲɯ-nɯɣme-a}\hspace{5pt}\pcmn{我怕老虎}\end{exemple}
\begin{exemple}\pjya{qapri ɲɯ-nɯɣme-a}\hspace{5pt}\pcmn{我怕蛇}\end{exemple}
\begin{exemple}\pjya{ɬɤndʐi ɲɯ-nɯɣme-a}\hspace{5pt}\pcmn{我怕鬼}\end{exemple}
\begin{exemple}\pjya{nɯ-nɯɣmu-t-a}\hspace{5pt}\pcmn{我怕了}\end{exemple}\relationsémantique{参考}{\lien{ⓔmuⓗ1}{mu₁}}\relationsémantique{参考}{\lien{ⓔɕɯɣmu}{ɕɯɣmu}}\relationsémantique{参考}{\lien{ⓔsɤɣmu}{sɤɣmu}}\end{entrée}

\begin{entrée}{nɯɣmaz}{}{ⓔnɯɣmaz} 
\classe{vi}  
\grammaire{denom} \paradigme{dir}{tɤ-}\paradigme{dir}{tɤ-}
\begin{définition}\pfra{se blesser}\end{définition}
\begin{définition}\pcmn{受伤}\end{définition}
\begin{définition}\pfra{blesser}\end{définition}
\begin{définition}\pcmn{弄伤}\end{définition}
\begin{exemple}\pjya{to-nɯɣmaz}\hspace{5pt}\pcmn{他受了伤}\end{exemple}
\begin{exemple}\pjya{aʑo tɤ-nɯɣmaz-a}\hspace{5pt}\pcmn{我受了伤}\end{exemple}
\begin{exemple}\pjya{a-jaʁ tɤ-nɯɣmaz}\hspace{5pt}\pcmn{我的手受伤了}\end{exemple}
\begin{exemple}\pjya{ɯ-mi to-nɯ-znɯɣmaz}\hspace{5pt}\pcmn{他不小心把脚弄伤了}\end{exemple}
\begin{exemple}\pjya{qartshaz ɕɤmɯɣdɯ kɯ tó-wɣ-znɯɣmaz}\hspace{5pt}\pcmn{鹿被枪打伤了}\end{exemple}\relationsémantique{参考}{\lien{ⓔtɯ-ɣmaz}{tɯ-ɣmaz}}
\begin{sous-entrée}{znɯɣmaz}{ⓔnɯɣmazⓝznɯɣmaz} 
\classe{vt} \end{sous-entrée}

\end{entrée}

\begin{entrée}{nɯɣmbɤβ}{}{ⓔnɯɣmbɤβ} 
\classe{vs}  
\grammaire{denom} \paradigme{dir}{thɯ-}\paradigme{dir}{thɯ-}
\begin{définition}\pfra{enfler}\end{définition}
\begin{définition}\pcmn{肿起来}\end{définition}
\begin{exemple}\pjya{βɣɤza (βɣɤrtshi) kɯ kó-wɣ-mtsɯɣ-a tɕe chɤ-nɯɣmbɤβ}\hspace{5pt}\pcmn{苍蝇(蚊子)叮了我就肿了}\end{exemple}
\begin{exemple}\pjya{a-jaʁ, a-mi chɤ-nɯɣmbɤβ}\hspace{5pt}\pcmn{我的手,我的脚肿了}\end{exemple}
\begin{exemple}\pjya{pɯ́-wɣ-ʁndɯ tɕe chɤ-nɯɣmbɤβ}\hspace{5pt}\pcmn{被打了就肿了}\end{exemple}\relationsémantique{参考}{\lien{ⓔtɯ-ɣmbɤβ}{tɯ-ɣmbɤβ}}
\begin{sous-entrée}{znɯɣmbɤβ}{ⓔnɯɣmbɤβⓝznɯɣmbɤβ} 
\classe{vt} \end{sous-entrée}

\end{entrée}

\begin{entrée}{nɯɣmɯm}{}{ⓔnɯɣmɯm} 
\classe{vs} \paradigme{dir}{tɤ-}
\begin{définition}\pfra{avoir des exigences sur la nourriture}\end{définition}
\begin{définition}\pcmn{讲究食物}\end{définition}
\begin{exemple}\pjya{tɯrme tsuku wuma ʑo kɯ-nɯɣmɯm tu}\hspace{5pt}\pcmn{有的人很讲究食物}\end{exemple}\relationsémantique{参考}{\lien{ⓔmɯm}{mɯm}}\end{entrée}

\begin{entrée}{nɯɣɲo}{}{ⓔnɯɣɲo}\relationsémantique{参考}{\lien{ⓔɲo}{ɲo}}\end{entrée}

\begin{entrée}{nɯɣur}{}{ⓔnɯɣur} 
\classe{vi}  
\grammaire{denom} \paradigme{dir}{pɯ-}
\begin{définition}\pfra{subir le gel (plante)}\end{définition}
\begin{définition}\pcmn{遭霜了}\end{définition}
\begin{exemple}\pjya{@cai pjɤ-nɯɣur}\hspace{5pt}\pcmn{菜遭霜了}\end{exemple}
\begin{exemple}\pjya{@yangyu pjɤ-nɯɣur}\hspace{5pt}\pcmn{土豆遭霜了}\end{exemple}
\begin{exemple}\pjya{tɯqe pjɯ-nɯɣur ra}\hspace{5pt}\pcmn{要经历各种艰难才懂得某些道理}\end{exemple}\relationsémantique{参考}{\lien{ⓔtɯɣur}{tɯɣur}}\end{entrée}

\begin{entrée}{nɯɣrɤβ}{}{ⓔnɯɣrɤβ} 
\classe{vt} \paradigme{dir}{\_}
\begin{définition}\pfra{tendre la main pour attraper}\end{définition}
\begin{définition}\pcmn{伸手去抓}\end{définition}
\begin{exemple}\pjya{lɤ-nɯɣraβ-a ri mɯ́j-ɕaβ-a}\hspace{5pt}\pcmn{我把手伸过去,但是够不着}\end{exemple}\end{entrée}

\begin{entrée}{nɯɣɯβzjoz}{}{ⓔnɯɣɯβzjoz}\relationsémantique{参考}{\lien{ⓔβzjoz}{βzjoz}}\end{entrée}

\begin{entrée}{nɯɣɯcɯm}{}{ⓔnɯɣɯcɯm}\relationsémantique{参考}{\lien{ⓔcɯm}{cɯm}}\end{entrée}

\begin{entrée}{nɯɣɯɕe}{}{ⓔnɯɣɯɕe} 
\classe{vs} 
\begin{définition}\pfra{praticable}\end{définition}
\begin{définition}\pcmn{方便去,好走}\end{définition}\relationsémantique{参考}{\lien{ⓔɕe}{ɕe}}\end{entrée}

\begin{entrée}{nɯɣɯɕɯftaʁ}{}{ⓔnɯɣɯɕɯftaʁ}\relationsémantique{参考}{\lien{ⓔɕɯftaʁ}{ɕɯftaʁ}}\end{entrée}

\begin{entrée}{nɯɣɯfɕɤt}{}{ⓔnɯɣɯfɕɤt}\relationsémantique{参考}{\lien{ⓔfɕɤtⓗ1}{fɕɤt₁}}\end{entrée}

\begin{entrée}{nɯɣɯfsu}{}{ⓔnɯɣɯfsu} 
\classe{vt}  
\grammaire{denom} \paradigme{dir}{tɤ-}
\begin{définition}\pfra{devenir ami}\end{définition}
\begin{définition}\pcmn{交朋友}\end{définition}
\begin{exemple}\pjya{tɤ-nɯɣɯfsu-t-a}\hspace{5pt}\pcmn{我跟他交了朋友}\end{exemple}
\begin{exemple}\pjya{jiɕqha nɯ tɯrme kɯ-pe ci ɲɯ-ŋu, tɤ-nɯɣɯfsu-t-a}\hspace{5pt}\pcmn{这个人很好,我跟他交了朋友}\end{exemple}\relationsémantique{同义词}{\lien{ⓔnɯβzaŋsa}{nɯβzaŋsa}}\relationsémantique{参考}{\lien{ⓔɣɯfsu}{ɣɯfsu}}\end{entrée}

\begin{entrée}{nɯɣɯftɯl}{}{ⓔnɯɣɯftɯl} 
\classe{vi}  
\grammaire{facil} \paradigme{dir}{pɯ-}
\begin{définition}\pfra{être facile à apprivoiser}\end{définition}
\begin{définition}\pcmn{容易驯服}\end{définition}
\begin{exemple}\pjya{mbro nɯ ɲɯ-nɯɣɯftɯl}\hspace{5pt}\pcmn{那匹马容易驯服}\end{exemple}
\begin{exemple}\pjya{jla ɲɯ-nɯɣɯftɯl}\hspace{5pt}\pcmn{犏牛容易驯服}\end{exemple}\relationsémantique{同义词}{\lien{ⓔɣɤndɯl}{ɣɤndɯl}}\relationsémantique{参考}{\lien{ⓔftɯlⓗ1}{ftɯl₁}}\end{entrée}

\begin{entrée}{nɯɣɯjmɯt}{}{ⓔnɯɣɯjmɯt}\relationsémantique{参考}{\lien{ⓔjmɯt}{jmɯt}}\end{entrée}

\begin{entrée}{nɯɣɯjpa}{}{ⓔnɯɣɯjpa} 
\classe{vs}  
\grammaire{facil} \paradigme{dir}{tɤ-}
\begin{définition}\pfra{facile, pratique}\end{définition}
\begin{définition}\pcmn{方便;好办}\end{définition}
\begin{exemple}\pjya{ʑɤŋgɯz kɯ-nɯɣɯjpa nɯ-βzu-tɕi}\hspace{5pt}\pcmn{我们互相提供方便}\end{exemple}
\begin{exemple}\pjya{tɤ-tɯt-a nɯ a-tɤ-tɯ-ste tɕe, ʑɤŋgɯz a-pɯ-nɯɣɯjpa}\hspace{5pt}\pcmn{你如果照我说的去办,你我都方便}\end{exemple}\end{entrée}

\begin{entrée}{nɯɣɯmto}{}{ⓔnɯɣɯmto} 
\classe{vs} \paradigme{dir}{tɤ-}
\begin{définition}\pfra{très visible}\end{définition}
\begin{définition}\pcmn{容易发现}\end{définition}
\begin{exemple}\pjya{tɤɣal ɲɯ-rɤʑi, ɲɯ-nɯɣɯmto}\hspace{5pt}\pcmn{很明显,容易发现}\end{exemple}\relationsémantique{参考}{\lien{ⓔmtoⓝmto}{mto}}\end{entrée}

\begin{entrée}{nɯɣɯnɤma}{}{ⓔnɯɣɯnɤma}\relationsémantique{参考}{\lien{ⓔnɤma}{nɤma}}\end{entrée}

\begin{entrée}{nɯɣɯndo}{}{ⓔnɯɣɯndo}\relationsémantique{参考}{\lien{ⓔndo}{ndo}}\end{entrée}

\begin{entrée}{nɯɣɯndza}{}{ⓔnɯɣɯndza} 
\classe{vs}  
\grammaire{facil} \paradigme{dir}{tɤ-}
\begin{définition}\pfra{bon à manger}\end{définition}
\begin{définition}\pcmn{好吃}\end{définition}
\begin{exemple}\pjya{tɤ-mthɯm ɲɯ-nɯɣɯndza, mɯ́j-nɯɣɯndza}\hspace{5pt}\pcmn{肉好吃,不好吃}\end{exemple}
\begin{exemple}\pjya{@cai ɲɯ-nɯɣɯndza, mɯ́j-nɯɣɯndza}\hspace{5pt}\pcmn{菜好吃,不好吃}\end{exemple}\relationsémantique{同义词}{\lien{ⓔmɯm}{mɯm}}\relationsémantique{参考}{\lien{ⓔndza}{ndza}}\end{entrée}

\begin{entrée}{nɯɣɯntɕhoz}{}{ⓔnɯɣɯntɕhoz} 
\classe{vs}  
\grammaire{facil} \paradigme{dir}{tɤ-}\sens{1}
\begin{définition}\pfra{facile à utiliser}\end{définition}
\begin{définition}\pcmn{好用;用起来很方便}\end{définition}
\begin{exemple}\pjya{laʁdɯn ɲɯ-nɯɣɯntɕhoz, mɯ́j-nɯɣɯntɕhoz}\hspace{5pt}\pcmn{工具好用,不好用}\end{exemple}\sens{2}
\begin{définition}\pfra{qui sait tout faire}\end{définition}
\begin{définition}\pcmn{勤快,什么都会做}\end{définition}
\begin{exemple}\pjya{jiɕqha tɯrme nɯ kɯ-nɯɣɯntɕhoz ci ɲɯ-ŋu}\hspace{5pt}\pcmn{这个人是个很勤快,什么都会做的人}\end{exemple}\relationsémantique{参考}{\lien{ⓔntɕhoz}{ntɕhoz}}\end{entrée}

\begin{entrée}{nɯɣɯŋga}{}{ⓔnɯɣɯŋga}\relationsémantique{参考}{\lien{ⓔŋga}{ŋga}}\end{entrée}

\begin{entrée}{nɯɣɯŋke}{}{ⓔnɯɣɯŋke} 
\classe{vs} \paradigme{dir}{tɤ-}
\begin{définition}\pfra{praticable (chemin)}\end{définition}
\begin{définition}\pcmn{好走(路)}\end{définition}
\begin{exemple}\pjya{kɯki tʂu ki mɯ-to-pe tɕe mɯ́j-nɯɣɯŋke}\hspace{5pt}\pcmn{这条路不好走}\end{exemple}
\begin{exemple}\pjya{tʂu to-ɣɤβdi-nɯ tɕe to-nɯɣɯŋke tɕe to-pe}\hspace{5pt}\pcmn{他们修了路以后就好走}\end{exemple}\relationsémantique{参考}{\lien{ⓔŋke}{ŋke}}\end{entrée}

\begin{entrée}{nɯɣɯphɯt}{}{ⓔnɯɣɯphɯt}\relationsémantique{参考}{\lien{ⓔphɯt}{phɯt}}\end{entrée}

\begin{entrée}{nɯɣɯqaʁ}{}{ⓔnɯɣɯqaʁ}\relationsémantique{参考}{\lien{ⓔqaʁⓗ1}{qaʁ₁}}\end{entrée}

\begin{entrée}{nɯɣɯsɤlɤɣɯ}{}{ⓔnɯɣɯsɤlɤɣɯ}\relationsémantique{参考}{\lien{ⓔsɤlɤɣɯ}{sɤlɤɣɯ}}\end{entrée}

\begin{entrée}{nɯɣɯt}{}{ⓔnɯɣɯt} 
\classe{vt}  
\grammaire{vert} \paradigme{dir}{\_}
\begin{définition}\pfra{ramener}\end{définition}
\begin{définition}\pcmn{拿回来}\end{définition}
\begin{exemple}\pjya{sɯɴɢoʁ tɤ-nɯɣɯt-a}\hspace{5pt}\pcmn{我把干柴拿回来了}\end{exemple}
\begin{exemple}\pjya{@yangyu tɤ-nɯɣɯt-a}\hspace{5pt}\pcmn{我把土豆拿回来了}\end{exemple}\relationsémantique{参考}{\lien{ⓔɣɯt}{ɣɯt}}\end{entrée}

\begin{entrée}{nɯɣɯti}{}{ⓔnɯɣɯti}\relationsémantique{参考}{\lien{ⓔti}{ti}}\end{entrée}

\begin{entrée}{nɯɣɯtshi}{}{ⓔnɯɣɯtshi}\relationsémantique{参考}{\lien{ⓔtshiⓗ1}{tshi₁}}\end{entrée}

\begin{entrée}{nɯɣɯʑɴɢu}{}{ⓔnɯɣɯʑɴɢu}\relationsémantique{参考}{\lien{ⓔʑɴɢu}{ʑɴɢu}}\end{entrée}

\begin{entrée}{nɯɣʑɯr}{}{ⓔnɯɣʑɯr} 
\classe{vi} \paradigme{dir}{tɤ-}\sens{1}
\begin{définition}\pfra{être en état d'alerte}\end{définition}
\begin{définition}\pcmn{警惕;害怕出事}\end{définition}
\begin{exemple}\pjya{ɲɯ-nɯɣʑɯr-a ɕti ma kɯmaʁ ɯβrɤ-fse ma ɲɯ-sɯsam-a}\hspace{5pt}\pcmn{我怕会出事}\end{exemple}\sens{2}\paradigme{dir}{tɤ-}
\begin{définition}\pfra{ne pas oser (manger)}\end{définition}
\begin{définition}\pcmn{不好意思(吃)}\end{définition}
\begin{exemple}\pjya{nɯnɯ ndʐuwa kɤ-rɯndzɤtshi mɯ́j-cha ma ɲɯ-nɯɣʑɯr}\hspace{5pt}\pcmn{客人不敢吃,不好意思吃}\end{exemple}\relationsémantique{同义词}{\lien{ⓔraʁle}{raʁle}}
\begin{sous-entrée}{znɯɣʑɯr}{ⓔnɯɣʑɯrⓢ2ⓝznɯɣʑɯr} 
\classe{vt} \end{sous-entrée}

\sens{1}
\begin{définition}\pfra{embarrasser}\end{définition}
\begin{définition}\pcmn{令人不好意思}\end{définition}
\begin{exemple}\pjya{nɯ kɯ-fse ma-tɤ-tɯ-ti ma tɯ-znɯɣʑɯr}\hspace{5pt}\pcmn{你不要说那些话,令他不好意思}\end{exemple}\sens{2}
\begin{définition}\pfra{faire peur}\end{définition}
\begin{définition}\pcmn{令人觉得危险;令人担心会出事}\end{définition}
\begin{exemple}\pjya{nɯ kɯ-fse paχɕi chɯ-tɯ-βʑoʁ tɕe ɲɯ-kɯ-znɯɣʑɯr-a}\hspace{5pt}\pcmn{你这样削苹果,令我害怕会出事}\end{exemple}\end{entrée}

\begin{entrée}{nɯhɯɲi}{}{ⓔnɯhɯɲi} 
\classe{vt} 
\begin{définition}\pfra{travailler}\end{définition}
\begin{définition}\pcmn{打工}\end{définition}\étymologie{fn:副业}\end{entrée}

\begin{entrée}{nɯjaŋsa}{}{ⓔnɯjaŋsa} 
\classe{vi} 
\begin{définition}\pfra{oisif}\end{définition}
\begin{définition}\pcmn{闲着}\end{définition}
\begin{exemple}\pjya{aʑo ɕ-pɯ-nɯjaŋsa-a ɕti wo}\hspace{5pt}\pcmn{我去那边闲逛}\end{exemple}\end{entrée}

\begin{entrée}{nɯjɤntɤn}{}{ⓔnɯjɤntɤn} 
\classe{vt}  
\grammaire{denom} \paradigme{dir}{pɯ-}
\begin{définition}\pfra{avoir comme passion}\end{définition}
\begin{définition}\pcmn{有这个爱好}\end{définition}
\begin{exemple}\pjya{kɤ-rɤβzjoz ntsɯ pjɯ-tɯ-nɯjɤntɤn ŋu}\hspace{5pt}\pcmn{你很专心地学习}\end{exemple}\relationsémantique{参考}{\lien{ⓔjɤntɤn}{jɤntɤn}}\étymologie{jon.tan}\end{entrée}

\begin{entrée}{nɯjɣa}{}{ⓔnɯjɣa} 
\classe{vi} \paradigme{dir}{tɤ-}
\begin{définition}\pfra{dire toujours oui}\end{définition}
\begin{définition}\pcmn{总是答应别人}\end{définition}
\begin{exemple}\pjya{kɯ-nɯjɣa ci ɲɯ-ŋu}\hspace{5pt}\pcmn{他是一个总是答应别人的人}\end{exemple}\relationsémantique{参考}{\lien{ⓔɣa}{ɣa}}\end{entrée}

\begin{entrée}{nɯjlɤlɤɣ}{}{ⓔnɯjlɤlɤɣ} 
\classe{vi}  
\grammaire{incorp} \paradigme{dir}{nɯ-}\paradigme{dir}{pɯ-}
\begin{définition}\pfra{faire paître un yak hybride}\end{définition}
\begin{définition}\pcmn{放犏牛}\end{définition}
\begin{exemple}\pjya{nɯ-nɯjlɤlaɣ-a}\hspace{5pt}\pcmn{我放了犏牛}\end{exemple}
\begin{exemple}\pjya{kɯ-nɯjlɤlɤɣ lɤ-ari-a}\hspace{5pt}\pcmn{我去放犏牛了}\end{exemple}
\begin{exemple}\pjya{kɯ-nɯjlɤlɤɣ lo-ɕe}\end{exemple}\relationsémantique{参考}{\lien{ⓔjla}{jla}}\relationsémantique{参考}{\lien{ⓔlɤɣ}{lɤɣ}}\end{entrée}

\begin{entrée}{nɯjlɤmtshi}{}{ⓔnɯjlɤmtshi} 
\classe{vi}  
\grammaire{incorp} 
\begin{définition}\pfra{mener un yak hybride (pendant le labour)}\end{définition}
\begin{définition}\pcmn{牵犏牛(耕地的时候)}\end{définition}
\begin{exemple}\pjya{tɤ-nɯjlɤmtshi-a}\hspace{5pt}\pcmn{我牵了犏牛}\end{exemple}
\begin{exemple}\pjya{tɤ-pɤtso pɯ-ŋu-a tɕe, kɤ-nɯjlɤmtshi pɯ-rɲo-t-a}\hspace{5pt}\pcmn{我小的时候,我曾经牵过犏牛}\end{exemple}\relationsémantique{参考}{\lien{ⓔjlɤmtshi}{jlɤmtshi}}\end{entrée}

\begin{entrée}{nɯjlɤndza/\variante{nɤjlɤndza}}{₁}{ⓔnɯjlɤndzaⓗ1} 
\classe{vt} \paradigme{dir}{thɯ-}\paradigme{dir}{tɤ-}
\begin{définition}\pfra{aimer manger des petites collations}\end{définition}
\begin{définition}\pcmn{爱吃零食}\end{définition}
\begin{exemple}\pjya{ɲɯ-ɤz-nɯjlɤndza}\hspace{5pt}\pcmn{他在吃零食}\end{exemple}
\begin{exemple}\pjya{nɤʑo kɯchi ɲɯ-tɯ-ɤz-nɯjlɤndza}\hspace{5pt}\pcmn{你在吃糖}\end{exemple}
\begin{exemple}\pjya{@guazi thɯ-nɯjlɤndza-t-a}\hspace{5pt}\pcmn{我吃了瓜子}\end{exemple}
\begin{exemple}\pjya{kɯchi tɤ-nɯjlɤndza-t-a}\hspace{5pt}\pcmn{我吃了糖}\end{exemple}\relationsémantique{参考}{\lien{ⓔndza}{ndza}}\end{entrée}

\begin{entrée}{nɯjlɤndza}{₂}{ⓔnɯjlɤndzaⓗ2} 
\classe{vi} 
\begin{définition}\pfra{couper de l'herbe pour les yaks hybrides}\end{définition}
\begin{définition}\pcmn{割牛草}\end{définition}
\begin{exemple}\pjya{kɯ-nɯjlɤndza jɤ-ari-a}\hspace{5pt}\pcmn{我去割牛草了}\end{exemple}\relationsémantique{同义词}{\lien{ⓔɣɤxtɕɤβ}{ɣɤxtɕɤβ}}\end{entrée}

\begin{entrée}{nɯjmɤzdɤβ}{}{ⓔnɯjmɤzdɤβ} 
\classe{vi}  
\grammaire{incorp} \paradigme{dir}{tɤ-}
\begin{définition}\pfra{dormir dans une direction inverse}\end{définition}
\begin{définition}\pcmn{打脚蹬(朝相反的方向睡,交叉着脚)}\end{définition}
\begin{exemple}\pjya{ʑɤni to-nɯjmɤzdɤβ-ndʑi}\hspace{5pt}\pcmn{他们俩朝相反的方向睡了}\end{exemple}\relationsémantique{参考}{\lien{ⓔzdɤβ}{zdɤβ}}\end{entrée}

\begin{entrée}{nɯjmŋo}{}{ⓔnɯjmŋo} 
\classe{vi}  
\grammaire{denom} \paradigme{dir}{pɯ-}
\begin{définition}\pfra{être l'objet du rêve de quelqu'un}\end{définition}
\begin{définition}\pcmn{出现在别人的梦中}\end{définition}
\begin{exemple}\pjya{jɯfɕɯɕɤr pɯ-ta-ɣɤjmŋo tɕe ɲɯ-tɯ-nɯjmŋo}\hspace{5pt}\pcmn{昨天我梦见你了(是好兆头)}\end{exemple}
\begin{exemple}\pjya{mɯ́j-nɯjmŋo-nɯ}\hspace{5pt}\pcmn{梦见他们是不好的兆头}\end{exemple}\relationsémantique{参考}{\lien{ⓔtɯ-jmŋo}{tɯ-jmŋo}}\relationsémantique{参考}{\lien{ⓔɣɤjmŋo}{ɣɤjmŋo}}\end{entrée}

\begin{entrée}{nɯjroʁ}{}{ⓔnɯjroʁ} 
\classe{vt}  
\grammaire{denom} \paradigme{dir}{tɤ-}
\begin{définition}\pfra{suivre à la trace}\end{définition}
\begin{définition}\pcmn{追踪}\end{définition}
\begin{exemple}\pjya{jɯfɕɯr, pri ci tɤ-nɯɕɤmɯɣdɯ-t-a tɕe jɤ-anɯri tɕe tɤ-nɯjroʁ-a}\hspace{5pt}\pcmn{昨天我射了一头熊,它逃走了,我追踪了它}\end{exemple}\relationsémantique{参考}{\lien{ⓔtɤ-jroʁ}{tɤ-jroʁ}}\relationsémantique{参考}{\lien{ⓔrɤjroʁ}{rɤjroʁ}}\end{entrée}

\begin{entrée}{nɯjʁo}{}{ⓔnɯjʁo} 
\classe{vi} \paradigme{dir}{tɤ-}\paradigme{dir}{tɤ-}
\begin{définition}\pfra{insulter, gronder}\end{définition}
\begin{définition}\pcmn{骂}\end{définition}
\begin{définition}\pfra{insulter}\end{définition}
\begin{définition}\pcmn{乱骂}\end{définition}
\begin{exemple}\pjya{ɲɯ-tɯ-nɯjʁo}\hspace{5pt}\pcmn{你在骂人}\end{exemple}
\begin{exemple}\pjya{tɤ-nɯjʁo-a}\hspace{5pt}\pcmn{我骂了他}\end{exemple}
\begin{exemple}\pjya{ma-tɤ-tɯ-nɯjʁo}\hspace{5pt}\pcmn{你不要骂(我)}\end{exemple}\relationsémantique{同义词}{\lien{ⓔnɤmqe}{nɤmqe}}
\begin{sous-entrée}{nɯjʁojʁe}{ⓔnɯjʁoⓝnɯjʁojʁe} 
\classe{vi} \end{sous-entrée}

\end{entrée}

\begin{entrée}{nɯjʁojʁe}{}{ⓔnɯjʁojʁe}\relationsémantique{参考}{\lien{ⓔnɯjʁo}{nɯjʁo}}\end{entrée}

\begin{entrée}{nɯjtʂhɤβ}{}{ⓔnɯjtʂhɤβ} 
\classe{vt} \paradigme{dir}{pɯ-}\sens{1}
\begin{définition}\pfra{érafler avec force}\end{définition}
\begin{définition}\pcmn{使劲地刮}\end{définition}
\begin{exemple}\pjya{ɯ-ŋga ra pjɯ-nɯjtʂhɤβ ʑo ju-rɤɕi pjɤ-ŋu}\hspace{5pt}\pcmn{把它的衣服使劲地刮破了}\end{exemple}\sens{2}
\begin{définition}\pfra{abîmer en griffant}\end{définition}
\begin{définition}\pcmn{抓烂}\end{définition}
\begin{exemple}\pjya{lɯlu tɯ-mɯrʁɯz kɯ pjɯ-kɯ-nɯjtʂhɤβ ʑo ŋgrɤl}\hspace{5pt}\pcmn{猫会在身上乱抓(把衣服和皮肤抓烂)}\end{exemple}\end{entrée}

\begin{entrée}{nɯɟɯɣɟɯɣ}{}{ⓔnɯɟɯɣɟɯɣ}\relationsémantique{参考}{\lien{ⓔɟɯɣɟɯɣ}{ɟɯɣɟɯɣ}}\end{entrée}

\begin{entrée}{nɯkɤntɕhaʁ}{}{ⓔnɯkɤntɕhaʁ} 
\classe{vi}  
\grammaire{denom} \paradigme{dir}{pɯ-}
\begin{définition}\pfra{aller dans la rue}\end{définition}
\begin{définition}\pcmn{上街}\end{définition}
\begin{exemple}\pjya{jiɕqha jiʑora @jieshang ɕ-pɯ-nɯkɤntɕhaʁ-i}\hspace{5pt}\pcmn{我们刚才上街了}\end{exemple}
\begin{exemple}\pjya{@chenlaoshi cho jiʑora pɯ-nɯkɤntɕhaʁ-i}\hspace{5pt}\pcmn{我们跟陈老师上街了}\end{exemple}\relationsémantique{参考}{\lien{ⓔkɤntɕhaʁ}{kɤntɕhaʁ}}\end{entrée}

\begin{entrée}{nɯkɤrŋi}{}{ⓔnɯkɤrŋi} 
\classe{vi}  
\grammaire{denom} \paradigme{dir}{nɯ-}\paradigme{dir}{pɯ-}
\begin{définition}\pfra{aller chercher des herbes sauvages}\end{définition}
\begin{définition}\pcmn{去采集野菜}\end{définition}
\begin{exemple}\pjya{sɯŋgɯ ʑ-nɯ-nɯkɤrŋi-a}\hspace{5pt}\pcmn{我到森林去采集野菜了}\end{exemple}\relationsémantique{参考}{\lien{ⓔarŋi}{arŋi}}\end{entrée}

\begin{entrée}{nɯkhamu}{}{ⓔnɯkhamu} 
\classe{vi}  
\grammaire{denom} \paradigme{dir}{tɤ-}
\begin{définition}\pfra{faire à manger}\end{définition}
\begin{définition}\pcmn{做饭}\end{définition}
\begin{exemple}\pjya{jɯfɕɯr pɯ-nɯkhamu-a}\hspace{5pt}\pcmn{我昨天做了饭}\end{exemple}
\begin{exemple}\pjya{jisŋi nɤʑo tɤ-nɯkhamu}\hspace{5pt}\pcmn{你今天做饭吧}\end{exemple}
\begin{exemple}\pjya{aʑo nɯkhamu-a ra}\hspace{5pt}\pcmn{我要做饭了}\end{exemple}\relationsémantique{参考}{\lien{ⓔkhamu}{khamu}}\end{entrée}

\begin{entrée}{nɯkhaŋrcɤl}{}{ⓔnɯkhaŋrcɤl} 
\classe{vi} \paradigme{dir}{thɯ-}\paradigme{dir}{thɯ-}
\begin{définition}\pfra{les quatre fers en l'air}\end{définition}
\begin{définition}\pcmn{四脚朝天}\end{définition}
\begin{définition}\pfra{mettre les quatre fers en l'air}\end{définition}
\begin{définition}\pcmn{让……四脚朝天}\end{définition}
\begin{exemple}\pjya{ɯʑo pɯ-znɤjpɯjpe ri, ɯʑo sɤz kɯ-cha ra jo-ɣi-nɯ tɕe chɤ́-wɣ-znɯkhaŋrcɤl ɕti}\hspace{5pt}\pcmn{他以前很傲慢,但是他遇见了比自己厉害的人,挫了他的傲气}\end{exemple}
\begin{sous-entrée}{znɯkhaŋrcɤl}{ⓔnɯkhaŋrcɤlⓝznɯkhaŋrcɤl} 
\classe{vt} \end{sous-entrée}

\end{entrée}

\begin{entrée}{nɯkhaŋχɯ}{}{ⓔnɯkhaŋχɯ} 
\classe{vi} \paradigme{dir}{thɯ-}
\begin{définition}\pfra{être accroupi}\end{définition}
\begin{définition}\pcmn{蹲}\end{définition}
\begin{exemple}\pjya{ma-thɯ-tɯ-nɯkhaŋχɯ}\hspace{5pt}\pcmn{你不要蹲下(没有礼貌)}\end{exemple}\end{entrée}

\begin{entrée}{nɯkharwut}{}{ⓔnɯkharwut} 
\classe{vi}  
\grammaire{denom} \paradigme{dir}{tɤ-}
\begin{définition}\pfra{avoir la fièvre aphteuse}\end{définition}
\begin{définition}\pcmn{得口蹄疫}\end{définition}
\begin{exemple}\pjya{jla ɲɯ-nɯkharwut}\hspace{5pt}\pcmn{犏牛有口蹄疫}\end{exemple}\relationsémantique{参考}{\lien{ⓔkharwut}{kharwut}}\end{entrée}

\begin{entrée}{nɯkhɤβdɤr}{}{ⓔnɯkhɤβdɤr} 
\classe{vt}  
\grammaire{denom} \paradigme{dir}{tɤ-}
\begin{définition}\pfra{dire des blagues}\end{définition}
\begin{définition}\pcmn{开玩笑;讲笑话}\end{définition}
\begin{exemple}\pjya{jɯfɕɯr tɤ-kɯ-nɯkhɤβdar-a}\hspace{5pt}\pcmn{你昨天跟我开了玩笑}\end{exemple}
\begin{exemple}\pjya{tɤ-ta-nɯkhɤβdɤr ɕti ma a-stu maʁ}\hspace{5pt}\pcmn{我只跟你开了玩笑}\end{exemple}\relationsémantique{参考}{\lien{ⓔkhɤβdɤr}{khɤβdɤr}}\end{entrée}

\begin{entrée}{nɯkhɤβɣa}{}{ⓔnɯkhɤβɣa} 
\classe{vi} 
\begin{définition}\pfra{ne pas rester en place, aller à droite et à gauche (oiseau)}\end{définition}
\begin{définition}\pcmn{在地面来回转动(鸟)}\end{définition}
\begin{exemple}\pjya{qro ɲɯ-nɯkhɤβɣa}\hspace{5pt}\pcmn{鸽子在来回转动}\end{exemple}
\begin{exemple}\pjya{nɤ-stu ku-kɤ-rɤʑi maŋe tɕe qro kɯ-nɯkhɤβɣa ʑo ɲɯ-tɯ-fse}\hspace{5pt}\pcmn{你坐好,不要坐立不安}\end{exemple}\end{entrée}

\begin{entrée}{nɯkhɤda}{}{ⓔnɯkhɤda} 
\classe{vt} \paradigme{dir}{nɯ-}
\begin{définition}\pfra{convaincre, calmer, raisonner qqn}\end{définition}
\begin{définition}\pcmn{劝说}\end{définition}
\begin{exemple}\pjya{ɲɯ-tɯ-anɯmqaj-ndʑi tɕe nɯ-ta-nɯkhɤda}\hspace{5pt}\pcmn{你们俩吵架了,我劝了你一下了}\end{exemple}\relationsémantique{反义词}{\lien{ⓔɣɤɕphɤr}{ɣɤɕphɤr}}\end{entrée}

\begin{entrée}{nɯkhɤja}{}{ⓔnɯkhɤja} 
\classe{vl} \paradigme{dir}{tɤ-}
\begin{définition}\pfra{tenir tête à; répondre de façon insolente}\end{définition}
\begin{définition}\pcmn{顶嘴}\end{définition}
\begin{exemple}\pjya{a-mu tɤ-nɯkhɤja-t-a}\hspace{5pt}\pcmn{我跟我母亲顶嘴了}\end{exemple}
\begin{exemple}\pjya{a-wa tɤ-nɯkhɤja-t-a}\hspace{5pt}\pcmn{我跟我父亲顶嘴了}\end{exemple}
\begin{exemple}\pjya{nɤʑo ɲɯ-tɯ-nɯkhɤja}\hspace{5pt}\pcmn{你在顶嘴}\end{exemple}
\begin{exemple}\pjya{nɤʑo ma-tɤ-kɯ-nɯkhɤja-a}\hspace{5pt}\pcmn{你不要跟我顶嘴}\end{exemple}\relationsémantique{同义词}{\lien{ⓔnɯɣɤja}{nɯɣɤja}}\end{entrée}

\begin{entrée}{nɯkhɤjlɤn}{}{ⓔnɯkhɤjlɤn} 
\classe{vt}  
\grammaire{denom} \paradigme{dir}{tɤ-}
\begin{définition}\pfra{faire un souhait}\end{définition}
\begin{définition}\pcmn{许愿}\end{définition}
\begin{exemple}\pjya{tɤ-nɯkhɤjlan-a}\hspace{5pt}\pcmn{我许愿了}\end{exemple}
\begin{exemple}\pjya{ɬasa tu-ɕe-a nɯ-sɯso-t-a ri, mɯ-pɯ-ŋgrɯ tɕe tɤ-nɯkhɤjlan-a}\hspace{5pt}\pcmn{我想过要去拉萨,没有去成,但是我许了愿将来一定会去}\end{exemple}\relationsémantique{参考}{\lien{ⓔkhɤjlɤn}{khɤjlɤn}}\end{entrée}

\begin{entrée}{nɯkhɤlɤmdzɯmdzɯ}{}{ⓔnɯkhɤlɤmdzɯmdzɯ} 
\classe{vi} \paradigme{dir}{pɯ-}\paradigme{dir}{thɯ-}
\begin{définition}\pfra{s'accroupir}\end{définition}
\begin{définition}\pcmn{蹲}\end{définition}
\begin{exemple}\pjya{ɲɯ-tɯ-nɯkhɤlɤmdzɯmdzɯ}\hspace{5pt}\pcmn{你是蹲着的}\end{exemple}\relationsémantique{同义词}{\lien{ⓔnɯkhaŋχɯ}{nɯkhaŋχɯ}}\relationsémantique{参考}{\lien{ⓔamdzɯ}{amdzɯ}}\end{entrée}

\begin{entrée}{nɯkhɤphrɯ}{}{ⓔnɯkhɤphrɯ} 
\classe{vt} \paradigme{dir}{tɤ-}
\begin{définition}\pfra{asperger (avec la bouche)}\end{définition}
\begin{définition}\pcmn{喷(水)}\end{définition}
\begin{exemple}\pjya{tɯ-ndʐi to-khrɯ tɕe, tɤ-nɯkhɤphrɯ-t-a tɕe nɯ-ɣɤla-t-a}\hspace{5pt}\pcmn{皮子干了,我喷了一下口水就令它湿润}\end{exemple}\relationsémantique{参考}{\lien{ⓔkhɤphrɯ}{khɤphrɯ}}\end{entrée}

\begin{entrée}{nɯkhɤrŋgɯ}{}{ⓔnɯkhɤrŋgɯ} 
\classe{vi}  
\grammaire{incorp} \paradigme{dir}{lɤ-}
\begin{définition}\pfra{s'allonger n'importe où pour se reposer}\end{définition}
\begin{définition}\pcmn{随便躺在某个地方休息}\end{définition}\relationsémantique{参考}{\lien{ⓔkhɤjmu}{khɤjmu}}\relationsémantique{参考}{\lien{ⓔrŋgɯⓗ2}{rŋgɯ}}\end{entrée}

\begin{entrée}{nɯkhɤsnɯm}{}{ⓔnɯkhɤsnɯm} 
\classe{vt} \paradigme{}{thɯ-}
\begin{définition}\pfra{mouiller avec de la salive}\end{définition}
\begin{définition}\pcmn{用口水弄湿}\end{définition}\relationsémantique{参考}{\lien{ⓔkhɤsnɯm}{khɤsnɯm}}\end{entrée}

\begin{entrée}{nɯkho}{}{ⓔnɯkho} 
\classe{vi}  
\grammaire{denom} \paradigme{dir}{tɤ-}\paradigme{dir}{tɤ-}
\begin{définition}\pfra{passer la nuit chez quelqu'un}\end{définition}
\begin{définition}\pcmn{借宿}\end{définition}
\begin{définition}\pfra{inviter chez soi pour la nuit}\end{définition}
\begin{définition}\pcmn{请人留宿}\end{définition}
\begin{exemple}\pjya{jɯfɕɯr ji-kɯ-nɯkho ʁnɯz pɯ-tu, χsɯm pɯ-tu}\hspace{5pt}\pcmn{昨天我们家有两三个客人}\end{exemple}
\begin{exemple}\pjya{tɤ-nɯkho-a}\hspace{5pt}\pcmn{我在他家借宿了}\end{exemple}
\begin{exemple}\pjya{tɤ-znɯkho-t-a}\hspace{5pt}\pcmn{我请他留宿了}\end{exemple}
\begin{exemple}\pjya{aʑo ci tɯ-rʑaʁ tu-kɯ-znɯkho-a-nɯ ɯ́-jɤɣ?}\hspace{5pt}\pcmn{请问能否让我借宿一晚?}\end{exemple}\relationsémantique{参考}{\lien{ⓔkhoⓗ2}{kho}}
\begin{sous-entrée}{znɯkho}{ⓔnɯkhoⓝznɯkho} 
\classe{vt}  
\grammaire{caus} \end{sous-entrée}

\end{entrée}

\begin{entrée}{nɯkhramba}{}{ⓔnɯkhramba} 
\classe{vt}  
\grammaire{denom}
\grammaire{refl} \paradigme{dir}{tɤ-}\paradigme{dir}{tɤ-}\paradigme{dir}{tɤ-}
\begin{définition}\pfra{tromper, mentir à quelqu'un}\end{définition}
\begin{définition}\pcmn{欺骗;撒谎}\end{définition}
\begin{définition}\pfra{tromper les gens}\end{définition}
\begin{définition}\pcmn{骗别人}\end{définition}
\begin{exemple}\pjya{jɯfɕɯr tɤ-ta-nɯkhramba}\hspace{5pt}\pcmn{我昨天骗了你}\end{exemple}
\begin{exemple}\pjya{nɤʑo ma-tɤ-kɯ-nɯkhramba-a}\hspace{5pt}\pcmn{你不要骗我}\end{exemple}
\begin{sous-entrée}{sɤnɯkhramba}{ⓔnɯkhrambaⓝsɤnɯkhramba} 
\classe{vi}  
\grammaire{apass} \end{sous-entrée}

\begin{sous-entrée}{ʑɣɤnɯkhramba}{ⓔnɯkhrambaⓝʑɣɤnɯkhramba} 
\classe{vi} \end{sous-entrée}

\begin{définition}\pfra{être trompé}\end{définition}
\begin{définition}\pcmn{被骗}\end{définition}\relationsémantique{参考}{\lien{ⓔkhramba}{khramba}}\relationsémantique{参考}{\lien{ⓔrɯkhramba}{rɯkhramba}}\end{entrée}

\begin{entrée}{nɯkhrɯɣ}{}{ⓔnɯkhrɯɣ} 
\classe{vt} \paradigme{dir}{nɯ-}
\begin{définition}\pfra{accrocher et déchirer}\end{définition}
\begin{définition}\pcmn{钩住;钩破}\end{définition}
\begin{exemple}\pjya{kɤ-tɯ-ari nɤ ɲɯ-kɯ-nɯkhrɯɣ-a, nɯ-tɯ-ɣe nɤ ɲɯ-kɯ-nɯkhrɯɣ-a}\hspace{5pt}\pcmn{你过去就把我的衣服钩破,过来也把我的衣服钩破}\end{exemple}
\begin{exemple}\pjya{si kɯ a-ŋga na-nɯkhrɯɣ}\hspace{5pt}\pcmn{我的衣服被树钩破了}\end{exemple}\end{entrée}

\begin{entrée}{nɯkhrɯm}{}{ⓔnɯkhrɯm} 
\classe{vi}  
\grammaire{denom} \paradigme{dir}{kɤ-}\sens{1}
\begin{définition}\pfra{être puni, recevoir un châtiment}\end{définition}
\begin{définition}\pcmn{受罚}\end{définition}\sens{2}
\begin{définition}\pfra{aller en prison}\end{définition}
\begin{définition}\pcmn{坐牢}\end{définition}\paradigme{dir}{kɤ-}
\begin{définition}\pfra{infliger une punition, châtier}\end{définition}
\begin{définition}\pcmn{惩罚,用刑}\end{définition}
\begin{exemple}\pjya{ɲɯ-nɯkhrɯm-nɯ}\hspace{5pt}\pcmn{他们在坐牢}\end{exemple}
\begin{sous-entrée}{znɯkhrɯm}{ⓔnɯkhrɯmⓝznɯkhrɯm} 
\classe{vt}  
\grammaire{caus} \end{sous-entrée}

\begin{sous-entrée}{sɤznɯkhrɯm}{ⓔnɯkhrɯmⓝsɤznɯkhrɯm} 
\classe{vi} 
\begin{définition}\pfra{infliger des punitions}\end{définition}
\begin{définition}\pcmn{惩罚人}\end{définition}
\begin{exemple}\pjya{kɯ-znɯkhrɯm}\hspace{5pt}\pcmn{刽子手}\end{exemple}\end{sous-entrée}

\étymologie{kʰrims}\end{entrée}

\begin{entrée}{nɯkhɯɣ}{}{ⓔnɯkhɯɣ} 
\classe{vt} \paradigme{div}{kɤ-}
\begin{définition}\pfra{boire à longs traits}\end{définition}
\begin{définition}\pcmn{大口大口地喝(很急的样子)}\end{définition}
\begin{exemple}\pjya{tɯ-ci ko-nɯkhɯɣ ʑo ko-tshi}\hspace{5pt}\pcmn{他大口大口地喝了水}\end{exemple}\relationsémantique{同义词}{\lien{ⓔnɯchɯβ}{nɯchɯβ}}\end{entrée}

\begin{entrée}{nɯkhɯr}{}{ⓔnɯkhɯr} 
\classe{vt}  
\grammaire{denom} \paradigme{dir}{pɯ-}
\begin{définition}\pfra{commander, gérer}\end{définition}
\begin{définition}\pcmn{担当;管理}\end{définition}
\begin{exemple}\pjya{aʑo @daduizhang pɯ-az-nɯkhɯr-a}\hspace{5pt}\pcmn{我以前当大队长}\end{exemple}\étymologie{kʰur}\end{entrée}

\begin{entrée}{nɯkhɯrthaŋ}{}{ⓔnɯkhɯrthaŋ} 
\classe{vi}  
\grammaire{denom} \paradigme{dir}{tɤ-}
\begin{définition}\pfra{occuper un poste}\end{définition}
\begin{définition}\pcmn{当官}\end{définition}
\begin{exemple}\pjya{χsɯ-xpa pɯ-nɯkhɯrthaŋ-a}\hspace{5pt}\pcmn{我当了三年官}\end{exemple}\end{entrée}

\begin{entrée}{nɯkhɯrwum}{}{ⓔnɯkhɯrwum} 
\classe{vi}  
\grammaire{denom} \paradigme{dir}{nɯ-}
\begin{définition}\pfra{moisir}\end{définition}
\begin{définition}\pcmn{发霉}\end{définition}
\begin{exemple}\pjya{kɤ-ndza ɲɤ-nɯkhɯrwum}\hspace{5pt}\pcmn{食物发霉了}\end{exemple}
\begin{exemple}\pjya{nɯ-kɯ-nɯkhɯrwum kɤ-ndza mɤ-sna}\hspace{5pt}\pcmn{发霉的食物不能吃}\end{exemple}\relationsémantique{参考}{\lien{ⓔkhɯrwum}{khɯrwum}}\end{entrée}

\begin{entrée}{nɯkumbrɤl}{}{ⓔnɯkumbrɤl} 
\classe{vi} \paradigme{dir}{pɯ-}
\begin{définition}\pfra{jouer aux échecs}\end{définition}
\begin{définition}\pcmn{下棋}\end{définition}
\begin{exemple}\pjya{pjɤ-nɯkumbrɤl}\hspace{5pt}\pcmn{他下棋了}\end{exemple}\relationsémantique{参考}{\lien{ⓔkumbrɤl}{kumbrɤl}}\end{entrée}

\begin{entrée}{nɯkon}{}{ⓔnɯkon} 
\classe{vt}  
\grammaire{denom} \paradigme{dir}{tɤ-}
\begin{définition}\pfra{gérer, s’occuper de}\end{définition}
\begin{définition}\pcmn{管理}\end{définition}
\begin{exemple}\pjya{thamtham aʑo @linyegongzuo ku-oz-nɯkon-a}\hspace{5pt}\pcmn{我现在管理林业工作}\end{exemple}
\begin{définition}\pfra{ne se soucier que de soi}\end{définition}
\begin{définition}\pcmn{只顾自己}\end{définition}
\begin{sous-entrée}{znɯkon}{ⓔnɯkonⓝznɯkon} 
\classe{vt} 
\begin{définition}\pfra{pouvoir contrôler}\end{définition}
\begin{définition}\pcmn{能管住}\end{définition}
\begin{exemple}\pjya{iɕqha tɤ-pɤtso nɯ ɯ-tɯ-li kɯ ɯ-kɯ-znɯkon me}\hspace{5pt}\pcmn{这个小孩子很调皮,没有管得住他}\end{exemple}\end{sous-entrée}

\begin{sous-entrée}{ʑɣɤnɯkon}{ⓔnɯkonⓝʑɣɤnɯkon} 
\classe{vi}  
\grammaire{refl} \paradigme{dir}{tɤ-}\end{sous-entrée}

\étymologie{fn:管}\end{entrée}

\begin{entrée}{nɯkoŋ}{}{ⓔnɯkoŋ} 
\classe{vs}  
\grammaire{denom}
\grammaire{denom} \paradigme{dir}{tɤ-}
\begin{définition}\pfra{cher}\end{définition}
\begin{définition}\pcmn{贵}\end{définition}
\begin{exemple}\pjya{laχtɕha ɲɯ-nɯkoŋ, mɯ́j-nɯkoŋ}\hspace{5pt}\pcmn{东西很便宜,不便宜}\end{exemple}\étymologie{goŋ}\end{entrée}

\begin{entrée}{nɯkowa}{}{ⓔnɯkowa} 
\classe{vt}  
\grammaire{denom} \paradigme{dir}{tɤ-}
\begin{définition}\pfra{préparer}\end{définition}
\begin{définition}\pcmn{准备;想办法}\end{définition}
\begin{exemple}\pjya{tɤ-nɯkowa-t-a}\hspace{5pt}\pcmn{我准备了}\end{exemple}
\begin{exemple}\pjya{ku-oz-nɯkowa-a}\hspace{5pt}\pcmn{我正在准备}\end{exemple}
\begin{exemple}\pjya{kha kɤ-βzu tɤ-nɯkowa-t-a}\hspace{5pt}\pcmn{我准备修房子了}\end{exemple}
\begin{exemple}\pjya{kɤ-rɤβzjoz tɤ-nɯkowa-ta}\hspace{5pt}\pcmn{我做了读书的准备}\end{exemple}
\begin{exemple}\pjya{ki tɤ-nɯkowa-t-a}\hspace{5pt}\pcmn{我想了这个办法}\end{exemple}
\begin{exemple}\pjya{tshi tsuku ʑo to-nɯkowa ri mɯ-pjɤ-cha}\hspace{5pt}\pcmn{他想尽办法,但是没有成功}\end{exemple}\relationsémantique{参考}{\lien{ⓔkowa}{kowa}}\relationsémantique{同义词}{\lien{ⓔnɯftɕaka}{nɯftɕaka}}\relationsémantique{同义词}{\lien{}{mɲo}}\relationsémantique{参考}{\lien{ⓔkowa}{kowa}}\end{entrée}

\begin{entrée}{nɯkrɤlma}{}{ⓔnɯkrɤlma} 
\classe{vi} \paradigme{dir}{tɤ-}
\begin{définition}\pfra{attraper une maladie de l'intestin}\end{définition}
\begin{définition}\pcmn{拉肚子(孩子)}\end{définition}
\begin{exemple}\pjya{tɤ-pɤtso ɲɯ-nɯkrɤlma}\hspace{5pt}\pcmn{小孩子在拉肚子}\end{exemple}\end{entrée}

\begin{entrée}{nɯkrɤz}{}{ⓔnɯkrɤz} 
\classe{vt}  
\grammaire{denom} \paradigme{dir}{tɤ-}
\begin{définition}\pfra{discuter}\end{définition}
\begin{définition}\pcmn{商量}\end{définition}
\begin{exemple}\pjya{kɤ-nɯkhɤjhwi pɯ-az-nɯkrɤz-tɕi}\hspace{5pt}\pcmn{我们俩在商量开会的事}\end{exemple}
\begin{exemple}\pjya{ɯ-koŋ ta-nɯkrɤz-nɯ}\hspace{5pt}\pcmn{他们商量了价钱}\end{exemple}
\begin{exemple}\pjya{tɤ-nɯkrɤz-tɕi}\hspace{5pt}\pcmn{我们俩商量了}\end{exemple}
\begin{exemple}\pjya{ju-kɤ-ɕe to-nɯkrɤz-nɯ}\hspace{5pt}\pcmn{他们商量了要不要去}\end{exemple}\relationsémantique{参考}{\lien{ⓔtɯkrɤz}{tɯkrɤz}}\relationsémantique{参考}{\lien{ⓔrɤkrɤz}{rɤkrɤz}}\end{entrée}

\begin{entrée}{nɯkro}{}{ⓔnɯkro}\relationsémantique{参考}{\lien{ⓔkro}{kro}}\end{entrée}

\begin{entrée}{nɯkrɯβ}{}{ⓔnɯkrɯβ} 
\classe{vi} \paradigme{dir}{pɯ-}\paradigme{dir}{pɯ-}
\begin{définition}\pfra{tomber malade à cause de nourriture avariée}\end{définition}
\begin{définition}\pcmn{食物中毒}\end{définition}
\begin{définition}\pfra{rendre malade (nourriture avariée)}\end{définition}
\begin{définition}\pcmn{使中毒}\end{définition}
\begin{exemple}\pjya{tɤ-mthɯm ɲɤ-ɣɤdi tɕe pɯ́-wɣ-znɯkrɯβ-a}\hspace{5pt}\pcmn{肉坏了,令我生病了}\end{exemple}
\begin{exemple}\pjya{kɤ-ndza kɯ pjɤ́-wɣ-znɯkrɯβ-a}\hspace{5pt}\pcmn{食物把我吃生病}\end{exemple}
\begin{sous-entrée}{znɯkrɯβ}{ⓔnɯkrɯβⓝznɯkrɯβ} 
\classe{vt} \end{sous-entrée}

\end{entrée}

\begin{entrée}{nɯkɯɕnom}{}{ⓔnɯkɯɕnom} 
\classe{vi}  
\grammaire{denom} \paradigme{dir}{pɯ-}
\begin{définition}\pfra{ramasser les épis tombés sur le sol après la récolte}\end{définition}
\begin{définition}\pcmn{收割后捡地上的青稞穗}\end{définition}
\begin{exemple}\pjya{tɤɕi kɤ-phɯt-i tɕe, pɯ-nɯkɯɕnom-a}\hspace{5pt}\pcmn{我们收割的时候,我捡了青稞穗}\end{exemple}\relationsémantique{参考}{\lien{ⓔkɯɕnom}{kɯɕnom}}\end{entrée}

\begin{entrée}{nɯkɯjŋu}{}{ⓔnɯkɯjŋu} 
\classe{vt} \paradigme{dir}{tɤ-}\paradigme{dir}{nɯ-}
\begin{définition}\pfra{jurer}\end{définition}
\begin{définition}\pcmn{发誓}\end{définition}
\begin{exemple}\pjya{ɲɤ-nɯkɯjŋu}\hspace{5pt}\pcmn{他立下誓言}\end{exemple}
\begin{exemple}\pjya{tɤ-nɯkɯjŋu-t-a (= kɯjŋu tɤ-joʁ-a)}\hspace{5pt}\pcmn{我立下誓言}\end{exemple}\relationsémantique{参考}{\lien{ⓔkɯjŋu}{kɯjŋu}}\relationsémantique{参考}{\lien{ⓔanɯkɯjŋɯjŋu}{anɯkɯjŋɯjŋu}}\relationsémantique{参考}{\lien{ⓔanɯkɯjŋɤŋgɯ}{anɯkɯjŋɤŋgɯ}}\end{entrée}

\begin{entrée}{nɯkɯlu}{}{ⓔnɯkɯlu} 
\classe{vi}  
\grammaire{caus} \paradigme{dir}{pɯ-}\paradigme{dir}{pɯ-}
\begin{définition}\pfra{se perdre}\end{définition}
\begin{définition}\pcmn{迷路}\end{définition}
\begin{exemple}\pjya{tʂu pjɤ-nɯkɯlu-a}\hspace{5pt}\pcmn{我迷路了}\end{exemple}
\begin{exemple}\pjya{nɯ zgoku nɯ kɤ-ɕe mɯ-pɯ-rɲo-t-a tɕe, pɯ-nɯɣi-a ri pɯ-nɯkɯlu-a}\hspace{5pt}\pcmn{因为我从来没有去过这座山,我从上面回来的时候就迷路了}\end{exemple}
\begin{sous-entrée}{znɯkɯlu}{ⓔnɯkɯluⓝznɯkɯlu} 
\classe{vt} \end{sous-entrée}

\begin{définition}\pfra{faire se perdre}\end{définition}
\begin{définition}\pcmn{令……迷路}\end{définition}
\begin{exemple}\pjya{pɯ́-wɣ-znɯkɯlu-a}\hspace{5pt}\pcmn{他令我迷路了}\end{exemple}\end{entrée}

\begin{entrée}{nɯkɯmaʁ}{}{ⓔnɯkɯmaʁ} 
\classe{vi}  
\grammaire{denom} \paradigme{dir}{thɯ-}\paradigme{dir}{nɯ-}
\begin{définition}\pfra{se tromper}\end{définition}
\begin{définition}\pcmn{无意中犯错误}\end{définition}
\begin{exemple}\pjya{tɯ-rju ɲɯ-nɯkɯmaʁ}\hspace{5pt}\pcmn{他说错了}\end{exemple}
\begin{exemple}\pjya{kɤ-rɤt ɲɤ-tɯ-nɯkɯmaʁ ri, mɯ́j-tɯ-nɯsɯrtoʁ}\hspace{5pt}\pcmn{你写错了,但是你没有发现}\end{exemple}
\begin{exemple}\pjya{jiɕqha kɤ-sthoʁ ɲɤ-nɯkɯmaʁ-a}\hspace{5pt}\pcmn{我刚才按错了(手机)}\end{exemple}\relationsémantique{参考}{\lien{ⓔkɯmaʁ}{kɯmaʁ}}\relationsémantique{同义词}{\lien{ⓔnor}{nor}}\end{entrée}

\begin{entrée}{nɯkɯmpɕɤr}{}{ⓔnɯkɯmpɕɤr} 
\classe{vt}  
\grammaire{denom} \paradigme{dir}{thɯ-}
\begin{définition}\pfra{porter aux grandes occasions}\end{définition}
\begin{définition}\pcmn{打扮(在特定的情况才穿的衣服)}\end{définition}
\begin{exemple}\pjya{ki tɤrɣe ki nɯ chɯ-nɯkɯmpɕɤr ɕti ma ɯ-xso tɕe mɤ-ntɕhoz}\hspace{5pt}\pcmn{珍珠只在特殊的情况才戴,平时不戴}\end{exemple}\relationsémantique{参考}{\lien{ⓔmpɕɤr}{mpɕɤr}}\end{entrée}

\begin{entrée}{nɯkɯmtɕhɯ}{}{ⓔnɯkɯmtɕhɯ} 
\classe{vt} \paradigme{dir}{nɯ-}
\begin{définition}\pfra{jouer avec}\end{définition}
\begin{définition}\pcmn{把……当做玩具}\end{définition}
\begin{exemple}\pjya{a-tɕɯ kɯ qartshaz ɯ-χpi ɲɯ-nɯkɯmtɕhi ŋu}\hspace{5pt}\pcmn{我儿子在玩鹿形状的玩具}\end{exemple}\relationsémantique{参考}{\lien{ⓔkɯmtɕhɯ}{kɯmtɕhɯ}}\end{entrée}

\begin{entrée}{nɯlaʁjoʁ}{}{ⓔnɯlaʁjoʁ} 
\classe{vi}  
\grammaire{incorp} 
\begin{définition}\pfra{servir d'assistant}\end{définition}
\begin{définition}\pcmn{当帮手}\end{définition}\relationsémantique{参考}{\lien{ⓔlaʁjoʁ}{laʁjoʁ}}\end{entrée}

\begin{entrée}{nɯlɤmba}{}{ⓔnɯlɤmba} 
\classe{vt} \paradigme{dir}{tɤ-}
\begin{définition}\pfra{soutenir, s'occuper de}\end{définition}
\begin{définition}\pcmn{扶持;照顾}\end{définition}
\begin{exemple}\pjya{jiɕqha lo-βzi tɕe tɤ-nɯlɤmba-t-a}\hspace{5pt}\pcmn{他醉了,我就把他扶起来了}\end{exemple}\relationsémantique{同义词}{\lien{ⓔɣɤrndi}{ɣɤrndi}}\end{entrée}

\begin{entrée}{nɯlɤn}{}{ⓔnɯlɤn}\relationsémantique{参考}{\lien{ⓔlɤn}{lɤn}}\end{entrée}

\begin{entrée}{nɯlɤsɤr}{}{ⓔnɯlɤsɤr} 
\classe{vi}  
\grammaire{denom} \paradigme{dir}{tɤ-}
\begin{définition}\pfra{fêter le nouvel an}\end{définition}
\begin{définition}\pcmn{过年}\end{définition}
\begin{exemple}\pjya{kɤ-nɯlɤsɤr to-mda}\hspace{5pt}\pcmn{到了过年的时候了}\end{exemple}\relationsémantique{参考}{\lien{ⓔlɤsɤr}{lɤsɤr}}\étymologie{lo.gsar}\end{entrée}

\begin{entrée}{nɯlɤʑɯn}{}{ⓔnɯlɤʑɯn} 
\classe{vt} 
\begin{définition}\pfra{faire un procès d'intention, interpréter de travers, déformer les faits}\end{définition}
\begin{définition}\pcmn{冤枉;歪曲事实}\end{définition}\end{entrée}

\begin{entrée}{nɯlŋɤβ}{}{ⓔnɯlŋɤβ} 
\classe{vt} \paradigme{dir}{tɤ-}
\begin{définition}\pfra{s'empiffrer}\end{définition}
\begin{définition}\pcmn{大口大口地吃}\end{définition}
\begin{exemple}\pjya{tɤ-nɯlŋaβ-a ʑo tɤ-ndza-t-a}\hspace{5pt}\pcmn{我大口大口地吃了}\end{exemple}\relationsémantique{同义词}{\lien{ⓔnɯchɯβ}{nɯchɯβ}}\end{entrée}

\begin{entrée}{nɯlɯka}{}{ⓔnɯlɯka} 
\classe{vs} \paradigme{dir}{tɤ-}
\begin{définition}\pfra{être séparé et ne pas être dérangé par les autres, ne pas avoir besoin de s'occuper de toutes sortes de choses}\end{définition}
\begin{définition}\pcmn{被隔开(不受别人的干扰,不需要管多种事情)}\end{définition}
\begin{exemple}\pjya{nɯ ɕɯŋgɯ tɕe, thɯci tsuku ʑo tɯtɯrca ɣɯ-nɤma pɯ-ra tɕe pɯ-sɤɣdɯɣ ma tham tɕe tɯ-tɯphu ma ɣɯ-nɤma mɯ́j-ra tɕe, ɲɯ-sɤscit ma ɲɯ-kɯ-nɯlɯka tɕe}\hspace{5pt}\pcmn{以前要同时做好几种事情,现在只需要做一种事,不再需要管那么多,很轻松}\end{exemple}\end{entrée}

\begin{entrée}{nɯɬoʁ}{}{ⓔnɯɬoʁ} 
\classe{vi}  
\grammaire{autoben} \paradigme{dir}{\_}
\begin{définition}\pfra{se détacher}\end{définition}
\begin{définition}\pcmn{散开;自动的出来}\end{définition}
\begin{exemple}\pjya{laχtɕha nɯɬoʁ ɲɯ-ŋu}\hspace{5pt}\pcmn{这个东西快要掉出来了}\end{exemple}
\begin{exemple}\pjya{tɯ-xtsɤ-ri ɲɤ-nɯɬoʁ}\hspace{5pt}\pcmn{鞋带散了}\end{exemple}
\begin{exemple}\pjya{a-mi ɲɤ-nɯɬoʁ}\hspace{5pt}\pcmn{我脚脱臼了}\end{exemple}
\begin{exemple}\pjya{paʁ ɯ-naŋtɕɯ chɤ-nɯɬoʁ}\hspace{5pt}\pcmn{猪的内脏出来了}\end{exemple}\relationsémantique{参考}{\lien{ⓔɬoʁⓗ2}{ɬoʁ₂}}\end{entrée}

\begin{entrée}{nɯmɤɕɯŋgɯ}{}{ⓔnɯmɤɕɯŋgɯ} 
\classe{adv} 
\begin{définition}\pfra{autrefois}\end{définition}
\begin{définition}\pcmn{以前}\end{définition}\end{entrée}

\begin{entrée}{nɯmbe}{}{ⓔnɯmbe} 
\classe{vt} \paradigme{dir}{tɤ-}
\begin{définition}\pfra{dédommager}\end{définition}
\begin{définition}\pcmn{赔偿}\end{définition}
\begin{exemple}\pjya{a-ŋga ɲɤ-tɯ-βde tɕe tɤ-nɯmbe}\hspace{5pt}\pcmn{你把我的衣服弄丢了,你要给我赔}\end{exemple}
\begin{exemple}\pjya{ɯ-laχtɕha ɲɤ-nɯβde-t-a tɕe tɤ-nɯmbe-t-a}\hspace{5pt}\pcmn{我把他的东西弄丢了,就给他赔了}\end{exemple}\relationsémantique{同义词}{\lien{ⓔrɤli}{rɤli}}\end{entrée}

\begin{entrée}{nɯmbɣom}{}{ⓔnɯmbɣom} 
\classe{vt}  
\grammaire{appl} \paradigme{dir}{tɤ-}
\begin{définition}\pfra{avoir hâte de}\end{définition}
\begin{définition}\pcmn{盼望}\end{définition}
\begin{exemple}\pjya{tɤ-ta-nɯmbɣom}\hspace{5pt}\pcmn{我很想你了}\end{exemple}
\begin{exemple}\pjya{ɯ-mu to-nɯmbɣom}\hspace{5pt}\pcmn{他很想妈妈了}\end{exemple}
\begin{exemple}\pjya{ɯʑo ju-nɯɣi ɲɯ-nɯmbɣom-a}\hspace{5pt}\pcmn{我盼望他早日回来}\end{exemple}
\begin{exemple}\pjya{lɤsɤr ju-zɣɯt ɲɯ-nɯmbɣom-a}\hspace{5pt}\pcmn{我盼望新年}\end{exemple}
\begin{exemple}\pjya{jɯfɕɯr a-ʑɯβ mɯ-pɯ-ɣe tɕe, lu-fsoʁ tɤ-nɯmbɣom-a}\hspace{5pt}\pcmn{昨天睡不着,盼望天亮}\end{exemple}\relationsémantique{同义词}{\lien{ⓔnɯɣbɯɣ}{nɯɣbɯɣ}}\relationsémantique{参考}{\lien{ⓔmbɣom}{mbɣom}}
\begin{sous-entrée}{anɯmbɯmbɣom}{ⓔnɯmbɣomⓝanɯmbɯmbɣom} 
\classe{vi} 
\begin{définition}\pfra{se manquer les uns aux autres}\end{définition}
\begin{définition}\pcmn{互相思念}\end{définition}\end{sous-entrée}

\end{entrée}

\begin{entrée}{nɯmbjɯm}{}{ⓔnɯmbjɯm} 
\classe{vi} \paradigme{dir}{thɯ-}
\begin{définition}\pfra{se chauffer au feu}\end{définition}
\begin{définition}\pcmn{烤火取暖}\end{définition}
\begin{exemple}\pjya{smi ɯ-phe thɯ-nɯmbjɯm}\hspace{5pt}\pcmn{你烤火取暖吧}\end{exemple}
\begin{exemple}\pjya{smi ɯ-phe thɯ-nɯmbjɯm-a}\hspace{5pt}\pcmn{我烤火取暖了}\end{exemple}\relationsémantique{同义词}{\lien{ⓔnɯsmɯɣjɯm}{nɯsmɯɣjɯm}}\end{entrée}

\begin{entrée}{nɯmbrɤpɯ}{}{ⓔnɯmbrɤpɯ} 
\classe{v}  
\grammaire{incorp} \paradigme{dir}{tɤ-}\paradigme{dir}{tɤ-}
\begin{définition}\pfra{monter (à cheval )}\end{définition}
\begin{définition}\pcmn{骑}\end{définition}
\begin{définition}\pfra{se laisser monter}\end{définition}
\begin{définition}\pcmn{让……骑在自己背上}\end{définition}
\begin{exemple}\pjya{mbro tɤ-nɯmbrɤpɯ-t-a}\hspace{5pt}\pcmn{我骑了马}\end{exemple}
\begin{exemple}\pjya{qambrɯ tɤ-nɯmbrɤpɯ-t-a}\hspace{5pt}\pcmn{我骑了牦牛}\end{exemple}
\begin{exemple}\pjya{mbro tɯrme nɯ kɯ to-ʑɣɤnɯmbrɤpɯ}\hspace{5pt}\pcmn{马让人骑在它背上了}\end{exemple}\relationsémantique{参考}{\lien{ⓔmbroⓗ2}{mbro₂}}
\begin{sous-entrée}{ʑɣɤnɯmbrɤpɯ}{ⓔnɯmbrɤpɯⓝʑɣɤnɯmbrɤpɯ} 
\classe{vi} \end{sous-entrée}

\end{entrée}

\begin{entrée}{nɯmbrɤrɟɯɣ}{}{ⓔnɯmbrɤrɟɯɣ} 
\classe{vi}  
\grammaire{incorp} \paradigme{dir}{tɤ-}
\begin{définition}\pfra{faire une course de cheval}\end{définition}
\begin{définition}\pcmn{赛马}\end{définition}
\begin{exemple}\pjya{lɤsɤr ɯ-raŋ pɯ-nɯmbrɤrɟɯɣ-nɯ}\hspace{5pt}\pcmn{过年的时候,他们在赛马}\end{exemple}\relationsémantique{参考}{\lien{ⓔmbrɤrɟɯɣ}{mbrɤrɟɯɣ}}\end{entrée}

\begin{entrée}{nɯmbrɤzɯ}{}{ⓔnɯmbrɤzɯ} 
\classe{vt}  
\grammaire{denom} \paradigme{dir}{pɯ-}
\begin{définition}\pfra{obtenir le fruit de son travail}\end{définition}
\begin{définition}\pcmn{得到自己的劳动成果}\end{définition}
\begin{exemple}\pjya{pɯ-nɯmbrɤzɯ-j}\hspace{5pt}\pcmn{我们得到自己的劳动成果}\end{exemple}
\begin{exemple}\pjya{japa ndɤre sɤrwa pɯ-tu tɕe kɤ-nɯmbrɤzɯ pɯ-rkɯn}\hspace{5pt}\pcmn{去年下了冰雹,(农民们的)收获不多}\end{exemple}\relationsémantique{参考}{\lien{ⓔɯ-mbrɤzɯ}{ɯ-mbrɤzɯ}}\end{entrée}

\begin{entrée}{nɯmbrɯmtsaʁ}{}{ⓔnɯmbrɯmtsaʁ} 
\classe{vi}  
\grammaire{incorp} \paradigme{dir}{}\paradigme{}{tɤ-}
\begin{définition}\pfra{sauter à la corde}\end{définition}
\begin{définition}\pcmn{跳绳}\end{définition}\relationsémantique{参考}{\lien{ⓔtɯmbri}{tɯmbri}}\relationsémantique{参考}{\lien{ⓔmtsaʁ}{mtsaʁ}}\end{entrée}

\begin{entrée}{nɯmbɯrlɤn}{}{ⓔnɯmbɯrlɤn} 
\classe{vt}  
\grammaire{denom} \paradigme{dir}{thɯ-}
\begin{définition}\pfra{raboter}\end{définition}
\begin{définition}\pcmn{刨}\end{définition}
\begin{exemple}\pjya{si thɯ-nɯmbɯrlɤn}\hspace{5pt}\pcmn{你刨一下树}\end{exemple}
\begin{exemple}\pjya{tɤrɤm thɯ-nɯmbɯrlan-a}\hspace{5pt}\pcmn{我刨了木板}\end{exemple}\relationsémantique{参考}{\lien{ⓔmbɯrlɤn}{mbɯrlɤn}}\étymologie{ⁿbur.len}\end{entrée}

\begin{entrée}{nɯmbɯsɯt}{}{ⓔnɯmbɯsɯt} 
\classe{vt}  
\grammaire{denom} \paradigme{dir}{thɯ-}
\begin{définition}\pfra{râper}\end{définition}
\begin{définition}\pcmn{擦成丝丝}\end{définition}\relationsémantique{参考}{\lien{ⓔmbɯsɯt}{mbɯsɯt}}\end{entrée}

\begin{entrée}{nɯmdar}{}{ⓔnɯmdar} 
\classe{vi} \paradigme{dir}{\_}
\begin{définition}\pfra{sauter}\end{définition}
\begin{définition}\pcmn{跳}\end{définition}
\begin{exemple}\pjya{kɤ-nɯmdar-a}\hspace{5pt}\pcmn{我跳了}\end{exemple}
\begin{exemple}\pjya{tɕɤki tɯ-ci ɣɤʑu tɕe, kɤ-ŋke mɯ́j-khɯ tɕe thɯ-nɯmdar-a}\hspace{5pt}\pcmn{下面有水,我不能走就跳过去了}\end{exemple}\relationsémantique{同义词}{\lien{ⓔmtsaʁ}{mtsaʁ}}
\begin{sous-entrée}{nɤmdɯmdar}{ⓔnɯmdarⓝnɤmdɯmdar} 
\classe{vi}  
\grammaire{n.orient} 
\begin{définition}\pfra{sauter dans tous les sens}\end{définition}
\begin{définition}\pcmn{跳来跳去}\end{définition}\end{sous-entrée}

\end{entrée}

\begin{entrée}{nɯmdaʁʑɯɣ}{}{ⓔnɯmdaʁʑɯɣ} 
\classe{vt}  
\grammaire{denom} \paradigme{dir}{tɤ-}
\begin{définition}\pfra{tirer à l'arc}\end{définition}
\begin{définition}\pcmn{射箭}\end{définition}
\begin{exemple}\pjya{tɤ-fsɯr tɤ-nɯmdaʁzɯɣ-a}\hspace{5pt}\pcmn{我对着靶子射箭了}\end{exemple}\relationsémantique{参考}{\lien{ⓔmdaʁʑɯɣ}{mdaʁʑɯɣ}}\end{entrée}

\begin{entrée}{nɯmdoʁ}{}{ⓔnɯmdoʁ} 
\classe{vi}  
\grammaire{denom} \paradigme{dir}{tɤ-}
\begin{définition}\pfra{avoir l'air de}\end{définition}
\begin{définition}\pcmn{好像;看起来}\end{définition}
\begin{exemple}\pjya{jisŋi tɯ-mɯ kɯ-lɤt ɲɯ-nɯmdoʁ}\hspace{5pt}\pcmn{今天好像要下雨}\end{exemple}
\begin{exemple}\pjya{ɲɯ-tɯ-nɯmdoʁ}\hspace{5pt}\pcmn{你又强壮又高大}\end{exemple}\relationsémantique{参考}{\lien{ⓔɯ-mdoʁ}{ɯ-mdoʁ}}
\begin{sous-entrée}{znɯmdoʁ}{ⓔnɯmdoʁⓝznɯmdoʁ} 
\classe{vt} 
\begin{définition}\pfra{bien faire (un certain type de travail)}\end{définition}
\begin{définition}\pcmn{【好样】(那一方面的工作)做得好}\end{définition}
\begin{exemple}\pjya{nɤki nɯ kɤ-sɤsɯxɕɤt kɯ-znɯmdoʁ ci pɯ-ŋu}\hspace{5pt}\pcmn{那个人以前是个很好的老师}\end{exemple}
\begin{exemple}\pjya{ɯʑo kɯ tɯ-ta-nɤma ʑo nɯ tu-znɯmdoʁɕti}\hspace{5pt}\pcmn{他做的每一样工作都做得非常好}\end{exemple}\end{sous-entrée}

\end{entrée}

\begin{entrée}{nɯmdɯm}{}{ⓔnɯmdɯm} 
\classe{vt} \paradigme{dir}{\_}
\begin{définition}\pfra{manger en marchant}\end{définition}
\begin{définition}\pcmn{一边走一边吃}\end{définition}
\begin{exemple}\pjya{kɯ-chi ɲɯ-tɯ-ɤz-nɯmdɯm}\hspace{5pt}\pcmn{你一边走一边吃糖}\end{exemple}
\begin{exemple}\pjya{@paopaotang ɲɯ-tɯ-ɤz-nɯmdɯm}\hspace{5pt}\pcmn{你一边走一边吃泡泡糖}\end{exemple}\relationsémantique{参考}{\lien{ⓔnɯndzɤmdɯm}{nɯndzɤmdɯm}}\end{entrée}

\begin{entrée}{nɯmga}{}{ⓔnɯmga} 
\classe{vt} \paradigme{dir}{kɤ-}\sens{1}
\begin{définition}\pfra{avoir l'intention}\end{définition}
\begin{définition}\pcmn{有意}\end{définition}\sens{2}
\begin{définition}\pfra{être bien fait pour}\end{définition}
\begin{définition}\pcmn{活该}\end{définition}
\begin{exemple}\pjya{nɯ kɤ-tɯ-nɯmga-t}\hspace{5pt}\pcmn{你活该!}\end{exemple}
\begin{exemple}\pjya{pɯ-tɯ-sɤnɯrtɕa ntsɯ tɕe taχphe ci nɯ-tɯ́-wɣ-sɤʁe tɕe, nɯ kɤ-tɯ-nɯmga-t}\hspace{5pt}\pcmn{你总是惹人家,你被打耳光是你活该}\end{exemple}
\begin{sous-entrée}{kɤ-nɯmga}{ⓔnɯmgaⓢ2ⓝkɤ-nɯmga}
\begin{définition}\pfra{afin de}\end{définition}
\begin{définition}\pcmn{为了}\end{définition}
\begin{exemple}\pjya{ftsoʁ nɯ tɕe tɕe ɯ-lu kɤ-nɯmga ɲɯ-ŋu}\hspace{5pt}\pcmn{母犏牛是为了牛奶(而养的)}\end{exemple}\end{sous-entrée}

\end{entrée}

\begin{entrée}{nɯmgo}{}{ⓔnɯmgo} 
\classe{vi}  
\grammaire{denom} \paradigme{dir}{tɤ-}
\begin{définition}\pfra{déjeuner}\end{définition}
\begin{définition}\pcmn{吃中午饭}\end{définition}\relationsémantique{同义词}{\lien{ⓔrɯndzɤtshi}{rɯndzɤtshi}}\relationsémantique{参考}{\lien{ⓔtɯ-mgo}{tɯ-mgo}}\end{entrée}

\begin{entrée}{nɯmgro}{}{ⓔnɯmgro} 
\classe{vt} \paradigme{dir}{thɯ-}\paradigme{dir}{pɯ-}\sens{1}
\begin{définition}\pfra{attendre}\end{définition}
\begin{définition}\pcmn{盼望}\end{définition}
\begin{exemple}\pjya{pɯ-ta-nɯmgro}\hspace{5pt}\pcmn{我盼望你}\end{exemple}
\begin{exemple}\pjya{pjɯ-tɯ-ɣi pɯ-ta-nɯmgro}\hspace{5pt}\pcmn{我盼望你来}\end{exemple}\sens{2}
\begin{définition}\pfra{espérer}\end{définition}
\begin{définition}\pcmn{希望}\end{définition}
\begin{exemple}\pjya{nɤʑo kɤ-si ma nɤ-kɤ-nɯmgro ɲɤ-me}\hspace{5pt}\pcmn{你只有死路一条}\end{exemple}\relationsémantique{参考}{\lien{ⓔsɤmgro}{sɤmgro}}
\begin{sous-entrée}{anɯmgɯmgro}{ⓔnɯmgroⓢ2ⓝanɯmgɯmgro} 
\classe{vi} 
\begin{définition}\pfra{s'attendre les uns les autres}\end{définition}
\begin{définition}\pcmn{互相盼望}\end{définition}\end{sous-entrée}

\end{entrée}

\begin{entrée}{nɯmɢla}{}{ⓔnɯmɢla} 
\classe{vt}  
\grammaire{denom} \paradigme{dir}{\_}
\begin{définition}\pfra{passer par dessus}\end{définition}
\begin{définition}\pcmn{跨过}\end{définition}
\begin{exemple}\pjya{si kɤ-nɯmɢla-t-a}\hspace{5pt}\pcmn{我跨过了(倒下的)树干}\end{exemple}
\begin{exemple}\pjya{rdɤstaʁ kɤ-nɯmɢla-t-a}\hspace{5pt}\pcmn{我从石头上跨过去了}\end{exemple}\relationsémantique{参考}{\lien{ⓔtɯ-mɢla}{tɯ-mɢla}}\end{entrée}

\begin{entrée}{nɯmɟa}{}{ⓔnɯmɟa} 
\classe{vt} \paradigme{dir}{tɤ-}\paradigme{dir}{pɯ-}
\begin{définition}\pfra{obtenir}\end{définition}
\begin{définition}\pcmn{得到,捡}\end{définition}
\begin{exemple}\pjya{tɤ-nɯmɟa-t-a}\hspace{5pt}\pcmn{我捡了}\end{exemple}
\begin{exemple}\pjya{ɯ-thoʁ nɯ ɲɯ-ɤta tɕe, tɤ-nɯmɟa-t-a}\hspace{5pt}\pcmn{地上有那个东西,我就捡了}\end{exemple}\relationsémantique{参考}{\lien{ⓔmɟa}{mɟa}}\end{entrée}

\begin{entrée}{nɯmkɤɣɯr}{}{ⓔnɯmkɤɣɯr} 
\classe{vt}  
\grammaire{denom} \paradigme{dir}{thɯ-}
\begin{définition}\pfra{porter sur le cou comme un collier}\end{définition}
\begin{définition}\pcmn{把某物体当项链戴在脖子上}\end{définition}
\begin{exemple}\pjya{laχtɕha thɯ-nɯmkɤɣɯr-a}\hspace{5pt}\pcmn{我把东西戴在脖子上了}\end{exemple}
\begin{exemple}\pjya{ɯʑo kɯ laχtɕha tha-nɯmkɤɣɯr}\hspace{5pt}\pcmn{他把东西戴在脖子上}\end{exemple}\relationsémantique{参考}{\lien{ⓔmkɤɣɯr}{mkɤɣɯr}}\end{entrée}

\begin{entrée}{nɯmkɤqloʁ}{}{ⓔnɯmkɤqloʁ} 
\classe{vi} \paradigme{dir}{thɯ-}\paradigme{dir}{thɯ-}
\begin{définition}\pfra{se prendre les pieds dans quelque chose et tomber}\end{définition}
\begin{définition}\pcmn{绊倒}\end{définition}
\begin{définition}\pfra{faire un croc-en-jambe (faire tomber vers l'avant)}\end{définition}
\begin{définition}\pcmn{绊拽}\end{définition}
\begin{exemple}\pjya{thɯ-nɯkɤqloʁ-a}\hspace{5pt}\pcmn{我被绊倒了}\end{exemple}
\begin{exemple}\pjya{thɯ-kɯ-z-nɯmkɤqloʁ-a (=pɯ-kɯ-tʂaβ-a)}\hspace{5pt}\pcmn{你把我绊拽住了}\end{exemple}
\begin{exemple}\pjya{ma-tɯ-ste ma tɯ-z-nɯmkɤqloʁ}\hspace{5pt}\pcmn{你别这样,不然你会把他绊拽住}\end{exemple}
\begin{sous-entrée}{znɯmkɤqloʁ}{ⓔnɯmkɤqloʁⓝznɯmkɤqloʁ} 
\classe{vt}  
\grammaire{caus} \end{sous-entrée}

\end{entrée}

\begin{entrée}{nɯmnɤl}{}{ⓔnɯmnɤl} 
\classe{vi}  
\grammaire{caus} \paradigme{dir}{pɯ-}\paradigme{dir}{pɯ-}
\begin{définition}\pfra{être sali, être rendu impur}\end{définition}
\begin{définition}\pcmn{被玷污,受晦气(迷信的说法)}\end{définition}
\begin{sous-entrée}{znɯmnɤl}{ⓔnɯmnɤlⓝznɯmnɤl} 
\classe{vt} \end{sous-entrée}

\begin{définition}\pfra{salir, rendre impur}\end{définition}
\begin{définition}\pcmn{令……沾上玷污气,沾上晦气}\end{définition}
\begin{exemple}\pjya{tɯ-ŋga nɯnɯ ma-thɯ-tɯ-ŋge ma mɤ-χtso tɕe tú-wɣ-z-nɯmnɤl}\hspace{5pt}\pcmn{你别穿这件衣服,不干净,你会沾上晦气的}\end{exemple}
\begin{exemple}\pjya{tɕhi tɤ-tɯ-ari tɕe, tɯrme ɯ-pa a-mɤ-tɯ-ɕe ma tú-wɣ-z-nɯmnɤl}\hspace{5pt}\pcmn{你上楼梯的时候,不要在别人下面不然你会沾上晦气的}\end{exemple}
\begin{exemple}\pjya{tɯ-ŋga kɯ pjɤ́-wɣ-z-nɯmnal-a}\hspace{5pt}\pcmn{这件令我受了晦气}\end{exemple}\end{entrée}

\begin{entrée}{nɯmɲaqrɯ}{}{ⓔnɯmɲaqrɯ} 
\classe{vt} \paradigme{dir}{tɤ-}
\begin{définition}\pfra{épier}\end{définition}
\begin{définition}\pcmn{瞪眼}\end{définition}
\begin{exemple}\pjya{ma-tɤ-kɯ-nɯmɲaqrɯ-a}\hspace{5pt}\pcmn{你不要瞪我}\end{exemple}\relationsémantique{参考}{\lien{ⓔmɲaqrɯ}{mɲaqrɯ}}\end{entrée}

\begin{entrée}{nɯmɲɯɣ}{}{ⓔnɯmɲɯɣ} 
\classe{vi}  
\grammaire{denom} \paradigme{dir}{kɤ-}
\begin{définition}\pfra{attraper le cancer de l'estomac}\end{définition}
\begin{définition}\pcmn{得胃癌}\end{définition}
\begin{exemple}\pjya{ko-nɯmɲɯɣ}\hspace{5pt}\pcmn{他得了胃癌}\end{exemple}\relationsémantique{参考}{\lien{ⓔtɯ-mɲɯɣ}{tɯ-mɲɯɣ}}\end{entrée}

\begin{entrée}{nɯmɲɯka}{}{ⓔnɯmɲɯka} 
\classe{vi}  
\grammaire{denom} \paradigme{dir}{pɯ-}
\begin{définition}\pfra{être humilié}\end{définition}
\begin{définition}\pcmn{被羞辱}\end{définition}
\begin{exemple}\pjya{jiɕqha ndɤre tɯrme ɯ-ʁɤri pɯ-nɯmɲɯka}\hspace{5pt}\pcmn{他在别人面前被羞辱了}\end{exemple}\relationsémantique{参考}{\lien{ⓔmɲɯka}{mɲɯka}}\end{entrée}

\begin{entrée}{nɯmɲɯʁʑi}{}{ⓔnɯmɲɯʁʑi} 
\classe{vs} 
\begin{définition}\pfra{avoir bon caractère}\end{définition}
\begin{définition}\pcmn{脾气好}\end{définition}\relationsémantique{同义词}{\lien{ⓔnɯmɲɯtɕhɤz}{nɯmɲɯtɕhɤz}}\relationsémantique{参考}{\lien{ⓔmɲɯʁʑi}{mɲɯʁʑi}}\end{entrée}

\begin{entrée}{nɯmɲɯtɕhɤz}{}{ⓔnɯmɲɯtɕhɤz} 
\classe{vs} \paradigme{dir}{tɤ-}
\begin{définition}\pfra{avoir bon caractère}\end{définition}
\begin{définition}\pcmn{脾气很好}\end{définition}\relationsémantique{同义词}{\lien{ⓔnɯmɲɯʁʑi}{nɯmɲɯʁʑi}}\relationsémantique{参考}{\lien{ⓔmɲɯtɕhɤz}{mɲɯtɕhɤz}}\end{entrée}

\begin{entrée}{nɯmŋu}{}{ⓔnɯmŋu} 
\classe{vt} \paradigme{dir}{kɤ-}
\begin{définition}\pfra{boire sans main, en mettant directement sa bouche sur...}\end{définition}
\begin{définition}\pcmn{(不用手)直接用嘴对着……的口喝}\end{définition}
\begin{exemple}\pjya{khɯtsa kɤ-nɯmŋu-t-a}\hspace{5pt}\pcmn{我直接用嘴对着碗口喝了(水)}\end{exemple}\relationsémantique{参考}{\lien{ⓔɯ-mŋu}{ɯ-mŋu}}\end{entrée}

\begin{entrée}{nɯmŋa}{}{ⓔnɯmŋa} 
\classe{vi} \paradigme{dir}{tɤ-}
\begin{définition}\pfra{être impressionnant}\end{définition}
\begin{définition}\pcmn{醒目耀眼}\end{définition}
\begin{exemple}\pjya{ɯ-ŋga ɲɯ-nɯmŋa}\hspace{5pt}\pcmn{他的衣服醒目耀眼}\end{exemple}\étymologie{mŋa}\end{entrée}

\begin{entrée}{nɯmpa}{}{ⓔnɯmpa} 
\classe{vt} \paradigme{dir}{nɯ-}
\begin{définition}\pfra{s'occuper de}\end{définition}
\begin{définition}\pcmn{照顾}\end{définition}
\begin{exemple}\pjya{tɤ-pɤtso ra kɤ-nɯmpa ɲɯ-ɴqa}\hspace{5pt}\pcmn{照顾小孩子很难}\end{exemple}
\begin{sous-entrée}{ʑɣɤnɯmpa}{ⓔnɯmpaⓝʑɣɤnɯmpa} 
\classe{vi}  
\grammaire{refl} 
\begin{définition}\pfra{s'occuper de soi-même}\end{définition}
\begin{définition}\pcmn{照顾自己}\end{définition}
\begin{exemple}\pjya{kɤ-ʑɣɤnɯmpa pjɯ-kɯ-cha ra}\hspace{5pt}\pcmn{一定要会照顾自己}\end{exemple}\end{sous-entrée}

\begin{sous-entrée}{sɤnɯmpa}{ⓔnɯmpaⓝsɤnɯmpa} 
\classe{vi}  
\grammaire{apass} 
\begin{définition}\pfra{s'occuper des gens}\end{définition}
\begin{définition}\pcmn{照顾别人}\end{définition}\end{sous-entrée}

\end{entrée}

\begin{entrée}{nɯmtɕhu}{}{ⓔnɯmtɕhu} 
\classe{vt} \paradigme{dir}{tɤ-}\paradigme{dir}{tɤ-}
\begin{définition}\pfra{dire du mal}\end{définition}
\begin{définition}\pcmn{说别人的坏话}\end{définition}
\begin{définition}\pfra{dire du mal des gens}\end{définition}
\begin{définition}\pcmn{说别人的坏话}\end{définition}
\begin{exemple}\pjya{jiɕqha nɯ kɯ a-qhu ɲɯ́-wɣ-nɯmtɕhu-a}\hspace{5pt}\pcmn{那个人在我背后说我的坏话}\end{exemple}
\begin{exemple}\pjya{jiɕqha nɯ ɲɯ-sɤnɯmtɕhu}\hspace{5pt}\pcmn{那个人说别人的坏话}\end{exemple}
\begin{sous-entrée}{sɤnɯmtɕhu}{ⓔnɯmtɕhuⓝsɤnɯmtɕhu} 
\classe{vi} \end{sous-entrée}

\end{entrée}

\begin{entrée}{nɯmtɕhɯtsaʁ}{}{ⓔnɯmtɕhɯtsaʁ} 
\classe{vi} 
\begin{définition}\pfra{avoir des ulcères sur la bouche}\end{définition}
\begin{définition}\pcmn{嘴上生疮}\end{définition}\relationsémantique{参考}{\lien{ⓔmtɕhɯtsaʁ}{mtɕhɯtsaʁ}}\end{entrée}

\begin{entrée}{nɯmtɕi}{}{ⓔnɯmtɕi} 
\classe{vi}  
\grammaire{denom} \paradigme{dir}{thɯ-}
\begin{définition}\pfra{tôt}\end{définition}
\begin{définition}\pcmn{起得早;来得早}\end{définition}
\begin{exemple}\pjya{aʑo ɲɯ-nɯmtɕi-a, nɤʑo mɯ́j-tɯ-nɯmtɕi}\hspace{5pt}\pcmn{我起得早,你起得晚}\end{exemple}
\begin{exemple}\pjya{ɯʑo sɤskɯsku ʑo chɯ-nɯmtɕi ɲɯ-ŋu}\hspace{5pt}\pcmn{他每天早上早起}\end{exemple}
\begin{exemple}\pjya{fso tɕe a-tu-sɤ-nɯmtɕɯmtɕi, ʑa ku-nɯ-rŋgɯ-a ra}\hspace{5pt}\pcmn{我为了明天早起,就要早点睡觉}\end{exemple}
\begin{exemple}\pjya{pɣɤtɕɯ mɤ-kɯ-nɯtɕi qajɯ mɤ-aʁe}\hspace{5pt}\pcmn{晚起的鸟吃不到虫子}\end{exemple}\relationsémantique{参考}{\lien{ⓔtɯmtɕi}{tɯmtɕi}}\end{entrée}

\begin{entrée}{nɯmthɯ}{₁}{ⓔnɯmthɯⓗ1} 
\classe{vt} \paradigme{dir}{tɤ-}
\begin{définition}\pfra{faire un bénéfice au dépend de}\end{définition}
\begin{définition}\pcmn{赚……的钱}\end{définition}
\begin{exemple}\pjya{tɤ-ta-nɯmthɯ}\hspace{5pt}\pcmn{我赚了你的钱}\end{exemple}
\begin{exemple}\pjya{tɤ́-wɣ-nɯmthɯ-a}\hspace{5pt}\pcmn{他赚了我的钱}\end{exemple}
\begin{sous-entrée}{sɤnɯmthɯ}{ⓔnɯmthɯⓗ1ⓝsɤnɯmthɯ} 
\classe{vi} 
\begin{définition}\pfra{faire un bénéfice}\end{définition}
\begin{définition}\pcmn{赚别人的钱}\end{définition}\end{sous-entrée}

\étymologie{mtʰo}\end{entrée}

\begin{entrée}{nɯmthɯ}{₂}{ⓔnɯmthɯⓗ2} 
\classe{vt}  
\grammaire{denom} \paradigme{dir}{thɯ-}
\begin{définition}\pfra{maudire}\end{définition}
\begin{définition}\pcmn{诅咒(念咒经)}\end{définition}
\begin{exemple}\pjya{cho-nɯmthɯ (=mthɯ cho-lɤt)}\hspace{5pt}\pcmn{(喇嘛)诅咒了他}\end{exemple}\relationsémantique{参考}{\lien{ⓔmthɯ}{mthɯ}}\end{entrée}

\begin{entrée}{nɯmto}{}{ⓔnɯmto} 
\classe{vt} \sens{1}\paradigme{dir}{pɯ-}
\begin{définition}\pfra{trouver qqch par terre}\end{définition}
\begin{définition}\pcmn{捡东西}\end{définition}
\begin{exemple}\pjya{laχtɕha pa-nɯmto}\hspace{5pt}\pcmn{他捡到东西了}\end{exemple}
\begin{exemple}\pjya{@gangbi pa-nɯmto}\hspace{5pt}\pcmn{他捡到钢笔了}\end{exemple}
\begin{exemple}\pjya{pɕawtsɯ pa-nɯmto}\hspace{5pt}\pcmn{他捡到钱了}\end{exemple}\sens{2}\paradigme{dir}{tɤ-}
\begin{définition}\pfra{viser}\end{définition}
\begin{définition}\pcmn{瞄准}\end{définition}
\begin{exemple}\pjya{tɤfsɯr tɤ-nɯmto-t-a}\hspace{5pt}\pcmn{我瞄准了靶子}\end{exemple}
\begin{exemple}\pjya{tɤfsɯr ɲɯ-ɤz-nɯmto-nɯ}\hspace{5pt}\pcmn{他们在瞄准靶子}\end{exemple}\relationsémantique{参考}{\lien{ⓔmtoⓝmto}{mto}}\end{entrée}

\begin{entrée}{nɯmtshalu}{}{ⓔnɯmtshalu} 
\classe{vi}  
\grammaire{denom} 
\begin{définition}\pfra{ramasser des orties}\end{définition}
\begin{définition}\pcmn{找荨麻}\end{définition}\relationsémantique{参考}{\lien{ⓔmtshalu}{mtshalu}}\end{entrée}

\begin{entrée}{nɯna}{}{ⓔnɯna} 
\classe{vi} \paradigme{dir}{tɤ-}
\begin{définition}\pfra{se reposer}\end{définition}
\begin{définition}\pcmn{休息}\end{définition}
\begin{exemple}\pjya{tʂu tɤ-nɯna-a}\hspace{5pt}\pcmn{我在路上休息了}\end{exemple}
\begin{exemple}\pjya{ɲɯ-ɴqa tɕe tɤ-nɯna-a}\hspace{5pt}\pcmn{很辛苦,我就休息了}\end{exemple}
\begin{exemple}\pjya{jisŋi toʁde tɤ-nɯna tɕe, jɤ-anɯri}\hspace{5pt}\pcmn{她今天(在我家)休息了一会就回去了}\end{exemple}
\begin{exemple}\pjya{kɤ-nɯβlu mɤ-tɯ-cha, tɤ-nɯna}\hspace{5pt}\pcmn{你休想骗我}\end{exemple}\relationsémantique{参考}{\lien{ⓔznɯna}{znɯna}}\end{entrée}

\begin{entrée}{nɯndzɤmbɣom}{}{ⓔnɯndzɤmbɣom} 
\classe{vi}  
\grammaire{comp} \paradigme{dir}{tɤ-}
\begin{définition}\pfra{être pressé de manger}\end{définition}
\begin{définition}\pcmn{急着要吃;馋嘴}\end{définition}
\begin{exemple}\pjya{tɤ-nɯndzɤmbɣom-a}\hspace{5pt}\pcmn{我急着要吃了}\end{exemple}
\begin{exemple}\pjya{nɤʑo nɤ-tɯ-nɯndzɤmbɣom nɯ}\hspace{5pt}\pcmn{你很馋嘴}\end{exemple}\relationsémantique{同义词}{\lien{ⓔfkrɯz}{fkrɯz}}\relationsémantique{参考}{\lien{ⓔndza}{ndza}}\relationsémantique{参考}{\lien{ⓔmbɣom}{mbɣom}}\end{entrée}

\begin{entrée}{nɯndzɤmdɯm}{}{ⓔnɯndzɤmdɯm} 
\classe{vi}  
\grammaire{comp} \paradigme{dir}{tɤ-}
\begin{définition}\pfra{aimer manger des petites collations}\end{définition}
\begin{définition}\pcmn{爱吃零食}\end{définition}
\begin{exemple}\pjya{ɲɯ-tɯ-nɯndzɤmdɯm}\hspace{5pt}\pcmn{你爱吃零食}\end{exemple}
\begin{exemple}\pjya{aj tɤ-nɯndzɤmdɯm-a}\hspace{5pt}\pcmn{我爱吃零食}\end{exemple}
\begin{exemple}\pjya{kɤ-nɯndzɤmdɯm χɕu}\hspace{5pt}\pcmn{他最喜欢吃零食}\end{exemple}\relationsémantique{参考}{\lien{ⓔnɯmdɯm}{nɯmdɯm}}\relationsémantique{参考}{\lien{ⓔndza}{ndza}}\end{entrée}

\begin{entrée}{nɯndzɤqɤr}{}{ⓔnɯndzɤqɤr} 
\classe{vt}  
\grammaire{comp} \paradigme{dir}{pɯ-}
\begin{définition}\pfra{ne pas laisser quelqu'un manger avec soi}\end{définition}
\begin{définition}\pcmn{不让别人吃}\end{définition}
\begin{exemple}\pjya{pɯ-kɯ-nɯndzɤqar-a}\hspace{5pt}\pcmn{你没有叫我吃}\end{exemple}
\begin{sous-entrée}{anɯndzɤqɯqɤr}{ⓔnɯndzɤqɤrⓝanɯndzɤqɯqɤr} 
\classe{vi} 
\begin{définition}\pfra{manger chacun dans son coin}\end{définition}
\begin{définition}\pcmn{各自吃各的}\end{définition}
\begin{exemple}\pjya{tɯrme ʁnɯ-rdoʁ ma maŋe-tɕi tɕe, kɤ-ɤnɯndzɤqɯqɤr mɤ-nɯ-cha-tɕi}\hspace{5pt}\pcmn{只有我们俩,不能各自吃的各的}\end{exemple}\relationsémantique{参考}{\lien{ⓔndza}{ndza}}\relationsémantique{参考}{\lien{ⓔqɤr}{qɤr}}\end{sous-entrée}

\end{entrée}

\begin{entrée}{nɯndzɤqhɤjɯ}{}{ⓔnɯndzɤqhɤjɯ}\relationsémantique{参考}{\lien{ⓔndzɤqhɤjɯ}{ndzɤqhɤjɯ}}\end{entrée}

\begin{entrée}{nɯndzɤsma}{}{ⓔnɯndzɤsma} 
\classe{vi} 
\begin{définition}\pfra{vouloir manger}\end{définition}
\begin{définition}\pcmn{想吃东西}\end{définition}
\begin{exemple}\pjya{tɤ-ngo-a tɕe ɲɯ-nɯndzɤsma-a}\hspace{5pt}\pcmn{我病了,现在就想吃东西}\end{exemple}\relationsémantique{参考}{\lien{ⓔndza}{ndza}}\relationsémantique{参考}{\lien{ⓔnɤsma}{nɤsma}}\end{entrée}

\begin{entrée}{nɯndzom}{}{ⓔnɯndzom} 
\classe{vi} 
\begin{définition}\pfra{couler le long}\end{définition}
\begin{définition}\pcmn{顺着某个东西流下来}\end{définition}
\begin{exemple}\pjya{tɯ-ci nɯ sɯku ɯ-taʁ pjɤ-nɯndzom}\hspace{5pt}\pcmn{水顺着树梢流下来了}\end{exemple}
\begin{exemple}\pjya{tɯ-ci nɯ si ɯ-rtaʁ ɯ-taʁ pjɤ-nɯndzom}\hspace{5pt}\pcmn{水顺着树枝流下来了}\end{exemple}
\begin{exemple}\pjya{ɯ-kɤrme ɯ-taʁ tɯ-ci pjɤ-nɯndzom}\hspace{5pt}\pcmn{水顺着他的头发流下来了}\end{exemple}\end{entrée}

\begin{entrée}{nɯndzɯ}{}{ⓔnɯndzɯ} 
\classe{vi} \paradigme{dir}{tɤ-}\paradigme{dir}{lɤ-}
\begin{définition}\pfra{vertical}\end{définition}
\begin{définition}\pcmn{竖}\end{définition}
\begin{définition}\pfra{mettre à la verticale}\end{définition}
\begin{définition}\pcmn{竖起来}\end{définition}
\begin{exemple}\pjya{laχtɕha lɤ-znɯndzɯ-t-a}\hspace{5pt}\pcmn{我把东西竖起来了}\end{exemple}
\begin{exemple}\pjya{ɕoŋtɕa lɤ-znɯndzɯ-t-a}\hspace{5pt}\pcmn{我把木料竖起来了}\end{exemple}\relationsémantique{同义词}{\lien{ⓔftɕhur}{ftɕhur}}
\begin{sous-entrée}{znɯndzɯ}{ⓔnɯndzɯⓝznɯndzɯ} 
\classe{vt}  
\grammaire{caus} \end{sous-entrée}

\end{entrée}

\begin{entrée}{nɯndzɯlŋɯz}{}{ⓔnɯndzɯlŋɯz} 
\classe{vi} \paradigme{dir}{pɯ-}\paradigme{dir}{thɯ-}
\begin{définition}\pfra{somnoler}\end{définition}
\begin{définition}\pcmn{打瞌睡}\end{définition}
\begin{exemple}\pjya{a-ʑɯβ ɲɯ-ɣi, pɯ-nɯndzɯlŋɯz-a}\hspace{5pt}\pcmn{我想睡了,我在打瞌睡}\end{exemple}
\begin{exemple}\pjya{ma-thɯ-tɯ-nɯndzɯlŋɯz}\hspace{5pt}\pcmn{你不要打瞌睡}\end{exemple}
\begin{sous-entrée}{ɣɤnɯndzɯlŋɯz}{ⓔnɯndzɯlŋɯzⓝɣɤnɯndzɯlŋɯz} 
\classe{vs}  
\grammaire{facil} 
\begin{définition}\pfra{somnoler facilement}\end{définition}
\begin{définition}\pcmn{容易打瞌睡}\end{définition}
\begin{exemple}\pjya{ɲɯ-ɣɤnɯndzɯlŋɯz}\hspace{5pt}\pcmn{他容易打瞌睡}\end{exemple}\end{sous-entrée}

\end{entrée}

\begin{entrée}{nɯndʐɯnbu}{}{ⓔnɯndʐɯnbu} 
\classe{vi} \paradigme{dir}{nɯ-}
\begin{définition}\pfra{partir de chez soi}\end{définition}
\begin{définition}\pcmn{出门;出差}\end{définition}
\begin{exemple}\pjya{nɤʑo kɯsthɯci ʑo kɯ-ɤrqhi kɯ-nɯndʐɯnbu jo-tɯ-ɣi}\hspace{5pt}\pcmn{你从这么远的地方出差来这里}\end{exemple}\relationsémantique{参考}{\lien{ⓔndʐɯnbu}{ndʐɯnbu}}
\begin{sous-entrée}{znɯndʐɯnbu}{ⓔnɯndʐɯnbuⓝznɯndʐɯnbu} 
\classe{vt}  
\grammaire{caus} 
\begin{définition}\pfra{faire voyager}\end{définition}
\begin{définition}\pcmn{让……出差}\end{définition}\end{sous-entrée}

\end{entrée}

\begin{entrée}{nɯni}{}{ⓔnɯni} 
\classe{dem} 
\begin{définition}\pfra{ces deux choses}\end{définition}
\begin{définition}\pcmn{那两个}\end{définition}\end{entrée}

\begin{entrée}{nɯnŋɤtʂo}{}{ⓔnɯnŋɤtʂo} 
\classe{vi}  
\grammaire{incorp} \paradigme{dir}{nɯ-}\paradigme{dir}{pɯ-}
\begin{définition}\pfra{rembourser sa dette}\end{définition}
\begin{définition}\pcmn{还债}\end{définition}
\begin{exemple}\pjya{nɯ-nɯnŋɤtʂo-a}\hspace{5pt}\pcmn{我还了债}\end{exemple}\relationsémantique{参考}{\lien{ⓔtɯ-nŋa}{tɯ-nŋa}}\relationsémantique{参考}{\lien{ⓔtʂo}{tʂo}}\end{entrée}

\begin{entrée}{nɯno}{}{ⓔnɯno} 
\classe{vt}  
\grammaire{vert} \paradigme{dir}{pɯ-}\paradigme{dir}{\_}
\begin{définition}\pfra{ramener (le bétail) à la maison}\end{définition}
\begin{définition}\pcmn{把牲畜赶回家}\end{définition}
\begin{exemple}\pjya{pɯ-nɯno-t-a}\hspace{5pt}\pcmn{我(把牲畜)赶回家了}\end{exemple}\relationsémantique{参考}{\lien{ⓔno}{no}}\end{entrée}

\begin{entrée}{nɯnthoʁnthɯɣ}{}{ⓔnɯnthoʁnthɯɣ} 
\classe{vi} \paradigme{dir}{tɤ-}\paradigme{dir}{thɯ-}
\begin{définition}\pfra{ramasser les détritus}\end{définition}
\begin{définition}\pcmn{捡废物}\end{définition}
\begin{exemple}\pjya{ɯ-thoʁ ra tɤ-nɯnthoʁnthɯɣ-a}\hspace{5pt}\pcmn{我捡了地上的垃圾}\end{exemple}
\begin{exemple}\pjya{hanɯni ɕ-tu-nɯnthoʁnthɯɣ-a nɤ}\hspace{5pt}\pcmn{我要去捡一下!}\end{exemple}
\begin{exemple}\pjya{kɤ-nɯthoʁnthɯɣ mɤ-ra}\hspace{5pt}\pcmn{不要到处捡垃圾}\end{exemple}\end{entrée}

\begin{entrée}{nɯntsho}{}{ⓔnɯntsho} 
\classe{vt}  
\grammaire{caus} \paradigme{dir}{thɯ-}\paradigme{dir}{nɯ-}\paradigme{dir}{nɯ-}
\begin{définition}\pfra{manger la viande sur les os}\end{définition}
\begin{définition}\pcmn{吃骨头上面剩下的肉}\end{définition}
\begin{exemple}\pjya{ɕɤrɯ thɯ-nɯntshɤm}\hspace{5pt}\pcmn{你把骨头上的肉吃了}\end{exemple}
\begin{exemple}\pjya{ɕɤrɯ na-nɯntsho}\hspace{5pt}\pcmn{他吃骨头上的肉}\end{exemple}
\begin{sous-entrée}{znɯntsho}{ⓔnɯntshoⓝznɯntsho} 
\classe{vt} \end{sous-entrée}

\begin{définition}\pfra{faire manger la viande sur les os}\end{définition}
\begin{définition}\pcmn{令人吃骨头上面剩下的肉}\end{définition}\end{entrée}

\begin{entrée}{nɯntsɯɣ}{}{ⓔnɯntsɯɣ} 
\classe{vt} \paradigme{dir}{tɤ-}
\begin{définition}\pfra{lécher}\end{définition}
\begin{définition}\pcmn{舔}\end{définition}
\begin{exemple}\pjya{khɯtsa tɤ-nɯntsɯɣ-a}\hspace{5pt}\pcmn{我舔了碗}\end{exemple}
\begin{exemple}\pjya{khɯna kɯ ɯ-jŋgɯ to-nɯntsɯɣ}\hspace{5pt}\pcmn{狗舔了它的碗}\end{exemple}
\begin{sous-entrée}{znɯntsɯɣ}{ⓔnɯntsɯɣⓝznɯntsɯɣ} 
\classe{vt} 
\begin{définition}\pfra{faire lécher}\end{définition}
\begin{définition}\pcmn{让……舔}\end{définition}
\begin{exemple}\pjya{tɕhɯrkɯ ta-znɯntsɯɣ}\hspace{5pt}\pcmn{让你舔狗碗(骂人的话)}\end{exemple}\end{sous-entrée}

\end{entrée}

\begin{entrée}{nɯnɯ}{₂}{ⓔnɯnɯⓗ2} 
\classe{dem} 
\begin{définition}\pfra{celà}\end{définition}
\begin{définition}\pcmn{那个}\end{définition}\end{entrée}

\begin{entrée}{nɯnɯ}{₁}{ⓔnɯnɯⓗ1} 
\classe{vt}  
\grammaire{denom} \paradigme{dir}{kɤ-}
\begin{définition}\pfra{sucer, aspirer}\end{définition}
\begin{définition}\pcmn{吸; 吸吮}\end{définition}
\begin{exemple}\pjya{chɤmdɤru kɤ-nɯnɯ-t-a}\hspace{5pt}\pcmn{我吸了坛吸管(喝坛坛酒)}\end{exemple}\relationsémantique{参考}{\lien{ⓔtɯ-nɯ}{tɯ-nɯ}}\end{entrée}

\begin{entrée}{nɯɲɤmkhe}{}{ⓔnɯɲɤmkhe} 
\classe{vs}  
\grammaire{incorp}
\grammaire{refl}
\grammaire{caus} \paradigme{dir}{nɯ-}\paradigme{dir}{nɯ-}
\begin{définition}\pfra{maigre}\end{définition}
\begin{définition}\pcmn{瘦}\end{définition}
\begin{exemple}\pjya{jiɕqha nɯ ɲɯ-nɯɲɤmkhe}\hspace{5pt}\pcmn{那个很瘦}\end{exemple}
\begin{exemple}\pjya{nɯ-fsapaʁ ɲɯ-nɯɲɤmkhe}\hspace{5pt}\pcmn{他们的牲畜很瘦}\end{exemple}
\begin{sous-entrée}{ʑɣɤznɯɲɤmkhe}{ⓔnɯɲɤmkheⓝʑɣɤznɯɲɤmkhe} 
\classe{vi} \end{sous-entrée}

\begin{définition}\pfra{se faire maigrir}\end{définition}
\begin{définition}\pcmn{令自己变瘦}\end{définition}\relationsémantique{反义词}{\lien{ⓔnɯɲɤmsɯ}{nɯɲɤmsɯ}}\relationsémantique{参考}{\lien{ⓔtɯ-ɲɤm}{tɯ-ɲɤm}}\relationsémantique{参考}{\lien{ⓔkhe}{khe}}\end{entrée}

\begin{entrée}{nɯɲɤmsɯ}{}{ⓔnɯɲɤmsɯ} 
\classe{vi}  
\grammaire{incorp} \paradigme{dir}{thɯ-}
\begin{définition}\pfra{gros, gras}\end{définition}
\begin{définition}\pcmn{肥;胖}\end{définition}
\begin{exemple}\pjya{mbala ɲɯ-nɯɲɤmsɯ}\hspace{5pt}\pcmn{牛很肥壮}\end{exemple}
\begin{exemple}\pjya{mbala cho-nɯɲɤmsɯ}\hspace{5pt}\pcmn{牛变胖了}\end{exemple}\relationsémantique{反义词}{\lien{ⓔnɯɲɤmkhe}{nɯɲɤmkhe}}\relationsémantique{参考}{\lien{ⓔtɯ-ɲɤm}{tɯ-ɲɤm}}\end{entrée}

\begin{entrée}{nɯŋa}{₁}{ⓔnɯŋaⓗ1} 
\classe{n} 
\begin{définition}\pfra{vache}\end{définition}
\begin{définition}\pcmn{母牛}\end{définition}\end{entrée}

\begin{entrée}{nɯŋa}{₂}{ⓔnɯŋaⓗ2} 
\classe{vt} \paradigme{dir}{tɤ-}
\begin{définition}\pfra{accepter de faire qqch pour qqn}\end{définition}
\begin{définition}\pcmn{答应为别人做事}\end{définition}
\begin{exemple}\pjya{tɤ-nɯŋa-t-a}\hspace{5pt}\pcmn{我答应了}\end{exemple}
\begin{exemple}\pjya{nɤ-kɯ-qur tɤ-nɯŋa-t-a}\hspace{5pt}\pcmn{我答应要帮你了}\end{exemple}\end{entrée}

\begin{entrée}{nɯŋɤdo}{}{ⓔnɯŋɤdo} 
\classe{n} 
\begin{définition}\pfra{vieille vache}\end{définition}
\begin{définition}\pcmn{老奶牛}\end{définition}\end{entrée}

\begin{entrée}{nɯŋgu}{}{ⓔnɯŋgu} 
\classe{vs} \paradigme{dir}{\_}\paradigme{dir}{\_}\paradigme{dir}{tɤ-}
\begin{définition}\pfra{prématuré}\end{définition}
\begin{définition}\pcmn{过早}\end{définition}
\begin{définition}\pfra{faire de façon prématurée}\end{définition}
\begin{définition}\pcmn{做得太早}\end{définition}
\begin{exemple}\pjya{saχsɯ to-nɯŋgu}\hspace{5pt}\pcmn{午餐吃得太早}\end{exemple}
\begin{exemple}\pjya{kɤ-ji lo-nɯŋgu}\hspace{5pt}\pcmn{他种得太早}\end{exemple}
\begin{exemple}\pjya{(tɤ-rɤku) kɤ-phɯt ko-nɯŋgu}\hspace{5pt}\pcmn{(庄稼)割得太早}\end{exemple}
\begin{exemple}\pjya{jɯxɕo aʑo kɤ-ɣi ko-nɯŋgu-a}\hspace{5pt}\pcmn{我今天早上来早了}\end{exemple}
\begin{exemple}\pjya{tɤ-znɯŋgu-t-a}\hspace{5pt}\pcmn{我做得太早了}\end{exemple}
\begin{sous-entrée}{znɯŋgu}{ⓔnɯŋguⓝznɯŋgu} 
\classe{vt} \end{sous-entrée}

\end{entrée}

\begin{entrée}{nɯŋgɤkhe}{}{ⓔnɯŋgɤkhe} 
\classe{vi} 
\begin{définition}\pfra{porter habituellement de vieux habits}\end{définition}
\begin{définition}\pcmn{(习惯)穿破旧的衣服}\end{définition}
\begin{exemple}\pjya{aʑo nɯŋgɤkhe-a}\hspace{5pt}\pcmn{我习惯穿破烂的衣服}\end{exemple}\relationsémantique{参考}{\lien{ⓔtɯ-ŋga}{tɯ-ŋga}}\relationsémantique{参考}{\lien{ⓔkhe}{khe}}\end{entrée}

\begin{entrée}{nɯŋgɤxtsa}{}{ⓔnɯŋgɤxtsa} 
\classe{vt} \paradigme{dir}{tɤ-}
\begin{définition}\pfra{s'habiller richement, être prêt à faire des dépenses dans les habits}\end{définition}
\begin{définition}\pcmn{穿得很豪华}\end{définition}
\begin{exemple}\pjya{tɤ-nɯŋgɤxtsa-t-a}\hspace{5pt}\pcmn{我舍得穿了}\end{exemple}
\begin{exemple}\pjya{sɯŋgi kɤ-nɯŋgɤxtsa cha, mgɯnbu mɤ-nɯŋgɤxtse}\hspace{5pt}\pcmn{僧吉舍得穿,袞布不舍得}\end{exemple}\relationsémantique{参考}{\lien{ⓔtɯ-ŋga}{tɯ-ŋga}}\relationsémantique{参考}{\lien{ⓔtɯ-xtsa}{tɯ-xtsa}}\end{entrée}

\begin{entrée}{nɯŋgumdʑɯɣ}{}{ⓔnɯŋgumdʑɯɣ} 
\classe{vi} \paradigme{dir}{lɤ-}
\begin{définition}\pfra{devenir chef}\end{définition}
\begin{définition}\pcmn{当领导}\end{définition}
\begin{exemple}\pjya{lo-nɯŋgumdʑɯɣ}\hspace{5pt}\pcmn{他当了领导}\end{exemple}\relationsémantique{参考}{\lien{ⓔŋgumdʑɯɣ}{ŋgumdʑɯɣ}}\end{entrée}

\begin{entrée}{nɯŋgumtha}{}{ⓔnɯŋgumtha} 
\classe{vt} \paradigme{dir}{tɤ-}
\begin{définition}\pfra{s'occuper de}\end{définition}
\begin{définition}\pcmn{照顾}\end{définition}
\begin{exemple}\pjya{tɤ-pɤtso tɤ-nɯŋgumtha-t-a}\hspace{5pt}\pcmn{我照顾了孩子}\end{exemple}
\begin{exemple}\pjya{rgargɯn ɲɯ-ŋu tɕe tɤ-nɯŋgumtha-t-a}\hspace{5pt}\pcmn{我照顾了老人家}\end{exemple}
\begin{exemple}\pjya{ɲɯ-ɲɯ-βzi nɤ kɯpɯpe tɤ-nɯŋgumthe}\hspace{5pt}\pcmn{如果他醉了的话,请你好好照顾他!}\end{exemple}\end{entrée}

\begin{entrée}{nɯŋgra}{}{ⓔnɯŋgra} 
\classe{vi}  
\grammaire{denom}
\grammaire{caus} \paradigme{dir}{tɤ-}\paradigme{dir}{tɤ-}
\begin{définition}\pfra{être payé pour un travail}\end{définition}
\begin{définition}\pcmn{拿到工钱}\end{définition}
\begin{exemple}\pjya{laχtɕha kɤ-tsɯm tɤ-nɯŋgra-a}\hspace{5pt}\pcmn{我把东西拿去了,得到了工钱}\end{exemple}
\begin{sous-entrée}{znɯŋgra}{ⓔnɯŋgraⓝznɯŋgra} 
\classe{vt} \end{sous-entrée}

\sens{1}
\begin{définition}\pfra{engager}\end{définition}
\begin{définition}\pcmn{雇佣}\end{définition}
\begin{exemple}\pjya{ɯʑo kɯ tɯrme ci a-tɤ-znɯŋgre ɲɯ-ntshi}\hspace{5pt}\pcmn{他只好雇佣人}\end{exemple}\sens{2}
\begin{définition}\pfra{louer}\end{définition}
\begin{définition}\pcmn{租(房子)}\end{définition}
\begin{exemple}\pjya{kha ci to-znɯŋgra}\hspace{5pt}\pcmn{他租了房子}\end{exemple}\relationsémantique{参考}{\lien{ⓔtɯ-ŋgra}{tɯ-ŋgra}}\end{entrée}

\begin{entrée}{nɯŋgurtɕaʁ}{}{ⓔnɯŋgurtɕaʁ} 
\classe{vt} \paradigme{dir}{pɯ-}
\begin{définition}\pfra{coudre selon un type de pas d'aiguille}\end{définition}
\begin{définition}\pcmn{缝针的方法}\end{définition}\relationsémantique{参考}{\lien{ⓔŋgurtɕaʁ}{ŋgurtɕaʁ}}\end{entrée}

\begin{entrée}{nɯŋke}{}{ⓔnɯŋke} 
\classe{vt}  
\grammaire{appl} \paradigme{dir}{pɯ-}\paradigme{dir}{\_}
\begin{définition}\pfra{aller pour faire quelque chose}\end{définition}
\begin{définition}\pcmn{到处走做某件事情}\end{définition}
\begin{exemple}\pjya{@yangyu kɤ-χtɯ ɕ-pɯ-nɯŋke-t-a}\hspace{5pt}\pcmn{我为了去买土豆走了一趟}\end{exemple}
\begin{exemple}\pjya{tɯ-ŋga kɤ-χtɯ ɕ-pɯ-nɯŋke-t-a}\hspace{5pt}\pcmn{我为了去买衣服走了一趟}\end{exemple}
\begin{exemple}\pjya{smɤnba ɕ-pɯ-nɯŋke-t-a}\hspace{5pt}\pcmn{我为了找医生走了一趟}\end{exemple}\relationsémantique{参考}{\lien{ⓔŋke}{ŋke}}\end{entrée}

\begin{entrée}{nɯŋumit}{}{ⓔnɯŋumit} 
\classe{vt} \paradigme{dir}{tɤ-}
\begin{définition}\pfra{humilier}\end{définition}
\begin{définition}\pcmn{侮辱;欺负}\end{définition}
\begin{exemple}\pjya{jiɕqha tɤ-pɤtso nɯ ɲɯ-ɤz-nɯŋumit-nɯ}\hspace{5pt}\pcmn{他们在欺负那个小孩子}\end{exemple}
\begin{exemple}\pjya{jiɕqha tɯrme ɲɯ-ɤz-nɯŋumit-nɯ}\hspace{5pt}\pcmn{他们在欺负那个人}\end{exemple}
\begin{exemple}\pjya{mɤ-ta-nɤkhe, mɤ-ta-nɯŋumit}\hspace{5pt}\pcmn{我不会欺负你的}\end{exemple}
\begin{sous-entrée}{sɤnɯŋumit}{ⓔnɯŋumitⓝsɤnɯŋumit} 
\classe{vi}  
\grammaire{apass} 
\begin{définition}\pfra{humilier les gens}\end{définition}
\begin{définition}\pcmn{侮辱人}\end{définition}\end{sous-entrée}

\étymologie{ŋo.med}\end{entrée}

\begin{entrée}{nɯŋundʑu}{}{ⓔnɯŋundʑu} 
\classe{vt} \paradigme{dir}{tɤ-}\sens{1}
\begin{définition}\pfra{attirer (animal)}\end{définition}
\begin{définition}\pcmn{引过来(动物)}\end{définition}
\begin{exemple}\pjya{jla tɤ-nɯŋundʑu-t-a}\hspace{5pt}\pcmn{我把犏牛引过来了(用盐)}\end{exemple}\sens{2}
\begin{définition}\pfra{calmer, apaiser (quelqu'un qui est fâché)}\end{définition}
\begin{définition}\pcmn{说几句好话,令别人没有那么生气}\end{définition}
\begin{exemple}\pjya{ɯ-mbrɯ ɲɯ-ŋgɯ tɕe, tɤ-nɯŋundʑu-t-a tɕe nɯ-ʑi}\hspace{5pt}\pcmn{我说了几句好话,他就平静下来了}\end{exemple}
\begin{exemple}\pjya{tɤ́-wɣ-nɯŋundʑu-a}\hspace{5pt}\pcmn{他跟我说了几句好话}\end{exemple}\relationsémantique{参考}{\lien{ⓔrɯŋundʑu}{rɯŋundʑu}}\end{entrée}

\begin{entrée}{nɯɴɢɯlɯjɤt}{}{ⓔnɯɴɢɯlɯjɤt} 
\classe{vi} \paradigme{dir}{nɯ-}
\begin{définition}\pfra{se séparer}\end{définition}
\begin{définition}\pcmn{分散;走散}\end{définition}
\begin{exemple}\pjya{nɯ-nɯɴɢɯlɯjɤt-i}\hspace{5pt}\pcmn{我们走散了}\end{exemple}\relationsémantique{同义词}{\lien{ⓔɴɢɤt}{ɴɢɤt}}\end{entrée}

\begin{entrée}{nɯɴqhu}{}{ⓔnɯɴqhu} 
\classe{vt}  
\grammaire{denom} \paradigme{dir}{\_}\paradigme{dir}{pɯ-}\paradigme{dir}{nɯ-}
\begin{définition}\pfra{suivre}\end{définition}
\begin{définition}\pcmn{跟踪(偷偷地)}\end{définition}
\begin{définition}\pfra{suivre, se conformer à}\end{définition}
\begin{définition}\pcmn{照办}\end{définition}
\begin{exemple}\pjya{kɤ-anɯri tɕe ɯ-qhu kɤ-nɯɴqhu-t-a}\hspace{5pt}\pcmn{我回去了,我就跟踪了他}\end{exemple}
\begin{exemple}\pjya{nɤʑo nɤ-kɤti nɯ pjɯ-znɯɴqhe-a ŋu}\hspace{5pt}\pcmn{我依着你的说法去做}\end{exemple}\relationsémantique{参考}{\lien{ⓔɯ-qhu}{ɯ-qhu}}\relationsémantique{同义词}{\lien{ⓔznɯjɯn}{znɯjɯn}}
\begin{sous-entrée}{znɯɴqhu}{ⓔnɯɴqhuⓝznɯɴqhu} 
\classe{vt} \end{sous-entrée}

\end{entrée}

\begin{entrée}{nɯpa}{}{ⓔnɯpa}\relationsémantique{参考}{\lien{ⓔpaⓗ3}{pa}}\end{entrée}

\begin{entrée}{nɯpaʁlɤɣ}{}{ⓔnɯpaʁlɤɣ} 
\classe{vi}  
\grammaire{incorp} \paradigme{dir}{nɯ-}
\begin{définition}\pfra{laisser sortir un cochon}\end{définition}
\begin{définition}\pcmn{放猪}\end{définition}
\begin{exemple}\pjya{nɯ-nɯpaʁlɤɣ}\hspace{5pt}\pcmn{他放了猪}\end{exemple}
\begin{exemple}\pjya{nɯ-nɯpaʁlaɣ-a}\hspace{5pt}\pcmn{我放了猪}\end{exemple}\relationsémantique{参考}{\lien{ⓔpaʁ}{paʁ}}\relationsémantique{参考}{\lien{ⓔlɤɣ}{lɤɣ}}\end{entrée}

\begin{entrée}{nɯpaχɕi}{}{ⓔnɯpaχɕi} 
\classe{vi} 
\begin{définition}\pfra{aller cueillir des pommes}\end{définition}
\begin{définition}\pcmn{采苹果}\end{définition}\relationsémantique{参考}{\lien{}{paχɕi}}\end{entrée}

\begin{entrée}{nɯpɤŋgɯŋgru}{}{ⓔnɯpɤŋgɯŋgru} 
\classe{vs} \paradigme{dir}{nɯ-}
\begin{définition}\pfra{avoir une crampe}\end{définition}
\begin{définition}\pcmn{抽筋}\end{définition}
\begin{exemple}\pjya{a-mi ɲɯ-nɯpɤŋgɯŋgru ɲɯ-ŋu}\hspace{5pt}\pcmn{我的脚经常抽筋}\end{exemple}
\begin{exemple}\pjya{nɤ-mi nɯ-nɯpɤŋgɯŋgru tɕe, ɯ-thoʁ pɯ-te tɕe pɯ-sthoʁ ʑo tɕe phɤn}\hspace{5pt}\pcmn{脚抽筋的时候,把脚板着地,使劲地蹬就会好}\end{exemple}\relationsémantique{参考}{\lien{ⓔtɯ-ŋgru}{tɯ-ŋgru}}\end{entrée}

\begin{entrée}{nɯpɤɴqi}{}{ⓔnɯpɤɴqi} 
\classe{vs} \paradigme{dir}{nɯ-}
\begin{définition}\pfra{paresseux}\end{définition}
\begin{définition}\pcmn{懒}\end{définition}\relationsémantique{参考}{\lien{ⓔnɤɴqi}{nɤɴqi}}\end{entrée}

\begin{entrée}{nɯpɕɯru}{}{ⓔnɯpɕɯru} 
\classe{vs} 
\begin{définition}\pfra{agréable à regarder}\end{définition}
\begin{définition}\pcmn{外表好看}\end{définition}
\begin{exemple}\pjya{ɯ-ʁzɯɣ ndɤre ɲɯ-nɯpɕɯru}\hspace{5pt}\pcmn{她的外表倒是美观}\end{exemple}\relationsémantique{参考}{\lien{ⓔruⓗ1}{ru₁}}\end{entrée}

\begin{entrée}{nɯpɣa}{}{ⓔnɯpɣa} 
\classe{vi} \paradigme{dir}{pɯ-}
\begin{définition}\pfra{chasser des oiseaux}\end{définition}
\begin{définition}\pcmn{打鸟}\end{définition}
\begin{exemple}\pjya{ɯʑo ɕ-pɯ-nɯpɣa}\hspace{5pt}\pcmn{他去打鸟了}\end{exemple}\relationsémantique{参考}{\lien{ⓔpɣa}{pɣa}}\end{entrée}

\begin{entrée}{nɯpɣɤɲaʁ}{}{ⓔnɯpɣɤɲaʁ} 
\classe{vi} \paradigme{dir}{pɯ-}
\begin{définition}\pfra{chasser le faisan}\end{définition}
\begin{définition}\pcmn{打勺鸡}\end{définition}
\begin{exemple}\pjya{ɕ-pɯ-nɯpɣɤɲaʁ-a}\hspace{5pt}\pcmn{我去打勺鸡}\end{exemple}\relationsémantique{参考}{\lien{ⓔpɣɤɲaʁ}{pɣɤɲaʁ}}\end{entrée}

\begin{entrée}{nɯphu}{}{ⓔnɯphu} 
\classe{vi} \paradigme{dir}{tɤ-}
\begin{définition}\pfra{s'accoupler}\end{définition}
\begin{définition}\pcmn{交配}\end{définition}\end{entrée}

\begin{entrée}{nɯphaʁɲɤl}{}{ⓔnɯphaʁɲɤl} 
\classe{vi}  
\grammaire{incorp} \paradigme{dir}{nɯ-}\paradigme{dir}{\_}
\begin{définition}\pfra{s’allonger}\end{définition}
\begin{définition}\pcmn{躺}\end{définition}
\begin{exemple}\pjya{ma-nɯ-tɯ-nɯphaʁɲɤl}\hspace{5pt}\pcmn{你不要躺下}\end{exemple}
\begin{exemple}\pjya{ɲɯ-tɯ-nɤɴqi tɕe ɲɯ-tɯ-nɯphaʁɲɤl}\hspace{5pt}\pcmn{你很懒,你在那里躺着}\end{exemple}
\begin{exemple}\pjya{pɯ-tɯ-nɯphaʁɲɤl ntsɯ}\hspace{5pt}\pcmn{你总是在那里躺着}\end{exemple}
\begin{exemple}\pjya{tɤ-tɕɯ nɯ ɯ-rkɯ nɯ tɕu pjɤ-nɯphaʁɲɤl}\hspace{5pt}\pcmn{那个男子在旁边躺着}\end{exemple}\relationsémantique{参考}{\lien{ⓔphaʁɲɤl}{phaʁɲɤl}}\end{entrée}

\begin{entrée}{nɯphawu}{}{ⓔnɯphawu} 
\classe{vt} \paradigme{dir}{nɯ-}
\begin{définition}\pfra{dépendre de, profiter (du l'influence d'autres gens)}\end{définition}
\begin{définition}\pcmn{依靠别人;借别人的势力}\end{définition}
\begin{exemple}\pjya{ɯ-βɣo ɲɯ-ɤz-nɯphawu}\hspace{5pt}\pcmn{他在依赖他的伯父}\end{exemple}
\begin{exemple}\pjya{ɯ-wa ɲɯ-ɤz-nɯphawu}\hspace{5pt}\pcmn{他在依赖他的父亲}\end{exemple}
\begin{exemple}\pjya{tɤru ɲɯ-ɤz-nɯphawu}\hspace{5pt}\pcmn{他在依赖头人}\end{exemple}
\begin{exemple}\pjya{nɤ-wa ma-nɯ-tɯ-nɯphawe}\hspace{5pt}\pcmn{你不要依赖你的父亲}\end{exemple}\end{entrée}

\begin{entrée}{nɯphɯ}{}{ⓔnɯphɯ} 
\classe{vs} \paradigme{dir}{tɤ-}
\begin{définition}\pfra{d'un prix convenable}\end{définition}
\begin{définition}\pcmn{价格合适}\end{définition}
\begin{exemple}\pjya{ɲɯ-nɯphɯ}\hspace{5pt}\pcmn{价格合适}\end{exemple}
\begin{exemple}\pjya{ki ɯ-phɯ ɯ-tɯ-wxti nɯ nɯ sthɯci ndɤre mɯ́j-nɯphɯ}\hspace{5pt}\pcmn{这个东西很贵,价格太高了}\end{exemple}\relationsémantique{参考}{\lien{ⓔɯ-phɯ}{ɯ-phɯ}}\end{entrée}

\begin{entrée}{nɯphɯrɤm}{}{ⓔnɯphɯrɤm} 
\classe{vt} \paradigme{dir}{thɯ-}\paradigme{dir}{thɯ-}
\begin{définition}\pfra{herser}\end{définition}
\begin{définition}\pcmn{耙地}\end{définition}
\begin{définition}\pfra{herser}\end{définition}
\begin{définition}\pcmn{耙地}\end{définition}
\begin{exemple}\pjya{ki tɯji ki thɯ-nɯphɯram-a}\hspace{5pt}\pcmn{我耙了这块地}\end{exemple}\relationsémantique{参考}{\lien{ⓔphɯrɤm}{phɯrɤm}}
\begin{sous-entrée}{rɯphɯrɤm}{ⓔnɯphɯrɤmⓝrɯphɯrɤm} 
\classe{vi} \end{sous-entrée}

\end{entrée}

\begin{entrée}{nɯpjaχpa}{}{ⓔnɯpjaχpa} 
\classe{vt} \paradigme{dir}{tɤ-}\sens{1}
\begin{définition}\pfra{tenir sous l'aisselle}\end{définition}
\begin{définition}\pcmn{夹在腋下}\end{définition}
\begin{exemple}\pjya{ɯʑo kɯ jɯɣi to-nɯpjaχpa}\hspace{5pt}\pcmn{他把书夹在腋下}\end{exemple}\sens{2}
\begin{définition}\pfra{en profiter pour prendre}\end{définition}
\begin{définition}\pcmn{顺便带}\end{définition}\relationsémantique{同义词}{\lien{ⓔnɤxtʂɯ}{nɤxtʂɯ}}\end{entrée}

\begin{entrée}{nɯpodɯdi}{}{ⓔnɯpodɯdi} 
\classe{vs}  
\grammaire{caus} \paradigme{dir}{nɯ-}
\begin{définition}\pfra{chatouiller}\end{définition}
\begin{définition}\pcmn{腰、胳肢窝发痒【入劲】}\end{définition}
\begin{sous-entrée}{znɯpodɯdi}{ⓔnɯpodɯdiⓝznɯpodɯdi} 
\classe{vt} \end{sous-entrée}

\begin{définition}\pfra{chatouiller}\end{définition}
\begin{définition}\pcmn{挠(别人)痒}\end{définition}
\begin{exemple}\pjya{nɯ́-wɣ-znɯpodɯdi-a}\hspace{5pt}\pcmn{他挠我痒痒了}\end{exemple}
\begin{exemple}\pjya{nɯ-znɯpodɯdi-t-a}\hspace{5pt}\pcmn{我挠他痒痒了}\end{exemple}\end{entrée}

\begin{entrée}{nɯpolɯli}{}{ⓔnɯpolɯli} 
\classe{vi} \paradigme{dir}{thɯ-}
\begin{définition}\pfra{s'allonger sur le ventre}\end{définition}
\begin{définition}\pcmn{俯卧,趴}\end{définition}
\begin{exemple}\pjya{khɤxtu zɯ chɤ-nɯpolɯli}\hspace{5pt}\pcmn{他趴在房背上了}\end{exemple}
\begin{exemple}\pjya{stɤmku ri chɤ-nɯpolɯli}\hspace{5pt}\pcmn{他趴在草地上了}\end{exemple}\end{entrée}

\begin{entrée}{nɯpoʁ}{}{ⓔnɯpoʁ} 
\classe{vt} \paradigme{dir}{kɤ-}\paradigme{dir}{kɤ-}
\begin{définition}\pfra{embrasser}\end{définition}
\begin{définition}\pcmn{亲吻}\end{définition}
\begin{définition}\pfra{laisser embrasser}\end{définition}
\begin{définition}\pcmn{让……亲吻}\end{définition}
\begin{exemple}\pjya{ɯ-rʑaβ ka-nɯpoʁ}\hspace{5pt}\pcmn{他把妻子亲了一下了}\end{exemple}
\begin{exemple}\pjya{tɤ-pɤtso ka-nɯpoʁ}\hspace{5pt}\pcmn{他把孩子亲了一下}\end{exemple}
\begin{exemple}\pjya{@waiguoren ra kɤ-nɯpoʁ rga-nɯ}\hspace{5pt}\pcmn{外国人喜欢亲嘴}\end{exemple}
\begin{exemple}\pjya{kɤ́-wɣ-nɯpoʁ-a}\hspace{5pt}\pcmn{他亲了我一下}\end{exemple}
\begin{sous-entrée}{znɯpoʁ}{ⓔnɯpoʁⓝznɯpoʁ} 
\classe{vt} \end{sous-entrée}

\end{entrée}

\begin{entrée}{nɯproʁmba}{}{ⓔnɯproʁmba} 
\classe{vl} \paradigme{dir}{tɤ-}
\begin{définition}\pfra{imiter les gestes}\end{définition}
\begin{définition}\pcmn{学别人的动作}\end{définition}
\begin{exemple}\pjya{ma-tɤ-kɯ-nɯproʁmba-a}\hspace{5pt}\pcmn{你不要学我!}\end{exemple}\end{entrée}

\begin{entrée}{nɯpɯmɲɯɣ}{}{ⓔnɯpɯmɲɯɣ} 
\classe{vi} \paradigme{dir}{tɤ-}
\begin{définition}\pfra{viser}\end{définition}
\begin{définition}\pcmn{瞄准}\end{définition}
\begin{exemple}\pjya{tɤ-nɯpɯmɲɯɣ-a}\hspace{5pt}\pcmn{我瞄准了}\end{exemple}\end{entrée}

\begin{entrée}{nɯqaɕti}{}{ⓔnɯqaɕti} 
\classe{vi} \paradigme{dir}{\_}
\begin{définition}\pfra{aller chercher des pêches}\end{définition}
\begin{définition}\pcmn{捡桃子}\end{définition}\relationsémantique{参考}{\lien{ⓔqaɕti}{qaɕti}}\end{entrée}

\begin{entrée}{nɯqajɯ}{}{ⓔnɯqajɯ} 
\classe{vi} \paradigme{dir}{pɯ-}\paradigme{dir}{lɤ-}\paradigme{dir}{nɯ-}
\begin{définition}\pfra{chercher des vers}\end{définition}
\begin{définition}\pcmn{找虫子}\end{définition}\relationsémantique{同义词}{\lien{ⓔnɯqandʐe}{nɯqandʐe}}\relationsémantique{参考}{\lien{ⓔrɯqajɯ}{rɯqajɯ}}\relationsémantique{参考}{\lien{ⓔqajɯ}{qajɯ}}\end{entrée}

\begin{entrée}{nɯqaɟy}{}{ⓔnɯqaɟy} 
\classe{vi}  
\grammaire{denom} \paradigme{dir}{lɤ-}\paradigme{dir}{pɯ-}
\begin{définition}\pfra{pêcher du poisson}\end{définition}
\begin{définition}\pcmn{钓鱼}\end{définition}
\begin{exemple}\pjya{pɯ-nɯqaɟya}\hspace{5pt}\pcmn{我钓了鱼}\end{exemple}
\begin{exemple}\pjya{ɲɯ-nɯqaɟy}\hspace{5pt}\pcmn{他在钓鱼}\end{exemple}
\begin{exemple}\pjya{ma-tɯ-nɯqaɟy-nɯ ma mɤj-ɤɣ}\hspace{5pt}\pcmn{不准钓鱼}\end{exemple}\relationsémantique{参考}{\lien{ⓔqaɟy}{qaɟy}}\end{entrée}

\begin{entrée}{nɯqambɯmbjom}{}{ⓔnɯqambɯmbjom} 
\classe{vi} \paradigme{dir}{\_}
\begin{définition}\pfra{voler}\end{définition}
\begin{définition}\pcmn{飞}\end{définition}
\begin{exemple}\pjya{tsɯʁot thɯ-nɯqambɯmbjom}\hspace{5pt}\pcmn{野鸡飞走了}\end{exemple}
\begin{exemple}\pjya{pɣa thɯ-nɯqambɯmbjom}\hspace{5pt}\pcmn{鸟飞走了}\end{exemple}\end{entrée}

\begin{entrée}{nɯqandʐe}{}{ⓔnɯqandʐe} 
\classe{vi} \paradigme{dir}{tɤ-}\paradigme{dir}{pɯ-}\paradigme{dir}{nɯ-}
\begin{définition}\pfra{chercher des vers de terre}\end{définition}
\begin{définition}\pcmn{找蚯蚓}\end{définition}\relationsémantique{参考}{\lien{ⓔqandʐe}{qandʐe}}\end{entrée}

\begin{entrée}{nɯqarma}{}{ⓔnɯqarma} 
\classe{vi} \paradigme{dir}{pɯ-}
\begin{définition}\pfra{chasser des crossoptilons}\end{définition}
\begin{définition}\pcmn{打马鸡}\end{définition}
\begin{exemple}\pjya{ɯʑo ɕ-pɯ-nɯqarma}\hspace{5pt}\pcmn{他去打马鸡了}\end{exemple}\relationsémantique{参考}{\lien{ⓔqarma}{qarma}}\end{entrée}

\begin{entrée}{nɯqhaɴɢaʁ}{}{ⓔnɯqhaɴɢaʁ} 
\classe{vi}  
\grammaire{incorp} \sens{1}\paradigme{dir}{pɯ-}
\begin{définition}\pfra{tomber en arrière, se pencher vers l'arrière}\end{définition}
\begin{définition}\pcmn{往后仰;往后摔下来(人)}\end{définition}
\begin{exemple}\pjya{pɯ-nɯqhaɴɢaʁ-a tɕe pɯ-ndʐaβ-a}\hspace{5pt}\pcmn{我往后一仰就摔倒了}\end{exemple}
\begin{exemple}\pjya{rɟɤlpu tɯ-ɲɟɤt kɯ ɯ-qhu pjɤ-nɯqhaɴɢaʁ tɕe pjɤ-si}\hspace{5pt}\pcmn{国王后悔不已,往后仰摔下来了就死了}\end{exemple}\sens{2}\paradigme{dir}{\_}
\begin{définition}\pfra{s'écrouler vers l'arrière}\end{définition}
\begin{définition}\pcmn{往后倒(房子)}\end{définition}
\begin{exemple}\pjya{kha nɯ-nɯqhaɴɢaʁ (tɕe pjɯ-ndʐaβ ɕti)}\hspace{5pt}\pcmn{房子倒塌了}\end{exemple}\relationsémantique{参考}{\lien{ⓔɯ-qhu}{ɯ-qhu}}\end{entrée}

\begin{entrée}{nɯqhapa}{}{ⓔnɯqhapa} 
\classe{vi} 
\begin{définition}\pfra{surveiller la maison (de quelqu'un d'autre)}\end{définition}
\begin{définition}\pcmn{看家(别人的)}\end{définition}
\begin{exemple}\pjya{aʑo ɲɯ-nɯqhapa-a}\hspace{5pt}\pcmn{我在看家}\end{exemple}\end{entrée}

\begin{entrée}{nɯqhaχɕu}{}{ⓔnɯqhaχɕu}\relationsémantique{参考}{\lien{ⓔrɯqhaχɕu}{rɯqhaχɕu}}\end{entrée}

\begin{entrée}{nɯqhɤcit}{}{ⓔnɯqhɤcit} 
\classe{vi}  
\grammaire{incorp} \paradigme{dir}{nɯ-}
\begin{définition}\pfra{reculer}\end{définition}
\begin{définition}\pcmn{后退}\end{définition}
\begin{exemple}\pjya{kɤ-nɤma thamtɕɤt nɯ pɯ-nnɯ-sɤɣdɯɣ kɯnɤ tu-kɯ-stu tu-kɯ-mbat ra ma ɲɯ-kɯ-nɯqhɤcit tɕe mɤ-pe}\hspace{5pt}\pcmn{工作无论再艰苦都要坚持下去,不要退却}\end{exemple}\relationsémantique{参考}{\lien{ⓔcit}{cit}}\relationsémantique{参考}{\lien{ⓔɯ-qhu}{ɯ-qhu}}\end{entrée}

\begin{entrée}{nɯqhɤjŋɯjŋa}{}{ⓔnɯqhɤjŋɯjŋa} 
\classe{vi}  
\grammaire{incorp} \paradigme{dir}{nɯ-}
\begin{définition}\pfra{se tenir très droit, avoir la tête presque courbée vers l'arrière}\end{définition}
\begin{définition}\pcmn{挺着身子仰着头;头往后仰;趾高气昂}\end{définition}
\begin{exemple}\pjya{jiɕqha tɯrme ɲɯ-nɯqhɤjŋɯjŋa}\hspace{5pt}\pcmn{那个人把身体挺得很直}\end{exemple}
\begin{exemple}\pjya{tɕheme nɯ ɲɯ-znɤjpɯjpe ɲɯ-nɯqhɤjŋɯjŋa}\hspace{5pt}\pcmn{那个女孩子趾高气昂}\end{exemple}
\begin{exemple}\pjya{tɕheme nɯ ɲɯ-rɯɕaŋchi tɕe ɲɯ-nɯqhɤjŋɯjŋa}\hspace{5pt}\pcmn{那个女孩子摇摆弄姿,趾高气昂}\end{exemple}\relationsémantique{参考}{\lien{ⓔɯ-qhu}{ɯ-qhu}}\end{entrée}

\begin{entrée}{nɯqhɤstɯstu}{}{ⓔnɯqhɤstɯstu} 
\classe{vi}  
\grammaire{incorp} \paradigme{dir}{\_}\paradigme{dir}{\_}
\begin{définition}\pfra{reculer}\end{définition}
\begin{définition}\pcmn{退后}\end{définition}
\begin{définition}\pfra{faire reculer}\end{définition}
\begin{définition}\pcmn{使……退后}\end{définition}
\begin{exemple}\pjya{kɤ-nɯqhɤstɯstu-a}\hspace{5pt}\pcmn{我后退了}\end{exemple}
\begin{exemple}\pjya{qapri pjɤ-mto tɕe kɤ-nɯqhɤstɯstu}\hspace{5pt}\pcmn{他看到蛇就后退了}\end{exemple}\relationsémantique{参考}{\lien{ⓔɯ-qhu}{ɯ-qhu}}\relationsémantique{参考}{\lien{ⓔastu}{astu}}
\begin{sous-entrée}{znɯqhɤstɯstu}{ⓔnɯqhɤstɯstuⓝznɯqhɤstɯstu} 
\classe{vt}  
\grammaire{caus} \end{sous-entrée}

\end{entrée}

\begin{entrée}{nɯqhoχɕɤr}{}{ⓔnɯqhoχɕɤr} 
\classe{vi} \paradigme{dir}{pɯ-}
\begin{définition}\pfra{avoir une diarrhée}\end{définition}
\begin{définition}\pcmn{拉肚子}\end{définition}\relationsémantique{同义词}{\lien{ⓔnɯtɯfɕɤl}{nɯtɯfɕɤl}}\end{entrée}

\begin{entrée}{nɯqru}{}{ⓔnɯqru} 
\classe{vt}  
\grammaire{vert} 
\begin{définition}\pfra{ramener à la maison}\end{définition}
\begin{définition}\pcmn{接回来}\end{définition}
\begin{exemple}\pjya{kɯmaʁ sɤtɕha pjɤ-rɤʑi tɕeri, wuma ʑo pjɤ-nɯmbɣom tɕe z-jo-nɯqru}\hspace{5pt}\pcmn{(他女儿)在另外一个地方,很想念她,就把她接回来了}\end{exemple}\relationsémantique{参考}{\lien{ⓔqru}{qru}}\end{entrée}

\begin{entrée}{nɯqro}{}{ⓔnɯqro} 
\classe{vi} \paradigme{dir}{lɤ-}\paradigme{dir}{tu-}
\begin{définition}\pfra{chercher des fourmis}\end{définition}
\begin{définition}\pcmn{找蚂蚁(熊)}\end{définition}
\begin{exemple}\pjya{pri lu-nɯqro ŋu}\hspace{5pt}\pcmn{熊在找蚂蚁}\end{exemple}\relationsémantique{参考}{\lien{ⓔqroⓗ2}{qro₂}}\end{entrée}

\begin{entrée}{nɯqɯqoʁ}{}{ⓔnɯqɯqoʁ} 
\classe{vi} \paradigme{dir}{nɯ-}
\begin{définition}\pfra{utiliser toutes ses forces}\end{définition}
\begin{définition}\pcmn{用尽全身的力气}\end{définition}
\begin{exemple}\pjya{ta-ma ɯ-tshɤt ɯ-tsa ra ma kɤ-nɯqɯqoʁ mɤ-βdi}\hspace{5pt}\pcmn{劳动不要太过卖力}\end{exemple}\end{entrée}

\begin{entrée}{nɯra/\variante{nɯnɯra}}{}{ⓔnɯra} 
\classe{dem} 
\begin{définition}\pfra{ces choses}\end{définition}
\begin{définition}\pcmn{那些}\end{définition}\end{entrée}

\begin{entrée}{nɯrɤɣo}{}{ⓔnɯrɤɣo} 
\classe{vi} \paradigme{dir}{thɯ-}
\begin{définition}\pfra{chanter}\end{définition}
\begin{définition}\pcmn{唱}\end{définition}
\begin{exemple}\pjya{jiɕqha nɯ kɤ-nɯrɤɣo rga}\hspace{5pt}\pcmn{那个人喜欢唱歌}\end{exemple}
\begin{exemple}\pjya{thɯ-nɯrɤɣo-a (=rɤɣo thɯ-βzu-t-a)}\hspace{5pt}\pcmn{他唱了歌}\end{exemple}\relationsémantique{参考}{\lien{ⓔrɤɣo}{rɤɣo}}\end{entrée}

\begin{entrée}{nɯrɤŋom}{}{ⓔnɯrɤŋom} 
\classe{vt} \paradigme{dir}{nɯ-}
\begin{définition}\pfra{subir un outrage}\end{définition}
\begin{définition}\pcmn{受气}\end{définition}
\begin{exemple}\pjya{na-nɯrɤŋom}\hspace{5pt}\pcmn{他受了气}\end{exemple}
\begin{exemple}\pjya{tɤ-nɤmqe-t-a tɕe ɲɯ-nɯrɤŋom}\hspace{5pt}\pcmn{我骂了他,他就受气}\end{exemple}\relationsémantique{参考}{\lien{ⓔrɤŋom}{rɤŋom}}\end{entrée}

\begin{entrée}{nɯrɤscoz}{}{ⓔnɯrɤscoz}\relationsémantique{参考}{\lien{ⓔrɤscoz}{rɤscoz}}\end{entrée}

\begin{entrée}{nɯrɤtʂha}{}{ⓔnɯrɤtʂha} 
\classe{vi}  
\grammaire{denom} \paradigme{dir}{kɤ-}
\begin{définition}\pfra{manger à l'extérieur}\end{définition}
\begin{définition}\pcmn{在野外吃便饭(到外面劳动时)}\end{définition}
\begin{exemple}\pjya{kɯ-lɤɣ tɤ-ari-tɕi tɕe rɯŋgu kɤ-nɯrɤtʂha-tɕi}\hspace{5pt}\pcmn{我们俩去放牧时在野外吃了午餐}\end{exemple}\relationsémantique{参考}{\lien{ⓔtʂha}{tʂha}}\end{entrée}

\begin{entrée}{nɯrɤʑi}{}{ⓔnɯrɤʑi}\relationsémantique{参考}{\lien{ⓔrɤʑi}{rɤʑi}}\end{entrée}

\begin{entrée}{nɯrchɯrchɤβ}{}{ⓔnɯrchɯrchɤβ} 
\classe{vi}  
\grammaire{denom} \paradigme{dir}{pɯ-}
\begin{définition}\pfra{aller dans des endroits où il y a peu d'espace}\end{définition}
\begin{définition}\pcmn{在密集的地方走来走去;钻来钻去(森林里,人群里)}\end{définition}
\begin{exemple}\pjya{ɯʑo pɯ-nɯrchɯrchɤβ}\hspace{5pt}\pcmn{他钻来钻去了}\end{exemple}
\begin{exemple}\pjya{sɯŋgɯ pɯ-nɯrchɯrchaβ-a}\hspace{5pt}\pcmn{我在森林里钻来钻去}\end{exemple}
\begin{exemple}\pjya{tɯrme nɯra nɯ-rchɤβ pɯ-nɯrchɯrchaβ-a}\hspace{5pt}\pcmn{我在人群中窜来窜去}\end{exemple}\relationsémantique{参考}{\lien{ⓔɯ-rchɤβ}{ɯ-rchɤβ}}\end{entrée}

\begin{entrée}{nɯrchɯrchɯɣ}{}{ⓔnɯrchɯrchɯɣ}\relationsémantique{参考}{\lien{}{rchɯrchɯɣ}}\end{entrée}

\begin{entrée}{nɯrɕɤt}{}{ⓔnɯrɕɤt} 
\classe{vi}  
\grammaire{caus} \sens{1}\paradigme{dir}{pɯ-}
\begin{définition}\pfra{faire une crise d'épilepsie, tomber inconscient}\end{définition}
\begin{définition}\pcmn{癫痫发作;昏迷}\end{définition}\sens{2}\paradigme{dir}{\_}\paradigme{dir}{\_}
\begin{définition}\pfra{toucher légèrement, frotter en passant}\end{définition}
\begin{définition}\pcmn{轻轻地擦过去;碰一下}\end{définition}
\begin{exemple}\pjya{jiɕqha nɯ ɯ-taʁ nɯ-nɯrɕɤt}\hspace{5pt}\pcmn{他轻轻地碰了一下}\end{exemple}
\begin{exemple}\pjya{nɯ-nɯrɕat-a}\hspace{5pt}\pcmn{我轻轻地碰了一下}\end{exemple}
\begin{exemple}\pjya{ɯ-mɲaʁ ɲɯ-nɯrɕɤt cinɤ mɤ-kɯ-ɤtsɯtsu qhe ɲɤ-ɕqhlɤt}\hspace{5pt}\pcmn{还没有来得及看一眼就消失了}\end{exemple}
\begin{sous-entrée}{znɯrɕɤt}{ⓔnɯrɕɤtⓢ2ⓝznɯrɕɤt} 
\classe{vt} \end{sous-entrée}

\begin{définition}\pfra{toucher légèrement, frotter en passant (avec un objet)}\end{définition}
\begin{définition}\pcmn{(用某个东西)轻轻地碰一下}\end{définition}
\begin{exemple}\pjya{nɯ-ari-ndʑi tɕe, lo rŋgɯ nɯ ɯ-ɕki nɯ tɕu phuɲi ɲɤ-znɯrɕɤt-ndʑi}\hspace{5pt}\pcmn{他们俩到了之后,在大石包上用树枝轻轻地擦过去了}\end{exemple}\end{entrée}

\begin{entrée}{nɯrdɤstaʁ}{}{ⓔnɯrdɤstaʁ} 
\classe{vt}  
\grammaire{denom} \paradigme{dir}{tɤ-}
\begin{définition}\pfra{jeter une pierre à}\end{définition}
\begin{définition}\pcmn{扔石头}\end{définition}
\begin{exemple}\pjya{tɤ́-wɣ-nɯrdɤstaʁ-a}\hspace{5pt}\pcmn{他向我扔了石头}\end{exemple}
\begin{exemple}\pjya{khɯna tɤ-nɯrdɤstaʁ-a (khɯna ɯ-ɕki rdɤstaʁ tɤ-lat-a)}\hspace{5pt}\pcmn{我向狗扔石头}\end{exemple}
\begin{exemple}\pjya{ndzɯrnaʁ ɯ-kha nɯ to-nɯrdɤstaʁ-nɯ tɕe, ndzɯrnaʁ pjɤ-nɯɬoʁ-nɯ tɕe kó-wɣ-mtsɯɣ tɕe pjɤ́-wɣ-tʂaβ}\hspace{5pt}\pcmn{他们向马蜂窝扔了石头,马蜂出来了,把他们蛰了,他们痛得摔倒了}\end{exemple}\relationsémantique{参考}{\lien{ⓔrdɤstaʁ}{rdɤstaʁ}}\end{entrée}

\begin{entrée}{nɯrdoʁ}{}{ⓔnɯrdoʁ} 
\classe{vt}  
\grammaire{denom} \paradigme{dir}{tɤ-}
\begin{définition}\pfra{ramasser un à un des petits morceaux}\end{définition}
\begin{définition}\pcmn{一个一个地捡起来}\end{définition}
\begin{exemple}\pjya{ta-nɯrdoʁ}\hspace{5pt}\pcmn{他捡了}\end{exemple}
\begin{exemple}\pjya{kumpɣa kɯ ɯ-kɤ-ndza ɲɯ-ɤz-nɯrdoʁ}\hspace{5pt}\pcmn{鸡在啄食物}\end{exemple}
\begin{exemple}\pjya{staχpɯ ta-nɯrdoʁ}\hspace{5pt}\pcmn{他捡了豌豆}\end{exemple}
\begin{exemple}\pjya{stoʁ ta-nɯrdoʁ}\hspace{5pt}\pcmn{他捡了胡豆}\end{exemple}
\begin{exemple}\pjya{ʑɴɢɯloʁ ta-nɯrdoʁ}\hspace{5pt}\pcmn{他捡了核桃}\end{exemple}\relationsémantique{参考}{\lien{ⓔtɯ-rdoʁ}{tɯ-rdoʁ}}\end{entrée}

\begin{entrée}{nɯrdɯl}{}{ⓔnɯrdɯl} 
\classe{vi} \paradigme{dir}{kɤ-}\paradigme{dir}{kɤ-}
\begin{définition}\pfra{avoir plein de poussière}\end{définition}
\begin{définition}\pcmn{沾满了灰尘}\end{définition}
\begin{définition}\pfra{rendre plein de poussière}\end{définition}
\begin{définition}\pcmn{使沾满灰尘}\end{définition}
\begin{exemple}\pjya{ki tɯ-ŋga ki ko-nɯrdɯl}\hspace{5pt}\pcmn{这件衣服沾上了灰尘}\end{exemple}
\begin{exemple}\pjya{kɤntɕhaʁ ɯ-tɯ-ɣɤrdɯl ɲɯ-saχaʁ tɕe, tɤ-kɯ-ŋke tɕe, ku-kɯ-z-nɯrdɯl ɲɯ-ŋu}\hspace{5pt}\pcmn{街上很多灰尘,走路的时候身上会沾到灰尘}\end{exemple}\relationsémantique{参考}{\lien{ⓔɣɤrdɯl}{ɣɤrdɯl}}\relationsémantique{参考}{\lien{ⓔrdɯl}{rdɯl}}
\begin{sous-entrée}{znɯrdɯl}{ⓔnɯrdɯlⓝznɯrdɯl} 
\classe{vt} \end{sous-entrée}

\end{entrée}

\begin{entrée}{nɯre}{}{ⓔnɯre} 
\classe{adv} 
\begin{définition}\pfra{là}\end{définition}
\begin{définition}\pcmn{在(你)那里}\end{définition}\end{entrée}

\begin{entrée}{nɯrga}{}{ⓔnɯrga} 
\classe{vt}  
\grammaire{appl} \paradigme{dir}{nɯ-}
\begin{définition}\pfra{aimer}\end{définition}
\begin{définition}\pcmn{喜欢}\end{définition}
\begin{exemple}\pjya{jiɕqha tɕheme nɯ ɲɯ-nɯrge-a}\hspace{5pt}\pcmn{我喜欢那个女子}\end{exemple}
\begin{exemple}\pjya{tɕheme nɯ nɯ-nɯrga-t-a}\hspace{5pt}\pcmn{我喜欢上那个女子}\end{exemple}
\begin{exemple}\pjya{pɯ-nɯrga-t-a}\hspace{5pt}\pcmn{我以前喜欢她}\end{exemple}
\begin{exemple}\pjya{ɲɯ-ta-nɯrga}\hspace{5pt}\pcmn{我喜欢你}\end{exemple}\relationsémantique{参考}{\lien{ⓔrgaⓗ1ⓝrga}{rga}}\relationsémantique{反义词}{\lien{ⓔqha}{qha}}
\begin{sous-entrée}{sɤnɯrga}{ⓔnɯrgaⓝsɤnɯrga} 
\classe{vi}  
\grammaire{appl}
\grammaire{apass} 
\begin{définition}\pfra{qui aime les gens}\end{définition}
\begin{définition}\pcmn{喜欢别人的}\end{définition}
\begin{exemple}\pjya{nɤʑo ɲɯ-tɯ-sɤnɯrga}\hspace{5pt}\pcmn{你喜欢别人}\end{exemple}
\begin{exemple}\pjya{nɤki tɤ-pɤtso nɯ kɯ-sɤnɯrga ci ŋu}\hspace{5pt}\pcmn{那个小孩子喜欢人}\end{exemple}\end{sous-entrée}

\begin{sous-entrée}{anɯrgɯrga}{ⓔnɯrgaⓝanɯrgɯrga} 
\classe{vi}  
\grammaire{appl}
\grammaire{recip} 
\begin{définition}\pfra{s'aimer l'un l'autre}\end{définition}
\begin{définition}\pcmn{相爱}\end{définition}\end{sous-entrée}

\étymologie{dga}\end{entrée}

\begin{entrée}{nɯrɟaŋ}{}{ⓔnɯrɟaŋ} 
\classe{vs} \sens{1}
\begin{définition}\pfra{qui se transmet loin}\end{définition}
\begin{définition}\pcmn{传得远}\end{définition}
\begin{exemple}\pjya{ɯ-skɤt ɲɯ-nɯrɟaŋ}\hspace{5pt}\pcmn{他的声音传得很远}\end{exemple}\sens{2}
\begin{définition}\pfra{répandue (langue)}\end{définition}
\begin{définition}\pcmn{使用范围最广(语言)}\end{définition}
\begin{exemple}\pjya{stu kɯ-nɯrɟaŋ nɯ ɲɯ-tɯ-spe}\hspace{5pt}\pcmn{你学会了最实用的(那个语言)(反话,指的是茶堡话)}\end{exemple}\relationsémantique{参考}{\lien{ⓔrɟaŋ}{rɟaŋ}}\end{entrée}

\begin{entrée}{nɯrɟɤntɕa}{}{ⓔnɯrɟɤntɕa} 
\classe{vt}  
\grammaire{denom} \paradigme{dir}{tɤ-}
\begin{définition}\pfra{se parer (de bijoux)}\end{définition}
\begin{définition}\pcmn{打扮(一身带满装饰)}\end{définition}
\begin{exemple}\pjya{tɤ-nɯrɟɤntɕa-t-a}\hspace{5pt}\pcmn{我打扮了}\end{exemple}\étymologie{rgʲan.tɕʰa}\end{entrée}

\begin{entrée}{nɯrɟɯrŋom}{}{ⓔnɯrɟɯrŋom} 
\classe{vi}  
\grammaire{incorp} \paradigme{dir}{nɯ-}
\begin{définition}\pfra{convoiter des richesses}\end{définition}
\begin{définition}\pcmn{贪财}\end{définition}
\begin{exemple}\pjya{ɲɯ-nɯrɟɯrŋom}\hspace{5pt}\pcmn{他贪财}\end{exemple}
\begin{exemple}\pjya{kɤ-nɯrɟɯrŋom sɤzraʁ}\hspace{5pt}\pcmn{贪财是一件可耻的事情}\end{exemple}\relationsémantique{参考}{\lien{ⓔrɟɯrŋom}{rɟɯrŋom}}\relationsémantique{参考}{\lien{ⓔsŋom}{sŋom}}\relationsémantique{参考}{\lien{ⓔtɯ-rɟɯ}{tɯ-rɟɯ}}\end{entrée}

\begin{entrée}{nɯrkorlɯt}{}{ⓔnɯrkorlɯt} 
\classe{vs}  
\grammaire{comp} \paradigme{dir}{tɤ-}
\begin{définition}\pfra{être entêté}\end{définition}
\begin{définition}\pcmn{顽固;固执}\end{définition}
\begin{exemple}\pjya{ɲɯ-nɯrkorlɯt}\hspace{5pt}\pcmn{他很顽固}\end{exemple}
\begin{exemple}\pjya{nɯki tɯrme nɯ ɲɯ-nɯrkorlɯt tɕe mɯ́j-khɯ}\hspace{5pt}\pcmn{这个人很固执,没有同意}\end{exemple}
\begin{exemple}\pjya{to-nɯrkorlɯt}\hspace{5pt}\pcmn{他变得很固执;他当时很固执(没有答应别人)}\end{exemple}
\begin{exemple}\pjya{ma-tɤ-tɯ-nɯrkorlɯt}\hspace{5pt}\pcmn{你不要这么固执}\end{exemple}\relationsémantique{参考}{\lien{ⓔrko}{rko}}\relationsémantique{参考}{\lien{ⓔarlɯt}{arlɯt}}\end{entrée}

\begin{entrée}{nɯrlɤn}{}{ⓔnɯrlɤn} 
\classe{vs} \paradigme{dir}{nɯ-}
\begin{définition}\pfra{vert (bois), humide}\end{définition}
\begin{définition}\pcmn{湿;生(木头)}\end{définition}
\begin{exemple}\pjya{pjɤ-nɯrlɤn tɕe sɯtɕhaʁ ko-ɕe}\hspace{5pt}\pcmn{木头是湿的,所以缩了}\end{exemple}
\begin{exemple}\pjya{kɯ-nɯrlɤn}\hspace{5pt}\pcmn{生木头}\end{exemple}
\begin{sous-entrée}{znɯrlɤn}{ⓔnɯrlɤnⓝznɯrlɤn} 
\classe{vt} 
\begin{définition}\pfra{rendre humide}\end{définition}
\begin{définition}\pcmn{弄湿}\end{définition}\end{sous-entrée}

\étymologie{rlan}\end{entrée}

\begin{entrée}{nɯrmɤβlɯ}{}{ⓔnɯrmɤβlɯ} 
\classe{vt} \paradigme{dir}{tɤ-}
\begin{définition}\pfra{dépendre de}\end{définition}
\begin{définition}\pcmn{(家庭)靠……}\end{définition}
\begin{exemple}\pjya{nɤʑo ku-ta-nɯrmɤβlɯ ɕti}\hspace{5pt}\pcmn{我们家全靠你}\end{exemple}
\begin{exemple}\pjya{nɤʑo ji-kɤ-nɯrmɤβlɯ tɯ-ɕti}\hspace{5pt}\pcmn{我们家全靠你}\end{exemple}\relationsémantique{参考}{\lien{ⓔrma}{rma}}\relationsémantique{参考}{\lien{ⓔβlɯ}{βlɯ}}\end{entrée}

\begin{entrée}{nɯrmɤkro}{}{ⓔnɯrmɤkro} 
\classe{vi}  
\grammaire{incorp} \paradigme{dir}{pɯ-}
\begin{définition}\pfra{partager le patrimoine}\end{définition}
\begin{définition}\pcmn{分家;分财产}\end{définition}
\begin{exemple}\pjya{pɯ-nɯrmɤkro-tɕi}\hspace{5pt}\pcmn{我们俩分家了}\end{exemple}\relationsémantique{参考}{\lien{ⓔtɯrma}{tɯrma}}\relationsémantique{参考}{\lien{ⓔkro}{kro}}\end{entrée}

\begin{entrée}{nɯrmɤmbe}{}{ⓔnɯrmɤmbe} 
\classe{vi}  
\grammaire{incorp} \paradigme{dir}{thɯ-}
\begin{définition}\pfra{muer (mammifère)}\end{définition}
\begin{définition}\pcmn{脱毛}\end{définition}
\begin{exemple}\pjya{ji-fsapaʁ chɤ-pe, thɯ-nɯrmɤmbe}\hspace{5pt}\pcmn{我们家的牲畜身体很壮就换了毛}\end{exemple}
\begin{exemple}\pjya{nɯŋa thɯ-nɯrmɤmbe}\hspace{5pt}\pcmn{奶牛换了毛}\end{exemple}
\begin{exemple}\pjya{jla thɯ-nɯrmɤmbe}\hspace{5pt}\pcmn{犏牛换了毛}\end{exemple}\relationsémantique{参考}{\lien{ⓔtɤ-rme}{tɤ-rme}}\relationsémantique{参考}{\lien{ⓔmbe}{mbe}}\relationsémantique{参考}{\lien{ⓔrmɤmbe}{rmɤmbe}}\end{entrée}

\begin{entrée}{nɯrmɤʑu}{}{ⓔnɯrmɤʑu} 
\classe{vi} \paradigme{dir}{tɤ-}\paradigme{dir}{thɯ-}
\begin{définition}\pfra{faire son intéressant}\end{définition}
\begin{définition}\pcmn{爱在人多的场合中表现自己【耍人来疯】}\end{définition}
\begin{exemple}\pjya{jiɕqha nɯ wuma nɯrmɤʑu}\hspace{5pt}\pcmn{那个人很喜欢表现自己}\end{exemple}
\begin{exemple}\pjya{ma-tɤ-tɯ-nɯrmɤʑu}\hspace{5pt}\pcmn{你不要耍人来疯}\end{exemple}
\begin{exemple}\pjya{tɤ-pɤtso ɲɯ-nɯrmɤʑu}\hspace{5pt}\pcmn{小孩子很喜欢表现自己}\end{exemple}\end{entrée}

\begin{entrée}{nɯrmbɯχtɕi}{}{ⓔnɯrmbɯχtɕi} 
\classe{vt}  
\grammaire{incorp} \paradigme{dir}{tɤ-}
\begin{définition}\pfra{asperger de liquide}\end{définition}
\begin{définition}\pcmn{朝人喷尿}\end{définition}
\begin{exemple}\pjya{tɤ-kɯ-nɯrmbɯχtɕi-a}\hspace{5pt}\pcmn{你朝我喷了液体}\end{exemple}\relationsémantique{参考}{\lien{ⓔχtɕi}{χtɕi}}\relationsémantique{参考}{\lien{ⓔtɯ-rmbi}{tɯ-rmbi}}\end{entrée}

\begin{entrée}{nɯrmɯ}{}{ⓔnɯrmɯ} 
\classe{vi} \paradigme{dir}{nɯ-}
\begin{définition}\pfra{se coucher tard}\end{définition}
\begin{définition}\pcmn{晚睡}\end{définition}\relationsémantique{参考}{\lien{ⓔtɯrmɯ}{tɯrmɯ}}\relationsémantique{参考}{\lien{ⓔnɯrmɯsoz}{nɯrmɯsoz}}\end{entrée}

\begin{entrée}{nɯrmɯsoz}{}{ⓔnɯrmɯsoz} 
\classe{vi} \paradigme{dir}{tɤ-}
\begin{définition}\pfra{se lever tôt et se coucher tard}\end{définition}
\begin{définition}\pcmn{早起晚睡}\end{définition}\relationsémantique{参考}{\lien{ⓔsoz}{soz}}\relationsémantique{参考}{\lien{ⓔtɯrmɯ}{tɯrmɯ}}\end{entrée}

\begin{entrée}{nɯrŋu}{}{ⓔnɯrŋu} 
\classe{vi} \paradigme{dir}{kɤ-}
\begin{définition}\pfra{attraper une maladie de la peau}\end{définition}
\begin{définition}\pcmn{患上一种皮肤病}\end{définition}
\begin{exemple}\pjya{paʁ ko-nɯrŋu}\hspace{5pt}\pcmn{猪得了皮肤病}\end{exemple}
\begin{exemple}\pjya{mbro ko-nɯrŋu}\hspace{5pt}\pcmn{马得了皮肤病}\end{exemple}
\begin{exemple}\pjya{kɤ-kɯ-nɯrŋu}\hspace{5pt}\pcmn{皮肤病患者}\end{exemple}
\begin{exemple}\pjya{kɤ-kɯ-nɯrŋu nɯ tɯ-βri thamtɕɤt ʑo ʑmbɤr kɯ-fse ɲɯ-mtshɤt ʑo ɲɯ-ŋu, tɤ-rme ra chɯ-ɤʁɟa tɕe ɲɯ-pɣi ʑo ɲɯ-rʁom ʑo ɲɯ-ŋu.}\hspace{5pt}\pcmn{\lien{}{kɤ-kɯ-nɯrŋu} 是一种皮肤病,全身长满疮一样的东西,毛全脱光,皮肤变灰色,很粗糙。}\end{exemple}\end{entrée}

\begin{entrée}{nɯrŋgɯ}{}{ⓔnɯrŋgɯ}\relationsémantique{参考}{\lien{ⓔrŋgɯⓗ1}{rŋgɯ₁}}\end{entrée}

\begin{entrée}{nɯrŋgɯmbri}{}{ⓔnɯrŋgɯmbri} 
\classe{vi}  
\grammaire{comp} 
\begin{définition}\pfra{gémir}\end{définition}
\begin{définition}\pcmn{呻吟}\end{définition}\relationsémantique{参考}{\lien{ⓔrŋgɯⓗ1}{rŋgɯ₁}}\relationsémantique{参考}{\lien{ⓔmbriⓗ1}{mbri₁}}\end{entrée}

\begin{entrée}{nɯrpu}{}{ⓔnɯrpu}\relationsémantique{参考}{\lien{ⓔrpu}{rpu}}\end{entrée}

\begin{entrée}{nɯrqhoʁ}{}{ⓔnɯrqhoʁ}\relationsémantique{参考}{\lien{ⓔɣɤrqhoʁrqhoʁ}{ɣɤrqhoʁrqhoʁ}}\end{entrée}

\begin{entrée}{nɯrʁe}{}{ⓔnɯrʁe} 
\classe{vt}  
\grammaire{autoben} \paradigme{dir}{lɤ-}
\begin{définition}\pfra{porter (un bracelet)}\end{définition}
\begin{définition}\pcmn{戴(手镯、耳环)}\end{définition}
\begin{exemple}\pjya{zgroʁ la-nɯrʁe}\hspace{5pt}\pcmn{他戴了手镯}\end{exemple}
\begin{exemple}\pjya{zgroʁ lɤ-nɯrʁe-t-a}\hspace{5pt}\pcmn{我戴了手镯}\end{exemple}
\begin{exemple}\pjya{rnɤjɯ la-nɯrʁe}\hspace{5pt}\pcmn{他戴了耳环}\end{exemple}
\begin{exemple}\pjya{srɯnloʁ lɤ-nɯrʁe-t-a}\hspace{5pt}\pcmn{我戴了戒指}\end{exemple}\relationsémantique{参考}{\lien{ⓔrʁe}{rʁe}}\end{entrée}

\begin{entrée}{nɯrʁɯrpu}{}{ⓔnɯrʁɯrpu} 
\classe{vt}  
\grammaire{incorp} \paradigme{dir}{tɤ-}
\begin{définition}\pfra{donner des coups de corne}\end{définition}
\begin{définition}\pcmn{用角打}\end{définition}
\begin{exemple}\pjya{jla kɯ tɤ́-wɣ-nɯʁrɯrpu-a}\hspace{5pt}\pcmn{犏牛用角打了我}\end{exemple}
\begin{sous-entrée}{sɤnɯʁrɯrpu}{ⓔnɯrʁɯrpuⓝsɤnɯʁrɯrpu} 
\classe{vi} 
\begin{définition}\pfra{donner des coups de corne aux gens}\end{définition}
\begin{définition}\pcmn{用角打人}\end{définition}\end{sous-entrée}

\end{entrée}

\begin{entrée}{nɯrʁɯrʁa}{}{ⓔnɯrʁɯrʁa} 
\classe{vi} \paradigme{dir}{tɤ-}
\begin{définition}\pfra{grimper}\end{définition}
\begin{définition}\pcmn{爬}\end{définition}
\begin{exemple}\pjya{znde ɯ-taʁ tɤ-nɯrʁɯrʁa}\hspace{5pt}\pcmn{他爬了墙}\end{exemple}
\begin{exemple}\pjya{znde ɯ-taʁ ma-tɤ-tɯ-nɯrʁɯrʁa ma tɯ-atɤr}\hspace{5pt}\pcmn{你不要爬墙,小心摔下来}\end{exemple}
\begin{exemple}\pjya{si ɯ-taʁ tɤ-nɯrʁɯrʁa}\hspace{5pt}\pcmn{他爬了树}\end{exemple}\end{entrée}

\begin{entrée}{nɯrtɤβ}{}{ⓔnɯrtɤβ} 
\classe{vt} \paradigme{dir}{nɯ-}\paradigme{dir}{thɯ-}
\begin{définition}\pfra{porter}\end{définition}
\begin{définition}\pcmn{戴;穿;系}\end{définition}
\begin{exemple}\pjya{tɕhoma nɯ-nɯrtaβ-a}\hspace{5pt}\pcmn{我穿了皮带}\end{exemple}
\begin{exemple}\pjya{βʑɯndi nɯ-nɯrtaβ-a}\hspace{5pt}\pcmn{我穿了裹腿}\end{exemple}
\begin{exemple}\pjya{mthɯxtɕɤr nɯ nɯ-nɯrtaβ-a}\hspace{5pt}\pcmn{我穿了腰带}\end{exemple}
\begin{exemple}\pjya{@weijin thɯ-nɯrtaβ-a}\hspace{5pt}\pcmn{我戴了围巾}\end{exemple}\relationsémantique{参考}{\lien{ⓔrtɤβ}{rtɤβ}}\end{entrée}

\begin{entrée}{nɯrtɕa}{}{ⓔnɯrtɕa} 
\classe{vt} \paradigme{dir}{kɤ-}\paradigme{dir}{kɤ-}
\begin{définition}\pfra{taquiner}\end{définition}
\begin{définition}\pcmn{逗弄(话)、骚扰}\end{définition}
\begin{définition}\pfra{taquiner les gens}\end{définition}
\begin{définition}\pcmn{惹人家}\end{définition}
\begin{exemple}\pjya{kɤ-nɯrtɕa-t-a}\hspace{5pt}\pcmn{我逗弄他了}\end{exemple}
\begin{exemple}\pjya{kɤ́-wɣ-nɯrtɕa-a}\hspace{5pt}\pcmn{他逗弄我了}\end{exemple}
\begin{exemple}\pjya{kɤ-ta-nɯrtɕa}\hspace{5pt}\pcmn{我逗弄你了}\end{exemple}\relationsémantique{同义词}{\lien{ⓔnɤjndɤt}{nɤjndɤt}}
\begin{sous-entrée}{sɤnɯrtɕa}{ⓔnɯrtɕaⓝsɤnɯrtɕa} 
\classe{vi}  
\grammaire{apass} \end{sous-entrée}

\end{entrée}

\begin{entrée}{nɯrtɕhɯɴɢɯɴɢaʁ}{}{ⓔnɯrtɕhɯɴɢɯɴɢaʁ} 
\classe{vi} \paradigme{dir}{lɤ-}
\begin{définition}\pfra{s'écailler}\end{définition}
\begin{définition}\pcmn{一片一片地裂开}\end{définition}
\begin{exemple}\pjya{tɤtho ɯ-mat nɯ ɯ-rɣi nɯ-ɬoʁ tɤkha tɕe ɯ-rqhu nɯ lu-nɯrtɕhɯɴɢ̣ɴɢaʁ ŋu, tɕe ɯ-rqhu nɯ tɤ-ɴɢaʁ tɕe, nɯ ɯ-ŋgɯ ɯ-rɣi nɯ pjɯ-nɯɬoʁ ɲɯ-ŋu}\hspace{5pt}\pcmn{松树的果子里面的种子快要脱落的时候,它的果皮一片一片地裂开,然后脱落}\end{exemple}
\begin{exemple}\pjya{pɯ-kɯ-ndʐaβ tɕe, tɯ-ndʐi lu-nɯrtɕhɯɴɢɯɴɢaʁ ŋgrɤl}\hspace{5pt}\pcmn{摔倒的时候,经常会把皮肤擦伤}\end{exemple}\relationsémantique{参考}{\lien{ⓔɴɢaʁ}{ɴɢaʁ}}\end{entrée}

\begin{entrée}{nɯrtsa}{}{ⓔnɯrtsa} 
\classe{vt} \paradigme{dir}{nɯ-}
\begin{définition}\pfra{rechercher la cause de}\end{définition}
\begin{définition}\pcmn{追究}\end{définition}
\begin{exemple}\pjya{nɯɕɯŋgɯ pɯ-kɯ-fse mɯ-ɲɤ-nɯrtsa}\hspace{5pt}\pcmn{他没有追究以前发生的事情}\end{exemple}
\begin{exemple}\pjya{tɕhi pjɤ-fse ɲɯ-ŋu kɯ a-nɯ-tɯ-nɯrtse}\hspace{5pt}\pcmn{你要追究到底发生了什么事情}\end{exemple}\relationsémantique{参考}{\lien{ⓔɯ-rtsa,tɕɤt}{ɯ-rtsa,tɕɤt}}\end{entrée}

\begin{entrée}{nɯrtsɤl}{}{ⓔnɯrtsɤl} 
\classe{vs} \paradigme{dir}{tɤ-}
\begin{définition}\pfra{être bon en équitation}\end{définition}
\begin{définition}\pcmn{骑马的技术好}\end{définition}
\begin{exemple}\pjya{mbro kɤ-nɯmbrɤpɯ wuma ɲɯ-nɯrtsɤl}\hspace{5pt}\pcmn{他骑马的技术非常好}\end{exemple}\étymologie{rtsal}\end{entrée}

\begin{entrée}{nɯrtsɤtɯɣ}{}{ⓔnɯrtsɤtɯɣ} 
\classe{vi}  
\grammaire{denom} \paradigme{dir}{pɯ-}
\begin{définition}\pfra{s'empoisonner en mangeant une herbe (bovidé)}\end{définition}
\begin{définition}\pcmn{草中毒(牛)}\end{définition}
\begin{exemple}\pjya{jla pjɤ-nɯrtsɤtɯɣ}\hspace{5pt}\pcmn{犏牛吃草中毒了}\end{exemple}
\begin{exemple}\pjya{qambrɯ pjɤ-nɯrtsɤtɯɣ}\hspace{5pt}\pcmn{犏牦牛吃草中毒了}\end{exemple}\relationsémantique{参考}{\lien{ⓔnɯtɕhɯtɯɣ}{nɯtɕhɯtɯɣ}}\étymologie{rtsʷa.dug}\end{entrée}

\begin{entrée}{nɯrtsɯ}{}{ⓔnɯrtsɯ} 
\classe{vi} \paradigme{dir}{\_}
\begin{définition}\pfra{ramper}\end{définition}
\begin{définition}\pcmn{爬行}\end{définition}
\begin{exemple}\pjya{tɤ-pɤtso ɲɯ-nɯrtsɯ}\hspace{5pt}\pcmn{小孩子在爬行}\end{exemple}\end{entrée}

\begin{entrée}{nɯrtsɯpɣaʁ}{}{ⓔnɯrtsɯpɣaʁ} 
\classe{vt} \paradigme{dir}{lɤ-}
\begin{définition}\pfra{retourner la terre après la récolte}\end{définition}
\begin{définition}\pcmn{庄稼收割了以后重新翻地}\end{définition}
\begin{exemple}\pjya{tɯji lɤ-nɯrtsɯpɣaʁ-a}\hspace{5pt}\pcmn{我翻了地}\end{exemple}
\begin{exemple}\pjya{tɯtɣa pɯ-jɤɣ tɕe, kɤ-nɯrtsɯpɣaʁ mda}\hspace{5pt}\pcmn{收割结束了之后就是翻地的时候了}\end{exemple}
\begin{exemple}\pjya{tɯji mɤ-kɤ-nɯrtsɯpɣaʁ mɤ-khɯ}\hspace{5pt}\pcmn{不翻田地是不行的}\end{exemple}
\begin{exemple}\pjya{pɯ-nɯrtsɯpɣaʁ-a ri mɯ́j-phɤn}\hspace{5pt}\pcmn{我耕了这块地,但还是不行}\end{exemple}
\begin{exemple}\pjya{mɤ-kɤ-tɣa kɤ-nɯrtsɯpɣaʁ mɤ-ŋgrɤl}\hspace{5pt}\pcmn{在没有收割之前不能翻地}\end{exemple}\relationsémantique{参考}{\lien{ⓔpɣaʁ}{pɣaʁ}}\relationsémantique{参考}{\lien{ⓔrtsɯpɣaʁ}{rtsɯpɣaʁ}}\relationsémantique{参考}{\lien{ⓔrɯrtsɯpɣaʁ}{rɯrtsɯpɣaʁ}}\end{entrée}

\begin{entrée}{nɯrɯ}{}{ⓔnɯrɯ} 
\classe{vi} \paradigme{dir}{nɯ-}
\begin{définition}\pfra{brouter l'herbe}\end{définition}
\begin{définition}\pcmn{吃草}\end{définition}
\begin{exemple}\pjya{fsapaʁ ɲɯ-nɯrɯ}\hspace{5pt}\pcmn{牲畜在吃草}\end{exemple}
\begin{exemple}\pjya{fsapaʁ ra ɲɯ-nɯrɯ-nɯ}\hspace{5pt}\pcmn{牲畜在吃草}\end{exemple}\relationsémantique{同义词}{\lien{ⓔnɯsɤlɤɣ}{nɯsɤlɤɣ}}\end{entrée}

\begin{entrée}{nɯrɯcu}{}{ⓔnɯrɯcu} 
\classe{vs} 
\begin{définition}\pfra{bien s'entendre avec}\end{définition}
\begin{définition}\pcmn{合得来}\end{définition}
\begin{exemple}\pjya{ndʐɯɣlɤm kɯ-nɯrɯcu}\hspace{5pt}\pcmn{合法的}\end{exemple}
\begin{exemple}\pjya{tɯrme ra nɯ-rca ɲɯ-tɯ-nɯrɯcu}\hspace{5pt}\pcmn{你跟这些人合得来}\end{exemple}\relationsémantique{参考}{\lien{ⓔcuⓝnɤcu}{nɤcu}}\end{entrée}

\begin{entrée}{nɯrɯtʂa}{}{ⓔnɯrɯtʂa} 
\classe{vt}  
\grammaire{denom} \paradigme{dir}{tɤ-}
\begin{définition}\pfra{envier}\end{définition}
\begin{définition}\pcmn{妒忌}\end{définition}
\begin{exemple}\pjya{jiɕqha nɯ kɯ ɲɯ́-wɣ-nɯrɯtʂa-a}\hspace{5pt}\pcmn{那个人妒忌我}\end{exemple}
\begin{exemple}\pjya{ma-tɤ-kɯ-nɯrɯtʂa-a}\hspace{5pt}\pcmn{你不要妒忌我}\end{exemple}
\begin{sous-entrée}{anɯrɯtʂɯtʂa}{ⓔnɯrɯtʂaⓝanɯrɯtʂɯtʂa}
\begin{définition}\pfra{s'envier les uns les autres}\end{définition}
\begin{définition}\pcmn{互相妒忌}\end{définition}\end{sous-entrée}

\begin{sous-entrée}{sɤnɯrɯtʂa}{ⓔnɯrɯtʂaⓝsɤnɯrɯtʂa} 
\classe{vi}  
\grammaire{apass} 
\begin{définition}\pfra{envier les gens}\end{définition}
\begin{définition}\pcmn{妒忌人家}\end{définition}\relationsémantique{参考}{\lien{ⓔrɯtʂa}{rɯtʂa}}\end{sous-entrée}

\end{entrée}

\begin{entrée}{nɯrɯz}{}{ⓔnɯrɯz} 
\classe{vi} \paradigme{dir}{thɯ-}\paradigme{dir}{tɤ-}
\begin{définition}\pfra{faire l'un après l'autre}\end{définition}
\begin{définition}\pcmn{轮流}\end{définition}
\begin{définition}\pfra{utiliser l'un après l'autre}\end{définition}
\begin{définition}\pcmn{轮流着用}\end{définition}
\begin{exemple}\pjya{@zhiban chɯ-nɯrɯz-nɯ ɲɯ-ra}\hspace{5pt}\pcmn{他们要轮流值班}\end{exemple}
\begin{exemple}\pjya{tɕiʑo ni kɤ-rɤma tu-nɯrɯz-tɕi ŋu}\hspace{5pt}\pcmn{我们轮流劳动}\end{exemple}
\begin{exemple}\pjya{kɯki a-ŋga ʁnɯz ki tu-znɯrɯrɯz-a ŋu, jisŋi ki tɤ-ŋga-t-a tɕe fso tɕe ci nɯ ɯ-βra tu-ŋge-a ŋu}\hspace{5pt}\pcmn{我把这两件衣服轮流穿,今天穿了这件,明天就会穿那件}\end{exemple}
\begin{sous-entrée}{znɯrɯrɯz}{ⓔnɯrɯzⓝznɯrɯrɯz} 
\classe{vt} \end{sous-entrée}

\end{entrée}

\begin{entrée}{nɯrzandɤɣ}{}{ⓔnɯrzandɤɣ} 
\classe{vi} \paradigme{dir}{pɯ-}
\begin{définition}\pfra{attraper le mal des hauteurs}\end{définition}
\begin{définition}\pcmn{发生高山反应}\end{définition}
\begin{exemple}\pjya{pjɤ-nɯrzandaɣ-a}\hspace{5pt}\pcmn{我有了高山反应}\end{exemple}
\begin{exemple}\pjya{aj mucin ʑo mɯ́j-nɯrzandaɣ-a}\hspace{5pt}\pcmn{我根本就不会有高山反应}\end{exemple}\étymologie{rdza.dug}\end{entrée}

\begin{entrée}{nɯrʑɯɣ}{}{ⓔnɯrʑɯɣ} 
\classe{vt} \paradigme{dir}{pɯ-}
\begin{définition}\pfra{couper très vite grâce à un couteau bien aiguisé}\end{définition}
\begin{définition}\pcmn{切得很快}\end{définition}
\begin{exemple}\pjya{paʁndza pa-nɯrʑɯɣ ʑo pa-rɤkrɯ}\hspace{5pt}\pcmn{他把猪草切得很快(刀很锋利)}\end{exemple}
\begin{exemple}\pjya{pɯ-nɯrʑɯɣ-a pɯ-ʁndzar-a}\hspace{5pt}\pcmn{我锯得很快(锯子很锋利)}\end{exemple}
\begin{exemple}\pjya{ftɕɤfkɤt pa-nɯrʑɯɣ ʑo}\hspace{5pt}\pcmn{他很果断的指挥了别人}\end{exemple}\end{entrée}

\begin{entrée}{nɯʁɤri}{}{ⓔnɯʁɤri} 
\classe{vt} \paradigme{dir}{\_}\sens{1}
\begin{définition}\pfra{se mettre en face de}\end{définition}
\begin{définition}\pcmn{转身向}\end{définition}
\begin{exemple}\pjya{aʑo tɕoχtsi lu-nɯʁɤri-a ŋu}\hspace{5pt}\pcmn{我转身面向桌子}\end{exemple}
\begin{exemple}\pjya{aʑo khɯɣɲɟɯ nɯ-nɯʁɤri-t-a}\hspace{5pt}\pcmn{我向窗子转身了}\end{exemple}\sens{2}
\begin{définition}\pfra{faire face à, tenir tête à}\end{définition}
\begin{définition}\pcmn{对付;阻挡}\end{définition}
\begin{exemple}\pjya{aʑo nɤʑo tu-ta-nɯʁɤri jɤɣ}\hspace{5pt}\pcmn{我可以对付你}\end{exemple}\relationsémantique{反义词}{\lien{ⓔnɯɕqhu}{nɯɕqhu}}\relationsémantique{参考}{\lien{ⓔɯ-ʁɤri}{ɯ-ʁɤri}}\end{entrée}

\begin{entrée}{nɯʁgra}{}{ⓔnɯʁgra} 
\classe{vt}  
\grammaire{denom} \paradigme{dir}{tɤ-}
\begin{définition}\pfra{considérer comme un ennemi}\end{définition}
\begin{définition}\pcmn{敌视}\end{définition}
\begin{exemple}\pjya{jiɕqha kɯ tú-wɣ-nɯʁgra-a ɲɯ-ŋu}\hspace{5pt}\pcmn{那个人跟我有仇}\end{exemple}
\begin{sous-entrée}{anɯʁgrɯʁgra}{ⓔnɯʁgraⓝanɯʁgrɯʁgra} 
\classe{vi} 
\begin{définition}\pfra{se considérer les uns les autres comme des ennemis}\end{définition}
\begin{définition}\pcmn{互相敌视}\end{définition}\relationsémantique{参考}{\lien{ⓔʁgra}{ʁgra}}\end{sous-entrée}

\étymologie{dgra}\end{entrée}

\begin{entrée}{nɯʁjoʁ}{}{ⓔnɯʁjoʁ} 
\classe{vt}  
\grammaire{denom} \paradigme{dir}{nɯ-}\sens{1}
\begin{définition}\pfra{commander}\end{définition}
\begin{définition}\pcmn{使唤}\end{définition}
\begin{exemple}\pjya{nɤʑo kɯ aʑo ɲɯ-kɯ-nɯʁjoʁ-a ɲɯ-ŋu}\hspace{5pt}\pcmn{你在使唤我}\end{exemple}
\begin{exemple}\pjya{aj ɲɯ-ta-nɯʁjoʁ, a-tʂha ci pɯ-rke}\hspace{5pt}\pcmn{请你给我倒一点茶}\end{exemple}
\begin{exemple}\pjya{nɯ-nɯʁjoʁ-a tɕe laχtɕha ka-nɯxtʂɯ}\hspace{5pt}\pcmn{我使唤了他,他就把东西顺便带来了}\end{exemple}
\begin{exemple}\pjya{nɯ-nɯʁjoʁ-a tɕe kɤ-z-nɯxtʂɯ-t-a}\hspace{5pt}\pcmn{我使唤了他,令他把东西顺便带来了}\end{exemple}\sens{2}
\begin{définition}\pfra{travailler pour quelqu'un}\end{définition}
\begin{définition}\pcmn{当别人的帮工}\end{définition}
\begin{exemple}\pjya{a-kɯ-nɯʁjoʁ jɤ-sɯɣe-t-a}\hspace{5pt}\pcmn{我雇佣了他}\end{exemple}\relationsémantique{参考}{\lien{ⓔʁjoʁ}{ʁjoʁ}}\end{entrée}

\begin{entrée}{nɯʁjɯβtshɤt}{}{ⓔnɯʁjɯβtshɤt} 
\classe{vt}  
\grammaire{denom} \paradigme{dir}{tɤ-}
\begin{définition}\pfra{estimer}\end{définition}
\begin{définition}\pcmn{估计}\end{définition}
\begin{exemple}\pjya{fsusqi-tɯrpa tɤ-nɯʁjɯβtshat-a}\hspace{5pt}\pcmn{我估计有三十斤}\end{exemple}
\begin{exemple}\pjya{tɤ-nɯʁjɯβtshat-a tɕe ɕoŋtɕa pɯ-ʁndzar-a}\hspace{5pt}\pcmn{我估计了一下就锯了木料}\end{exemple}\relationsémantique{参考}{\lien{ⓔʁjɯβtshɤt}{ʁjɯβtshɤt}}\end{entrée}

\begin{entrée}{nɯʁlɤwɯr}{}{ⓔnɯʁlɤwɯr} 
\classe{vi}  
\grammaire{denom} \paradigme{dir}{tɤ-}
\begin{définition}\pfra{faire soudainement}\end{définition}
\begin{définition}\pcmn{突然做}\end{définition}
\begin{exemple}\pjya{tɤ-nɯʁlɤwɯr ʑo tɕe jɤ-ari}\hspace{5pt}\pcmn{他突然就走了}\end{exemple}\relationsémantique{参考}{\lien{ⓔʁlɤwɯr}{ʁlɤwɯr}}\end{entrée}

\begin{entrée}{nɯʁlɯmbɯɣ}{}{ⓔnɯʁlɯmbɯɣ} 
\classe{vt} \paradigme{dir}{tɤ-}
\begin{définition}\pfra{estimer}\end{définition}
\begin{définition}\pcmn{估计}\end{définition}
\begin{exemple}\pjya{tɤ-nɯʁlɯmbɯɣa}\hspace{5pt}\pcmn{我估计了一下}\end{exemple}
\begin{exemple}\pjya{@liangmi ɲɯ-ra ri tɤ-nɯʁlɯmbɯɣ-a tɕe pɯ-ʁndzar-a}\hspace{5pt}\pcmn{需要两米的木头,我估计了一下就锯了}\end{exemple}
\begin{exemple}\pjya{fso tɕe tɯ-mɯ lɤt mɤ-lɤt mɤxsi ma aj tɤ-nɯʁlɯmbɯɣ-a ɕti}\hspace{5pt}\pcmn{明天不知道下不下雨,我只是估计一下}\end{exemple}
\begin{exemple}\pjya{kɯki ŋu maʁ mɤxsi ma tɤ-nɯʁlɯmbɯɣ-a ɕti}\hspace{5pt}\pcmn{不知道是不是正确的,这是我猜想的}\end{exemple}
\begin{exemple}\pjya{tu-nɯʁlɯmbɯɣ-a ɕti wo}\hspace{5pt}\pcmn{我猜的(一句话的意思)}\end{exemple}
\begin{exemple}\pjya{nɯ tu-nɯʁlɯmbɯɣ-a ɕti ri, ɯʑo kɯ kɤ-nɤma nɯ sɤpe}\hspace{5pt}\pcmn{我估计他会把工作做好}\end{exemple}\relationsémantique{同义词}{\lien{ⓔnɯʁjɯβtshɤt}{nɯʁjɯβtshɤt}}\end{entrée}

\begin{entrée}{nɯʁmaʁmi}{}{ⓔnɯʁmaʁmi} 
\classe{vi}  
\grammaire{denom} \paradigme{dir}{tɤ-}
\begin{définition}\pfra{être soldat, faire son service militaire}\end{définition}
\begin{définition}\pcmn{当兵}\end{définition}\relationsémantique{参考}{\lien{ⓔʁmaʁmi}{ʁmaʁmi}}\end{entrée}

\begin{entrée}{nɯʁndomnɤt}{}{ⓔnɯʁndomnɤt} 
\classe{vi}  
\grammaire{incorp} \paradigme{dir}{pɯ-}\paradigme{dir}{tɤ-}
\begin{définition}\pfra{répéter sans arrêt, radoter}\end{définition}
\begin{définition}\pcmn{重复讲说过的话;啰唆}\end{définition}
\begin{exemple}\pjya{ma-pɯ-tɯ-nɯʁndomnɤt ntsɯ}\hspace{5pt}\pcmn{你不要不停地重复讲同一句话}\end{exemple}
\begin{exemple}\pjya{a-mu cho-rgɤz tɕe, wuma ʑo nɯʁndomnɤt}\hspace{5pt}\pcmn{我母亲老了,不停地重复讲说过的话}\end{exemple}
\begin{exemple}\pjya{ɯʑo cha ku-tshi tɕe, tu-nɯʁndomnɤt ɲɯ-ŋu}\hspace{5pt}\pcmn{他喝了酒就会重复讲说过的话}\end{exemple}\relationsémantique{参考}{\lien{ⓔtaʁndo}{taʁndo}}\end{entrée}

\begin{entrée}{nɯʁnoŋ}{}{ⓔnɯʁnoŋ} 
\classe{vi} 
\begin{définition}\pfra{avoir des remors}\end{définition}
\begin{définition}\pcmn{内疚;有愧}\end{définition}\relationsémantique{同义词}{\lien{ⓔɲɟɤt}{ɲɟɤt}}\relationsémantique{参考}{\lien{ⓔtɯ-ʁnoŋ}{tɯ-ʁnoŋ}}\end{entrée}

\begin{entrée}{nɯʁɲɤlwa}{}{ⓔnɯʁɲɤlwa} 
\classe{vi} \paradigme{dir}{pɯ-}
\begin{définition}\pfra{souffrir le martyre}\end{définition}
\begin{définition}\pcmn{受苦难}\end{définition}
\begin{exemple}\pjya{pɯ-nɯʁɲɤlwa-a}\hspace{5pt}\pcmn{我受了很多苦}\end{exemple}\relationsémantique{参考}{\lien{ⓔʁɲɤlwa}{ʁɲɤlwa}}\end{entrée}

\begin{entrée}{nɯʁzɯɣ}{}{ⓔnɯʁzɯɣ} 
\classe{vs} \paradigme{dir}{thɯ-}
\begin{définition}\pfra{beau}\end{définition}
\begin{définition}\pcmn{美观}\end{définition}
\begin{exemple}\pjya{jiɕqha tɤ-tɕɯ nɯ ɲɯ-nɯʁzɯɣ}\hspace{5pt}\pcmn{那个男子很英俊}\end{exemple}
\begin{exemple}\pjya{jiɕqha laχtɕha ɲɯ-nɯʁzɯɣ}\hspace{5pt}\pcmn{那个东西好看}\end{exemple}\étymologie{gzigs}\end{entrée}

\begin{entrée}{nɯʁʑɯnɯ}{}{ⓔnɯʁʑɯnɯ} 
\classe{vs}  
\grammaire{denom} \paradigme{dir}{thɯ-}\paradigme{dir}{tɤ-}
\begin{définition}\pfra{grandir et devenir un jeune homme}\end{définition}
\begin{définition}\pcmn{长成青年}\end{définition}\relationsémantique{参考}{\lien{ⓔnɯʁʑɯnɯ}{nɯʁʑɯnɯ}}\end{entrée}

\begin{entrée}{nɯsarsi}{}{ⓔnɯsarsi} 
\classe{vi}  
\grammaire{denom} \paradigme{dir}{\_}
\begin{définition}\pfra{ramasser des abricots}\end{définition}
\begin{définition}\pcmn{摘杏}\end{définition}\relationsémantique{参考}{\lien{ⓔsarsi}{sarsi}}\end{entrée}

\begin{entrée}{nɯsaχɕɯβ}{}{ⓔnɯsaχɕɯβ} 
\classe{vi} 
\begin{définition}\pfra{faire une compétition}\end{définition}
\begin{définition}\pcmn{比试}\end{définition}
\begin{exemple}\pjya{tɕiʑo ni tɤ-nɯ-saχɕɯβ-tɕi ri, ɯʑo kɯ pɯ́-wɣ-ɕɯnŋo-a}\hspace{5pt}\pcmn{我们俩比试一下了,他把我打败了}\end{exemple}\end{entrée}

\begin{entrée}{nɯsaχsɯ}{}{ⓔnɯsaχsɯ} 
\classe{vi}  
\grammaire{denom} \paradigme{dir}{tɤ-}
\begin{définition}\pfra{prendre le repas de midi}\end{définition}
\begin{définition}\pcmn{吃中午饭}\end{définition}
\begin{exemple}\pjya{nɯsaχsɯ-j}\hspace{5pt}\pcmn{我们吃中午饭}\end{exemple}
\begin{exemple}\pjya{jɯfɕɯr sloχpɯn cho tɯɣrɤz tɤ-nɯsaχsɯ-j}\hspace{5pt}\pcmn{我们昨天跟老师一起吃了中午饭}\end{exemple}
\begin{exemple}\pjya{ɯ-tɤ-tɯ-nɯsaχsɯ?}\hspace{5pt}\pcmn{你吃饭了吗?}\end{exemple}\relationsémantique{参考}{\lien{ⓔsaχsɯⓗ1}{saχsɯ}}\end{entrée}

\begin{entrée}{nɯsɤlɤɣ}{}{ⓔnɯsɤlɤɣ} 
\classe{vi} \paradigme{dir}{nɯ}\paradigme{dir}{nɯ-}
\begin{définition}\pfra{manger de l'herbe}\end{définition}
\begin{définition}\pcmn{吃草}\end{définition}
\begin{définition}\pfra{faire manger de l'herbe}\end{définition}
\begin{définition}\pcmn{给动物喂草}\end{définition}
\begin{définition}\pcmn{我给你吃草(人对动物说)}\end{définition}
\begin{exemple}\pjya{nɯŋa ɲɯ-nɯsɤlɤɣ}\hspace{5pt}\pcmn{牛在吃草}\end{exemple}
\begin{exemple}\pjya{nɤʑo nɯ-nɯsɤlɤɣ}\hspace{5pt}\pcmn{你吃草!(人对动物说)}\end{exemple}
\begin{exemple}\pjya{ɲɯ-ta-znɯsɤlɤɣ}\end{exemple}\relationsémantique{同义词}{\lien{ⓔnɯrɯ}{nɯrɯ}}
\begin{sous-entrée}{znɯsɤlɤɣ}{ⓔnɯsɤlɤɣⓝznɯsɤlɤɣ} 
\classe{vt}  
\grammaire{caus} \end{sous-entrée}

\end{entrée}

\begin{entrée}{nɯsɤra}{}{ⓔnɯsɤra} 
\classe{vi} \paradigme{dir}{tɤ-}
\begin{définition}\pfra{se gratter (contre des arbres)}\end{définition}
\begin{définition}\pcmn{抓痒(奶牛;牦牛)}\end{définition}\end{entrée}

\begin{entrée}{nɯsɤraʁ}{}{ⓔnɯsɤraʁ} 
\classe{vt} \paradigme{dir}{tɤ-}
\begin{définition}\pfra{parier}\end{définition}
\begin{définition}\pcmn{打赌}\end{définition}
\begin{exemple}\pjya{nɯ-sɤraʁ-tɕi}\hspace{5pt}\pcmn{我们俩打赌}\end{exemple}
\begin{exemple}\pjya{tɤ-nɯsɤraʁ-tɕi}\hspace{5pt}\pcmn{我们俩打赌了}\end{exemple}
\begin{exemple}\pjya{tɤ-sɯxtsɯɣ-a tɕe tɕhi a-tɤ-fse, mɯ-tɤ-sɯxtsɯɣ-a tɕe tɕhi fse nɯ-sɤraʁ-tɕi}\hspace{5pt}\pcmn{我们俩打赌,如果我打中了的话就怎么样,打不中的话又怎么样}\end{exemple}\relationsémantique{同义词}{\lien{ⓔnɯtɤraʁ}{nɯtɤraʁ}}\end{entrée}

\begin{entrée}{nɯscɯʁzɯɣ}{}{ⓔnɯscɯʁzɯɣ} 
\classe{vs} 
\begin{définition}\pfra{beau}\end{définition}
\begin{définition}\pcmn{美丽;漂亮}\end{définition}
\begin{exemple}\pjya{jiɕqha nɯ mɯ́j-nɯscɯʁzɯɣ}\hspace{5pt}\pcmn{那个人不漂亮}\end{exemple}
\begin{exemple}\pjya{jiɕqha nɯ ɲɯ-nɯscɯʁzɯɣ}\hspace{5pt}\pcmn{那个人很漂亮}\end{exemple}\étymologie{skʲe.gzugs}\end{entrée}

\begin{entrée}{nɯsɣa}{}{ⓔnɯsɣa} 
\classe{vs}  
\grammaire{denom} \paradigme{dir}{kɤ-}
\begin{définition}\pfra{rouiller}\end{définition}
\begin{définition}\pcmn{生锈}\end{définition}
\begin{exemple}\pjya{ɕom ko-nɯsɣa}\hspace{5pt}\pcmn{铁生锈了}\end{exemple}
\begin{exemple}\pjya{laʁdɯn ɲɯ-nɯsɣa}\hspace{5pt}\pcmn{工具生锈}\end{exemple}\relationsémantique{参考}{\lien{ⓔsɣa}{sɣa}}\end{entrée}

\begin{entrée}{nɯskɤt}{}{ⓔnɯskɤt} 
\classe{vt} 
\begin{définition}\pfra{parler}\end{définition}
\begin{définition}\pcmn{说话}\end{définition}
\begin{exemple}\pjya{nɤʑo mɤ-tɯ-nɯskɤt me}\hspace{5pt}\pcmn{你无话不说}\end{exemple}
\begin{exemple}\pjya{ɯʑo kɯ mɯ-ta-nɯskɤt ʑo me}\hspace{5pt}\pcmn{他以前说很多话}\end{exemple}
\begin{exemple}\pjya{nɯ kɯ-fse kɤ-nɯskɤt mɤ-ra}\hspace{5pt}\pcmn{不要这样说}\end{exemple}
\begin{exemple}\pjya{mɤ-nɯskɤt-ndʑi ʑo maŋe}\hspace{5pt}\pcmn{他们俩没有什么不说的}\end{exemple}\relationsémantique{参考}{\lien{ⓔtɯ-skɤt}{tɯ-skɤt}}\end{entrée}

\begin{entrée}{nɯskhrɯ}{}{ⓔnɯskhrɯ} 
\classe{vt} \paradigme{dir}{kɤ-}
\begin{définition}\pfra{être enceinte d'un enfant}\end{définition}
\begin{définition}\pcmn{怀上(孩子)}\end{définition}
\begin{exemple}\pjya{ɯ-rɟit ko-nɯskhrɯ}\hspace{5pt}\pcmn{她怀上了孩子}\end{exemple}\relationsémantique{参考}{\lien{ⓔtɯ-skhrɯ}{tɯ-skhrɯ}}\end{entrée}

\begin{entrée}{nɯslɯɣ}{}{ⓔnɯslɯɣ} 
\classe{vi} \paradigme{dir}{tɤ-}\paradigme{dir}{tɤ-}
\begin{définition}\pfra{être hourdé de, être sali}\end{définition}
\begin{définition}\pcmn{沾上}\end{définition}
\begin{définition}\pfra{salir avec}\end{définition}
\begin{définition}\pcmn{使沾上}\end{définition}
\begin{exemple}\pjya{a-ŋga ɯ-thoʁ pjɤ-k-ɤtɤr-ci tɕe to-nɯslɯɣ}\hspace{5pt}\pcmn{我衣服掉到地上,沾上了灰尘}\end{exemple}
\begin{exemple}\pjya{soʁma ɯ-ŋgɯ ju-kɯ-ɕe tɕe, tɯ-ŋga tu-nɯslɯɣ ŋu}\hspace{5pt}\pcmn{走进麦草里,衣服会沾上麦草}\end{exemple}
\begin{exemple}\pjya{ɯʑo kɯ ɯ-ŋga tɤrcoʁ to-znɯslɯɣ}\hspace{5pt}\pcmn{他把自己衣服沾上了泥巴}\end{exemple}
\begin{sous-entrée}{znɯslɯɣ}{ⓔnɯslɯɣⓝznɯslɯɣ} 
\classe{vt} \end{sous-entrée}

\end{entrée}

\begin{entrée}{nɯslɯt}{}{ⓔnɯslɯt} 
\classe{vi} \paradigme{dir}{tɤ-}\paradigme{dir}{tɤ-}
\begin{définition}\pfra{être plein de poussières et de saletés (après avoir été laissé dans un endroit sale)}\end{définition}
\begin{définition}\pcmn{(不管地下脏躺在哪里)满身都是灰尘,沾满灰尘和脏东西}\end{définition}
\begin{exemple}\pjya{a-ŋga to-nɯslɯt}\hspace{5pt}\pcmn{我衣服上沾满了灰尘}\end{exemple}
\begin{exemple}\pjya{nɤ-tɕɯ kɯ ɯ-ŋga ɯ-tó-znɯslɯt?}\hspace{5pt}\pcmn{你儿子把衣服放在那里沾上灰尘了吗?}\end{exemple}
\begin{exemple}\pjya{ɯ-thoʁ ɲɯ-ɴqhi tɕe ɲɯ-kɯ-znɯslɯt}\hspace{5pt}\pcmn{地面很脏,令人沾上灰尘}\end{exemple}
\begin{sous-entrée}{znɯslɯt}{ⓔnɯslɯtⓝznɯslɯt} 
\classe{vt} \end{sous-entrée}

\end{entrée}

\begin{entrée}{nɯsmɤn}{}{ⓔnɯsmɤn} 
\classe{vt}  
\grammaire{denom} \paradigme{dir}{tɤ-}\paradigme{dir}{tɤ-}
\begin{définition}\pfra{guérir}\end{définition}
\begin{définition}\pcmn{治病}\end{définition}
\begin{définition}\pfra{guérir avec}\end{définition}
\begin{définition}\pcmn{用……治病}\end{définition}
\begin{exemple}\pjya{nɯnɯ ɣɯ ɯ-jaʁ pjɤ-ɴɢraʁ tɕe to-nɯsmɤn}\hspace{5pt}\pcmn{他的手破了,(医生)就把它治好了}\end{exemple}
\begin{exemple}\pjya{smɤnba kɯ to-nɯsmɤn}\hspace{5pt}\pcmn{医生把他的病治好了}\end{exemple}
\begin{exemple}\pjya{smɤnba kɯ smɤn to-βzu tɕe to-nɯsmɤn}\hspace{5pt}\pcmn{医生用药把他治好了}\end{exemple}
\begin{exemple}\pjya{ɲɯ-ngo tɕe ta-nɯsmɤn}\hspace{5pt}\pcmn{他病了,(医生)把病治好了}\end{exemple}
\begin{exemple}\pjya{smɤnba kɯ tɤ́-wɣ-nɯsman-a}\hspace{5pt}\pcmn{医生给我治了病}\end{exemple}
\begin{exemple}\pjya{ɯ-smɤn ra la-tsɯm tɕe ɕ-to-z-nɯsmɤn tɕe to-pe}\hspace{5pt}\pcmn{他带了药,就把她治好了}\end{exemple}\relationsémantique{参考}{\lien{ⓔaɣɯsmɤn}{aɣɯsmɤn}}\relationsémantique{参考}{\lien{ⓔsmɤn}{smɤn}}
\begin{sous-entrée}{ʑɣɤnɯsmɤn}{ⓔnɯsmɤnⓝʑɣɤnɯsmɤn} 
\classe{vi}  
\grammaire{refl} 
\begin{définition}\pfra{se traiter}\end{définition}
\begin{définition}\pcmn{看病}\end{définition}
\begin{exemple}\pjya{nɤʑo ɕɯ-tɯ-ʑɣɤnɯsmɤn ɯ-ŋu?}\hspace{5pt}\pcmn{你去看病}\end{exemple}\end{sous-entrée}

\begin{sous-entrée}{znɯsmɤn}{ⓔnɯsmɤnⓝznɯsmɤn} 
\classe{vt}  
\grammaire{caus} \end{sous-entrée}

\end{entrée}

\begin{entrée}{nɯsmɤphɤβ}{}{ⓔnɯsmɤphɤβ} 
\classe{vt} \paradigme{dir}{pɯ-}
\begin{définition}\pfra{humilier}\end{définition}
\begin{définition}\pcmn{侮辱;污蔑}\end{définition}
\begin{exemple}\pjya{pjɤ-nɯsmɤphɤβ}\hspace{5pt}\pcmn{他侮辱了他}\end{exemple}
\begin{exemple}\pjya{pjɯ-kɯ-nɯsmɤphaβ-a ɲɯ-ŋu}\hspace{5pt}\pcmn{你在侮辱我}\end{exemple}\end{entrée}

\begin{entrée}{nɯsmɯɣjɯm}{}{ⓔnɯsmɯɣjɯm} 
\classe{vi} 
\begin{définition}\pfra{se chauffer au feu}\end{définition}
\begin{définition}\pcmn{烤火取暖}\end{définition}\relationsémantique{同义词}{\lien{ⓔnɯmbjɯm}{nɯmbjɯm}}\end{entrée}

\begin{entrée}{nɯsmɯlɤm}{}{ⓔnɯsmɯlɤm} 
\classe{vt} \paradigme{dir}{nɯ-}
\begin{définition}\pfra{espérer}\end{définition}
\begin{définition}\pcmn{愿望;盼望}\end{définition}
\begin{exemple}\pjya{ɬasa kɤ-ɕe ɲɯ-nɯsmɯlɤm}\hspace{5pt}\pcmn{他很希望去拉萨}\end{exemple}
\begin{exemple}\pjya{nɤ-kɤ-nɯsmɯlɤm nɯ a-pɯ-fse}\hspace{5pt}\pcmn{祝你愿望实现}\end{exemple}
\begin{exemple}\pjya{pjɯ-kɤ-cha nɯ ɲɯ-nɯsmɯlam-a}\hspace{5pt}\pcmn{我希望(这件事情)成功}\end{exemple}\relationsémantique{参考}{\lien{ⓔsmɯlɤm}{smɯlɤm}}\end{entrée}

\begin{entrée}{nɯsmɯrjɯɣ}{}{ⓔnɯsmɯrjɯɣ} 
\classe{vt} \paradigme{dir}{thɯ-}
\begin{définition}\pfra{courber à la chaleur}\end{définition}
\begin{définition}\pcmn{用高温令木条变形;弄弯}\end{définition}
\begin{exemple}\pjya{ʑmbrɯɟoʁ ci ɣɤʑu tɕe nɯ-nɯsmɯrjɯɣ-a}\hspace{5pt}\pcmn{我把杨柳枝条弄弯了}\end{exemple}
\begin{exemple}\pjya{kɯki si ki thɯ-nɯsmɯrjɯɣ-a}\hspace{5pt}\pcmn{我把这块木条弄弯了}\end{exemple}
\begin{exemple}\pjya{kɯki ɲɯ-jpum tɕe, kɤ-nɯsmɯrjɯɣ mɯ́j-khɯ}\hspace{5pt}\pcmn{这个东西很粗,不能弄弯}\end{exemple}\end{entrée}

\begin{entrée}{nɯsnɯɲaʁ}{}{ⓔnɯsnɯɲaʁ} 
\classe{vt}  
\grammaire{incorp} \paradigme{dir}{tɤ-}
\begin{définition}\pfra{causer du mal}\end{définition}
\begin{définition}\pcmn{伤害,陷害}\end{définition}
\begin{exemple}\pjya{tɤ-nɯsnɯɲaʁ-a}\hspace{5pt}\pcmn{我陷害了他}\end{exemple}
\begin{exemple}\pjya{jiɕqha nɯ kɯ tú-wɣ-nɯsnɯsɲaʁ-a ɲɯ-ŋu}\hspace{5pt}\pcmn{那个人在陷害我}\end{exemple}
\begin{exemple}\pjya{tɤ́-wɣ-nɯsnɯɲaʁ-a}\hspace{5pt}\pcmn{他陷害了我}\end{exemple}\relationsémantique{参考}{\lien{ⓔsnɯɲaʁ}{snɯɲaʁ}}
\begin{sous-entrée}{sɤnɯsnɯɲaʁ}{ⓔnɯsnɯɲaʁⓝsɤnɯsnɯɲaʁ} 
\classe{vi}  
\grammaire{incorp}
\grammaire{apass} 
\begin{définition}\pfra{causer du mal aux gens}\end{définition}
\begin{définition}\pcmn{害人}\end{définition}\end{sous-entrée}

\end{entrée}

\begin{entrée}{nɯsɲaŋne}{}{ⓔnɯsɲaŋne} 
\classe{vi} \paradigme{dir}{pɯ-}
\begin{définition}\pfra{jeûner}\end{définition}
\begin{définition}\pcmn{念哑巴经(禁食斋)}\end{définition}
\begin{exemple}\pjya{pɯ-nɯsɲaŋne-a}\hspace{5pt}\pcmn{我念了哑巴经}\end{exemple}\relationsémantique{参考}{\lien{ⓔsɲaŋne}{sɲaŋne}}\relationsémantique{参考}{\lien{ⓔrɯsɲaŋne}{rɯsɲaŋne}}\end{entrée}

\begin{entrée}{nɯsɲɤtqha}{}{ⓔnɯsɲɤtqha} 
\classe{vi}  
\grammaire{incorp} \paradigme{dir}{tɤ-}
\begin{définition}\pfra{faire une ruade (lorsque le cheval ne supporte plus sa selle)}\end{définition}
\begin{définition}\pcmn{(因为受不了鞍子)尥蹶子}\end{définition}
\begin{exemple}\pjya{mbro to-nɯsɲɤtqha}\hspace{5pt}\pcmn{马尥蹶子了}\end{exemple}\relationsémantique{同义词}{\lien{ⓔcɤmtsaʁ}{cɤmtsaʁ}}\relationsémantique{参考}{\lien{ⓔsɲɤt}{sɲɤt}}\relationsémantique{参考}{\lien{ⓔqha}{qha}}\end{entrée}

\begin{entrée}{nɯsɲɯβri}{}{ⓔnɯsɲɯβri} 
\classe{vt} 
\begin{définition}\pfra{chérir}\end{définition}
\begin{définition}\pcmn{心疼}\end{définition}\relationsémantique{同义词}{\lien{ⓔnɤrɕɤmŋɤm}{nɤrɕɤmŋɤm}}\end{entrée}

\begin{entrée}{nɯsŋom}{}{ⓔnɯsŋom} 
\classe{vt}  
\grammaire{appl} \paradigme{dir}{nɯ-}
\begin{définition}\pfra{désirer, convoiter}\end{définition}
\begin{définition}\pcmn{贪}\end{définition}
\begin{exemple}\pjya{nɯ-nɯsŋom-a}\hspace{5pt}\pcmn{我贪(这个东西)了}\end{exemple}
\begin{exemple}\pjya{aʑo jiɕqha tɯ-xtsa nɯ ɲɯ-nɯsŋom-a}\hspace{5pt}\pcmn{我很想得到那双鞋子}\end{exemple}
\begin{exemple}\pjya{tɯ-ŋga nɯ ɲɯ-nɯsŋom-a}\hspace{5pt}\pcmn{我很想得到那件衣服}\end{exemple}
\begin{exemple}\pjya{nɯ kɤ-χtɯ ma-nɯ-tɯ-nɯsŋom}\hspace{5pt}\pcmn{你不要有买那个的欲望(买不完)}\end{exemple}
\begin{exemple}\pjya{tɯ-rɟɯ nɯ-nɯsŋom-a}\hspace{5pt}\pcmn{我贪了财}\end{exemple}\relationsémantique{参考}{\lien{ⓔsŋom}{sŋom}}\end{entrée}

\begin{entrée}{nɯspjɤtɕha}{}{ⓔnɯspjɤtɕha} 
\classe{vt} \paradigme{dir}{tɤ-}
\begin{définition}\pfra{faire une mauvaise action}\end{définition}
\begin{définition}\pcmn{做坏事;搞鬼}\end{définition}
\begin{exemple}\pjya{ki nɤʑo tɤ-tɯ-nɯspjɤtɕha-t ŋu!}\hspace{5pt}\pcmn{这是你搞的鬼!}\end{exemple}\relationsémantique{参考}{\lien{ⓔspjɤtɕha}{spjɤtɕha}}\end{entrée}

\begin{entrée}{nɯsqar}{}{ⓔnɯsqar} 
\classe{vs} \paradigme{dir}{tɤ-}
\begin{définition}\pfra{facile à séparer (fils)}\end{définition}
\begin{définition}\pcmn{容易分开}\end{définition}
\begin{exemple}\pjya{ki kɤ-taʁ ɲɯ-mbat ma ɲɯ-nɯsqar}\hspace{5pt}\pcmn{因为(线)容易分开,所以很好织}\end{exemple}\relationsémantique{参考}{\lien{ⓔɯ-sqar}{ɯ-sqar}}\end{entrée}

\begin{entrée}{nɯsroʁmbrɤt}{}{ⓔnɯsroʁmbrɤt} 
\classe{vi}  
\grammaire{incorp} \paradigme{dir}{thɯ-}
\begin{définition}\pfra{se débattre en agonisant}\end{définition}
\begin{définition}\pcmn{临死挣扎}\end{définition}\relationsémantique{参考}{\lien{ⓔmbrɤt}{mbrɤt}}\relationsémantique{参考}{\lien{ⓔtɯ-sroʁ}{tɯ-sroʁ}}\end{entrée}

\begin{entrée}{nɯsrɯɣndɤr}{}{ⓔnɯsrɯɣndɤr} 
\classe{vi} \paradigme{dir}{nɯ-}
\begin{définition}\pfra{avoir de l'acné}\end{définition}
\begin{définition}\pcmn{长青春痘}\end{définition}
\begin{exemple}\pjya{ɯʑo ɲɯ-nɯsrɯɣndɤr}\hspace{5pt}\pcmn{他长青春痘}\end{exemple}\relationsémantique{参考}{\lien{ⓔsrɯndɤr}{srɯndɤr}}\end{entrée}

\begin{entrée}{nɯstu}{}{ⓔnɯstu} 
\classe{vs} \paradigme{dir}{tɤ-}\paradigme{dir}{\_}
\begin{définition}\pfra{être correct}\end{définition}
\begin{définition}\pcmn{准}\end{définition}
\begin{exemple}\pjya{ɯ-jaʁ ɲɯ-nɯstu}\hspace{5pt}\pcmn{他的枪法很准}\end{exemple}\relationsémantique{参考}{\lien{ⓔɯ-stuⓗ2}{ɯ-stu₂}}
\begin{sous-entrée}{znɯstu}{ⓔnɯstuⓝznɯstu} 
\classe{vt}  
\grammaire{caus} \end{sous-entrée}

\sens{1}
\begin{définition}\pfra{faire correctement}\end{définition}
\begin{définition}\pcmn{做得准}\end{définition}
\begin{exemple}\pjya{kɤ-ti ta-z-nɯstu}\hspace{5pt}\pcmn{他说话说准了}\end{exemple}
\begin{exemple}\pjya{kɤ-ɕe ka-z-nɯstu}\hspace{5pt}\pcmn{他走准了}\end{exemple}\sens{2}
\begin{définition}\pfra{atteindre la cible}\end{définition}
\begin{définition}\pcmn{射中}\end{définition}\end{entrée}

\begin{entrée}{nɯstɤβtshɤt}{}{ⓔnɯstɤβtshɤt} 
\classe{vi} \paradigme{dir}{tɤ-}
\begin{définition}\pfra{faire un concours de force}\end{définition}
\begin{définition}\pcmn{比力气}\end{définition}\relationsémantique{参考}{\lien{ⓔstɤβtshɤt}{stɤβtshɤt}}\end{entrée}

\begin{entrée}{nɯstɤraʁndo}{}{ⓔnɯstɤraʁndo} 
\classe{vi} \paradigme{dir}{tɤ-}
\begin{définition}\pfra{se parler à soi-même}\end{définition}
\begin{définition}\pcmn{喃喃自语;自己跟自己说话}\end{définition}
\begin{exemple}\pjya{tɤ-nɯstɤraʁndo-a}\hspace{5pt}\pcmn{我自己跟自己说话了}\end{exemple}\relationsémantique{参考}{\lien{ⓔtaʁndo}{taʁndo}}\relationsémantique{参考}{\lien{}{ɯ-sti}}\end{entrée}

\begin{entrée}{nɯstɤrɟɯɣ}{}{ⓔnɯstɤrɟɯɣ} 
\classe{vt}  
\grammaire{deidph} \paradigme{dir}{\_}
\begin{définition}\pfra{courir}\end{définition}
\begin{définition}\pcmn{跑}\end{définition}
\begin{exemple}\pjya{ɲɯ-nɯstɤrɟɯɣ}\hspace{5pt}\pcmn{他在跑}\end{exemple}
\begin{exemple}\pjya{ɲɯ-nɯstɤrɟɯɣ tɕe kɤ-ari}\hspace{5pt}\pcmn{他跑去了}\end{exemple}\relationsémantique{参考}{\lien{ⓔstɤrɟɯɣ}{stɤrɟɯɣ}}\relationsémantique{参考}{\lien{ⓔrɟɯɣⓗ1}{rɟɯɣ₁}}\end{entrée}

\begin{entrée}{nɯsthamtɕɤt}{}{ⓔnɯsthamtɕɤt} 
\classe{adv} 
\begin{définition}\pfra{autant}\end{définition}
\begin{définition}\pcmn{那么多}\end{définition}\étymologie{tʰams.tɕad}\end{entrée}

\begin{entrée}{nɯsthoʁ}{}{ⓔnɯsthoʁ} 
\classe{vt} \paradigme{dir}{pɯ-}
\begin{définition}\pfra{avoir des relations sexuelles}\end{définition}
\begin{définition}\pcmn{性交}\end{définition}
\begin{exemple}\pjya{tɕheme pɯ-nɯsthoʁ-a, pɯ-tɯ-nɯsthoʁ}\end{exemple}\relationsémantique{参考}{\lien{ⓔsthoʁ}{sthoʁ}}\end{entrée}

\begin{entrée}{nɯsthɯt}{}{ⓔnɯsthɯt}\relationsémantique{参考}{\lien{ⓔsthɯt}{sthɯt}}\end{entrée}

\begin{entrée}{nɯsɯku}{}{ⓔnɯsɯku} 
\classe{vi}  
\grammaire{denom} \paradigme{dir}{tɤ-}
\begin{définition}\pfra{grimper aux arbres}\end{définition}
\begin{définition}\pcmn{爬树}\end{définition}
\begin{exemple}\pjya{ɣzɯ kɤ-nɯsɯku ɲɯ-cha}\hspace{5pt}\pcmn{猴子会爬树}\end{exemple}
\begin{exemple}\pjya{tɯrme kɤ-nɯsɯku kɯ-cha tu, nɯnɯ kɯ-cha mɤ-kɯ-cha tu}\hspace{5pt}\pcmn{有的人会爬树,这件事,有的会有的不会}\end{exemple}
\begin{exemple}\pjya{ma-tɯ-nɯsɯku ma tɯ-atɤr}\hspace{5pt}\pcmn{你不要爬树,你会摔下来的}\end{exemple}\relationsémantique{参考}{\lien{ⓔsɯku}{sɯku}}\end{entrée}

\begin{entrée}{nɯsɯkho}{}{ⓔnɯsɯkho} 
\classe{vt} \paradigme{dir}{nɯ-}
\begin{définition}\pfra{extorquer, dévaliser}\end{définition}
\begin{définition}\pcmn{抢}\end{définition}
\begin{exemple}\pjya{nɯ ma-nɯ-tɯ-nɯsɯkhɤm}\hspace{5pt}\pcmn{你不要抢这个东西}\end{exemple}
\begin{exemple}\pjya{paʁmu kɯ ɯ-rɟit kɤ-ndza ɲɯ-nɯsɯkhɤm ɲɯ-ŋu}\hspace{5pt}\pcmn{母猪总是抢它的崽子的食物}\end{exemple}
\begin{exemple}\pjya{tɤ-pɤtso kɯ ɯ-zda ɯ-kɯmtɕhɯ na-nɯsɯkho}\hspace{5pt}\pcmn{小孩子抢了他的同学的玩具}\end{exemple}
\begin{exemple}\pjya{ma-nɯ-kɯ-nɯsɯkho-a}\hspace{5pt}\pcmn{你别抢我的东西}\end{exemple}\relationsémantique{参考}{\lien{ⓔkhoⓗ1}{kho₁}}\relationsémantique{参考}{\lien{ⓔanɯsɯkhɯkho}{anɯsɯkhɯkho}}
\begin{sous-entrée}{sɤnɯsɯkho/\variante{nɯsɤsɯkho}}{ⓔnɯsɯkhoⓝsɤnɯsɯkho} 
\classe{vi} 
\begin{définition}\pfra{extorquer, dévaliser les gens}\end{définition}
\begin{définition}\pcmn{抢人家的东西}\end{définition}
\begin{exemple}\pjya{ɯʑo sɤnɯsɯkho ŋgrɤl}\hspace{5pt}\pcmn{他抢别人的东西}\end{exemple}\relationsémantique{同义词}{\lien{ⓔnɯtɕaχpa}{nɯtɕaχpa}}\end{sous-entrée}

\begin{sous-entrée}{znɯsɤsɯkho}{ⓔnɯsɯkhoⓝznɯsɤsɯkho} 
\classe{vt} 
\begin{définition}\pfra{inciter / forcer à dévaliser les gens}\end{définition}
\begin{définition}\pcmn{致使……抢东西}\end{définition}\end{sous-entrée}

\end{entrée}

\begin{entrée}{nɯsɯmŋɤn}{}{ⓔnɯsɯmŋɤn} 
\classe{vt} \paradigme{dir}{tɤ-}
\begin{définition}\pfra{se méfier}\end{définition}
\begin{définition}\pcmn{怀疑}\end{définition}
\begin{exemple}\pjya{jiɕqha nɯ ŋu maʁ mɤ-xsi ri tɤ-nɯsɯmŋan-a}\hspace{5pt}\pcmn{不知道是不是他(偷了东西),但是我怀疑他了}\end{exemple}
\begin{exemple}\pjya{ma-tɤ-kɯ-nɯsɯmŋan-a ma aʑo maʁ-a}\hspace{5pt}\pcmn{你不要怀疑我,不是我做的}\end{exemple}
\begin{exemple}\pjya{ma-tɤ-tɯ-nɯsɯmŋɤn}\hspace{5pt}\pcmn{你不要怀疑他}\end{exemple}
\begin{exemple}\pjya{sɯmŋɤn}\end{exemple}\étymologie{sems.ŋan}\end{entrée}

\begin{entrée}{nɯsɯmʁɲiz}{}{ⓔnɯsɯmʁɲiz} 
\classe{vi} \paradigme{dir}{tɤ-}
\begin{définition}\pfra{hésiter}\end{définition}
\begin{définition}\pcmn{犹豫}\end{définition}
\begin{exemple}\pjya{nɤʑo ɲɯ-tɯ-nɯsɯmʁɲiz netɕi}\hspace{5pt}\pcmn{你还在犹豫啊}\end{exemple}\étymologie{sems.gɲis}\end{entrée}

\begin{entrée}{nɯsɯmɯzdɯɣ}{}{ⓔnɯsɯmɯzdɯɣ} 
\classe{vi}  
\grammaire{denom} \paradigme{dir}{thɯ-}\sens{1}
\begin{définition}\pfra{être inquiet}\end{définition}
\begin{définition}\pcmn{担心}\end{définition}
\begin{exemple}\pjya{nɤ-rɟit ɲɯ-ngo, ma-tɯ-nɯsɯmɯzdɯɣ}\hspace{5pt}\pcmn{你儿子生病了,但是不要担心}\end{exemple}\sens{2}
\begin{définition}\pfra{être malheureux, être triste}\end{définition}
\begin{définition}\pcmn{不高兴;伤心}\end{définition}
\begin{exemple}\pjya{thɯ-nɯsɯmɯzdɯɣ mɤ-ra}\hspace{5pt}\pcmn{你不要伤心}\end{exemple}
\begin{exemple}\pjya{ma-tɯ-nɯsɯmɯzdɯɣ}\hspace{5pt}\pcmn{你不要伤心}\end{exemple}
\begin{exemple}\pjya{tɕhindʐa ku-tɯ-nɯsɯmɯzdɯɣ?}\hspace{5pt}\pcmn{你为什么不高兴?}\end{exemple}\relationsémantique{参考}{\lien{ⓔsɯmɯzdɯɣ}{sɯmɯzdɯɣ}}\étymologie{sems.sdug}\end{entrée}

\begin{entrée}{nɯsɯŋgɯ}{}{ⓔnɯsɯŋgɯ} 
\classe{vi} 
\begin{définition}\pfra{aller dans la forêt}\end{définition}
\begin{définition}\pcmn{在森林里逛;打猎}\end{définition}
\begin{exemple}\pjya{kɯ-nɯsɯŋgɯ jɤ-ari-a}\hspace{5pt}\pcmn{我去了森林}\end{exemple}
\begin{exemple}\pjya{ɕɯ-nɯsɯŋgɯ-a}\hspace{5pt}\pcmn{我要去森林}\end{exemple}
\begin{exemple}\pjya{nɤʑo ɕɯ-tɯ-nɯsɯŋgɯ ɯ-ŋu?}\hspace{5pt}\pcmn{你去森林吗?}\end{exemple}\relationsémantique{参考}{\lien{ⓔsɯŋgɯ}{sɯŋgɯ}}\end{entrée}

\begin{entrée}{nɯsɯɴɢoʁ}{}{ⓔnɯsɯɴɢoʁ} 
\classe{vi}  
\grammaire{denom} \paradigme{dir}{nɯ-}\paradigme{dir}{\_}
\begin{définition}\pfra{ramasser du bois mort}\end{définition}
\begin{définition}\pcmn{捡干柴}\end{définition}
\begin{exemple}\pjya{ji-si chɤ-k-ɤrɕo-ci tɕe, ʑ-nɯ-nɯsɯɴɢoʁ-a pɯ-ra}\hspace{5pt}\pcmn{我们的柴用完了,所以我就捡柴去了}\end{exemple}
\begin{exemple}\pjya{sɯŋgɯ pɯ-nɯsɯɴɢoʁ-a}\hspace{5pt}\pcmn{我在森林里捡柴了}\end{exemple}\relationsémantique{参考}{\lien{ⓔsiⓗ1}{si₁}}\relationsémantique{参考}{\lien{ⓔtaɴɢoʁ}{taɴɢoʁ}}\end{entrée}

\begin{entrée}{nɯsɯqaʁ}{}{ⓔnɯsɯqaʁ} 
\classe{vt} \paradigme{dir}{tɤ-}
\begin{définition}\pfra{délimiter un endroit à construire}\end{définition}
\begin{définition}\pcmn{划定,定下(将来修房子、种田的地方)}\end{définition}
\begin{exemple}\pjya{kha ɯ-sta tɤ-nɯsɯqaʁ-a}\hspace{5pt}\pcmn{我选定了修房子的地皮}\end{exemple}
\begin{exemple}\pjya{ɯʑo ɯ-ji to-nɯsɯqaʁ}\hspace{5pt}\pcmn{他划定了田界}\end{exemple}\end{entrée}

\begin{entrée}{nɯsɯrtoʁ}{}{ⓔnɯsɯrtoʁ} 
\classe{vt} 
\begin{définition}\pfra{s'apercevoir}\end{définition}
\begin{définition}\pcmn{发觉,看得出}\end{définition}
\begin{exemple}\pjya{tɯʑo mɤ-kɯ-pe nɯ a-pɯ́-wɣ-nɯsɯrtoʁ tɕe pe}\hspace{5pt}\pcmn{发现自己的缺点是一件好事}\end{exemple}
\begin{exemple}\pjya{ɲɯ-nɯsɯrtoʁ-a}\hspace{5pt}\pcmn{我看得出}\end{exemple}
\begin{exemple}\pjya{nɤki tɤ-scoz nɯ kɤ-rɤt ɲɤ-tɯ-nɯkɯmaʁ ri, mɯ́j-tɯ-nɯsɯrtoʁ}\hspace{5pt}\pcmn{你写错了字,但是你没有看出来}\end{exemple}
\begin{exemple}\pjya{nɤki tʂu kɯmaʁ jo-tɯ-nɯɕe ri mɯ́j-tɯ-nɯsɯrtoʁ}\hspace{5pt}\pcmn{你走错了路,但是你没有看出来}\end{exemple}\relationsémantique{同义词}{\lien{ⓔsɯχsɤl}{sɯχsɤl}}\relationsémantique{参考}{\lien{ⓔrtoʁ}{rtoʁ}}\end{entrée}

\begin{entrée}{nɯsɯzʁe}{}{ⓔnɯsɯzʁe} 
\classe{vi}  
\grammaire{incorp} \paradigme{dir}{\_}
\begin{définition}\pfra{porter du bois (sur le dos)}\end{définition}
\begin{définition}\pcmn{(来回)背柴}\end{définition}\relationsémantique{参考}{\lien{ⓔsiⓗ1}{si}}\relationsémantique{参考}{\lien{ⓔnɯzʁe}{nɯzʁe}}\relationsémantique{参考}{\lien{ⓔsɯzʁe}{sɯzʁe}}\end{entrée}

\begin{entrée}{nɯsuwa}{}{ⓔnɯsuwa} 
\classe{vi} \paradigme{dir}{pɯ-}
\begin{définition}\pfra{monter la garde}\end{définition}
\begin{définition}\pcmn{站岗;放哨}\end{définition}
\begin{exemple}\pjya{ɯʑo ku-nɯsuwa}\hspace{5pt}\pcmn{他在放哨}\end{exemple}\relationsémantique{同义词}{\lien{ⓔnɯchɯra}{nɯchɯra}}\relationsémantique{参考}{\lien{ⓔsuwa}{suwa}}\end{entrée}

\begin{entrée}{nɯt}{}{ⓔnɯt} 
\classe{vi} \paradigme{dir}{tɤ-}
\begin{définition}\pfra{brûler}\end{définition}
\begin{définition}\pcmn{燃起}\end{définition}
\begin{exemple}\pjya{smi to-nɯt}\hspace{5pt}\pcmn{火燃起来了}\end{exemple}
\begin{exemple}\pjya{si to-nɯt}\hspace{5pt}\pcmn{柴火燃起来了}\end{exemple}\end{entrée}

\begin{entrée}{nɯta}{}{ⓔnɯta}\relationsémantique{参考}{\lien{ⓔta}{ta}}\end{entrée}

\begin{entrée}{nɯtal}{}{ⓔnɯtal} 
\classe{vi} \paradigme{dir}{tɤ-}
\begin{définition}\pfra{jouer au jianzi}\end{définition}
\begin{définition}\pcmn{踢毽子}\end{définition}\relationsémantique{参考}{\lien{ⓔtal}{tal}}\end{entrée}

\begin{entrée}{nɯtɤpɤtso}{}{ⓔnɯtɤpɤtso} 
\classe{vt} \paradigme{dir}{tɤ-}
\begin{définition}\pfra{considérer comme un enfant}\end{définition}
\begin{définition}\pcmn{把……当成小孩子}\end{définition}
\begin{exemple}\pjya{tú-wɣ-nɯtɤpɤtso-a ŋgrɤl}\hspace{5pt}\pcmn{他把我当成小孩子}\end{exemple}\relationsémantique{参考}{\lien{ⓔtɤ-pɤtso}{tɤ-pɤtso}}\end{entrée}

\begin{entrée}{nɯtɤraʁ}{}{ⓔnɯtɤraʁ} 
\classe{vt}  
\grammaire{denom} \paradigme{dir}{tɤ-}
\begin{définition}\pfra{parier}\end{définition}
\begin{définition}\pcmn{打赌}\end{définition}
\begin{exemple}\pjya{tɤ-nɯtɤraʁ-ndʑi}\hspace{5pt}\pcmn{他们俩打赌了}\end{exemple}\relationsémantique{同义词}{\lien{ⓔnɯsɤraʁ}{nɯsɤraʁ}}\end{entrée}

\begin{entrée}{nɯtɕu}{}{ⓔnɯtɕu} 
\classe{adv} \sens{1}
\begin{définition}\pfra{là-bas}\end{définition}
\begin{définition}\pcmn{在那里}\end{définition}\sens{2}
\begin{définition}\pfra{à ce moment-là}\end{définition}
\begin{définition}\pcmn{在那个时候}\end{définition}\end{entrée}

\begin{entrée}{nɯtɕarloŋ}{}{ⓔnɯtɕarloŋ} 
\classe{vi} \paradigme{dir}{pɯ-}\paradigme{dir}{pɯ-}
\begin{définition}\pfra{avoir une sensation désagréable (après avoir bu un thé trop fort)}\end{définition}
\begin{définition}\pcmn{醉茶}\end{définition}
\begin{définition}\pfra{causer une réaction désagréable (thé)}\end{définition}
\begin{définition}\pcmn{使……醉茶}\end{définition}
\begin{exemple}\pjya{tʂha kɯ-sna kɯ-tɕhom kɤ-tshi-t-a pɯ́-wɣ-znɯtɕarloŋ-a}\hspace{5pt}\pcmn{我喝了过浓的茶就醉茶了}\end{exemple}
\begin{sous-entrée}{znɯtɕarloŋ}{ⓔnɯtɕarloŋⓝznɯtɕarloŋ} 
\classe{vi} \end{sous-entrée}

\end{entrée}

\begin{entrée}{nɯtɕaχpa}{}{ⓔnɯtɕaχpa} 
\classe{vt}  
\grammaire{denom} \paradigme{dir}{pɯ-}
\begin{définition}\pfra{extorquer}\end{définition}
\begin{définition}\pcmn{抢}\end{définition}
\begin{exemple}\pjya{tʂu tɕe pjɤ́-wɣ-nɯtɕaχpa}\hspace{5pt}\pcmn{他在路上被抢了}\end{exemple}\relationsémantique{同义词}{\lien{ⓔnɯsɯkho}{nɯsɯkho}}\relationsémantique{参考}{\lien{ⓔtɕaχpa}{tɕaχpa}}\étymologie{dʑag.pa}\end{entrée}

\begin{entrée}{nɯtɕɤmɯ}{}{ⓔnɯtɕɤmɯ} 
\classe{vi}  
\grammaire{denom} \paradigme{dir}{lɤ-}
\begin{définition}\pfra{devenir none}\end{définition}
\begin{définition}\pcmn{当尼姑}\end{définition}\relationsémantique{参考}{\lien{ⓔtɕɤmɯ}{tɕɤmɯ}}\relationsémantique{参考}{\lien{ⓔrɯtɕɤmɯ}{rɯtɕɤmɯ}}\end{entrée}

\begin{entrée}{nɯtɕetha}{}{ⓔnɯtɕetha} 
\classe{vt} \paradigme{dir}{kɤ-}
\begin{définition}\pfra{sonder}\end{définition}
\begin{définition}\pcmn{试探}\end{définition}
\begin{exemple}\pjya{ɯʑo kɯ kɤ́-wɣ-nɯtɕetha}\hspace{5pt}\pcmn{他试探我了}\end{exemple}
\begin{sous-entrée}{sɤnɯtɕetha}{ⓔnɯtɕethaⓝsɤnɯtɕetha} 
\classe{vi}  
\grammaire{apass} 
\begin{définition}\pfra{sonder les gens}\end{définition}
\begin{définition}\pcmn{试探人}\end{définition}
\begin{exemple}\pjya{kɤ-sɤnɯtɕetha mɯ́j-tɯ-spe wo!}\hspace{5pt}\pcmn{你不会试探别人}\end{exemple}\relationsémantique{参考}{\lien{ⓔtɕetha}{tɕetha}}\end{sous-entrée}

\end{entrée}

\begin{entrée}{nɯtɕɣom}{}{ⓔnɯtɕɣom} 
\classe{vi}  
\grammaire{denom} \paradigme{dir}{pɯ-}
\begin{définition}\pfra{ramasser du xanthoxyle}\end{définition}
\begin{définition}\pcmn{摘花椒}\end{définition}
\begin{exemple}\pjya{ɕ-pɯ-tɕɣom}\hspace{5pt}\pcmn{我去摘花椒了}\end{exemple}\relationsémantique{参考}{\lien{ⓔtɕɣom}{tɕɣom}}\end{entrée}

\begin{entrée}{nɯtɕhaʁ}{}{ⓔnɯtɕhaʁ} 
\classe{vi} \paradigme{dir}{tɤ-}
\begin{définition}\pfra{manger du fourrage (cheval)}\end{définition}
\begin{définition}\pcmn{吃饲料(马)}\end{définition}
\begin{exemple}\pjya{mbro ɲɯ-nɯtɕhaʁ}\hspace{5pt}\pcmn{马在吃饲料}\end{exemple}\relationsémantique{参考}{\lien{ⓔɯ-tɕhaʁⓗ1}{ɯ-tɕhaʁ₁}}\end{entrée}

\begin{entrée}{nɯtɕhɤjlɯz}{}{ⓔnɯtɕhɤjlɯz} 
\classe{vi} 
\begin{définition}\pfra{observer une coutume}\end{définition}
\begin{définition}\pcmn{遵循某种风俗}\end{définition}
\begin{exemple}\pjya{kɯki kɯ-fse aʑo kɤ-nɯtɕhɤjlɯz mɤ-cha-a}\hspace{5pt}\pcmn{我不能接受这种风俗}\end{exemple}\relationsémantique{参考}{\lien{ⓔtɕhɤjlɯz}{tɕhɤjlɯz}}\end{entrée}

\begin{entrée}{nɯtɕhɤjʁo}{}{ⓔnɯtɕhɤjʁo} 
\classe{vs} 
\begin{définition}\pfra{être en état d'ébriété}\end{définition}
\begin{définition}\pcmn{发酒疯}\end{définition}
\begin{exemple}\pjya{jɯfɕɯr lɤ-βzi tɕe tɤ-nɯtɕhɤjʁo}\hspace{5pt}\pcmn{他昨天喝醉了就发了酒疯}\end{exemple}\relationsémantique{参考}{\lien{ⓔtɕhɤjʁo}{tɕhɤjʁo}}\end{entrée}

\begin{entrée}{nɯtɕhɤl}{}{ⓔnɯtɕhɤl} 
\classe{vi} \paradigme{dir}{pɯ-}
\begin{définition}\pfra{être puni}\end{définition}
\begin{définition}\pcmn{受到惩罚}\end{définition}
\begin{sous-entrée}{znɯtɕhɤl}{ⓔnɯtɕhɤlⓝznɯtɕhɤl} 
\classe{vt} \paradigme{dir}{pɯ-}
\begin{définition}\pfra{punir}\end{définition}
\begin{définition}\pcmn{惩罚}\end{définition}
\begin{exemple}\pjya{pɯ́-wɣ-znɯtɕhal-a tɕe nɯ-kho-t-a}\hspace{5pt}\pcmn{我被罚就交了罚款}\end{exemple}
\begin{exemple}\pjya{ɯʑo ɕ-to-mɯrkɯ tɕe, pɯ-znɯtɕhal-a}\hspace{5pt}\pcmn{他偷了东西我就惩罚了他}\end{exemple}\relationsémantique{同义词}{\lien{ⓔznɯtɕhɤtpa}{znɯtɕhɤtpa}}\relationsémantique{参考}{\lien{ⓔɯ-tɕhɤl}{ɯ-tɕhɤl}}\end{sous-entrée}

\étymologie{tɕʰad}\end{entrée}

\begin{entrée}{nɯtɕhomba}{}{ⓔnɯtɕhomba} 
\classe{vi} \paradigme{dir}{tɤ-}
\begin{définition}\pfra{attraper un rhume}\end{définition}
\begin{définition}\pcmn{感冒}\end{définition}
\begin{exemple}\pjya{adianhua jɤ-tɯ-lɤt ri, pɯ-nɯtɕhomba-a}\hspace{5pt}\pcmn{你给我打电话的时候,我在感冒}\end{exemple}\relationsémantique{参考}{\lien{ⓔtɕhomba}{tɕhomba}}\end{entrée}

\begin{entrée}{nɯtɕhʁɯβ}{}{ⓔnɯtɕhʁɯβ}\relationsémantique{参考}{\lien{ⓔtɕhʁɯβnɤtɕhʁɯβ}{tɕhʁɯβnɤtɕhʁɯβ}}\end{entrée}

\begin{entrée}{nɯtɕhɯrɟɯɣ}{}{ⓔnɯtɕhɯrɟɯɣ} 
\classe{vi} 
\begin{définition}\pfra{où l'eau coule vite}\end{définition}
\begin{définition}\pcmn{水流得很急(地方)}\end{définition}
\begin{exemple}\pjya{ɯ-sta a-pɯ-ɣɤʑɯn tɕe nɯtɕhɯrɟɯɣ}\hspace{5pt}\pcmn{如果有斜坡的话水流得很急}\end{exemple}\end{entrée}

\begin{entrée}{nɯtɕhɯtɕɯnpaχɕi}{}{ⓔnɯtɕhɯtɕɯnpaχɕi} 
\classe{vi} \paradigme{dir}{nɯ-}
\begin{définition}\pfra{aller cueillir des poires}\end{définition}
\begin{définition}\pcmn{采梨子}\end{définition}\end{entrée}

\begin{entrée}{nɯtɕhɯtɯɣ}{}{ⓔnɯtɕhɯtɯɣ} 
\classe{vi}  
\grammaire{denom} \paradigme{dir}{pɯ-}
\begin{définition}\pfra{s'empoisonner en buvant de l'eau (bovidé)}\end{définition}
\begin{définition}\pcmn{水中毒(牛)}\end{définition}
\begin{exemple}\pjya{jla pjɤ-nɯtɕhɯtɯɣ}\hspace{5pt}\pcmn{犏牛喝水中毒了}\end{exemple}
\begin{exemple}\pjya{qambrɯ pjɤ-nɯtɕhɯtɯɣ}\hspace{5pt}\pcmn{牦牛喝水中毒了}\end{exemple}\relationsémantique{参考}{\lien{ⓔnɯrtsɤtɯɣ}{nɯrtsɤtɯɣ}}\étymologie{tɕʰu.dug}\end{entrée}

\begin{entrée}{nɯtɕhɯwɯt}{}{ⓔnɯtɕhɯwɯt} 
\classe{vt}  
\grammaire{denom} \paradigme{dir}{nɯ-}
\begin{définition}\pfra{faire bouillir la peau afin de pouvoir enlever les poils}\end{définition}
\begin{définition}\pcmn{烫;用开水将毛褪掉}\end{définition}
\begin{exemple}\pjya{paʁ ɲɤ-nɯtɕhɯwɯt}\hspace{5pt}\pcmn{他烫了猪皮}\end{exemple}\end{entrée}

\begin{entrée}{nɯtɕʁɯβ}{}{ⓔnɯtɕʁɯβ}\relationsémantique{参考}{\lien{ⓔtɕʁɯβnɤtɕʁɯβ}{tɕʁɯβnɤtɕʁɯβ}}\end{entrée}

\begin{entrée}{nɯthaj}{}{ⓔnɯthaj} 
\classe{vt} \paradigme{dir}{tɤ-}
\begin{définition}\pfra{soulever (à plusieurs)}\end{définition}
\begin{définition}\pcmn{抬(几个人一起)}\end{définition}
\begin{exemple}\pjya{ɕoŋtɕa tɤ-nɯthaj-tɕi}\hspace{5pt}\pcmn{我们俩把木料抬起来了}\end{exemple}\étymologie{fn:抬}\end{entrée}

\begin{entrée}{nɯthɤstɯɣ}{}{ⓔnɯthɤstɯɣ} 
\classe{vi}  
\grammaire{denom} \paradigme{dir}{tɤ-}
\begin{définition}\pfra{jouer à un jeu de hasard (où il faut deviner combien d'objets son adversaire tient dans la main)}\end{définition}
\begin{définition}\pcmn{赌胡豆的游戏}\end{définition}
\begin{exemple}\pjya{stoʁ tɤ-nɯthɤstɯɣ-tɕi}\hspace{5pt}\pcmn{我们俩赌了胡豆}\end{exemple}
\begin{exemple}\pjya{khɯtsa ɯ-ŋgɯ stoʁ tɤ-rku-tɕi tɕe tɤ-nɯthɤstɯɣ-tɕi, a-pɯ-tɕhaʁ tɕe nɤʑo ɲɯ-kɯ-ɣɤjɯ-a, ɯ-tsa ɲɯ-βze nɤ nɤʑo jɤ-nɯtsɯm}\hspace{5pt}\pcmn{我们俩在碗里装了胡豆就赌了有多少,要是猜少了的话就给我赔,要是猜对了的话你就把它带走}\end{exemple}\relationsémantique{参考}{\lien{ⓔthɤstɯɣ}{thɤstɯɣ}}\end{entrée}

\begin{entrée}{nɯthɣe}{}{ⓔnɯthɣe} 
\classe{vi}  
\grammaire{denom} \paradigme{dir}{\_}
\begin{définition}\pfra{ramasser des glands}\end{définition}
\begin{définition}\pcmn{捡青冈籽}\end{définition}\relationsémantique{参考}{\lien{ⓔthɣe}{thɣe}}\end{entrée}

\begin{entrée}{nɯthɯ}{}{ⓔnɯthɯ} 
\classe{vt} \paradigme{dir}{kɤ-}
\begin{définition}\pfra{utiliser comme une casserole}\end{définition}
\begin{définition}\pcmn{用锅子;当锅子用}\end{définition}
\begin{exemple}\pjya{nɤʑo tʂha kɤ-tɯ-ta-t tɕe ko-tɯ-nɯthɯ-t}\hspace{5pt}\pcmn{你煮茶的时候用了锅子}\end{exemple}\relationsémantique{参考}{\lien{}{tɯthɯ}}\end{entrée}

\begin{entrée}{nɯtsa}{}{ⓔnɯtsa} 
\classe{vs} \paradigme{dir}{tɤ-}
\begin{définition}\pfra{convenir}\end{définition}
\begin{définition}\pcmn{适合}\end{définition}
\begin{exemple}\pjya{ki tɤ-rte ki nɤʑɯɣ ɲɯ-nɯtsa}\hspace{5pt}\pcmn{你适合戴这种帽子}\end{exemple}
\begin{exemple}\pjya{ki tɤ-rte ki ɲɯ-tɯ-nɯtsa}\hspace{5pt}\pcmn{你适合戴这种帽子}\end{exemple}
\begin{exemple}\pjya{tu-tɯ-nɯtsa ʑo tɕe phɣo-a ɕti}\hspace{5pt}\pcmn{一到合适的时间,我就会逃跑}\end{exemple}\relationsémantique{参考}{\lien{ⓔɯ-tsa}{ɯ-tsa}}\relationsémantique{参考}{\lien{ⓔnɤtsa}{nɤtsa}}\end{entrée}

\begin{entrée}{nɯtshɤβ}{}{ⓔnɯtshɤβ} 
\classe{vt} \sens{1}\paradigme{dir}{kɤ-}
\begin{définition}\pfra{confronter ensemble}\end{définition}
\begin{définition}\pcmn{一起对付一个人}\end{définition}
\begin{exemple}\pjya{ʑara kɯ ɯʑo kɤ-ʁndɯ ko-nɯtshɤβ-nɯ}\hspace{5pt}\pcmn{他们一起打了他}\end{exemple}
\begin{exemple}\pjya{ʑara kɯ ɯʑo ko-nɯtshɤβ-nɯ ʑo to-nɤmqe-nɯ}\hspace{5pt}\pcmn{他们一起骂了他}\end{exemple}\sens{2}\paradigme{dir}{nɯ-}
\begin{définition}\pfra{faire ensemble}\end{définition}
\begin{définition}\pcmn{共同做一件事情}\end{définition}
\begin{exemple}\pjya{tɕiʑo kɯki laχtɕha kɤ-χtɯ tɤ-nɯtshɤβ-tɕi ŋu}\hspace{5pt}\pcmn{我们一起买了这个东西}\end{exemple}
\begin{exemple}\pjya{ki tɯ-khɯtsa kɤndza tɤ-nɯtshɤβ-tɕi}\hspace{5pt}\pcmn{我们一起吃了这一碗}\end{exemple}\end{entrée}

\begin{entrée}{nɯtshɤdɯɣ}{}{ⓔnɯtshɤdɯɣ} 
\classe{vi} \paradigme{dir}{nɯ-}
\begin{définition}\pfra{souffrir de la chaleur}\end{définition}
\begin{définition}\pcmn{受热,中暑}\end{définition}
\begin{exemple}\pjya{jisŋi tɤŋe ɲɯ-sɤɕke tɕe ɲɯ-nɯtshɤdɯɣ-a}\hspace{5pt}\pcmn{今天太阳很大,我觉得很难受}\end{exemple}\relationsémantique{参考}{\lien{ⓔtshɤdɯɣ}{tshɤdɯɣ}}\relationsémantique{参考}{\lien{ⓔɣɯtshɤdɯɣ}{ɣɯtshɤdɯɣ}}\end{entrée}

\begin{entrée}{nɯtshɤtʂot}{}{ⓔnɯtshɤtʂot} 
\classe{vi}  
\grammaire{denom} \paradigme{dir}{nɯ-}\paradigme{dir}{tɤ-}
\begin{définition}\pfra{avoir la fièvre}\end{définition}
\begin{définition}\pcmn{发烧}\end{définition}
\begin{exemple}\pjya{ɲɯ-nɯtɕhomba tɕe ɲɯ-nɯtshɤtʂot}\hspace{5pt}\pcmn{他感冒就发烧}\end{exemple}
\begin{exemple}\pjya{nɯ-nɯtshɤtʂot-a}\hspace{5pt}\pcmn{我发烧了}\end{exemple}\relationsémantique{参考}{\lien{ⓔtshɤtʂot}{tshɤtʂot}}\étymologie{tsʰa.drod}\end{entrée}

\begin{entrée}{nɯtsɯm}{}{ⓔnɯtsɯm}\relationsémantique{参考}{\lien{ⓔtsɯm}{tsɯm}}\end{entrée}

\begin{entrée}{nɯtsɯʁot}{}{ⓔnɯtsɯʁot} 
\classe{vi} \paradigme{dir}{pɯ-}
\begin{définition}\pfra{chasser le faisan}\end{définition}
\begin{définition}\pcmn{打野鸡}\end{définition}
\begin{exemple}\pjya{ɯʑo ɕ-pɯ-nɯtsɯʁot ri, mɯ-pjɤ-cha}\hspace{5pt}\pcmn{他去打野鸡,但是没有成功}\end{exemple}\relationsémantique{参考}{\lien{ⓔtsɯʁot}{tsɯʁot}}\end{entrée}

\begin{entrée}{nɯtʂu}{₁}{ⓔnɯtʂuⓗ1} 
\classe{vi} \paradigme{dir}{tɤ-}
\begin{définition}\pfra{bien se passer}\end{définition}
\begin{définition}\pcmn{顺利}\end{définition}
\begin{exemple}\pjya{jɤxtshi tɕi-tɯtsɣe pɯ-nɯtʂu}\hspace{5pt}\pcmn{最近我们俩的生意很好}\end{exemple}\relationsémantique{参考}{\lien{ⓔtʂu}{tʂu}}\end{entrée}

\begin{entrée}{nɯtʂu}{₂}{ⓔnɯtʂuⓗ2} 
\classe{vt} \paradigme{dir}{kɤ-}
\begin{définition}\pfra{prendre en passant}\end{définition}
\begin{définition}\pcmn{(一路上看见什么东西就)顺便拿走}\end{définition}
\begin{exemple}\pjya{sɯmat ɣɤʑu tɕe kɤ-nɯtʂu-t-a}\hspace{5pt}\pcmn{(路上)有水果,我顺便拿了}\end{exemple}
\begin{exemple}\pjya{paʁndza tʂɯtʂu kɤ-nɯtʂu-t-a tɕe nɯ-phɯt-a}\hspace{5pt}\pcmn{我在路上看到猪草,顺便割了一把}\end{exemple}\relationsémantique{参考}{\lien{ⓔtʂu}{tʂu}}\end{entrée}

\begin{entrée}{nɯtʂawku}{}{ⓔnɯtʂawku} 
\classe{vt} \paradigme{dir}{tɤ-}
\begin{définition}\pfra{prendre soin}\end{définition}
\begin{définition}\pcmn{照顾}\end{définition}
\begin{sous-entrée}{ʑɣɤnɯtʂawku}{ⓔnɯtʂawkuⓝʑɣɤnɯtʂawku} 
\grammaire{refl} 
\begin{définition}\pfra{prendre soin de soi}\end{définition}
\begin{définition}\pcmn{照顾自己}\end{définition}
\begin{exemple}\pjya{nɤʑo tu-tɯ-ʑɣɤnɯtʂawku mɯ́j-tɯ-cha ɕti}\hspace{5pt}\pcmn{你都不会照顾自己}\end{exemple}\end{sous-entrée}

\étymologie{fn:照顾}\end{entrée}

\begin{entrée}{nɯtʂɤqɤsti}{}{ⓔnɯtʂɤqɤsti} 
\classe{vt}  
\grammaire{incorp} \paradigme{dir}{tɤ-}
\begin{définition}\pfra{bloquer le chemin}\end{définition}
\begin{définition}\pcmn{挡路(走在前面的人,挡后面行人的去路)}\end{définition}
\begin{exemple}\pjya{ma-tɤ-kɯ-nɯtʂɤqɤsti-a}\hspace{5pt}\pcmn{你不要挡我的路}\end{exemple}\relationsémantique{参考}{\lien{ⓔstiⓗ1}{sti₁}}\relationsémantique{参考}{\lien{ⓔtʂu}{tʂu}}\end{entrée}

\begin{entrée}{nɯtʂha}{}{ⓔnɯtʂha} 
\classe{vi}  
\grammaire{denom} \paradigme{dir}{kɤ-}
\begin{définition}\pfra{prendre le petit déjeuner}\end{définition}
\begin{définition}\pcmn{吃早饭}\end{définition}
\begin{exemple}\pjya{kɤ-nɯtʂha-ndʑi}\hspace{5pt}\pcmn{他们俩吃了早餐}\end{exemple}\relationsémantique{参考}{\lien{ⓔtʂha}{tʂha}}\end{entrée}

\begin{entrée}{nɯtʂhɤɣndʑɤr}{}{ⓔnɯtʂhɤɣndʑɤr} 
\classe{vi}  
\grammaire{denom} \paradigme{dir}{tɤ-}
\begin{définition}\pfra{manger (serviteur)}\end{définition}
\begin{définition}\pcmn{吃糌粑}\end{définition}
\begin{exemple}\pjya{soz to-nɯtʂhɤɣndʑɤr-ndʑi}\hspace{5pt}\pcmn{他们俩早上吃了}\end{exemple}\relationsémantique{参考}{\lien{ⓔtɯ-ɣndʑɤr}{tɯ-ɣndʑɤr}}\end{entrée}

\begin{entrée}{nɯtʂhɤlu}{}{ⓔnɯtʂhɤlu} 
\classe{vt}  
\grammaire{denom} 
\begin{définition}\pfra{verser du lait dans le thé}\end{définition}
\begin{définition}\pcmn{把牛奶倒到马茶里}\end{définition}
\begin{exemple}\pjya{alo ji-tɤ-lu tha-ɣɯt-nɯ tɤ-nɯtʂhɤlu-j / (tʂhɤlu tɤ-nɯlɤt-i)}\hspace{5pt}\pcmn{他们把牛奶带下来了,我们就倒在茶里喝了}\end{exemple}\end{entrée}

\begin{entrée}{nɯtɯcizʁe}{}{ⓔnɯtɯcizʁe} 
\classe{vi}  
\grammaire{incorp} 
\begin{définition}\pfra{transporter de l'eau}\end{définition}
\begin{définition}\pcmn{背水}\end{définition}
\begin{exemple}\pjya{ɕ-pɯ-nɯtɯcizʁe-a}\hspace{5pt}\pcmn{我去背水了}\end{exemple}\relationsémantique{参考}{\lien{ⓔtɯ-ci}{tɯ-ci}}\relationsémantique{参考}{\lien{ⓔtɯcizʁe}{tɯcizʁe}}\relationsémantique{参考}{\lien{ⓔnɯzʁe}{nɯzʁe}}\end{entrée}

\begin{entrée}{nɯtɯfɕɤl}{}{ⓔnɯtɯfɕɤl} 
\classe{vi}  
\grammaire{denom} \paradigme{dir}{nɯ-}\paradigme{dir}{nɯ-}
\begin{définition}\pfra{avoir la diarrhée}\end{définition}
\begin{définition}\pcmn{拉肚子}\end{définition}
\begin{définition}\pfra{causer la diarrhée}\end{définition}
\begin{définition}\pcmn{令人拉肚子}\end{définition}
\begin{exemple}\pjya{nɯ-nɯtɯfɕal-a}\hspace{5pt}\pcmn{我拉了肚子}\end{exemple}
\begin{exemple}\pjya{a-xtu ɲɯ-mŋɤm tɕe nɯ-nɯtɯfɕal-a}\hspace{5pt}\pcmn{我肚子疼,拉了肚子}\end{exemple}\relationsémantique{参考}{\lien{ⓔfɕɤl}{fɕɤl}}\relationsémantique{同义词}{\lien{ⓔnɯqhoχɕɤr}{nɯqhoχɕɤr}}
\begin{sous-entrée}{znɯtɯfɕɤl}{ⓔnɯtɯfɕɤlⓝznɯtɯfɕɤl} 
\classe{vt} \end{sous-entrée}

\end{entrée}

\begin{entrée}{nɯtɯrgi}{}{ⓔnɯtɯrgi} 
\classe{vi} \paradigme{dir}{pɯ-}
\begin{définition}\pfra{couper et ramasser des branches de sapin pour faire des fumigations}\end{définition}
\begin{définition}\pcmn{把杉树枝桠砍回来}\end{définition}\end{entrée}

\begin{entrée}{nɯtɯrgilaŋlaŋ}{}{ⓔnɯtɯrgilaŋlaŋ} 
\classe{vi}  
\grammaire{denom} \paradigme{dir}{\_}
\begin{définition}\pfra{ramasser des cônes de pin}\end{définition}
\begin{définition}\pcmn{捡杉木果}\end{définition}
\begin{exemple}\pjya{ɕ-pɯ-nɯtɯrgilaŋlaŋ-a}\hspace{5pt}\pcmn{我去捡杉木果了}\end{exemple}\relationsémantique{参考}{\lien{ⓔtɯrgilaŋlaŋ}{tɯrgilaŋlaŋ}}\end{entrée}

\begin{entrée}{nɯtɯtɕhɯ}{}{ⓔnɯtɯtɕhɯ} 
\classe{vt} \paradigme{dir}{tɤ-}
\begin{définition}\pfra{poignarder}\end{définition}
\begin{définition}\pcmn{刺杀}\end{définition}
\begin{exemple}\pjya{tshjencɯ kɯ to-znɯtɯtɕhɯ tɕe pjɤ-sat}\hspace{5pt}\pcmn{用短刀把他刺杀了}\end{exemple}\relationsémantique{参考}{\lien{ⓔtɕhɯ}{tɕhɯ}}
\begin{sous-entrée}{sɤnɯtɯtɕhɯ}{ⓔnɯtɯtɕhɯⓝsɤnɯtɯtɕhɯ} 
\classe{vi}  
\grammaire{antipass} \end{sous-entrée}

\end{entrée}

\begin{entrée}{nɯtɯtso}{}{ⓔnɯtɯtso} 
\classe{vt} \paradigme{dir}{pɯ-}
\begin{définition}\pfra{avoir de l'expérience}\end{définition}
\begin{définition}\pcmn{懂事,有经验,得到教训}\end{définition}
\begin{exemple}\pjya{pjɯ-kɯ-nɯtɯtso ra ma nɯ ɯ-qhu tɕe kɯmaʁ kɤ-nɤma tɤ-ra tɕe kɤ-sɤpe kɯ-cha (=tɯ-kɯ-tso pjɯ-tu ra)}\hspace{5pt}\pcmn{要经过经验和教训才能把以后的事情做好}\end{exemple}\relationsémantique{参考}{\lien{ⓔtso}{tso}}\end{entrée}

\begin{entrée}{nɯtɯtʂaŋ}{}{ⓔnɯtɯtʂaŋ}\relationsémantique{参考}{\lien{ⓔtʂaŋ}{tʂaŋ}}\end{entrée}

\begin{entrée}{nɯxso}{}{ⓔnɯxso} 
\classe{vs} \paradigme{dir}{tɤ-}
\begin{définition}\pfra{être vide}\end{définition}
\begin{définition}\pcmn{空的}\end{définition}
\begin{exemple}\pjya{khɯtsa to-nɯxso tɕe ɯ-ŋgɯ kɤ-rku ɲɤ-me}\hspace{5pt}\pcmn{碗空了}\end{exemple}\relationsémantique{参考}{\lien{ⓔso}{so}}\end{entrée}

\begin{entrée}{nɯxsɯ}{}{ⓔnɯxsɯ} 
\classe{vi} \sens{1}\paradigme{dir}{lɤ-}\paradigme{dir}{thɯ-}
\begin{définition}\pfra{regarder en cachette}\end{définition}
\begin{définition}\pcmn{偷看}\end{définition}
\begin{exemple}\pjya{ku-nɯxsɯ ɲɯ-ŋu}\hspace{5pt}\pcmn{他在偷看}\end{exemple}
\begin{exemple}\pjya{tɤ-pɤtso ɲɯ-nɯxsɯ}\hspace{5pt}\pcmn{小孩子在偷看}\end{exemple}
\begin{exemple}\pjya{nɯ tɯrme ɲɯ-nɯxsɯ}\hspace{5pt}\pcmn{那个人在偷看}\end{exemple}
\begin{exemple}\pjya{ma-lɤ-tɯ-nɯxsɯ tɕe lɤ-ɣi}\hspace{5pt}\pcmn{你不要偷看,你进来吧}\end{exemple}\sens{2}\paradigme{dir}{lɤ-}
\begin{définition}\pfra{laisser apparaître un petit bout}\end{définition}
\begin{définition}\pcmn{露出了一部分}\end{définition}
\begin{exemple}\pjya{nɤ-laχtɕha ɲɯ-nɯxsɯ tɕe tɯ-nɯ-βde ma}\hspace{5pt}\pcmn{你的东西露出来,消息不要丢掉}\end{exemple}\end{entrée}

\begin{entrée}{nɯxtɕhɤz}{}{ⓔnɯxtɕhɤz} 
\classe{vi} \paradigme{dir}{tɤ-}
\begin{définition}\pfra{avoir un tel tempérament, une telle habitude, une telle propension}\end{définition}
\begin{définition}\pcmn{有这样的性格,有这样的习惯}\end{définition}
\begin{exemple}\pjya{ɯʑo ɲɯ-nɯxtɕhɤz}\hspace{5pt}\pcmn{他有那个习惯}\end{exemple}
\begin{exemple}\pjya{nɤʑo kɯnɤ tɯ-nɯxtɕhɤz}\hspace{5pt}\pcmn{你也有那个习惯}\end{exemple}
\begin{exemple}\pjya{aʑo ɲɯ-nɯxtɕhaz-a}\hspace{5pt}\pcmn{我有这个习惯}\end{exemple}
\begin{exemple}\pjya{nɤki nɤ-kɤ-fse nɯnɯ nɤʑo tɯ-nɯxtɕhɤz ɕti}\hspace{5pt}\pcmn{你有这样的习惯}\end{exemple}
\begin{exemple}\pjya{nɯ kɯ-fse kɤ-ti tɯ-nɯxtɕhɤz ɕti}\hspace{5pt}\pcmn{你习惯这样说}\end{exemple}
\begin{exemple}\pjya{kɤ-ŋɤn tɯ-nɯxtɕhɤz ɕti}\hspace{5pt}\pcmn{你习惯做坏事}\end{exemple}
\begin{exemple}\pjya{nɯ kɯ-fse kɤ-βzu nɯxtɕhɤz}\hspace{5pt}\pcmn{他习惯那样做}\end{exemple}
\begin{exemple}\pjya{tɤresɤpɯpa kɤ-βzu nɯxtɕhɤz}\hspace{5pt}\pcmn{他习惯取笑别人}\end{exemple}
\begin{exemple}\pjya{khramba kɤ-βzu mɤ-nɯxtɕhɤz}\hspace{5pt}\pcmn{他不习惯说谎}\end{exemple}\end{entrée}

\begin{entrée}{nɯxtshi}{}{ⓔnɯxtshi} 
\classe{adv} 
\begin{définition}\pfra{cette fois-là}\end{définition}
\begin{définition}\pcmn{那一次}\end{définition}
\begin{exemple}\pjya{rgɯnba tɤ-ari tɕe, nɯ́xtshi nɯ tɯrme wuma pɯ-dɤn}\hspace{5pt}\pcmn{我们去寺庙大那一次,人很多}\end{exemple}\end{entrée}

\begin{entrée}{nɯχɤnloʁ}{}{ⓔnɯχɤnloʁ} 
\classe{vi} \paradigme{dir}{pɯ-}
\begin{définition}\pfra{être peu réactif (à cause de l'âge)}\end{définition}
\begin{définition}\pcmn{思维迟钝(因为年老)}\end{définition}
\begin{exemple}\pjya{kɤ-nɯχɤnloʁnɯ nɯ mɯ́j-pe}\hspace{5pt}\pcmn{上了年纪思维迟钝很不好}\end{exemple}\end{entrée}

\begin{entrée}{nɯχpɯn}{}{ⓔnɯχpɯn} 
\classe{vi}  
\grammaire{denom} \paradigme{dir}{lɤ-}
\begin{définition}\pfra{devenir moine}\end{définition}
\begin{définition}\pcmn{当和尚}\end{définition}\relationsémantique{参考}{\lien{ⓔχpɯn}{χpɯn}}\relationsémantique{参考}{\lien{ⓔrɤχpɯn}{rɤχpɯn}}\end{entrée}

\begin{entrée}{nɯχpɯnbu}{}{ⓔnɯχpɯnbu} 
\classe{vi} \paradigme{dir}{lɤ-}
\begin{définition}\pfra{avoir le pouvoir}\end{définition}
\begin{définition}\pcmn{掌权}\end{définition}
\begin{exemple}\pjya{lɤ-nɯχpɯnbu (=χpɯnbu la-ndo)}\hspace{5pt}\pcmn{他掌权了}\end{exemple}\relationsémantique{参考}{\lien{ⓔχpɯnbu}{χpɯnbu}}\end{entrée}

\begin{entrée}{nɯχsɯmtoʁ}{}{ⓔnɯχsɯmtoʁ} 
\classe{vi} 
\begin{définition}\pfra{vivre}\end{définition}
\begin{définition}\pcmn{活;生存}\end{définition}
\begin{exemple}\pjya{jiɕqha nɯ ɲɯ-nɯχsɯmtoʁ}\hspace{5pt}\pcmn{那个还活着}\end{exemple}
\begin{exemple}\pjya{kɯtɕu kɤ-nɯχsɯmtoʁ ɲɯ-ɴqa ɕti}\hspace{5pt}\pcmn{这里生存很辛苦}\end{exemple}
\begin{exemple}\pjya{ɯ-ɲɯ-nɯχsɯmtoʁ}\hspace{5pt}\pcmn{他还活着吗}\end{exemple}\end{entrée}

\begin{entrée}{nɯχtɕɤn}{}{ⓔnɯχtɕɤn} 
\classe{vs} \paradigme{dir}{kɤ-}
\begin{définition}\pfra{terrible}\end{définition}
\begin{définition}\pcmn{恐怖;凶猛}\end{définition}
\begin{exemple}\pjya{ndzaʁlaŋ tɯrme jo-ɣi tɕe, tɯ-ci kɤ-kɯ-nɯχtɕɤn ɯ-ŋgɯ ɕ-pjɯ́-wɣ-βde ɲɯ-ra}\hspace{5pt}\pcmn{(妖界)来了凡人,我们要把他扔进恐怖的水里!}\end{exemple}
\begin{exemple}\pjya{tɯ-ci kɤ-kɯ-nɯχtɕɤn nɯnɯ tɕe tɕe tu-ola ʑo kɯ-fse pjɤ-ɕti tɕe, nɯ pjɯ-tɯ-βde nɯ tɯrme pjɯ-kɯ-si pjɤ-ŋgrɤl}\hspace{5pt}\pcmn{“恐怖的水”在沸腾一样,(妖)把人一扔进去就必死无疑}\end{exemple}\end{entrée}

\begin{entrée}{nɯχtɕɯrɯ}{}{ⓔnɯχtɕɯrɯ} 
\classe{vi} \paradigme{dir}{nɯ-}
\begin{définition}\pfra{se déshabiller complètement}\end{définition}
\begin{définition}\pcmn{把衣服脱光}\end{définition}
\begin{sous-entrée}{znɯχtɕɯrɯ}{ⓔnɯχtɕɯrɯⓝznɯχtɕɯrɯ} 
\classe{vt} 
\begin{définition}\pfra{déshabiller complètement}\end{définition}
\begin{définition}\pcmn{把……的衣服脱光}\end{définition}
\begin{exemple}\pjya{tɤ-pɤtso nɯ-znɯχtɕɯrɯ-t-a}\hspace{5pt}\pcmn{我把小孩子的衣服脱光了}\end{exemple}\relationsémantique{参考}{\lien{ⓔχtɕɯrɯ}{χtɕɯrɯ}}\end{sous-entrée}

\end{entrée}

\begin{entrée}{nɯχtɯntʂu}{}{ⓔnɯχtɯntʂu} 
\classe{vi} \paradigme{dir}{tɤ-}
\begin{définition}\pfra{convivial, sociable}\end{définition}
\begin{définition}\pcmn{合得来;合群}\end{définition}
\begin{exemple}\pjya{jiɕqha nɯ ɲɯ-nɯχtɯntʂu}\hspace{5pt}\pcmn{他跟别人合得来}\end{exemple}
\begin{exemple}\pjya{ɯ-zda ra nɯ-rca ɲɯ-nɯχtɯntʂu}\hspace{5pt}\pcmn{他跟他的伙伴很合得来}\end{exemple}
\begin{exemple}\pjya{jiɕqha nɯ mɯ́j-nɯχtɯntʂu}\hspace{5pt}\pcmn{他很孤僻}\end{exemple}
\begin{exemple}\pjya{tu-nɯχtɯntʂu-a}\hspace{5pt}\pcmn{我跟别人合得来}\end{exemple}\end{entrée}

\begin{entrée}{nɯzarzɯr}{}{ⓔnɯzarzɯr} 
\classe{vi} \paradigme{dir}{pɯ-}
\begin{définition}\pfra{avoir la tête qui tourne (ovins, tremblante du mouton?)}\end{définition}
\begin{définition}\pcmn{头晕倒下(羊)}\end{définition}
\begin{exemple}\pjya{qaʑo pjɤ-nɯzarzɯr}\hspace{5pt}\pcmn{绵羊头晕倒下了}\end{exemple}
\begin{exemple}\pjya{tshɤt kɤ-nɯzarzɯr nɯxtɕhɤz}\hspace{5pt}\pcmn{山羊经常头晕倒下}\end{exemple}
\begin{exemple}\pjya{qaʑo cho tshɤt ɯ-kɤrnoʁ ɲɯ-mtɕɯr tɕe pjɯ-ndʐaβ tɕe nɯ pjɯ-nɯzarzɯr ŋu tɕe ɯ-rna ɯ-taʁ pjɯ́-wɣ-qraʁ tɕe tɤ-se a-pɯ-ɬoʁ tɕe tu-mna ŋgrɤl}\hspace{5pt}\pcmn{绵羊和山羊头晕突然倒下,在羊的耳朵上割一刀,让血流出来就会好。}\end{exemple}\end{entrée}

\begin{entrée}{nɯzaχtɤt}{}{ⓔnɯzaχtɤt} 
\classe{vi} \paradigme{dir}{thɯ-}
\begin{définition}\pfra{manger de la nourriture pour les morts}\end{définition}
\begin{définition}\pcmn{吃死人的食物(骂人的话)}\end{définition}\relationsémantique{参考}{\lien{ⓔzaχtɤt}{zaχtɤt}}\relationsémantique{同义词}{\lien{ⓔnɯzɤmpo}{nɯzɤmpo}}\end{entrée}

\begin{entrée}{nɯzɤmpo}{}{ⓔnɯzɤmpo} 
\classe{vi}  
\grammaire{denom} \paradigme{dir}{thɯ-}
\begin{définition}\pfra{manger de la nourriture pour les morts}\end{définition}
\begin{définition}\pcmn{吃死人的食物(骂人的话)}\end{définition}\relationsémantique{参考}{\lien{ⓔnɯzaχtɤt}{nɯzaχtɤt}}\relationsémantique{参考}{\lien{ⓔzɤmpo}{zɤmpo}}\end{entrée}

\begin{entrée}{nɯzɤsna}{}{ⓔnɯzɤsna} 
\classe{vl}  
\grammaire{denom} \paradigme{dir}{thɯ-}
\begin{définition}\pfra{manger la nourriture pour les morts}\end{définition}
\begin{définition}\pcmn{吃死人的食物}\end{définition}
\begin{exemple}\pjya{thɯ-nɯzɤsne}\hspace{5pt}\pcmn{你去死吧!}\end{exemple}\relationsémantique{参考}{\lien{ⓔzɤsna}{zɤsna}}\end{entrée}

\begin{entrée}{nɯzdɯɣ}{}{ⓔnɯzdɯɣ} 
\classe{vt}  
\grammaire{caus} \paradigme{dir}{thɯ-}
\begin{définition}\pfra{s'inquiéter pour quelqu'un}\end{définition}
\begin{définition}\pcmn{为别人担心}\end{définition}\paradigme{dir}{nɯ-}
\begin{exemple}\pjya{a-mu ɲɯ-nɯzdɯɣ-a}\hspace{5pt}\pcmn{我担心我的母亲}\end{exemple}
\begin{exemple}\pjya{a-rɟit ɲɯ-nɯzdɯɣ-a}\hspace{5pt}\pcmn{我担心我的儿子}\end{exemple}
\begin{exemple}\pjya{nɤj nɤ-rʑaβ ɲɯ-tɯ-nɯzdɯɣ}\hspace{5pt}\pcmn{你担心你的妻子}\end{exemple}
\begin{exemple}\pjya{ɲɯ-ta-nɯzdɯɣ}\hspace{5pt}\pcmn{我为你担心}\end{exemple}
\begin{sous-entrée}{znɯzdɯɣ}{ⓔnɯzdɯɣⓝznɯzdɯɣ} 
\classe{vt} \end{sous-entrée}

\begin{définition}\pfra{causer de l'inquiétude à qqun}\end{définition}
\begin{définition}\pcmn{令……担心}\end{définition}
\begin{exemple}\pjya{nɯ-znɯzdɯɣ-a}\hspace{5pt}\pcmn{我让他担心了}\end{exemple}
\begin{sous-entrée}{sɤnɯzdɯɣ}{ⓔnɯzdɯɣⓝsɤnɯzdɯɣ} 
\classe{vs}  
\grammaire{deexp} 
\begin{définition}\pfra{causer de l'inquiétude}\end{définition}
\begin{définition}\pcmn{令人担心}\end{définition}\end{sous-entrée}

\étymologie{sdug}\end{entrée}

\begin{entrée}{nɯzdɯsŋɤl}{}{ⓔnɯzdɯsŋɤl} 
\classe{vi} \paradigme{dir}{pɯ-}
\begin{définition}\pfra{supporter toutes sortes de difficultés}\end{définition}
\begin{définition}\pcmn{受尽苦难}\end{définition}
\begin{exemple}\pjya{kɯ-rtaʁ ʑo pɯ-nɯzdɯsŋal-a}\hspace{5pt}\pcmn{我受够了苦难}\end{exemple}\relationsémantique{参考}{\lien{ⓔtɤzdɯɣ}{tɤzdɯɣ}}\étymologie{sdug.bsŋal}\end{entrée}

\begin{entrée}{nɯzdɯxpa/\variante{znɯzdɯxpa}}{}{ⓔnɯzdɯxpa} 
\classe{vt} \paradigme{dir}{nɯ-}
\begin{définition}\pfra{avoir pitié de}\end{définition}
\begin{définition}\pcmn{可怜别人}\end{définition}
\begin{exemple}\pjya{nɯ-nɯzdɯxpa-t-a-nɯ}\hspace{5pt}\pcmn{我可怜他们}\end{exemple}
\begin{exemple}\pjya{jiɕqha nɯ ɯ-ŋgu mɯ́j-thon, ɲɯ-nɯzdɯxpe-a}\hspace{5pt}\pcmn{他很穷,我很可怜他}\end{exemple}\relationsémantique{参考}{\lien{ⓔsɤzdɯxpa}{sɤzdɯxpa}}\end{entrée}

\begin{entrée}{nɯzgomdʑo}{}{ⓔnɯzgomdʑo} 
\classe{vi} \paradigme{dir}{\_}
\begin{définition}\pfra{se promener dans la montagne et admirer le paysage}\end{définition}
\begin{définition}\pcmn{在山上观光}\end{définition}
\begin{exemple}\pjya{jiɕqha nɯ kɯ-nɯzgomdʑo jɤ-ari}\hspace{5pt}\pcmn{他去山上观光了}\end{exemple}\relationsémantique{同义词}{\lien{ⓔnɤmɲole}{nɤmɲole}}\end{entrée}

\begin{entrée}{nɯzgrɯtɕhɯ}{}{ⓔnɯzgrɯtɕhɯ} 
\classe{vt}  
\grammaire{incorp} \paradigme{dir}{tɤ-}
\begin{définition}\pfra{donner un coup de coude}\end{définition}
\begin{définition}\pcmn{用肘碰}\end{définition}
\begin{exemple}\pjya{tɤ́-wɣ-nɯzgrɯtɕhɯ-a}\hspace{5pt}\pcmn{他用肘打了我}\end{exemple}\relationsémantique{参考}{\lien{ⓔzgrɯtɕhɯ}{zgrɯtɕhɯ}}\end{entrée}

\begin{entrée}{nɯzɣɯt}{}{ⓔnɯzɣɯt}\relationsémantique{参考}{\lien{ⓔzɣɯt}{zɣɯt}}\end{entrée}

\begin{entrée}{nɯzjaŋ}{}{ⓔnɯzjaŋ}\relationsémantique{参考}{\lien{ⓔɣɤzjaŋlaŋ}{ɣɤzjaŋlaŋ}}\end{entrée}

\begin{entrée}{nɯzɟɯ}{}{ⓔnɯzɟɯ} 
\classe{vi} \paradigme{dir}{pɯ-}\paradigme{dir}{pɯ-}
\begin{définition}\pfra{pâtir de quelque chose}\end{définition}
\begin{définition}\pcmn{吃亏}\end{définition}
\begin{définition}\pfra{faire pâtir quelqu'un de quelque chose}\end{définition}
\begin{définition}\pcmn{使吃亏}\end{définition}
\begin{exemple}\pjya{pɯ-nɯzɟɯ}\hspace{5pt}\pcmn{他吃亏了}\end{exemple}
\begin{exemple}\pjya{tɯtsɣe tɤ-βzu-tɕi, aʑo pɯ-nɯzɟɯ-a, nɤʑo pɯ-tɯ-nɤndʑe}\hspace{5pt}\pcmn{我们俩做了生意,我吃亏了,你占了便宜}\end{exemple}
\begin{exemple}\pjya{ʑɴɢɯloʁ pɯ-nɯkro-tɕi, nɤʑo pɯ-tɯ-nɤndʑe, aʑo pɯ-nɯzɟɯ-a}\hspace{5pt}\pcmn{我们俩分核桃,你占了便宜,我吃亏了}\end{exemple}
\begin{exemple}\pjya{pɯ-kɯ-znɯzɟɯ-a}\hspace{5pt}\pcmn{你让我吃亏了}\end{exemple}\relationsémantique{反义词}{\lien{ⓔnɤndʑe}{nɤndʑe}}
\begin{sous-entrée}{znɯzɟɯ}{ⓔnɯzɟɯⓝznɯzɟɯ} 
\classe{vt}  
\grammaire{caus} \end{sous-entrée}

\end{entrée}

\begin{entrée}{nɯzrɯɣru}{}{ⓔnɯzrɯɣru} 
\classe{vi}  
\grammaire{incorp} \paradigme{dir}{pɯ-}
\begin{définition}\pfra{chercher les poux}\end{définition}
\begin{définition}\pcmn{找虱子}\end{définition}
\begin{exemple}\pjya{jiɕqha mbroχpa ɲɯ-nɯzrɯɣru}\hspace{5pt}\pcmn{那个牧民在找虱子}\end{exemple}
\begin{exemple}\pjya{pɯ-nɯzrɯɣru-a}\hspace{5pt}\pcmn{我在找虱子}\end{exemple}\relationsémantique{参考}{\lien{ⓔzrɯɣru}{zrɯɣru}}\end{entrée}

\begin{entrée}{nɯzʁe}{}{ⓔnɯzʁe} 
\classe{vt} \paradigme{dir}{\_}
\begin{définition}\pfra{transporter un par un}\end{définition}
\begin{définition}\pcmn{一个一个地搬运}\end{définition}
\begin{exemple}\pjya{rdɤstaʁ lɤ-nɯzʁe-t-a}\hspace{5pt}\pcmn{我把石头搬过去了}\end{exemple}\relationsémantique{参考}{\lien{ⓔnɯsɯzʁe}{nɯsɯzʁe}}\end{entrée}

\begin{entrée}{nɯʑɤla}{}{ⓔnɯʑɤla} 
\classe{vt} \paradigme{dir}{kɤ-}
\begin{définition}\pfra{transmettre (poux)}\end{définition}
\begin{définition}\pcmn{传染(虱子)}\end{définition}
\begin{exemple}\pjya{a-tɕɯ ɯ-zda ra kɯ zrɯɣ kó-wɣ-nɯʑɤla}\hspace{5pt}\pcmn{我儿子的同学给他传染了虱子}\end{exemple}\relationsémantique{同义词}{\lien{ⓔɕte}{ɕte}}\end{entrée}

\begin{entrée}{nɯʑɤŋɤn}{}{ⓔnɯʑɤŋɤn} 
\classe{vt} \paradigme{dir}{tɤ-}
\begin{définition}\pfra{taquiner}\end{définition}
\begin{définition}\pcmn{逗着……玩}\end{définition}
\begin{exemple}\pjya{tu-ta-nɯʑɤŋɤn ɕti tɕe ma-tɤ-tɯ-qhe je}\hspace{5pt}\pcmn{我只是逗你的,你不要生气}\end{exemple}\relationsémantique{同义词}{\lien{ⓔnɯrtɕa}{nɯrtɕa}}\end{entrée}

\begin{entrée}{nɯʑɤzdaŋ}{}{ⓔnɯʑɤzdaŋ} 
\classe{vt} \paradigme{dir}{tɤ-}
\begin{définition}\pfra{envier, vouloir imiter}\end{définition}
\begin{définition}\pcmn{妒忌}\end{définition}
\begin{sous-entrée}{sɤnɯʑɤzdaŋ}{ⓔnɯʑɤzdaŋⓝsɤnɯʑɤzdaŋ} 
\classe{vi} 
\begin{définition}\pfra{envier les gens}\end{définition}
\begin{définition}\pcmn{妒忌别人}\end{définition}
\begin{exemple}\pjya{ma-tɯ-sɤnɯʑɤzdaŋ}\hspace{5pt}\pcmn{你不要妒忌别人}\end{exemple}\relationsémantique{同义词}{\lien{ⓔnɤʑɤmŋɤn}{nɤʑɤmŋɤn}}\end{sous-entrée}

\étymologie{ʑe.sdaŋ}\end{entrée}

\begin{entrée}{nɯʑgrɤɣ}{}{ⓔnɯʑgrɤɣ} 
\classe{vt} \paradigme{dir}{pɯ-}
\begin{définition}\pfra{renverser avec force}\end{définition}
\begin{définition}\pcmn{很轻松地把对方摔下去}\end{définition}
\begin{exemple}\pjya{pa-nɯʑgrɤɣ ʑo pa-tʂaβ}\hspace{5pt}\pcmn{他很轻松地把他摔下去了}\end{exemple}\relationsémantique{同义词}{\lien{ⓔnɯɕkrɤɣ}{nɯɕkrɤɣ}}\relationsémantique{参考}{\lien{ⓔʑgrɤɣʑgrɤɣ}{ʑgrɤɣʑgrɤɣ}}\end{entrée}

\begin{entrée}{nɯʑɣɤβri}{}{ⓔnɯʑɣɤβri}\relationsémantique{参考}{\lien{ⓔβri}{βri}}\end{entrée}

\begin{entrée}{nɯʑmbɤr}{}{ⓔnɯʑmbɤr} 
\classe{vi}  
\grammaire{denom} \paradigme{dir}{nɯ-}
\begin{définition}\pfra{avoir une pustule}\end{définition}
\begin{définition}\pcmn{生疮}\end{définition}
\begin{exemple}\pjya{a-rŋa ɯ-ʑmbɤr ɲɤ-ɬoʁ, ɲɯ-nɯʑmbar-a}\hspace{5pt}\pcmn{我脸上生了疮}\end{exemple}
\begin{exemple}\pjya{pɯ-nɯʑmbɤr tɕe ɯ-sta ʁmazgrɯβ tu}\hspace{5pt}\pcmn{他生过疮,有伤疤}\end{exemple}\relationsémantique{参考}{\lien{ⓔʑmbɤr}{ʑmbɤr}}\end{entrée}

\begin{entrée}{nɯʑo}{}{ⓔnɯʑo} 
\classe{pro} 
\begin{définition}\pfra{vous}\end{définition}
\begin{définition}\pcmn{你们}\end{définition}\end{entrée}

\begin{entrée}{nɯʑɯβ}{}{ⓔnɯʑɯβ} 
\classe{vi} \paradigme{dir}{pɯ-}\paradigme{dir}{kɤ-}
\begin{définition}\pfra{s’endormir}\end{définition}
\begin{définition}\pcmn{睡着}\end{définition}
\begin{exemple}\pjya{pɯ-nɯʑɯβ-a}\hspace{5pt}\pcmn{我睡着了}\end{exemple}
\begin{exemple}\pjya{ma-tɯ-ɤrju-nɯ tɕe, aj pjɯ-nɯʑɯβ-a}\hspace{5pt}\pcmn{你们不要说话,我在睡觉}\end{exemple}
\begin{sous-entrée}{ɣɤnɯʑɯβ}{ⓔnɯʑɯβⓝɣɤnɯʑɯβ} 
\classe{vs} 
\begin{définition}\pfra{qui arrive facilement à s'endormir}\end{définition}
\begin{définition}\pcmn{容易入眠;容易睡着}\end{définition}
\begin{exemple}\pjya{ɲɯ-ɣɤnɯʑɯβ}\hspace{5pt}\pcmn{(小孩子)容易入眠}\end{exemple}\relationsémantique{参考}{\lien{}{ɯ-ʑɯβ}}\end{sous-entrée}

\end{entrée}

\begin{entrée}{nɯʑɯβri}{}{ⓔnɯʑɯβri} 
\classe{vi} \paradigme{dir}{pɯ-}
\begin{définition}\pfra{veiller; avoir une insomnie}\end{définition}
\begin{définition}\pcmn{失眠;熬夜‘没有睡够}\end{définition}
\begin{exemple}\pjya{pjɯ-nɯʑɯβri-a}\hspace{5pt}\pcmn{我熬夜}\end{exemple}
\begin{exemple}\pjya{a-ndʐuwa pɯ-tu tɕe kɤ-nɯʑɯβ mɯ-pɯ-ŋgrɯ tɕe pjɤ-nɯʑɯβri-a}\hspace{5pt}\pcmn{因为有客人没有睡成}\end{exemple}
\begin{exemple}\pjya{jɯfɕɯr pɯ-nɯʑɯβri-a, jisŋi a-ʑɯβ ɲɯ-ɣi}\hspace{5pt}\pcmn{我昨天熬夜了,今天就打瞌睡}\end{exemple}\relationsémantique{参考}{\lien{ⓔtɯ-ʑɯβ}{tɯ-ʑɯβ}}\relationsémantique{参考}{\lien{ⓔri}{ri}}\end{entrée}

\newpage\caractère{ɲ}

\begin{entrée}{ɲakhri}{}{ⓔɲakhri} 
\classe{n} 
\begin{définition}\pfra{lit}\end{définition}
\begin{définition}\pcmn{床}\end{définition}\étymologie{ɲal.kʰri}\end{entrée}

\begin{entrée}{ɲaɲa}{}{ⓔɲaɲa} 
\classe{n} 
\begin{définition}\pfra{agneau}\end{définition}
\begin{définition}\pcmn{绵羊羔}\end{définition}
\begin{exemple}\pjya{ɲaɲa ɣɯ ɯ-rme}\hspace{5pt}\pcmn{羊羔的毛}\end{exemple}\end{entrée}

\begin{entrée}{ɲaʁ}{}{ⓔɲaʁ} 
\classe{vs} \paradigme{dir}{nɯ-}
\begin{définition}\pfra{noir}\end{définition}
\begin{définition}\pcmn{黑(颜色)}\end{définition}\relationsémantique{参考}{\lien{ⓔsɯɣɲaʁ}{sɯɣɲaʁ}}\end{entrée}

\begin{entrée}{ɲaʁtɣi}{}{ⓔɲaʁtɣi} 
\classe{n} 
\begin{définition}\pfra{empan (pouce et majeur)}\end{définition}
\begin{définition}\pcmn{一拃(大拇指和中指之间的距离)}\end{définition}
\begin{exemple}\pjya{ɲaʁtɣi tɯ-tɣa}\hspace{5pt}\pcmn{一拃}\end{exemple}\end{entrée}

\begin{entrée}{ɲat}{}{ⓔɲat} 
\classe{vi} \paradigme{dir}{nɯ-}\paradigme{dir}{nɯ-}
\begin{définition}\pfra{être fatigué}\end{définition}
\begin{définition}\pcmn{累}\end{définition}
\begin{définition}\pfra{fatiguer}\end{définition}
\begin{définition}\pcmn{令人累}\end{définition}
\begin{exemple}\pjya{aʑo pɯ-rɤma-a tɕe ɲɤ-ɲat-a}\hspace{5pt}\pcmn{我劳动了就累了}\end{exemple}
\begin{exemple}\pjya{ki kɤ-nɤma ɲɯ-ɴqa tɕe ɲɯ-kɯ-sɯɣɲat}\hspace{5pt}\pcmn{这个工作很辛苦,令人很累}\end{exemple}\relationsémantique{参考}{\lien{ⓔsɤɣɲat}{sɤɣɲat}}
\begin{sous-entrée}{sɯɣɲat}{ⓔɲatⓝsɯɣɲat} 
\classe{vt} \end{sous-entrée}

\end{entrée}

\begin{entrée}{ɲawa}{}{ⓔɲawa} 
\classe{n} 
\begin{définition}\pfra{accouplement (animaux)}\end{définition}
\begin{définition}\pcmn{交配(动物)}\end{définition}\end{entrée}

\begin{entrée}{ɲɤβɲɤβ}{}{ⓔɲɤβɲɤβ} 
\classe{idph.2} 
\begin{définition}\pfra{très mou}\end{définition}
\begin{définition}\pcmn{软绵绵}\end{définition}
\begin{exemple}\pjya{ɲɯ-ngo tɕe, ɲɤβɲɤβ ʑo ɲɯ-rɤʑi}\hspace{5pt}\pcmn{他病了就没有精神}\end{exemple}
\begin{exemple}\pjya{ko-smi ɲɤβɲɤβ ʑo}\hspace{5pt}\pcmn{熟得很软}\end{exemple}\end{entrée}

\begin{entrée}{ɲɤβrɯɣ}{}{ⓔɲɤβrɯɣ} 
\classe{n} 
\begin{définition}\pfra{espèce d'arbre}\end{définition}
\begin{définition}\pcmn{树的一种}\end{définition}\end{entrée}

\begin{entrée}{ɲɤndɤpa}{}{ⓔɲɤndɤpa} 
\classe{adv} 
\begin{définition}\pfra{dans quatre ans}\end{définition}
\begin{définition}\pcmn{四年以后}\end{définition}\end{entrée}

\begin{entrée}{ɲɤndi}{}{ⓔɲɤndi} 
\classe{adv} 
\begin{définition}\pfra{dans quatre jours}\end{définition}
\begin{définition}\pcmn{四天以后}\end{définition}\end{entrée}

\begin{entrée}{ɲɤntsho}{}{ⓔɲɤntsho} 
\classe{n} 
\begin{définition}\pfra{pistolet}\end{définition}
\begin{définition}\pcmn{手枪}\end{définition}\end{entrée}

\begin{entrée}{ɲɤsma}{}{ⓔɲɤsma} 
\classe{n} 
\begin{définition}\pfra{feue (décédée)}\end{définition}
\begin{définition}\pcmn{去世了的女子}\end{définition}\relationsémantique{参考}{\lien{ⓔɲɤspa}{ɲɤspa}}\end{entrée}

\begin{entrée}{ɲɤspa}{}{ⓔɲɤspa} 
\classe{n} 
\begin{définition}\pfra{feu (homme décédé)}\end{définition}
\begin{définition}\pcmn{去世了的男人}\end{définition}
\begin{exemple}\pjya{ɲɤspa a-nɯ-tɯ-βze}\hspace{5pt}\pcmn{你去死吧!}\end{exemple}\étymologie{ɲes.pa}\end{entrée}

\begin{entrée}{ɲɤzma}{}{ⓔɲɤzma} 
\classe{n} 
\begin{définition}\pfra{feue (femme décédée)}\end{définition}
\begin{définition}\pcmn{去世了的女子}\end{définition}\étymologie{ɲes.ma}\end{entrée}

\begin{entrée}{ɲcɤr}{}{ⓔɲcɤr} 
\classe{vt} \paradigme{dir}{pɯ-}
\begin{définition}\pfra{appuyer}\end{définition}
\begin{définition}\pcmn{(用全身的力量) 按;压住}\end{définition}
\begin{exemple}\pjya{rdɤstaʁ kɯ si nɯ pɯ-sɯ-ɲcɤr}\hspace{5pt}\pcmn{你用石头把柴压着(防止柴被人拿)}\end{exemple}
\begin{exemple}\pjya{ki ɕoʁɕoʁ ki thɯci kɯ pɯ-sɯ-ɲcɤr ma qale kɯ nɯtsɯm}\hspace{5pt}\pcmn{你用什么东西把纸压住,不然会被风吹走}\end{exemple}
\begin{exemple}\pjya{paʁ pɯ-ɲcar-a}\hspace{5pt}\pcmn{我把猪压住了}\end{exemple}
\begin{exemple}\pjya{tɯrme ɯ-stu mɤ-kɯ-fse nɯ pjɯ́-wɣ-ɲcɤr tɕe ɯ-stu kú-wɣ-z-rɤʑi ŋu}\hspace{5pt}\pcmn{人不听话要压一下,使他温顺}\end{exemple}\end{entrée}

\begin{entrée}{ɲcɣɤɲcɣɤt}{}{ⓔɲcɣɤɲcɣɤt} 
\classe{idph.2} 
\begin{définition}\pfra{très nombreux}\end{définition}
\begin{définition}\pcmn{很多(朝着同一个方向);嘈杂(声音);旺盛}\end{définition}
\begin{exemple}\pjya{ɲcɣɤɲcɣɤt ʑo chɤ-k-ɤkhar-nɯ-ci}\hspace{5pt}\pcmn{很多人围着坐}\end{exemple}
\begin{exemple}\pjya{ɲcɣɤɲcɣɤt ʑo pjɤ-ɕkho}\hspace{5pt}\pcmn{铺了很多(东西)在地上}\end{exemple}
\begin{exemple}\pjya{tɤtʂu ɲcɣɤɲcɣɤt ʑo to-zwɤr-nɯ}\hspace{5pt}\pcmn{他们把灯开得亮堂堂的}\end{exemple}
\begin{exemple}\pjya{laχtɕha ɲcɣɤɲcɣɤt ʑo to-fɕɤm-nɯ}\hspace{5pt}\pcmn{他们把(很多)东西摆出来了}\end{exemple}
\begin{exemple}\pjya{tɯrme ɲcɣɤɲcɣɤt ʑo ɲɯ-ɤmdzɯt-nɯ}\hspace{5pt}\pcmn{很多人在那里坐着,面向一个方向}\end{exemple}
\begin{sous-entrée}{ɲcɣɤnɤɲcɣɤt}{ⓔɲcɣɤɲcɣɤtⓝɲcɣɤnɤɲcɣɤt} 
\classe{idph.3} 
\begin{exemple}\pjya{ɲcɣɤnɤɲcɣɤt ɲɯ-rɟaʁ-nɯ}\hspace{5pt}\pcmn{很多人在跳舞}\end{exemple}\relationsémantique{参考}{\lien{ⓔɣɤɲcɣɤɲcɣɤt}{ɣɤɲcɣɤɲcɣɤt}}\end{sous-entrée}

\end{entrée}

\begin{entrée}{ɲchaʁɲchaʁ}{}{ⓔɲchaʁɲchaʁ} 
\classe{idph.2} 
\begin{définition}\pfra{un peu froid}\end{définition}
\begin{définition}\pcmn{形容些微的寒冷}\end{définition}
\begin{exemple}\pjya{jɯxɕo tɯ-mɯ ko-lɤt tɕe, ɯ-pɕi ɲɯ-mɯɕtaʁ ɲchaʁɲchaʁ}\hspace{5pt}\pcmn{今天早上下了雨,外面有点冷}\end{exemple}\end{entrée}

\begin{entrée}{ɲchɤɲchɤr}{}{ⓔɲchɤɲchɤr} 
\classe{idph.2} 
\begin{définition}\pfra{dilué}\end{définition}
\begin{définition}\pcmn{形容流体稀}\end{définition}
\begin{exemple}\pjya{a-rɟɤɣi ɯ-ci pjɤ-dɤn tɕe ɲɯ-ŋgri ʑo ɲchɤɲchɤr, tɕe kɤ-rɤlaj mɯ́j-khɯ}\hspace{5pt}\pcmn{糌粑里的水分太多,太稀了,没有办法挼}\end{exemple}\end{entrée}

\begin{entrée}{ɲchɣaʁ}{}{ⓔɲchɣaʁ} 
\classe{n} 
\begin{définition}\pfra{écorce de bouleau}\end{définition}
\begin{définition}\pcmn{白桦树皮}\end{définition}\end{entrée}

\begin{entrée}{ɲchɣaʁʑɤr}{}{ⓔɲchɣaʁʑɤr} 
\classe{n} 
\begin{définition}\pfra{verglas}\end{définition}
\begin{définition}\pcmn{冰雪路}\end{définition}\end{entrée}

\begin{entrée}{ɲchoʁ}{}{ⓔɲchoʁ} 
\classe{vi} \paradigme{dir}{pɯ-}\paradigme{dir}{nɯ-}
\begin{définition}\pfra{se dégonfler}\end{définition}
\begin{définition}\pcmn{瘪下去}\end{définition}
\begin{exemple}\pjya{@piqiu pjɤ-ɲchoʁ}\hspace{5pt}\pcmn{皮球瘪下去了}\end{exemple}
\begin{exemple}\pjya{kɤrŋi kú-wɣ-sqa tɕe, lú-wɣ-sqa ɕɯmɯma dɤn ri, kɤ-smi tɕe tɕe ɲɯ-rkɯn ŋu tɕe tɕe ɲɤ-ɲchoʁ tu-kɯ-ti ŋu}\hspace{5pt}\pcmn{蔬菜熟了以后就萎缩了}\end{exemple}
\begin{exemple}\pjya{ɯ-lɯm ɲɯ-kɯ-xtɕi nɯ tɕe ɲɯ-kɯ-ɲchoʁ tu-kɯ-ti}\hspace{5pt}\pcmn{体积变小就做“瘪下去”}\end{exemple}
\begin{exemple}\pjya{ɲɯ-mtsɯr-a tɕe a-xtu ɲɤ-ɲchoʁ}\hspace{5pt}\pcmn{我很饿,肚子瘪了}\end{exemple}\relationsémantique{同义词}{\lien{ⓔxɕɯβ}{xɕɯβ}}\end{entrée}

\begin{entrée}{ɲcriɲcri/\variante{ɲcɯɲcri}}{}{ⓔɲcriɲcri} 
\classe{idph.2} 
\begin{définition}\pfra{liquide, dilué (boue)}\end{définition}
\begin{définition}\pcmn{稀(泥巴)}\end{définition}\relationsémantique{参考}{\lien{ⓔɲcrɯɣɲcrɯɣ}{ɲcrɯɣɲcrɯɣ}}\relationsémantique{参考}{\lien{ⓔscrɯscri}{scrɯscri}}\relationsémantique{参考}{\lien{ⓔχcrɯχcri}{χcrɯχcri}}\relationsémantique{参考}{\lien{ⓔscrɯscri}{scrɯscri}}\end{entrée}

\begin{entrée}{ɲcrɯɣɲcrɯɣ}{}{ⓔɲcrɯɣɲcrɯɣ} 
\classe{idph.2} 
\begin{définition}\pfra{mou et peu épais (boue, crotte)}\end{définition}
\begin{définition}\pcmn{形容(泥巴、牛粪) 又稀又软的样子}\end{définition}
\begin{exemple}\pjya{tɤrcoʁ ɲcrɯɣɲcrɯɣ ɲɯ-ŋu}\hspace{5pt}\pcmn{泥巴很稀}\end{exemple}\relationsémantique{参考}{\lien{ⓔɲcriɲcri}{ɲcriɲcri}}\relationsémantique{参考}{\lien{ⓔscrɯscri}{scrɯscri}}\end{entrée}

\begin{entrée}{ɲɟa}{}{ⓔɲɟa} 
\classe{vs} \paradigme{dir}{kɤ-}
\begin{définition}\pfra{trop vieux (animal)}\end{définition}
\begin{définition}\pcmn{老得不行(动物)}\end{définition}
\begin{exemple}\pjya{jiɕqha nɯ ko-ɲɟa}\hspace{5pt}\pcmn{那个老得不行了}\end{exemple}
\begin{exemple}\pjya{fsapaʁ ko-ɲɟa}\hspace{5pt}\pcmn{牲畜老得不行了}\end{exemple}\end{entrée}

\begin{entrée}{ɲɟɤβ}{}{ⓔɲɟɤβ} 
\classe{vi}  
\grammaire{acaus} \paradigme{dir}{pɯ-}\paradigme{dir}{kɤ-}\sens{1}
\begin{définition}\pfra{s’aplatir}\end{définition}
\begin{définition}\pcmn{压扁;凹进去}\end{définition}
\begin{exemple}\pjya{tɯthɯ pjɤ-ɲɟɤβ}\hspace{5pt}\pcmn{锅子凹进去了}\end{exemple}
\begin{exemple}\pjya{jɤmtsa pjɤ-ɲɟɤβ}\hspace{5pt}\pcmn{炒菜锅凹进去了}\end{exemple}
\begin{exemple}\pjya{khɯtsa pjɤ-ɲɟɤβ}\hspace{5pt}\pcmn{碗凹进去了}\end{exemple}\sens{2}
\begin{définition}\pfra{avoir une fracture}\end{définition}
\begin{définition}\pcmn{骨折}\end{définition}
\begin{exemple}\pjya{a-rnom ko-ɲɟɤβ}\hspace{5pt}\pcmn{我肋骨折了}\end{exemple}
\begin{exemple}\pjya{a-ɣrɯmke pjɤ-ɲɟɤβ}\hspace{5pt}\pcmn{我的手腕骨折了}\end{exemple}\relationsémantique{参考}{\lien{ⓔchɤβ}{chɤβ}}\end{entrée}

\begin{entrée}{ɲɟɤlɤsnɯrsnɯr/\variante{ɲɟɤlɤsnɯsnɯr}}{}{ⓔɲɟɤlɤsnɯrsnɯr} 
\classe{n} 
\begin{définition}\pfra{viande séchée conservée dans les intestins}\end{définition}
\begin{définition}\pcmn{装在肠子里的瘦肉}\end{définition}\end{entrée}

\begin{entrée}{ɲɟɤrɲɟɤr}{}{ⓔɲɟɤrɲɟɤr} 
\classe{idph.2} 
\begin{définition}\pfra{dodu}\end{définition}
\begin{définition}\pcmn{形容体貌肥嘟嘟,庞大的样子}\end{définition}
\begin{exemple}\pjya{mtshu ɲɟɤrɲɟɤr ʑo ɲɯ-pa}\hspace{5pt}\pcmn{湖又大又满}\end{exemple}
\begin{exemple}\pjya{ɯʑo ɯ-tɯ-tshu kɯ ɲɟɤrɲɟɤr ʑo ɲɯ-pa}\hspace{5pt}\pcmn{他胖得满身都是肉}\end{exemple}
\begin{exemple}\pjya{jla ɲɟɤrɲɟɤr ɲɯ-rɤʑi}\hspace{5pt}\pcmn{犏牛身躯庞大地站在那里}\end{exemple}
\begin{exemple}\pjya{paʁ ɲɯ-tshu ɲɟɤrɲɟɤr ʑo}\hspace{5pt}\pcmn{猪肥嘟嘟的}\end{exemple}
\begin{exemple}\pjya{tɤɕi tɯ-fkur ʑo ɲɟɤrɲɟɤr to-rku}\hspace{5pt}\pcmn{青稞袋子装得很满(大)}\end{exemple}
\begin{sous-entrée}{ɲɟɤrnɤɲɟɤr}{ⓔɲɟɤrɲɟɤrⓝɲɟɤrnɤɲɟɤr} 
\classe{idph.3} 
\begin{exemple}\pjya{kɯ-tshu ci ɲɟɤrnɤɲɟɤr ɲɯ-ŋu ɲɯ-nɤŋkɯŋke}\hspace{5pt}\pcmn{有个胖子在走来走去}\end{exemple}
\begin{exemple}\pjya{paʁ ɲɟɤrnɤɲɟɤr ʑo ɲɯ-ŋke}\hspace{5pt}\pcmn{肥嘟嘟的猪在走}\end{exemple}\end{sous-entrée}

\begin{sous-entrée}{mɤlɤɲɟɤr}{ⓔɲɟɤrɲɟɤrⓝmɤlɤɲɟɤr} 
\classe{idph.6} 
\begin{définition}\pfra{énorme}\end{définition}
\begin{définition}\pcmn{形容高大(搬都搬不动)}\end{définition}
\begin{exemple}\pjya{rŋgɯ mɤlɤɲɟɤr ci ɲɯ-ŋu tɕe kɤ-ɣɤrɤt mɯ́j-sɤcha}\hspace{5pt}\pcmn{石包很高大,搬也搬不动}\end{exemple}\relationsémantique{参考}{\lien{ⓔɣɤɲɟɤrɲɟɤr}{ɣɤɲɟɤrɲɟɤr}}\end{sous-entrée}

\end{entrée}

\begin{entrée}{ɲɟɤt}{}{ⓔɲɟɤt} 
\classe{vi} \paradigme{dir}{tɤ-}
\begin{définition}\pfra{regretter}\end{définition}
\begin{définition}\pcmn{后悔}\end{définition}
\begin{exemple}\pjya{ɲɯ-ɲɟat-a}\hspace{5pt}\pcmn{我后悔}\end{exemple}
\begin{exemple}\pjya{tɤ-ɲɟat-a}\hspace{5pt}\pcmn{我后悔了}\end{exemple}
\begin{exemple}\pjya{to-ɲɟɤt}\hspace{5pt}\pcmn{他后悔了}\end{exemple}
\begin{exemple}\pjya{jiɕqha tɤ-tɯt-a nɯ ɲɯ-ɲɟat-a}\hspace{5pt}\pcmn{我后悔刚才讲的话}\end{exemple}\étymologie{ⁿgʲod}\end{entrée}

\begin{entrée}{ɲɟo}{}{ⓔɲɟo} 
\classe{vi} \paradigme{dir}{pɯ-}\paradigme{dir}{pɯ-}
\begin{définition}\pfra{subir des dommages}\end{définition}
\begin{définition}\pcmn{受损;受伤;受灾}\end{définition}
\begin{définition}\pfra{abîmer}\end{définition}
\begin{définition}\pcmn{弄坏}\end{définition}
\begin{exemple}\pjya{ɯʑo pɯ-ɲɟo}\hspace{5pt}\pcmn{他遇到灾难了}\end{exemple}
\begin{exemple}\pjya{nɤʑo pɯ-tɯ-ɲɟo}\hspace{5pt}\pcmn{你遇到灾难了}\end{exemple}
\begin{exemple}\pjya{laχtɕha pjɤ-ɲɟo}\hspace{5pt}\pcmn{东西损坏了}\end{exemple}
\begin{exemple}\pjya{ɯ-mi pjɤ-ɲɟo}\hspace{5pt}\pcmn{他的脚坏了}\end{exemple}
\begin{exemple}\pjya{pɯ-ndʐaβ-a tɕe pɯ-ɲɟo-a}\hspace{5pt}\pcmn{他摔跤了就受伤了}\end{exemple}
\begin{exemple}\pjya{kɯ-ɲɟo a-pɯme tɕe mɤ-kɯ-pe me}\hspace{5pt}\pcmn{只要没有损失就没有什么不好的}\end{exemple}
\begin{exemple}\pjya{ɣɯjpa tɯ-xpa kɯ-ɲɟo me}\hspace{5pt}\pcmn{今年一年都没有损失}\end{exemple}
\begin{exemple}\pjya{pɯ-kɯ-ɲɟo me-a}\hspace{5pt}\pcmn{我没有受到损伤}\end{exemple}
\begin{exemple}\pjya{ɲɟɯ-ɲɟo ʑo pɯ-ɲɟo}\hspace{5pt}\pcmn{他遭到很多灾难}\end{exemple}
\begin{exemple}\pjya{@tuolaji pjɤ-sɯɣɲɟo}\hspace{5pt}\pcmn{他把拖拉机弄坏了}\end{exemple}
\begin{exemple}\pjya{pɯ-sɯɣɲɟo-t-a}\hspace{5pt}\pcmn{我弄坏了}\end{exemple}
\begin{sous-entrée}{sɯɣɲɟo}{ⓔɲɟoⓝsɯɣɲɟo} 
\classe{vt}  
\grammaire{caus} \end{sous-entrée}

\end{entrée}

\begin{entrée}{ɲɟoʁ}{}{ⓔɲɟoʁ} 
\classe{vt} \paradigme{dir}{kɤ-}\paradigme{dir}{nɯ-}\paradigme{dir}{pɯ-}\paradigme{dir}{kɤ-}
\begin{définition}\pfra{coller}\end{définition}
\begin{définition}\pcmn{贴}\end{définition}
\begin{définition}\pfra{se coller sur}\end{définition}
\begin{définition}\pcmn{把(自己的)身体贴在}\end{définition}
\begin{exemple}\pjya{nɯ-ɲɟoʁ-a, pɯ-ɲɟoʁ-a}\hspace{5pt}\pcmn{我贴了}\end{exemple}
\begin{exemple}\pjya{tɕetu ki kɤ-ɲɟoʁ-a}\hspace{5pt}\pcmn{我把这个东西贴在上面了}\end{exemple}
\begin{exemple}\pjya{znde ɯ-taʁ ko-ɲɟoʁ}\hspace{5pt}\pcmn{他贴在墙上了}\end{exemple}
\begin{exemple}\pjya{kɤ-anbaʁ-a tɕe znde ɯ-taʁ kɤ-ʑɣɤsɤɲɟoʁ-a}\hspace{5pt}\pcmn{我躲起来的时候把身体贴在墙上了}\end{exemple}
\begin{sous-entrée}{aɲɟoʁ}{ⓔɲɟoʁⓝaɲɟoʁ} 
\classe{vi}  
\grammaire{pass} 
\begin{définition}\pfra{être collé}\end{définition}
\begin{définition}\pcmn{贴着}\end{définition}\end{sous-entrée}

\begin{sous-entrée}{ʑɣɤsɤɲɟoʁ}{ⓔɲɟoʁⓝʑɣɤsɤɲɟoʁ} 
\classe{vi}  
\grammaire{refl}
\grammaire{caus} \end{sous-entrée}

\end{entrée}

\begin{entrée}{ɲɟɯ}{}{ⓔɲɟɯ} 
\classe{vi}  
\grammaire{acaus} \paradigme{dir}{kɤ-}\paradigme{dir}{tɤ-}
\begin{définition}\pfra{être ouvert, s'ouvrir}\end{définition}
\begin{définition}\pcmn{开着;自动打开}\end{définition}
\begin{exemple}\pjya{qale ta-βzu tɕe, kɯm kɤ-ɲɟɯ}\hspace{5pt}\pcmn{风一吹,门就开了}\end{exemple}
\begin{exemple}\pjya{tʂu to-ɲɟɯ}\hspace{5pt}\pcmn{路开了}\end{exemple}
\begin{exemple}\pjya{rgɤm tɤ-ɲɟɯ}\hspace{5pt}\pcmn{盒子开了}\end{exemple}
\begin{exemple}\pjya{kɯm a-pɯ-nɯ-ɲɟɯ je!}\hspace{5pt}\pcmn{让门开着}\end{exemple}\relationsémantique{参考}{\lien{ⓔcɯⓗ1}{cɯ₁}}\relationsémantique{参考}{\lien{ⓔɯ-ɣɲɟɯ}{ɯ-ɣɲɟɯ}}\end{entrée}

\begin{entrée}{ɲɟɯɣ}{}{ⓔɲɟɯɣ} 
\classe{vi} \paradigme{dir}{tɤ-}
\begin{définition}\pfra{s'entendre}\end{définition}
\begin{définition}\pcmn{合得来}\end{définition}
\begin{exemple}\pjya{ɲɯ-ɲɟɯɣ-ndʑi}\hspace{5pt}\pcmn{他们合得来}\end{exemple}
\begin{exemple}\pjya{ki ɯ-ʁɤri mɯ-pɯ-amɯmi-ndʑi ri, tham to-ɲɟɯɣ-ndʑi}\hspace{5pt}\pcmn{他们以前关系不好,现在合得来了}\end{exemple}\relationsémantique{同义词}{\lien{ⓔamɯmiⓗ2}{amɯmi}}\end{entrée}

\begin{entrée}{ɲɟɯr}{}{ⓔɲɟɯr} 
\classe{vi} \paradigme{dir}{nɯ-}\paradigme{dir}{tɤ-}
\begin{définition}\pfra{changer}\end{définition}
\begin{définition}\pcmn{变化}\end{définition}
\begin{exemple}\pjya{kutɕu kɤ-tɯ-nɤrʑaʁ tɕe nɤ-skɤt ɲɤ-ɲɟɯr}\hspace{5pt}\pcmn{你在这里待久了,你的语言就变了}\end{exemple}\étymologie{ⁿgʲur}\end{entrée}

\begin{entrée}{ɲɟɯrmbloʁ/\variante{nɤɲɟɯrmbloʁ}}{}{ⓔɲɟɯrmbloʁ} 
\classe{vi} \paradigme{dir}{tɤ-}
\begin{définition}\pfra{être instable}\end{définition}
\begin{définition}\pcmn{动摇不定;不稳定}\end{définition}
\begin{exemple}\pjya{ɯʑo kɯ-nɤɲɟɯrmbloʁ ci ɲɯ-ɕti tɕe, tɤ-mda tɕe tɕhi ɲɯ-fse mɤxsi}\hspace{5pt}\pcmn{他是个动摇不定的人,不知道最好会怎么样}\end{exemple}
\begin{exemple}\pjya{a-mɤ-tɯ-nɤɲɟɯrmbloʁ je!}\hspace{5pt}\pcmn{你不要动摇不定}\end{exemple}\relationsémantique{反义词}{\lien{ⓔarɤstoʁsta}{arɤstoʁsta}}\relationsémantique{参考}{\lien{ⓔɲɟɯr}{ɲɟɯr}}\end{entrée}

\begin{entrée}{ɲɟɯrnor}{}{ⓔɲɟɯrnor} 
\classe{n} 
\begin{définition}\pfra{erreur}\end{définition}
\begin{définition}\pcmn{错觉}\end{définition}\relationsémantique{参考}{\lien{ⓔsɤɲɟɯrnor}{sɤɲɟɯrnor}}\end{entrée}

\begin{entrée}{ɲo}{}{ⓔɲo} 
\classe{vs} \paradigme{dir}{tɤ-}
\begin{définition}\pfra{déjà préparé}\end{définition}
\begin{définition}\pcmn{现成}\end{définition}
\begin{exemple}\pjya{ji-saχsɯ to-ɲo}\hspace{5pt}\pcmn{我们的午餐已经准备好了}\end{exemple}
\begin{exemple}\pjya{tɤ-pɤri to-ɲo}\hspace{5pt}\pcmn{晚餐已经做好了}\end{exemple}
\begin{exemple}\pjya{kɯ-ɲo-ndza}\hspace{5pt}\pcmn{现成的食物}\end{exemple}
\begin{exemple}\pjya{kɯ-ɲo-ŋga}\hspace{5pt}\pcmn{现成的衣服}\end{exemple}\relationsémantique{参考}{\lien{ⓔsɯɣɲo}{sɯɣɲo}}\relationsémantique{参考}{\lien{ⓔmɲoⓗ1}{mɲo₁}}\relationsémantique{同义词}{\lien{ⓔndzu}{ndzu}}
\begin{sous-entrée}{nɯɣɲo}{ⓔɲoⓝnɯɣɲo} 
\classe{vt}  
\grammaire{appl} 
\begin{définition}\pfra{être certain que}\end{définition}
\begin{définition}\pcmn{觉得……一定会……}\end{définition}\end{sous-entrée}

\begin{exemple}\pjya{aʑo kɤ-βʁa nɯ kɤ-nɯɣɲo ɕti}\hspace{5pt}\pcmn{我注定会赢}\end{exemple}
\begin{exemple}\pjya{aʑo kɤ-βʁa nɯ kú-wɣ-nɯɣɲo-a-nɯ ɕti}\hspace{5pt}\pcmn{他们觉得我一定会赢}\end{exemple}\end{entrée}

\begin{entrée}{ɲur}{}{ⓔɲur} 
\classe{vs} \paradigme{dir}{nɯ-}
\begin{définition}\pfra{être fatigué}\end{définition}
\begin{définition}\pcmn{困倦}\end{définition}\end{entrée}

\begin{entrée}{ɲɯɣɲɯɣ}{}{ⓔɲɯɣɲɯɣ} 
\classe{idph.2} 
\begin{définition}\pfra{objets granulaires entassés}\end{définition}
\begin{définition}\pcmn{形容粉状的东西堆得很高、很软的样子}\end{définition}
\begin{exemple}\pjya{qajβɣi ɲɯɣɲɯɣ ʑo ɲɯ-ɤrmbɯ}\hspace{5pt}\pcmn{糠秕堆得很高}\end{exemple}
\begin{exemple}\pjya{ɲɯɣɲɯɣ ʑo ɲɯ-ɤta}\hspace{5pt}\pcmn{堆得很高}\end{exemple}
\begin{exemple}\pjya{qambɯt (tɤ-rɤku, tɯ-ɣli) ɲɯɣɲɯɣ ʑo ɲɯ-ɤrmbɯ}\hspace{5pt}\pcmn{沙子(庄稼、肥料)堆得很高}\end{exemple}
\begin{sous-entrée}{ɲɯɣnɤɲɯɣ}{ⓔɲɯɣɲɯɣⓝɲɯɣnɤɲɯɣ} 
\classe{idph.3} 
\begin{définition}\pfra{(se déplacer sur) un sol poudreux et mou}\end{définition}
\begin{définition}\pcmn{形容在泥沙里走动时很不方便(地很软)的样子}\end{définition}
\begin{exemple}\pjya{nɯ ɯ-taʁ tu-kɯ-ŋke tɕe, ɲɯɣnɤɲɯɣ ʑo ɲɯ-ti}\hspace{5pt}\pcmn{在那上面走动很不方便}\end{exemple}\relationsémantique{参考}{\lien{ⓔlɲɯɣlɲɯɣ}{lɲɯɣlɲɯɣ}}\end{sous-entrée}

\end{entrée}

\newpage\caractère{ŋ}

\begin{entrée}{ŋu}{}{ⓔŋu} 
\classe{vs} \paradigme{dir}{tɤ-}
\begin{définition}\pfra{être}\end{définition}
\begin{définition}\pcmn{是}\end{définition}
\begin{sous-entrée}{pɯpɯŋu nɤ}{ⓔŋuⓝpɯpɯŋu nɤ}
\begin{définition}\pfra{en ce qui concerne, à propos}\end{définition}
\begin{définition}\pcmn{至于,关于}\end{définition}\end{sous-entrée}

\begin{sous-entrée}{ŋu nɤ}{ⓔŋuⓝŋu nɤ}
\begin{définition}\pfra{en ce qui concerne, à propos}\end{définition}
\begin{définition}\pcmn{至于,关于}\end{définition}
\begin{exemple}\pjya{aʑo ŋu nɤ, mɯ-tu-kɯ-nɯkon-a-nɯ ɲɯ-ŋu}\hspace{5pt}\pcmn{我呢,你们都不关心我}\end{exemple}\end{sous-entrée}

\begin{sous-entrée}{wuma tɤ-ŋu tɕe}{ⓔŋuⓝwuma tɤ-ŋu tɕe}
\begin{définition}\pfra{en réalité}\end{définition}
\begin{définition}\pcmn{实际上}\end{définition}\end{sous-entrée}

\end{entrée}

\begin{entrée}{ŋa}{}{ⓔŋa} 
\classe{vt} \paradigme{dir}{kɤ-}
\begin{définition}\pfra{devoir de l'argent, faire un crédit}\end{définition}
\begin{définition}\pcmn{赊帐}\end{définition}
\begin{exemple}\pjya{kɤ-ŋa-t-a}\hspace{5pt}\pcmn{我欠了钱}\end{exemple}
\begin{exemple}\pjya{nɤ-rŋɯl kɤ-ŋa-t-a, ɯ-qhu tɕe ɲɯ-kham-a}\hspace{5pt}\pcmn{我欠了你钱,以后会还}\end{exemple}
\begin{exemple}\pjya{ɣnɤsqi ɲɯ-ra, sqɯ-mpɕar nɯ-kho-t-a, nɯma mɯ́j-rtaʁ tɕe sqɯ-mpɕar kɤ-ŋa-t-a}\hspace{5pt}\pcmn{原来需要二十块钱,我给了十块然后钱不够,我就赊了十块钱}\end{exemple}
\begin{exemple}\pjya{kɯ-rɤχtɯ jɤ-ari-a ri, a-rŋɯl mɯ́j-rtaʁ tɕe kɤ-ŋa-t-a, tɕe a-nŋa nɯ-ɬoʁ}\hspace{5pt}\pcmn{我去买东西钱不够就赊了账}\end{exemple}\relationsémantique{参考}{\lien{ⓔrɤnŋa}{rɤnŋa}}\relationsémantique{参考}{\lien{ⓔtɯ-nŋa}{tɯ-nŋa}}\end{entrée}

\begin{entrée}{ŋɤjɕtsa}{}{ⓔŋɤjɕtsa} 
\classe{n} 
\begin{définition}\pfra{chef de village}\end{définition}
\begin{définition}\pcmn{村长}\end{définition}\end{entrée}

\begin{entrée}{ŋɤlitɕaʁmbɯm}{}{ⓔŋɤlitɕaʁmbɯm} 
\classe{n} 
\begin{définition}\pfra{bousier}\end{définition}
\begin{définition}\pcmn{蜣螂【牛屎虫】}\end{définition}\end{entrée}

\begin{entrée}{ŋɤn}{}{ⓔŋɤn} 
\classe{vi} \paradigme{dir}{nɯ-}\sens{1}
\begin{définition}\pfra{mauvais}\end{définition}
\begin{définition}\pcmn{坏}\end{définition}\sens{2}
\begin{définition}\pfra{féroce}\end{définition}
\begin{définition}\pcmn{凶}\end{définition}
\begin{sous-entrée}{ɣɤŋɤn}{ⓔŋɤnⓝɣɤŋɤn} 
\classe{vt} \paradigme{dir}{tɤ-}
\begin{exemple}\pjya{ɯ-sɯm to-ɣɤŋɤn tɕe ɯʑo ɯ-mɤ-kɯ-pe to-nɯ-βzu}\hspace{5pt}\pcmn{他起了坏心,反而自食其果}\end{exemple}\end{sous-entrée}

\étymologie{ŋan}\end{entrée}

\begin{entrée}{ŋɤnŋɤt}{}{ⓔŋɤnŋɤt} 
\classe{idph.2} 
\begin{définition}\pfra{la tête vers le bas}\end{définition}
\begin{définition}\pcmn{形容低头的样子}\end{définition}
\begin{exemple}\pjya{pjɤ-nɯʑɯβ tɕe ɯ-ku ŋɤnŋɤt ʑo pjɤ-phɤβ}\hspace{5pt}\pcmn{他低着头睡着了}\end{exemple}\relationsémantique{反义词}{\lien{ⓔŋɤrŋɤr}{ŋɤrŋɤr}}\end{entrée}

\begin{entrée}{ŋɤnɯ}{}{ⓔŋɤnɯ} 
\classe{n} 
\begin{définition}\pfra{pis de la vache}\end{définition}
\begin{définition}\pcmn{奶牛的乳房}\end{définition}\end{entrée}

\begin{entrée}{ŋɤnɯkɯmtsɯɣ}{}{ⓔŋɤnɯkɯmtsɯɣ} 
\classe{n} 
\begin{définition}\pfra{une espèce d'arbrisseau}\end{définition}
\begin{définition}\pcmn{【红青椒】}\end{définition}
\begin{exemple}\pjya{ŋɤnɯ kɯmtsɯɣ nɯ ruŋgu sɯŋgɯ arɤndɯndɤt ʑo tu-ɬoʁ ɕti, ɯ-zrɤm wuma ʑo wxti, kɯ-ɣɯrni ŋu. smɤnrɯɣ ci ɲɯ-ŋu. ɯ-ru nɯ kɯ-ɤβʑɯrdu ŋu, ɯ-jwaʁ kɯ-ɤβzɯrχsɯm ŋu, kɯ-pɣi tsa ŋu, ɯ-qhu nɯ ɯ-rme kɯ-fse tu, ɯ-ru kɯ-zri tsa tu-ɬoʁ tɕe, ɯ-rtaʁ ɲɯ-ɬoʁ ŋu, ɯ-mɯntoʁ nɯ ɯ-ru ɯ-taʁ lu-oʑɯrja ŋu, kɯ-ɣɯrni tsa ŋu. ɣʑo wuma ʑo rga.}\hspace{5pt}\pcmn{红青椒到处都可以生长,包括草坪里和森林里。根长得很大,是红色的,是一种药材。茎是四方形的,叶子是三角形的,呈灰色,背面有细毛。茎长了以后就会分杈,花排列在茎上,是淡红色的。蜜蜂喜欢这种植物。}\end{exemple}\end{entrée}

\begin{entrée}{ŋɤqa}{}{ⓔŋɤqa} 
\classe{n} 
\begin{définition}\pfra{une espèce de champignon}\end{définition}
\begin{définition}\pcmn{一种蘑菇}\end{définition}
\begin{exemple}\pjya{ŋɤqa nɯ stɤmku ri tu-ɬoʁ ŋu tɕe, ɯ-mgɯr ɯ-qhu nɯ kɯ-wɣrum ŋu, ɯ-rʑɯɣ nɯ kɯ-ɤɣɯrnɯɕɯr ŋu, ɯ-ru nɯ li kɯ-wɣrum ŋu, kɤ-ndza mɯm, phaʁzla ɯ-ŋgɯ tu-ɬoʁ ŋu tɕe phazla ŋɤqa kɤ-ti tu.}\hspace{5pt}\pcmn{\lien{ⓔŋɤqa}{ŋɤqa} 长在草地上,背面和干是白色的,菌褶带有红色,好吃。长在五月份,所以叫“五月\lien{ⓔŋɤqa}{ŋɤqa}”。}\end{exemple}\end{entrée}

\begin{entrée}{ŋɤqe}{}{ⓔŋɤqe} 
\classe{n} 
\begin{définition}\pfra{bouse de vache}\end{définition}
\begin{définition}\pcmn{牛粪}\end{définition}\end{entrée}

\begin{entrée}{ŋɤrŋɤr}{}{ⓔŋɤrŋɤr} 
\classe{idph.2} 
\begin{définition}\pfra{qui tend le cou}\end{définition}
\begin{définition}\pcmn{形容伸脖子的样子}\end{définition}
\begin{exemple}\pjya{ɯ-ku ŋɤrŋɤr ʑo to-joʁ}\hspace{5pt}\pcmn{他伸了脖子(看)}\end{exemple}
\begin{exemple}\pjya{ɯ-ku ŋɤrŋɤr ʑo chɤ-tɕɤt}\hspace{5pt}\pcmn{他把头(从窗户)探出来伸着脖子看}\end{exemple}\relationsémantique{反义词}{\lien{ⓔŋɤnŋɤt}{ŋɤnŋɤt}}
\begin{sous-entrée}{ŋɤrnɤŋɤr}{ⓔŋɤrŋɤrⓝŋɤrnɤŋɤr} 
\classe{idph.2} \end{sous-entrée}

\end{entrée}

\begin{entrée}{ŋɤtɕɯkɤti,khɯ}{}{ⓔŋɤtɕɯkɤti,khɯ} 
\classe{n}
\classe{vs} \paradigme{dir}{tɤ-}
\begin{définition}\pfra{obéir en tous points}\end{définition}
\begin{définition}\pcmn{完全顺从}\end{définition}
\begin{exemple}\pjya{ŋɤtɕɯkɤti a-tɤ-tɯ-khɯ ra}\hspace{5pt}\pcmn{你要完全顺从吩咐}\end{exemple}
\begin{exemple}\pjya{ŋɤtɕɯkɤti tu-sɯkhi-a ra}\hspace{5pt}\pcmn{我要令他完全顺从}\end{exemple}\relationsémantique{Component 1}{\lien{}{ŋɤtɕɯkɤti}}\relationsémantique{Component 2}{\lien{ⓔkhɯⓗ2}{khɯ}}\relationsémantique{参考}{\lien{ⓔŋotɕu}{ŋotɕu}}\end{entrée}

\begin{entrée}{ŋga}{}{ⓔŋga} 
\classe{vt} \paradigme{dir}{tɤ-}\paradigme{dir}{tɤ-}
\begin{définition}\pfra{mettre (un vêtement)}\end{définition}
\begin{définition}\pcmn{穿衣服}\end{définition}
\begin{définition}\pfra{mettre n'importe quel habit}\end{définition}
\begin{définition}\pcmn{随便穿}\end{définition}
\begin{exemple}\pjya{tɤ-rte tɤ-ŋge}\hspace{5pt}\pcmn{你戴上帽子}\end{exemple}
\begin{exemple}\pjya{tɯ-ŋga pɯ-ŋga-t-a}\hspace{5pt}\pcmn{(睡觉的时候)我把衣服盖在身上了}\end{exemple}
\begin{exemple}\pjya{tɯrme ɯ-ŋga ra ma-tɯ-nɤŋgɯŋge ma nɤʑɯɣ tɤ-nɯ-ŋge}\hspace{5pt}\pcmn{不要随便穿别人的衣服,穿自己的}\end{exemple}\relationsémantique{参考}{\lien{ⓔʑŋga}{ʑŋga}}
\begin{sous-entrée}{nɯɣɯŋga}{ⓔŋgaⓝnɯɣɯŋga} 
\classe{vs}  
\grammaire{facil} 
\begin{définition}\pfra{facile, agréable à mettre (habit)}\end{définition}
\begin{définition}\pcmn{容易穿;穿着舒服}\end{définition}
\begin{exemple}\pjya{kɯki a-ŋga ki wuma ʑo ɲɯ-nɯɣɯŋga}\hspace{5pt}\pcmn{我这件衣服穿着很舒服}\end{exemple}
\begin{exemple}\pjya{nɤki nɯ ɯ-ɲɯ́-nɯɣɯŋga?}\hspace{5pt}\pcmn{你那件好穿吗?}\end{exemple}\end{sous-entrée}

\begin{sous-entrée}{nɤŋgɯŋga}{ⓔŋgaⓝnɤŋgɯŋga} 
\classe{vt} \end{sous-entrée}

\end{entrée}

\begin{entrée}{ŋgɤɣ}{}{ⓔŋgɤɣ} 
\classe{vi.nh}  
\grammaire{acaus} \paradigme{dir}{\_}
\begin{définition}\pfra{courbé}\end{définition}
\begin{définition}\pcmn{弯(树)}\end{définition}
\begin{exemple}\pjya{tɯɲcɣa chɯ-ŋgɤɣ ɲɯ-ŋu}\hspace{5pt}\pcmn{镰刀是弯的}\end{exemple}
\begin{exemple}\pjya{ɕɤmiŋoʁ chɯ-ŋgɤɣ ɲɯ-ŋu}\hspace{5pt}\pcmn{铁钩是弯的}\end{exemple}
\begin{exemple}\pjya{paχɕi ɲɤ-ɣɤmat tɕe, ɯ-rtaʁ ra pjɤ-ŋgɤɣ ʑo}\hspace{5pt}\pcmn{因为苹果熟了,导致树枝弯下来了}\end{exemple}\relationsémantique{参考}{\lien{ⓔkɤɣ}{kɤɣ}}\end{entrée}

\begin{entrée}{ŋgɤjpɤn}{}{ⓔŋgɤjpɤn} 
\classe{n} 
\begin{définition}\pfra{planche de bois}\end{définition}
\begin{définition}\pcmn{木板}\end{définition}
\begin{exemple}\pjya{ŋgɤjpɤn nɯ rɟaŋsoʁ kɯ thɯ-kɤ-sɯ-phaʁ ŋu. tɤrɤm nɯ tɯ-rpa kɯ ɯ-rɯmu thɯ-kɤ-z-nɯɴqhu tɕe thɯ-kɤ-phaʁ ŋu.}\hspace{5pt}\pcmn{\lien{ⓔŋgɤjpɤn}{ŋgɤjpɤn}是用锯子锯成的板子,\lien{ⓔtɤrɤm}{tɤrɤm}是用斧头顺着纹路劈成的。}\end{exemple}\end{entrée}

\begin{entrée}{ŋgɤlɤʁɟa}{}{ⓔŋgɤlɤʁɟa} 
\classe{n} 
\begin{définition}\pfra{chauve}\end{définition}
\begin{définition}\pcmn{秃子}\end{définition}\end{entrée}

\begin{entrée}{ŋgɤm}{}{ⓔŋgɤm} 
\classe{n} 
\begin{définition}\pfra{pente de terre (à 90 degrés)}\end{définition}
\begin{définition}\pcmn{(垂直)的土坡【土崖】}\end{définition}\end{entrée}

\begin{entrée}{ŋgɤr}{}{ⓔŋgɤr} 
\classe{vs}  
\grammaire{trop} \paradigme{dir}{kɤ-}
\begin{définition}\pfra{étroit}\end{définition}
\begin{définition}\pcmn{狭窄}\end{définition}
\begin{exemple}\pjya{kɯspoʁ ɲɯ-ŋgɤr}\hspace{5pt}\pcmn{洞很狭窄}\end{exemple}
\begin{exemple}\pjya{sɤxɕe ɲɯ-ŋgɤr}\hspace{5pt}\pcmn{去的地方很狭窄}\end{exemple}
\begin{exemple}\pjya{nɤ-ro ɯ-tɯ-ŋgɤr}\hspace{5pt}\pcmn{你很小气}\end{exemple}\relationsémantique{反义词}{\lien{ⓔjom}{jom}}\relationsémantique{参考}{\lien{ⓔaŋgɤrŋgɤr}{aŋgɤrŋgɤr}}\relationsémantique{参考}{\lien{}{znɤngɤr}}\relationsémantique{参考}{\lien{}{vt}}
\begin{sous-entrée}{nɤŋgɤr}{ⓔŋgɤrⓝnɤŋgɤr}\end{sous-entrée}

\begin{définition}\pfra{trouver trop étroit}\end{définition}
\begin{définition}\pcmn{觉得狭窄}\end{définition}\end{entrée}

\begin{entrée}{ŋgɤrom}{}{ⓔŋgɤrom} 
\classe{n} 
\begin{définition}\pfra{méthode de tissage}\end{définition}
\begin{définition}\pcmn{织布的方法,两根线交错着【单巴子】}\end{définition}\end{entrée}

\begin{entrée}{ŋgio}{}{ⓔŋgio} 
\classe{vi}  
\grammaire{acaus} \paradigme{dir}{pɯ-}
\begin{définition}\pfra{glisser}\end{définition}
\begin{définition}\pcmn{(从高处一直)滑下来}\end{définition}
\begin{exemple}\pjya{pɯ-ŋgio}\hspace{5pt}\pcmn{他滑下来了}\end{exemple}
\begin{exemple}\pjya{pɯ-ŋgio-a}\hspace{5pt}\pcmn{我滑下来了}\end{exemple}
\begin{exemple}\pjya{ɕoŋtɕa pɯ-ŋgio}\hspace{5pt}\pcmn{木料滑下来了}\end{exemple}
\begin{exemple}\pjya{rdɤstaʁ pjɤ-ŋgio}\hspace{5pt}\pcmn{石头滑下来了}\end{exemple}
\begin{sous-entrée}{nɤŋgiolo}{ⓔŋgioⓝnɤŋgiolo} 
\classe{vi}  
\grammaire{n.orient} 
\begin{définition}\pfra{glisser dans tous les sens}\end{définition}
\begin{définition}\pcmn{滑来滑去}\end{définition}\relationsémantique{参考}{\lien{ⓔkio}{kio}}\relationsémantique{同义词}{\lien{ⓔaʁdɤt}{aʁdɤt}}\end{sous-entrée}

\end{entrée}

\begin{entrée}{ŋgumdʑɯɣ}{}{ⓔŋgumdʑɯɣ} 
\classe{n} 
\begin{définition}\pfra{chef}\end{définition}
\begin{définition}\pcmn{领导}\end{définition}
\begin{exemple}\pjya{aʑo nɯ-ŋgumdʑɯɣ ŋu-a}\hspace{5pt}\pcmn{我是你们的领导}\end{exemple}\relationsémantique{参考}{\lien{}{nɯŋgɯdʑɯɣ}}
\begin{sous-entrée}{ŋgumdʑɯxpa}{ⓔŋgumdʑɯɣⓝŋgumdʑɯxpa} 
\grammaire{n} 
\begin{définition}\pfra{chef}\end{définition}
\begin{définition}\pcmn{领导人}\end{définition}
\begin{exemple}\pjya{ŋgumdʑɯxpa to-ndo}\hspace{5pt}\pcmn{他当了领导}\end{exemple}\end{sous-entrée}

\étymologie{ⁿgo.mdʑug}\end{entrée}

\begin{entrée}{ŋgumdʑɯxpa}{}{ⓔŋgumdʑɯxpa}\relationsémantique{参考}{\lien{ⓔŋgumdʑɯɣ}{ŋgumdʑɯɣ}}\end{entrée}

\begin{entrée}{ŋgo}{}{ⓔŋgo} 
\classe{vi} \paradigme{dir}{tɤ-}
\begin{définition}\pfra{cuire (les momos)}\end{définition}
\begin{définition}\pcmn{烤熟(馍馍)}\end{définition}
\begin{exemple}\pjya{qajɣi to-ŋgo}\hspace{5pt}\pcmn{馍馍烤熟了}\end{exemple}
\begin{sous-entrée}{sɯŋgo}{ⓔŋgoⓝsɯŋgo} 
\classe{vt} 
\begin{exemple}\pjya{qajɣi tɤ-sɯŋgo-t-a}\hspace{5pt}\pcmn{我把馍馍烤熟了}\end{exemple}\end{sous-entrée}

\end{entrée}

\begin{entrée}{ŋgoŋpu}{}{ⓔŋgoŋpu} 
\classe{n} 
\begin{définition}\pfra{désastre}\end{définition}
\begin{définition}\pcmn{祸事}\end{définition}\étymologie{ⁿgoŋ.po}\end{entrée}

\begin{entrée}{ŋgorli}{}{ⓔŋgorli} 
\classe{n} 
\begin{définition}\pfra{bovidé sans corne}\end{définition}
\begin{définition}\pcmn{无角牛}\end{définition}\end{entrée}

\begin{entrée}{ŋgra}{}{ⓔŋgra} 
\classe{vi}  
\grammaire{acaus} \paradigme{dir}{pɯ-}
\begin{définition}\pfra{tomber}\end{définition}
\begin{définition}\pcmn{掉下来}\end{définition}
\begin{exemple}\pjya{ʑɴɢɯloʁ pjɤ-ŋgra}\hspace{5pt}\pcmn{核桃掉下来了}\end{exemple}
\begin{exemple}\pjya{sɯmat pjɤ-ŋgra}\hspace{5pt}\pcmn{水果掉下来了}\end{exemple}
\begin{exemple}\pjya{rdɤstaʁ pjɤ-ŋgra}\hspace{5pt}\pcmn{石头掉下来了}\end{exemple}
\begin{exemple}\pjya{tɤɕi ɲɯ-ŋgra}\hspace{5pt}\pcmn{青稞颗粒掉落}\end{exemple}
\begin{exemple}\pjya{qaj ɯ-tɯ-mda thɯ-tɕhom tɕe kɤ́-wɣ-tɣa tɕe pjɯ-ŋgra ɕti}\hspace{5pt}\pcmn{小麦太熟了,收割的时候掉到地面上}\end{exemple}\relationsémantique{参考}{\lien{ⓔkra}{kra}}\end{entrée}

\begin{entrée}{ŋgrɤl}{}{ⓔŋgrɤl} 
\classe{vs} \paradigme{dir}{tɤ-}
\begin{définition}\pfra{être habituellement ainsi}\end{définition}
\begin{définition}\pcmn{(平时)是这样}\end{définition}
\begin{sous-entrée}{sɯŋgrɤl}{ⓔŋgrɤlⓝsɯŋgrɤl} 
\classe{vt}  
\grammaire{caus} \end{sous-entrée}

\begin{sous-entrée}{zrɯŋgrɤl}{ⓔŋgrɤlⓝzrɯŋgrɤl} 
\classe{vt}  
\grammaire{caus} 
\begin{définition}\pfra{changer de}\end{définition}
\begin{définition}\pcmn{改用}\end{définition}
\begin{exemple}\pjya{kɯɕɯŋgɯ tɕe tɯ-ci kɯ βɣa chɯ-sɯmtɕɯr-i tɕe kɤ-ɣndʑɯr chɯ-sɯ-lɤt-i pɯ-ŋu ri, tham tɕe mkhɯrlu kɯ kɤ-sɯɣndʑɯr nɯ-zrɯŋgrɤl-i}\hspace{5pt}\pcmn{我以前用水力推磨,现在改用机器磨面}\end{exemple}\end{sous-entrée}

\end{entrée}

\begin{entrée}{ŋgri}{}{ⓔŋgri} 
\classe{vs} \paradigme{dir}{nɯ-}
\begin{définition}\pfra{fin (gruau)}\end{définition}
\begin{définition}\pcmn{稀(粥)}\end{définition}
\begin{exemple}\pjya{tɯtshi ɲɯ-ŋgri}\hspace{5pt}\pcmn{粥很稀}\end{exemple}\relationsémantique{反义词}{\lien{ⓔndzɤβ}{ndzɤβ}}\end{entrée}

\begin{entrée}{ŋgro}{}{ⓔŋgro} 
\classe{vs} \paradigme{dir}{thɯ-}
\begin{définition}\pfra{puissant, important, honorable}\end{définition}
\begin{définition}\pcmn{有权有势;有地位;值得尊重的人}\end{définition}
\begin{exemple}\pjya{βlama ɲɯ-ŋgro}\hspace{5pt}\pcmn{喇嘛很受尊重}\end{exemple}\relationsémantique{同义词}{\lien{ⓔɣɤʁre}{ɣɤʁre}}\end{entrée}

\begin{entrée}{ŋgurtɕaʁ}{}{ⓔŋgurtɕaʁ} 
\classe{n} 
\begin{définition}\pfra{type de pas d'aiguille}\end{définition}
\begin{définition}\pcmn{缝针的方法}\end{définition}\relationsémantique{参考}{\lien{ⓔnɯŋgurtɕaʁ}{nɯŋgurtɕaʁ}}\end{entrée}

\begin{entrée}{ŋgrɯ}{}{ⓔŋgrɯ} 
\classe{vi} \paradigme{dir}{pɯ-}
\begin{définition}\pfra{accomplir}\end{définition}
\begin{définition}\pcmn{成功;完成}\end{définition}
\begin{exemple}\pjya{jisŋi ji-sɯphɯt pɯ-ŋgrɯ}\hspace{5pt}\pcmn{今天我们砍柴的任务完成了}\end{exemple}
\begin{exemple}\pjya{jisŋi ji-ma pɯ-ŋgrɯ}\hspace{5pt}\pcmn{今天我们的工作完成了}\end{exemple}
\begin{exemple}\pjya{qartsɤβ pɯ-ŋgrɯ}\hspace{5pt}\pcmn{收割完成了}\end{exemple}
\begin{exemple}\pjya{aʑɯɣ pɯ-ŋgrɯ}\hspace{5pt}\pcmn{我成功了}\end{exemple}
\begin{exemple}\pjya{aʑo a-kɤ-sɯso nɯ pɯ-ŋgrɯ}\hspace{5pt}\pcmn{我的愿望实现了}\end{exemple}
\begin{exemple}\pjya{aʑo a-kɤ-nɤma nɯ pɯ-ŋgrɯ}\hspace{5pt}\pcmn{我的工作完成了}\end{exemple}\relationsémantique{同义词}{\lien{ⓔngrɯβ}{ngrɯβ}}\relationsémantique{参考}{\lien{}{sɯŋgrɯ}}\étymologie{ⁿgrub}\end{entrée}

\begin{entrée}{ŋguskor}{}{ⓔŋguskor} 
\classe{n} 
\begin{définition}\pfra{fouet}\end{définition}
\begin{définition}\pcmn{皮鞭(打猪用的)}\end{définition}\end{entrée}

\begin{entrée}{ŋgute}{}{ⓔŋgute} 
\classe{np} 
\begin{définition}\pfra{qui a une grande tête}\end{définition}
\begin{définition}\pcmn{大头}\end{définition}\end{entrée}

\begin{entrée}{ŋgɯ}{}{ⓔŋgɯ} 
\classe{vs} \paradigme{dir}{nɯ-}
\begin{définition}\pfra{pauvre}\end{définition}
\begin{définition}\pcmn{穷}\end{définition}
\begin{exemple}\pjya{jiɕqha nɯ kɯ-ŋgɯ ci ɲɯ-ŋu}\hspace{5pt}\pcmn{他是个穷人}\end{exemple}\end{entrée}

\begin{entrée}{ŋgɯŋgri/\variante{ŋgringri}}{}{ⓔŋgɯŋgri} 
\classe{idph.2} 
\begin{définition}\pfra{long et dur, qui ne casse pas facilement}\end{définition}
\begin{définition}\pcmn{形容硬而细,不易折断的样子}\end{définition}\end{entrée}

\begin{entrée}{ŋgɯr}{₁}{ⓔŋgɯrⓗ1} 
\classe{n} 
\begin{définition}\pfra{chant mystique}\end{définition}
\begin{définition}\pcmn{道情}\end{définition}\étymologie{mgur}\end{entrée}

\begin{entrée}{ŋgɯr}{₂}{ⓔŋgɯrⓗ2} 
\classe{idph.2} 
\begin{définition}\pfra{bruit du canon}\end{définition}
\begin{définition}\pcmn{放炮的声音}\end{définition}\relationsémantique{参考}{\lien{ⓔɣɤŋgɯrŋgɯr}{ɣɤŋgɯrŋgɯr}}\end{entrée}

\begin{entrée}{ŋgɯrŋgɯr}{}{ⓔŋgɯrŋgɯr} 
\classe{idph.2} 
\begin{définition}\pfra{large et profonde (étendue d'eau)}\end{définition}
\begin{définition}\pcmn{形容水面很宽,水很深的样子}\end{définition}
\begin{exemple}\pjya{tɯ-ci ŋgɯrŋgɯr ʑo ɲɯ-pa}\hspace{5pt}\pcmn{水很深}\end{exemple}
\begin{exemple}\pjya{cha tɯ-khɯtsa ʑo ŋgɯrŋgɯr kɤ-tshi-t-a}\hspace{5pt}\pcmn{我喝了满满的一碗酒}\end{exemple}\end{entrée}

\begin{entrée}{ŋgɯsɯ}{}{ⓔŋgɯsɯ} 
\classe{n} 
\begin{définition}\pfra{Adenophora sp.}\end{définition}
\begin{définition}\pcmn{沙参}\end{définition}
\begin{exemple}\pjya{ŋgɯsɯ tʂu ɯ-rkɯ tɯ-ji ɯ-rkɯ zgoku pɕoʁ ra tu-ɬoʁ ŋu, ɯ-qa nɯ smɤn ɲɯ-ŋu khi, ɯ-ŋgɯ nɯ kɯ-wɣrum ŋu, ɯ-pɕi nɯ kɯ-pɣi tsa ŋu, ɯ-zrɤm nɯ mɤ-ndoʁ, ɯ-ru nɯ kɯ-xtshɯm kɯ-zri tsa ŋu, kɯ-qandʐi tsa ŋu, ɯ-jwaʁ nɯ kɯ-ɤrtɯm tɕe kɯ-ndɯβ tsa ŋu, mɤʑɯ tɯ-tɯphu nɯ ɯ-jwaʁ kɯ-tɕɤr tɕe kɯ-ɤmtɕoʁ tsa ŋu, ɯ-mɯntoʁ nɯ tshaŋlaŋ kɯ-fse tɕe ɯ-mdoʁ lɯŋkɤr ŋu tɕɤn ɯ-ru ɯ-taʁ lu-oʑɯrja ŋu.}\hspace{5pt}\pcmn{\lien{ⓔŋgɯsɯ}{ŋgɯsɯ}生长在地边,路边和山上,听说它的根是药材,根里面是白色的,外面是土灰色的,不脆。茎细长,乌色。一种\lien{ⓔŋgɯsɯ}{ŋgɯsɯ}有小而圆的叶子,而另一种有窄而尖的叶子。两种\lien{ⓔŋgɯsɯ}{ŋgɯsɯ}的花像铃铛,天蓝色,排列在茎的顶端上。}\end{exemple}\end{entrée}

\begin{entrée}{ŋkɤɲɟo}{}{ⓔŋkɤɲɟo} 
\classe{vi} \paradigme{dir}{pɯ-}
\begin{définition}\pfra{passer}\end{définition}
\begin{définition}\pcmn{路过;来往}\end{définition}
\begin{exemple}\pjya{ɯʑo pɯ-ŋkɤɲɟo}\hspace{5pt}\pcmn{他路过(这里)}\end{exemple}\end{entrée}

\begin{entrée}{ŋke}{}{ⓔŋke} 
\classe{vi} \paradigme{dir}{\_}\paradigme{dir}{caus}
\begin{définition}\pfra{marcher}\end{définition}
\begin{définition}\pcmn{走路}\end{définition}
\begin{exemple}\pjya{kɤ-ŋke kɤ-ari-a}\hspace{5pt}\pcmn{我走路去了}\end{exemple}
\begin{exemple}\pjya{pɯ-ŋke-a}\hspace{5pt}\pcmn{我走了一下}\end{exemple}
\begin{exemple}\pjya{ɯʑo kɤ-ŋke kɤ-anɯri}\hspace{5pt}\pcmn{他走路回去了}\end{exemple}
\begin{exemple}\pjya{aʑo nɯ-nɯ-ŋke-a, mkhɯrlu pɯ-me}\hspace{5pt}\pcmn{我走路去了,没有车}\end{exemple}
\begin{exemple}\pjya{@yangma tɤ-me tɕe, kɤ-nɯ-ŋke ɬoʁ}\hspace{5pt}\pcmn{没有自行车的时候只好自己走}\end{exemple}
\begin{exemple}\pjya{nɯɕɯŋgɯ tɕiʑo lɤ-ari-tɕi nɯ tɤ-ŋke-tɕi pɯ-ra ma tham tɕe @qiche ɯ-ŋgu tu-kɯ-ɕe khɯ}\hspace{5pt}\pcmn{以前我们去的时候必须走路,现在可以坐车}\end{exemple}
\begin{exemple}\pjya{sɤŋke mɯ́j-pe}\hspace{5pt}\pcmn{不好走}\end{exemple}
\begin{exemple}\pjya{@dian ko-znɯna tɕe @dianti kɯnɤ mɯ́j-ŋke}\hspace{5pt}\pcmn{停电了,电梯也不走}\end{exemple}
\begin{sous-entrée}{ɕɯŋke}{ⓔŋkeⓝɕɯŋke} 
\classe{vt} \end{sous-entrée}

\sens{1}
\begin{définition}\pfra{faire marcher}\end{définition}
\begin{définition}\pcmn{使走动}\end{définition}\sens{2}
\begin{définition}\pfra{emporter}\end{définition}
\begin{définition}\pcmn{带走}\end{définition}
\begin{exemple}\pjya{aʑo tɤ-pɤtso tu-fkur-a tɕe tu-ɕɯŋke-a ŋu}\hspace{5pt}\pcmn{我背着小孩子走动}\end{exemple}\relationsémantique{参考}{\lien{ⓔnɯɣɯŋke}{nɯɣɯŋke}}\relationsémantique{参考}{\lien{ⓔnɯŋke}{nɯŋke}}\end{entrée}

\begin{entrée}{ŋkhor}{}{ⓔŋkhor} 
\classe{vi} \paradigme{dir}{\_}
\begin{définition}\pfra{se rapprocher (animal)}\end{définition}
\begin{définition}\pcmn{慢慢地接近(动物)}\end{définition}
\begin{exemple}\pjya{kɯki khɯna ki ɕɤfɕo aʑo a-phe ku-ŋkhor ɲɯ-ŋu}\hspace{5pt}\pcmn{这几天这条狗开始接近我了}\end{exemple}\étymologie{ⁿkʰor}\end{entrée}

\begin{entrée}{ŋkhorwapa}{}{ⓔŋkhorwapa} 
\classe{n} 
\begin{définition}\pfra{paysan}\end{définition}
\begin{définition}\pcmn{农民}\end{définition}\étymologie{ⁿkʰor.ba.pa}\end{entrée}

\begin{entrée}{ŋkhrɯl}{}{ⓔŋkhrɯl} 
\classe{vi} \paradigme{dir}{tɤ-}\sens{1}
\begin{définition}\pfra{se desserrer}\end{définition}
\begin{définition}\pcmn{变松(螺丝;门)}\end{définition}
\begin{exemple}\pjya{ɯ-ŋkhrɯli to-ŋkhrɯl}\hspace{5pt}\pcmn{螺丝松了}\end{exemple}
\begin{exemple}\pjya{ɯ-kɯm to-ŋkhrɯl kɤ-nɤcɯpa mɯ́j-khɯ}\hspace{5pt}\pcmn{门变松,开关都不方便了}\end{exemple}\sens{2}
\begin{définition}\pfra{fléchir (résolution)}\end{définition}
\begin{définition}\pcmn{动摇(决心)}\end{définition}
\begin{exemple}\pjya{ɯ-sɯm to-ŋkhrɯl (=ɲɤ-ɲɟɯr)}\hspace{5pt}\pcmn{他开始动摇了}\end{exemple}\end{entrée}

\begin{entrée}{ŋkhrɯli}{}{ⓔŋkhrɯli} 
\classe{n} 
\begin{définition}\pfra{vis}\end{définition}
\begin{définition}\pcmn{螺丝}\end{définition}
\begin{exemple}\pjya{ŋkhrɯli tɤ-spra-t-a}\hspace{5pt}\pcmn{我拧了螺丝}\end{exemple}\end{entrée}

\begin{entrée}{ŋoj}{}{ⓔŋoj} 
\classe{pro} 
\begin{définition}\pfra{où}\end{définition}
\begin{définition}\pcmn{哪里}\end{définition}
\begin{exemple}\pjya{nɤʑo ŋoj ku-tɯ-rɤʑi?}\hspace{5pt}\pcmn{你在哪里?}\end{exemple}
\begin{exemple}\pjya{ŋoj nɯ-ari ma mɯ-ɲɤ-mto-t-a}\hspace{5pt}\pcmn{(书包)丢到哪里了,我看不到}\end{exemple}
\begin{exemple}\pjya{ŋoj tɯ-ɕe}\hspace{5pt}\pcmn{你去哪里?}\end{exemple}\relationsémantique{参考}{\lien{ⓔŋotɕu}{ŋotɕu}}\end{entrée}

\begin{entrée}{ŋoʁ}{}{ⓔŋoʁ} 
\classe{n} 
\begin{définition}\pfra{crochet (pour attacher les vêtements)}\end{définition}
\begin{définition}\pcmn{用来勾住披衫(雨衣)的铁钩}\end{définition}\end{entrée}

\begin{entrée}{ŋotɕu}{}{ⓔŋotɕu} 
\classe{pro} 
\begin{définition}\pfra{où}\end{définition}
\begin{définition}\pcmn{哪里}\end{définition}
\begin{exemple}\pjya{ŋotɕu ku-tɯ-rɤʑi?}\hspace{5pt}\pcmn{你在哪里?}\end{exemple}
\begin{exemple}\pjya{ŋotɕu tɯ-ɕe?}\hspace{5pt}\pcmn{你去哪儿?}\end{exemple}
\begin{exemple}\pjya{kɯki ŋotɕu ɲɯ-ŋgrɤl?}\hspace{5pt}\pcmn{怎么可以这样?}\end{exemple}
\begin{exemple}\pjya{ŋotɕu chiz ku-tɯ-nɯ-rɤʑi kɯ?}\hspace{5pt}\pcmn{不知道你住在哪个地方?}\end{exemple}
\begin{exemple}\pjya{ŋotɕu sɤtɕha ɣɯ ɯ-tɯrme nɯ pɯ-nnɯ-ŋɯ-ŋu ʑo khɯ}\hspace{5pt}\pcmn{什么地方的人都可以}\end{exemple}\relationsémantique{参考}{\lien{ⓔŋoj}{ŋoj}}\end{entrée}

\begin{entrée}{ŋotɕuŋondɤt}{}{ⓔŋotɕuŋondɤt} 
\classe{adv} 
\begin{définition}\pfra{partout}\end{définition}
\begin{définition}\pcmn{到处}\end{définition}
\begin{exemple}\pjya{ŋotɕuŋondɤt ʑo tu}\hspace{5pt}\pcmn{到处都有}\end{exemple}\relationsémantique{同义词}{\lien{ⓔaʁɤndɯndɤt}{aʁɤndɯndɤt}}\relationsémantique{参考}{\lien{ⓔŋotɕu}{ŋotɕu}}\relationsémantique{参考}{\lien{ⓔɕɯⓝɕɯmɤɕɯ}{ɕɯmɤɕɯ}}\relationsémantique{参考}{\lien{ⓔnɤndɯndɤt}{nɤndɯndɤt}}\end{entrée}

\newpage\caractère{ɴ}

\begin{entrée}{ɴɢu}{}{ⓔɴɢu} 
\classe{vs} \paradigme{dir}{nɯ-}\paradigme{dir}{nɯ-}
\begin{définition}\pfra{relâché}\end{définition}
\begin{définition}\pcmn{松}\end{définition}
\begin{définition}\pfra{desserrer}\end{définition}
\begin{définition}\pcmn{松开,放开}\end{définition}
\begin{exemple}\pjya{ta-ma ɲɯ-ɴɢu}\hspace{5pt}\pcmn{工作很轻松}\end{exemple}
\begin{exemple}\pjya{ki ɯ-xtɕɤr ki ɲɯ-ɴɢu}\hspace{5pt}\pcmn{系得很松}\end{exemple}
\begin{exemple}\pjya{ɯ-sɤ-xtɕɤr ɲɯ-ɴɢu}\hspace{5pt}\pcmn{系得很松}\end{exemple}
\begin{exemple}\pjya{tɤ-mtɯ ɲɯ-ɴɢu}\hspace{5pt}\pcmn{结很松}\end{exemple}
\begin{exemple}\pjya{nɤki tɤ-mtɯ ɯ-tɯ-ɤsɯɣ ɲɯ-tɕhom tɕe, ɲo-ɣɤɴɢu}\hspace{5pt}\pcmn{那个结太紧了,他把它松了一下}\end{exemple}\relationsémantique{反义词}{\lien{ⓔasɯɣ}{asɯɣ}}\relationsémantique{参考}{\lien{ⓔɴɢule}{ɴɢule}}
\begin{sous-entrée}{ɣɤɴɢu}{ⓔɴɢuⓝɣɤɴɢu}\end{sous-entrée}

\end{entrée}

\begin{entrée}{ɴɢarmɯ}{}{ⓔɴɢarmɯ} 
\classe{n} 
\begin{définition}\pfra{vache bâtarde}\end{définition}
\begin{définition}\pcmn{母杂种牛}\end{définition}\end{entrée}

\begin{entrée}{ɴɢarpa}{}{ⓔɴɢarpa} 
\classe{n} 
\begin{définition}\pfra{bœuf bâtard}\end{définition}
\begin{définition}\pcmn{杂种牛}\end{définition}\end{entrée}

\begin{entrée}{ɴɢartɯm,ɣɯt/\variante{ɴɢaftɯm}}{}{ⓔɴɢartɯm,ɣɯt} 
\classe{n}
\classe{vt} 
\begin{définition}\pcmn{俯冲}\end{définition}
\begin{exemple}\pjya{qaliaʁ kɯ ɴɢartɯm pjɤ-ɣɯt}\hspace{5pt}\pcmn{老鹰俯冲(下去)了}\end{exemple}
\begin{exemple}\pjya{kɯjka kɯ ɴɢartɯm pa-ɣɯt}\hspace{5pt}\pcmn{红嘴乌鸦俯冲(下去)了}\end{exemple}\relationsémantique{Component 1}{\lien{}{ɴɢartɯm}}\relationsémantique{Component 2}{\lien{ⓔɣɯt}{ɣɯt}}\end{entrée}

\begin{entrée}{ɴɢaʁ}{}{ⓔɴɢaʁ} 
\classe{vi}  
\grammaire{acaus} \paradigme{dir}{pɯ-}
\begin{définition}\pfra{perdre sa peau}\end{définition}
\begin{définition}\pcmn{自动脱皮}\end{définition}
\begin{exemple}\pjya{sɯrqhu pjɤ-ɴɢaʁ}\hspace{5pt}\pcmn{树皮脱了}\end{exemple}
\begin{exemple}\pjya{a-mi ɯ-rqhu pjɤ-ɴɢaʁ}\hspace{5pt}\pcmn{我的脚脱皮了}\end{exemple}
\begin{exemple}\pjya{ʑmbɤr ɯ-rqhu pjɤ-ɴɢaʁ}\hspace{5pt}\pcmn{疮脱皮了}\end{exemple}\relationsémantique{参考}{\lien{ⓔqaʁⓗ1}{qaʁ₁}}\end{entrée}

\begin{entrée}{ɴɢaʁrɯm}{}{ⓔɴɢaʁrɯm} 
\classe{n} 
\begin{définition}\pfra{ombre (bâtiments, montagne)}\end{définition}
\begin{définition}\pcmn{阴影}\end{définition}\end{entrée}

\begin{entrée}{ɴɢɤjom}{}{ⓔɴɢɤjom} 
\classe{n} 
\begin{définition}\pfra{Rumex crispus}\end{définition}
\begin{définition}\pcmn{皱叶酸模}\end{définition}
\begin{exemple}\pjya{ɴɢɤjom nɯ ruŋgu kɯ-mbro tsa tu-ɬoʁ ŋu, ɯ-ru nɯ qhɤjmbaʁ ɯ-ru cho naχtɕɯɣ, ɯ-jwaʁ nɯ qhɤjmbaʁ ɯ-jwaʁ sɤznɤ khro ʑo ndɯβ, ɯ-rme kɯ-fse kɯ-xtɕɯ-xtɕi tu, ɯ-mɯntoʁ kɯ-wɣrum ŋu, kɯ-ndɯ-ndɯβ ʑo ŋu, ɯ-ru nɯ tú-wɣ-ndza tɕe wuma ʑo tɕur. zgoku pa pɕoʁ ra maka mɤ-ɬoʁ.}\hspace{5pt}\pcmn{酸模生长在比较高的草山上。茎和\lien{ⓔqhɤjmbaʁ}{qhɤjmbaʁ}的一样,叶子比\lien{ⓔqhɤjmbaʁ}{qhɤjmbaʁ}的叶子小得多,有点细毛。花是白色的,细小。茎吃起来很酸。山下根本不能生长。}\end{exemple}\end{entrée}

\begin{entrée}{ɴɢɤt}{}{ⓔɴɢɤt} 
\classe{vi} \paradigme{dir}{nɯ-}
\begin{définition}\pfra{se séparer}\end{définition}
\begin{définition}\pcmn{分散;分手}\end{définition}
\begin{exemple}\pjya{tɯtɯrca pɯ-rɤʑi-tɕi, tɕe nɯ-nɯ-ɴɢɤt-tɕi}\hspace{5pt}\pcmn{我们俩原来在一起,然后就分手了}\end{exemple}
\begin{exemple}\pjya{ʁzɤmi ni pɯ-nɯ-ɴɢɤt-tɕi}\hspace{5pt}\pcmn{我们离婚了}\end{exemple}
\begin{exemple}\pjya{tɤ-pi nɯ tɤ-rɯstɯnmɯ tɕe, kɤndʑɯxtɤɣ ni nɯ-nɯ-ɴɢɤt-ndʑi}\hspace{5pt}\pcmn{哥哥结婚了,兄弟俩就分开了}\end{exemple}\relationsémantique{参考}{\lien{ⓔqɤt}{qɤt}}\relationsémantique{参考}{\lien{ⓔnɯɴɢɯlɯjɤt}{nɯɴɢɯlɯjɤt}}\relationsémantique{参考}{\lien{ⓔznɯɴɢɤt}{znɯɴɢɤt}}\end{entrée}

\begin{entrée}{ɴɢia}{}{ⓔɴɢia} 
\classe{vi}  
\grammaire{acaus} 
\begin{définition}\pfra{se détacher, se dérouler (fil)}\end{définition}
\begin{définition}\pcmn{散(线)}\end{définition}
\begin{exemple}\pjya{kɤtɯm pjɤ-ɴɢia tɕe ɲɤ-ɬɯt}\hspace{5pt}\pcmn{线团散了就乱了}\end{exemple}
\begin{exemple}\pjya{tɤ-mtsɯ ɲɤ-ɴɢia}\hspace{5pt}\pcmn{结散了}\end{exemple}
\begin{exemple}\pjya{tɤ-fkɯm ɯ-mŋu ɲɤ-nɯ-ɴɢia (=tɤ-fkɯm ɯ-xtɕɤr ɲɤ-nɯ-ɬoʁ)}\hspace{5pt}\pcmn{口袋的口自动解开了}\end{exemple}\relationsémantique{参考}{\lien{ⓔqia}{qia}}\end{entrée}

\begin{entrée}{ɴɢiɤβɴɢiɤβ}{}{ⓔɴɢiɤβɴɢiɤβ} 
\classe{idph.2} 
\begin{définition}\pfra{nonchalant}\end{définition}
\begin{définition}\pcmn{形容不慌不忙的样子}\end{définition}
\begin{sous-entrée}{ɴɢiɤβnɤɴɢiɤβ}{ⓔɴɢiɤβɴɢiɤβⓝɴɢiɤβnɤɴɢiɤβ} 
\classe{idph.3} 
\begin{définition}\pfra{nonchalant}\end{définition}
\begin{définition}\pcmn{不慌不忙地(做事)}\end{définition}
\begin{exemple}\pjya{ɴɢiɤβnɤɴɢiɤβ kɤ-ari}\hspace{5pt}\pcmn{他不慌不忙地去了}\end{exemple}
\begin{exemple}\pjya{ɴɢiɤβnɤɴɢiɤβ ɲɯ-rɤma}\hspace{5pt}\pcmn{他不慌不忙地劳动}\end{exemple}\end{sous-entrée}

\end{entrée}

\begin{entrée}{ɴɢiɤt}{}{ⓔɴɢiɤt} 
\classe{vs} \paradigme{dir}{tɤ-}\paradigme{dir}{tɤ-}
\begin{définition}\pfra{désordonné}\end{définition}
\begin{définition}\pcmn{不爱整理的}\end{définition}
\begin{définition}\pfra{ne pas faire attention à}\end{définition}
\begin{définition}\pcmn{不重视}\end{définition}
\begin{exemple}\pjya{ma-tɤ-tɯ-ɴɢiɤt tɕe nɤ-ŋga ra tɤ-rɤwum}\hspace{5pt}\pcmn{你不要真么乱,收拾一下你的衣服}\end{exemple}
\begin{exemple}\pjya{nɤ-xtu ɲɯ-tɯ-nɤɴɢiɤt}\hspace{5pt}\pcmn{你亏待你的肚子(吃得太少)}\end{exemple}
\begin{exemple}\pjya{nɤ-kɤ-nɤma ra ma-tɤ-tɯ-nɤɴɢiɤt ma kɤ-nɤɴɢiɤt kɯ kɯ-nɤɴɢiɤt kɤ-ti}\hspace{5pt}\pcmn{你要专心地做这个工作,如果你不重视它的话,它也会不重视你(工作就做不出来)}\end{exemple}
\begin{sous-entrée}{nɤɴɢiɤt}{ⓔɴɢiɤtⓝnɤɴɢiɤt} 
\classe{vt} \end{sous-entrée}

\end{entrée}

\begin{entrée}{ɴɢule}{}{ⓔɴɢule} 
\classe{vs} \paradigme{dir}{nɯ-}
\begin{définition}\pfra{oisif}\end{définition}
\begin{définition}\pcmn{松懈}\end{définition}
\begin{exemple}\pjya{ta-ma kɯ-ɴɢule ci ɲɯ-ŋu}\hspace{5pt}\pcmn{他是工作不勤快的人}\end{exemple}
\begin{exemple}\pjya{kɤ-nɤma ɲɯ-ɴɢule}\hspace{5pt}\pcmn{他工作得不勤快}\end{exemple}\relationsémantique{参考}{\lien{ⓔɴɢu}{ɴɢu}}\end{entrée}

\begin{entrée}{ɴɢlɯt}{}{ⓔɴɢlɯt} 
\classe{vi}  
\grammaire{acaus} \paradigme{dir}{pɯ-}
\begin{définition}\pfra{se casser}\end{définition}
\begin{définition}\pcmn{折;断(自动)}\end{définition}
\begin{exemple}\pjya{pɯ-ɴɢlɯt}\hspace{5pt}\pcmn{折了}\end{exemple}
\begin{exemple}\pjya{ɕoŋtɕa pɯ-ɴɢlɯt}\hspace{5pt}\pcmn{木料折了}\end{exemple}
\begin{exemple}\pjya{laʁdɯn pjɤ-ɴɢlɯt}\hspace{5pt}\pcmn{工具断了}\end{exemple}
\begin{exemple}\pjya{a-jaʁ pjɤ-ɴɢlɯt}\hspace{5pt}\pcmn{我的手折了}\end{exemple}
\begin{exemple}\pjya{ɯ-mi pjɤ-ɴɢlɯt}\hspace{5pt}\pcmn{他的脚折了}\end{exemple}
\begin{exemple}\pjya{ɯ-rnom ko-ɴɢlɯt}\hspace{5pt}\pcmn{他的肋骨折了}\end{exemple}
\begin{exemple}\pjya{mbɣo pjɤ-ɴɢlɯt}\hspace{5pt}\pcmn{犁断了}\end{exemple}
\begin{exemple}\pjya{ɯ-jɯ pjɤ-ɴɢlɯt}\hspace{5pt}\pcmn{把子断了}\end{exemple}
\begin{sous-entrée}{ɣɤɴɢlɯt}{ⓔɴɢlɯtⓝɣɤɴɢlɯt} 
\classe{vs}  
\grammaire{facil} 
\begin{définition}\pfra{qui se casse facilement}\end{définition}
\begin{définition}\pcmn{容易断}\end{définition}\relationsémantique{参考}{\lien{ⓔqlɯt}{qlɯt}}\end{sous-entrée}

\end{entrée}

\begin{entrée}{ɴɢoɕna}{}{ⓔɴɢoɕna} 
\classe{n} 
\begin{définition}\pfra{grosse araignée}\end{définition}
\begin{définition}\pcmn{大蜘蛛}\end{définition}
\begin{exemple}\pjya{ɴɢoɕna cho porɤt ni ɲɯ-naχtɕɯɣ-ndʑi, ɴɢoɕna kɯ-wxti, ɯ-mi ra jpum, porɤt nɯ xtɕi, ɯ-mi ra xtshɯm, kɯ-pɣi ŋu-ndʑi, ɴɢoɕna kɯ aɲaʁndzɯm. ɴɢoɕna cho porɤt ni ndʑi-xtu ɯ-ŋgɯ ndʑi-ri kɯ-fse chɯ-nɯ-tɕɤt-ndʑi tɕe, kha tu-nɯ-βzu-nɯ ɲɯ-ŋgrɤl. nɯ kha nɯ ɴɢoɕnamɤjɯ rmi. ndʑi-kha nɯ tɕu kɯmaʁ qajɯ ra nɯ-fsa tu-sɯpa-nɯ tɕe, ka-ndo tɕe tu-ndza-nɯ ɲɯ-ŋu. nɯ-pɯ wuma ɲɯ-dɤn, koŋla mɯ-thɯ-wxti-nɯ mɤɕtʂa ɯ-fkɯm ci ɣɤʑu tɕe tu-nɤfkɯ-fkur ɲɯ-ra.}\hspace{5pt}\pcmn{大蜘蛛和小蜘蛛很相似,大蜘蛛比较大,脚粗一点,小蜘蛛比较小,脚细一点。两种都是灰色的,大蜘蛛是暗灰色的。它们从肚子里抽出丝来制造“房子”,这种房子叫蜘蛛网,用这个网作抓其它虫子的圈套,一旦网缠住了(小虫),它们就会把它吃掉。蜘蛛有很多幼虫,在幼虫尚未长大之前,一直把它们放在一种袋子里背来背去。}\end{exemple}\end{entrée}

\begin{entrée}{ɴɢoɕnamɤjɯ}{}{ⓔɴɢoɕnamɤjɯ} 
\classe{n} 
\begin{définition}\pfra{toile d'araignée}\end{définition}
\begin{définition}\pcmn{蜘蛛网}\end{définition}\end{entrée}

\begin{entrée}{ɴɢolo}{}{ⓔɴɢolo} 
\classe{n} 
\begin{définition}\pfra{une espèce d'arbrisseau}\end{définition}
\begin{définition}\pcmn{灌木的一种}\end{définition}
\begin{exemple}\pjya{ɴɢolo nɯ si kɯ-mbɤr tsa ci ŋu. ɯ-ru ɯ-pɕi nɯ ra kɯ-pɣi ci ŋu. ɯ-mdzu kɯ-rɲɟi kɯ-mtɕoʁ tɯ-khɤl ʑo χsɯ-ldʑa ntsɯ ku-ndzoʁ ŋu. ɯ-rtaʁ ɯ-taʁ ra kɯnɤ ɯ-mdzu tu. ɯ-jwaʁ ɯ-tshɯɣa nɯ babɯ ɯ-jwaʁ cho naχtɕɯɣ. ɯ-mɯntoʁ tshaŋlaŋ kɯ-fse ɲɯ-βze ŋu, kɯ-wɣrum ɯ-rkɯ zɯ kɯ-ɣɯrni tu-fskɤr ŋu. kɯ-ɤrqhi jɯ-kɯ-ru mɤ-saχsɤl jɯ-kɯ-ɤrmbat tɕe, ɯ-mɯntoʁ tú-wɣ-rtoʁ tɕe mpɕɤr. ɯ-jɯ tu tɕe, pjɯ-ɴqoʁ tu-fse ŋu. ɯ-mat thɯ-aβzu tɕe, arŋi tɕe ʂɣɤlʂɣɤl ʑo pa, ɯ-ŋgɯ ɯ-rdoʁ ra kɯnɤ saχsɤl. ɯ-mat ɯ-βri ɯ-rme kɯ-xtɯt tsa tu. tú-wɣ-ndza tɕe, ɯ-tɯ-tɕur saχaʁ. tɤ-pɤtso ra kɤ-nɤɣro kɯ-fse ma kɤ-ndza mɤ-sna. ɯ-mnɯ kɯ-ɕɤɣ tɤ-kɯ-ɬoʁ nɯ ɣɯrni ɯ-mdzu tu ri ɯ-rqhu pjɯ́-wɣ-qaʁ tɕe tú-wɣ-ndza tɕe mpɯ, kɯ-xtɕɯ-xtɕi chi. tɤmdzɤqaqa rmi.}\hspace{5pt}\pcmn{\lien{ⓔɴɢolo}{ɴɢolo}是一种矮小的树,树皮是灰色的,刺又长又锋利,三根长在一起。枝桠上也有刺。叶子的形状和\lien{ⓔbabɯ}{babɯ}的叶子一样。花像铃铛一样,花瓣是白色的,边上镶有红色。远处看不出,走近了看就觉得花很美。花有花梗,是吊着的。果实结了以后,看起来很透明,连里面的种子也看得见。果实外面也长有短毛。吃起来很酸。除了小孩子吃着玩以外就不能吃。发出的新苗是红色的,虽然有刺,但剥了皮吃起来很嫩,有点甜。这种新苗叫\lien{ⓔtɤmdzɤqaqa}{tɤmdzɤqaqa}。}\end{exemple}\end{entrée}

\begin{entrée}{ɴɢolophɯcɯ}{}{ⓔɴɢolophɯcɯ} 
\classe{n} 
\begin{définition}\pfra{grès}\end{définition}
\begin{définition}\pcmn{砂岩}\end{définition}
\begin{exemple}\pjya{ɴɢolophɯcɯ, kɤ́-wɣ-rtoʁ tɕe, tɤ-phɯ fse ri rdɤstaʁ jamar rko}\hspace{5pt}\pcmn{砂岩看起来像土块,但是像石头那么硬}\end{exemple}\end{entrée}

\begin{entrée}{ɴɢorna}{}{ⓔɴɢorna} 
\classe{n} 
\begin{définition}\pfra{une plante}\end{définition}
\begin{définition}\pcmn{植物的一种}\end{définition}
\begin{exemple}\pjya{ɴɢorna nɯ sɯjno kɯ-mpɯ-mpɯ ci ŋu. ɯ-ru kɯ-zɯ-zri ŋu tɕe, ɯ-zda sɯjno ɯ-taʁ tu-nɯrʁɯrʁa ra ma ɯʑo tu-nɯ-ndzur mɤ-cha ma mpɯ. ɯ-jwaʁ cho ɯ-ru ra kɯ-rʁɯ-rʁom ŋu. tɯ-ɕa ra pjɯ-qraʁ cha. tɯ-ŋga ɯ-taʁ ku-ndzoʁ ŋu. ɯ-mat nɯ kɯ-ɤrtɯ-rtɯm tɕe, li ɯ-taʁ ɯ-rme kɯ-tu ŋu, ɯ-jwaʁ ɯ-rchɤβ ri ku-ndzoʁ ŋu. thɯ-tɯt tɕe ɲaʁ. ɯ-jwaʁ kɯ-ndɯ-ndɯβ ŋu, ɯ-ru ɯ-taʁ kɤ-fskɤr tɕe, tɯ-rtsɤɣ tɯ-rtsɤɣ ku-ndzoʁ ŋu. ɴɢorna nɯ fsapaʁndza sna.}\hspace{5pt}\pcmn{\lien{ⓔɴɢorna}{ɴɢorna}是一种很软的草,茎长得很长,必须爬在其它草上因为它身子软,立不起来。叶子和茎很粗糙,可以把皮肉刮破,还粘在衣服上。果实是球形的,上面也有毛,生长在叶子的中间。成熟了以后是黑色的。叶子很小,是绕着茎一节一节地长着。\lien{ⓔɴɢorna}{ɴɢorna}是牲畜的饲料。}\end{exemple}\end{entrée}

\begin{entrée}{ɴɢraʁ}{}{ⓔɴɢraʁ} 
\classe{vi}
\classe{n}
\classe{vi}  
\grammaire{acaus} \paradigme{dir}{pɯ-}\paradigme{dir}{thɯ-}\paradigme{dir}{nɯ-}
\begin{définition}\pfra{se déchirer}\end{définition}
\begin{définition}\pcmn{破烂(衣服、皮)}\end{définition}
\begin{définition}\pfra{percer, se lever (aube)}\end{définition}
\begin{définition}\pcmn{破晓}\end{définition}
\begin{exemple}\pjya{a-ŋga pjɤ-ɴɢraʁ}\hspace{5pt}\pcmn{我的衣服破了}\end{exemple}
\begin{exemple}\pjya{nɤ-ŋga nɤ-xtsa pjɤ-ɴɢraʁ}\hspace{5pt}\pcmn{你的衣服鞋子都破了}\end{exemple}
\begin{exemple}\pjya{a-jaʁ pjɤ-ɴɢraʁ}\hspace{5pt}\pcmn{我的手破了(皮肤破了)}\end{exemple}
\begin{exemple}\pjya{nɤ-ŋga cho-ɴɢraʁ}\hspace{5pt}\pcmn{你的衣服破了}\end{exemple}
\begin{exemple}\pjya{ɕoʁɕoʁ chɤ-ɴɢraʁ}\hspace{5pt}\pcmn{纸撕破了}\end{exemple}
\begin{exemple}\pjya{ɕɤrkha ɲɤ-ɴɢraʁ}\hspace{5pt}\pcmn{破晓了}\end{exemple}\relationsémantique{参考}{\lien{ⓔqraʁⓗ1}{qraʁ₁}}\relationsémantique{Component 1}{\lien{ⓔɕɤrkha}{ɕɤrkha}}\relationsémantique{Component 2}{\lien{ⓔɴɢraʁ}{ɴɢraʁ}}
\begin{sous-entrée}{ɕɤrkha,ɴɢraʁ}{ⓔɴɢraʁⓝɕɤrkha,ɴɢraʁ}\end{sous-entrée}

\end{entrée}

\begin{entrée}{ɴɢrɤz}{}{ⓔɴɢrɤz} 
\classe{vi}  
\grammaire{acaus} \paradigme{dir}{nɯ-}
\begin{définition}\pfra{se réduire en poussière au moindre toucher (objets secs)}\end{définition}
\begin{définition}\pcmn{一摸就烂(干的东西)}\end{définition}
\begin{exemple}\pjya{xɕaj pjɤ-rom tɕe ɲɤ-ɴɢrɤz ɲɯ-ɕti}\hspace{5pt}\pcmn{草干了以后一摸就烂了}\end{exemple}
\begin{exemple}\pjya{tɤ-jwaʁ ɲɤ-ɴɢrɤz}\hspace{5pt}\pcmn{叶子一摸就烂了}\end{exemple}\relationsémantique{参考}{\lien{ⓔqrɤz}{qrɤz}}\end{entrée}

\begin{entrée}{ɴɢrɯ}{}{ⓔɴɢrɯ} 
\classe{vi}  
\grammaire{acaus} \paradigme{dir}{pɯ-}
\begin{définition}\pfra{se casser}\end{définition}
\begin{définition}\pcmn{破;碎(自动)}\end{définition}
\begin{exemple}\pjya{χɕɤlzgoŋ ki pjɤ-ɴɢrɯ}\hspace{5pt}\pcmn{镜子破了}\end{exemple}
\begin{exemple}\pjya{khɯtsa pɯ-ɴɢrɯ}\hspace{5pt}\pcmn{碗破了}\end{exemple}
\begin{exemple}\pjya{popo pjɤ-ɴɢrɯ}\hspace{5pt}\pcmn{砂锅破了}\end{exemple}
\begin{exemple}\pjya{ɕɯ-ɴɢrɯ nɯ-sɯsota}\hspace{5pt}\pcmn{我怕会破}\end{exemple}
\begin{sous-entrée}{ɣɤɴɢrɯ}{ⓔɴɢrɯⓝɣɤɴɢrɯ} 
\classe{vs} 
\begin{définition}\pfra{qui se casse facilement}\end{définition}
\begin{définition}\pcmn{容易破}\end{définition}
\begin{exemple}\pjya{ki khɯtsa ki mɯ́j-ngɯt tɕe ɲɯ-ɣɤɴɢrɯ}\hspace{5pt}\pcmn{这个碗不结实,容易破}\end{exemple}\relationsémantique{参考}{\lien{ⓔqrɯ}{qrɯ}}\end{sous-entrée}

\end{entrée}

\begin{entrée}{ɴɢuʁɤr/\variante{ŋguʁɤr}}{}{ⓔɴɢuʁɤr} 
\classe{n} 
\begin{définition}\pfra{tissu de laine}\end{définition}
\begin{définition}\pcmn{呢子}\end{définition}\étymologie{bal}\end{entrée}

\begin{entrée}{ɴɢɯɴɢli}{}{ⓔɴɢɯɴɢli} 
\classe{idph.2} 
\begin{définition}\pfra{écarquillant les yeux}\end{définition}
\begin{définition}\pcmn{眼睛睁得很大的样子}\end{définition}
\begin{exemple}\pjya{qala kɯ ɯ-mɲaʁ ɴɢɯɴɢli ʑo to-stu}\hspace{5pt}\pcmn{兔子把眼睛睁得很大}\end{exemple}\relationsémantique{参考}{\lien{ⓔqɯqli}{qɯqli}}\end{entrée}

\begin{entrée}{ɴqa}{}{ⓔɴqa} 
\classe{vs} \paradigme{dir}{thɯ-}\paradigme{dir}{tɤ-}
\begin{définition}\pfra{dur (travail)}\end{définition}
\begin{définition}\pcmn{辛苦;难做}\end{définition}
\begin{définition}\pfra{rendre difficile}\end{définition}
\begin{définition}\pcmn{使困难}\end{définition}
\begin{exemple}\pjya{ta-ma ɲɯ-ɴqa}\hspace{5pt}\pcmn{工作很辛苦}\end{exemple}
\begin{exemple}\pjya{ɯ-pɯ́-ɴqa}\hspace{5pt}\pcmn{辛苦了吗}\end{exemple}
\begin{exemple}\pjya{nɤ-tʂha ɲɯ-ɴqa}\hspace{5pt}\pcmn{你放了很多茶叶,茶很浓}\end{exemple}
\begin{exemple}\pjya{ɯ-phɯ to-ɣɤɴqa}\hspace{5pt}\pcmn{他加价了}\end{exemple}\relationsémantique{反义词}{\lien{ⓔmbat}{mbat}}\relationsémantique{参考}{\lien{ⓔnɤɴqa}{nɤɴqa}}
\begin{sous-entrée}{ɣɤɴqa}{ⓔɴqaⓝɣɤɴqa} 
\classe{vt} \end{sous-entrée}

\end{entrée}

\begin{entrée}{ɴqhɤβɴqhɤβ}{}{ⓔɴqhɤβɴqhɤβ} 
\classe{idph.2} 
\begin{définition}\pfra{épais et dur}\end{définition}
\begin{définition}\pcmn{形容厚实的样子}\end{définition}
\begin{exemple}\pjya{qhɤjmbaʁ ɣɯ ɯ-jwaʁ ɴqhɤβɴqhɤβ ʑo pa}\hspace{5pt}\pcmn{红青椒的叶子很厚实的样子}\end{exemple}\relationsémantique{同义词}{\lien{ⓔɴqhɯɴqhi}{ɴqhɯɴqhi}}\end{entrée}

\begin{entrée}{ɴqhi}{}{ⓔɴqhi} 
\classe{vs} \paradigme{dir}{kɤ-}\paradigme{dir}{kɤ-}
\begin{définition}\pfra{sale}\end{définition}
\begin{définition}\pcmn{脏}\end{définition}
\begin{définition}\pfra{salir}\end{définition}
\begin{définition}\pcmn{弄脏}\end{définition}
\begin{exemple}\pjya{ki khɯtsa ɲɯ-ɴqhi}\hspace{5pt}\pcmn{这个碗很脏}\end{exemple}
\begin{exemple}\pjya{βɣɤza nɯ kɯ-ɴqhi ɲɯ-rga}\hspace{5pt}\pcmn{苍蝇爱脏的东西}\end{exemple}
\begin{exemple}\pjya{ko-ɴqhi}\hspace{5pt}\pcmn{变脏了}\end{exemple}
\begin{exemple}\pjya{tɤ-pɤtso kɯ ɯ-ŋga ko-sɯɴqhi}\hspace{5pt}\pcmn{小孩子把自己衣服弄脏了}\end{exemple}
\begin{exemple}\pjya{tɤ-lu kɯ ɯ-sɤ-rku thamtɕɤt sɯ-ɴqhi ɕti}\hspace{5pt}\pcmn{牛奶会弄脏容器}\end{exemple}\relationsémantique{参考}{\lien{ⓔtɤlɤɴqhi}{tɤlɤɴqhi}}
\begin{sous-entrée}{nɤɴqhi}{ⓔɴqhiⓝnɤɴqhi} 
\classe{vt}  
\grammaire{trop} 
\begin{exemple}\pjya{khɯtsa ɲɯ-nɤɴqhi tɕe, mɯ́j-nɤpe}\hspace{5pt}\pcmn{他觉得碗很脏,觉得不好}\end{exemple}\end{sous-entrée}

\begin{sous-entrée}{ɣɤɴqhi}{ⓔɴqhiⓝɣɤɴqhi} 
\classe{vs} 
\begin{définition}\pfra{qui se salit vite}\end{définition}
\begin{définition}\pcmn{脏得快}\end{définition}\end{sous-entrée}

\begin{sous-entrée}{sɯɴqhi}{ⓔɴqhiⓝsɯɴqhi} 
\classe{vt}  
\grammaire{caus} \end{sous-entrée}

\end{entrée}

\begin{entrée}{ɴqhɯɴqhi}{}{ⓔɴqhɯɴqhi} 
\classe{idph.2} 
\begin{définition}\pfra{épais}\end{définition}
\begin{définition}\pcmn{形容叶子、菌子、脸等厚实的样子}\end{définition}\relationsémantique{同义词}{\lien{ⓔɴqhɤβɴqhɤβ}{ɴqhɤβɴqhɤβ}}\end{entrée}

\begin{entrée}{ɴqiaβ}{}{ⓔɴqiaβ} 
\classe{n} 
\begin{définition}\pfra{ubac}\end{définition}
\begin{définition}\pcmn{山阴,背阴的山坡}\end{définition}\end{entrée}

\begin{entrée}{ɴqiazwɤr}{}{ⓔɴqiazwɤr} 
\classe{n} 
\begin{définition}\pfra{espèce d'armoise}\end{définition}
\begin{définition}\pcmn{阳山的艾蒿}\end{définition}\relationsémantique{参考}{\lien{ⓔzwɤrⓗ2}{zwɤr₂}}\relationsémantique{参考}{\lien{ⓔɴqiaβ}{ɴqiaβ}}\end{entrée}

\begin{entrée}{ɴqoʁ}{}{ⓔɴqoʁ} 
\classe{vi} \paradigme{dir}{kɤ-}\paradigme{dir}{tɤ-}
\begin{définition}\pfra{être accroché, se tenir}\end{définition}
\begin{définition}\pcmn{悬挂;扶住}\end{définition}
\begin{exemple}\pjya{kɤ-ɴqoʁ ma tɯ-atɤr}\hspace{5pt}\pcmn{你抓稳,小心不要掉下来}\end{exemple}
\begin{exemple}\pjya{aʑo si ɯ-taʁ zɯ kɤ-ɴqoʁ-a}\hspace{5pt}\pcmn{我抓住了树}\end{exemple}
\begin{exemple}\pjya{ndʐaβ-a ɲɯ-ŋu tɕe, @langan ɯ-taʁ kɤ-ɴqoʁ-a}\hspace{5pt}\pcmn{我差一点跌倒了,但抓住了栏杆}\end{exemple}
\begin{exemple}\pjya{ɕomskrɯt ɯ-taʁ tɯ-ŋga χsɯm ɲɯ-ɴqoʁ}\hspace{5pt}\pcmn{三件衣服挂在铁丝上}\end{exemple}\relationsémantique{参考}{\lien{ⓔɕɯɴqoʁ}{ɕɯɴqoʁ}}\relationsémantique{参考}{\lien{ⓔʑɴɢoʁ}{ʑɴɢoʁ}}\end{entrée}

\newpage\caractère{o}

\begin{entrée}{ommanipɤnmehoŋʂi}{}{ⓔommanipɤnmehoŋʂi} 
\classe{n} 
\begin{définition}\pfra{un mantra}\end{définition}
\begin{définition}\pcmn{六字真言}\end{définition}\étymologie{om.ma.ni.padme.hum.hri}\end{entrée}

\newpage\caractère{p}

\begin{entrée}{pu}{}{ⓔpu} 
\classe{vt} \paradigme{dir}{thɯ-}\paradigme{dir}{lɤ-}
\begin{définition}\pfra{cuire dans les braises}\end{définition}
\begin{définition}\pcmn{煨}\end{définition}
\begin{exemple}\pjya{jaŋjy lɤ-pe}\hspace{5pt}\pcmn{你把土豆烤一下}\end{exemple}
\begin{exemple}\pjya{qajɣi lɤ-pe}\hspace{5pt}\pcmn{你把馍馍烤一下}\end{exemple}
\begin{exemple}\pjya{tɤ-mthɯm thɯ-pu-t-a}\hspace{5pt}\pcmn{我烤了肉}\end{exemple}
\begin{exemple}\pjya{pɤjka lɤ-pu-t-a}\hspace{5pt}\pcmn{我烤了白瓜}\end{exemple}\relationsémantique{参考}{\lien{ⓔtɯpu}{tɯpu}}\end{entrée}

\begin{entrée}{pa}{₃}{ⓔpaⓗ3} 
\classe{adv} 
\begin{définition}\pfra{en bas}\end{définition}
\begin{définition}\pcmn{下面}\end{définition}
\begin{sous-entrée}{ɯ-pa}{ⓔpaⓗ3ⓝɯ-pa} 
\classe{np} 
\begin{définition}\pfra{le bas}\end{définition}
\begin{définition}\pcmn{下面}\end{définition}\end{sous-entrée}

\end{entrée}

\begin{entrée}{pa}{₁}{ⓔpaⓗ1} 
\classe{vt}
\classe{np}  
\grammaire{autoben}
\grammaire{refl}
\grammaire{autoben} \sens{1}\paradigme{dir}{tɤ-}
\begin{définition}\pfra{faire}\end{définition}
\begin{définition}\pcmn{办}\end{définition}
\begin{exemple}\pjya{tɕhi tu-pe-a ɲɯ-ra?}\hspace{5pt}\pcmn{我应该怎么办?}\end{exemple}
\begin{exemple}\pjya{aʑo akɯ ɕe-a, nɤʑo tɕhi tɯ-pe}\hspace{5pt}\pcmn{我往东边去,你呢?}\end{exemple}
\begin{exemple}\pjya{tɕhi tú-wɣ-pa?}\hspace{5pt}\pcmn{怎么办?}\end{exemple}
\begin{exemple}\pjya{kutɕu tɕhi ɯ-kɯ-pa jɤ-tɯ-ɣe?}\hspace{5pt}\pcmn{你来这里做什么?}\end{exemple}
\begin{exemple}\pjya{dal tsa tɤ-pe}\hspace{5pt}\pcmn{慢慢做}\end{exemple}\relationsémantique{参考}{\lien{ⓔkɤpa}{kɤpa}}\sens{2}\paradigme{dir}{kɤ-}\paradigme{dir}{tɤ-}
\begin{définition}\pfra{fermer}\end{définition}
\begin{définition}\pcmn{关}\end{définition}
\begin{définition}\pfra{se mettre d'accord}\end{définition}
\begin{définition}\pcmn{商量好}\end{définition}
\begin{exemple}\pjya{kɯm kɤ-pe}\hspace{5pt}\pcmn{关门吧!}\end{exemple}
\begin{exemple}\pjya{@diandeng kɤ-pe}\hspace{5pt}\pcmn{关电灯吧!}\end{exemple}
\begin{exemple}\pjya{khɯɣɲɟɯ kɤ-pe}\hspace{5pt}\pcmn{关窗户吧!}\end{exemple}
\begin{exemple}\pjya{ɯʑo kɯ a-@dianhua pjɤ-pa}\hspace{5pt}\pcmn{他挂了我电话}\end{exemple}
\begin{exemple}\pjya{nɤ-@shouji kɤ-nɯ-pe ɯ́-jɤɣ?}\hspace{5pt}\pcmn{麻烦你把手机关一下}\end{exemple}\relationsémantique{参考}{\lien{ⓔapa}{apa}}\relationsémantique{参考}{\lien{ⓔsɤpa}{sɤpa}}
\begin{sous-entrée}{nɯpa}{ⓔpaⓗ1ⓢ2ⓝnɯpa} 
\classe{vt} \end{sous-entrée}

\paradigme{dir}{kɤ-}\paradigme{dir}{thɯ-}
\begin{exemple}\pjya{ju-kɤ-ɕe tɤ-nɯpa-tɕi}\hspace{5pt}\pcmn{我们商量好要(一起)出发}\end{exemple}
\begin{exemple}\pjya{ɣufsu (βzaŋsa) tɤ-nɯ-pa-tɕi}\hspace{5pt}\pcmn{我们俩交了朋友}\end{exemple}
\begin{sous-entrée}{nɯʑɣɤpa}{ⓔpaⓗ1ⓝnɯʑɣɤpa} 
\classe{vi} \end{sous-entrée}

\paradigme{dir}{tɤ-}
\begin{définition}\pfra{se fermer par soi-même}\end{définition}
\begin{définition}\pcmn{自动关}\end{définition}
\begin{définition}\pfra{conserver}\end{définition}
\begin{définition}\pcmn{保管}\end{définition}
\begin{exemple}\pjya{kɯm ko-nɯʑɣɤpa}\hspace{5pt}\pcmn{门自动关了}\end{exemple}
\begin{exemple}\pjya{laχtɕha ɯ-pɯ tɤ-pe}\hspace{5pt}\pcmn{你把东西保管好!}\end{exemple}
\begin{exemple}\pjya{tɤ-lu tɤ-rʑaʁ kɯ-rɲɟi ɯ-pɯ kɤ-pa mɯ́j-khɯ tu-rpjɯ ɲɯ-ɕti}\hspace{5pt}\pcmn{牛奶不能长时间保存,不然就会腥(变质)}\end{exemple}
\begin{exemple}\pjya{nɤ-jaʁ ɯ-pɯ ma-tɤ-tɯ-nɯ-pe}\hspace{5pt}\pcmn{你不要不管,要帮一下忙}\end{exemple}\relationsémantique{Component 1}{\lien{}{ɯ-pɯ}}\relationsémantique{Component 2}{\lien{ⓔpaⓗ3}{pa}}
\begin{sous-entrée}{ɯ-pɯ,pa}{ⓔpaⓗ1ⓝɯ-pɯ,pa}\end{sous-entrée}

\end{entrée}

\begin{entrée}{pa}{₂}{ⓔpaⓗ2} 
\classe{vs} \paradigme{dir}{tɤ-}\sens{1}
\begin{définition}\pfra{auxiliaire employé avec les idéophones}\end{définition}
\begin{définition}\pcmn{助动词(和状貌词连用)}\end{définition}
\begin{exemple}\pjya{ko-nɯ-rŋgɯ tɕe, ɯ-sɤ-rŋgɯ-ŋga loŋloŋ ʑo ɲɯ-pa}\hspace{5pt}\pcmn{他睡着了以后,床上是拱起来的}\end{exemple}\sens{2}
\begin{définition}\pfra{se passer X années}\end{définition}
\begin{définition}\pcmn{过……年}\end{définition}
\begin{exemple}\pjya{tɕiʑo ni kɤ-amɯfse-tɕi nɯ jinde kɯβdɤsqi ɯ-ro to-pa}\hspace{5pt}\pcmn{我们俩认识已经四十多年了}\end{exemple}\end{entrée}

\begin{entrée}{pakuku/\variante{xpakuku}}{}{ⓔpakuku} 
\classe{adv} 
\begin{définition}\pfra{tous les ans}\end{définition}
\begin{définition}\pcmn{每年}\end{définition}\end{entrée}

\begin{entrée}{palaxtsa}{}{ⓔpalaxtsa} 
\classe{n} 
\begin{définition}\pfra{botte en peau de chevrotain à la couture grossière}\end{définition}
\begin{définition}\pcmn{靴筒全是獐皮子的一种靴子,针脚缝得比较粗糙}\end{définition}
\begin{exemple}\pjya{palaxtsa nɯ ɯ-rkɯ kosca ŋu, tɕeri ɯ-ɕna ɣɯ ɯ-komɤr me, ɯ-xtsɤrkɯ ni ci kɤ-kɤ-sɯpa tɕe kɤ-kɤ-tʂɯβ ŋu. ɯ-ɕna kɯ-ɤmtɕoʁ tɕe tu-kɯ-ŋgɤɣ kɯ-fse nɯ tu, ɯ-xsɤrkɯ cho ɯ-xtsɤku ɣɯ kɤ-tʂɯβ nɯ konaʁ xtsa ɣɯ ɯ-tʂɯβ cho mɤ-naχtɕɯɣ, nɯ ɯ-ro nɯ ra konaʁxtsa cho lonba naχtɕɯɣ. palaxtsa nɯ mɤ-mpɕɤr ri ngɯt, kɯ-rɤma kɤ-ŋga nɤtsa.}\hspace{5pt}\pcmn{\lien{ⓔpalaxtsa}{palaxtsa}的鞋边是用没有染过的皮子做成的,脚尖部分没有红皮子。鞋边直接合在一起缝成的,脚尖部分很尖,有钩起来的部分。鞋边和鞋筒连接的缝法和黑皮鞋不一样,其它完全一样。\lien{ⓔpalaxtsa}{palaxtsa}不美观但很结实,适合干活的人穿。}\end{exemple}\end{entrée}

\begin{entrée}{paltsaʁ/\variante{pɤltsaʁ}}{}{ⓔpaltsaʁ} 
\classe{n} 
\begin{définition}\pfra{glaise qui est appliquée sur les plaques de pierres}\end{définition}
\begin{définition}\pcmn{涂在石板上的稀泥}\end{définition}\end{entrée}

\begin{entrée}{pandɤɕku}{}{ⓔpandɤɕku} 
\classe{n} 
\begin{définition}\pfra{renoncule}\end{définition}
\begin{définition}\pcmn{毛茛}\end{définition}
\begin{exemple}\pjya{pandɤɕku nɯ sɯjno kɯ-mbɯ-mbɤr ci ŋu, ɯ-ru cho ɯ-jwaʁ ra kɯ-ɤrŋi ŋu, ɯ-zrɤm xtɕi, stɤmku tʂu ɯ-rkɯ tu-ɬoʁ ŋu, ɯ-mɯntoʁ kɯ-qarŋe ŋu, ɯ-ru ɣɯ ɯ-βzɯr lu-ɕe ŋu, tú-wɣ-ndza tɕe mɤrtsaβ tɕe núndʐa pandɤɕku rmi.}\hspace{5pt}\pcmn{毛茛是矮小的植物,茎和叶子全是绿色的,根细小。生长在山上的路边上,开黄色花,茎上长出很多菱角。吃起来很辣,所以叫\lien{ⓔpandɤɕku}{pandɤɕku}(班地的葱)}\end{exemple}\end{entrée}

\begin{entrée}{pandi}{}{ⓔpandi} 
\classe{n} 
\begin{définition}\pfra{gelugspa}\end{définition}
\begin{définition}\pcmn{黄教}\end{définition}\étymologie{panɖita}\end{entrée}

\begin{entrée}{panja}{}{ⓔpanja} 
\classe{n} 
\begin{définition}\pfra{feu (moine)}\end{définition}
\begin{définition}\pcmn{已故(指和尚)}\end{définition}\end{entrée}

\begin{entrée}{paqe}{}{ⓔpaqe} 
\classe{n} 
\begin{définition}\pfra{purin}\end{définition}
\begin{définition}\pcmn{猪屎}\end{définition}\relationsémantique{参考}{\lien{ⓔpaʁ}{paʁ}}\relationsémantique{参考}{\lien{ⓔtɯ-qe}{tɯ-qe}}\end{entrée}

\begin{entrée}{parɕaŋ}{}{ⓔparɕaŋ} 
\classe{n} 
\begin{définition}\pfra{xylographe}\end{définition}
\begin{définition}\pcmn{印版}\end{définition}
\begin{exemple}\pjya{parɕaŋ pɯ-rkaz-a}\hspace{5pt}\pcmn{我刻了印版}\end{exemple}\étymologie{par.ɕiŋ}\end{entrée}

\begin{entrée}{paʁ}{}{ⓔpaʁ} 
\classe{n} 
\begin{définition}\pfra{porc}\end{définition}
\begin{définition}\pcmn{猪}\end{définition}
\begin{exemple}\pjya{paʁ nɯ kha pjɯ-kɤ-nɯ-χsu ŋu tɕe, ɯ-ɕa kɤ-ndza ɯ-spa ŋu, paʁ nɯ ɯ-rme kɯ-ɲaʁ tu, kɯ-ɤkhra tu, kɯ-ɣɯrni tu, kɯ-wɣrum tu, tsuku ɯ-xtu kɯ-wɣrum ɯ-ro ra kɯ-ɲaʁ tu, ɯ-mi kɯ-wɣrum tu, ɯ-jme ɯ-ku kɯ-wɣrum tu, ɯ-laz tɯ-snaʁ kɯ-wɣrum tu, paʁ kɯ mɤ-ndze ʑo me, tɯ-xpa ʁnɯ-xpa jamar cho-χsu tɕe kɤ-ntɕha tu-βze ŋu, paʁ ɯ-pɯ ta-ndo tɕe, kɯβdɤ-sla tɕe chɯ-rɤpɯ ŋu, ɯ-pɯ nɯ tɯ-ɣjɤn na-lɤt tɕe, kɯ-dɤn sqi sqaptɯɣ jamar kɯ-tu tu, ʁnɯz jamar ma kɯ-me tu, paʁɕa nɯ ɯ-ɕɤmi ɕa cho ɯ-rnom ɯ-ɕa nɯ stu kɯ-mɯm ŋu. paʁɕa laʑu chɯ́-wɣ-tɕɤt khɯ, tɤ-ŋgɤr pjɯ́-wɣ-qaʁ khɯ, tɕhi kɯ-ra tú-wɣ-stu khɯ.}\hspace{5pt}\pcmn{猪是自己在家里喂的,可以用来吃肉。猪有黑毛的,有花毛的,有红毛的,有白毛的。有些腹部是白色的,其它的腹部是黑色的。有的脚是白的,有的尾端白,有的额头上有白点。猪什么都吃,喂了一两年就可以宰了。猪怀小猪要四个月才下,每胎多的有十一头,少的只有两头。猪肉是大腿上和排骨最香。可以弄成腊肉,也可以剥成猪膘,根据需要怎么弄都行。}\end{exemple}\relationsémantique{参考}{\lien{ⓔpaqe}{paqe}}\relationsémantique{参考}{\lien{ⓔpaskɤɣ}{paskɤɣ}}\relationsémantique{参考}{\lien{ⓔpɤjŋgɯ}{pɤjŋgɯ}}\étymologie{pʰag}\end{entrée}

\begin{entrée}{paʁɕɤɣ}{}{ⓔpaʁɕɤɣ} 
\classe{n} 
\begin{définition}\pfra{Ephedra sinica}\end{définition}
\begin{définition}\pcmn{草麻黄}\end{définition}
\begin{exemple}\pjya{paʁɕɤɣ nɯ kɤntɕhɯ-tɯphu tu, si ci tu, sɯjno ci tu. si nɯ wuma sthɯci mɤ-mbro, ɕɤɣ ɯ-ŋgɯ tɯ-tɯphu ŋu. ɯ-jwaʁ nɯ kɯ-ɤrqhi ju-kɯ-ru tɕe, kɯ-ɤndɯndo kɯ-fse ŋu, kɯ-ɤrmbat tɕe, tɕe kɯ-ɤndɯndo maʁ, ɯ-ru ɕɤɣ cho naχtɕɯɣ, ɯ-rtaʁ wuma ʑo xtɯt, lu-rɲɟi mɤ-cha, ɯ-jwaʁ nɯ qartsɯmɤftɕar ʑo ɯ-tɯ-ɤrŋi kɯ nɤmar ʑo, pjɯ́-wɣ-sɤkhɯ tɕe, ɯ-di ɕɤɣ ɣɯ sthɯci mɤ-mɯm. paʁɕɤɣ li sɯjno ci tu tɕe, tɯ-ji ɯ-ŋgɯ tɤ-rɤku ɯ-rca tu-ɬoʁ ŋu, ɯ-zrɤm xtɕi, kɤ-phɯt mbat, ɯ-ru me, ɯ-jwaʁ nɯ taqaβ kɯ-fse tɕe, kɯ-rɲɟi ŋu, ɯ-rtsɤɣ tu. ɯ-rtsɤɣ ɯ-taʁ ɯ-mɯntoʁ ɲɯ-βze cha, ɯ-di ɯʑoz ci tu.}\hspace{5pt}\pcmn{\lien{ⓔpaʁɕɤɣ}{paʁɕɤɣ}有几种,有一种是树,另一种是草。树的那种长得不太高,是柏树的一种。叶子从远方看好像是连在一起的,近看却不是连在一起的。树干和柏树的一样,枝桠很短,长不长,叶子一年四季的都是油绿色的。熏的时候,味道不如柏树香。另一种\lien{ⓔpaʁɕɤɣ}{paʁɕɤɣ} 是草,和庄稼长在一起,根小,容易拔掉,没有茎,叶子像针一样,很长,有节。节上开花。有异样的味道。}\end{exemple}\end{entrée}

\begin{entrée}{paʁɕi}{}{ⓔpaʁɕi} 
\classe{n} 
\begin{définition}\pfra{pomme}\end{définition}
\begin{définition}\pcmn{苹果}\end{définition}\end{entrée}

\begin{entrée}{paʁɟu}{}{ⓔpaʁɟu} 
\classe{n} 
\begin{définition}\pfra{verrat}\end{définition}
\begin{définition}\pcmn{种公猪}\end{définition}\end{entrée}

\begin{entrée}{paʁkɤjlɤβ}{}{ⓔpaʁkɤjlɤβ} 
\classe{n} 
\begin{définition}\pfra{tête de cochon}\end{définition}
\begin{définition}\pcmn{猪头(把骨头取出而剩下的皮肉缝成筒形,两只耳朵各一头,鼻子在中间)}\end{définition}\end{entrée}

\begin{entrée}{paʁkhar}{}{ⓔpaʁkhar} 
\classe{n} 
\begin{définition}\pfra{enclos à cochons}\end{définition}
\begin{définition}\pcmn{猪圈}\end{définition}\étymologie{ⁿkʰor}\end{entrée}

\begin{entrée}{paʁkho}{}{ⓔpaʁkho} 
\classe{n} 
\begin{définition}\pfra{porcherie}\end{définition}
\begin{définition}\pcmn{猪的房间}\end{définition}\end{entrée}

\begin{entrée}{paʁlu}{}{ⓔpaʁlu} 
\classe{n} 
\begin{définition}\pfra{année du porc}\end{définition}
\begin{définition}\pcmn{猪年}\end{définition}\end{entrée}

\begin{entrée}{paʁmu}{}{ⓔpaʁmu} 
\classe{n} 
\begin{définition}\pfra{truie}\end{définition}
\begin{définition}\pcmn{母猪}\end{définition}\end{entrée}

\begin{entrée}{paʁmurɯ}{}{ⓔpaʁmurɯ} 
\classe{n} 
\begin{définition}\pfra{os de cochon conservés dans l'intestin}\end{définition}
\begin{définition}\pcmn{猪骨头(装在大肠里保存)}\end{définition}\end{entrée}

\begin{entrée}{paʁndza}{}{ⓔpaʁndza} 
\classe{n} 
\begin{définition}\pfra{nourriture du cochons}\end{définition}
\begin{définition}\pcmn{猪食}\end{définition}\end{entrée}

\begin{entrée}{paʁtsa}{}{ⓔpaʁtsa} 
\classe{n} 
\begin{définition}\pfra{porcelet}\end{définition}
\begin{définition}\pcmn{猪崽}\end{définition}\end{entrée}

\begin{entrée}{paʁtsarna}{}{ⓔpaʁtsarna} 
\classe{n} 
\begin{définition}\pfra{une plante}\end{définition}
\begin{définition}\pcmn{植物的一种}\end{définition}
\begin{exemple}\pjya{paʁtsarna nɯ sɯjno kɯ-tshu tsa ci ŋu, ɯ-ru cho ɯ-jwaʁ ɯ-ru nɯ ra wuma ʑo mpɯ, wuma ʑo arŋi. ɯ-ru ɯ-ŋgɯ kɯ-so ŋu. ɯ-jwaʁ nɯ kɤ-βzɯr χsɯm ŋu tɕe, paʁrna tsa fse, tɕe núndʐa paʁtsarna ɲɯ-rmi. ɯ-ru ɯ-kɤχcɤl zɯ tɤ-fkɯm kɯ-fse ci ɯ-ku kɯ-ɤmtɕoʁ ci tu-ɬoʁ. ʑɯrɯʑɤri tɕe ɲɯ-mbaʁ tɕe ɯ-mɯntoʁ ɯ-spa kɯɕnom kɯ-fse tu-βze tɕe ɯ-mɯntoʁ kɯ-qarŋe tu-oʑɯrja ɲɯ-ŋu. ki sɯjno ki ɲɯ-tshu ɲɯ-mpɯ tɕe paʁ cho nɯŋa jla ra kɤsɯfse ɲɯ-rga-nɯ, tɯrme kɤ-ndza mɤ-sna.}\hspace{5pt}\pcmn{\lien{ⓔpaʁtsarna}{paʁtsarna} 是一种又肥又嫩的草。茎和叶子的很嫩,很绿。茎是空心的。叶子是三角形的,有点像猪耳朵,所以称它为\lien{ⓔpaʁtsarna}{paʁtsarna}。茎的顶端上长出像口袋一样,上面尖的东西,逐渐破开了以后,里面漏出穗状的花,花是黄色的,成行的排列在茎上。}\end{exemple}\end{entrée}

\begin{entrée}{paʁtshi}{}{ⓔpaʁtshi} 
\classe{n} 
\begin{définition}\pfra{nourriture pour cochon}\end{définition}
\begin{définition}\pcmn{喂猪的}\end{définition}\end{entrée}

\begin{entrée}{paʁzwamɯntoʁ}{}{ⓔpaʁzwamɯntoʁ} 
\classe{n} 
\begin{définition}\pfra{pissenlit}\end{définition}
\begin{définition}\pcmn{蒲公英}\end{définition}
\begin{exemple}\pjya{paʁzwa mɯntoʁ nɯ ɯ-jwaʁ lu-orɤrkhɯrkhe, ɯ-ku lu-omtɕoʁ ŋu, ɯ-spjɯŋ nɯ qhoʁsjɯβ ŋu. ɯ-spjɯŋ tu-ɬoʁ tɕe, ɯ-kɤχcɤl ri ɲɯ-rɯmɯntoʁ ŋu. ɯ-mɯntoʁ nɯ-rom tɕe, tu-ɣɤmɯt tɕe qale rca ju-nɯɕe ŋu. ki sɯjno ki tɤ-ɕqhe smɤn, tɯ-ɕɣa kɯ-mŋɤm smɤn, tshɤtʂot smɤn ŋu}\hspace{5pt}\pcmn{蒲公英叶子不连贯而尖,茎是空心的。茎长出来时,顶上开花。当花凋谢的时候,一吹花絮就顺着风飞走。这种草有治咳、治牙痛,退烧等作用。}\end{exemple}\end{entrée}

\begin{entrée}{paskɤɣ}{}{ⓔpaskɤɣ} 
\classe{n} 
\begin{définition}\pfra{porc à engraisser}\end{définition}
\begin{définition}\pcmn{催肥的猪;肥猪}\end{définition}\relationsémantique{参考}{\lien{ⓔpaʁ}{paʁ}}\relationsémantique{参考}{\lien{ⓔskɤɣ}{skɤɣ}}\end{entrée}

\begin{entrée}{pasrɯ}{}{ⓔpasrɯ} 
\classe{n} 
\begin{définition}\pfra{ivoire}\end{définition}
\begin{définition}\pcmn{象牙}\end{définition}\étymologie{ba.so}\end{entrée}

\begin{entrée}{paχpatʂɯβ}{}{ⓔpaχpatʂɯβ} 
\classe{n} 
\begin{définition}\pfra{type de pas d'aiguille}\end{définition}
\begin{définition}\pcmn{缝针的方法}\end{définition}\end{entrée}

\begin{entrée}{pɤβ}{}{ⓔpɤβ} 
\classe{n} 
\begin{définition}\pfra{natte}\end{définition}
\begin{définition}\pcmn{席子}\end{définition}\end{entrée}

\begin{entrée}{pɤɕɤt}{}{ⓔpɤɕɤt} 
\classe{n} 
\begin{définition}\pfra{pouce}\end{définition}
\begin{définition}\pcmn{一寸}\end{définition}\end{entrée}

\begin{entrée}{pɤɕthɤɣ}{}{ⓔpɤɕthɤɣ} 
\classe{n} 
\begin{définition}\pfra{sangle ventrale}\end{définition}
\begin{définition}\pcmn{马肚带}\end{définition}\end{entrée}

\begin{entrée}{pɤjka}{}{ⓔpɤjka} 
\classe{n} 
\begin{définition}\pfra{courge}\end{définition}
\begin{définition}\pcmn{瓜子的一种【南瓜】}\end{définition}
\begin{exemple}\pjya{pɤjka nɯ lu-kɤ-ji ci ŋu, tɤ-rɤku rca pjɯ́-wɣ-ji ŋu. sɤtɕha kɯ-mpja tsa ɬoʁ, rpɣo pɕoʁ kɯ-mɯɕtaʁ tu-ɬoʁ mɤ-cha. ɯ-jwaʁ ɯ-qhu pɕoʁ ɯ-mdzu dɤn tɕe ɯ-rʁom ʑo tɤjmɤɣ ɯ-sɤ-χtɕi pe, ɯ-jwaʁ wxti, ɯ-βzɯr kɯmŋu tu, ɯ-ru kɯ-zri tsa jɯ-ɕe ŋu, ɯ-taʁ tu-ɬoʁ mɤ-cha, ɯ-thoʁ pjɯ-ɤɲɟoʁ tɕe jɯ-ɕe ŋu, si cho sɯjno ru na-tɯɣ tɕe, li tu-ortɯ-rtɤβ tɕe ɯ-taʁ tu-ɕe cha. ɯ-ru ɯ-taʁ kɯnɤ ɯ-mdzu tu. tɕe ɯ-ru ɯ-taʁ tɕe, ɯ-jwaʁ ku-ndzoʁ tɕe, ɯ-jwaʁ χsɯm jamar nɯ-ɬoʁ tɕe, ɯ-mɯntoʁ ku-ndzoʁ ŋu. ɯ-mɯntoʁ ʁnɯ-tɯphu tɕe, tɯ-tɯphu nɯ ɯ-mat chɯ-βze mɤ-cha tɕe, nɯ phu ŋu tu-ti-nɯ ŋgrɤl, li ci tɯ-tɯphu nɯ, ɯ-mat chɯ-βze cha tɕe, nɯ mu ŋu tu-ti-nɯ ŋu, ɯ-mɯntoʁ ɯ-mdoʁ kɯ-qarŋɯ-rŋe ʑo ŋu, kɯ-ɤrɯlaba ŋu, ɯ-βzɯr kɯmŋu tu. pɤjka ɯ-mat kɯ-wxtɯ-wxti chɯ-βze cha, kɯ-ɤrtɯm tɕe kɯ-zri ŋu. sɤtɕha ɯ-taʁ pjɯ-kɯ-ru nɯ aqarŋɯrŋe, tɯ-mɯ ɯ-pɕoʁ tu-kɯ-ru nɯ, ldʑaŋnaʁ ŋu. pɤjka stu kɯ-wxti nɯ ɣnɤsqi tɯ-rpa jarma chɯ-βze cha. thɯ-do tɕe ɯ-rqhu nɯ rko tɕe ɯ-ɕa ɣɯ ɯ-ŋgɯ ɯ-rɣi chɯ-fka ŋu. pɤjka nɯ kɤ-ndza sna, tɯ-mgo zmɤrɤβ ŋu. thɯ-do tɕe, nɯ fse kú-wɣ-nɯ-sqa tɕe, ɯ-lpɤɣ tú-wɣ-ndza tɕe mɯm.}\hspace{5pt}\pcmn{南瓜是自己种的一种植物,是种在庄稼地里,种在气候比较温和的地方,山上气候冷的地方的能生长。叶子的背面有很多小刺,很粗糙,可以用来刷洗菌子。叶子大,有个棱角,茎可以长得比较长,不是往上长,是爬在地上的。遇到了树干或草茎就会缠着往上长。茎上也有刺。茎上长出三四片叶子的时候就开花。有两种花,一种是不能结果的,一种是不能结果的,有人说它是公的,另一种是可以结果的,有人说它是母的。花是大黄色的,形状像喇叭,有五个棱角。南瓜能结很大的果实,是椭圆形的。朝地的那面是淡黄色是,朝天的那面是深绿色的。最大的南瓜有20多斤左右。老了皮子变硬,种子也就饱满了。南瓜可以吃,是一种菜。老南瓜煮以后,成块的吃了很好吃。}\end{exemple}
\begin{exemple}\pjya{pɤjka tɤ-kɯ-ɬoʁ nɯ, ɯ-mɯntoʁ ku-ndzoʁ ŋu, ɯ-mɯntoʁ kɤ-ndzoʁ tɕe, tɤ-pɤtso (ɯ-ɕki) ``ma-ɕ-kɤ-tɯ-sɯjaʁndze ma mɯntoʁ mbɯt" tu-kɯ-ti ŋu}\hspace{5pt}\pcmn{白瓜开花的时候,给孩子说:“你不要指,不然它的花会掉下来”}\end{exemple}\étymologie{fn:北瓜}\end{entrée}

\begin{entrée}{pɤjkhu}{}{ⓔpɤjkhu} 
\classe{adv} 
\begin{définition}\pfra{encore}\end{définition}
\begin{définition}\pcmn{还;先;暂时}\end{définition}
\begin{exemple}\pjya{pɤjkhu a-ŋga tu-ŋge-a, pɤjkhu a-rte tu-nɤrte-a ra}\hspace{5pt}\pcmn{我还要穿衣服,还要戴帽子}\end{exemple}
\begin{exemple}\pjya{@shiwuhao ri lɤ-ari tɕe, pɤjkhu mɯ-thɯ-nɯɣe}\hspace{5pt}\pcmn{他十五号就去了,还没有回来}\end{exemple}
\begin{exemple}\pjya{pɤjkhu nɤʑo thɯ-ɣi ra}\hspace{5pt}\pcmn{暂时先要你一个人下来}\end{exemple}
\begin{exemple}\pjya{pɤjkhu aʑo mɤ-ndzu-a}\hspace{5pt}\pcmn{我还没有准备好}\end{exemple}\end{entrée}

\begin{entrée}{pɤjmu}{}{ⓔpɤjmu} 
\classe{n} 
\begin{définition}\pfra{beimu}\end{définition}
\begin{définition}\pcmn{贝母}\end{définition}
\begin{exemple}\pjya{pɤjmu ruŋgu}\hspace{5pt}\pcmn{贝母生长的草山(最寒冷的地方)}\end{exemple}\étymologie{fn:贝母}\end{entrée}

\begin{entrée}{pɤjmɤtsɯŋgɤr}{}{ⓔpɤjmɤtsɯŋgɤr} 
\classe{n} 
\begin{définition}\pfra{lard au dessus de la queue}\end{définition}
\begin{définition}\pcmn{猪尾部划出来的膘}\end{définition}\end{entrée}

\begin{entrée}{pɤjŋgɯ}{}{ⓔpɤjŋgɯ} 
\classe{n} 
\begin{définition}\pfra{auge}\end{définition}
\begin{définition}\pcmn{猪槽}\end{définition}\relationsémantique{参考}{\lien{ⓔpaʁ}{paʁ}}\end{entrée}

\begin{entrée}{pɤjpe}{}{ⓔpɤjpe} 
\classe{n} 
\begin{définition}\pfra{pâte aplatie}\end{définition}
\begin{définition}\pcmn{面块}\end{définition}
\begin{exemple}\pjya{kupa pɤjpe}\hspace{5pt}\pcmn{面条}\end{exemple}\end{entrée}

\begin{entrée}{pɤkhije}{}{ⓔpɤkhije} 
\classe{intj} 
\begin{définition}\pfra{attend un peu}\end{définition}
\begin{définition}\pcmn{等一下}\end{définition}\end{entrée}

\begin{entrée}{pɤkɯ}{}{ⓔpɤkɯ} 
\classe{n} 
\begin{définition}\pfra{viande du tronc du cochon}\end{définition}
\begin{définition}\pcmn{猪的排骨}\end{définition}\end{entrée}

\begin{entrée}{pɤlɤjlɯz}{}{ⓔpɤlɤjlɯz} 
\classe{n} 
\begin{définition}\pfra{méthode}\end{définition}
\begin{définition}\pcmn{办法}\end{définition}\end{entrée}

\begin{entrée}{pɤlɤtɕɯ}{}{ⓔpɤlɤtɕɯ} 
\classe{n} 
\begin{définition}\pfra{momo au beurre}\end{définition}
\begin{définition}\pcmn{酥油馍馍}\end{définition}\end{entrée}

\begin{entrée}{pɤliaʁ}{}{ⓔpɤliaʁ} 
\classe{n} 
\begin{définition}\pfra{rouleau à pâtisserie}\end{définition}
\begin{définition}\pcmn{擀面棍}\end{définition}\end{entrée}

\begin{entrée}{pɤŋɤxɕaj}{}{ⓔpɤŋɤxɕaj} 
\classe{n} 
\begin{définition}\pfra{espèce d'herbe}\end{définition}
\begin{définition}\pcmn{草的一种(猪可以吃的)}\end{définition}\relationsémantique{参考}{\lien{ⓔxɕajⓗ1}{xɕaj₁}}\end{entrée}

\begin{entrée}{pɤqa}{}{ⓔpɤqa} 
\classe{n} 
\begin{définition}\pfra{pattes de cochon farcies}\end{définition}
\begin{définition}\pcmn{有肉馅的猪脚}\end{définition}\relationsémantique{参考}{\lien{ⓔpaʁ}{paʁ}}\relationsémantique{参考}{\lien{ⓔtɯ-qa}{tɯ-qa}}\end{entrée}

\begin{entrée}{pɤrmɤloŋ}{}{ⓔpɤrmɤloŋ} 
\classe{n} 
\begin{définition}\pfra{gaspillage}\end{définition}
\begin{définition}\pcmn{浪费}\end{définition}
\begin{exemple}\pjya{pɤrmɤloŋ mɯ-nɯ-ari}\hspace{5pt}\pcmn{没有浪费}\end{exemple}
\begin{exemple}\pjya{pɤrmɤloŋ ma-nɯ-tɯ-sɯxɕe}\hspace{5pt}\pcmn{你别浪费}\end{exemple}\end{entrée}

\begin{entrée}{pɕaʁ}{}{ⓔpɕaʁ} 
\classe{n} 
\begin{définition}\pfra{prosternation}\end{définition}
\begin{définition}\pcmn{礼拜}\end{définition}
\begin{exemple}\pjya{sroŋma ɯ-ʁɤri pɕaʁ lɤ-βzu-t-a}\hspace{5pt}\pcmn{我在护神前做了礼拜}\end{exemple}\relationsémantique{参考}{\lien{ⓔrɤpɕaʁ}{rɤpɕaʁ}}\end{entrée}

\begin{entrée}{pɕawtsɯ}{}{ⓔpɕawtsɯ} 
\classe{n} 
\begin{définition}\pfra{billet de banque}\end{définition}
\begin{définition}\pcmn{钞票}\end{définition}\étymologie{fn:票子}\end{entrée}

\begin{entrée}{pɕintɕɤt}{}{ⓔpɕintɕɤt} 
\classe{adv} 
\begin{définition}\pfra{à partir de (ce moment là)}\end{définition}
\begin{définition}\pcmn{从此以后}\end{définition}
\begin{exemple}\pjya{nɯ ɯ-qhu pɕintɕɤt tɕe}\hspace{5pt}\pcmn{从此以后}\end{exemple}\étymologie{pʰʲin.tɕʰad}\end{entrée}

\begin{entrée}{pɕirɯ}{}{ⓔpɕirɯ} 
\classe{n} 
\begin{définition}\pfra{arrière de la selle}\end{définition}
\begin{définition}\pcmn{后鞍桥}\end{définition}\étymologie{pʰʲi.ru}\end{entrée}

\begin{entrée}{pɕiz}{}{ⓔpɕiz} 
\classe{vt} \paradigme{dir}{nɯ-}\paradigme{dir}{\_}
\begin{définition}\pfra{essuyer}\end{définition}
\begin{définition}\pcmn{擦}\end{définition}
\begin{exemple}\pjya{@zhuozi nɯ-pɕiz}\hspace{5pt}\pcmn{擦一下桌子}\end{exemple}
\begin{exemple}\pjya{nɤ-rŋa pɯ-pɕiz}\hspace{5pt}\pcmn{擦一下脸}\end{exemple}
\begin{exemple}\pjya{nɤ-ŋga nɯ-pɕiz}\hspace{5pt}\pcmn{擦一下衣服}\end{exemple}
\begin{exemple}\pjya{nɤ-ɕnaβ nɯ-pɕiz}\hspace{5pt}\pcmn{擦一下鼻涕}\end{exemple}
\begin{exemple}\pjya{kɯki kɯ a-jaʁ mɯ́j-pɕiz}\hspace{5pt}\pcmn{我的手用这个擦不到}\end{exemple}\étymologie{pʰʲis}\end{entrée}

\begin{entrée}{pɕizgɤkɯm}{}{ⓔpɕizgɤkɯm} 
\classe{n} 
\begin{définition}\pfra{grande porte}\end{définition}
\begin{définition}\pcmn{大门}\end{définition}\étymologie{pʰʲi.sgo}\end{entrée}

\begin{entrée}{pɕoʁβʑi}{}{ⓔpɕoʁβʑi} 
\classe{n} 
\begin{définition}\pfra{dans toutes les directions}\end{définition}
\begin{définition}\pcmn{四周}\end{définition}\étymologie{pʰʲogs.bʑi}\end{entrée}

\begin{entrée}{pɕɯɣ}{}{ⓔpɕɯɣ} 
\classe{idph.1} 
\begin{définition}\pfra{bruit de l'eau qu'on jette, bruit d'une vague qui se brise sur le bord du fleuve}\end{définition}
\begin{définition}\pcmn{泼水的声音、波浪撞水边的声音}\end{définition}
\begin{exemple}\pjya{tɯ-ci pɕɯɣ ʑo ta-lɤt}\hspace{5pt}\pcmn{他扑一声把水泼了}\end{exemple}
\begin{sous-entrée}{ɣɤpɕɯlɯɣ}{ⓔpɕɯɣⓝɣɤpɕɯlɯɣ}
\begin{définition}\pfra{asperger dans tous les sens}\end{définition}
\begin{définition}\pcmn{泼过去泼过来}\end{définition}
\begin{exemple}\pjya{khɯtsa ɯ-ŋgɯ tɯ-ci tɯ-asɯ-ndo tɕe, dal tsa tɤ-ŋke ma ɲɯ-ɣɤpɕɯlɯɣ ʑo tɕe jit}\hspace{5pt}\pcmn{你在碗里端水,走路慢一点,不然就会到处流出来}\end{exemple}\end{sous-entrée}

\begin{sous-entrée}{sɤpɕɯlɯɣ}{ⓔpɕɯɣⓝsɤpɕɯlɯɣ} 
\classe{vt} 
\begin{définition}\pfra{faire tomber de l'eau dans tous les sens (en transportant un bol)}\end{définition}
\begin{définition}\pcmn{端水的时候走路不稳当,把水泼过去泼过来}\end{définition}
\begin{exemple}\pjya{tɯ-ci ɲɯ-sɤpɕɯɣlɯɣ}\hspace{5pt}\pcmn{他把水泼过去泼过来}\end{exemple}\end{sous-entrée}

\end{entrée}

\begin{entrée}{pe}{}{ⓔpe} 
\classe{vs} \paradigme{dir}{tɤ-}\sens{1}
\begin{définition}\pfra{bien}\end{définition}
\begin{définition}\pcmn{好}\end{définition}
\begin{exemple}\pjya{nɯ ɲɯ-pe mɤ-kɯ-pe maŋe}\hspace{5pt}\pcmn{那很好,没有错误}\end{exemple}
\begin{exemple}\pjya{nɤʑɯɣ nɤ-kha ra ɯ-ɲɯ-pe-nɯ?}\hspace{5pt}\pcmn{你家的人好不好?}\end{exemple}
\begin{exemple}\pjya{nɯ kɯ-fse kɯ-pe me}\hspace{5pt}\pcmn{那是最好的了,没有比这个好的}\end{exemple}
\begin{exemple}\pjya{mɯ-tu-tɯ-pe mɤ-βdi ma tɯ-kɤ-stu pjɯ-tu ra}\hspace{5pt}\pcmn{如果你哪一天遇到不幸的事情,就要有解决的办法}\end{exemple}
\begin{exemple}\pjya{nɯ ɕɯŋgɯ aʑo mɯ-tɤ-pe-a zɯ, ɯʑo wuma ʑo pɯ-stu}\hspace{5pt}\pcmn{以前我遇到问题的时候,他帮了我很多}\end{exemple}\sens{2}\paradigme{dir}{tɤ-}
\begin{définition}\pfra{bon}\end{définition}
\begin{définition}\pcmn{善良}\end{définition}
\begin{exemple}\pjya{tɤ-pɤtso pɯ-ŋke ri mɤ-kɯ-pe kɯ-me ʑo pɯ-ndʐaβ}\hspace{5pt}\pcmn{小孩子走路的时候,没有什么原因就摔倒了}\end{exemple}\relationsémantique{反义词}{\lien{ⓔŋɤn}{ŋɤn}}
\begin{sous-entrée}{nɤpe}{ⓔpeⓢ2ⓝnɤpe} 
\classe{vt} \end{sous-entrée}

\sens{1}
\begin{définition}\pfra{aimer}\end{définition}
\begin{définition}\pcmn{喜欢}\end{définition}\sens{2}
\begin{définition}\pfra{être content de}\end{définition}
\begin{définition}\pcmn{因为……觉得高兴}\end{définition}
\begin{sous-entrée}{sɤpe}{ⓔpeⓢ2ⓝsɤpe} 
\classe{vt} 
\begin{définition}\pfra{améliorer}\end{définition}
\begin{définition}\pcmn{做得更好}\end{définition}\end{sous-entrée}

\end{entrée}

\begin{entrée}{pɣa}{}{ⓔpɣa} 
\classe{n} 
\begin{définition}\pfra{oiseau}\end{définition}
\begin{définition}\pcmn{鸟}\end{définition}\end{entrée}

\begin{entrée}{pɣaʁ}{}{ⓔpɣaʁ} 
\classe{vt} \sens{1}\paradigme{dir}{\_}
\begin{définition}\pfra{retourner}\end{définition}
\begin{définition}\pcmn{打翻}\end{définition}
\begin{exemple}\pjya{nɤ-ŋga ɲɤ-tɯ-pɣaʁ}\hspace{5pt}\pcmn{你把衣服穿反了}\end{exemple}
\begin{exemple}\pjya{sɯpɣo tha-pɣaʁ}\hspace{5pt}\pcmn{他把柴垛弄翻了}\end{exemple}
\begin{exemple}\pjya{a-ŋga kɤ-ŋga ɲɤ-nɯ-pɣaʁ-a ri, mɯ-pjɤ-sɯχsal-a}\hspace{5pt}\pcmn{我把衣服穿反了但是没有注意}\end{exemple}\sens{2}
\begin{définition}\pfra{labourer}\end{définition}
\begin{définition}\pcmn{耕(地)}\end{définition}
\begin{exemple}\pjya{tɯji la-pɣaʁ, lɤ-pɣaʁ-a}\hspace{5pt}\pcmn{他耕了地,我耕了地}\end{exemple}
\begin{exemple}\pjya{stɤmku la-pɣaʁ}\hspace{5pt}\pcmn{他耕了草坪}\end{exemple}\sens{3}
\begin{définition}\pfra{faire s'effondrer (mur)}\end{définition}
\begin{définition}\pcmn{推倒(墙)}\end{définition}
\begin{exemple}\pjya{znde tha-pɣaʁ}\hspace{5pt}\pcmn{他把墙推倒了}\end{exemple}\sens{4}\paradigme{dir}{\_}
\begin{définition}\pfra{ouvrir (couvercle)}\end{définition}
\begin{définition}\pcmn{揭开}\end{définition}
\begin{définition}\pfra{se retourner}\end{définition}
\begin{définition}\pcmn{翻身}\end{définition}
\begin{exemple}\pjya{ɯ-fkaβ tɤ-pɣaʁ-a (=tɤ-mɟa-t-a)}\hspace{5pt}\pcmn{我揭开了盖子}\end{exemple}
\begin{exemple}\pjya{aʑo stɤmku ri pɯ-rŋgɯ-a tɕe thɯ-ʑɣɤpɣaʁ-a ma a-pa qajɯ ɣɤʑu}\hspace{5pt}\pcmn{我躺在草坪的时候翻了身,因为我的下面有虫子}\end{exemple}
\begin{exemple}\pjya{thɯ-ʑɣɤpɣaʁ nɤ thɯ-ʑɣɤpɣaʁ}\hspace{5pt}\pcmn{他(在地上)滚动}\end{exemple}\relationsémantique{参考}{\lien{ⓔmbɣaʁ}{mbɣaʁ}}\relationsémantique{参考}{\lien{ⓔapɣaʁsci}{apɣaʁsci}}
\begin{sous-entrée}{rɤpɣaʁ}{ⓔpɣaʁⓢ4ⓝrɤpɣaʁ} 
\classe{vi}  
\grammaire{apass} 
\begin{définition}\pfra{défricher}\end{définition}
\begin{définition}\pcmn{开荒}\end{définition}\end{sous-entrée}

\begin{sous-entrée}{ʑɣɤpɣaʁ}{ⓔpɣaʁⓢ4ⓝʑɣɤpɣaʁ} 
\classe{vi}  
\grammaire{refl} \end{sous-entrée}

\end{entrée}

\begin{entrée}{pɣatɤste}{}{ⓔpɣatɤste} 
\classe{n} 
\begin{définition}\pfra{espèce de rapace}\end{définition}
\begin{définition}\pcmn{猛禽的一种}\end{définition}
\begin{exemple}\pjya{pɣatɤste nɯ qale kɯtshi cho naχtɕɯɣ, qajdo sɤznɤ xtɕi, ɯ-βri kɯ-pɣi ŋu, rɯdaʁ kɯ-xtɕi ra tu-ndze ŋgrɤl, tɤ-rɤku tu-ndze mɤ-ŋgrɤl.}\hspace{5pt}\pcmn{\lien{ⓔpɣatɤste}{pɣatɤste}和\lien{ⓔqalekɯtshi}{qalekɯtshi}相似,比乌鸦小,身子是灰色的,吃小动物,不吃粮食。}\end{exemple}\end{entrée}

\begin{entrée}{pɣɤjmɤt}{}{ⓔpɣɤjmɤt} 
\classe{n} 
\begin{définition}\pfra{type d'herbe}\end{définition}
\begin{définition}\pcmn{草的一种}\end{définition}\end{entrée}

\begin{entrée}{pɣɤkhɯ}{}{ⓔpɣɤkhɯ} 
\classe{n} 
\begin{définition}\pfra{hibou}\end{définition}
\begin{définition}\pcmn{猫头鹰}\end{définition}
\begin{exemple}\pjya{pɣɤkhɯ nɯ pɣa ci ŋu, wxti tsa qandʑɣi jamar tu, lɤŋɤtʂɤ-tɯrpa jamar zɣɯt, ɯ-phoŋbu nɯ kɯ-pɣi tɕe kɯ-qandʐi tsa ŋu, ɯ-ku nɯ lɯlu ku tsa fse ri pɣa ŋu tɕe ɯ-mtsioʁ tu, ɯ-mtsioʁ moŋtaʁ ɣɯ ɯ-ku nɯ lu-ŋgɤɣ tsa ɲɯ-ŋu, ɯ-mɲaʁ nɯ ɕɤr tɕe ɲɯ-mto ma sŋi tɕe mɯ́j-mto tɕe ɯʑo ɯ-kɤ-ndza kɯ-ɕar nɯ ɕɤr ʁɟa ju-ɬoʁ ɲɯ-ŋu, tɕe βʑɯ tu-ndze, qaɲi tu-ndze, qala kɯ-fse nɯ ra tu-ndze ɲɯ-ŋu. ɕɤr tɕe tu-mbri ŋgrɤl, tɕe `uhu' tu-ti ɲɯ-ŋu, tɕe tsuku tɯrme wuma ʑo kɯ-nɯrmɯ kɯ-nɯmtɕi tu-nɯ tɕe ɕɤr ɲɯ-kɯ-rɤma-nɯ pɣɤkhɯ tu-sɤrmi-nɯ ŋgrɤl.}\hspace{5pt}\pcmn{猫头鹰是一种鸟,和隼一样大,可以达到五六斤重,身子灰里带黑。头有点像猫,但有鸟喙,因为是鸟。上鸟喙的顶端是钩着的。眼睛只有夜里才看见东西,白天看不见。它只有晚上才出来觅食,吃老鼠、鼹鼠、兔子等动物。它会在夜里叫,叫声是‘uhu’。所以人们把喜欢晚收工,早出工,晚上工作的人称作“猫头鹰”。}\end{exemple}\end{entrée}

\begin{entrée}{pɣɤloʁ}{}{ⓔpɣɤloʁ} 
\classe{n} 
\begin{définition}\pfra{nid}\end{définition}
\begin{définition}\pcmn{鸟巢}\end{définition}\end{entrée}

\begin{entrée}{pɣɤlpɣɤl}{}{ⓔpɣɤlpɣɤl} 
\classe{idph.2} 
\begin{définition}\pfra{brillant}\end{définition}
\begin{définition}\pcmn{形容光芒耀眼,金光闪闪的样子}\end{définition}
\begin{exemple}\pjya{χɕɤl ɲɯ-nɤmbju pɣɤlpɣɤl ʑo}\hspace{5pt}\pcmn{玻璃金光闪闪}\end{exemple}
\begin{exemple}\pjya{ʁmbɣi pɣɤlpɣɤl ʑo ɲɤ-ɬoʁ}\hspace{5pt}\pcmn{太阳出来了,光芒四射}\end{exemple}
\begin{sous-entrée}{pɣɤlnɤpɣɤl}{ⓔpɣɤlpɣɤlⓝpɣɤlnɤpɣɤl} 
\classe{idph.3} \sens{1}
\begin{définition}\pfra{scintillant}\end{définition}
\begin{définition}\pcmn{形容一闪一闪的样子}\end{définition}\end{sous-entrée}

\sens{2}
\begin{définition}\pfra{en marchant}\end{définition}
\begin{définition}\pcmn{形容一步一步走动的样子}\end{définition}
\begin{sous-entrée}{pɣɤlpɣɤl nɤ pɣɤlpɣɤl}{ⓔpɣɤlpɣɤlⓢ2ⓝpɣɤlpɣɤl nɤ pɣɤlpɣɤl} 
\classe{idph.3} 
\begin{définition}\pfra{en courant}\end{définition}
\begin{définition}\pcmn{形容步子迈得快的样子}\end{définition}
\begin{exemple}\pjya{pɣɤlpɣɤl nɤ pɣɤlpɣɤl ʑo pjɤ-ɣi}\hspace{5pt}\pcmn{她跑回来了}\end{exemple}\end{sous-entrée}

\end{entrée}

\begin{entrée}{pɣɤmbri}{}{ⓔpɣɤmbri} 
\classe{n} 
\begin{définition}\pfra{chant d'oiseau}\end{définition}
\begin{définition}\pcmn{鸟的叫声}\end{définition}\relationsémantique{参考}{\lien{ⓔmbriⓗ1}{mbri₁}}\relationsémantique{参考}{\lien{ⓔpɣa}{pɣa}}\end{entrée}

\begin{entrée}{pɣɤmuj}{}{ⓔpɣɤmuj} 
\classe{n} 
\begin{définition}\pfra{plumes d'oiseau}\end{définition}
\begin{définition}\pcmn{羽毛}\end{définition}\relationsémantique{参考}{\lien{ⓔtɤ-muj}{tɤ-muj}}\relationsémantique{参考}{\lien{ⓔpɣa}{pɣa}}\end{entrée}

\begin{entrée}{pɣɤɲaʁ}{}{ⓔpɣɤɲaʁ} 
\classe{n} 
\begin{définition}\pfra{faisan (pucrasia macrolopha)}\end{définition}
\begin{définition}\pcmn{勺鸡}\end{définition}\relationsémantique{参考}{\lien{ⓔɲaʁ}{ɲaʁ}}\end{entrée}

\begin{entrée}{pɣɤrnoʁ}{}{ⓔpɣɤrnoʁ} 
\classe{n} 
\begin{définition}\pfra{une espèce de champignon}\end{définition}
\begin{définition}\pcmn{【鸡油菌】}\end{définition}
\begin{exemple}\pjya{pɣɤrnoʁ nɯ si kɯ-xtɕi tsa ɯ-taʁ ku-ndzoʁ tɕe kɯ-qarŋe tsa ŋu, kɤ-ndza sna}\hspace{5pt}\pcmn{鸡油菌是长在小树上的一种菌子,呈黄色,可以吃。}\end{exemple}\relationsémantique{参考}{\lien{ⓔtɯ-rnoʁ}{tɯ-rnoʁ}}\end{entrée}

\begin{entrée}{pɣɤrtsɤɣ}{}{ⓔpɣɤrtsɤɣ} 
\classe{n} 
\begin{définition}\pfra{maladie de la peau de la main}\end{définition}
\begin{définition}\pcmn{疣}\end{définition}\end{entrée}

\begin{entrée}{pɣɤtɕɯtɤŋgɤr}{}{ⓔpɣɤtɕɯtɤŋgɤr} 
\classe{n} 
\begin{définition}\pfra{une plante}\end{définition}
\begin{définition}\pcmn{植物的一种}\end{définition}
\begin{exemple}\pjya{pɣɤtɕɯ tɤŋgɤr nɯ stɤmku tu-ɬoʁ ŋu. ɯ-jwaʁ nɯ kɯ-ɤrtɯm ŋu, tɕe ɯ-thoʁ pjɯ-ɤɲɟoʁ ʑo ɲɯ-ŋu. ɯ-jwaʁ ʁnɯz ma maŋe, kɯ-jɯ-jaʁ ɲɯ-ŋu, tɕe ɲɯ-ndoʁ. pjɯ́-wɣ-qrɯt tɕe ɯ-ci kɯ-ɤrɤmtʂɯmtʂaj kɯ-fse ɲɯ-ɬoʁ ɲɯ-ŋu. ɯ-ru tu-ɬoʁ tɕe, ɯ-ru ʁnɯ-tɣa jamar tɤ-zri tɕe, ɯ-mɯntoʁ ɲɯ-ɬoʁ ɲɯ-ŋu. ɯ-mɯntoʁ pa zɯ tɤ-fkɯm kɯ-fse kɯ-xtɕɯ-xtɕi ɣɤʑɯ tɕe, ɯ-mɯntoʁ ɯ-pa pjɤ-ɴqoʁ, ɯ-mŋu ɯ-taʁ tu-ru ɲɯ-ŋu. ɯ-ŋgɯ tɯ-ci ɲɯ-mtshɤt tɕe, ɲɯ-chi. ɯ-mɯntoʁ nɯ kɯ-qarŋe tɕe, staχpɯ ɯ-mɯntoʁ tsa ɲɯ-fse. ɯ-mɯntoʁ kɯ-dɤn tu-oʑɯrja ɲɯ-ŋu.}\hspace{5pt}\pcmn{\lien{}{pɣɤtɕɯ tɤ-ŋgɤr}生长在草地上。叶子是圆形的,贴在地面上。只有两片叶子,又厚又脆,掰开时里面流出有粘性的液体。茎长出两拃的时候就开花。花的下面有一个像口袋一样的小东西,吊在花的下面,口朝上。里面装满水,很甜。花是黄色的,像豌豆的花一样,花很多,成行地长在茎上。}\end{exemple}\end{entrée}

\begin{entrée}{pɣɤzraʁ}{}{ⓔpɣɤzraʁ} 
\classe{n} 
\begin{définition}\pfra{espèce d'oiseau}\end{définition}
\begin{définition}\pcmn{一种鸟}\end{définition}
\begin{exemple}\pjya{pɣɤzraʁ nɯ pɣa kɯ-xtɕi tsa ci ŋu, ɯ-phoŋbu nɯ tɤŋkhɯt jamar ma me, ɯ-βri kɯ-pɣi ɯ-taʁ kɯ-ɲaʁ kɯ-ɤkhra ŋu, zgo ɯ-χcɤl ɕaŋtaʁ sɤtɕha kɯ-mbro ku-rɤʑi ŋu, qajɯ kɯ-fse, sɯmat kɯ-fse tu-ndze ŋgrɤl. kɯmŋu kɯtʂɤɣ jamar tɯtɯrca ku-rɤʑi-nɯ ŋu. qartsɯmɤftɕar kɤ-mto tu ɕti. ɯ-mtɕhi nɯ kɯ-ɲaʁ ŋu, ɯ-mi nɯ ra kɯ-wɣrum tɕe kɯ-pɣi kɯ-fse ŋu. tɤ-rɤku ɯ-taʁ rɯŋɯŋɤn mɤ-ŋgrɤl}\hspace{5pt}\pcmn{\lien{ⓔpɣɤzraʁ}{pɣɤzraʁ}是一种比较小的鸟,整个身子只有拳头那么大,身子是灰色上面有黑条纹。栖息在半山以上,海拔高的地方,吃虫子和野果等,五六只成群一起生活。一年四季都可以看到它。嘴是黑色的,脚是白里带灰色。一般不会破坏庄稼。}\end{exemple}\end{entrée}

\begin{entrée}{pɣi}{}{ⓔpɣi} 
\classe{vs} \paradigme{dir}{nɯ-}
\begin{définition}\pfra{gris}\end{définition}
\begin{définition}\pcmn{灰}\end{définition}\end{entrée}

\begin{entrée}{pɣo}{}{ⓔpɣo} 
\classe{vt} \paradigme{dir}{lɤ-}
\begin{définition}\pfra{faire de la ficelle en roulant dans les mains (dans le sens des aiguilles d'une montre)}\end{définition}
\begin{définition}\pcmn{捻线(顺时针)}\end{définition}
\begin{exemple}\pjya{smɤɣ lɤ-pɣo-t-a}\hspace{5pt}\pcmn{我捻了羊毛}\end{exemple}
\begin{exemple}\pjya{srɯn lɤ-pɣo-t-a}\hspace{5pt}\pcmn{我捻了棉花}\end{exemple}
\begin{exemple}\pjya{tɤ-rme lɤpɣo-t-a}\hspace{5pt}\pcmn{我捻了毛}\end{exemple}
\begin{sous-entrée}{nɯɣɯpɣo}{ⓔpɣoⓝnɯɣɯpɣo} 
\classe{vs} 
\begin{définition}\pfra{facile à enfiler}\end{définition}
\begin{définition}\pcmn{容易捻}\end{définition}\end{sous-entrée}

\end{entrée}

\begin{entrée}{pɣotaʁ}{}{ⓔpɣotaʁ} 
\classe{n} 
\begin{définition}\pfra{tissage}\end{définition}
\begin{définition}\pcmn{吊线和织布}\end{définition}\relationsémantique{参考}{\lien{ⓔpɣo}{pɣo}}\relationsémantique{参考}{\lien{ⓔtaʁⓗ1}{taʁ₁}}\end{entrée}

\begin{entrée}{phu}{}{ⓔphu} 
\classe{vs} 
\begin{définition}\pfra{fier}\end{définition}
\begin{définition}\pcmn{有自信}\end{définition}
\begin{exemple}\pjya{ɯ-sɯm ɲɯ-phu}\hspace{5pt}\pcmn{他很有自信}\end{exemple}\end{entrée}

\begin{entrée}{pha}{}{ⓔpha} 
\classe{n} 
\begin{définition}\pfra{entier, complet}\end{définition}
\begin{définition}\pcmn{整个}\end{définition}
\begin{exemple}\pjya{pha ɯ-ku ʑo cho-wɣrum}\hspace{5pt}\pcmn{整个头都白了}\end{exemple}
\begin{exemple}\pjya{pha ɯ-phoŋbu ʑo ɲɯ-mŋɤm}\hspace{5pt}\pcmn{他全身痛}\end{exemple}
\begin{exemple}\pjya{pha kɤntɕhaʁ ʑo ɕ-tɤ-khat-a}\hspace{5pt}\pcmn{我走了遍了整个城市}\end{exemple}\étymologie{pʰal(tɕʰer)}\end{entrée}

\begin{entrée}{phaloŋri}{}{ⓔphaloŋri} 
\classe{adv} 
\begin{définition}\pfra{pleins d'endroits (dans la montagne)}\end{définition}
\begin{définition}\pcmn{很多地方(山川)}\end{définition}
\begin{exemple}\pjya{phaloŋri ʑo ɕ-to-khɤt/ɕ-to-ŋke}\hspace{5pt}\pcmn{他走遍了所有的山川}\end{exemple}\end{entrée}

\begin{entrée}{phama}{}{ⓔphama} 
\classe{n} 
\begin{définition}\pfra{parents}\end{définition}
\begin{définition}\pcmn{父母}\end{définition}\étymologie{pʰa.ma}\end{entrée}

\begin{entrée}{phantsɯt}{}{ⓔphantsɯt} 
\classe{n} 
\begin{définition}\pfra{assiette}\end{définition}
\begin{définition}\pcmn{盘子}\end{définition}\étymologie{fn:盘子}\end{entrée}

\begin{entrée}{phaʁ}{}{ⓔphaʁ} 
\classe{vt} \paradigme{dir}{nɯ-}\paradigme{dir}{thɯ-}\paradigme{dir}{tɤ-}
\begin{définition}\pfra{couper}\end{définition}
\begin{définition}\pcmn{劈;剖}\end{définition}
\begin{exemple}\pjya{si nɯ-phaʁ-a}\hspace{5pt}\pcmn{我劈了木头}\end{exemple}
\begin{exemple}\pjya{ɕoŋtɕa nɯ-phaʁ-a}\hspace{5pt}\pcmn{我劈了柴}\end{exemple}
\begin{exemple}\pjya{tɤ-fkɯm nɯ-phaʁ-a}\hspace{5pt}\pcmn{我把口袋弄坏了(装的东西太多就破了)}\end{exemple}
\begin{exemple}\pjya{tɯ-ŋga thɯ-phaʁ-a}\hspace{5pt}\pcmn{我撕了衣服}\end{exemple}
\begin{exemple}\pjya{tɤ-ri thɯ-phaʁ-a}\hspace{5pt}\pcmn{我撕了线}\end{exemple}
\begin{exemple}\pjya{paχɕi ɯ-sɤ-phaʁ maŋe, mbrɯtɕɯ a-pɯ-tu tɕe pe ri}\hspace{5pt}\pcmn{没有东西切苹果,有刀子就好了}\end{exemple}\relationsémantique{参考}{\lien{ⓔmbaʁ}{mbaʁ}}\end{entrée}

\begin{entrée}{phaʁlu}{}{ⓔphaʁlu} 
\classe{n} 
\begin{définition}\pfra{année du porc}\end{définition}
\begin{définition}\pcmn{猪年}\end{définition}\étymologie{pʰag.lo}\end{entrée}

\begin{entrée}{phaʁɲɤl}{}{ⓔphaʁɲɤl} 
\classe{n} 
\begin{définition}\pfra{fait de s'allonger sur le côté}\end{définition}
\begin{définition}\pcmn{半身躺着}\end{définition}\relationsémantique{参考}{\lien{ⓔtɯ-phaʁ}{tɯ-phaʁ}}\relationsémantique{参考}{\lien{ⓔnɯphaʁɲɤl}{nɯphaʁɲɤl}}\end{entrée}

\begin{entrée}{phaʁrgot}{}{ⓔphaʁrgot} 
\classe{n} 
\begin{définition}\pfra{sanglier}\end{définition}
\begin{définition}\pcmn{野猪}\end{définition}\étymologie{pʰag.rgod}\end{entrée}

\begin{entrée}{phaʁrzi}{}{ⓔphaʁrzi} 
\classe{n} 
\begin{définition}\pfra{brosse à dent traditionnelle en poil de cochon}\end{définition}
\begin{définition}\pcmn{用猪鬃毛作成的牙刷}\end{définition}\étymologie{pʰag.ze}\end{entrée}

\begin{entrée}{phaʁzlasqamŋu}{}{ⓔphaʁzlasqamŋu} 
\classe{n} 
\begin{définition}\pfra{deux semaines}\end{définition}
\begin{définition}\pcmn{半个月}\end{définition}\étymologie{zla}\end{entrée}

\begin{entrée}{phaʁzoŋ}{}{ⓔphaʁzoŋ} 
\classe{n} 
\begin{définition}\pfra{sûtra lu pour les cochons lorsqu'ils sont malades}\end{définition}
\begin{définition}\pcmn{为猪念的经}\end{définition}\étymologie{pʰag.??}\end{entrée}

\begin{entrée}{phɤβ}{₂}{ⓔphɤβⓗ2} 
\classe{n} 
\begin{définition}\pfra{ferment de vin}\end{définition}
\begin{définition}\pcmn{酿酒用的曲子}\end{définition}\end{entrée}

\begin{entrée}{phɤβ}{₁}{ⓔphɤβⓗ1} 
\classe{vt} \sens{1}\paradigme{dir}{pɯ-}\paradigme{dir}{thɯ-}
\begin{définition}\pfra{abaisser}\end{définition}
\begin{définition}\pcmn{弄低,降下来}\end{définition}
\begin{exemple}\pjya{a-ku pɯ-phaβ-a ma nɯ-rpe-a ɲɯ-ŋu}\hspace{5pt}\pcmn{我低了头,因为差一点撞了}\end{exemple}
\begin{exemple}\pjya{nɤ-ku ko-tɯ-nɯ-rpu-t tɕe pjɤ-tɯ-phɤβ pjɤ-ra}\hspace{5pt}\pcmn{你撞了头,你本来应该低头}\end{exemple}
\begin{exemple}\pjya{nɤ-ku pɯ-phɤβ ma tɯ-nɯ-rpe}\hspace{5pt}\pcmn{你低头,小心撞了}\end{exemple}
\begin{exemple}\pjya{ki laχtɕha ɯ-phɯ ɲɯ-wxti, ɯ-koŋ pɯ-phɤβ}\hspace{5pt}\pcmn{这个东西很贵,价格卖便宜一点吧}\end{exemple}\relationsémantique{参考}{\lien{ⓔmbɤβⓗ2}{mbɤβ₂}}\sens{2}\paradigme{dir}{pɯ-}
\begin{définition}\pfra{peigner}\end{définition}
\begin{définition}\pcmn{梳理(头发)}\end{définition}
\begin{exemple}\pjya{nɤ-kɤrme pɯ-phɤβ}\hspace{5pt}\pcmn{梳理一下头发}\end{exemple}\sens{3}\paradigme{dir}{nɯ-}
\begin{définition}\pfra{appliquer une couche de graisse (sur les poteries qui viennent de sortir du four)}\end{définition}
\begin{définition}\pcmn{上油(给出窑的坛子)}\end{définition}
\begin{exemple}\pjya{tɯfcɤr nɯ-phɤβ-i}\hspace{5pt}\pcmn{我们给刚出窑的坛子上了一层油}\end{exemple}\end{entrée}

\begin{entrée}{phɤn}{}{ⓔphɤn} 
\classe{vs} \paradigme{dir}{tɤ-}
\begin{définition}\pfra{avoir un (bon) effet}\end{définition}
\begin{définition}\pcmn{有效果,起作用}\end{définition}
\begin{exemple}\pjya{pɯ-me mɤ-phɤn ma ɣɯ-ndza ra, pɯ-tu mɤ-phɤn ma ɣɯ-ɕtʂat ra}\hspace{5pt}\pcmn{粮食再少也不能不吃,粮食再多也不能不节约}\end{exemple}
\begin{sous-entrée}{ɣɤphɤn}{ⓔphɤnⓝɣɤphɤn} 
\classe{vt} 
\begin{définition}\pfra{donner un effet}\end{définition}
\begin{définition}\pcmn{让……起作用}\end{définition}
\begin{exemple}\pjya{kɤ-rɤβzjoz tu-kɯ-stu tɕe kɤ-spa ɣɯ ɯ-kɯ-ɣɤphɤn ŋu}\hspace{5pt}\pcmn{只有努力才有可能学会}\end{exemple}\end{sous-entrée}

\étymologie{pʰan}\end{entrée}

\begin{entrée}{phɤnba}{}{ⓔphɤnba} 
\classe{n} 
\begin{définition}\pfra{bienfait}\end{définition}
\begin{définition}\pcmn{好处}\end{définition}
\begin{exemple}\pjya{kɯrɯ skɤt pjɯ-βzjoz-a tɕe a-phɤnba tu}\hspace{5pt}\pcmn{学藏语对我是有用的}\end{exemple}\end{entrée}

\begin{entrée}{phɤnthoʁ}{}{ⓔphɤnthoʁ} 
\classe{n} 
\begin{définition}\pfra{avantage}\end{définition}
\begin{définition}\pcmn{好处}\end{définition}\relationsémantique{参考}{\lien{ⓔphɤn}{phɤn}}\étymologie{pʰan.tʰog}\end{entrée}

\begin{entrée}{phɤr}{}{ⓔphɤr} 
\classe{vt} \paradigme{dir}{nɯ-}\paradigme{dir}{kɤ-}
\begin{définition}\pfra{secouer, agiter pour faire tout tomber d'un récipient}\end{définition}
\begin{définition}\pcmn{抖掉;倒完}\end{définition}
\begin{exemple}\pjya{nɤki tɤ-fkɯm ɯ-ŋgɯ kɤ-phɤr}\hspace{5pt}\pcmn{你(把青稞)抖进口袋里吧}\end{exemple}
\begin{exemple}\pjya{ɯ-kɯr ɲɤ-phɤr}\hspace{5pt}\pcmn{我张开了嘴巴(目瞪口呆)}\end{exemple}
\begin{exemple}\pjya{pjɤ-nɤscɤr tɕe, ɯ-kɯr ɲɤ-phɤr}\hspace{5pt}\pcmn{我被吓到了,就张开了嘴巴}\end{exemple}\relationsémantique{参考}{\lien{ⓔsɤphɤr}{sɤphɤr}}\relationsémantique{参考}{\lien{ⓔmbɤrⓗ2}{mbɤr₂}}\end{entrée}

\begin{entrée}{phɤri}{}{ⓔphɤri} 
\classe{adv} 
\begin{définition}\pfra{de l'autre côté}\end{définition}
\begin{définition}\pcmn{对面;对岸}\end{définition}
\begin{exemple}\pjya{kɯchu phɤri, ndɯchu phɤri}\end{exemple}\étymologie{pʰa.rol}\end{entrée}

\begin{entrée}{phɤrikɯnɤwu}{}{ⓔphɤrikɯnɤwu} 
\classe{n} 
\begin{définition}\pfra{une plante}\end{définition}
\begin{définition}\pcmn{植物的一种}\end{définition}
\begin{exemple}\pjya{phɤri kɯnɤwu nɯ kha ɯ-rkɯ sɯjno kɯ-dɤn kɯ-fse fsapaʁ ɯ-ɣli kɯ-dɤn kɯ-fse ra tu-ɬoʁ ŋu. tɯrme tɯ-fsu jamar tu-βze cha. ɯ-jwaʁ nɯ ɕɤɣ ɯ-jwaʁ ɯ-tshɯɣa fse ri nɯ sɤz rɟum, ɯ-ru nɯ khro mɤ-jpum, ɯ-ru kɯ-zɯ-zri tɤ-ɬoʁ tɕe, ɯ-taʁ ɯ-mɯntoʁ ku-ndzoʁ ŋu, tɕe ɯ-mɯntoʁ ɯ-tshɯɣa nɯ stoʁ ɯ-mɯntoʁ fse, ɯ-mdoʁ nɯ nɤmkha ɯ-mdoʁ kɯ-ɤɲaʁndzɯm tsa ŋu. ɯʑo sɯjno nɯ tɯ-ɕɣa ɯ-smɤn kɤ-βʑu ɲɯ-khɯ khi, kɯɕɯŋgɯ kɯ-wxti ra kɯ phɤri kɯnɤwu tu-sɤrmi-nɯ pɯ-ŋu. kɤ-ndza mɤ-sna. fsapaʁ ndza kɯnɤ mɤ-sna}\hspace{5pt}\pcmn{\lien{}{phɤri kɯnɤwu} 是长在房子周围,草多、牲畜粪多的地方。可以长到人的高度。叶子形状像柏树的叶子,但宽一些,茎不粗,茎长得长了,就在上面开花。花的形状像胡豆的花,是深蓝色的。据说这种草可以作成牙痛的药。以前老人们称它是\lien{}{phɤri kɯnɤwu}(为对面哭的意思)。不能吃,连牲畜都不能吃。}\end{exemple}\end{entrée}

\begin{entrée}{phɤrtɕaʁ}{}{ⓔphɤrtɕaʁ} 
\classe{n} 
\begin{définition}\pfra{support sur lequel on met le bout inférieur du fuseau}\end{définition}
\begin{définition}\pcmn{用来放纺锤的下一端的托子(一般是碗的底部,或者小石板)}\end{définition}\end{entrée}

\begin{entrée}{phɤtɕhɯχtɤr}{}{ⓔphɤtɕhɯχtɤr} 
\classe{n} 
\begin{définition}\pfra{éparpiller partout}\end{définition}
\begin{définition}\pcmn{撒得一地都是}\end{définition}
\begin{exemple}\pjya{phɤtɕhɯχtɤr ʑo pɯ-ta-t-a (pɯ-lat-a)}\hspace{5pt}\pcmn{我撒得一地都是了}\end{exemple}\relationsémantique{参考}{\lien{ⓔrɯtɕhɯχtɤr}{rɯtɕhɯχtɤr}}\end{entrée}

\begin{entrée}{phɣo}{}{ⓔphɣo} 
\classe{vi} \paradigme{dir}{nɯ-}\paradigme{dir}{nɯ-}
\begin{définition}\pfra{fuir}\end{définition}
\begin{définition}\pcmn{逃跑}\end{définition}
\begin{définition}\pfra{laisser fuir}\end{définition}
\begin{définition}\pcmn{使逃跑}\end{définition}
\begin{exemple}\pjya{khɯna ɣɤʑu tɕe nɯ-phɣo-a}\hspace{5pt}\pcmn{我看到狗就跑了}\end{exemple}
\begin{exemple}\pjya{ɯʑo kɤ-phɣo ɲɯ-sɯsɤm ri, mɯ-nɯ-sɯphɣo-t-a}\hspace{5pt}\pcmn{他想逃跑,但是我没有让他(得逞)}\end{exemple}\relationsémantique{参考}{\lien{ⓔɕphɣo}{ɕphɣo}}
\begin{sous-entrée}{sɯphɣo/\variante{ɕɯphɣo}}{ⓔphɣoⓝsɯphɣo} 
\classe{vt}  
\grammaire{caus} \end{sous-entrée}

\end{entrée}

\begin{entrée}{phima}{}{ⓔphima} 
\classe{n} 
\begin{définition}\pfra{partie extérieur des habits tibétains}\end{définition}
\begin{définition}\pcmn{衣服面子(藏装)}\end{définition}\étymologie{pʰʲi.ma}\end{entrée}

\begin{entrée}{phuɲi}{}{ⓔphuɲi} 
\classe{n} 
\begin{définition}\pfra{une espèce d'arbrisseau}\end{définition}
\begin{définition}\pcmn{灌木的一种}\end{définition}
\begin{exemple}\pjya{phuɲi nɯ si kɯ-mbɯ-mbɤr ci ŋu, ɯ-ru ɣɯ ɯ-mdoʁ nɯ aɣɯrnɯɕɯr, ɯ-rqhu rɕɯβrɕɯβ ʑo pa, ɯ-jwaʁ xtɕi ri jaʁ tsa ɯ-rme kɯ-fse tu. ɯ-mɯntoʁ kɯ-qarŋɯ-rŋe ŋu. ɯ-mnɯ nɯ ɲɯ́-wɣ-phɯt tɕe, kɯ-ndɯβ nɯ ra saχsɯ ɲɯ́-wɣ-βzu sna, kɯ-jndʐɤz nɯ ra zɣɤmbu ɲɯ́-wɣ-βzu sna.}\hspace{5pt}\pcmn{\lien{ⓔphuɲi}{phuɲi}是矮小的树种,树干是淡红色,树皮很粗糙(到处都裂开、快要脱落的样子),叶子比较小,但是有点厚,上面有毛。花是淡黄色的。把它的苗拔下后,小的可以作耍把,大的可以作扫把。}\end{exemple}\end{entrée}

\begin{entrée}{phoβraŋ}{}{ⓔphoβraŋ} 
\classe{n} 
\begin{définition}\pfra{palais}\end{définition}
\begin{définition}\pcmn{宫殿}\end{définition}\étymologie{pʰo.braŋ}\end{entrée}

\begin{entrée}{phochi}{}{ⓔphochi} 
\classe{n} 
\begin{définition}\pfra{chien}\end{définition}
\begin{définition}\pcmn{公狗}\end{définition}\étymologie{pho.kʰʲi}\end{entrée}

\begin{entrée}{pholi}{}{ⓔpholi} 
\classe{n} 
\begin{définition}\pfra{chat}\end{définition}
\begin{définition}\pcmn{公猫}\end{définition}\end{entrée}

\begin{entrée}{phoŋ}{}{ⓔphoŋ} 
\classe{n} 
\begin{définition}\pfra{bouteille}\end{définition}
\begin{définition}\pcmn{瓶子}\end{définition}\étymologie{bum.bu}\end{entrée}

\begin{entrée}{phoŋ}{}{ⓔphoŋ} 
\classe{n} 
\begin{définition}\pfra{bouteille}\end{définition}
\begin{définition}\pcmn{瓶}\end{définition}
\begin{sous-entrée}{tɯ-phoŋ}{ⓔphoŋⓝtɯ-phoŋ} 
\classe{clf} 
\begin{définition}\pfra{une bouteille}\end{définition}
\begin{définition}\pcmn{一瓶}\end{définition}
\begin{exemple}\pjya{cha tɯ-phoŋ}\hspace{5pt}\pcmn{一瓶酒}\end{exemple}\end{sous-entrée}

\end{entrée}

\begin{entrée}{phoŋnɤphoŋ}{}{ⓔphoŋnɤphoŋ} 
\classe{idph.3} 
\begin{définition}\pfra{bruit de morceaux de bois qui s'entrechoquent}\end{définition}
\begin{définition}\pcmn{形容木头撞击的声音}\end{définition}
\begin{exemple}\pjya{kɯm ɯ-taʁ laχtɕha a-kɤ-rpu tɕe phoŋnɤphoŋ ti}\hspace{5pt}\pcmn{东西撞击在木门上就发出砰砰声}\end{exemple}\relationsémantique{同义词}{\lien{}{ɕkhoŋnɤɕkhoŋ}}\end{entrée}

\begin{entrée}{phoŋsti}{}{ⓔphoŋsti} 
\classe{n} 
\begin{définition}\pfra{bouchon}\end{définition}
\begin{définition}\pcmn{瓶盖}\end{définition}\relationsémantique{参考}{\lien{ⓔstiⓗ1}{sti₁}}\end{entrée}

\begin{entrée}{phoroʁ/\variante{phɤroʁ}}{}{ⓔphoroʁ} 
\classe{n} 
\begin{définition}\pfra{corbeau (corvus corax)}\end{définition}
\begin{définition}\pcmn{渡鸦}\end{définition}\étymologie{pʰo.rog}\end{entrée}

\begin{entrée}{phoʁ}{}{ⓔphoʁ} 
\classe{vt} \paradigme{dir}{\_}
\begin{définition}\pfra{verser une partie}\end{définition}
\begin{définition}\pcmn{(从大袋子)舀出来,分装在(小袋子里)}\end{définition}
\begin{exemple}\pjya{tɯ-ci thɯ-phoʁ, lɤ-phoʁ}\hspace{5pt}\pcmn{把水舀出来}\end{exemple}
\begin{exemple}\pjya{tɤɕi kɤ-phoʁ}\hspace{5pt}\pcmn{把青稞分装在小袋子里吧}\end{exemple}
\begin{exemple}\pjya{nɤ-tʂha thɯ-phoʁ-a, kɤ-tshi ma}\hspace{5pt}\pcmn{我分了一点茶给你,你喝吧}\end{exemple}\relationsémantique{参考}{\lien{ⓔciⓗ2}{ci}}\end{entrée}

\begin{entrée}{phoʁphoʁ}{}{ⓔphoʁphoʁ} 
\classe{idph.2} 
\begin{définition}\pfra{solide}\end{définition}
\begin{définition}\pcmn{结实,安然无恙,衣冠端正}\end{définition}
\begin{exemple}\pjya{laχtɕha ra phoʁphoʁ ʑo ɯ-pɯ to-nɯ-panɯ}\hspace{5pt}\pcmn{他们很小心地保管了东西}\end{exemple}
\begin{exemple}\pjya{kɯm phoʁphoʁ ʑo ko-nɯ-pa-nɯ}\hspace{5pt}\pcmn{他们把门关得很紧}\end{exemple}
\begin{exemple}\pjya{tɯ-ŋga phoʁphoʁ ʑo ɲɤ-nɯ-ta (ɯ-kɯ-nɯ-ndo pjɤ-me)}\hspace{5pt}\pcmn{衣服放在那里很安全(没有人拿)}\end{exemple}
\begin{sous-entrée}{phoʁnɤphoʁ}{ⓔphoʁphoʁⓝphoʁnɤphoʁ} 
\classe{idph.3} 
\begin{exemple}\pjya{phoʁnɤphoʁ ko-nɯɕe}\hspace{5pt}\pcmn{他一点也没有耽搁就回去了}\end{exemple}\end{sous-entrée}

\end{entrée}

\begin{entrée}{phosɤr}{}{ⓔphosɤr} 
\classe{n} 
\begin{définition}\pfra{jeune garçon}\end{définition}
\begin{définition}\pcmn{青年}\end{définition}\relationsémantique{参考}{\lien{}{pho.gsar}}\end{entrée}

\begin{entrée}{phot}{}{ⓔphot} 
\classe{vi} \paradigme{dir}{tɤ-}
\begin{définition}\pfra{oser}\end{définition}
\begin{définition}\pcmn{敢}\end{définition}
\begin{exemple}\pjya{jiɕqha nɯ kɤ-saʁndɯ ɲɯ-phot}\hspace{5pt}\pcmn{他敢打人}\end{exemple}
\begin{exemple}\pjya{ku-ɕe ɲɯ-phot}\hspace{5pt}\pcmn{他敢去}\end{exemple}
\begin{exemple}\pjya{praʁ ɯ-ku ku-ɕe ɲɯ-phot}\hspace{5pt}\pcmn{他敢到悬崖去}\end{exemple}
\begin{exemple}\pjya{sɯku tu-ɕe ɲɯ-phot}\hspace{5pt}\pcmn{他敢到树上}\end{exemple}
\begin{exemple}\pjya{tɕhi pɯ-nɯ-ŋɯ-ŋu phot}\hspace{5pt}\pcmn{他什么都敢做}\end{exemple}
\begin{exemple}\pjya{ɯ-mɤ-kɤ-phot me}\hspace{5pt}\pcmn{他什么都敢做}\end{exemple}\étymologie{pʰod}\end{entrée}

\begin{entrée}{phozgra}{}{ⓔphozgra} 
\classe{n} 
\begin{définition}\pfra{cri (lama)}\end{définition}
\begin{définition}\pcmn{吼声(喇嘛的)}\end{définition}\end{entrée}

\begin{entrée}{phrɤβ}{}{ⓔphrɤβ} 
\classe{idph.1} 
\begin{définition}\pfra{bruit de grains ou de billes qui s'éparpillent sur le sol}\end{définition}
\begin{définition}\pcmn{很多颗粒一下子撒在地上的声音}\end{définition}
\begin{exemple}\pjya{ɯʑo kɯ stoʁ tɯ-spra to-ndo tɕe phrɤβ ʑo pa-βde}\hspace{5pt}\pcmn{他抓了一把胡豆一下子扔下去了}\end{exemple}
\begin{sous-entrée}{phrɤβnɤphrɤβ}{ⓔphrɤβⓝphrɤβnɤphrɤβ} 
\classe{idph.3} 
\begin{exemple}\pjya{sɤrwa pa-lɤt tɕe, ɯ-zgra phrɤβphrɤβ nɤ phrɤβphrɤβ ʑo ɲɯ-ti}\hspace{5pt}\pcmn{打了冰雹,发出噼噼啪啪的声音}\end{exemple}\end{sous-entrée}

\begin{sous-entrée}{ɣɤphrɤβphrɤβ}{ⓔphrɤβⓝɣɤphrɤβphrɤβ} 
\classe{vs} 
\begin{exemple}\pjya{sɤrwa chɯ-lɤt tɕe ɲɯ-ɣɤphrɤβphrɤβ ʑo}\hspace{5pt}\pcmn{冰雹发出噼噼啪啪的声音(连续不停地,很快)}\end{exemple}\end{sous-entrée}

\end{entrée}

\begin{entrée}{phrɤl}{}{ⓔphrɤl} 
\classe{vt} \paradigme{dir}{pɯ-}
\begin{définition}\pfra{expliquer}\end{définition}
\begin{définition}\pcmn{解释}\end{définition}
\begin{exemple}\pjya{tɯ-rju pɯ-phrɤl}\hspace{5pt}\pcmn{你解释一下}\end{exemple}
\begin{exemple}\pjya{tɯ-rju pɯ-phral-a tɕe nɤʑo tɯ-sɯz}\hspace{5pt}\pcmn{你已经解释了,你现在知道}\end{exemple}
\begin{exemple}\pjya{tsuku tɤ-mu kɯ ɯ-rɟit ɯ-ɕki ``ɲɯ-tɯ-nɯkhɤja" to-ti ri, ɯ-rɟit nɯ kɯ ``tu-nɯkhɤja-a maʁ, pjɯ-phral-a ɕti" to-ti}\hspace{5pt}\pcmn{有些母亲对孩子说“你顶嘴”,而孩子回答:“我不是顶嘴,是在解释”}\end{exemple}\end{entrée}

\begin{entrée}{phɯ}{}{ⓔphɯ} 
\classe{n} 
\begin{définition}\pfra{souffle}\end{définition}
\begin{définition}\pcmn{吹冷气}\end{définition}
\begin{exemple}\pjya{tɯ-ci ɲɯ-sɤɕke tɕe phɯ tɤ-ti tɕe kɤ-tshi}\hspace{5pt}\pcmn{水很烫,你吹一下才喝}\end{exemple}\end{entrée}

\begin{entrée}{phɯɣ}{}{ⓔphɯɣ} 
\classe{vt} \paradigme{dir}{pɯ-}
\begin{définition}\pfra{déployer, ouvrir (parapluie, tente)}\end{définition}
\begin{définition}\pcmn{撑开}\end{définition}
\begin{exemple}\pjya{@san pɯ-phɯɣ}\hspace{5pt}\pcmn{你把伞撑开吧}\end{exemple}
\begin{exemple}\pjya{zgɤr pɯ-phɯɣ}\hspace{5pt}\pcmn{你把帐篷撑开吧}\end{exemple}\end{entrée}

\begin{entrée}{phɯɣphɯɣ}{}{ⓔphɯɣphɯɣ} 
\classe{n} 
\begin{définition}\pfra{son}\end{définition}
\begin{définition}\pcmn{麦类的粗糠秕}\end{définition}\end{entrée}

\begin{entrée}{phɯl}{}{ⓔphɯl} 
\classe{vt} \paradigme{dir}{lɤ-}
\begin{définition}\pfra{offrir}\end{définition}
\begin{définition}\pcmn{献给;上供}\end{définition}
\begin{exemple}\pjya{βlama ɯ-ʁɤri lɤ-phɯl-a}\hspace{5pt}\pcmn{我献给喇嘛了}\end{exemple}
\begin{exemple}\pjya{tɯjpu lɤ-phɯl-a}\hspace{5pt}\pcmn{我给他献了粮食}\end{exemple}\étymologie{pʰul}\end{entrée}

\begin{entrée}{phɯqha}{}{ⓔphɯqha} 
\classe{n} 
\begin{définition}\pfra{grosse racine}\end{définition}
\begin{définition}\pcmn{干树根}\end{définition}\end{entrée}

\begin{entrée}{phɯrɤm}{}{ⓔphɯrɤm} 
\classe{n} 
\begin{définition}\pfra{herse}\end{définition}
\begin{définition}\pcmn{耙}\end{définition}
\begin{exemple}\pjya{phɯrɤm thɯ-rɤɕi-t-a}\hspace{5pt}\pcmn{我耙了地}\end{exemple}\relationsémantique{参考}{\lien{ⓔnɯphɯrɤmⓝrɯphɯrɤm}{rɯphɯrɤm}}\relationsémantique{参考}{\lien{ⓔnɯphɯrɤm}{nɯphɯrɤm}}\end{entrée}

\begin{entrée}{phɯrkhɯɣ}{}{ⓔphɯrkhɯɣ} 
\classe{n} 
\begin{définition}\pfra{sac que l'on porte en bandoulière}\end{définition}
\begin{définition}\pcmn{挎包}\end{définition}\relationsémantique{反义词}{\lien{ⓔɕɤkhoz}{ɕɤkhoz}}\étymologie{kʰug}\end{entrée}

\begin{entrée}{phɯsɤti}{}{ⓔphɯsɤti} 
\classe{n} 
\begin{définition}\pfra{soufflet}\end{définition}
\begin{définition}\pcmn{吹火筒}\end{définition}\relationsémantique{参考}{\lien{ⓔti}{ti}}\end{entrée}

\begin{entrée}{phɯt}{}{ⓔphɯt} 
\classe{vt} \sens{1}\paradigme{dir}{lɤ-}\paradigme{dir}{pɯ-}
\begin{définition}\pfra{couper}\end{définition}
\begin{définition}\pcmn{割;砍}\end{définition}
\begin{exemple}\pjya{si lɤ-phɯt-a}\hspace{5pt}\pcmn{我砍了树}\end{exemple}
\begin{exemple}\pjya{a-ndzrɯ nɯ-phɯt-a (=nɯ-kraɣ-a, nɯ-ʁndzar-a)}\hspace{5pt}\pcmn{我剪了指甲}\end{exemple}\sens{2}\paradigme{dir}{nɯ-}
\begin{définition}\pfra{arracher, cueillir}\end{définition}
\begin{définition}\pcmn{拔;扯;摘}\end{définition}
\begin{exemple}\pjya{mɯntoʁ nɯ-phɯt-a}\hspace{5pt}\pcmn{我摘了花}\end{exemple}
\begin{exemple}\pjya{sɯjno nɯ-phɯt-a, lɤ-phɯt-a}\hspace{5pt}\pcmn{我拔了草}\end{exemple}\sens{3}\paradigme{dir}{pɯ-}
\begin{définition}\pfra{enlever, diminuer}\end{définition}
\begin{définition}\pcmn{减}\end{définition}
\begin{exemple}\pjya{ɣnɤsqi ɯ-ŋgɯ χsɯm pjɯ́-wɣ-phɯt tɕe, sqaɕnɯz ma ɲɯ-me ŋu}\hspace{5pt}\pcmn{二十减三等于十七}\end{exemple}\sens{4}\paradigme{dir}{pɯ-}
\begin{définition}\pfra{démolir}\end{définition}
\begin{définition}\pcmn{拆}\end{définition}
\begin{exemple}\pjya{znde pɯ-phɯt-a}\hspace{5pt}\pcmn{我拆了墙}\end{exemple}\sens{5}\paradigme{dir}{kɤ-}
\begin{définition}\pfra{récolter}\end{définition}
\begin{définition}\pcmn{收割}\end{définition}
\begin{exemple}\pjya{tɤ-rɤku ka-phɯt}\hspace{5pt}\pcmn{他收割了庄稼}\end{exemple}\relationsémantique{参考}{\lien{ⓔɣɯsɯphɯt}{ɣɯsɯphɯt}}
\begin{sous-entrée}{nɯɣɯphɯt}{ⓔphɯtⓝnɯɣɯphɯt} 
\classe{vs}  
\grammaire{facil} 
\begin{définition}\pfra{être facile à cueillir}\end{définition}
\begin{définition}\pcmn{容易摘}\end{définition}\end{sous-entrée}

\begin{sous-entrée}{sphɯt}{ⓔphɯtⓝsphɯt} 
\classe{vt}  
\grammaire{habil} 
\begin{définition}\pfra{pouvoir couper}\end{définition}
\begin{définition}\pcmn{切得动;咬得动}\end{définition}
\begin{exemple}\pjya{a-ɕɣa kɯ kɤ-ndza ra mɯ́j-sphɯt / a-ɕɣa tɤ-lat-a ri mɯ́j-sphɯt}\hspace{5pt}\pcmn{我的牙齿吃不动那些食物}\end{exemple}
\begin{exemple}\pjya{mbrɯtɕɯ ki mɯ́j-mtɕoʁ tɕe mɯ́j-sphɯt}\hspace{5pt}\pcmn{这把刀不锋利,切不动}\end{exemple}\end{sous-entrée}

\begin{sous-entrée}{sɤsphɯt}{ⓔphɯtⓝsɤsphɯt} 
\classe{vs} 
\begin{définition}\pfra{que l'on peut casser (avec les dents)}\end{définition}
\begin{définition}\pcmn{吃得动}\end{définition}\end{sous-entrée}

\end{entrée}

\begin{entrée}{pijma}{}{ⓔpijma} 
\classe{n} 
\begin{définition}\pfra{terre jaune (de mauvaise qualité)}\end{définition}
\begin{définition}\pcmn{黄泥巴}\end{définition}\end{entrée}

\begin{entrée}{piwalu}{}{ⓔpiwalu} 
\classe{n} 
\begin{définition}\pfra{année du rat}\end{définition}
\begin{définition}\pcmn{鼠年}\end{définition}\étymologie{bʲi.ba.lo}\end{entrée}

\begin{entrée}{puj}{}{ⓔpuj} 
\classe{n} 
\begin{définition}\pfra{type de sapin}\end{définition}
\begin{définition}\pcmn{杉树的一种}\end{définition}\end{entrée}

\begin{entrée}{pja}{}{ⓔpja} 
\classe{n} 
\begin{définition}\pfra{oiseau}\end{définition}
\begin{définition}\pcmn{鸟}\end{définition}\end{entrée}

\begin{entrée}{pjalu}{}{ⓔpjalu} 
\classe{n} 
\begin{définition}\pfra{année du coq}\end{définition}
\begin{définition}\pcmn{鸡年}\end{définition}\étymologie{bʲa.lo}\end{entrée}

\begin{entrée}{pjɤβlaʁ}{}{ⓔpjɤβlaʁ} 
\classe{n} 
\begin{définition}\pfra{tromperie}\end{définition}
\begin{définition}\pcmn{阴谋;骗局}\end{définition}
\begin{exemple}\pjya{pjɤβlaʁ to-βzu}\hspace{5pt}\pcmn{他用了阴谋}\end{exemple}\relationsémantique{参考}{\lien{ⓔrɯpjɤβlaʁ}{rɯpjɤβlaʁ}}\end{entrée}

\begin{entrée}{pjɤl}{}{ⓔpjɤl} 
\classe{vt} \sens{1}\paradigme{dir}{tɤ-}
\begin{définition}\pfra{contourner}\end{définition}
\begin{définition}\pcmn{绕过}\end{définition}
\begin{exemple}\pjya{nɯtɕu khɯna ci ɣɤʑu, tɤ-pjal-a}\hspace{5pt}\pcmn{那边看到有狗,我就绕过去了}\end{exemple}
\begin{exemple}\pjya{mɯ-mɤ-ɲɯ-tɯ-mbɣom nɤ tu-kɯ-nɯ-pjal-a ma kɯsthɯci a-zrɯɣ a-ndʑrɯ ɲɯ-dɤn}\hspace{5pt}\pcmn{你不急的话你绕过我吧,我身上有这么多的虱子和虮子}\end{exemple}\sens{2}\paradigme{dir}{kɤ-}
\begin{définition}\pfra{traverser}\end{définition}
\begin{définition}\pcmn{穿过}\end{définition}
\begin{exemple}\pjya{sɯŋgɯ kɤ-pjal-a}\hspace{5pt}\pcmn{我穿过了森林}\end{exemple}\étymologie{bʲol}\end{entrée}

\begin{entrée}{pjɤrgɤt}{}{ⓔpjɤrgɤt} 
\classe{n} 
\begin{définition}\pfra{vautour (gyps himalayensis)}\end{définition}
\begin{définition}\pcmn{高山兀鹫}\end{définition}
\begin{exemple}\pjya{pjɤrgɤt nɯ praʁ kɯ-mbro ɯ-ku zɯ ɕ-ku-rma ɲɯ-ŋu, ɯ-ro cho ɯ-ku nɯ kɯ-wɣrum ɲɯ-ŋu, wuma ʑo ɲɯ-wxti. fsusqɯ-tɯrpa jamar tu, kɯ-ɤɲaʁndzɯm tu ɲɯ-ŋgrɤl, pjɤrgɤt ɯ-ŋgɯz kɯ-wxti ɲɯ-ŋu, pjɤrgɤt βlama ŋu nɯ tu-kɯ-ti ɲɯ-ŋgrɤl, fsapaʁ cho tɯrme ɕa tu-ndze ɲɯ-ŋgrɤl, co ra mɤ-ɣi, nɤmkha zɯ tu-nɯpjɤŋkhɤr rga, ɯ-mɤlɤjaʁ ɯ-ndzrɯ ɕɤmiɕtʂɤt fse. ɯ-rme kɯ-wɣrum kɯ-zri ɲɯ-ŋu, ɯ-mtsioʁ nɯ tu-ŋgɤɣ ŋu.}\hspace{5pt}\pcmn{兀鹫一般栖息在悬崖峭壁上,胸部和头部是白色的,很大,有三十来斤。有的是黑色的,是在兀鹫之中比较大的,据说是它们的喇嘛。它吃牲畜和人肉。不会飞到山谷里,喜欢在空中旋转。爪子像铁钩一样,羽毛白而长,嘴是勾起来的。}\end{exemple}\étymologie{bʲa.rgod}\end{entrée}

\begin{entrée}{pjɤt}{}{ⓔpjɤt} 
\classe{vt}  
\grammaire{apass} \paradigme{dir}{nɯ-}\paradigme{dir}{nɯ-}
\begin{définition}\pfra{bourrer (saucisson)}\end{définition}
\begin{définition}\pcmn{装满(肠子、肺)}\end{définition}
\begin{exemple}\pjya{tɯ-pu nɯ-pjɤt}\hspace{5pt}\pcmn{把肠子装满}\end{exemple}
\begin{exemple}\pjya{paʁ ɯ-rtshɤz nɯ-pjɤt}\hspace{5pt}\pcmn{把猪肺装满}\end{exemple}
\begin{sous-entrée}{rɤpjɤt}{ⓔpjɤtⓝrɤpjɤt} 
\classe{vi} \end{sous-entrée}

\begin{définition}\pfra{bourrer}\end{définition}
\begin{définition}\pcmn{装满(肠子、肺)}\end{définition}\end{entrée}

\begin{entrée}{pjɤʑŋgur}{}{ⓔpjɤʑŋgur} 
\classe{n} 
\begin{définition}\pfra{saucisson}\end{définition}
\begin{définition}\pcmn{猪肉香肠}\end{définition}
\begin{exemple}\pjya{pjɤʑŋgur ɲɯ́-wɣ-rku}\hspace{5pt}\pcmn{把肉泥灌入香肠}\end{exemple}\relationsémantique{参考}{\lien{ⓔpjɤt}{pjɤt}}\end{entrée}

\begin{entrée}{pjɤʑrɤz}{}{ⓔpjɤʑrɤz} 
\classe{n} 
\begin{définition}\pfra{méthode de tissage}\end{définition}
\begin{définition}\pcmn{织布的方法,四根线交错着,纹路的方向变来变去}\end{définition}\end{entrée}

\begin{entrée}{pjɯrɯ}{}{ⓔpjɯrɯ} 
\classe{n} 
\begin{définition}\pfra{corail}\end{définition}
\begin{définition}\pcmn{珊瑚}\end{définition}\étymologie{bʲu.ru}\end{entrée}

\begin{entrée}{puɟy}{}{ⓔpuɟy} 
\classe{n} 
\begin{définition}\pfra{instrument pour bourrer les saucisses et les boudins}\end{définition}
\begin{définition}\pcmn{装血肠的工具}\end{définition}\end{entrée}

\begin{entrée}{plaŋplaŋ}{}{ⓔplaŋplaŋ} 
\classe{idph.2} 
\begin{définition}\pfra{étendu et lisse}\end{définition}
\begin{définition}\pcmn{形容又光滑又宽的样子}\end{définition}
\begin{exemple}\pjya{tɤjpɣom plaŋplaŋ kɯ-pa ʑo nɯ-atɯɣ-ndʑi}\hspace{5pt}\pcmn{他们俩遇到一块又光滑又宽的冰片}\end{exemple}\end{entrée}

\begin{entrée}{plaʁplaʁ}{}{ⓔplaʁplaʁ} 
\classe{idph.2} \sens{1}
\begin{définition}\pfra{doux au toucher}\end{définition}
\begin{définition}\pcmn{形容摸起来很光滑的样子}\end{définition}\sens{2}
\begin{définition}\pfra{tout blanc}\end{définition}
\begin{définition}\pcmn{形容一片纯白的样子}\end{définition}
\begin{exemple}\pjya{tɤjpa plaʁplaʁ ʑo ko-sɯwɣrum}\hspace{5pt}\pcmn{雪把大地染成了白茫茫的一片}\end{exemple}
\begin{exemple}\pjya{ɕɤfɕo ndɤre, thɯ-nɯrmɤmbe-nɯ, koŋla plaʁplaʁ ʑo thɯ-pa-nɯ}\hspace{5pt}\pcmn{这几天它们脱了毛,变得光溜溜的}\end{exemple}
\begin{sous-entrée}{plaʁnɤplaʁ}{ⓔplaʁplaʁⓢ2ⓝplaʁnɤplaʁ} 
\classe{idph.3} 
\begin{exemple}\pjya{qapri kɯ ɯ-mdʑu plaʁnɤplaʁ ʑo ɲɯ-ɤsɯ-stu}\hspace{5pt}\pcmn{蛇(慢慢地)把舌头一伸一缩的}\end{exemple}\end{sous-entrée}

\begin{sous-entrée}{ɣɤplaʁplaʁ}{ⓔplaʁplaʁⓢ2ⓝɣɤplaʁplaʁ} 
\classe{vi} 
\begin{exemple}\pjya{qapri ɣɯ ɯ-mdʑu ɲɯ-ɣɤplaʁplaʁ ʑo}\hspace{5pt}\pcmn{蛇的舌头一伸一缩}\end{exemple}\end{sous-entrée}

\begin{sous-entrée}{sɤplaʁplaʁ}{ⓔplaʁplaʁⓢ2ⓝsɤplaʁplaʁ} 
\classe{vt} 
\begin{exemple}\pjya{qapri kɯ ɯ-mdʑɯ ɲɯ-sɤplaʁplaʁ}\hspace{5pt}\pcmn{蛇(快速地)把舌头一伸一缩}\end{exemple}\end{sous-entrée}

\end{entrée}

\begin{entrée}{ploʁploʁ}{}{ⓔploʁploʁ} 
\classe{idph.2} 
\begin{définition}\pfra{en boule}\end{définition}
\begin{définition}\pcmn{形容圆形的}\end{définition}\relationsémantique{参考}{\lien{}{χploχploʁ}}\relationsémantique{参考}{\lien{ⓔɕploʁɕploʁ}{ɕploʁɕploʁ}}
\begin{sous-entrée}{ɣɤploʁploʁ}{ⓔploʁploʁⓝɣɤploʁploʁ} 
\classe{vi} 
\begin{définition}\pfra{bouillir à gros bouillon}\end{définition}
\begin{définition}\pcmn{滚烫,冒出水泡}\end{définition}
\begin{exemple}\pjya{tɯ-ci ɲɯ-ɤla ɲɯ-ɣɤploʁploʁ ʑo}\hspace{5pt}\pcmn{水沸腾了,冒出水泡}\end{exemple}\end{sous-entrée}

\end{entrée}

\begin{entrée}{plɯt}{}{ⓔplɯt} 
\classe{vt} \paradigme{dir}{nɯ-}\paradigme{dir}{thɯ-}\sens{1}
\begin{définition}\pfra{détruire}\end{définition}
\begin{définition}\pcmn{灭亡}\end{définition}
\begin{exemple}\pjya{jima ɯ-rɣi nɯ-plɯt-i}\hspace{5pt}\pcmn{我们把玉米的种子用得一点都不剩了}\end{exemple}\sens{2}
\begin{définition}\pfra{anéantir la descendance}\end{définition}
\begin{définition}\pcmn{断根}\end{définition}\relationsémantique{参考}{\lien{ⓔmblɯt}{mblɯt}}\end{entrée}

\begin{entrée}{popo}{}{ⓔpopo} 
\classe{n} 
\begin{définition}\pfra{récipient en terre}\end{définition}
\begin{définition}\pcmn{沙锅}\end{définition}
\begin{exemple}\pjya{popo nɯ tɤ-rcoʁ kɯ tɤ-kɤ-sɯ-βzu ŋu, ɯ-mŋu artɯm, ɯ-xtu kɯnɤ artɯm, ɯ-xtu cho ɯ-mŋu ɯ-pɤrthɤβ ɯ-mke ci tu, ɯ-jɯ kɯ-rɟɯ-rɟum tu, tɤ-mthɯm sɤ-sqa pe, tɤ-ala pɯ-tsu tɕe ʑaʑa mɤ-fɕu tɕe tɤ-mthɯm pɯ́-wɣ-sqa tɕe wuma ʑo ku-smi cha tɕe rgargɯn ra nɯ-kɤ-ndza wuma ʑo nɤtsa.}\hspace{5pt}\pcmn{\lien{ⓔpopo}{popo}是用泥做成的,口和肚子都是圆形的,口和肚子之间有颈,把很粗,是煮肉的好器具。一旦煮开了,不容易凉,煮肉煮得很熟,很适合老年人吃。}\end{exemple}\end{entrée}

\begin{entrée}{porɤt}{}{ⓔporɤt} 
\classe{n} 
\begin{définition}\pfra{petite araignée}\end{définition}
\begin{définition}\pcmn{小蜘蛛}\end{définition}\end{entrée}

\begin{entrée}{poʁlɯ}{}{ⓔpoʁlɯ} 
\classe{n} 
\begin{définition}\pfra{type d'avoine}\end{définition}
\begin{définition}\pcmn{莜麦}\end{définition}\end{entrée}

\begin{entrée}{poʁlɯrmbjɤβ}{}{ⓔpoʁlɯrmbjɤβ} 
\classe{n} 
\begin{définition}\pfra{avoine en bottes}\end{définition}
\begin{définition}\pcmn{捆成一把的莜麦杆}\end{définition}\end{entrée}

\begin{entrée}{posti}{}{ⓔposti} 
\classe{n} 
\begin{définition}\pfra{planche en bois recouvrant le mur dans la cuisine}\end{définition}
\begin{définition}\pcmn{【墙裙】贴在墙壁上的木板,大概只有一米高}\end{définition}\end{entrée}

\begin{entrée}{pot}{}{ⓔpot} 
\classe{n} 
\begin{définition}\pfra{Tibet central}\end{définition}
\begin{définition}\pcmn{西藏}\end{définition}\étymologie{bod}\end{entrée}

\begin{entrée}{poxco}{}{ⓔpoxco} 
\classe{n} 
\begin{définition}\pfra{récipient en cuivre}\end{définition}
\begin{définition}\pcmn{铜罐子,口小腹大,有斜着的倒水用的嘴口,没有盖子}\end{définition}
\begin{exemple}\pjya{poxco nɯ kontsi nɯ ɯ-ŋgɯz kɯ-wxti tsa ci ŋu, tɯ-ci sqamnɯz tɯ-rpa jamar tɕhɯt, nɯ sɤz kɯ-xtɕi tɕi tu, poxco ɯ-mtɕhi ɯ-phaʁ ntsi ri ɲɯ-ru ŋu, ɯ-xtu kɯ-wxtɯ-wxti ŋu, ɯ-spa nɯ zaŋ ŋu, ɯ-jɯ wuma ʑo ngɯt. kɯɕɯŋgɯ tɕe tʂha ɯ-sɤ-ta pjɤ-ŋu, tɯrme kɯ-ɤro pjɤ-rkɯn. tɯrme kɯ-tshu nɯ-xtu kɯ-wxti nɯ ra poxco tu-sɤrmi-nɯ ŋgrɤl.}\hspace{5pt}\pcmn{\lien{ⓔpoxco}{poxco}是罐子里面比较大的一种,可以容下十二斤水,比较小的也有。罐子嘴是斜的,有很大的肚子,是用红铜铸成的,把很结实。过去,是用来熬茶的罐子,拥有这种罐子的人不多。肚子很大,比较胖的人有时候说他们是\lien{ⓔpoxco}{poxco}。}\end{exemple}\end{entrée}

\begin{entrée}{pru}{}{ⓔpru} 
\classe{n}  
\grammaire{n.lieu} 
\begin{définition}\pfra{un nom de hameau}\end{définition}
\begin{définition}\pcmn{房名}\end{définition}\end{entrée}

\begin{entrée}{praʁ}{}{ⓔpraʁ} 
\classe{n} 
\begin{définition}\pfra{falaise}\end{définition}
\begin{définition}\pcmn{崖山}\end{définition}\étymologie{brag}\end{entrée}

\begin{entrée}{praʁɕku}{}{ⓔpraʁɕku} 
\classe{n} 
\begin{définition}\pfra{oignon}\end{définition}
\begin{définition}\pcmn{葱}\end{définition}
\begin{exemple}\pjya{praʁɕku nɯ tɯ-ji ɯ-ŋgɯ lu-kɤ-nɯ-ji ci ŋu, ɯ-jwaʁ nɯ kɤ-ɕpɯ-ɕpa kɯ-tɕɤr tɕe kɯ-rɲɟi tsa ŋu, ɯ-ru me, ɯ-tho tu, kɤ-ndza mɯm. ɯ-mdoʁ ldʑaŋnaʁ ŋu.}\hspace{5pt}\pcmn{\lien{ⓔpraʁɕku}{praʁɕku}是自己种在地里的(农作物),叶子又扁又窄又长,没有茎,有花梗,好吃。颜色是深绿色。}\end{exemple}\end{entrée}

\begin{entrée}{praʁkɤsi}{}{ⓔpraʁkɤsi} 
\classe{n} 
\begin{définition}\pfra{une espèce de chêne}\end{définition}
\begin{définition}\pcmn{槲栎的一种}\end{définition}\end{entrée}

\begin{entrée}{praʁkhaŋ}{}{ⓔpraʁkhaŋ} 
\classe{n} 
\begin{définition}\pfra{grotte}\end{définition}
\begin{définition}\pcmn{山洞}\end{définition}\end{entrée}

\begin{entrée}{praʁɬɤrɲaŋ}{}{ⓔpraʁɬɤrɲaŋ} 
\classe{n} 
\begin{définition}\pfra{falaise}\end{définition}
\begin{définition}\pcmn{悬崖}\end{définition}\étymologie{brag lha.rɲiŋ}\end{entrée}

\begin{entrée}{praʁɲɤl}{}{ⓔpraʁɲɤl} 
\classe{n} 
\begin{définition}\pfra{grotte}\end{définition}
\begin{définition}\pcmn{岩洞}\end{définition}\end{entrée}

\begin{entrée}{praʁsrɯm/\variante{praʁʂɯm}}{}{ⓔpraʁsrɯm} 
\classe{n} 
\begin{définition}\pfra{un mammifère}\end{définition}
\begin{définition}\pcmn{哺乳动物的一种}\end{définition}
\begin{exemple}\pjya{praʁsrɯm nɯ praʁ ɯ-ŋgɯ ku-kɯ-rɤʑi tɕe ɯ-rme wuma ɲɯ-mpɕɤr tɕe spoŋsrɤm cho tɕhɯɕrɤm nɯra kɯ-naχtɕɯɣ ɲɯ-ŋu-nɯ ɯʑo praʁ ɯ-ŋgɯ ku-rɤʑi ɲɯ-ŋu tɕe núndʐa praʁsrɯm ɲɯ-rmi}\hspace{5pt}\pcmn{\lien{ⓔpraʁsrɯm}{praʁsrɯm}是生活在岩石里的一种动物,毛长得很美,同水獭和\lien{ⓔspoŋsrɤm}{spoŋsrɤm} 一样,因为生活在岩石里所以叫 \lien{ⓔpraʁsrɯm}{praʁsrɯm}}\end{exemple}\étymologie{brag.sram}\end{entrée}

\begin{entrée}{praʁsrɯn}{}{ⓔpraʁsrɯn} 
\classe{n} 
\begin{définition}\pfra{nain, une sorte de démon}\end{définition}
\begin{définition}\pcmn{鬼的一种(矮人,投小石头)}\end{définition}\étymologie{brag.srin}\end{entrée}

\begin{entrée}{praʁʑɯn}{}{ⓔpraʁʑɯn} 
\classe{n} 
\begin{définition}\pfra{grotte}\end{définition}
\begin{définition}\pcmn{山洞}\end{définition}\end{entrée}

\begin{entrée}{praχpa}{}{ⓔpraχpa} 
\classe{n} 
\begin{définition}\pfra{caverne sous la falaise}\end{définition}
\begin{définition}\pcmn{岩洞}\end{définition}\end{entrée}

\begin{entrée}{prɤdɤja,ta}{}{ⓔprɤdɤja,ta} 
\classe{n}
\classe{vt} \sens{1}
\begin{définition}\pfra{griffer partout}\end{définition}
\begin{définition}\pcmn{到处乱抓}\end{définition}\sens{2}
\begin{définition}\pfra{mettre en désordre (oiseau)}\end{définition}
\begin{définition}\pcmn{乱撒东西(鸟)}\end{définition}
\begin{exemple}\pjya{kumpɣa kɯ ɯ-thoʁ ra chɯ-rɤβraʁ tɕe, prɤdɤja ʑo pjɯ-te ŋu}\hspace{5pt}\pcmn{鸡把地面抓得到处都是抓过的痕迹}\end{exemple}\relationsémantique{Component 1}{\lien{}{prɤdɤja}}\relationsémantique{Component 2}{\lien{ⓔta}{ta}}\relationsémantique{参考}{\lien{ⓔpriⓗ1}{pri₁}}\end{entrée}

\begin{entrée}{prɤftsa}{}{ⓔprɤftsa} 
\classe{n} 
\begin{définition}\pfra{ours noir}\end{définition}
\begin{définition}\pcmn{狗熊}\end{définition}\relationsémantique{参考}{\lien{ⓔpriⓗ2}{pri₂}}\relationsémantique{参考}{\lien{ⓔtɤ-ftsa}{tɤ-ftsa}}\end{entrée}

\begin{entrée}{prɤku}{}{ⓔprɤku} 
\classe{n}  
\grammaire{n.lieu} 
\begin{définition}\pfra{l'un des hameaux de Kamnyu}\end{définition}
\begin{définition}\pcmn{干木鸟的大队之一}\end{définition}\end{entrée}

\begin{entrée}{prɤm}{}{ⓔprɤm} 
\classe{vt} \paradigme{dir}{pɯ-}
\begin{définition}\pfra{ajouter de la farine}\end{définition}
\begin{définition}\pcmn{加面粉(水、汤里)}\end{définition}
\begin{exemple}\pjya{paʁtshi pɯ-prɤm}\hspace{5pt}\pcmn{在猪食里加一点面粉吧}\end{exemple}
\begin{exemple}\pjya{a-tʂha ci pɯ-prɤm}\hspace{5pt}\pcmn{给我的茶加点面粉吧}\end{exemple}\relationsémantique{参考}{\lien{ⓔtɤ-prɤm}{tɤ-prɤm}}\end{entrée}

\begin{entrée}{prɤmɤl/\variante{n-prAmAl}}{}{ⓔprɤmɤl} 
\classe{n} 
\begin{définition}\pfra{partie couverte sous les sapins où l'on peut se protéger de la pluie}\end{définition}
\begin{définition}\pcmn{高大的杉树能遮雨的地方}\end{définition}\end{entrée}

\begin{entrée}{prɤmchi}{}{ⓔprɤmchi} 
\classe{n} 
\begin{définition}\pfra{bile d'ours}\end{définition}
\begin{définition}\pcmn{熊胆}\end{définition}\relationsémantique{参考}{\lien{ⓔpriⓗ2}{pri₂}}\relationsémantique{参考}{\lien{ⓔtɯ-mchi}{tɯ-mchi}}\end{entrée}

\begin{entrée}{prɤndʐi}{}{ⓔprɤndʐi} 
\classe{n} 
\begin{définition}\pfra{peau d'ours}\end{définition}
\begin{définition}\pcmn{熊皮子}\end{définition}\relationsémantique{参考}{\lien{ⓔpriⓗ2}{pri₂}}\relationsémantique{参考}{\lien{ⓔtɯ-ndʐi}{tɯ-ndʐi}}\end{entrée}

\begin{entrée}{prɤɲaʁ}{}{ⓔprɤɲaʁ} 
\classe{n} 
\begin{définition}\pfra{ours noir}\end{définition}
\begin{définition}\pcmn{狗熊}\end{définition}\relationsémantique{参考}{\lien{ⓔpriⓗ2}{pri₂}}\end{entrée}

\begin{entrée}{prɤɲi}{}{ⓔprɤɲi} 
\classe{n} 
\begin{définition}\pfra{reflets empourprés (au crépuscule ou à l'aube)}\end{définition}
\begin{définition}\pcmn{早霞;晚霞}\end{définition}\end{entrée}

\begin{entrée}{prɤschɯ}{}{ⓔprɤschɯ} 
\classe{n}  
\grammaire{n.lieu} 
\begin{définition}\pfra{l'un des hameaux de Kamnyu}\end{définition}
\begin{définition}\pcmn{干木鸟的大队之一}\end{définition}\end{entrée}

\begin{entrée}{prɤt}{}{ⓔprɤt} 
\classe{vt} \paradigme{dir}{nɯ-}\paradigme{dir}{pɯ-}\paradigme{dir}{pɯ-}
\begin{définition}\pfra{casser (fil, corde), couper}\end{définition}
\begin{définition}\pcmn{弄断(线);迸断}\end{définition}
\begin{définition}\pfra{faire abandonner (une mauvaise habitude) à qqn}\end{définition}
\begin{définition}\pcmn{让……断绝(坏习惯)}\end{définition}
\begin{exemple}\pjya{tɯmbri pɯ-prɤt}\hspace{5pt}\pcmn{你把绳子弄断吧}\end{exemple}
\begin{exemple}\pjya{ɕomskrɯt nɯ-prɤt}\hspace{5pt}\pcmn{你把铁丝弄断吧}\end{exemple}
\begin{exemple}\pjya{tɤ-ri nɯ-prɤt}\hspace{5pt}\pcmn{你把线弄断吧}\end{exemple}
\begin{exemple}\pjya{a-wa ɯ-thamakha-sko kɤ-sɯprɤt mɯ-pɯ-cha-a}\hspace{5pt}\pcmn{我没能控制住我父亲抽烟的习惯}\end{exemple}\relationsémantique{参考}{\lien{ⓔmbrɤt}{mbrɤt}}\relationsémantique{参考}{\lien{ⓔrɤmprɤt}{rɤmprɤt}}
\begin{sous-entrée}{sɯprɤt}{ⓔprɤtⓝsɯprɤt} 
\classe{vt} \end{sous-entrée}

\begin{sous-entrée}{nɤpɯprɤt}{ⓔprɤtⓝnɤpɯprɤt} 
\classe{vt} 
\begin{définition}\pfra{couper dans tous les sens}\end{définition}
\begin{définition}\pcmn{断来断去}\end{définition}\end{sous-entrée}

\end{entrée}

\begin{entrée}{pri}{₂}{ⓔpriⓗ2} 
\classe{n} 
\begin{définition}\pfra{ours}\end{définition}
\begin{définition}\pcmn{熊}\end{définition}
\begin{exemple}\pjya{pri nɯ χsɯ-tɯphu tu, ndzɤpri kɯ-rmi ci tu, prɤɲaʁ kɯ-rmi ci tu, prɤftsa kɯ-rmi ci tu, ndzɤpri nɯ pɤjmu ruŋgɯ nɯ tɕu tu ɲɯ-ngrɤl, sɤ-ndza ɲɯ-ŋgrɤl, ɯ-mdoʁ nɯ kɯ-pɣi kɯ-ɤɣɯrnɯɕɯr ɲɯ-ŋu, ɯ-mɤlɤjaʁ nɯ prɤɲaʁ cho kɯ-naχtɕɯɣ ɕti, qrorni cho sɯmat tu-ndze ɲɯ-ŋgrɤl. prɤɲaʁ nɯ kɯ-wxti ŋu, tɯ-ji ɯ-ŋgɯ kɯ-nɤru ju-ɣi ŋgrɤl, jima stoʁ ndze ŋgrɤl, tú-wɣ-nɤrʁaʁ tɕe tɯ-phe jɤ-armbat tɕe χsɯ-mɢla jamar kɯ-tu tɕe tu-ndzur ɲɯ-ŋgrɤl, tɕe tu-kɯ-mtsɯɣ tɤ-ɣɯɣu ŋgrɤl. kɤ-nɯsɯku wuma rkaŋ, ɕkrɤz ɯ-mat tu-ndze tɕe pjɯ-χtsɤβ ŋgrɤl ɕti. ɯʑo nɯ kɯ-ɲaʁ ŋu, ɯ-ro χcɤl zɯ kɯ-wɣrum tɯ-snaʁ tu, ɯ-mtɕhi amtɕoʁ tsa. ɯ-mɤlɤjaʁ tɯrme ɣɯ tsa fse. prɤftsa kɯ-xtɕi tsa ŋu, prɤɲaʁ cho naχtɕɯɣ tsa, sɤ-ndza tu-kɯ-ti ɲɯ-ŋu.}\hspace{5pt}\pcmn{熊有三种,一种叫马熊、一种叫狗熊、另一种叫\lien{ⓔprɤftsa}{prɤftsa}。马熊一般生活在贝母山上,会吃人,颜色是灰里带红的,四肢和狗熊的一样,吃红蚂蚁和树果。狗熊很大,经常到田地里来偷吃粮食,吃玉米和胡豆。捕猎它的时候,相隔距离只有三步的时候它就会站立起来,准备咬人。善于爬树,吃青冈树的果实的时候还要把(树的枝桠)搓揉起来。身子是黑色的,在胸部中间有白毛,嘴巴有点尖。四只脚和人的一样。\lien{ⓔprɤftsa}{prɤftsa}长得比较小,有点像狗熊,据说吃人。}\end{exemple}\end{entrée}

\begin{entrée}{pri}{₁}{ⓔpriⓗ1} 
\classe{vt} \paradigme{dir}{thɯ-}
\begin{définition}\pfra{déchirer}\end{définition}
\begin{définition}\pcmn{撕}\end{définition}
\begin{exemple}\pjya{tɯ-ŋga thɯ-pri-t-a}\hspace{5pt}\pcmn{我撕了衣服}\end{exemple}
\begin{exemple}\pjya{jɯɣi thɯ-pri-t-a}\hspace{5pt}\pcmn{我撕了书}\end{exemple}\relationsémantique{参考}{\lien{ⓔmbriⓗ2}{mbri₂}}\relationsémantique{参考}{\lien{ⓔprɤdɤja,ta}{prɤdɤja,ta}}\end{entrée}

\begin{entrée}{prɯ}{}{ⓔprɯ} 
\classe{vs} 
\begin{définition}\pfra{imperméable}\end{définition}
\begin{définition}\pcmn{不进水}\end{définition}
\begin{exemple}\pjya{tɯwɯr nɯ tú-wɣ-nɤqhɤŋga jɤɣ ma wuma ʑo prɯ}\hspace{5pt}\pcmn{雨衣可以披在肩上,不容易进水}\end{exemple}\end{entrée}

\begin{entrée}{prɯɣprɯɣ}{}{ⓔprɯɣprɯɣ} 
\classe{idph.2} \sens{1}
\begin{définition}\pfra{très serré}\end{définition}
\begin{définition}\pcmn{形容很紧,不容易解开}\end{définition}
\begin{exemple}\pjya{rgɤm ɯ-ŋgɯ prɯɣprɯɣ ʑo pjɯ-ɤrku ɕti}\hspace{5pt}\pcmn{箱子里装得很满}\end{exemple}
\begin{exemple}\pjya{tɤ-fkɯm ɯ-ŋgɯ prɯɣprɯɣ ʑo chɯ-ɤrku}\hspace{5pt}\pcmn{口袋里装得很满}\end{exemple}
\begin{exemple}\pjya{tɤ-mtɯ prɯɣprɯɣ chɤ-lɤt}\hspace{5pt}\pcmn{把扣子扣得很紧}\end{exemple}\sens{2}
\begin{définition}\pfra{bien rassasié}\end{définition}
\begin{définition}\pcmn{形容吃饱的样子}\end{définition}
\begin{exemple}\pjya{a-tɯ-fka kɯ prɯɣprɯɣ ʑo ɲɯ-pa}\hspace{5pt}\pcmn{我吃得很饱}\end{exemple}
\begin{exemple}\pjya{tɤ-fka prɯɣprɯɣ ʑo}\hspace{5pt}\pcmn{他很饱}\end{exemple}\end{entrée}

\begin{entrée}{prɯŋprɯŋ}{}{ⓔprɯŋprɯŋ} 
\classe{idph.2} 
\begin{définition}\pfra{solide}\end{définition}
\begin{définition}\pcmn{形容(绑得)很紧、很扎实}\end{définition}
\begin{exemple}\pjya{prɯŋprɯŋ ʑo ko-tʂɯβ}\hspace{5pt}\pcmn{他缝得很扎实}\end{exemple}\end{entrée}

\begin{entrée}{pɯɕɯɣ}{}{ⓔpɯɕɯɣ} 
\classe{n} 
\begin{définition}\pfra{sac à poudre}\end{définition}
\begin{définition}\pcmn{装火药的皮袋}\end{définition}
\begin{exemple}\pjya{pɯɕɯɣ nɯ ɕɤmɯɣdɯ ɣɯ ɯ-fkɯm mɯ-ɲɯ-kɯ-sɤci ɯ-spa ŋu}\hspace{5pt}\pcmn{枪袋是用来防止枪杆受潮的。}\end{exemple}\end{entrée}

\begin{entrée}{pɯɣ}{}{ⓔpɯɣ} 
\classe{vs} \paradigme{dir}{tɤ-}
\begin{définition}\pfra{se gonfler}\end{définition}
\begin{définition}\pcmn{膨胀}\end{définition}
\begin{exemple}\pjya{tɤɕi kɤ́-wɣ-sqa tɕe to-pɯɣ tɕe tɯthɯ ɯ-ŋgɯ mɯ-ɲɤ-xtɕhɯt}\hspace{5pt}\pcmn{煮青稞的时候,煮熟了就发胀,锅里就装不下了}\end{exemple}\end{entrée}

\begin{entrée}{pɯlthi}{}{ⓔpɯlthi} 
\classe{n} 
\begin{définition}\pfra{mèche}\end{définition}
\begin{définition}\pcmn{火绳}\end{définition}
\begin{exemple}\pjya{pɯlthi nɯ smi sɤ-zwɤr ŋu}\hspace{5pt}\pcmn{火绳是用来点火的部件。}\end{exemple}\end{entrée}

\begin{entrée}{pɯnbu}{}{ⓔpɯnbu} 
\classe{n} 
\begin{définition}\pfra{bonpo}\end{définition}
\begin{définition}\pcmn{黑教}\end{définition}\étymologie{bon.po}\end{entrée}

\begin{entrée}{pɯthaŋ}{}{ⓔpɯthaŋ} 
\classe{n} 
\begin{définition}\pfra{type de métal}\end{définition}
\begin{définition}\pcmn{白铜}\end{définition}\end{entrée}

\begin{entrée}{pɯtshɯŋ}{}{ⓔpɯtshɯŋ} 
\classe{n} 
\begin{définition}\pfra{petit récipient en cuivre}\end{définition}
\begin{définition}\pcmn{红铜小罐子}\end{définition}
\begin{exemple}\pjya{pɯtshɯŋ nɯ kontsi kɯ-xtɕɯ-xtɕi ci ŋu, tɯ-ci rɟɤpɕɤt jamar ma mɤ-tɕhɯt, ɯ-spa nɯ li tu-kɯ-ti ɲɯ-ŋu, ɯ-mdoʁ aɣrɤɣrum tsa, ɯ-mŋu zɯ ɯ-mtɕhi kɯ-zri tsa ʑo tu, ɯ-jɯ wuma ʑo ngɯt, ɯ-xtu mɤ-wxti, ɯ-qa nɯ kɯ-ntɯ-ntaβ ŋu. kɯɕɯŋgɯ tɕe, tɯrme kɯ-ŋgro ra ɣɯ nɯ-cha ɯ-z-nɤrkɯrku pjɤ-ŋu.}\hspace{5pt}\pcmn{\lien{ⓔpɯtshɯŋ}{pɯtshɯŋ}是比较小的罐子,只容得下半斤水,据说是用铜做成,颜色有点白,有比较长的嘴,把很结实,肚子不大,底部放起来很稳。过去,是用来给贵宾倒酒的罐子。}\end{exemple}\end{entrée}

\begin{entrée}{pɯwɯ}{}{ⓔpɯwɯ} 
\classe{n} 
\begin{définition}\pfra{âne}\end{définition}
\begin{définition}\pcmn{驴子}\end{définition}\end{entrée}

\begin{entrée}{pɯz}{}{ⓔpɯz} 
\classe{vs} \paradigme{dir}{tɤ-}
\begin{définition}\pfra{pourri (bois), usé jusqu'à la corde (habits)}\end{définition}
\begin{définition}\pcmn{朽烂(木头),破烂(衣服)}\end{définition}\end{entrée}

\newpage\caractère{q}

\begin{entrée}{qachɣa}{}{ⓔqachɣa} 
\classe{n} 
\begin{définition}\pfra{renard}\end{définition}
\begin{définition}\pcmn{狐狸}\end{définition}\end{entrée}

\begin{entrée}{qachɣamɯntoʁ}{}{ⓔqachɣamɯntoʁ} 
\classe{n} 
\begin{définition}\pfra{Eugeron breviscapus}\end{définition}
\begin{définition}\pcmn{短葶飞蓬}\end{définition}\relationsémantique{反义词}{\lien{ⓔqachɣarte}{qachɣarte}}\end{entrée}

\begin{entrée}{qachɣarte}{}{ⓔqachɣarte} 
\classe{n} 
\begin{définition}\pfra{Eugeron breviscapus}\end{définition}
\begin{définition}\pcmn{短葶飞蓬}\end{définition}\relationsémantique{参考}{\lien{ⓔqachɣamɯntoʁ}{qachɣamɯntoʁ}}\end{entrée}

\begin{entrée}{qachɣɤndʐi}{}{ⓔqachɣɤndʐi} 
\classe{n} 
\begin{définition}\pfra{peau de renard}\end{définition}
\begin{définition}\pcmn{狐狸皮子}\end{définition}
\begin{exemple}\pjya{qachɣɤndʐi ɲɯ-fkra}\hspace{5pt}\pcmn{狐狸皮子彩色斑斓}\end{exemple}\relationsémantique{参考}{\lien{ⓔqachɣa}{qachɣa}}\end{entrée}

\begin{entrée}{qaɕɣi}{}{ⓔqaɕɣi} 
\classe{n} 
\begin{définition}\pfra{grosse mouche}\end{définition}
\begin{définition}\pcmn{大苍蝇}\end{définition}
\begin{exemple}\pjya{qaɕɣi nɯ βɣɤza sɤznɤ wxti, βɣɤza kɯβde jamar tu, ɯ-tshɯɣa nɯ βɣɤza cho naχtɕɯɣ, ɯ-ku artɯm, ɯ-mɲaʁ kɯ-wxtɯ-wxti tu, ɯ-ʁrɯ tu, ɯ-mɤlɤjaʁ kɯtʂɤ-ldʑa tu, ɯ-ʁar mba ɯ-rɯmu tu, mɤ-sɤmtsɯɣ, tɕeri wuma ʑo ŋɤn ma ftɕar tɕe ɕɤkhe ɯ-taʁ ʑo ɯ-qe ku-lɤt ŋu, nɯ ɯ-qe nɯ laʁnɤ-rʑaʁ ʑo tɕe tɯ-mɯnmɯ tu-ʑe, tɕe chɯ-wxti tɕe, ɕa tu-ndze ɲɯ-ŋu, kɯ-mɯm ɯ-stu ʑo tu-ndze ɲɯ-ŋu, tɕe tu-ndze mɤ-kɯ-jɤɣ kɯ ɯ-di ɲɯ-ɕɯmnɤm tɕe, mɯ-ta-ndza nɯ ra kɯnɤ ɯ-di ɲɯ-ɕɯmnɤm tɕe kɤ-ndza mɯ-ɲɯ-ɣɤsne ŋu. thu-wxti tɕe, pjɯ-nɯɬoʁ tɕe, kɯm ɯ-qhu zndɤrchɤβ nɯ ra ju-nɯrtsɯ-nɯ tɕe ju-ormbɯrmbɯ-nɯ ŋu, ɯntɕe nɯ-rqhu ɲɯ-βze tɕe, qartsɯ nɯ tɕu ku-rɤʑi-nɯ, χɕitka tɕe, ɯ-rqhu ɯ-ŋgɯ ri qaɕɣi ɲɯ-βze tɕe, li ju-nɯɬoʁ-nɯ ŋu, tɕe tɯrme ra kɯ qaɕɣi pa-mto-nɯ tɕe pjɯ-sat-nɯ ɕti ma ɯ-kɯ-qha ʁɟa ŋu.}\hspace{5pt}\pcmn{大苍蝇比小苍蝇大四倍,形状完全和小苍蝇一样,头部是圆形的,有一对大眼睛,有触角,有六只脚。翅膀很薄,有纹路。不蜇人,但是很坏。夏天在瘦肉上下卵子(屙屎),过一两就开始动,吃肉,专门吃最好吃的部位,不但吃,还使肉变臭,连没有吃到的地方都会发臭,使得肉不能再吃了。长成了以后就出来掉到地上,爬到门后面墙壁缝里聚集在一起,然后变成蛹,在那里过冬天。到了春天蛹就变成苍蝇,从里面飞出来。人们看到苍蝇时就会把它弄死,因为都很讨厌它。}\end{exemple}\end{entrée}

\begin{entrée}{qaɕpa}{}{ⓔqaɕpa} 
\classe{n} 
\begin{définition}\pfra{grenouille}\end{définition}
\begin{définition}\pcmn{青蛙}\end{définition}\étymologie{sbal}\end{entrée}

\begin{entrée}{qaɕparaz}{}{ⓔqaɕparaz} 
\classe{n} 
\begin{définition}\pfra{espèce d'herbe}\end{définition}
\begin{définition}\pcmn{草的一种}\end{définition}\end{entrée}

\begin{entrée}{qaɕpɤrnoʁ}{}{ⓔqaɕpɤrnoʁ} 
\classe{n} 
\begin{définition}\pfra{fraise sauvage}\end{définition}
\begin{définition}\pcmn{野草莓}\end{définition}\relationsémantique{参考}{\lien{ⓔqaɕpa}{qaɕpa}}\relationsémantique{参考}{\lien{ⓔtɯ-rnoʁ}{tɯ-rnoʁ}}\end{entrée}

\begin{entrée}{qaɕti}{}{ⓔqaɕti} 
\classe{n} 
\begin{définition}\pfra{pêche}\end{définition}
\begin{définition}\pcmn{桃子}\end{définition}
\begin{exemple}\pjya{qaɕti nɯ si kɯ-wxti tsa ci ŋu, ɯ-ru nɯ kɯ-pɣi ŋu, ɯ-rtaʁ dɤn, wuma ʑo ɲɯ-jpum mɤ-cha, ɯ-jwaʁ nɯ kɯ-tɕɤr tɕe kɯ-ɤmtɕoʁ tsa ci ŋu. ɯ-mɯntoʁ nɯ ɯ-jwaʁ ɲɤ-lɤt ɕɯŋgɯ ɲɯ-lɤt ŋu, ɯ-mɯntoʁ kɯ-wɣrum tu, kɯ-ɣɯrni tu, ɯ-mat nɯ ɯ-rme kɯ-fse sɯβ-sɯβ tu, arŋi tsa thɯ-tɯt tɕe qarŋe. tú-wɣ-ndza tɕe kɯ-chi tu, kɯ-tɕur tu. ɯ-si nɯ ngɯt tɕe laʁdɯn ɯ-jɯ kɤ-nɯ-βzu sna.}\hspace{5pt}\pcmn{桃树是比较高大的树种,树干是灰色的,枝桠多,都长得不是很粗,叶子细而尖,花开在叶子长出来之前,有的是白色的,有的是红色的。果实上有绒毛,先是绿色的,成熟后变黄。吃起来有的是甜的,有的是酸的。木质结实,可以用来作农具的把子。}\end{exemple}\relationsémantique{参考}{\lien{ⓔnɯqaɕti}{nɯqaɕti}}\end{entrée}

\begin{entrée}{qafsa}{}{ⓔqafsa} 
\classe{n} 
\begin{définition}\pfra{type de plante rampante}\end{définition}
\begin{définition}\pcmn{藤树的一种}\end{définition}\end{entrée}

\begin{entrée}{qaj}{}{ⓔqaj} 
\classe{n} 
\begin{définition}\pfra{blé}\end{définition}
\begin{définition}\pcmn{麦}\end{définition}\end{entrée}

\begin{entrée}{qajdu}{}{ⓔqajdu} 
\classe{n} 
\begin{définition}\pfra{une espèce d'arbre}\end{définition}
\begin{définition}\pcmn{乔木的一种}\end{définition}
\begin{exemple}\pjya{qajdu nɯ si khro mɤ-kɯ-mbro tsa ci ŋu, zgoku aʁɤndɯndɤt tu-ɬoʁ cha, si ɯ-βri nɯ kɯ-ɲaʁ ʁɟa ʑo ŋu, ɯ-jwaʁ kɯnɤ ldʑaŋnaʁ ɕti, ɯ-mdzu wuma ʑo mtɕoʁ, tɯ-ɕa ɯ-taʁ kɤ-tsa tɕe tɤ-spɯ ʑo tu-sɤβze cha. ɯ-mat kɯnɤ thɯ-tɯt tɕe kɯ-ɲaʁ chɯ-βze ŋu}\hspace{5pt}\pcmn{\lien{ⓔqajdu}{qajdu}是长得不是高的树种,山上山下到处都可以生长,树皮(树身)全是黑色的,连树叶都是深绿色的。刺很锋利,刺到皮肉上会长脓包。果实成熟后也是黑色的}\end{exemple}
\begin{exemple}\pjya{nɤ-sŋi ɯ-tɯ-ɲaʁ kɯ qajdu ʑo ɲɯ-tɯ-fse}\hspace{5pt}\pcmn{你的心黑得像\lien{ⓔqajdu}{qajdu}一样(你真狠心)。}\end{exemple}\end{entrée}

\begin{entrée}{qajdo}{}{ⓔqajdo} 
\classe{n} 
\begin{définition}\pfra{corbeau (corvus corone)}\end{définition}
\begin{définition}\pcmn{小嘴乌鸦}\end{définition}\end{entrée}

\begin{entrée}{qajdoɕku}{}{ⓔqajdoɕku} 
\classe{n} 
\begin{définition}\pfra{poireau sauvage}\end{définition}
\begin{définition}\pcmn{高山上的野韭菜}\end{définition}
\begin{exemple}\pjya{qajdoɕku nɯ aʁɤndɯndɤt sɯŋgɯ tu-ɬoʁ cha, ɯ-jwaʁ ɯ-qhuchu nɯ wɣrum, nɯ maʁ nɯ, ɕkɤpja tsa fse, ri ɯ-ru ɯ-taʁ kɯ-ɤrɤʑɯʑrɤz nɯ me. ɯ-mɯntoʁ nɤmkha mdoʁ ɲɯ-lɤt ŋu. tú-wɣ-ndza tɕe, ɯ-di mɤ-mɯm.}\hspace{5pt}\pcmn{\lien{ⓔqajdoɕku}{qajdoɕku}能在森林里到处生长,叶子背面是白色的,其他部位跟 \lien{ⓔɕkɤpja}{ɕkɤpja}一样,但茎上没有纹路,开天蓝色的花。吃起来不香。}\end{exemple}\end{entrée}

\begin{entrée}{qajɣi}{}{ⓔqajɣi} 
\classe{n} 
\begin{définition}\pfra{pain}\end{définition}
\begin{définition}\pcmn{馍馍}\end{définition}\end{entrée}

\begin{entrée}{qajo}{}{ⓔqajo} 
\classe{n} 
\begin{définition}\pfra{récipient en terre}\end{définition}
\begin{définition}\pcmn{土罐子}\end{définition}\end{entrée}

\begin{entrée}{qajpɣom}{}{ⓔqajpɣom} 
\classe{vs} \paradigme{dir}{nɯ-}
\begin{définition}\pfra{geler}\end{définition}
\begin{définition}\pcmn{冻到}\end{définition}
\begin{exemple}\pjya{@yangyu ɲɤ-qajpɣom}\hspace{5pt}\pcmn{洋芋冻到了}\end{exemple}
\begin{exemple}\pjya{lɤpɯɣ ɲɤ-qajpɣom}\hspace{5pt}\pcmn{萝卜冻到了}\end{exemple}\relationsémantique{参考}{\lien{ⓔjpɣom}{jpɣom}}\end{entrée}

\begin{entrée}{qajru}{}{ⓔqajru} 
\classe{n} 
\begin{définition}\pfra{tige de blé}\end{définition}
\begin{définition}\pcmn{麦秸}\end{définition}\end{entrée}

\begin{entrée}{qajrqhu}{}{ⓔqajrqhu} 
\classe{n} 
\begin{définition}\pfra{son}\end{définition}
\begin{définition}\pcmn{麦麸}\end{définition}\end{entrée}

\begin{entrée}{qajrutʂu}{}{ⓔqajrutʂu} 
\classe{n} 
\begin{définition}\pfra{torche}\end{définition}
\begin{définition}\pcmn{火把}\end{définition}\end{entrée}

\begin{entrée}{qajtsrɯ}{}{ⓔqajtsrɯ} 
\classe{n} 
\begin{définition}\pfra{pousses de blé}\end{définition}
\begin{définition}\pcmn{麦苗}\end{définition}\end{entrée}

\begin{entrée}{qajtʂha}{}{ⓔqajtʂha} 
\classe{n} 
\begin{définition}\pfra{vautour (aegyptius monachus)}\end{définition}
\begin{définition}\pcmn{秃鹫}\end{définition}
\begin{exemple}\pjya{qajtʂha nɯ kuwu cho naχtɕɯɣ ri sɤznɤ xtɕi tsa, ɯ-ku kɯ-ɲaʁ ŋu, ɯ-mtɕhi-rme tu tɕe nɯ kɯnɤ kɯ-ɲaʁ ŋu, ɯ-phoŋbu kɯ-pɣi kɯ-ɤɲaʁndzɯm tsa ŋu, tɕe praʁ ɯ-ŋgɯ ku-rɤʑi ŋu, kɯ-rgɤz ra kɯ qajtʂha tɤ-ŋke tɕe tɯ-mɯ khe tu-ti-nɯ ŋgrɤl. qajɯ kɯ-fse ra tu-ndze ma kɯmaʁ rɯdaʁ kɯ-fse ra mɤ-ndze, tɤ-rɤku ri mɤ-ndze.}\hspace{5pt}\pcmn{秃鹫和胡兀鹫长得差不多,但小一点,头是黑色的,有胡须,也是黑色的,身子是灰里带黑的,栖息在岩洞里。老年人们说,当秃鹫出现时,天会变阴。它吃虫子,不吃其它动物,也不吃粮食。}\end{exemple}\end{entrée}

\begin{entrée}{qajɯ}{}{ⓔqajɯ} 
\classe{n} 
\begin{définition}\pfra{insecte, vers}\end{définition}
\begin{définition}\pcmn{虫}\end{définition}\relationsémantique{参考}{\lien{ⓔrɯqajɯ}{rɯqajɯ}}\end{entrée}

\begin{entrée}{qajɯβlama}{}{ⓔqajɯβlama} 
\classe{n} 
\begin{définition}\pfra{petite sangsue}\end{définition}
\begin{définition}\pcmn{小水蛭}\end{définition}\end{entrée}

\begin{entrée}{qajɯkɯrɤtɣa}{}{ⓔqajɯkɯrɤtɣa} 
\classe{n} 
\begin{définition}\pfra{chenille arpenteuse}\end{définition}
\begin{définition}\pcmn{尺蠖}\end{définition}
\begin{exemple}\pjya{qajɯ kɯ-rɤtɣa nɯ qajɯ kɯ-mpɯ-mpɯ kɯ-xtɕɯ-xtɕi ci ŋu, tɕe ɯ-zda ra kɯ-fse tu-nɯrtsɯ ɲɯ-maʁ, ju-rɤtɣe tɕe, ju-ɕe ɲɯ-ŋu, ɯ-ku nɯ kɯ-ɤrqhi tsa ju-tsrɤt ju-te tɕe, jme nɯ ju-mɟe tɕe ɯ-ku ɯ-ɕki ʑo ju-sɤzɣɯt tɕe, li ɯ-ku nɯ ju-tsrɤt, tɕe nɯ kɯ-fse ju-ɕe ɲɯ-ŋu tɕe, núndʐa qajɯ kɯ-rɤtɣa ɲɯ-rmi.}\hspace{5pt}\pcmn{尺蠖是一种又软又小的虫子,不像其它虫子一样爬着走,好像是人在用手指量尺寸那样走动,先把头部往前伸出去,然后把尾部收到头部那里,然后又把头部伸出去,就这样行走,所以叫作尺蠖。}\end{exemple}\end{entrée}

\begin{entrée}{qajɯkɯsɤtʂot}{}{ⓔqajɯkɯsɤtʂot} 
\classe{n} 
\begin{définition}\pfra{luciole}\end{définition}
\begin{définition}\pcmn{萤火虫}\end{définition}\end{entrée}

\begin{entrée}{qajɯsmɤnba}{}{ⓔqajɯsmɤnba} 
\classe{n} 
\begin{définition}\pfra{sangsue}\end{définition}
\begin{définition}\pcmn{水蛭}\end{définition}\end{entrée}

\begin{entrée}{qajɯstoʁ}{}{ⓔqajɯstoʁ} 
\classe{n} 
\begin{définition}\pfra{une espèce d'insecte}\end{définition}
\begin{définition}\pcmn{昆虫的一种}\end{définition}
\begin{exemple}\pjya{qajɯstoʁ nɯ qajɯ kɯ-ɲaʁ ci ŋu, ɯ-ʁar ɯ-rqhu nɯ kɯ-rkɯ-rko ci ŋu, tɕe nɤmbju, ɯ-xtu fka, ɯ-smɤt tɕe chɯ-ɤmtɕoʁ ŋu, ɯ-ku kɯ-xtɕɯ-xtɕi ŋu, ɯ-mtɕhi amtɕoʁ, ɯ-mɤlɤjaʁ kɯtʂɤ-ldʑa tu, zndɤrchɤβ, soʁma ɯ-rchɤβ, aʁɤndɯndɤt kɯ-ɴqhi nɯ ra ku-rɤʑi ŋu.}\hspace{5pt}\pcmn{\lien{}{qajɯ stoʁ}是一种黑色的虫,翅膀的壳很硬,有光泽,肚子很胀,尾部是尖的,头部很小,嘴很尖,有六只脚,生活所有脏的地方,比如在墙壁缝和干草堆}\end{exemple}\end{entrée}

\begin{entrée}{qajʑmbraʁ}{}{ⓔqajʑmbraʁ} 
\classe{n} 
\begin{définition}\pfra{barbe de blé}\end{définition}
\begin{définition}\pcmn{麦芒}\end{définition}\end{entrée}

\begin{entrée}{qaɟy/\variante{qaɟwi}}{}{ⓔqaɟy} 
\classe{n} 
\begin{définition}\pfra{poisson}\end{définition}
\begin{définition}\pcmn{鱼}\end{définition}\end{entrée}

\begin{entrée}{qaɟɤɣi}{}{ⓔqaɟɤɣi} 
\classe{n} 
\begin{définition}\pfra{avoine}\end{définition}
\begin{définition}\pcmn{燕麦}\end{définition}
\begin{exemple}\pjya{qaɟɤɣi nɯ sɯjno ci ŋu, ɯ-jwaʁ ɯ-ru nɯ ra tɤɕi cho naχtɕɯɣ, ɯ-jwaʁ ɯ-βzɯr nɯ ɯ-rme kɯ-xtɕɯ-xtɕi tu, ɯ-mat nɯ tɤɕi cho mɤ-naχtɕɯɣ, mbrɤz kɯɕnom cho naχtɕɯɣ, ɯ-mat nɯ ɯ-rqhu wuma ʑo jaʁ tɕe, pjɯ-tsɣi mɤ-cha, tɕe tɕhi kɯ-fse tɤ-nɯ-nɤrʑaʁ kɯnɤ pjɯ-tsɣi mɤ-cha tɕe tu-ɬoʁ ɕti, tɕe ɯ-ʑmbraʁ nɯ chɤ-ndzɯ-ndzri ŋu tɕe, nɯ-aci tɕe lu-orlɯ-rla ŋu, tɕe ɯ-mat tu-sɯ-mtɕɯr cha. qaɟɤɣi tɤ-rɤku ɯ-rchɤβ tu-ɬoʁ tɕe, tɤ-rɤku ɯ-taʁ ʁnɤt tɕe, sɯjno wuma ʑo kɯ-ŋɤn ŋu. tɕe kɯ-rɤma ra kɯ wuma ʑo qha-nɯ tɕe, tɕhi kɤ-cha ɲɯ-kɤ-ɣɤme ftɕaka tu-βzu-nɯ ŋu.}\hspace{5pt}\pcmn{燕麦是一种草,叶子和茎和青稞一样,但是叶子的边缘有小毛,果实和青稞的不一样,和大米的穗一样。果实的皮很厚,不容易腐烂,再长的时间也不会腐烂,能生长。它的芒是拧着长的,受潮时就会自然松开,使果实转动。燕麦生长在庄稼里,对庄稼造成危害,是很坏的草,所以劳动人民很讨厌它,用一切办法来消灭它。}\end{exemple}\end{entrée}

\begin{entrée}{qaɟyri}{}{ⓔqaɟyri} 
\classe{n} 
\begin{définition}\pfra{type de pas d'aiguille}\end{définition}
\begin{définition}\pcmn{缝针的方法}\end{définition}\end{entrée}

\begin{entrée}{qala}{}{ⓔqala} 
\classe{n} 
\begin{définition}\pfra{lapin}\end{définition}
\begin{définition}\pcmn{兔子}\end{définition}\end{entrée}

\begin{entrée}{qalalu}{}{ⓔqalalu} 
\classe{n} 
\begin{définition}\pfra{année du lapin}\end{définition}
\begin{définition}\pcmn{兔年}\end{définition}\end{entrée}

\begin{entrée}{qalamɯjpɤt}{}{ⓔqalamɯjpɤt} 
\classe{n} 
\begin{définition}\pfra{coton sauvage}\end{définition}
\begin{définition}\pcmn{野棉花}\end{définition}
\begin{exemple}\pjya{qalamɯjpɤt nɯ tɯ-ji ɯ-rkɯ si ɯ-rchɤβ ra tu-ɬoʁ ŋu, ɯ-jwaʁ nɯ aɣrɤɣrum tɕe ɯ-rme kɯ-fse tu, ɯ-ru nɯ kɯ-ngɯ-ngɯt ʑo ŋu, ɯ-mɯntoʁ nɯ ɯ-ʁɤri nɯ kɯ-wɣrum, ɯ-qhuchu nɯ ʁmɤrsmɯɣ ŋu, ɯ-mat nɯ thɯ-tɯt tɕe, ɯ-ŋgɯ srɯn kɯ-fse ɲɯ-nɯɬoʁ ŋu, ɯ-rɣi nɯ kɯ-ndɯ-ndɯβ ʑo ŋu.}\hspace{5pt}\pcmn{野棉花生长在田地边缘和树木之间,叶子淡白,有毛,茎很结实,花正面白色,背面紫色,果实成熟后,里面就漏出像棉花一样的东西,种子非常小。}\end{exemple}\end{entrée}

\begin{entrée}{qalarnaftɕɯχa}{}{ⓔqalarnaftɕɯχa} 
\classe{n} 
\begin{définition}\pfra{lapin ayant dix trous dans les oreilles}\end{définition}
\begin{définition}\pcmn{耳朵上有十个缺口的兔子(故事里)}\end{définition}\end{entrée}

\begin{entrée}{qale}{}{ⓔqale} 
\classe{n} 
\begin{définition}\pfra{vent}\end{définition}
\begin{définition}\pcmn{风}\end{définition}
\begin{exemple}\pjya{qale to-βzu (jo-ɣɯt)}\hspace{5pt}\pcmn{风刮起来了}\end{exemple}
\begin{exemple}\pjya{iʑo pɯ-nɤʁaʁ-i tɕe pɯ-scit-i ri, tɯ-mɯ qale ja-ɣɯt tɕe tɤ-nɯɣe-j pɯ-ra}\hspace{5pt}\pcmn{我们在外面晒太阳的时候,风刮起来了,下雨了,我们只好回家了}\end{exemple}\relationsémantique{参考}{\lien{ⓔakɯchoʁle}{akɯchoʁle}}\end{entrée}

\begin{entrée}{qalekɯtshi}{}{ⓔqalekɯtshi} 
\classe{n} 
\begin{définition}\pfra{rapace}\end{définition}
\begin{définition}\pcmn{鹰科,无法定到种}\end{définition}
\begin{exemple}\pjya{qalekɯtshi nɯ pɣa kɯ-xtɕi tsa ci ŋu, ɲɯ-nɯqambɯmbjom ɯ-raŋ zɯ, nɤmkha zɯ tɯtshot χsɯ-skɤrma jamar ɯ-stu ku-rɤʑi ɲɯ-ŋgrɤl, tɕe tsuku kɯ tɕe qale ɯ-kɯ-tshi, tsɯku kɯ tɕe qale ɯ-kɯ-ndza tu-kɯ-ti ŋu, tɕhi tu-ste ŋgrɤl nɯ mɤ-xsi. ɯ-βri nɯ kɯ-pɣi tsa ci ŋu.}\hspace{5pt}\pcmn{\lien{ⓔqalekɯtshi}{qalekɯtshi}是一种比较小的鸟,飞的时候,可以在空中停住三分钟左右,有的人说它在挡风,有的说在吃风,不知道哪一种说法是对的。身子是灰色的。}\end{exemple}\end{entrée}

\begin{entrée}{qaliaʁ}{}{ⓔqaliaʁ} 
\classe{n} 
\begin{définition}\pfra{aigle}\end{définition}
\begin{définition}\pcmn{雕}\end{définition}
\begin{exemple}\pjya{qaliaʁ nɯ praʁ ɯ-ŋgɯ zɯ ku-rɤʑi ŋu, ʁnɯ-tɯphu tu, tɯ-tɯphu nɯ thaŋkɤr rmi, ɯʑo ɯ-mdoʁ nɯ kɯ-ɲaʁ ŋu, ɯ-ʁar nɯ kɯ-wɣrum tɯ-tɯ-snaʁ kɯ-tu tu, qala, βʑɯ, paʁtsa cho ɲaɲa nɯ ra tu-ndze ŋgrɤl. li ci thaŋnaʁ kɯ-rmi ci tu tɕe, kɯ-ɲaʁ ʁɟa ʑo ŋu, nɯ kɯ-cha nɯ ŋu, paʁtsa cho ɲaɲa nɯ phɤri ɲɯ-nɯ-tsɯm cha, ca pjɤ-sat ŋgrɤl. ɯ-mi nɯ ɕɤmiɕtʂɤt fse. mphrɯmɯ pjɯ-re ŋgrɤl tu-kɯ-ti ɲɯ-ŋu ma nɤmkha ɯ-stu ʑo tu-ɕe tɕe ɯ-stu li pjɯ-jɣɤt tɕe ɴɢartɯm pjɯ-ɣɯt ŋu. ɯ-mɲaʁ wuma ʑo mto tu-kɯ-ti ɲɯ-ŋgrɤl, phɤri ku-ru tɕe qaʑo kɯ ɯ-qe thɤstɯ-rdoʁ pa-lɤt mtɤm tu-kɯ-ti ɲɯ-ŋgrɤl. ɕa a-tɤ-ndze tɤ-fka ɯ-qhu tɕe, co a-pɯ-ŋu tɕe chɯ-nɯqambɯmbjom mɤ-cha.}\hspace{5pt}\pcmn{雕栖息在岩洞里,有两种,一种叫\lien{}{thaŋkɤr},是黑色的,翅膀下面有白点,吃兔子、老鼠、小猪、小羊等。另一种叫\lien{}{thaŋnaʁ},全身都是黑色的,比较凶,不但可以把小猪和小羊带走,而且能杀死麝香鹿。爪子像铁钩一样。人家说它会算卦,因为它在空中往上直飞,又转回往下直飞。据说它视力很强,能看清对面山上的羊屙的屎有多少颗。如果在山沟里的话,肉吃饱了以后就飞不起来。}\end{exemple}\étymologie{glag}\end{entrée}

\begin{entrée}{qalpɕa}{}{ⓔqalpɕa} 
\classe{vi} \paradigme{dir}{tɤ-}
\begin{définition}\pfra{s'ouvrir (feuille de fougère)}\end{définition}
\begin{définition}\pcmn{展开(蕨苔的叶子)【开反】}\end{définition}
\begin{exemple}\pjya{dɤrʁɯ to-qalpɕa}\hspace{5pt}\pcmn{蕨苔展开了}\end{exemple}\end{entrée}

\begin{entrée}{qambalɯla}{}{ⓔqambalɯla} 
\classe{n} 
\begin{définition}\pfra{papillon}\end{définition}
\begin{définition}\pcmn{蝴蝶}\end{définition}
\begin{exemple}\pjya{qambalɯla kɯ rŋgɯ mɤ-fkaβ}\hspace{5pt}\pcmn{蝴蝶盖不住大石包}\end{exemple}\end{entrée}

\begin{entrée}{qambɣo}{}{ⓔqambɣo} 
\classe{n} 
\begin{définition}\pfra{cérumen}\end{définition}
\begin{définition}\pcmn{耳垢}\end{définition}\end{entrée}

\begin{entrée}{qambrɯ}{}{ⓔqambrɯ} 
\classe{n} 
\begin{définition}\pfra{yak}\end{définition}
\begin{définition}\pcmn{公牦牛}\end{définition}\end{entrée}

\begin{entrée}{qambɯt}{}{ⓔqambɯt} 
\classe{n} 
\begin{définition}\pfra{sable}\end{définition}
\begin{définition}\pcmn{沙子}\end{définition}\end{entrée}

\begin{entrée}{qamdɯxtsa}{}{ⓔqamdɯxtsa} 
\classe{n} 
\begin{définition}\pfra{botte dont le haut est est peau de chevrotain, et le milieu en cuir teint en rouge}\end{définition}
\begin{définition}\pcmn{靴筒上部是獐皮子,下部是染成红色的牛皮的一种靴子}\end{définition}\end{entrée}

\begin{entrée}{qamdzi}{}{ⓔqamdzi} 
\classe{n}  
\grammaire{n.lieu} 
\begin{définition}\pfra{Dkarmdzes}\end{définition}
\begin{définition}\pcmn{甘孜州}\end{définition}\end{entrée}

\begin{entrée}{qame}{}{ⓔqame} 
\classe{n} 
\begin{définition}\pfra{grain de beauté}\end{définition}
\begin{définition}\pcmn{黑痣}\end{définition}\end{entrée}

\begin{entrée}{qamphoʁ}{}{ⓔqamphoʁ} 
\classe{n} 
\begin{définition}\pfra{feuille de chêne}\end{définition}
\begin{définition}\pcmn{青冈树的叶子}\end{définition}\end{entrée}

\begin{entrée}{qamphoʁɕɯrʁaʁ}{}{ⓔqamphoʁɕɯrʁaʁ} 
\classe{n} 
\begin{définition}\pfra{espèce de champignon}\end{définition}
\begin{définition}\pcmn{菌子的一种}\end{définition}\end{entrée}

\begin{entrée}{qamtɕɯr}{}{ⓔqamtɕɯr} 
\classe{n} 
\begin{définition}\pfra{musaraigne}\end{définition}
\begin{définition}\pcmn{尖鼠;鼩鼱}\end{définition}
\begin{exemple}\pjya{qamtɕɯr nɯ βʑɯ cho ndʑi-rme ra naχtɕɯɣ, tɕeri qamtɕɯr nɯ ɯ-mtɕhi mɤʑɯ ʑo amtɕoʁ, ɯ-jme xtɯt cho xtshɯm, qamtɕɯr nɯ βʑɯ sɤznɤ khro xtɕi, βʑɯ kɯ tɤ-mthɯm, tɯ-jpu, tɯ-ŋga tɕhi kɯ-tu tu-ndze ŋu ma qamtɕɯr kɯ tɤ-mthɯm kɯnɤ kɯ-tshu ma mɤ-ndze tɕeri tɤ-mthɯm tu-ndze ɯ-qhu ɯ-sta rmbi ɲɯ-lɤt ŋu tɕe ɯ-di wuma ʑo sɤjloʁ tɤ-mthɯm kɤ-ndza mɤ-kɯ-sna ɲɯ-sɤβze cha, tɕe wuma ʑo ŋɤn. tu-mbri tɕe ɯ-skɤt wuma ʑo amtɕoʁ χɕɤβ.}\hspace{5pt}\pcmn{鼩鼱和老鼠的毛一样,但鼩鼱的嘴比较尖,尾巴细而短。鼩鼱比老鼠小得多。老鼠吃肉、衣服、粮食,有什么就吃什么,而鼩鼱只吃肥肉,吃肉之后还在那里撒尿,味道很臭,使得肉不能吃,是很坏的动物。叫起来声音又尖又大。}\end{exemple}\end{entrée}

\begin{entrée}{qamtsɯrmdzu}{}{ⓔqamtsɯrmdzu} 
\classe{n} 
\begin{définition}\pfra{espèce de plante}\end{définition}
\begin{définition}\pcmn{植物的一种}\end{définition}\end{entrée}

\begin{entrée}{qamtsɯrpɣɤtɕɯ}{}{ⓔqamtsɯrpɣɤtɕɯ} 
\classe{n} 
\begin{définition}\pfra{espèce indéterminée}\end{définition}
\begin{définition}\pcmn{鸟的一种}\end{définition}\end{entrée}

\begin{entrée}{qamɯrwa}{}{ⓔqamɯrwa} 
\classe{n} 
\begin{définition}\pfra{chauve-souris}\end{définition}
\begin{définition}\pcmn{蝙蝠}\end{définition}\end{entrée}

\begin{entrée}{qandzɤjo}{}{ⓔqandzɤjo} 
\classe{n} 
\begin{définition}\pfra{espèce d'arbrisseau}\end{définition}
\begin{définition}\pcmn{灌木的一种}\end{définition}
\begin{exemple}\pjya{qandzɤjo nɯ si kɯ-mbɤr tsa ci ŋu, ɯ-jwaʁ ndɯβ, ɯʑo tu-mbro cho ɲɯ-jpum mɤ-cha, ɯ-si ngɯt tɕe kɯɕɯŋgɯ tɕe ndʑu ɯ-spa pɯ-ŋu. ɯ-mɯntoʁ kɯ-ndɯβ tsa ɲɯ-lɤt ŋu tɕe, wɣrum.}\hspace{5pt}\pcmn{\lien{ⓔqandzɤjo}{qandzɤjo}是一种矮小的树,叶子细小,长不高也长不粗,木质结实,是过去作筷子的材料。开细小的白花。}\end{exemple}\end{entrée}

\begin{entrée}{qandzɤjoɕku}{}{ⓔqandzɤjoɕku} 
\classe{n} 
\begin{définition}\pfra{poireau sauvage}\end{définition}
\begin{définition}\pcmn{高山上的野韭菜}\end{définition}
\begin{exemple}\pjya{qandzɤjo ɕku nɯ zgoku kɯ-mbro ʑo tu-ɬoʁ ŋu, sɤtɕha kɯ-mɯɕtaʁ tsa tu-ɬoʁ cha. ɯ-mdoʁ pɣi, si kɯ-ndɯβ ɯ-rchɤβ ra tu-ɬoʁ ŋu. ɯ-qa ri ɯ-zrɤm sɤɣ-ndzoʁ ɯ-stu kɯ-xtɕɯ-xtɕi jpum. ɯ-zrɤm nɯ kɯ-xtshɯm, kɯ-wɣrum ŋu. ɯ-ru tu-tɯ-ɬoʁ ɯ-stu nɯ arɤʑɯʑrɤz, ɯ-ŋgɯ tɕe ɯ-spjɯŋ tu-ɬoʁ tɕe, ɯ-taʁ tɕe li ɯ-jwaʁ ɲɯ-βze ŋu. ɯ-jwaʁ nɯ tɕɤr tɕe rɲɟi, ɯ-jwaʁ χsɯm jamar tɤ-ɬoʁ ɯ-qhu tɕe, tɕe ɯ-spjɯŋ ɯ-ŋgɯ ɯ-ku nɯ tɕu ɲɯ-rɯmɯntoʁ ŋu. pha ɯ-phoŋbu nɯ mɤrtsaβ, ɯ-dɯχɯn χɕɤβ. kɤ-ndza sna.}\hspace{5pt}\pcmn{\lien{ⓔqandzɤjoɕku}{qandzɤjoɕku}生长在高山上,能生长在比较寒冷的地方,是灰色的,一般生长在灌木丛里。长根的部位有点粗。根又细又白。茎长出来的部位有条纹,里面长主心干,在上面又长叶子。叶子细而长。叶子长到两三片时,在顶部开花。全身都是辣的,香味很浓。可以吃。}\end{exemple}\end{entrée}

\begin{entrée}{qandzi}{}{ⓔqandzi} 
\classe{n} 
\begin{définition}\pfra{type de sapin}\end{définition}
\begin{définition}\pcmn{杉树的一种}\end{définition}\end{entrée}

\begin{entrée}{qandʑɣi}{}{ⓔqandʑɣi} 
\classe{n} 
\begin{définition}\pfra{faucon (falco cherrug)}\end{définition}
\begin{définition}\pcmn{猎隼}\end{définition}
\begin{exemple}\pjya{qandʑɣi nɯ kɯ-pɣi ci ŋu, ɯ-ʁar ra zɯmi kɯ-wɣrum tɯ-snaʁ ka tu, ɕa rga, stoʁ ji ɯ-raŋ tɕe ju-ɣi ŋu, qartsɯ tɕe ju-nɯɕe ŋu. kɯrɯ ra kɯ tu-ti-nɯ stu kɯ-mɤku pɯ́-wɣ-mto tɕe, ɯ-mgɯr ɯ-qhu a-pɯ-mto tɕe, tɯ-xpa kɯ-fka tu-kɯ-ti ŋgrɤl, ɯ-xtu kɤ-mto a-pɯ́-wɣ-z-mɤku tɕe kɯ-mtsɯr tu-kɯ-ti ŋɯ-ŋgrɤl. tɯ-jaʁ tɤ-mthɯm tú-wɣ-ndo tɕe pjɯ-ɣi tɕe tɯ-jaʁ pjɯ-qraʁ tɕe tɤ-mthɯm ɲɯ-kɯ-nɯsɯkho ɲɯ-ŋgrɤl.}\hspace{5pt}\pcmn{猎隼是灰色的,翅膀上有白色斑点,爱吃肉,在种胡豆的季节出现,到了冬天就回去。藏民有一种说法,当它第一次出现时,如果先看见它的背部,这一年吃得饱,如果先看见它的腹部就会挨饿。如果你手上拿着肉,它会飞下来,抓破你的手,然后把肉抢走了。}\end{exemple}\end{entrée}

\begin{entrée}{qandʑi}{}{ⓔqandʑi} 
\classe{n} 
\begin{définition}\pfra{étain}\end{définition}
\begin{définition}\pcmn{锡}\end{définition}\end{entrée}

\begin{entrée}{qandʐe}{}{ⓔqandʐe} 
\classe{n} \sens{1}
\begin{définition}\pfra{ver de terre}\end{définition}
\begin{définition}\pcmn{蚯蚓}\end{définition}\sens{2}
\begin{définition}\pfra{nom d'une constellation}\end{définition}
\begin{définition}\pcmn{星宿的名字,东边升起来落到西边去}\end{définition}
\begin{exemple}\pjya{qandʐe ɯ-jme}\hspace{5pt}\pcmn{蚯蚓星宿}\end{exemple}\end{entrée}

\begin{entrée}{qandʐethɤlwɤɕtʂat}{}{ⓔqandʐethɤlwɤɕtʂat} 
\classe{n} 
\begin{définition}\pfra{riche mais économe}\end{définition}
\begin{définition}\pcmn{虽然很富有但是非常节约的人}\end{définition}\end{entrée}

\begin{entrée}{qandʐi}{₂}{ⓔqandʐiⓗ2} 
\classe{n} 
\begin{définition}\pfra{salmonidé}\end{définition}
\begin{définition}\pcmn{鲑【猫儿鱼】}\end{définition}
\begin{exemple}\pjya{qandʐi nɯ qaɟy ɯ-ngɯz stu kɯ-wxti ŋu, ɯ-ɕɣa tu, ɯ-mtɕhirme tu, kú-wɣ-sqa tɕe ɯ-ɕa ɲɯ-ndʐi mɤ-cha, ɯ-punaŋtɕa tɕhɯtɯɣ ɯ-smɤn ŋu, ʁnɯz ʁnɯz tɯtɯrca tu-ŋke ɲɯ-ŋgrɤl, qaɟy kɯ-fse kɯ-dɤn maŋe, zlawa χsɯmba raŋ tɕe tɯ-ci ɯ-ŋgɯ tɤton lu-ɣi ɲɯ-ŋgrɤl. nɯ lɤ-ɣe ɯ-raŋ mdaʁʑɯɣ pjɯ́-wɣ-sɤtsa tɕe, pjɯ́-wɣ-sat pjɤ-ŋgrɤl. kɯ-wxti kɯ tɯrme ɯ-fsu tu ɲɯ-ŋgrɤl, kɯ-xtɕi nɯ tɯ-tɯ-ɣa jamar ma kɯ-me tu ɲɯ-ŋgrɤl. nɯ kɯnɤ qaɟy kɯ-wxti chɯ-ndze ɲɯ-ŋu.}\hspace{5pt}\pcmn{鲑在鱼当中是最大的一种,有牙齿,有胡须。肉煮不化,其内脏是一种药材,如果牛因喝水而中毒,可以用此解毒。一般是一对一对地游动,没有像其它鱼那么多。到了三月,它们往河流的上游游去,正当这个时候,插下长矛把它杀死。大的和人一样大,小的只有一拃长。鲑可以吃其它比较大的鱼。}\end{exemple}\end{entrée}

\begin{entrée}{qandʐi}{₁}{ⓔqandʐiⓗ1} 
\classe{vi} \paradigme{dir}{kɤ-}\paradigme{dir}{tɤ-}\paradigme{dir}{tɤ-}
\begin{définition}\pfra{noirâtre, sombre, violacé}\end{définition}
\begin{définition}\pcmn{乌;紫}\end{définition}
\begin{définition}\pfra{faire un bleu}\end{définition}
\begin{définition}\pcmn{弄紫}\end{définition}
\begin{exemple}\pjya{tɯ-mɯ ko-qandʐi}\hspace{5pt}\pcmn{天很阴(要下雨了)}\end{exemple}
\begin{exemple}\pjya{tɤ́-wɣ-xtsɯɣ-a tɕe, to-sqandʐi}\hspace{5pt}\pcmn{他打中我了,就弄紫了}\end{exemple}
\begin{exemple}\pjya{nɤ-rŋa kɤ-tɯ-nɯ-rpu-t tɕe to-sqandʐi}\hspace{5pt}\pcmn{你撞到脸,就弄紫了}\end{exemple}
\begin{sous-entrée}{sqandʐi}{ⓔqandʐiⓗ1ⓝsqandʐi} 
\classe{vt}  
\grammaire{caus} \end{sous-entrée}

\end{entrée}

\begin{entrée}{qanɯ}{}{ⓔqanɯ} 
\classe{vs} \paradigme{dir}{kɤ-}
\begin{définition}\pfra{sombre}\end{définition}
\begin{définition}\pcmn{暗,黑(天色)}\end{définition}
\begin{exemple}\pjya{ʑa qanɯ ɲɯ-ŋu}\hspace{5pt}\pcmn{快要黑了}\end{exemple}
\begin{exemple}\pjya{ko-qanɯ}\hspace{5pt}\pcmn{天黑了}\end{exemple}\relationsémantique{参考}{\lien{ⓔsqanɯ}{sqanɯ}}\end{entrée}

\begin{entrée}{qaɲi}{}{ⓔqaɲi} 
\classe{n} 
\begin{définition}\pfra{taupe}\end{définition}
\begin{définition}\pcmn{鼹鼠【田鼠】}\end{définition}
\begin{exemple}\pjya{qaɲi nɯ ɯ-mdoʁ cho ɯ-rme nɯ ra βʑɯ kɯ-fse ŋu, ɯʑo ɯ-tshɯɣa nɯ ɯ-mɤlɤjaʁ ra ɲɯ-xtɯt, ɯ-ndzrɯ wuma ɲɯ-mtɕoʁ, ɯ-phoŋbu ɲɯ-tshu, ɯ-jme kɯ-xtɯ-xtɯt kɯ-xtshɯm ci ɣɤʑu, ɯ-ku nɯ βʑɯ kɯ-fse mɯ-ɲɯ-ɤmtɕoʁ, ɯ-rna nɯ ɯ-rme ɯ-ŋgɯ ku-kɯ-raʁ kɯ-fse ci ma maŋe, ɯ-mɲaʁ kɯ-xtɕɯ-xtɕi ɲɯ-ŋu, ɯ-ɕna paʁ ɯ-ɕna tsa ɲɯ-fse tɕe ɲɯ-rko ma sɤtɕha ɯ-pa ɯ-ɕna kɯ ju-sɯ-sloʁ ɲɯ-ra. ɯ-mɤlɤjaʁ kɯ ju-z-rɤβraʁ ɲɯ-ra tɕe ɯ-jroʁ ju-tɕɤt tɕe ɯʑo ju-ɕe pjɯ-tɕhɯt ɲɯ-ra. tɕe ɯ-jroʁ ja-tɕɤt tɕe, ɯʑo tshɯrɟɯn tu-ŋke kɯ-ra nɯ ra ɲɯ-jom. tɕe stonka tɕe sɯjno ɯ-qa, tɤtsoʁ ɯ-qa, tɤ-rɤku ɯ-mat nɯ ra ɲɯ-sɤjti tɕe ɯ-jroʁ ɯ-ŋgɯ nɯ ju-sɯ-mtshɤt, tɕe qartsɯ ɯ-ndza spa ɲɯ-ŋu. qartsɯ tɕe mɯ-tha-ɕkɯt nɯ, ftɕar tɕe tu-ɬoʁ tɕe ɲɯ-saχsɤl. kɯ-sɤmtshɤr nɯ tɤ-rɤku ɯ-mat pa-nɯ-phɯt tɕe, ɯ-ru maka mɤ-kɯ-ɴɢlɯt, mɤ-kɯ-ɤjʁu tu-βze ɲɯ-cha. kɯɕnom nɯ kɤ́-ʑmbɯ-ʑmbraʁ ju-tsɯm ɲɯ-ŋu, stoʁ staχpɯ nɯ ɯ-qiɯ nɯ ɯ-rdoʁ ju-tsɯm ɲɯ-ŋu, ɯ-qiɯ nɯ kɤ́-rqhɯ-rqhu ju-tsɯm ɲɯ-ŋu.}\hspace{5pt}\pcmn{鼹鼠毛色和老鼠一样,它的形状是:四肢短、爪子很锐利,身子很肥,有一条又细又短的尾巴,头不像老鼠的那么尖,耳朵陷在毛里,眼睛很小。鼻子有点像猪的鼻子但比较硬,因为它在地下用鼻子拱土。它用四肢挖土,要挖出一条容得下它身子的通道。通道挖了以后,它经常经过的地方宽一些。到了秋天,它就把草根、人参果的根和粮食储存起来,塞满它的通道,作冬天的食物。冬天没有吃完的,到春天会长出来就可以发现。奇怪的是它摘了庄稼的果实,能做到秆不折断也不弯,穗子连同芒一起带去。胡豆和豌豆有一半只拿颗粒,另一半则连壳一起拿去。}\end{exemple}\relationsémantique{参考}{\lien{ⓔqaɲɯɣɲɟɯ}{qaɲɯɣɲɟɯ}}\end{entrée}

\begin{entrée}{qaɲɯɣɲɟɯ}{}{ⓔqaɲɯɣɲɟɯ} 
\classe{n} 
\begin{définition}\pfra{taupière}\end{définition}
\begin{définition}\pcmn{鼹鼠洞}\end{définition}\relationsémantique{参考}{\lien{ⓔqaɲi}{qaɲi}}\relationsémantique{参考}{\lien{ⓔɯ-ɣɲɟɯ}{ɯ-ɣɲɟɯ}}\end{entrée}

\begin{entrée}{qapar}{}{ⓔqapar} 
\classe{n} 
\begin{définition}\pfra{cuon alpinus}\end{définition}
\begin{définition}\pcmn{豺}\end{définition}
\begin{exemple}\pjya{qapar nɯ khɯna kɯ-fse ŋu, kɯ-pɣi kɯ-ɤɣɯrnɯɕɯr ŋu, ɕnɤcat tɯtɯrca tu ɲɯ-ngrɤl, fsapaʁ tu-ndze tɕe, tu-βɟi ɯ-ʑɤrʑɯr tɤ-sŋɯt tu-lɤt tɕe fsapaʁ tu-ndze ɲɯ-ŋgrɤl, fsapaʁ pjɯ-si ɕɯŋgɯ tɕe ɯ-xtu ɯ-ŋgɯ chɯ-ɕe tɕe ɯ-punaŋtɕa chɯ-ɕkɯt ɲɯ-ŋgrɤl. ɯ-mɤlɤjaʁ khɯna ɯ-mɤlɤjaʁ kɯ-fse ɲɯ-ŋu, ɯ-jme khɯna jme staʁ jpum, zoŋzoŋ pa.}\hspace{5pt}\pcmn{豺狗像狗一样,灰里带红色,八九只一起行动,吃牲畜,一边追一边咬住就吃,在牲畜死之前,它钻进牲畜肚子里把内脏吃完。四肢和狗的一样,尾巴比狗的粗,毛茸茸的。}\end{exemple}\end{entrée}

\begin{entrée}{qapɤtɯm/\variante{qapɯtɯm}}{}{ⓔqapɤtɯm} 
\classe{n} 
\begin{définition}\pfra{silex arrondi}\end{définition}
\begin{définition}\pcmn{圆形的燧石}\end{définition}\relationsémantique{参考}{\lien{ⓔqapiⓗ1}{qapi₁}}\end{entrée}

\begin{entrée}{qapɣɤmtɯmtɯ}{}{ⓔqapɣɤmtɯmtɯ} 
\classe{n} 
\begin{définition}\pfra{houppe}\end{définition}
\begin{définition}\pcmn{戴胜}\end{définition}
\begin{exemple}\pjya{qapɣɤmtɯmtɯ nɯ tɯ-ci ɯ-rkɯ ntsɯ ku-rɤʑi ŋu, tɯ-ci ɯ-ŋgɯ qajɯ ra tu-ndze ŋu, ɯʑo kɯ-xtɕi tsa ci ŋu, ɯ-mtsioʁ mɤ-rɲɟi, ɯ-kɤχcɤl ri ɯ-rme tɯ-mtɕoʁ tu tɕe nɯ ɯ-mtɯ ŋu tu-kɯ-ti ɲɯ-ŋu tɕe, tɕe tɤ-mbri tɕe ɯ-mtɯ nɯ ɲɤ-χtɤr kɯ-fse, ɯ-phoŋbu ɯ-muj ra mpɕɤr nɤmbju, ɯ-jme khro mɤ-rɲɟi, ɯ-mi qarŋe tsa ŋu.}\hspace{5pt}\pcmn{戴胜生活在大河边,吃河里的虫子,它身子小,嘴不长,在头顶有一撮毛(羽冠),人家说是它的髻。它每叫一声,就会把羽冠散开一下。身上的羽毛很美丽,有光泽,尾巴不长,脚是淡黄色的。}\end{exemple}\end{entrée}

\begin{entrée}{qapɣo}{}{ⓔqapɣo} 
\classe{n}  
\grammaire{n.lieu} 
\begin{définition}\pfra{un village de Sarndzu}\end{définition}
\begin{définition}\pcmn{沙尔宗的一个村}\end{définition}\end{entrée}

\begin{entrée}{qapi}{₁}{ⓔqapiⓗ1} 
\classe{n} 
\begin{définition}\pfra{silex}\end{définition}
\begin{définition}\pcmn{燧石}\end{définition}
\begin{exemple}\pjya{qapi nɯ rdɤstaʁ kɯ-wɣrum ŋu, kha znde ɣɯ ɯ-kɤχcɤl kɯ-ɤmtɕoʁ zɯ ɲɯ́-wɣ-ta ŋgrɤl tɕe βzɯrtɕoʁ ɲɯ-rmi. tɯ-ji kɯ-wxti ɣɯ ɯ-χcɤl ɲɯ́-wɣ-ta ɲɯ-ŋgrɤl tɕe, nɯ tɕu tɕe ʑaŋɬa ɲɯ-rmi. tɕe qapi nɯ ɯ-taʁ tɕaʁmɤr pjɯ́-wɣ-lɤt tɕe smɯtɕɣom tu-ɬoʁ ŋgrɤl tɕe kɯɕɯŋgɯ βʁɯz smi ɯ-sɤ-sɯ-mɟa nɯ qapi cho tɕaʁmɤr ni kɯ tú-wɣ-sɯ-βzu-nɯ pjɤ-ŋu. smi kɯ-me tɕe tɯrme tɯ-ndzɤtshi kɤ-βzu mɤ-khɯ tɕe kɯɕɯŋgɯ qapi nɯ wuma ʑo kɯ-ʁzi pjɤ-ɕti.}\hspace{5pt}\pcmn{\lien{ⓔqapiⓗ1}{qapi}是一种白石头,这种石头放在屋顶的四角上,这时候叫\lien{ⓔβzɯrtɕoʁ}{βzɯrtɕoʁ}。有人会把它放在最大的田地中间,这时候叫\lien{ⓔʑaŋɬa}{ʑaŋɬa}。在白石头上打火镰就会迸出火星。过去,引燃火绒的就是白石头和火镰。没有火,人们就不能做饭,所以过去白石头是非常重要的一个东西。}\end{exemple}\relationsémantique{参考}{\lien{ⓔqapɤtɯm}{qapɤtɯm}}\end{entrée}

\begin{entrée}{qapi}{₂}{ⓔqapiⓗ2} 
\classe{n} 
\begin{définition}\pfra{une espèce de cerisier}\end{définition}
\begin{définition}\pcmn{野樱桃的一种}\end{définition}
\begin{exemple}\pjya{qapi nɯ si kɯ-mbɯ-mbro ci ŋu, ɯ-ru wuma ɲɯ-jpum mɤ-cha, ɯ-ru ɯ-rtaʁ nɯ ra ɯ-mdzu kɯ-χɕu kɯ-jpum ʑo tu, ɯ-jwaʁ mɤ-jndʐɤz, kɯ-ɤrtɯm tsa ŋu. ɯ-mɯntoʁ kɯ-wɣrum ɲɯ-lɤt ŋu, ɯ-mat tha-βzu tɕe, kɯβde kɯmŋu jamar tɯtɯrca ku-ndzoʁ, ʑakastaka nɯ-jɯ kɯ-xtshɯm tɯ-ka tu. qapi zgo kɯ-mbro tsa kɯ-mbɤr tsa aʁɤndɯndɤt tu-ɬoʁ cha.}\hspace{5pt}\pcmn{野樱桃长得很高,但树干不是很粗,树干和树枝上都长有又尖又粗的刺,叶子不大,呈圆形。开白花。结果的时候,四五个结在一起,每一个都有细小的茎。山上山下都可以生长。}\end{exemple}\end{entrée}

\begin{entrée}{qaprɤftsa}{}{ⓔqaprɤftsa} 
\classe{n} 
\begin{définition}\pfra{mille patte}\end{définition}
\begin{définition}\pcmn{蜈蚣}\end{définition}\end{entrée}

\begin{entrée}{qaprɤkhɯsloŋ}{}{ⓔqaprɤkhɯsloŋ} 
\classe{n} 
\begin{définition}\pfra{Arisaema consanguineum}\end{définition}
\begin{définition}\pcmn{天南星}\end{définition}
\begin{exemple}\pjya{qaprɤkhɯsloŋ nɯ sɯjno ci ŋu, kɯ-mɤku ɯ-ru kɯ-jpum ʑo tu-ɬoʁ tɕe, ɯ-taʁ tsa ri tɕe, ɯ-jwaʁ ɲɯ-ɬoʁ, ɯ-jwaʁ ɯ-sɤɣɬoʁ tɯ-ldʑa ma me, ɯ-mat nɯ jima ɯ-mat cho naχtɕɯɣ, thɯ-tɯt tɕe chɯ-ɣɯrni ŋu. ɯ-qa nɯ kɯ-ɤrtɯm ɲɯ-βze ŋu, sɤndɤɣ.}\hspace{5pt}\pcmn{天南星是一种植物。首先,长出一根粗壮的茎,长了一段后,才开始长出叶子,长叶的枝只有一根,果实像玉米果实一样,成熟了就变红。根是圆形的。有毒性。}\end{exemple}\end{entrée}

\begin{entrée}{qaprɤsi}{}{ⓔqaprɤsi} 
\classe{n} 
\begin{définition}\pfra{une espèce de chêne}\end{définition}
\begin{définition}\pcmn{槲栎的一种}\end{définition}\end{entrée}

\begin{entrée}{qapri}{}{ⓔqapri} 
\classe{n} 
\begin{définition}\pfra{serpent}\end{définition}
\begin{définition}\pcmn{蛇}\end{définition}
\begin{exemple}\pjya{qapri ɯ-χsjɯβ chɤ-βde}\hspace{5pt}\pcmn{蛇脱皮了}\end{exemple}\relationsémantique{参考}{\lien{ⓔtɕhɯχpri}{tɕhɯχpri}}\étymologie{sbrul}\end{entrée}

\begin{entrée}{qapribɯxsi}{}{ⓔqapribɯxsi} 
\classe{n} 
\begin{définition}\pfra{serpent géant}\end{définition}
\begin{définition}\pcmn{巨蛇}\end{définition}\end{entrée}

\begin{entrée}{qaprilu}{}{ⓔqaprilu} 
\classe{n} 
\begin{définition}\pfra{année du serpent}\end{définition}
\begin{définition}\pcmn{蛇年}\end{définition}\end{entrée}

\begin{entrée}{qaprimdʑu}{}{ⓔqaprimdʑu} 
\classe{n} 
\begin{définition}\pfra{une composée}\end{définition}
\begin{définition}\pcmn{【剪刀菜】}\end{définition}
\begin{exemple}\pjya{qaprimdʑu nɯ sɯjno ɯ-zrɤm ra kɯ-xtɕi ci ŋu, ɯ-ru cho ɯ-jwaʁ nɯ ra kɯ-ɤrŋi ɯ-ŋgɯz kɯ-pɣi ŋu, ɯ-jwaʁ nɯ qapri mdʑu ɯ-tshɯɣa ɲɯ-fse, ɯ-mɯntoʁ nɤmkha mdoʁ ŋu, pjɯ́-wɣ-qlɯt tɕe ɯ-lu tu. fsapaʁ ra kɤ-ndza rga-nɯ, tɯrme kɤ-ndza mɤ-sna.}\hspace{5pt}\pcmn{剪刀菜是根部不发达的植物,茎和叶子绿里带有灰色,叶子的形状像蛇的舌头,花是天蓝色的。折断的时候有乳汁。牲畜喜欢吃,人不能吃。}\end{exemple}\end{entrée}

\begin{entrée}{qapɯ}{}{ⓔqapɯ} 
\classe{vi} \paradigme{dir}{nɯ-}\paradigme{dir}{nɯ-}
\begin{définition}\pfra{tomber en friche}\end{définition}
\begin{définition}\pcmn{变成荒地}\end{définition}
\begin{définition}\pfra{laisser en friche}\end{définition}
\begin{définition}\pcmn{让……变成荒地}\end{définition}
\begin{exemple}\pjya{tɯ-ji ra ɲɤ-qapɯ-nɯ}\hspace{5pt}\pcmn{田地变成荒地}\end{exemple}
\begin{sous-entrée}{sqapɯ}{ⓔqapɯⓝsqapɯ} 
\classe{vt} \end{sous-entrée}

\end{entrée}

\begin{entrée}{qar}{}{ⓔqar} 
\classe{vs} 
\begin{définition}\pfra{tabou}\end{définition}
\begin{définition}\pcmn{禁忌的地方}\end{définition}
\begin{exemple}\pjya{ki sɤtɕha ɲɯ-qar}\hspace{5pt}\pcmn{这个地方是禁忌}\end{exemple}
\begin{exemple}\pjya{si kɤ-phɯt mɤ-βdi, sɤtɕha kɤ-lɣa mɤ-βdi. tɯ-ŋgo ɲɯ-sɤβze ɲɯ-ŋgrɤl, tu-kɯ-ɕɯngo ɲɯ-ŋu.}\hspace{5pt}\pcmn{(在那种地方)砍树是不好的,挖地是不好的。这样会令人得病}\end{exemple}\end{entrée}

\begin{entrée}{qarɤt}{}{ⓔqarɤt} 
\classe{n} 
\begin{définition}\pfra{râteau}\end{définition}
\begin{définition}\pcmn{耙子}\end{définition}\end{entrée}

\begin{entrée}{qarcɯm}{}{ⓔqarcɯm} 
\classe{vs} \paradigme{dir}{thɯ-}
\begin{définition}\pfra{s'assombrir (ciel)}\end{définition}
\begin{définition}\pcmn{天阴}\end{définition}
\begin{exemple}\pjya{tɯ-mɯ chɤ-qarcɯm}\hspace{5pt}\pcmn{天阴了}\end{exemple}\relationsémantique{同义词}{\lien{ⓔqanɯ}{qanɯ}}\relationsémantique{参考}{\lien{ⓔsqarcɯm}{sqarcɯm}}\end{entrée}

\begin{entrée}{qarɣɤpɤt}{}{ⓔqarɣɤpɤt} 
\classe{n} 
\begin{définition}\pfra{une plante}\end{définition}
\begin{définition}\pcmn{【鹿茸花】}\end{définition}
\begin{exemple}\pjya{qarɣɤpɤt nɯ, zgo wuma ʑo kɯ-mbro, sɤtɕha kɯ-ɣɤndʐo ɯ-pɕoʁ tu-ɬoʁ ŋu. ɯ-zrɤm mɯ́j-wxti, ɯ-ru ɯ-ŋgɯ qhoʁsjɯβ ɲɯ-ŋu, ɯ-ru ɯ-taʁ ɯ-jwaʁ ɲɯ-ɬoʁ tɕe, nɯ ɯ-rchɤβ ri ɯ-mɯntoʁ tu-ɬoʁ ɲɯ-ŋu. ɯ-ru cho ɯ-jwaʁ nɯ ra ɯ-rme rsɯβrsɯβ ʑo ɲɯ-pa, ɯ-jwaʁ cho ɯ-ru kɯ-ɤɣrɤɣrum ɲɯ-ŋu, ɯ-rme nɯ kɯ-pɣi ɲɯ-ŋu, ɯ-mɯntoʁ tɯ-rdoʁ tɯ-rdoʁ ɲɯ-ŋu, ɯ-mɯntoʁ ɯ-rqhu nɯ li ɯ-rme rsɯβrsɯβ ɲɯ-pa, ɯ-rqhu nɯ pɯ-ɴɢaʁ tɕe, ɯ-mɯntoʁ wuma ʑo kɯ-qarŋe ɲɤ-ɬoʁ ɲɯ-ŋu, ɯ-mɯntoʁ ɲɯ-wxti, kɯ-ɤrtɯ-rtɯm ɲɯ-ŋu, wuma ʑo ɲɯ-mpɕɤr.}\hspace{5pt}\pcmn{鹿茸花生长在高山上,比较寒冷的地方里。根不大,茎是空心的,茎上长叶子,叶子和茎的中间长花。茎和叶子看起来毛茸茸的,茎和叶子带有一点白色,毛是灰色的,叶子是一朵一朵地长出来,花的外层也是毛茸茸的,外层剖开时,就开黄色的花,花很大,圆形,非常美。}\end{exemple}\end{entrée}

\begin{entrée}{qarɣe}{}{ⓔqarɣe} 
\classe{n} 
\begin{définition}\pfra{espèce d'arbrisseau}\end{définition}
\begin{définition}\pcmn{灌木的一种}\end{définition}
\begin{exemple}\pjya{qarɣe nɯ si ci ŋu, khro mɤ-mbro, ɯ-ru nɯ kɯ-ɣɯrni ɯ-ŋgɯ ri kɯ-qarŋe tsa ŋu, ɯ-ru nɯ ndoʁ, χɕitka tɕe kɯmaʁ si ra ɕɯŋgɯ ɯ-jwaʁ ɲɯ-lɤt ɕɯŋgɯ ɲɯ-rɯmɯntoʁ ŋu. ɯ-ru cho ɯ-mɯntoʁ nɯ ɯ-dɯχɯn mnɤm, ndzɤtshi ɯ-cu kɤ-nɯlɤt sna}\hspace{5pt}\pcmn{\lien{ⓔqarɣe}{qarɣe}是一种树,长的不高,树干红里带黄,木质很脆。到了春天,开花时间比其他树早,甚至在自己出叶之前开花,树干和花都有香味,可以用来作香料。}\end{exemple}\end{entrée}

\begin{entrée}{qarɣɯ}{}{ⓔqarɣɯ} 
\classe{n} 
\begin{définition}\pfra{rosacée}\end{définition}
\begin{définition}\pcmn{蔷薇科的一种}\end{définition}
\begin{exemple}\pjya{qarɣɯ nɯ si ci ŋu wuma sthɯci mɤ-mbro, ɯ-mdzu wuma ʑo jaʁ mtɕoʁ, ɯ-mdzu ɯ-kɤχcɤl ŋgɤɣ ʑo ŋu, ɯ-jwaʁ kɯ-ndɯβ tsa kɯ-ɤrtɯm ŋu, ɯ-jwaʁ βzɯr nɯ ra ɯ-mdzu kɯ-fse tu, ɯ-mɯntoʁ kɯ-ɣɯrni tu kɯ-wɣrum tu, ɯ-mat nɯ pɣa ra kɯ tu-ndza-nɯ ŋu ma nɯ ma ɯ-kɯ-ndza me}\hspace{5pt}\pcmn{\lien{ⓔqarɣɯ}{qarɣɯ}是一种树,长得不怎么高,刺长得又厚又尖锐,刺的顶端还有钩,叶子小而圆,边缘还有小刺,花有的是红色的,有的是白色的。果实只有鸟吃,其他动物都不吃。}\end{exemple}\end{entrée}

\begin{entrée}{qarma}{}{ⓔqarma} 
\classe{n} 
\begin{définition}\pfra{crossoptilon}\end{définition}
\begin{définition}\pcmn{马鸡}\end{définition}
\begin{exemple}\pjya{qarma nɯ ʁnɯ-tɯphu tu, kɯ-ɲaʁ ci tu, qarma ɲaʁ rmi, tɕe kɯ-wɣrum ci tu, qarmaɣrum rmi. qarma nɯ ɯ-ku ɲaʁ, ɯ-mɲaʁ ɯ-rkɯ nɯ ɣɯrni, ɯ-rna nɯ ɯ-rme ɯ-ŋgɯ ku-raʁ ʑo ŋu, kɯ-ɤrqhi ju-kɯ-ru mɤ-saχsɤl, ɯ-jme rɲɟi, qarmaɲaʁ nɯ ɯ-ʁar ɯ-ku ɲaʁ, ɯ-jme ɯ-kɯ-ɲaʁ dɤn, ɯ-mi qarŋe. qarmaɣrum nɯ kɯmaʁ ra naχtɕɯɣ, ɯ-ʁar lonba wɣrum, ɯ-jme ɯ-kɯ-ɲaʁ rkɯn. ɯ-mi li kɯ-qarne ŋu. sɯŋgɯ wuma ʑo ku-rɤʑi-ndʑi ɕti. nɯ-ɕpaʁ tɕe, qarma kɯβdɤsqi kɯmŋɤsqi jamar tɯtɯrca co zɯ kɯ-nɯci pjɯ-ɣi-nɯ ŋgrɤl. qarma nɯ pɣa kɯ-wxti ci ɕti tɕe, tɯ-rdoʁ kɯ sqamŋu tɯ-rpa jamar kɯ-zɣɯt tu, ɯ-ŋgɯm kɯnɤ kɯmpɣa ŋgɯm sɤz wxti, qajɯ cho sɯmat ra tu-ndze ɲɯ-ŋu, tɤ-rɤku ra li ɣɯ-tu-ndze ngrɤl. tu-mbri tɕe, ɯ-mtɕhi tu-mbri ɯ-rca ɯ-mphɯz kɯnɤ tu-mbri ŋu, xɕiri kɯ tu-ti tɕe, `a-pi qarma ɯ-ɕa mɯm ri, tɯ-sɤtɕha ɲɯ-kɯ-sɯ-spɤr ŋu' tu-ti ɲɯ-ŋgrɤl ma qarma lɤ-zo tɕe xɕiri kɯ ɯ-mke ku-ndɤm tɕe mɯ-ɲɯ-te tɕe, qarma ɲɯ-nɯqambɯmbjom tɕe phɤri ʑo ku-nɯɕe tɕe xɕiri tɤrcɯrca ku-tsɯm ɲɯ-ŋgrɤl, xɕiri kɯ nɯ kɯnɤ mɯ-ɲɯ-te tɕe núndʐa nɯ tu-ti ɲɯ-ŋu.}\hspace{5pt}\pcmn{马鸡有两种,一种是黑色的,叫黑马鸡(蓝马鸡),另一种是白色的,叫白马鸡。马鸡的头是黑色的,眼睛周围有一圈是红色的,耳朵陷在羽毛里,从远处看不清楚,尾巴长。黑马鸡的翅膀顶端是黑色的,尾巴非常黑,脚是黄色的。白马鸡其它部位和黑马鸡一样,只是翅膀全白,尾巴黑羽毛少一些,脚也是黄色的。两种都生活在森林深处。口渴的时候,就四五十只一起下到山沟边来喝水。马鸡是一种比较大的鸟,一只都有十五斤左右,蛋也比鸡蛋大。它喜欢吃虫子和野果,也会来偷吃粮食。叫起来的时候,嘴和尾巴都发出声音。据说黄鼠狼说:“虽然马鸡哥哥的肉好吃,但是它会使你离开自己的家乡”。因为马鸡着地时,黄鼠狼把它的脖子抓住不放,马鸡会飞到对面的山上,也会把黄鼠狼带到那边去,黄鼠狼也还是不放,所以就有这样的说法。}\end{exemple}\end{entrée}

\begin{entrée}{qarmaɣrum}{}{ⓔqarmaɣrum} 
\classe{n} 
\begin{définition}\pfra{faisan (crossoptilon crossoptilon)}\end{définition}
\begin{définition}\pcmn{白马鸡}\end{définition}\end{entrée}

\begin{entrée}{qarmami}{}{ⓔqarmami} 
\classe{n} 
\begin{définition}\pfra{plumes rouges de crossoptilon}\end{définition}
\begin{définition}\pcmn{马鸡的红羽毛}\end{définition}\end{entrée}

\begin{entrée}{qarmamtshalu}{}{ⓔqarmamtshalu} 
\classe{n} 
\begin{définition}\pfra{une plante}\end{définition}
\begin{définition}\pcmn{植物的一种}\end{définition}
\begin{exemple}\pjya{qarma mtshalu nɯ sɯjno ci ŋu, mtshalu cho naχtɕɯɣ, tɕeri ɯ-mdoʁ nɯ afsoʁŋgi, ɯ-rme kɯ-xtɕɯ-xtɕi tu ri mɤ-sɤmtsɯɣ, kɤ-ndza sna. qarma mtshalu kɤntɕhɯ-tɯphu tu tɕe, tɯtɯphu nɯ ɯ-mɯntoʁ kɯ-wɣrum ɲɯ-lɤt ŋu, ɯ-mat tha-βzu tɕe, kɯ-rɲɟi tsa chɯ-βze tɕe, thɯ-fka tsaʁ tɕe, kú-wɣ-ndo tɕe ɲɤ-mboʁ ŋu. ɲɤ-mboʁ tɕe ɯ-rdoʁ pjɯ-nɯɬoʁ, ɯ-fkɯm rʁoʁ ʑo tu-ti tɕe lo-rʁɯrʁu ŋu. tɯrme nɯɕɯmɯma ʑo tu-kɯ-sɤmbrɯ nɤ qarma mtshalu tu-sɤrmi-nɯ ŋgrɤl}\hspace{5pt}\pcmn{\lien{}{qarma mtshalu}是一种植物,和荨麻一样,但颜色浅一些,虽然有一些毛,但不蜇人,有几种\lien{}{qarma mtshalu}可以吃。其中一个开白花,结果实的时候,果实有点长,结得饱满时,用手一摸就会突然破掉,滑漏出种子,种子的壳就会卷起来。突然生气的人,人们说他是\lien{}{qarma mtshalu}。}\end{exemple}\end{entrée}

\begin{entrée}{qarmaɲaʁ}{}{ⓔqarmaɲaʁ} 
\classe{n} 
\begin{définition}\pfra{faisan (crossoptilon auritum)}\end{définition}
\begin{définition}\pcmn{蓝马鸡}\end{définition}\end{entrée}

\begin{entrée}{qarndɯm}{}{ⓔqarndɯm} 
\classe{vi} \paradigme{dir}{nɯ-}
\begin{définition}\pfra{trouble (eau)}\end{définition}
\begin{définition}\pcmn{混浊}\end{définition}
\begin{exemple}\pjya{tɯ-ci ɲɯ-qarndɯm}\hspace{5pt}\pcmn{水是浑浊的}\end{exemple}
\begin{sous-entrée}{sqarndɯm}{ⓔqarndɯmⓝsqarndɯm} 
\classe{vt} 
\begin{définition}\pfra{troubler (eau)}\end{définition}
\begin{définition}\pcmn{弄浊}\end{définition}\relationsémantique{反义词}{\lien{ⓔamgri}{amgri}}\end{sous-entrée}

\end{entrée}

\begin{entrée}{qarŋe}{}{ⓔqarŋe} 
\classe{vi} \paradigme{dir}{thɯ-}\paradigme{dir}{nɯ-}
\begin{définition}\pfra{jaune}\end{définition}
\begin{définition}\pcmn{黄}\end{définition}
\begin{exemple}\pjya{βlama ɯ-ŋga nɯ kɯ-qarŋe ɲɯ-ŋu}\hspace{5pt}\pcmn{喇嘛的衣服是黄色的}\end{exemple}
\begin{exemple}\pjya{koxtɕɯn kɯ-qarŋe}\hspace{5pt}\pcmn{黄色的丝缎}\end{exemple}\end{entrée}

\begin{entrée}{qarŋɯrŋe}{}{ⓔqarŋɯrŋe} 
\classe{vs} 
\begin{définition}\pfra{jaune foncé}\end{définition}
\begin{définition}\pcmn{深黄色}\end{définition}\relationsémantique{参考}{\lien{ⓔqarŋe}{qarŋe}}\relationsémantique{反义词}{\lien{ⓔaqarŋɯrŋe}{aqarŋɯrŋe}}\end{entrée}

\begin{entrée}{qartsɤβ}{}{ⓔqartsɤβ} 
\classe{n} 
\begin{définition}\pfra{récolte}\end{définition}
\begin{définition}\pcmn{秋收}\end{définition}\end{entrée}

\begin{entrée}{qartshaʁrɯ}{}{ⓔqartshaʁrɯ} 
\classe{n} 
\begin{définition}\pfra{ramure de cerf}\end{définition}
\begin{définition}\pcmn{鹿角}\end{définition}\end{entrée}

\begin{entrée}{qartshaz}{}{ⓔqartshaz} 
\classe{n} 
\begin{définition}\pfra{cerf}\end{définition}
\begin{définition}\pcmn{鹿}\end{définition}
\begin{exemple}\pjya{qartshaz nɯ ruŋgu ku-rɤʑi ŋu, sŋo ɯ-ŋgɯ zɯ ku-rɤʑi, mbro kɯ-fse kɯ-wxti ŋu, ɯ-rme nɯ kɯ-ɤɣɯrnɯɕɯr kɯ-qarŋe kɯ-fse ŋu, kɯ-wxti ɕasca nɯ ɯ-ʁrɯ sqarcɤt kɯ-tu tu ɲɯ-ŋgrɤl, mɤʑɯ ɯ-ʁrɯ tɯ-ldʑa ma kɯ-me tu ɲɯ-ŋgrɤl, nɤrwɯ ŋu tu-kɯ-ti ɲɯ-ŋgrɤl, nɯnɯ qartshaz ɕawu rambɯm ŋu tu-kɯ-ti ɲɯ-ŋgrɤl, ma kɤ-mto rkɯn. kɯ-xtɕi kɯ tʂɯɣlaʁ nɯ dɤn, zgo zɯ ʁɟa ku-rɤʑi, kɯ-rɯndzɤtshi cho tɯ-rmɯkha rndzɤkɤŋe kɤ-ɣe tɕe ku-ɬoʁ ɲɯ-ŋgrɤl, wuma ʑo ku-rɯndzaŋspa, sɯjno tɯ-tɯkɯr ɲɯ-nɯ-phɯt tɕe, ci ɲɯ-nɤrɯra ku-ŋgrɤl, ɯ-qataʁrɯ nɯ nɯŋa ɯ-qataʁrɯ fse. ɯ-mɤlɤjaʁ nɯ mbro ɯ-mɤlɤjaʁ fse, ɯ-jme kɯ-xtɕɯ-xtɕi ma me, ɯ-rna nɯ nɯŋa ɯ-rna fse, ɯ-mtɕhi nɯ nɯŋa ɯ-mtɕhi tsa fse. ɕɤŋɯ raŋ tɕe qartshaz phu nɯ zgo kɯ-mbro zɯ tu-ɕe tɕe nɯ tɕu chɯ-ɣɤwu tɕe, kɯ-kɯ-ɤmɯmtshɤm ɣɯ qartshaz mu nɯ ju-ɣi-nɯ ra ɲɯ-ŋgrɤl, tɕe nɯ thɯ-awɤwum-nɯ tɕe qartshaz ɕaphu nɯ kɯ pjɯ-fskɤr tɕe nɯ tɕu ku-je tɕe tɯ-ci tɯ-mɯm mɯ-pjɤ-sɯ-tshi ɲɯ-ŋgrɤl, nɯ tɕu tɕe ɯ-sŋi kɤrtsi ʑo ku-z-rɤʑi, tɕe qartshaz mu nɯ ra ɲɯ-ɕpaʁ-nɯ, tɕe tɯ-ci tshi ɯ-raŋ nɯ ku-naχthɤβ tɕe tu-nɯphu ŋu tu-kɯ-ti ɲɯ-ŋgrɤl. ɯ-ʁrɯ nɯ smɤn ŋu, tu-ɕaŋ ɕɯŋgɯ tɕe, ɯ-rme sɯβ-sɯβ tu raŋ tɕe mpɯ, tɤ-ɕaŋ tɕe kɯ-rkɯ-rko ŋu, staχpɯ kɤ-phɯt ɯ-raŋ tɕe ɯ-ʁrɯ tu-ɕaŋ ŋu, tɕe lu-nɯ-βde ŋu tu-kɯ-ti ɲɯ-ŋgrɤl, ɯ-ʁrɯ na-nɯ-βde tɕe, tɯ-rdoʁ ɲɯ-nɯ-βde ma tɯ-tɕha tɯtɯrca ɲɯ-nɯ-βde mɯ́j-ŋgrɤl. ɯ-sni ɯ-se nɯ smɤn stu kɯ-ʑru ŋu tu-kɯ-ti ɲɯ-ŋgrɤl.}\hspace{5pt}\pcmn{鹿生活在高山上,在高山羊角花的树林里,有马那么大,毛是红里带黄的。大公鹿有的长十八支叉形角,还有的只有一只角的。据说(十八支角)是宝贝,叫作\lien{}{ɕawu rambɯm},很罕见。一般小鹿有十二支角,这种比较多。鹿一直生活在高山上,到了黄昏,太阳落山(阴影出现)的时候,它就出来觅食和进行其它活动。它非常谨慎,每吃一口草就要左右张望一下。蹄子像牛蹄子一样,四肢像马的一样,只有很小的尾巴,耳朵和嘴巴像牛的一样。发情的时候,公鹿到山的最高处,在那边叫喊,所能听到叫声的母鹿就会往它的方向来。在它们聚集以后,公鹿就会在它们周围绕一圈,不让它们离开,而且一口水都不让它们喝,让它们在那里待数天。母鹿很渴,喝水时公鹿就乘机交配。鹿角是药材,老了之前,毛茸茸的时候,是软的,老了就变硬。据说在割豌豆时鹿角就会老掉,鹿角掉落时,每次只掉一只,不会同时掉一对。据说鹿心脏的血是一种很有疗效的药材。}\end{exemple}\end{entrée}

\begin{entrée}{qartshɤɕku}{}{ⓔqartshɤɕku} 
\classe{n} 
\begin{définition}\pfra{oignon}\end{définition}
\begin{définition}\pcmn{葱的一种}\end{définition}
\begin{exemple}\pjya{qartshaɕku nɯ ruŋgu kɯ-mbro ɣɯ tɕhɯtoʁ ku tsa tu-ɬoʁ ɲɯ-ŋu, ɯ-tshɯɣa nɯ ɕkɤtshoŋ cho ɲɯ-naχtɕɯɣ, ɯ-ru maŋe, ɯ-jwaʁ nɯ kɯ-ɤlɯlju tɕe qhoʁsjɯβ ɲɯ-ŋu, ɯ-tho ɣɤʑu, ɯ-qa li ɯ-rqhu kɤntɕhɯ-tɤlɤβ ɣɤʑu. ɯ-dɯχɯn wuma ɲɯ-mɯm.}\hspace{5pt}\pcmn{\lien{ⓔqartshɤɕku}{qartshɤɕku}生长在高山的水草地里,形状和葱子(即通常说的“野葱”)一样,没有茎,叶子是空心的、圆柱形的。有花梗。根也有很多层皮。味道很香。}\end{exemple}\relationsémantique{参考}{\lien{ⓔqartshaz}{qartshaz}}\end{entrée}

\begin{entrée}{qartshɤndʐi}{}{ⓔqartshɤndʐi} 
\classe{n} 
\begin{définition}\pfra{peau de cerf}\end{définition}
\begin{définition}\pcmn{鹿皮}\end{définition}\relationsémantique{参考}{\lien{ⓔqartshaz}{qartshaz}}\relationsémantique{参考}{\lien{ⓔtɯ-ndʐi}{tɯ-ndʐi}}\end{entrée}

\begin{entrée}{qartshi}{}{ⓔqartshi} 
\classe{n} 
\begin{définition}\pfra{cigale; criquet}\end{définition}
\begin{définition}\pcmn{蝉;蟋蟀}\end{définition}
\begin{exemple}\pjya{qartshi nɯ sɯjno ŋgɯ, tɤ-rɤku ɯ-ŋgɯ ku-rɤʑi ŋu, tɤ-rɤku tu-ndze ŋgrɤl, qartshi nɯ ɯ-ʁrɯ tu, ɯ-mɲaʁ kɯ-wxtɯ-wxti ŋu, χɕɤlmɯɣ tɤ-kɯ-nɯ-ta ʑo fse, ɯ-ku xtɕi tsa ɯ-phoŋbu wxti, ɯ-mi kɯ-wxti ʑo tu, ɯ-mɤpɤl ɲɯ-zɲɟa, ɯ-ʁar mba, ɯ-rʑɯɣ tu, tɤ-ŋke tɕe kɯ-ɤrqhi ʑo ju-mtsaʁ cha, ɯ-ʁar kɯ-dɤn kɤ-ntɕhoz mɤ-ra.}\hspace{5pt}\pcmn{蟋蟀生活在草丛和庄稼里,吃庄稼。蟋蟀有触角,眼睛很大,好像戴了眼镜一样。头小,身子大。有很大的脚,爪子抓东西很稳,翅膀很薄,有褶皱。走动时,可以跳到较远的距离,翅膀起不了很大的作用。}\end{exemple}\end{entrée}

\begin{entrée}{qartsɯ}{}{ⓔqartsɯ} 
\classe{n} 
\begin{définition}\pfra{hiver}\end{définition}
\begin{définition}\pcmn{冬天}\end{définition}
\begin{exemple}\pjya{qartsɯ kɤ-ndzoʁ}\hspace{5pt}\pcmn{冬天开始了}\end{exemple}\end{entrée}

\begin{entrée}{qartsɯmɤftɕar}{}{ⓔqartsɯmɤftɕar} 
\classe{adv} 
\begin{définition}\pfra{toute l'année}\end{définition}
\begin{définition}\pcmn{一年四季}\end{définition}\relationsémantique{参考}{\lien{ⓔqartsɯ}{qartsɯ}}\relationsémantique{参考}{\lien{ⓔftɕar}{ftɕar}}\end{entrée}

\begin{entrée}{qartsɯɲaʁ}{}{ⓔqartsɯɲaʁ} 
\classe{n} 
\begin{définition}\pfra{hiver très froid}\end{définition}
\begin{définition}\pcmn{寒冬腊月}\end{définition}\relationsémantique{参考}{\lien{ⓔɲaʁ}{ɲaʁ}}\end{entrée}

\begin{entrée}{qartsɯχtu}{}{ⓔqartsɯχtu} 
\classe{n} 
\begin{définition}\pfra{cérémonie bouddhique de l'hiver}\end{définition}
\begin{définition}\pcmn{在冬天举行的一种佛教活动}\end{définition}\end{entrée}

\begin{entrée}{qarwoʁ}{}{ⓔqarwoʁ} 
\classe{n} 
\begin{définition}\pfra{récipient en terre}\end{définition}
\begin{définition}\pcmn{沙锅}\end{définition}
\begin{exemple}\pjya{qarwoʁ nɯ tɤ-rcoʁ kɯ tɤ-kɤ-sɯ-βzu ŋu, kɯɕɯŋgɯ tɕe rgargɯn ra ɣɯ nɯ-tʂha ɯ-sɤ-ta pjɤ-ŋu, ɯ-mŋu ri ɯ-mtɕhi tu tɕe tʂha pɯ́-wɣ-rku tɕe mɤ-jit. qarwoʁ ɯ-ŋgɯ tʂha ŋotɕu tɤ-ala tɕe thɯ́-wɣ-nɯ-mɟa kɯnɤ pjɯ-mbuz pjɤ-ŋu, ʑaʑa ʑo ɯ-tɯ-ɤla mɯ-pjɯ-ʑi tɕe tɯrme ɯ-ro kɯ-ŋgɤr ɯ-mbrɯ ʑa mɤ-kɯ-ʑi nɯ ra qarwoʁ tu-sɤrmi-nɯ tu-ŋgrɤl.}\hspace{5pt}\pcmn{\lien{ⓔqarwoʁ}{qarwoʁ}是用泥捏成的,过去是专门给老年人熬茶的,开口呈嘴形,倒茶时不会洒。\lien{ⓔqarwoʁ}{qarwoʁ}里的茶一旦烧开了,即使把它从火边移开仍然要溢出来,开个不停,所以人们把脾气暴躁的人说成是\lien{ⓔqarwoʁ}{qarwoʁ}。}\end{exemple}\end{entrée}

\begin{entrée}{qaʁ}{₂}{ⓔqaʁⓗ2} 
\classe{n} 
\begin{définition}\pfra{bêche}\end{définition}
\begin{définition}\pcmn{锄头}\end{définition}\end{entrée}

\begin{entrée}{qaʁ}{₁}{ⓔqaʁⓗ1} 
\classe{vt} \sens{1}\paradigme{dir}{nɯ-}\paradigme{dir}{pɯ-}\paradigme{dir}{\_}
\begin{définition}\pfra{enlever (peau, écorce), écorcher}\end{définition}
\begin{définition}\pcmn{剥皮}\end{définition}
\begin{exemple}\pjya{nɤ-ŋga nɯ-qaʁ}\hspace{5pt}\pcmn{你脱衣服吧}\end{exemple}
\begin{exemple}\pjya{sɯrqhu pɯ-qaʁ}\hspace{5pt}\pcmn{你剥树皮吧}\end{exemple}
\begin{exemple}\pjya{ɯ-ndʐi nɯ-qaʁ}\hspace{5pt}\pcmn{你剥它的皮子吧}\end{exemple}
\begin{exemple}\pjya{skɤmndʐi pɯ-qaʁ-a}\hspace{5pt}\pcmn{我剥了牛皮}\end{exemple}
\begin{exemple}\pjya{ɲchɣaʁ tɤ-qaʁ-a}\hspace{5pt}\pcmn{我剥了白桦树皮(往上剥)}\end{exemple}\sens{2}\paradigme{dir}{nɯ-}
\begin{définition}\pfra{rejeter}\end{définition}
\begin{définition}\pcmn{排挤;当外人}\end{définition}
\begin{exemple}\pjya{aʑo ɲɯ-kɯ-qaʁ-a ɲɯ-ŋu}\hspace{5pt}\pcmn{你把我当外人}\end{exemple}\relationsémantique{参考}{\lien{ⓔɴɢaʁ}{ɴɢaʁ}}
\begin{sous-entrée}{nɯɣɯqaʁ}{ⓔqaʁⓗ1ⓢ2ⓝnɯɣɯqaʁ} 
\classe{vs}  
\grammaire{facil} 
\begin{définition}\pfra{facile à enlever}\end{définition}
\begin{définition}\pcmn{容易剥}\end{définition}\end{sous-entrée}

\end{entrée}

\begin{entrée}{qase}{}{ⓔqase} 
\classe{n} 
\begin{définition}\pfra{corde en peau}\end{définition}
\begin{définition}\pcmn{皮条}\end{définition}\end{entrée}

\begin{entrée}{qatɯkɯr}{}{ⓔqatɯkɯr} 
\classe{n} 
\begin{définition}\pfra{mauvais conseils}\end{définition}
\begin{définition}\pcmn{反面教育}\end{définition}\relationsémantique{参考}{\lien{ⓔznɯqatɯkɯr}{znɯqatɯkɯr}}\end{entrée}

\begin{entrée}{qawɤr/\variante{qawɯwɤr}}{}{ⓔqawɤr} 
\classe{vi} \paradigme{dir}{tɤ-}
\begin{définition}\pfra{s'ouvrir (champignon)}\end{définition}
\begin{définition}\pcmn{展开(菌子)【开反】}\end{définition}
\begin{exemple}\pjya{tɤjmɤɣ to-qawɤr}\hspace{5pt}\pcmn{蘑菇展开了}\end{exemple}\end{entrée}

\begin{entrée}{qawoʁ}{}{ⓔqawoʁ} 
\classe{n} 
\begin{définition}\pfra{grotte}\end{définition}
\begin{définition}\pcmn{洞}\end{définition}\end{entrée}

\begin{entrée}{qawɯz}{}{ⓔqawɯz} 
\classe{n} 
\begin{définition}\pfra{edelweiss}\end{définition}
\begin{définition}\pcmn{火绒草}\end{définition}
\begin{exemple}\pjya{qawɯz nɯ sɯjno kɯ-mbɯ-mbɤr ci ŋu, sɯjno ɯ-jwaʁ nɯ ra ɯ-rme kɯ-tu, kɯ-jaʁ kɯ-fse ci ŋu. ɯ-mɯntoʁ ɯ-jwaʁ ɯ-ru lonba ʑo kɯ-wɣrum ŋu, tɕe ɲɯ́-wɣ-sɯɣ-rom tɕe, tú-wɣ-rŋu, tɕe tɯ-ɕke ta-ʑa tɕe, pjɯ́-wɣ-nɤtar tɕe pɯ-ndɯβ tɕe srɯn kɯ-fse ɲɯ-ɤndɯndo ŋu tɕe, ɯnɯ qawɯz ŋu tɕe, tɕaʁmɤr tɤ́-wɣ-lɤt tɕe, ɯ-smi sɤ-mɟa ŋu. tɕaʁmɤr kɤ-lɤt tɕe, qawɯz tɯ-snaʁ cho qapi ni tɯ-tɯ-ndo kú-wɣ-βzu tɕe tɯ-jaʁ ntsi kɯ tɕaʁmɤr pjɯ́-wɣ-sɯ-lɤt tɕe, qapi cho qawɯz ni ndʑi-ɕnɤz nɯtɕu tɕaʁmɤr pjɯ́-wɣ-lɤt, tɕe qapi ɯ-taʁ smɯtɕɣom tu-ɬoʁ tɕe, qawɯz ɯ-taʁ smi nɯ tu-nɯt ŋu. kɯɕɯŋgɯ tɕe smi kɤ-sɯ-βzu nɯ tu-stu-nɯ pjɤ-ŋu, tham tɕe nɯ kɯ-ntɕhoz me.}\hspace{5pt}\pcmn{火绒草是一种矮小的植物,叶子长有毛,有点厚。花、叶子和茎都是白色的。晒干了以后,干炒到开始焦黄时就用细条抽打,打细了以后就像棉花一样连在一起,这也是火绒草。打火镰时,是用来引火的东西。打火镰时,把燧石和一小块火绒草一起拿在手上另一只手拿着火镰,打在燧石和火绒草的一角,燧石上迸出火星,火绒草就会着火。过去取火就是这样的,现在没有人用这种方法了。}\end{exemple}\end{entrée}

\begin{entrée}{qaʑmbri}{}{ⓔqaʑmbri} 
\classe{n} 
\begin{définition}\pfra{plante grimpante}\end{définition}
\begin{définition}\pcmn{藤子}\end{définition}
\begin{exemple}\pjya{qaʑmbri nɯ tɯmbri kɯ-fse kɯ-rɲɟɯ-rɲɟi tɕe ɯ-rtsɤɣ raŋri ɯ-mɯntoʁ cho ɯ-jwaʁ ku-ndzoʁ ŋu. ɯ-jwaʁ ɯ-mdoʁ ldʑaŋnaʁ ŋu. ɯ-mɯntoʁ tshanlaŋ kɯ-fse kɯ-qarŋe ŋu}\hspace{5pt}\pcmn{藤子长得像绳子一样,很长,分节长叶开花。叶子呈深绿色,花像铃铛一样,呈黄色}\end{exemple}\end{entrée}

\begin{entrée}{qaʑo}{}{ⓔqaʑo} 
\classe{n} 
\begin{définition}\pfra{mouton}\end{définition}
\begin{définition}\pcmn{绵羊}\end{définition}\étymologie{gjaŋ}\end{entrée}

\begin{entrée}{qaʑolu}{}{ⓔqaʑolu} 
\classe{n} 
\begin{définition}\pfra{année du mouton}\end{définition}
\begin{définition}\pcmn{羊年}\end{définition}\end{entrée}

\begin{entrée}{qɤr}{}{ⓔqɤr} 
\classe{vt}  
\grammaire{autoben} \paradigme{dir}{thɯ-}\paradigme{dir}{nɯ-}\paradigme{dir}{tɤ-}\paradigme{dir}{tɤ-}
\begin{définition}\pfra{trier}\end{définition}
\begin{définition}\pcmn{挑选}\end{définition}
\begin{exemple}\pjya{stoʁrɣi thɯ-qar-a}\hspace{5pt}\pcmn{我选了胡豆种子}\end{exemple}
\begin{exemple}\pjya{jaŋjy nɯ-qar-a}\hspace{5pt}\pcmn{我选了土豆}\end{exemple}
\begin{exemple}\pjya{pɯ-kɯ-tsɣi cho mɤ-kɯ-tsɣi nɯ-qar-a}\hspace{5pt}\pcmn{我选出了没有烂的}\end{exemple}\relationsémantique{参考}{\lien{ⓔsɤqɤrle}{sɤqɤrle}}\relationsémantique{参考}{\lien{ⓔnɯndzɤqɤr}{nɯndzɤqɤr}}\relationsémantique{参考}{\lien{ⓔʑɣɤqɤr}{ʑɣɤqɤr}}
\begin{sous-entrée}{nɯqɤr}{ⓔqɤrⓝnɯqɤr}\end{sous-entrée}

\begin{exemple}\pjya{tɤ-nɯ-qɤr}\hspace{5pt}\pcmn{你自己选吧}\end{exemple}
\begin{exemple}\pjya{tɤ-nɯqar-a}\hspace{5pt}\pcmn{我选了}\end{exemple}
\begin{exemple}\pjya{nɤ-@gongzuo ma tɤ-nɯqɤr}\hspace{5pt}\pcmn{你自己选你的工作}\end{exemple}
\begin{exemple}\pjya{a-sɯm kɯ-ɕe tu-nɯqar-a ɕti}\hspace{5pt}\pcmn{我选择我愿意做的工作}\end{exemple}
\begin{sous-entrée}{nɤqɤrqɤr}{ⓔqɤrⓝnɤqɤrqɤr} 
\classe{vt} 
\begin{définition}\pfra{être indécis et ne pas savoir quoi choisir}\end{définition}
\begin{définition}\pcmn{反复挑选}\end{définition}
\begin{exemple}\pjya{kɯ-tu nɯ tɤ-ndze ma kɤ-nɤqɤrqɤr ntsɯ me}\hspace{5pt}\pcmn{有的那个你就吃吧,不要东选西选了}\end{exemple}\end{sous-entrée}

\end{entrée}

\begin{entrée}{qɤt}{}{ⓔqɤt} 
\classe{vt} \paradigme{dir}{nɯ-}\paradigme{dir}{pɯ-}
\begin{définition}\pfra{séparer}\end{définition}
\begin{définition}\pcmn{分开}\end{définition}
\begin{exemple}\pjya{mbro jla nɯ-qɤt}\hspace{5pt}\pcmn{你把马和犏牛分开}\end{exemple}
\begin{exemple}\pjya{a-mi nɯ-qat-a}\hspace{5pt}\pcmn{我把脚分开了}\end{exemple}
\begin{exemple}\pjya{a-kɤrme pɯ-qat-a}\hspace{5pt}\pcmn{我分了头发}\end{exemple}
\begin{exemple}\pjya{a-ŋga nɯ-qat-a}\hspace{5pt}\pcmn{我把衣服解开了(由两边分)}\end{exemple}\relationsémantique{同义词}{\lien{ⓔsɤqɤrle}{sɤqɤrle}}\relationsémantique{参考}{\lien{ⓔɴɢɤt}{ɴɢɤt}}\relationsémantique{参考}{\lien{}{tɤqɤt}}\end{entrée}

\begin{entrée}{qha}{}{ⓔqha} 
\classe{vt} \paradigme{dir}{tɤ-}\paradigme{dir}{nɯ-}
\begin{définition}\pfra{s'énerver, détester}\end{définition}
\begin{définition}\pcmn{生气;讨厌}\end{définition}
\begin{exemple}\pjya{hajtsu qhe-a}\hspace{5pt}\pcmn{我讨厌黑椒}\end{exemple}
\begin{exemple}\pjya{jiɕqha nɯ kɯ nɯ ɲɯ-ti ɲɯ-qhe-a}\hspace{5pt}\pcmn{那个人说了这个,我很讨厌}\end{exemple}
\begin{exemple}\pjya{ɲɯ-ta-qha}\hspace{5pt}\pcmn{我讨厌你}\end{exemple}
\begin{exemple}\pjya{ma-tɤ-tɯ-qhe}\hspace{5pt}\pcmn{你不要讨厌他}\end{exemple}
\begin{exemple}\pjya{tɤrkoz pɯ-maʁ, ma-tɤ-tɯ-qhe}\hspace{5pt}\pcmn{他不是故意的,你不要讨厌他}\end{exemple}
\begin{exemple}\pjya{smi kɯ tɯ-ci ɲɯ-qhe}\hspace{5pt}\pcmn{火讨厌水}\end{exemple}
\begin{exemple}\pjya{khɯna kɯ lɯlu ɲɯ-qhe}\hspace{5pt}\pcmn{狗讨厌猫}\end{exemple}
\begin{exemple}\pjya{tɯ-ci kɤ-lɤt ɲɯ-qhe}\hspace{5pt}\pcmn{他讨厌灌水}\end{exemple}
\begin{exemple}\pjya{tɯ-ci ɕɯ-kɤ-ru ɲɯ-qhe}\hspace{5pt}\pcmn{他讨厌去背水}\end{exemple}\relationsémantique{反义词}{\lien{ⓔnɯrga}{nɯrga}}\end{entrée}

\begin{entrée}{qhajŋgɯ}{}{ⓔqhajŋgɯ} 
\classe{n} 
\begin{définition}\pfra{voie d'eau du moulin}\end{définition}
\begin{définition}\pcmn{磨坊引水槽}\end{définition}
\begin{exemple}\pjya{qhajŋgɯ nɯ ɕoŋtɕa kɯ-wxti ʑo chɯ́-wɣ-phɯt tɕe chɯ́-wɣ-nɯrtaʁ chɯ́-wɣ-nɤrqhu tɕe ɯ-rqhioʁ chɯ́-wɣ-tɕɤt tɕe βɣa ɯ-lɤcu chɯ́-wɣ-tshoʁ tɕe ɯ-ŋgɯ tɯ-ci chɯ́-wɣ-sɯ-ɣe tɕe tɕhɯŋkhɤr ɯ-taʁ chɯ́-wɣ-sɯ-lɤt ɯ-spa ŋu}\hspace{5pt}\pcmn{\lien{ⓔqhajŋgɯ}{qhajŋgɯ}(是一种引水槽)。把一棵大树砍下,砍掉枝桠,刨去树皮然后挖一道槽,然后装在磨坊上,引水通过槽冲转水车。}\end{exemple}\end{entrée}

\begin{entrée}{qhaqhu}{}{ⓔqhaqhu} 
\classe{n} 
\begin{définition}\pfra{derrière la maison}\end{définition}
\begin{définition}\pcmn{房子的后面}\end{définition}\relationsémantique{参考}{\lien{ⓔkha}{kha}}\end{entrée}

\begin{entrée}{qharu}{}{ⓔqharu} 
\classe{n} 
\begin{définition}\pfra{regard en arrière}\end{définition}
\begin{définition}\pcmn{回头}\end{définition}\relationsémantique{参考}{\lien{ⓔnɤqharu}{nɤqharu}}\relationsémantique{参考}{\lien{ⓔɯ-qhu}{ɯ-qhu}}\relationsémantique{参考}{\lien{ⓔruⓗ1}{ru₁}}\end{entrée}

\begin{entrée}{qhaχɕu}{}{ⓔqhaχɕu} 
\classe{n} 
\begin{définition}\pfra{vantardise}\end{définition}
\begin{définition}\pcmn{炫耀}\end{définition}
\begin{exemple}\pjya{qhaχɕu ma-tɤ-tɯ-βze}\hspace{5pt}\pcmn{你不要炫耀}\end{exemple}\relationsémantique{参考}{\lien{ⓔrɯqhaχɕu}{rɯqhaχɕu}}\relationsémantique{参考}{\lien{ⓔnɯqhaχɕu}{nɯqhaχɕu}}\end{entrée}

\begin{entrée}{qhɤjmbaʁ}{}{ⓔqhɤjmbaʁ} 
\classe{n} 
\begin{définition}\pfra{Rumex sp.}\end{définition}
\begin{définition}\pcmn{酸模}\end{définition}
\begin{exemple}\pjya{qhɤjmbaʁ nɯ kha ɯ-rkɯ ra tɯ-ɣli kɯ-dɤn ɣɯ sɤtɕha pɕoʁ tu-ɬoʁ ŋu, ɯ-jwaʁ ɯ-tshɯɣa nɯ ŋɤnɯkɯmtsɯɣ cho naχtɕɯɣ, tɕeri qhɤjmbaʁ ɯ-jwaʁ nɯ mba ɯ-rme me, arŋi, qhɤjmbaʁ ɯ-ru tu-ɬoʁ tɕe, tɯ-rtsɤɣ tɕe ci ntsɯ ɯ-jwaʁ ku-ndzoʁ ŋu. ɯ-ru tɤ-ari ɯ-jija ɯ-jwaʁ tu-xtɕi ŋu. ɯ-ru ɯ-stɤt tɕe, ɯ-mɯntoʁ ɲɯ-lɤt. ɯ-mɯntoʁ aɣɯrnɯɕɯr. ɯʑo ɯ-ru nɯ kɯ-wxti nɯ tɯrme fsu jamar tu, nɯ maʁ nɤ tɤ-mthɤɣ fsu jamar ma me, paʁndza sna ma tɯrme kɤ-ndza mɤ-sna. ɯ-di mɤ-χɕɤβ. qhɤjmbaʁ li ci tɯ-tɯphu tu tɕe, kɯmaʁ ra naχtɕɯɣ tɕeri ɯ-jwaʁ rɲɟi cho jaʁ, mpɯ, tɯrme kɤ-ndza sna. ɯ-jwaʁ ɯ-sɤɣ-ndzoʁ nɯ tɕu ɯ-ru cho ɯ-pɤrthɤβ nɯ tɕu, tɯ-ci kɯ-fse kɯ-ɤrɤmtʂɯmtʂaj tu, ɯ-ru jpum tsa tɕe mpɯ.}\hspace{5pt}\pcmn{酸模生长在房子旁边,肥料比较多的地方,叶子的形状和红青椒的相同,但是酸模的叶子比较薄,没有毛,是绿色的。\lien{ⓔqhɤjmbaʁ}{qhɤjmbaʁ}的茎上每一节都长叶子。随着茎的长高,叶子就变小。茎的顶端开花。花是淡红色的。茎可以长到一个人那么高,但一般的只长到人的腰部那么高。可以喂猪,人不能吃。味道不浓。还有一种\lien{ⓔqhɤjmbaʁ}{qhɤjmbaʁ},其他(部分)和(前一种)一样,就是叶子比较长,比较厚,嫩,人可以吃。在茎上长叶子的部位之间,有一种粘液。茎粗而嫩。}\end{exemple}\end{entrée}

\begin{entrée}{qhɤndɤpa}{}{ⓔqhɤndɤpa} 
\classe{num} 
\begin{définition}\pfra{dans trois ans}\end{définition}
\begin{définition}\pcmn{三年以后}\end{définition}\end{entrée}

\begin{entrée}{qhɤndi}{}{ⓔqhɤndi} 
\classe{num} 
\begin{définition}\pfra{dans trois jours}\end{définition}
\begin{définition}\pcmn{大后天}\end{définition}\end{entrée}

\begin{entrée}{qhɤtɯɣ}{}{ⓔqhɤtɯɣ} 
\classe{n} 
\begin{définition}\pfra{bâton qui sert à caler la porte}\end{définition}
\begin{définition}\pcmn{关上门后,在地上斜顶住门的木棒}\end{définition}\end{entrée}

\begin{entrée}{qhi}{}{ⓔqhi} 
\classe{vs} \paradigme{dir}{tɤ-}\paradigme{dir}{thɯ-}\sens{1}
\begin{définition}\pfra{difficile à dompter (cheval)}\end{définition}
\begin{définition}\pcmn{难以驯服的马}\end{définition}\sens{2}
\begin{définition}\pfra{insolent}\end{définition}
\begin{définition}\pcmn{放肆}\end{définition}\end{entrée}

\begin{entrée}{qhinɤqhi}{}{ⓔqhinɤqhi} 
\classe{idph.3} 
\begin{définition}\pfra{bruit d'essoufflement}\end{définition}
\begin{définition}\pcmn{形容气喘吁吁的样子}\end{définition}
\begin{exemple}\pjya{a-fkur ɲɯ-rʑi tɕe, qhjinɤqhji ʑo tɤ-tɯt-a pɯ-ra}\hspace{5pt}\pcmn{因为我背的东西太重,所以有点气喘吁吁}\end{exemple}\relationsémantique{同义词}{\lien{ⓔχinɤχi}{χinɤχi}}\end{entrée}

\begin{entrée}{qhuj}{}{ⓔqhuj} 
\classe{n} 
\begin{définition}\pfra{ce soir, cet après-midi}\end{définition}
\begin{définition}\pcmn{今天下午;今天晚上}\end{définition}\end{entrée}

\begin{entrée}{qhjɤβqhjɤβ}{}{ⓔqhjɤβqhjɤβ} 
\classe{idph.2} 
\begin{définition}\pfra{de couleur terne}\end{définition}
\begin{définition}\pcmn{颜色不鲜艳}\end{définition}
\begin{exemple}\pjya{tɤŋe qhjɤβqhjɤβ ci ɣɤʑu}\hspace{5pt}\pcmn{(云多)太阳出不来}\end{exemple}\relationsémantique{同义词}{\lien{ⓔqhjiqhji}{qhjiqhji}}\end{entrée}

\begin{entrée}{qhjiqhji}{}{ⓔqhjiqhji} 
\classe{idph.2} 
\begin{définition}\pfra{de couleur terne}\end{définition}
\begin{définition}\pcmn{颜色不鲜艳}\end{définition}
\begin{exemple}\pjya{tɕoχtsi ɯ-mdoʁ nɯ kɯ-qarŋe qhjiqhji ɲɯ-ŋu}\hspace{5pt}\pcmn{桌子是淡黄色的}\end{exemple}
\begin{exemple}\pjya{nɤ-@diannao ɯ-mdoʁ kɯ-ɤrŋi qhjiqhji ɲɯ-ŋu}\hspace{5pt}\pcmn{你电脑的颜色是淡蓝色的}\end{exemple}
\begin{exemple}\pjya{a-ʑi ɲɯ-loʁ tɕe, a-rqo qhjiqhji ʑo ɲɯ-pa}\hspace{5pt}\pcmn{我感到恶心,喉咙里很不舒服}\end{exemple}\relationsémantique{同义词}{\lien{ⓔqhjɤβqhjɤβ}{qhjɤβqhjɤβ}}\end{entrée}

\begin{entrée}{qhlaʁ}{}{ⓔqhlaʁ} 
\classe{idph.1} 
\begin{définition}\pfra{disparaître d'un coup}\end{définition}
\begin{définition}\pcmn{突然消失}\end{définition}
\begin{exemple}\pjya{qhlaʁ ʑo to-ɕqhlɤt}\hspace{5pt}\pcmn{突然间消失了}\end{exemple}\end{entrée}

\begin{entrée}{qhlɤβqhlɤβ}{}{ⓔqhlɤβqhlɤβ} 
\classe{idph.2} \sens{1}
\begin{définition}\pfra{moyen, pas extrême (température, lumière etc)}\end{définition}
\begin{définition}\pcmn{形容温度、光等中等,不极端}\end{définition}
\begin{exemple}\pjya{tɯ-mɯ kɯ-jɯm tɕi ɲɯ-maʁ, kɯ-lɯβ tɕi ɲɯ-maʁ, qhlɤβqhlɤβ ʑo ɲɯ-pa}\hspace{5pt}\pcmn{天不晴也不阴,灰扑扑的}\end{exemple}\sens{2}
\begin{définition}\pfra{grisâtre}\end{définition}
\begin{définition}\pcmn{灰扑扑的}\end{définition}\end{entrée}

\begin{entrée}{qhloŋ}{}{ⓔqhloŋ} 
\classe{idph.1} 
\begin{définition}\pfra{plouf}\end{définition}
\begin{définition}\pcmn{噗通(突然掉进水里的声音)}\end{définition}
\begin{exemple}\pjya{qhloŋ ʑo pjɤ-ɕqhlɤt}\hspace{5pt}\pcmn{噗通一声就掉进水里了}\end{exemple}
\begin{sous-entrée}{qhloŋnɤloŋ}{ⓔqhloŋⓝqhloŋnɤloŋ} 
\classe{idph.4} \end{sous-entrée}

\begin{sous-entrée}{phɯqhloŋ}{ⓔqhloŋⓝphɯqhloŋ} 
\classe{idph.7} \end{sous-entrée}

\begin{sous-entrée}{sɤqhloŋloŋ}{ⓔqhloŋⓝsɤqhloŋloŋ} 
\classe{vt} 
\begin{définition}\pfra{faire du bruit en se débattant dans l'eau}\end{définition}
\begin{définition}\pcmn{在水里不停地发出“噗通”的声音(动物在水里挣扎的声音)}\end{définition}
\begin{exemple}\pjya{tɯ-ci ɯ-ŋgɯ ɲɯ-nɯsroʁmbrɤt tɕe ɲɯ-sɤqhloŋloŋ ʑo}\hspace{5pt}\pcmn{他在水里挣扎,发出噗通噗通的声音}\end{exemple}\end{sous-entrée}

\end{entrée}

\begin{entrée}{qhloŋqhloŋ}{}{ⓔqhloŋqhloŋ} 
\classe{idph.2} 
\begin{définition}\pfra{trouble (eau)}\end{définition}
\begin{définition}\pcmn{形容浑浊的样子}\end{définition}
\begin{exemple}\pjya{tɯ-ci qhloŋqhloŋ ʑo ɲɯ-qarndɯm}\hspace{5pt}\pcmn{水很浑浊的样子}\end{exemple}\end{entrée}

\begin{entrée}{qhlɯ}{}{ⓔqhlɯ} 
\classe{n} 
\begin{définition}\pfra{naga}\end{définition}
\begin{définition}\pcmn{龙;水神}\end{définition}
\begin{exemple}\pjya{nɤʑo qhlɯ to-tɯ-sɤzmbrɯ-t}\hspace{5pt}\pcmn{你惹了水神}\end{exemple}\étymologie{klu}\end{entrée}

\begin{entrée}{qhlɯqhlu/\variante{qhluqhlu}}{}{ⓔqhlɯqhlu} 
\classe{idph.2} 
\begin{définition}\pfra{un peu laiteux}\end{définition}
\begin{définition}\pcmn{形容带有一点乳白色的(液体)}\end{définition}
\begin{exemple}\pjya{tɤ-lu ci qhluqhlu tɤ-lat-a}\hspace{5pt}\pcmn{我倒了一点点牛奶(茶水里只有一点乳白色)}\end{exemple}\end{entrée}

\begin{entrée}{qhoʁqhoʁ}{}{ⓔqhoʁqhoʁ} 
\classe{n} 
\begin{définition}\pfra{lingot}\end{définition}
\begin{définition}\pcmn{一块银子}\end{définition}\end{entrée}

\begin{entrée}{qhoʁsjɯβ}{}{ⓔqhoʁsjɯβ} 
\classe{n} 
\begin{définition}\pfra{creux}\end{définition}
\begin{définition}\pcmn{空心}\end{définition}\end{entrée}

\begin{entrée}{qhrɯt}{}{ⓔqhrɯt} 
\classe{vt} \paradigme{dir}{tɤ-}\paradigme{dir}{nɯ-}\paradigme{dir}{\_}
\begin{définition}\pfra{gratter complètement}\end{définition}
\begin{définition}\pcmn{刮干净}\end{définition}
\begin{exemple}\pjya{tɯthɯ ta-qhrɯt}\hspace{5pt}\pcmn{他刮了锅子}\end{exemple}
\begin{exemple}\pjya{ɲɤ-k-ɤtɕaʁ-ci ɲɤ-qhrɯt}\hspace{5pt}\pcmn{有东西粘上去了他就刮了}\end{exemple}
\begin{exemple}\pjya{nɯ-qhrɯt-a}\hspace{5pt}\pcmn{我刮了}\end{exemple}
\begin{exemple}\pjya{nɯ-tɯ-qhrɯt}\hspace{5pt}\pcmn{你刮了}\end{exemple}
\begin{exemple}\pjya{tɤrcoʁ ra nɯ-qhrɯt}\hspace{5pt}\pcmn{你把那些泥刮一下}\end{exemple}
\begin{exemple}\pjya{tɤndɤr nɯ-qhrɯt-a}\hspace{5pt}\pcmn{我把粉刺刮了一下}\end{exemple}
\begin{exemple}\pjya{βʑɯ kɯ tɕoχtsi ɯ-taʁ kɯ-ɴqhi nɯ ɲɯ-ɤz-nɯ-qhrɯt}\hspace{5pt}\pcmn{老鼠在啃桌子上的脏东西}\end{exemple}\sens{2}
\begin{définition}\pfra{faire complètement}\end{définition}
\begin{définition}\pcmn{做得彻底}\end{définition}\end{entrée}

\begin{entrée}{qia}{}{ⓔqia} 
\classe{vt} \paradigme{dir}{pɯ-}\paradigme{dir}{thɯ-}
\begin{définition}\pfra{déchirer, démolir}\end{définition}
\begin{définition}\pcmn{拆(线)}\end{définition}
\begin{exemple}\pjya{nɤki tɤ-tɯ-mphɯr nɯ pɯ-qie}\hspace{5pt}\pcmn{你把包了的东西拆开}\end{exemple}
\begin{exemple}\pjya{nɤj tɤ-tɯ-mphɯr aj pjɯ-qie-a}\hspace{5pt}\pcmn{我拆你包了的东西}\end{exemple}
\begin{exemple}\pjya{tɤ-mphɯr-a nɯ pɯ-qia-t-a}\hspace{5pt}\pcmn{我把包了的东西拆开}\end{exemple}
\begin{exemple}\pjya{nɤ-ŋga thɯ-qie}\hspace{5pt}\pcmn{你把(缝了的)衣服拆开}\end{exemple}
\begin{exemple}\pjya{tɤ-tɯ-βzu-t nɯ pɯ-qie}\hspace{5pt}\pcmn{你把做了的东西拆开}\end{exemple}
\begin{exemple}\pjya{ɯ-kɤpjɤz tha-qia}\hspace{5pt}\pcmn{他把辫子散开了}\end{exemple}\relationsémantique{同义词}{\lien{ⓔfɕɯɣ}{fɕɯɣ}}\relationsémantique{参考}{\lien{ⓔɴɢia}{ɴɢia}}\end{entrée}

\begin{entrée}{qiaβ}{}{ⓔqiaβ} 
\classe{vi} \paradigme{dir}{nɯ-}\paradigme{dir}{pɯ-}
\begin{définition}\pfra{amer}\end{définition}
\begin{définition}\pcmn{苦}\end{définition}
\begin{exemple}\pjya{ɲɯ-qiaβ}\end{exemple}
\begin{exemple}\pjya{pɯ-nɤqiaβ-a}\hspace{5pt}\pcmn{我觉得太苦了}\end{exemple}\relationsémantique{反义词}{\lien{ⓔchi}{chi}}
\begin{sous-entrée}{nɤqiaβ}{ⓔqiaβⓝnɤqiaβ} 
\classe{vt}  
\grammaire{trop} \end{sous-entrée}

\end{entrée}

\begin{entrée}{qioʁ}{}{ⓔqioʁ} 
\classe{vi} \paradigme{dir}{lɤ-}
\begin{définition}\pfra{vomir}\end{définition}
\begin{définition}\pcmn{呕吐}\end{définition}
\begin{exemple}\pjya{lɤ-qioʁ-a}\hspace{5pt}\pcmn{我吐了}\end{exemple}
\begin{exemple}\pjya{ɲɯ-ngo-a tɕe lɤ-qioʁ-a}\hspace{5pt}\pcmn{我病了就吐了}\end{exemple}
\begin{exemple}\pjya{lo-tɯ-qioʁ}\hspace{5pt}\pcmn{你吐了}\end{exemple}
\begin{exemple}\pjya{tɤ-kɯ-nɯtɕhomba tɕe ɲɯ-kɯ-qioʁ}\hspace{5pt}\pcmn{感冒的时候就会吐}\end{exemple}\relationsémantique{同义词}{\lien{ⓔmɯjphɤt}{mɯjphɤt}}\relationsémantique{参考}{\lien{ⓔtɯqioʁ}{tɯqioʁ}}\end{entrée}

\begin{entrée}{qlaqla}{}{ⓔqlaqla} 
\classe{idph.2} 
\begin{définition}\pfra{regard étonné, les yeux écarquillés}\end{définition}
\begin{définition}\pcmn{形容惊奇的眼光,眼睛睁得很大的样子}\end{définition}
\begin{exemple}\pjya{a-rŋa qlaqla ku-ru ɲɯ-ŋu}\hspace{5pt}\pcmn{他把眼睛睁得很大,看着我的脸}\end{exemple}
\begin{exemple}\pjya{ɯ-mɲaʁ qlaqla ʑo to-stu}\hspace{5pt}\pcmn{他把眼睛睁得很大}\end{exemple}\end{entrée}

\begin{entrée}{qlɤβqlɤβ}{}{ⓔqlɤβqlɤβ} 
\classe{idph.2} \sens{1}
\begin{définition}\pfra{les yeux ouverts}\end{définition}
\begin{définition}\pcmn{形容眼睛睁开着的样子}\end{définition}
\begin{exemple}\pjya{ɯ-mɲaʁ ɲɯ-ɲɟɯ qlɤβqlɤβ ʑo ɲɯ-pa}\hspace{5pt}\pcmn{他眼睛睁得大大的}\end{exemple}\sens{2}
\begin{définition}\pfra{trouble}\end{définition}
\begin{définition}\pcmn{形容水浑浊的样子}\end{définition}
\begin{exemple}\pjya{cɯβloʁ nɯ kɯ-xtɕɯ-xtɕi ɲɯ-qarndɯm qlɤβqlɤβ}\hspace{5pt}\pcmn{池塘的水有点浑浊}\end{exemple}
\begin{sous-entrée}{qlɤβnɤqlɤβ}{ⓔqlɤβqlɤβⓢ2ⓝqlɤβnɤqlɤβ} 
\classe{idph.3} 
\begin{exemple}\pjya{ɯ-mɲaʁ qlɤβnɤqlɤβ ɲɯ-ɤsɯ-stu}\hspace{5pt}\pcmn{他眼睛一眨一眨的}\end{exemple}\end{sous-entrée}

\end{entrée}

\begin{entrée}{qloŋ}{}{ⓔqloŋ}\relationsémantique{参考}{\lien{ⓔqhloŋ}{qhloŋ}}\end{entrée}

\begin{entrée}{qlɯβ}{}{ⓔqlɯβ} 
\classe{idph.1} 
\begin{définition}\pfra{bruit d'un objet jeté dans l'eau}\end{définition}
\begin{définition}\pcmn{形容东西跳到水里的声音}\end{définition}
\begin{exemple}\pjya{tɯ-ci ɯ-ŋgɯ qlɯβ ʑo pa-βde}\hspace{5pt}\pcmn{他啪哒一声扔到水里了}\end{exemple}
\begin{exemple}\pjya{qlɯβ ʑo ka-tshi}\hspace{5pt}\pcmn{咕噜一声就喝了}\end{exemple}
\begin{sous-entrée}{qlɯβnɤqlɯβ}{ⓔqlɯβⓝqlɯβnɤqlɯβ} 
\classe{idph.3} 
\begin{exemple}\pjya{qlɯβnɤqlɯβ ɲɯ-nɤŋkɯŋke}\hspace{5pt}\pcmn{啪哒啪哒地走}\end{exemple}\relationsémantique{参考}{\lien{ⓔɕqlɯβnɤɕqlɯβ}{ɕqlɯβnɤɕqlɯβ}}\end{sous-entrée}

\end{entrée}

\begin{entrée}{qlɯqlɯ}{}{ⓔqlɯqlɯ} 
\classe{idph.2} 
\begin{définition}\pfra{les yeux écarquillés}\end{définition}
\begin{définition}\pcmn{形容瞪着眼睛的样子}\end{définition}
\begin{exemple}\pjya{qlɯqlɯ ʑo ku-ru ɲɯ-ŋu}\hspace{5pt}\pcmn{他瞪着眼睛盯着他}\end{exemple}\relationsémantique{参考}{\lien{ⓔɕquɕqu}{ɕquɕqu}}\relationsémantique{参考}{\lien{ⓔɕqhɯɕqhi}{ɕqhɯɕqhi}}\end{entrée}

\begin{entrée}{qlɯt}{}{ⓔqlɯt} 
\classe{vt} \paradigme{dir}{pɯ-}
\begin{définition}\pfra{rompre, casser}\end{définition}
\begin{définition}\pcmn{折断(棍子)}\end{définition}
\begin{exemple}\pjya{si ɯ-rtaʁ pɯ-qlɯt-a}\hspace{5pt}\pcmn{我把树折断了}\end{exemple}
\begin{exemple}\pjya{jiɕqha @qiche ɯ-taʁ chɤ-mbɣaʁ tɕe, ɯ-mi pjɤ-nɯqlɯt}\hspace{5pt}\pcmn{他出了车祸,把脚弄断了}\end{exemple}
\begin{exemple}\pjya{nɤ-mu nɯ ndʑu tɯ-ldʑa cinɤ ʑo a-mɤ-nɯ-tɯ-sɯ-qlɯt ra}\hspace{5pt}\pcmn{你不要让你母亲做任何家务(连一个木棒都不要让她折断)}\end{exemple}\relationsémantique{参考}{\lien{ⓔɴɢlɯt}{ɴɢlɯt}}\end{entrée}

\begin{entrée}{qoʁmɢlɤcit}{}{ⓔqoʁmɢlɤcit} 
\classe{n} 
\begin{définition}\pfra{faire son premier pas (enfant)}\end{définition}
\begin{définition}\pcmn{会走路(婴儿)}\end{définition}
\begin{exemple}\pjya{a-tɕɯ ɯ-qoʁmɢlɤcit pɯ-cha}\hspace{5pt}\pcmn{我儿子会走路了}\end{exemple}\relationsémantique{参考}{\lien{ⓔɯ-qoʁ}{ɯ-qoʁ}}\relationsémantique{参考}{\lien{ⓔtɯ-mɢla}{tɯ-mɢla}}\relationsémantique{参考}{\lien{ⓔcit}{cit}}\end{entrée}

\begin{entrée}{qru}{}{ⓔqru} 
\classe{vt} \paradigme{dir}{tɤ-}\paradigme{dir}{\_}
\begin{définition}\pfra{accueillir, aller chercher}\end{définition}
\begin{définition}\pcmn{迎接}\end{définition}
\begin{exemple}\pjya{tɤ-kɯ-qru-a}\hspace{5pt}\pcmn{你接了我}\end{exemple}
\begin{exemple}\pjya{a-rʑaβ ɕ-tɤ-qru-t-a}\hspace{5pt}\pcmn{我娶了妻子}\end{exemple}
\begin{exemple}\pjya{tɤ-pɤtso ɕ-tɤ-qru-t-a}\hspace{5pt}\pcmn{我去接了孩子}\end{exemple}
\begin{exemple}\pjya{ndʐuwa ɕ-tɤ-qru-t-a}\hspace{5pt}\pcmn{我去接了客人}\end{exemple}
\begin{exemple}\pjya{ɯ-rʑaβ ja-qru}\hspace{5pt}\pcmn{他娶了妻子}\end{exemple}
\begin{exemple}\pjya{ɯʑo kɯ ``a-ɣɯ-jɤ-kɯ-qru-a ra" ɲɯ-ti}\hspace{5pt}\pcmn{他要求我接他(直译:他说“你要来接我”)}\end{exemple}\relationsémantique{参考}{\lien{ⓔnɯqru}{nɯqru}}
\begin{sous-entrée}{saqru}{ⓔqruⓝsaqru} 
\classe{vi}  
\grammaire{apass} \end{sous-entrée}

\end{entrée}

\begin{entrée}{qur}{}{ⓔqur} 
\classe{vt} \paradigme{dir}{tɤ-}
\begin{définition}\pfra{aider}\end{définition}
\begin{définition}\pcmn{帮助}\end{définition}
\begin{exemple}\pjya{tɤ-qur-a}\hspace{5pt}\pcmn{我帮了他}\end{exemple}
\begin{exemple}\pjya{tɤ́-wɣ-qur-a}\hspace{5pt}\pcmn{他帮了我}\end{exemple}
\begin{exemple}\pjya{ɣɯ-tɤ́-wɣ-qur-a}\hspace{5pt}\pcmn{他来帮我了}\end{exemple}
\begin{exemple}\pjya{ɕ-pɯ-qur-tɕi}\hspace{5pt}\pcmn{我们去帮他了}\end{exemple}
\begin{exemple}\pjya{tɤ-kɯ-qur-a}\hspace{5pt}\pcmn{你帮了我}\end{exemple}
\begin{exemple}\pjya{tɤ́-wɣ-qur-a tɕe, jɤ-scɤt-tɕi}\hspace{5pt}\pcmn{他帮了我,我们俩一起搬了}\end{exemple}
\begin{exemple}\pjya{tɤ-scoz kɤ-rɤt tɤ́-wɣ-qur-a}\hspace{5pt}\pcmn{他帮我写信了}\end{exemple}
\begin{exemple}\pjya{ɯʑo kɯ ``a-ɣɯ-jɤ-kɯ-qur-a ra" ɲɯ-ti}\hspace{5pt}\pcmn{他要求我帮他(直译:他说“你要帮我”)}\end{exemple}
\begin{sous-entrée}{sɤqur}{ⓔqurⓝsɤqur} 
\classe{vi}  
\grammaire{apass} 
\begin{définition}\pfra{aider les gens}\end{définition}
\begin{définition}\pcmn{帮人家}\end{définition}
\begin{exemple}\pjya{tɕheme kɯ-sɤqur}\hspace{5pt}\pcmn{女帮手}\end{exemple}\relationsémantique{参考}{\lien{ⓔqurɣa}{qurɣa}}\relationsémantique{参考}{\lien{ⓔaqurle}{aqurle}}\end{sous-entrée}

\end{entrée}

\begin{entrée}{qra}{}{ⓔqra} 
\classe{n} 
\begin{définition}\pfra{femelle de yak}\end{définition}
\begin{définition}\pcmn{母牦牛}\end{définition}\end{entrée}

\begin{entrée}{qraʁ}{₂}{ⓔqraʁⓗ2} 
\classe{n} 
\begin{définition}\pfra{soc}\end{définition}
\begin{définition}\pcmn{铧头}\end{définition}
\begin{exemple}\pjya{kɤ-ɕlu tɤ-mda tɕe mbɣo ɯ-pa qraʁ kɤ-tshoʁ ra}\hspace{5pt}\pcmn{当要耕地的时候,要在犁头下面装好铧头}\end{exemple}\end{entrée}

\begin{entrée}{qraʁ}{₁}{ⓔqraʁⓗ1} 
\classe{vt} \paradigme{dir}{thɯ-}\paradigme{dir}{pɯ-}
\begin{définition}\pfra{déchirer}\end{définition}
\begin{définition}\pcmn{撕}\end{définition}
\begin{exemple}\pjya{nɤ-ŋga chɤ-tɯ-qraʁ}\hspace{5pt}\pcmn{你把衣服撕破了}\end{exemple}
\begin{exemple}\pjya{lɯlu kɯ pɯ́-wɣ-mɯrʁɯz-a tɕe a-jaʁ pjɤ-qraʁ}\hspace{5pt}\pcmn{猫把我抓了一下,抓破了我的手}\end{exemple}\relationsémantique{参考}{\lien{ⓔɴɢraʁ}{ɴɢraʁ}}\end{entrée}

\begin{entrée}{qrɤβqrɤβ}{}{ⓔqrɤβqrɤβ} 
\classe{idph.2} 
\begin{définition}\pfra{bariolé et disharmonieux}\end{définition}
\begin{définition}\pcmn{形容驳杂,不好看的样子}\end{définition}
\begin{exemple}\pjya{ɯ-muj ɯ-mdoʁ ɲɯ-ɤkhra ʑo qrɤβqrɤβ tɕe mɯ́j-mpɕɤr}\hspace{5pt}\pcmn{它的羽毛颜色东一块西一块的,不好看}\end{exemple}\end{entrée}

\begin{entrée}{qrɤmɟoʁ}{}{ⓔqrɤmɟoʁ} 
\classe{n} 
\begin{définition}\pfra{une espèce d'arbre}\end{définition}
\begin{définition}\pcmn{乔木的一种}\end{définition}
\begin{exemple}\pjya{qrɤmɟoʁ nɯ si kɯ-rkɯn tsa ci ŋu, mbro tsa ɲɯ-jpum mɤ-cha, ɯ-ru nɯ kɯ-ɣɯrni ŋu, ɯ-jwaʁ nɯ kɯ-tɕɤr tɕe kɯ-ɤmtɕoʁ tsa ci ŋu, ɯ-si nɯ kɯ-ndoʁ tsa ci mɤ-kɯ-ngɯt, ɯ-mɯntoʁ cho ɯ-mat ra me, sɤtɕha kɯ-mpja tu-ɬoʁ, sɤtɕha kɯ-ɣɤndʐo tu-ɬoʁ mɤ-cha.}\hspace{5pt}\pcmn{\lien{ⓔqrɤmɟoʁ}{qrɤmɟoʁ}是罕见的树种,比较高但是长得不粗,树干是红色的,叶子细而尖,木质脆,不结实。既没有花也没有果实,生长在温暖的地方,寒冷的地方不能生长。}\end{exemple}\end{entrée}

\begin{entrée}{qrɤntshom}{}{ⓔqrɤntshom} 
\classe{n} 
\begin{définition}\pfra{espèce d'arbrisseau}\end{définition}
\begin{définition}\pcmn{灌木的一种}\end{définition}
\begin{exemple}\pjya{qrɤntshom nɯ si ci ŋu, ɯ-jwaʁ ɯ-qhuchu nɯ ra tɯ-ɣndʑɤr ʑo kɯ-fse tu, ɯ-jwaʁ kɯ-zri tsa kɯ-ɤmtɕoʁ tsa ŋu, si mɤ-mbro cɯrmbɯ ɯ-ŋgɯ tsa tu-ɬoʁ ŋu, ɯ-mɯntoʁ nɯ kɯ-ndɯ-ndɯβ kɯ-dɯ-dɤn kɯɕnom ʑo kɯ-fse ŋu. ɯ-mdoʁ nɯ kɯ-ɣɯrni tsa ŋu.}\hspace{5pt}\pcmn{\lien{ⓔqrɤntshom}{qrɤntshom}是一种树,叶子背面有像面粉一样的东西,叶子有点长和有点尖,树不高,生长在石头较多的地方,花小而密,穗状,浅红色。}\end{exemple}\end{entrée}

\begin{entrée}{qrɤz}{}{ⓔqrɤz} 
\classe{vt} \paradigme{dir}{thɯ-}\paradigme{dir}{pɯ-}
\begin{définition}\pfra{raser}\end{définition}
\begin{définition}\pcmn{剃}\end{définition}
\begin{exemple}\pjya{nɤ-ku pɯ-qrɤz}\hspace{5pt}\pcmn{你剃一下头}\end{exemple}
\begin{exemple}\pjya{tɤ-rme pɯ-qrɤz}\hspace{5pt}\pcmn{你剃一下毛}\end{exemple}
\begin{exemple}\pjya{ɯ-ku thɯ-qraz-a}\hspace{5pt}\pcmn{我给他剃了头}\end{exemple}
\begin{exemple}\pjya{a-mtɕhirme pɯ-nɯ-qraz-a}\hspace{5pt}\pcmn{我剃了胡子}\end{exemple}\relationsémantique{参考}{\lien{ⓔɴɢrɤz}{ɴɢrɤz}}\end{entrée}

\begin{entrée}{qrubu}{}{ⓔqrubu} 
\classe{n} 
\begin{définition}\pfra{coquillage}\end{définition}
\begin{définition}\pcmn{贝壳}\end{définition}\end{entrée}

\begin{entrée}{qurɣa}{}{ⓔqurɣa} 
\classe{vt} \paradigme{dir}{tɤ-}
\begin{définition}\pfra{aider}\end{définition}
\begin{définition}\pcmn{帮助}\end{définition}
\begin{exemple}\pjya{tɤ-qurɣa-t-a}\hspace{5pt}\pcmn{我帮了他}\end{exemple}\relationsémantique{参考}{\lien{ⓔqur}{qur}}\end{entrée}

\begin{entrée}{qro}{₁}{ⓔqroⓗ1} 
\classe{n} 
\begin{définition}\pfra{pigeon}\end{définition}
\begin{définition}\pcmn{鸽子}\end{définition}
\begin{exemple}\pjya{qro nɯ pɣa ci ŋu, khro mɤ-wxti, pha ɯ-phoŋbu ʑo wɣrum ri ɯ-jme ɯ-ku ri kɯ-ɲaʁ tu, ɯ-mi qarŋe, tɤ-rɤku cho qajɯ tu-ndze ŋu, ftɕar tɕe tɯ-tɕha tɯ-tɕha ku-rɤʑi ɲɯ-ŋu, qartsɯ tɕe ɯ-ɣurʑa ʑo tɯtɯrca ku-rɤʑi ɲɯ-ŋu. kha ku-kɤ-nɤχsu ci ɣɤʑu, ɯʑo ku-kɯ-nɯ-rɤʑi nɯ ɣɤʑu tɕe praʁ pa wuma ʑo kɯ-mbro zɯ ku-rma ɲɯ-ŋu. ɯ-ɕa nɯ wuma ʑo smɤn ɲɯ-ŋu, ɯ-ku nɯ tɯ-ku kɯ-mŋɤm smɤn ɲɯ-ŋu, ɯ-se nɯ li tɯ-ku tɯ-kɤrnoʁ kɯ-mtɕɯr kɯ-phɤn ɲɯ-ŋu. qro tɤ-rɤku ndze ri wuma mɤ-ʁnɤt tɕe ɯ-kɯ-qha rkɯn. qro kɯ-nɯkhɤβɣa kɤ-ti tu tɕe, khɤxtu zɯ ku-zo tɕe, qro nɯ χcha tɯ-tɤxɯr ku-mtɕɯr, ʁe tɯ-tɤxɯr ku-mtɕɯr, tɕe ɯ-khɯkha, `kuku kuku' tu-ti ŋu. nɯ kɤntɕhɯ-ɣjɤn tu-fse ŋu.}\hspace{5pt}\pcmn{鸽子是一种鸟,不是很大,身子是白的,但是尾巴末端有黑点,脚是黄色的,吃粮食和虫子。夏天,它们成对地生活,冬天数百只在一起。有的是人在家里喂的,有的是(在野外)自己生活的,栖息在很高的岩洞里。鸽子肉是一种药材,它的头可以治头疼病,血可以治头晕。虽然鸽子吃粮食,但是不是很厉害,讨厌它的人不多。有人也说鸽子会用手磨,因为着落在房背上的时候,就会向左转一圈,向右转一圈,同时又叫\lien{}{kuku kuku},这要重复很多次。}\end{exemple}\end{entrée}

\begin{entrée}{qro}{₂}{ⓔqroⓗ2} 
\classe{n} 
\begin{définition}\pfra{fourmi}\end{définition}
\begin{définition}\pcmn{蚂蚁}\end{définition}\relationsémantique{参考}{\lien{ⓔnɯqro}{nɯqro}}\end{entrée}

\begin{entrée}{qromkemdoʁ}{}{ⓔqromkemdoʁ} 
\classe{n} 
\begin{définition}\pfra{violet}\end{définition}
\begin{définition}\pcmn{紫色}\end{définition}\relationsémantique{参考}{\lien{ⓔqroⓗ1}{qro₁}}\relationsémantique{参考}{\lien{ⓔtɯ-mke}{tɯ-mke}}\relationsémantique{参考}{\lien{ⓔɯ-mdoʁ}{ɯ-mdoʁ}}\étymologie{mdog}\end{entrée}

\begin{entrée}{qroɲaʁ}{}{ⓔqroɲaʁ} 
\classe{n} 
\begin{définition}\pfra{petite fourmi}\end{définition}
\begin{définition}\pcmn{小蚂蚁}\end{définition}\relationsémantique{参考}{\lien{ⓔqroⓗ2}{qro₂}}\end{entrée}

\begin{entrée}{qrormbɯ}{}{ⓔqrormbɯ} 
\classe{n} 
\begin{définition}\pfra{fourmilière}\end{définition}
\begin{définition}\pcmn{蚂蚁巢}\end{définition}\relationsémantique{参考}{\lien{ⓔqroⓗ2}{qro₂}}\end{entrée}

\begin{entrée}{qrorni}{}{ⓔqrorni} 
\classe{n} 
\begin{définition}\pfra{fourmi}\end{définition}
\begin{définition}\pcmn{蚂蚁}\end{définition}\relationsémantique{参考}{\lien{ⓔqroⓗ2}{qro₂}}\relationsémantique{参考}{\lien{ⓔɣɯrni}{ɣɯrni}}\end{entrée}

\begin{entrée}{qrose}{}{ⓔqrose} 
\classe{n} 
\begin{définition}\pfra{une espèce d'arbrisseau}\end{définition}
\begin{définition}\pcmn{灌木的一种【酸酸泡儿】}\end{définition}
\begin{exemple}\pjya{qrose nɯ si wuma mɤ-kɯ-mbro ci ŋu. ɯ-ru ɣɯrni, aɣɯrtɯrtaʁ, mɤ-jpum. ɯ-jwaʁ ɯ-βzɯr kɯmŋu kɯ-tu nɯ ŋu, ɯ-jwaʁ ɯ-taʁ ɯ-rme kɯ-fse kɯ-xtɕi tu ri, mɤ-rʁom. babɯ ɯ-jwaʁ sɤznɤ jndʐɤz. ɯ-mat nɯ thɯ-kɤ-ɣɯri kɯ-fse tɯ-tɤri tɯ-tɤri pjɤ-ɴqoʁ ŋu, tɕe ɯ-mat thɯ-tɯt kɯ-mɤku ɣɯrni, konla thɯ-tɯt tɕe ɲaʁ. ɲɯ́-wɣ-tɕɣaʁ tɕe, ɯ-ŋgɯ tɤ-se kɯ-fse ɲɯ-nɯɬoʁ ŋu. tú-wɣ-ndza tɕe wuma ʑo tɕur, tɤ-pɤtso ra rga-nɯ ma tɯrme kɯ-wxti ra kɤ-ndza mɤ-kɯ-cha.}\hspace{5pt}\pcmn{酸酸泡儿是长得不高的一种树。树干是红色的,长很多枝条,不粗。叶子有五个角,叶上有毛,但不粗糙。比\lien{ⓔbabɯ}{babɯ}的叶子大。果实像穿起来的,一串一串地挂在枝上。果实开始成熟时,是红色,完全成熟时,是黑色的,把它捏破时,里面会流出血一样的液体。吃起来很酸,小孩子喜欢吃,但大人不敢吃。}\end{exemple}\end{entrée}

\begin{entrée}{qrotsoʁ}{}{ⓔqrotsoʁ} 
\classe{n} 
\begin{définition}\pfra{une plante}\end{définition}
\begin{définition}\pcmn{植物的一种}\end{définition}
\begin{exemple}\pjya{qrotsoʁ nɯ sɯjno kɯ-xtɕɯ-xtɕi ci ŋu, ɯ-ku nɯ kɯ-ɤrŋi ŋgɯz kɯ-ɲaʁ ci ŋu, ɯ-mɯntoʁ wɣrum, ɯ-mɯntoʁ nɯ ɯ-jwaʁ rca ri ku-ndzoʁ ŋu, ɯ-ru xtshɯm, ɯ-jwaʁ cho aɣɯmdoʁ, ɯ-qa nɯ kɯ-ɤɣrɤɣrum ci ɲɯ-ŋu, ɯ-tshɯɣa nɯ @ou tsa fse, wuma ʑo ndoʁ. mɤ-sɤndɤɣ ri ɯ-kɯ-ndza me.}\hspace{5pt}\pcmn{\lien{ⓔqrotsoʁ}{qrotsoʁ}是一种小草。苗是深绿色的,花是白色的,花和叶子长在一起,茎很细,颜色和叶子一样,根很白,形状和藕一样,很脆。没有毒性但是没有人吃。}\end{exemple}\end{entrée}

\begin{entrée}{qrɯ}{}{ⓔqrɯ} 
\classe{vt} \sens{1}\paradigme{dir}{pɯ-}
\begin{définition}\pfra{casser}\end{définition}
\begin{définition}\pcmn{捣碎}\end{définition}
\begin{exemple}\pjya{khɯtsa pa-qrɯ}\hspace{5pt}\pcmn{他把碗打破了}\end{exemple}
\begin{exemple}\pjya{phoŋ pa-qrɯ}\hspace{5pt}\pcmn{他把瓶子打破了}\end{exemple}
\begin{exemple}\pjya{cupa pa-qrɯ}\hspace{5pt}\pcmn{他把石板打破了}\end{exemple}\sens{2}\paradigme{dir}{nɯ-}
\begin{définition}\pfra{tailler (habit)}\end{définition}
\begin{définition}\pcmn{裁(衣服)}\end{définition}
\begin{exemple}\pjya{tɯ-ŋga na-qrɯ}\hspace{5pt}\pcmn{他裁了衣服}\end{exemple}
\begin{exemple}\pjya{aʑo tɯ-ŋga ɲɯ-qri-a}\hspace{5pt}\pcmn{我裁衣服}\end{exemple}\relationsémantique{参考}{\lien{ⓔɴɢrɯ}{ɴɢrɯ}}\end{entrée}

\begin{entrée}{qrɯnqrɯn}{}{ⓔqrɯnqrɯn} 
\classe{idph.2} 
\begin{définition}\pfra{dont les bandes sont très claires}\end{définition}
\begin{définition}\pcmn{形容花纹、纹路清晰}\end{définition}
\begin{exemple}\pjya{kɯrtsɤɣ ɲɯ-ɤkhra qrɯnqrɯn}\hspace{5pt}\pcmn{豹子的花纹很清晰}\end{exemple}\end{entrée}

\begin{entrée}{qrɯt}{}{ⓔqrɯt} 
\classe{idph.1} 
\begin{définition}\pfra{regard furtif}\end{définition}
\begin{définition}\pcmn{瞟一眼}\end{définition}
\begin{exemple}\pjya{pjɯ-rɤrɤt ɯ-khɯkha, qrɯt ntsɯ ku-ru ɲɯ-ŋu}\hspace{5pt}\pcmn{他一边写一边朝这里瞟}\end{exemple}\end{entrée}

\begin{entrée}{qusput}{}{ⓔqusput} 
\classe{n} 
\begin{définition}\pfra{coucou}\end{définition}
\begin{définition}\pcmn{杜鹃【布谷鸟】}\end{définition}
\begin{exemple}\pjya{qusput nɯ pɣa ci ŋu, khro mɤ-wxti, ɯ-phoŋbu nɯ ra kɯ-pɣi ci ŋu, ɯ-mtsioʁ ɲaʁ, ɯ-jme rɲɟi tsa, phaʁzla sqaptɯɣ tɕe ju-ɣi ŋu, tɕe ``phaʁzla sqaptɯɣ tɕe mɯ-mɤ-jɤ-azɣɯt-a nɤ pɯ-si-a ra ŋu" tu-ti ɲɯ-ŋgrɤl, qusput tɤ-mbri tɕe ɯ-jme kundi ju-sɯxɕe pɯ-pɯ-ŋu nɤ, ɣɯjpa taχpa nɤkɤro kɤ-ti ɲɯ-ŋu, qusput tɤ-mbri tɕe ɯ-jme taʁki tu-sɯ-khɤt pɯ-pɯ-ŋu nɤ taχpa wuma ʑo pe tu-kɯ-ti ɲɯ-ŋgrɤl ma kundi ku-ɕe tɕe tɤtɤɣ ɯ-mŋu ɯ-fsu ma me tu-kɯ-ti ɲɯ-ŋu, taʁki tu-βze pɯ-pɯ-ŋu nɤ tɤtɤɣ pjɯ-ɣnde ŋu tu-kɯ-ti ɲɯ-ŋu.}\hspace{5pt}\pcmn{杜鹃是一种鸟,长得不大,身子是灰色的,嘴是黑色的,尾巴有点长。它五月十一号就到(开始叫),据说“如果我五月十一号还没有到的话,那就是我死了的兆头”。杜鹃叫的时候,如果把尾巴左右摆动,今年的庄稼收成基本可以,如果是上下摆动的话今年的收成特别好,因为如果尾巴左右摆动的话就说明粮食只能装到柜子的口边,如果是上下摆动的话,就说明它在把柜子里的粮食捶得很紧(粮食柜很满)。}\end{exemple}\end{entrée}

\begin{entrée}{qusputmbro}{}{ⓔqusputmbro} 
\classe{n} 
\begin{définition}\pfra{libellule}\end{définition}
\begin{définition}\pcmn{蜻蜓}\end{définition}\end{entrée}

\begin{entrée}{qɯmdroŋ}{}{ⓔqɯmdroŋ} 
\classe{n} 
\begin{définition}\pfra{oie sauvage}\end{définition}
\begin{définition}\pcmn{大雁}\end{définition}
\begin{exemple}\pjya{qɯmdroŋ nɯ pɣa wuma ʑo kɯ-wxti, ɯ-mke kɯ-zɯ-zri ci ŋu, ɯ-mdoʁ kɯ-pɣi tsa ŋu, ɯ-ro ra kɯ-wɣrum ɣɤʑu. qartsɯ tɕe athi pɕoʁ chɯ-ɕe-nɯ, ftɕar tɕe alo mbroχpa pɕoʁ lu-ɕe-nɯ ɲɯ-ŋu, tɕe tɯtɯrca ʁɟa ʑo ɣnɤsqi ro jamar tu-ŋke-nɯ ɲɯ-ŋu. tɯ-tɯ-rdoʁ a-nɯ-nɯ-βde tɕe, nɯ pjɯ-nɯkɯlu tɕe kɤ-nɯɕe mɯ-ɲɯ-cha tɕe ɯ-zda nɯ kɯ kɤ-ɕar ɯ-xɕɤt kɯ ɯ-ʁar ndzom tu-rɤspɯ ʑo ɲɯ-ŋgrɤl.}\hspace{5pt}\pcmn{大雁是一种比较大的鸟,颈很长,带有灰色,胸部是白色的。冬天,它们飞往南方,夏天就飞往北方的牧区,二十多只成群飞行。如果有只掉队了,迷失了方向,就回不去了,它的伴侣会用尽全力地找它,一直到翅膀化脓了还在找。}\end{exemple}\end{entrée}

\begin{entrée}{qɯqli}{}{ⓔqɯqli} 
\classe{idph.2} 
\begin{définition}\pfra{qui a les yeux grand ouverts}\end{définition}
\begin{définition}\pcmn{形容眼睛睁得很大的样子}\end{définition}
\begin{exemple}\pjya{qala kɯ ɯ-mɲaʁ qɯqli ʑo to-stu}\hspace{5pt}\pcmn{兔子把眼睛睁得很大}\end{exemple}
\begin{exemple}\pjya{jiɕqha nɯ kɯ ɯ-mɲaʁ qɯqli ʑo to-stu}\hspace{5pt}\pcmn{他把眼睛睁得很大}\end{exemple}\relationsémantique{参考}{\lien{ⓔɴɢɯɴɢli}{ɴɢɯɴɢli}}\end{entrée}

\newpage\caractère{r}

\begin{entrée}{ru}{₁}{ⓔruⓗ1} 
\classe{vi} \paradigme{dir}{\_}\paradigme{case}{ɯ-ɕki}\paradigme{dir}{\_}
\begin{définition}\pfra{regarder}\end{définition}
\begin{définition}\pcmn{看}\end{définition}
\begin{définition}\pfra{laisser regarder, faire voir, montrer}\end{définition}
\begin{définition}\pcmn{让人看}\end{définition}
\begin{exemple}\pjya{a-tɤ-lu ɯ-jo-ɣɯt kɯ ɕ-pjɯ-ru-a}\hspace{5pt}\pcmn{我要去看牛奶送来了没有}\end{exemple}
\begin{exemple}\pjya{a-ɕki jɤ-ru}\hspace{5pt}\pcmn{你往我这边看}\end{exemple}
\begin{exemple}\pjya{tɯrme nɯra z-jɤ-ru}\hspace{5pt}\pcmn{你看一下这些人}\end{exemple}
\begin{exemple}\pjya{tɯrme ra nɯ-ɕki z-jɤ-ru}\hspace{5pt}\pcmn{你看一下这些人}\end{exemple}
\begin{exemple}\pjya{ɯ-jme ɯ-pɕoʁ jɤ-ru}\hspace{5pt}\pcmn{他颠倒着睡(头朝着脚的方向)}\end{exemple}
\begin{exemple}\pjya{ci a-mɤ-ɕ-tɤ-ru tɕe, ci ntsɯ tu-dɤn pjɤ-ŋgrɤl}\hspace{5pt}\pcmn{只要少看一次,就会变得多一点}\end{exemple}
\begin{exemple}\pjya{mɯ-jɤ-sɯɣru-t-a}\hspace{5pt}\pcmn{我没有让他看}\end{exemple}\relationsémantique{参考}{\lien{ⓔnɯpɕɯru}{nɯpɕɯru}}\relationsémantique{参考}{\lien{ⓔanɯrŋɤrɯru}{anɯrŋɤrɯru}}
\begin{sous-entrée}{sɯɣru}{ⓔruⓗ1ⓝsɯɣru} 
\classe{vt}  
\grammaire{caus} \end{sous-entrée}

\end{entrée}

\begin{entrée}{ru}{₂}{ⓔruⓗ2} 
\classe{vt} \paradigme{dir}{\_}
\begin{définition}\pfra{amener}\end{définition}
\begin{définition}\pcmn{带}\end{définition}
\begin{exemple}\pjya{si ɕ-pɯ-asɯ-ru-a}\hspace{5pt}\pcmn{(前一段时间)我背柴了回来}\end{exemple}
\begin{exemple}\pjya{a-zrɯɣ pɯ-re ra}\hspace{5pt}\pcmn{你帮我找虱子吧}\end{exemple}
\begin{exemple}\pjya{nɤ-ɕki jɯɣi ki ɣɯ-tɤ-ru-t-a}\hspace{5pt}\pcmn{我从你那里把书拿上来了}\end{exemple}
\begin{exemple}\pjya{a-tɤ-lu jo-ɣɯt tɕe ɕ-tɤ-ru-t-a}\hspace{5pt}\pcmn{牛奶带来了,我把它拿上来了}\end{exemple}
\begin{sous-entrée}{nɯru}{ⓔruⓗ2ⓝnɯru}
\begin{exemple}\pjya{ʑ-lɤ-nɯ-ru-t-a}\hspace{5pt}\pcmn{我自己去拿来}\end{exemple}\end{sous-entrée}

\end{entrée}

\begin{entrée}{ru}{₃}{ⓔruⓗ3} 
\classe{vt} \paradigme{dir}{pɯ-}\paradigme{dir}{pɯ-}
\begin{définition}\pfra{prédire l'avenir}\end{définition}
\begin{définition}\pcmn{算命}\end{définition}
\begin{définition}\pfra{demander de regarder l'avenir}\end{définition}
\begin{définition}\pcmn{请人算命}\end{définition}
\begin{exemple}\pjya{mphrɯmɯ pɯ-re}\hspace{5pt}\pcmn{你算命吧}\end{exemple}
\begin{exemple}\pjya{mphrɯmɯ pɯ-ru-t-a}\hspace{5pt}\pcmn{我算了命}\end{exemple}
\begin{exemple}\pjya{βlama nɯ-sqar-a tɕe mphrɯmɯ pa-ru}\hspace{5pt}\pcmn{我请了喇嘛算命}\end{exemple}
\begin{exemple}\pjya{mphrɯmɯ ɕ-pɯ-sɯre}\hspace{5pt}\pcmn{你去请人算命吧}\end{exemple}
\begin{exemple}\pjya{aʑɯɣ kɯnɤ a-mphrɯmɯ a-pɯ-tɯ-sɯre ɯ-tɯ́-cha}\hspace{5pt}\pcmn{你能不能叫别人帮我算命}\end{exemple}
\begin{sous-entrée}{sɯru}{ⓔruⓗ3ⓝsɯru} 
\classe{vt} \end{sous-entrée}

\end{entrée}

\begin{entrée}{ra}{₂}{ⓔraⓗ2} 
\classe{det} 
\begin{définition}\pfra{pluriel}\end{définition}
\begin{définition}\pcmn{复数}\end{définition}\end{entrée}

\begin{entrée}{ra}{₁}{ⓔraⓗ1} 
\classe{vs} \sens{1}
\begin{définition}\pfra{devoir}\end{définition}
\begin{définition}\pcmn{应该}\end{définition}
\begin{exemple}\pjya{kɯ-ɣɯsɯphɯt ɕe-a ra}\hspace{5pt}\pcmn{我要去砍柴}\end{exemple}\sens{2}
\begin{définition}\pfra{avoir besoin de}\end{définition}
\begin{définition}\pcmn{需要}\end{définition}
\begin{exemple}\pjya{nɤʑɯɣ tɕhi ra?}\hspace{5pt}\pcmn{你需要什么?}\end{exemple}
\begin{exemple}\pjya{aʑɯɣ kɯ-ra me}\hspace{5pt}\pcmn{我什么也不需要}\end{exemple}
\begin{exemple}\pjya{mɤ-tɯ-ra}\hspace{5pt}\pcmn{不需要你}\end{exemple}\relationsémantique{参考}{\lien{ⓔɣɤra}{ɣɤra}}
\begin{sous-entrée}{mɤra ma}{ⓔraⓗ1ⓢ2ⓝmɤra ma} 
\classe{cnj} 
\begin{définition}\pfra{non seulement}\end{définition}
\begin{définition}\pcmn{不但}\end{définition}
\begin{exemple}\pjya{jisŋi tɤ-rɯndzɤtshi-a tɕe tɯmgo nɯ mɤra ma tɤjko kɯnɤ tɤ-ndza-t-a}\hspace{5pt}\pcmn{我今天不但吃了饭,还吃了酸菜}\end{exemple}\end{sous-entrée}

\end{entrée}

\begin{entrée}{raka}{}{ⓔraka} 
\classe{n} 
\begin{définition}\pfra{façon dont poussent les cornes (animaux)}\end{définition}
\begin{définition}\pcmn{长势(角)}\end{définition}
\begin{exemple}\pjya{kɯki jla ki ɯ-raka wuma ʑo ɲɯ-βdi}\hspace{5pt}\pcmn{这头犏牛的角长得很好}\end{exemple}\end{entrée}

\begin{entrée}{ralarala}{}{ⓔralarala} 
\classe{adv} 
\begin{définition}\pfra{très rapide}\end{définition}
\begin{définition}\pcmn{飞快}\end{définition}\end{entrée}

\begin{entrée}{rambɯm}{}{ⓔrambɯm} 
\classe{n} 
\begin{définition}\pfra{ramure de cerf}\end{définition}
\begin{définition}\pcmn{雄鹿的角}\end{définition}\étymologie{rwa.ⁿbum}\end{entrée}

\begin{entrée}{raŋ}{₂}{ⓔraŋⓗ2} 
\classe{n} 
\begin{définition}\pfra{soi-même}\end{définition}
\begin{définition}\pcmn{自己}\end{définition}
\begin{exemple}\pjya{nɤʑo tu-tɯ-ti mɤ-ra ma aʑo raŋ tu-nɯ-ti-a jɤɣ}\hspace{5pt}\pcmn{不用你说,我自己说}\end{exemple}
\begin{exemple}\pjya{kɯki laχtɕha ki aʑo raŋ ɣɯ pɯ-kɯ-nɯ-tɯ-tu ɕti}\hspace{5pt}\pcmn{这个东西是我自己曾经有过的}\end{exemple}\étymologie{raŋ}\end{entrée}

\begin{entrée}{raŋ}{₁}{ⓔraŋⓗ1} 
\classe{vs} \paradigme{dir}{tɤ-}
\begin{définition}\pfra{long (temps)}\end{définition}
\begin{définition}\pcmn{长(时间,路程)}\end{définition}
\begin{exemple}\pjya{ki ɲɯ-fse nɯ ɲɯ-raŋ}\hspace{5pt}\pcmn{这样、那样}\end{exemple}
\begin{exemple}\pjya{mɤ-kɯ-fse mɤ-kɯ-raŋ ʑo maŋe}\hspace{5pt}\pcmn{有很多种现象,预料不到的现象都有}\end{exemple}\étymologie{riŋ}\end{entrée}

\begin{entrée}{raŋri}{}{ⓔraŋri} 
\classe{adv} 
\begin{définition}\pfra{chacun}\end{définition}
\begin{définition}\pcmn{每个}\end{définition}
\begin{exemple}\pjya{kɤndza thɯ́-wɣ-kro tɕe, tɯrme raŋri ɣɯ ɲɯ́-wɣ-kho ra}\hspace{5pt}\pcmn{分食物的时候,要分给每一个人}\end{exemple}
\begin{exemple}\pjya{tɯrme raŋri ɣɯ nɯ-mɲaʁ tu ɕti}\hspace{5pt}\pcmn{每个人有眼睛}\end{exemple}
\begin{sous-entrée}{rɯri}{ⓔraŋriⓝrɯri} 
\classe{adv} 
\begin{définition}\pfra{chacun}\end{définition}
\begin{définition}\pcmn{每个}\end{définition}\end{sous-entrée}

\étymologie{raŋ.re}\étymologie{re.re}\end{entrée}

\begin{entrée}{raŋɯŋi}{}{ⓔraŋɯŋi} 
\classe{idph.7} 
\begin{définition}\pfra{qui pend (substance visqueuse)}\end{définition}
\begin{définition}\pcmn{形容(黏稠的物质,如鼻涕)吊着的样子}\end{définition}
\begin{exemple}\pjya{nɤɕnaβ raŋɯŋi ʑo pɯ-nɯɬoʁ}\hspace{5pt}\pcmn{你的鼻涕流出来了}\end{exemple}
\begin{exemple}\pjya{ɲɯ-nɯtɕhomba tɕe, ɯ-ɕnɤmtsrɯɣ raŋɯŋi ɲɯ-ɤsɯ-stu}\hspace{5pt}\pcmn{他感冒了就吊着鼻涕}\end{exemple}
\begin{exemple}\pjya{tɤjko pjɤ-tɕur tɕe, raŋɯŋi ɲɯ-pa}\hspace{5pt}\pcmn{酸菜很酸,吊着有黏性的液体}\end{exemple}\étymologie{raŋ}\end{entrée}

\begin{entrée}{raʁ}{₂}{ⓔraʁⓗ2} 
\classe{n} 
\begin{définition}\pfra{laiton}\end{définition}
\begin{définition}\pcmn{黄铜}\end{définition}\étymologie{rag}\end{entrée}

\begin{entrée}{raʁ}{₁}{ⓔraʁⓗ1} 
\classe{vi} \sens{1}\paradigme{dir}{kɤ-}\paradigme{dir}{thɯ-}
\begin{définition}\pfra{être bloqué}\end{définition}
\begin{définition}\pcmn{卡住}\end{définition}
\begin{exemple}\pjya{tɤtshoʁ ko-raʁ}\hspace{5pt}\pcmn{钉子卡住了(取不出来)}\end{exemple}
\begin{exemple}\pjya{tʂhazwa a-rqo thɯ-raʁ}\hspace{5pt}\pcmn{茶叶卡在我的喉咙里了}\end{exemple}
\begin{exemple}\pjya{kɯ-spoʁ ɯ-ŋgɯ cho-raʁ}\hspace{5pt}\pcmn{在洞里卡住了}\end{exemple}\sens{2}\paradigme{dir}{nɯ-}
\begin{définition}\pfra{s’accrocher}\end{définition}
\begin{définition}\pcmn{钩住}\end{définition}
\begin{exemple}\pjya{a-ŋga nɯ-raʁ}\hspace{5pt}\pcmn{我的衣服被钩住了}\end{exemple}
\begin{sous-entrée}{sɯɣraʁ}{ⓔraʁⓗ1ⓢ2ⓝsɯɣraʁ} 
\classe{vt} 
\begin{définition}\pfra{accrocher}\end{définition}
\begin{définition}\pcmn{卡住;钩住}\end{définition}\end{sous-entrée}

\end{entrée}

\begin{entrée}{raʁdoŋ}{}{ⓔraʁdoŋ} 
\classe{n} 
\begin{définition}\pfra{trompette}\end{définition}
\begin{définition}\pcmn{长号角}\end{définition}\étymologie{rag.duŋ}\end{entrée}

\begin{entrée}{raʁjɯ}{}{ⓔraʁjɯ} 
\classe{vi} \paradigme{dir}{nɯ-}
\begin{définition}\pfra{faire semblant d'être incapable (pour ne pas avoir à travailler)}\end{définition}
\begin{définition}\pcmn{装作自己不会做}\end{définition}
\begin{exemple}\pjya{ma-nɯ-tɯ-raʁjɯ kɯ nɯ-rɤma}\hspace{5pt}\pcmn{你不要装作自己不会,要劳动!}\end{exemple}\relationsémantique{同义词}{\lien{ⓔnɯɕɯʁjɯ}{nɯɕɯʁjɯ}}\end{entrée}

\begin{entrée}{raʁle}{}{ⓔraʁle} 
\classe{vi} \paradigme{dir}{tɤ-}
\begin{définition}\pfra{être poli}\end{définition}
\begin{définition}\pcmn{客气}\end{définition}
\begin{exemple}\pjya{tɤ-raʁle-a tɕe kɯ-dɤn mɯ-tɤ-ndza-t-a}\hspace{5pt}\pcmn{我为了表现客气,没有吃很多}\end{exemple}
\begin{exemple}\pjya{ma-tɤ-tɯ-raʁle je}\hspace{5pt}\pcmn{不用客气}\end{exemple}\relationsémantique{参考}{\lien{ⓔɯ-ʁle}{ɯ-ʁle}}\end{entrée}

\begin{entrée}{raʁmaʁ}{}{ⓔraʁmaʁ} 
\classe{part} 
\begin{définition}\pfra{d'accord ?}\end{définition}
\begin{définition}\pcmn{好吗?}\end{définition}
\begin{exemple}\pjya{nɤ-ŋga tɤ-ŋge raʁmaʁ ma ɲɯ-ɣɤndʐo}\hspace{5pt}\pcmn{你多穿点衣服好吗,天气很冷。}\end{exemple}\end{entrée}

\begin{entrée}{raʁraʁ}{}{ⓔraʁraʁ} 
\classe{idph.2} 
\begin{définition}\pfra{noir comme du charbon}\end{définition}
\begin{définition}\pcmn{形容极黑}\end{définition}
\begin{exemple}\pjya{qaʑo kɯ ɯ-pɯ kɯ-ɲaʁ raʁraʁ ʑo ɲɤ-lɤt}\hspace{5pt}\pcmn{绵羊下了黑黢黢的小羊羔}\end{exemple}\end{entrée}

\begin{entrée}{raʁrɤt}{}{ⓔraʁrɤt} 
\classe{vi}  
\grammaire{denom} \paradigme{dir}{pɯ-}
\begin{définition}\pfra{faire du charbon}\end{définition}
\begin{définition}\pcmn{烧木炭}\end{définition}\relationsémantique{参考}{\lien{ⓔta-ʁrɤt}{ta-ʁrɤt}}\end{entrée}

\begin{entrée}{raʁrɯz}{}{ⓔraʁrɯz} 
\classe{vt} \paradigme{dir}{thɯ-}\paradigme{dir}{tɤ-}\paradigme{dir}{thɯ-}
\begin{définition}\pfra{balayer}\end{définition}
\begin{définition}\pcmn{扫}\end{définition}
\begin{définition}\pfra{changer les affaire et balayer le sol}\end{définition}
\begin{définition}\pcmn{收拾和扫地}\end{définition}
\begin{définition}\pfra{balayer avec}\end{définition}
\begin{définition}\pcmn{用……扫}\end{définition}
\begin{exemple}\pjya{ɯ-thoʁ ko-ɴqhi tɕe thɯ-raʁrɯz}\hspace{5pt}\pcmn{地变脏了,你扫一下吧}\end{exemple}
\begin{exemple}\pjya{kha ra ɲɯ-ɤdrɤt tɕe tɤ-rɤroʁrɯz-a}\hspace{5pt}\pcmn{家里很乱,我又收拾了,又扫了地}\end{exemple}
\begin{exemple}\pjya{zɣɤmbu kɯ thɯ-z-raʁrɯz-a}\hspace{5pt}\pcmn{我用扫把扫了}\end{exemple}\relationsémantique{参考}{\lien{ⓔroʁrɯz}{roʁrɯz}}
\begin{sous-entrée}{rɤroʁrɯz}{ⓔraʁrɯzⓝrɤroʁrɯz} 
\classe{vi}  
\grammaire{apass}
\grammaire{denom} \end{sous-entrée}

\begin{sous-entrée}{zraʁrɯz}{ⓔraʁrɯzⓝzraʁrɯz} 
\classe{vt}  
\grammaire{caus} \end{sous-entrée}

\end{entrée}

\begin{entrée}{rasqaβ}{}{ⓔrasqaβ} 
\classe{n} 
\begin{définition}\pfra{aiguille à coudre}\end{définition}
\begin{définition}\pcmn{缝衣针}\end{définition}\relationsémantique{参考}{\lien{ⓔtaqaβ}{taqaβ}}\end{entrée}

\begin{entrée}{rasti}{}{ⓔrasti} 
\classe{n} 
\begin{définition}\pfra{navet (Brassica sp.)}\end{définition}
\begin{définition}\pcmn{芜菁【圆根】}\end{définition}
\begin{exemple}\pjya{rasti nɯ zgoku tsa tu-ɬoʁ cha. ɯ-mdoʁ nɯ kɯ-pɣi tsa ŋu. ɯ-qa nɯ kɯ-ɤrtɯm tɕe, kɯ-wxti ʑo ɲɯ-βze cha. stu kɯ-wxti nɯ sqamŋu spaprɤɣ ɯ-tɯrpa jamar ɲɯ-βze cha. ɯ-taʁ tɕe ɯ-sku nɯ kɯ-dɯ-dɤn tu-ɬoʁ cha, ɯ-pɕi nɯ pjɯ́-wɣ-nɯ-phɯt tɕe, ɯ-ŋgɯ mɤʑɯ ɲɯ-βze ɕti, tɕe nɯ pjɯ-kɯ-phɯt nɤ, ɲɯ-kɤ-nɤrqaʁ rmi. nɯ ɯ-ku nɯ rasti rmi, ɯ-qa nɯ rɤjndoʁ rmi. rasti nɯ kú-wɣ-sqa tɕe, kɤ-smi tɕe, pjɯ́-wɣ-sɯɣ-tɕur tɕe, tɤ-jko rmi, tɕe tɯ-mgo zmɤrɤβ wuma ʑo pe. tɯrme kɤ-ndza kɯmɯxte ʑo rga-nɯ. ɯ-jndoʁ nɯ paʁndza wuma ʑɤ pe. kú-wɣ-sqa tɕe, kɤ-smi tɕe, tɯrme kɯnɤ tú-wɣ-ndza ŋgrɤl. ɯ-jndoʁ kɤ-kɤ-sqa nɯ rɤsqa rmi.}\hspace{5pt}\pcmn{圆根生长在比较高的山地里,有点灰色,它的根是圆形的,长得很大。最大可以长到15、16斤。上面长很多苗,外层的苗拔下后,里面的苗会继续长。这种拔苗的方法叫\lien{}{kɤ-nɤrqaʁ}。苗叫\lien{ⓔrasti}{rasti},根叫\lien{ⓔrɤjndoʁ}{rɤjndoʁ}。圆根苗煮熟了以后弄酸了,叫酸菜,是很好的菜。大多数人喜欢吃。圆根是喂猪的好饲料。煮熟后,人也可以吃。煮熟的圆根叫\lien{ⓔrɤsqa}{rɤsqa}。}\end{exemple}\relationsémantique{同义词}{\lien{ⓔtɤjkɤspa}{tɤjkɤspa}}\end{entrée}

\begin{entrée}{raχpi}{}{ⓔraχpi} 
\classe{vi}  
\grammaire{denom} \paradigme{dir}{tɤ-}
\begin{définition}\pfra{raconter}\end{définition}
\begin{définition}\pcmn{陈述}\end{définition}\relationsémantique{参考}{\lien{ⓔχpi}{χpi}}\end{entrée}

\begin{entrée}{raχtɕɤz}{}{ⓔraχtɕɤz} 
\classe{vt} \paradigme{dir}{tɤ-}
\begin{définition}\pfra{bien traiter, accueillir chaleureusement}\end{définition}
\begin{définition}\pcmn{款待}\end{définition}
\begin{exemple}\pjya{jɯfɕɯr kɯ-mɯm to-sɤftɕaka tɤ́-wɣ-raχtɕaz-a}\hspace{5pt}\pcmn{他昨天做了好吃的,款待了我}\end{exemple}
\begin{exemple}\pjya{ndʐuwa ta-raχtɕɤz}\hspace{5pt}\pcmn{他款待了客人}\end{exemple}
\begin{sous-entrée}{sɤraχtɕɤz}{ⓔraχtɕɤzⓝsɤraχtɕɤz} 
\classe{vi}  
\grammaire{apass} 
\begin{définition}\pfra{bien traiter les gens}\end{définition}
\begin{définition}\pcmn{关心别人}\end{définition}
\begin{exemple}\pjya{nɤki smɤnba nɯ ɲɯ-sɤraχtɕɤz}\hspace{5pt}\pcmn{那个医生关心别人}\end{exemple}
\begin{exemple}\pjya{nɤki tɤ-mu nɯ kɯ-sɤraχtɕɤz ci ŋu}\hspace{5pt}\pcmn{那位大娘是关心别人的}\end{exemple}\end{sous-entrée}

\begin{sous-entrée}{ʑɣɤraχtɕɤz}{ⓔraχtɕɤzⓝʑɣɤraχtɕɤz} 
\classe{vi}  
\grammaire{refl} 
\begin{définition}\pfra{bien se traiter soi-même}\end{définition}
\begin{définition}\pcmn{优待自己}\end{définition}\end{sous-entrée}

\end{entrée}

\begin{entrée}{raχtɕi}{}{ⓔraχtɕi}\relationsémantique{参考}{\lien{ⓔχtɕi}{χtɕi}}\end{entrée}

\begin{entrée}{raχtɕoŋ}{}{ⓔraχtɕoŋ} 
\classe{n} 
\begin{définition}\pfra{bovidé de couleur noire}\end{définition}
\begin{définition}\pcmn{黑色的牛}\end{définition}\étymologie{rwa.tɕʰuŋ?}\end{entrée}

\begin{entrée}{raχtɕɯmχtɕɤz}{}{ⓔraχtɕɯmχtɕɤz} 
\classe{vt} 
\begin{définition}\pfra{bien traiter}\end{définition}
\begin{définition}\pcmn{优待}\end{définition}
\begin{exemple}\pjya{nɤʑo ɲɯ-tɯ-raχtɕɯmχtɕɤz}\hspace{5pt}\pcmn{你优待他}\end{exemple}
\begin{exemple}\pjya{ɲɯ-tɯ́-wɣ-raχtɕɯmχtɕɤz}\hspace{5pt}\pcmn{他优待你}\end{exemple}\end{entrée}

\begin{entrée}{raχtɕɯʁɟo/\variante{rɤχtɕɯʁɟo}}{}{ⓔraχtɕɯʁɟo} 
\classe{vi} \paradigme{dir}{pɯ-}
\begin{définition}\pfra{laver}\end{définition}
\begin{définition}\pcmn{洗澡}\end{définition}
\begin{exemple}\pjya{paʁ ra cɯβloʁ ɯ-ŋgɯ pɯ-rɤχtɕɯʁɟo-nɯ}\hspace{5pt}\pcmn{猪在池塘里洗澡了}\end{exemple}\relationsémantique{参考}{\lien{ⓔχtɕi}{χtɕi}}\end{entrée}

\begin{entrée}{raχtsur}{}{ⓔraχtsur}\relationsémantique{参考}{\lien{ⓔχtsur}{χtsur}}\end{entrée}

\begin{entrée}{raχtɯ}{}{ⓔraχtɯ}\relationsémantique{参考}{\lien{ⓔχtɯ}{χtɯ}}\end{entrée}

\begin{entrée}{raχtɯtsɣe}{}{ⓔraχtɯtsɣe} 
\classe{vi}  
\grammaire{comp} \paradigme{dir}{thɯ-}
\begin{définition}\pfra{faire du commerce}\end{définition}
\begin{définition}\pcmn{做生意}\end{définition}\relationsémantique{参考}{\lien{ⓔχtɯ}{χtɯ}}\relationsémantique{参考}{\lien{ⓔntsɣe}{ntsɣe}}\end{entrée}

\begin{entrée}{raz}{}{ⓔraz} 
\classe{n} 
\begin{définition}\pfra{tissu}\end{définition}
\begin{définition}\pcmn{布}\end{définition}\étymologie{ras}\end{entrée}

\begin{entrée}{razmbe}{}{ⓔrazmbe} 
\classe{n} 
\begin{définition}\pfra{vieux tissu}\end{définition}
\begin{définition}\pcmn{旧的布料}\end{définition}\end{entrée}

\begin{entrée}{razɴɢu}{}{ⓔrazɴɢu} 
\classe{n} 
\begin{définition}\pfra{espèce d'arbrisseau}\end{définition}
\begin{définition}\pcmn{灌木的一种}\end{définition}
\begin{exemple}\pjya{razɴɢu nɯ si kɯ-mbɯ-mbɤr ci ŋu, rdɤstaʁ cho praʁ ɯ-taʁ jɯ-ʑɣɤɲɟoʁ tɕe, tu-ɬoʁ ŋu. ɯ-jwaʁ cho ɯ-ru nɯ ra ɯ-mat nɯ ra tɤ-ru cho aɣɯmdoʁ, ɯ-mat cho ɯ-jwaʁ ra fsapaʁ ra kɯ tu-ndza-nɯ ŋgrɤl}\hspace{5pt}\pcmn{\lien{ⓔrazɴɢu}{razɴɢu} 是一种非常矮小的树,爬在石头和岩石上。叶子、干和果实颜色和火棘一样,牲畜都吃它的果实和叶子。}\end{exemple}\end{entrée}

\begin{entrée}{razri}{}{ⓔrazri} 
\classe{n} 
\begin{définition}\pfra{fil en coton}\end{définition}
\begin{définition}\pcmn{棉线}\end{définition}\relationsémantique{参考}{\lien{ⓔtɤ-ri}{tɤ-ri}}\relationsémantique{参考}{\lien{ⓔraz}{raz}}\end{entrée}

\begin{entrée}{rɤβraʁ}{}{ⓔrɤβraʁ} 
\classe{vt} \sens{1}\paradigme{dir}{nɯ-}
\begin{exemple}\pjya{a-mgɯr nɯ-rɤβraʁ}\hspace{5pt}\pcmn{你抠我的背部吧}\end{exemple}
\begin{exemple}\pjya{nɯ-nɯ-rɤβraʁ-a}\hspace{5pt}\pcmn{我抠了一下痒}\end{exemple}
\begin{exemple}\pjya{ma-nɯ-tɯ-rɤβraʁ}\hspace{5pt}\pcmn{你不要抓痒!}\end{exemple}
\begin{exemple}\pjya{paʁ na-rɤβraʁ}\hspace{5pt}\pcmn{他给猪抓痒了}\end{exemple}
\begin{exemple}\pjya{paʁ ɲɯ́-wɣ-rɤβraʁ tɕe rga}\hspace{5pt}\pcmn{猪很喜欢抓痒}\end{exemple}\sens{2}\paradigme{dir}{pɯ-}
\begin{définition}\pfra{ratisser}\end{définition}
\begin{définition}\pcmn{耙}\end{définition}
\begin{exemple}\pjya{sɤtɕha ra pa-rɤβraʁ}\hspace{5pt}\pcmn{他耙了地}\end{exemple}
\begin{sous-entrée}{zrɤβraʁ}{ⓔrɤβraʁⓢ2ⓝzrɤβraʁ} 
\classe{vt}  
\grammaire{caus} \sens{1}\paradigme{dir}{nɯ-}
\begin{définition}\pfra{gratter avec}\end{définition}
\begin{définition}\pcmn{用……抠}\end{définition}
\begin{exemple}\pjya{tɤ-lu ɯ-sta kɯ-khrɯ ɲɯ-ŋu tɕe, kɤ-χtɕi ɲɯ-ɴqa, tɕe a-ndzrɯ kɯ nɯ-zrɤβraʁ-a ʑo pɯ-ra}\hspace{5pt}\pcmn{装过牛奶的瓶子里面干硬了,很难洗,我只好用指甲抠掉}\end{exemple}\end{sous-entrée}

\sens{2}\paradigme{dir}{pɯ-}
\begin{définition}\pfra{ratisser avec}\end{définition}
\begin{définition}\pcmn{用……耙}\end{définition}
\begin{exemple}\pjya{qarɤt kɯ pɯ-zrɤβraʁ-a}\hspace{5pt}\pcmn{我用耙子耙了}\end{exemple}
\begin{sous-entrée}{ʑɣɤrɤβraʁ}{ⓔrɤβraʁⓢ2ⓝʑɣɤrɤβraʁ} 
\classe{vi}  
\grammaire{refl} 
\begin{définition}\pfra{se gratter}\end{définition}
\begin{définition}\pcmn{给自己抓痒}\end{définition}
\begin{exemple}\pjya{paʁ khrɯɣnɤkhrɯɣ ɲɯ-ʑɣɤrɤβraʁ}\hspace{5pt}\pcmn{猪在抓痒}\end{exemple}\end{sous-entrée}

\end{entrée}

\begin{entrée}{rɤβʁa/\variante{rɤβʁɯʁa}}{}{ⓔrɤβʁa} 
\classe{vi} \paradigme{dir}{tɤ-}
\begin{définition}\pfra{rugir}\end{définition}
\begin{définition}\pcmn{(猫、豹子)吼叫}\end{définition}
\begin{exemple}\pjya{lɯlu ɲɯ-rɤβʁa}\hspace{5pt}\pcmn{猫在吼叫}\end{exemple}
\begin{exemple}\pjya{kɯrtsɤɣ ɲɯ-rɤβʁa}\hspace{5pt}\pcmn{豹子在吼叫}\end{exemple}
\begin{exemple}\pjya{mbala ɲɯ-rɤβʁa}\hspace{5pt}\pcmn{公牛在吼叫}\end{exemple}\end{entrée}

\begin{entrée}{rɤβzjoz}{}{ⓔrɤβzjoz} 
\classe{vi}  
\grammaire{apass} \paradigme{dir}{pɯ-}
\begin{définition}\pfra{faire des études}\end{définition}
\begin{définition}\pcmn{读书}\end{définition}
\begin{exemple}\pjya{pɯ-rɤβzjoz-a}\hspace{5pt}\pcmn{我读书了}\end{exemple}\relationsémantique{参考}{\lien{ⓔβzjoz}{βzjoz}}
\begin{sous-entrée}{zrɤβzjoz}{ⓔrɤβzjozⓝzrɤβzjoz} 
\classe{vt}  
\grammaire{apass}
\grammaire{caus} 
\begin{définition}\pfra{faire étudier}\end{définition}
\begin{définition}\pcmn{让……读书}\end{définition}\end{sous-entrée}

\end{entrée}

\begin{entrée}{rɤcɤβ}{}{ⓔrɤcɤβ} 
\classe{vi} \paradigme{dir}{thɯ-}
\begin{définition}\pfra{pousser des cosses}\end{définition}
\begin{définition}\pcmn{结荚果}\end{définition}\relationsémantique{参考}{\lien{ⓔɯ-cɤβ}{ɯ-cɤβ}}\end{entrée}

\begin{entrée}{rɤɕar}{}{ⓔrɤɕar}\relationsémantique{参考}{\lien{ⓔɕar}{ɕar}}\end{entrée}

\begin{entrée}{rɤɕi/\variante{rɤɕit}}{}{ⓔrɤɕi} 
\classe{vt} \paradigme{dir}{\_}
\begin{définition}\pfra{tirer}\end{définition}
\begin{définition}\pcmn{拉}\end{définition}
\begin{exemple}\pjya{tɯmbri kɤ-rɤɕi}\hspace{5pt}\pcmn{你拉一下绳子吧}\end{exemple}
\begin{exemple}\pjya{a-jaʁ kɤ-rɤɕi}\hspace{5pt}\pcmn{你拉我的手吧}\end{exemple}
\begin{exemple}\pjya{nɯ-rɤɕi-t-a}\hspace{5pt}\pcmn{我拉了}\end{exemple}
\begin{exemple}\pjya{tɯrɟaʁ kɯ-rɲɟɯ-rɲɟi ʑo ko-rɤɕi-nɯ (ko-mtshi-nɯ)}\hspace{5pt}\pcmn{他们跳舞的队伍拉得很长}\end{exemple}
\begin{sous-entrée}{ʑɣɤrɤɕi}{ⓔrɤɕiⓝʑɣɤrɤɕi} 
\classe{vi}  
\grammaire{refl} 
\begin{définition}\pfra{se défaire de}\end{définition}
\begin{définition}\pcmn{摆脱,挣脱}\end{définition}
\begin{exemple}\pjya{a-jaʁ ka-ndo ri, kɤ-ʑɣɤrɤɕi-a tɕe kɤ-sɯɕlɯɣ-a}\hspace{5pt}\pcmn{他抓住了我的手,我就挣脱了,让他松手了}\end{exemple}\end{sous-entrée}

\end{entrée}

\begin{entrée}{rɤɕkho}{}{ⓔrɤɕkho}\relationsémantique{参考}{\lien{ⓔɕkho}{ɕkho}}\end{entrée}

\begin{entrée}{rɤɕom}{}{ⓔrɤɕom} 
\classe{vi}  
\grammaire{denom} \paradigme{dir}{kɤ-}
\begin{définition}\pfra{apparaître (peau du lait)}\end{définition}
\begin{définition}\pcmn{结奶皮}\end{définition}
\begin{exemple}\pjya{tɤ-lu ko-rɤɕom}\hspace{5pt}\pcmn{牛奶结了奶皮}\end{exemple}\relationsémantique{参考}{\lien{ⓔɕomⓗ2}{ɕom₂}}\end{entrée}

\begin{entrée}{rɤɕon}{}{ⓔrɤɕon} 
\classe{vi} \paradigme{dir}{tɤ-}
\begin{définition}\pfra{porter témoigner}\end{définition}
\begin{définition}\pcmn{作证}\end{définition}
\begin{exemple}\pjya{tu-nɯzdɯɣ mɤ-ra ma aʑo tu-rɤɕon-a jɤɣ}\hspace{5pt}\pcmn{你不用担心(他冤枉你),我可以给你作证}\end{exemple}\end{entrée}

\begin{entrée}{rɤɕphɤt}{}{ⓔrɤɕphɤt}\relationsémantique{参考}{\lien{ⓔɕphɤt}{ɕphɤt}}\end{entrée}

\begin{entrée}{rɤɕtʂat}{}{ⓔrɤɕtʂat}\relationsémantique{参考}{\lien{ⓔɕtʂat}{ɕtʂat}}\end{entrée}

\begin{entrée}{rɤɕtʂo}{}{ⓔrɤɕtʂo}\relationsémantique{参考}{\lien{ⓔɕtʂo}{ɕtʂo}}\end{entrée}

\begin{entrée}{rɤɕtʂɯ}{}{ⓔrɤɕtʂɯ}\relationsémantique{参考}{\lien{ⓔɕtʂɯ}{ɕtʂɯ}}\end{entrée}

\begin{entrée}{rɤfcɤr}{}{ⓔrɤfcɤr} 
\classe{vi}  
\grammaire{denom} \paradigme{dir}{tɤ-}
\begin{définition}\pfra{faire de la poterie}\end{définition}
\begin{définition}\pcmn{做泥工}\end{définition}
\begin{exemple}\pjya{aʑo tu-rɤfcar-a ɲɯ-sɯsam-a ma jinde kɯ-rɤfcaʁ maŋe}\hspace{5pt}\pcmn{我想做泥工因为现在没有泥匠}\end{exemple}\relationsémantique{参考}{\lien{ⓔtɯfcɤr}{tɯfcɤr}}\end{entrée}

\begin{entrée}{rɤfɕɤt}{}{ⓔrɤfɕɤt}\relationsémantique{参考}{\lien{ⓔfɕɤtⓗ1}{fɕɤt₁}}\end{entrée}

\begin{entrée}{rɤfɕɯfɕɤt}{}{ⓔrɤfɕɯfɕɤt} 
\classe{vt} \paradigme{dir}{thɯ-}
\begin{définition}\pfra{déchirer dans tous les sens}\end{définition}
\begin{définition}\pcmn{乱撕}\end{définition}
\begin{exemple}\pjya{ɯ-ŋga chɤ-rɤfɕɯfɕɤt}\hspace{5pt}\pcmn{他乱撕了他的衣服}\end{exemple}
\begin{exemple}\pjya{ɯ-@benzi chɤ-rɤfɕɯfɕɤt}\hspace{5pt}\pcmn{他乱撕了他的本子}\end{exemple}
\begin{exemple}\pjya{tha-rɤfɕɯfɕɤt}\hspace{5pt}\pcmn{他乱撕了}\end{exemple}\end{entrée}

\begin{entrée}{rɤfse}{}{ⓔrɤfse}\relationsémantique{参考}{\lien{ⓔfseⓗ2}{fse₂}}\end{entrée}

\begin{entrée}{rɤfsjit}{}{ⓔrɤfsjit} 
\classe{vi} \paradigme{dir}{thɯ-}
\begin{définition}\pfra{siffler}\end{définition}
\begin{définition}\pcmn{吹口哨}\end{définition}
\begin{exemple}\pjya{ɯʑo ci thɯ-rɤfsjit}\hspace{5pt}\pcmn{他吹了口哨}\end{exemple}\relationsémantique{参考}{\lien{ⓔtɤfsjit}{tɤfsjit}}\end{entrée}

\begin{entrée}{rɤfsoʁ}{}{ⓔrɤfsoʁ}\relationsémantique{参考}{\lien{ⓔfsoʁⓗ1}{fsoʁ₁}}\end{entrée}

\begin{entrée}{rɤftɕɤz}{}{ⓔrɤftɕɤz}\relationsémantique{参考}{\lien{ⓔftɕɤz}{ftɕɤz}}\end{entrée}

\begin{entrée}{rɤɣdɤt}{}{ⓔrɤɣdɤt} 
\classe{vt} \paradigme{dir}{pɯ-}
\begin{définition}\pfra{découper en sections}\end{définition}
\begin{définition}\pcmn{砍;锯成一段一段}\end{définition}
\begin{exemple}\pjya{si pa-rɤɣdɤt}\hspace{5pt}\pcmn{他把木头锯成了几段}\end{exemple}
\begin{exemple}\pjya{ɕoŋtɕa pa-rɤɣdɤt}\hspace{5pt}\pcmn{他把木料锯成了几段}\end{exemple}
\begin{exemple}\pjya{ɕom pɯ-rɤɣdat-a}\hspace{5pt}\pcmn{我把铁锯了几段}\end{exemple}
\begin{exemple}\pjya{tɯmbri pɯ-rɤɣdat-a}\hspace{5pt}\pcmn{我把绳子剪了几段}\end{exemple}\relationsémantique{参考}{\lien{ⓔrɤrzɯɣ}{rɤrzɯɣ}}\end{entrée}

\begin{entrée}{rɤɣdɯt}{}{ⓔrɤɣdɯt} 
\classe{vt} \paradigme{dir}{thɯ-}\paradigme{dir}{nɯ-}
\begin{définition}\pfra{écorcher}\end{définition}
\begin{définition}\pcmn{剥皮(保持完整的皮子)}\end{définition}
\begin{exemple}\pjya{thɯ-rɤɣdɯt-a, tha-rɤɣdɯt}\hspace{5pt}\pcmn{我剥了皮、他剥了皮}\end{exemple}
\begin{exemple}\pjya{qala tha-rɤɣdɯt}\hspace{5pt}\pcmn{他剥了兔子的皮}\end{exemple}
\begin{exemple}\pjya{tshɤt qaʑo tha-rɤɣdɯt}\hspace{5pt}\pcmn{他剥了羊的皮}\end{exemple}
\begin{exemple}\pjya{xɕiri kɯ kumpɣa ɲɤ-rɤɣdɯt}\hspace{5pt}\pcmn{黄鼠狼喝光了鸡的血(还没有吃到肉)}\end{exemple}
\begin{exemple}\pjya{tɯrme kɯ tu-kɤ-ntɕha nɯnɯ, ɯ-rqhu mɯ-tu-kɤ-phaʁ nɯ, ɯ-ndʐi mɯ-tu-kɤ-phaʁ nɯ tɕe, chɤ-rɤɣdɯt tu-kɯ-ti ɲɯ-ŋu. xɕiri kɯ kumpɣa ta-ndza tɕe, ɯ-se ku-tshi ma, ɯ-ɕa mɯ-tu-ndze ɲɯ-βde nɯnɯ, li ɲɤ-rɤɣdɯt tu-kɯ-ti khɯ.}\hspace{5pt}\pcmn{宰动物的时候,不破皮子地剥皮叫做\lien{}{chɤrɤɣdɯt}。黄鼠狼喝鸡的血,不吃它的肉也叫做\lien{}{ɲɤrɤɣdɯt}}\end{exemple}\end{entrée}

\begin{entrée}{rɤɣlɤn}{}{ⓔrɤɣlɤn} 
\classe{vt}  
\grammaire{denom} \paradigme{dir}{tɤ-}
\begin{définition}\pfra{prendre pour prétexte}\end{définition}
\begin{définition}\pcmn{拿为借口}\end{définition}
\begin{exemple}\pjya{nɯ aʑo tu-kɯ-rɤɣlan-a ɲɯ-ŋu}\hspace{5pt}\pcmn{你在怪我}\end{exemple}
\begin{exemple}\pjya{nɤʑo tu-tɯ́-wɣ-ɲɯ-ŋu}\hspace{5pt}\pcmn{他在怪你}\end{exemple}
\begin{exemple}\pjya{nɤʑo ɲɤ-tɯ-nɯβde-t, aj tu-kɯ-rɤɣlan-a ɲɯ-ŋu}\hspace{5pt}\pcmn{你弄丢了还怪我}\end{exemple}\étymologie{len}\end{entrée}

\begin{entrée}{rɤɣndi}{}{ⓔrɤɣndi} 
\classe{vt} \paradigme{dir}{thɯ-}\paradigme{dir}{pɯ-}\paradigme{dir}{\_}
\begin{définition}\pfra{bourrer}\end{définition}
\begin{définition}\pcmn{硬塞}\end{définition}
\begin{exemple}\pjya{tɤ-fkɯm ɯ-ŋgɯ tha-rɤɣndi}\hspace{5pt}\pcmn{他硬塞进口袋里}\end{exemple}
\begin{exemple}\pjya{kɯ-spoʁ ɯ-ŋgɯ pa-rɤɣndi}\hspace{5pt}\pcmn{他硬塞进洞里}\end{exemple}
\begin{exemple}\pjya{tɤ-fkɯm ɯ-ŋgɯ tɯ-ŋga thɯ-rɤɣndi-t-a}\hspace{5pt}\pcmn{我把衣服硬塞进口袋里}\end{exemple}\end{entrée}

\begin{entrée}{rɤɣo}{}{ⓔrɤɣo} 
\classe{n} 
\begin{définition}\pfra{chanson}\end{définition}
\begin{définition}\pcmn{歌}\end{définition}
\begin{exemple}\pjya{rɤɣo thɯ-βzu-t-a}\hspace{5pt}\pcmn{我唱了歌}\end{exemple}
\begin{exemple}\pjya{rɤɣo pɯ-lat-a}\hspace{5pt}\pcmn{我演奏了音乐}\end{exemple}
\begin{exemple}\pjya{rɤɣo ci thɯ-tɯt-a}\hspace{5pt}\pcmn{我唱了一首歌}\end{exemple}
\begin{exemple}\pjya{a-rɤɣo ci thɯ-βze}\hspace{5pt}\pcmn{给我唱一首歌吧!}\end{exemple}\relationsémantique{参考}{\lien{ⓔnɯrɤɣo}{nɯrɤɣo}}\end{entrée}

\begin{entrée}{rɤɣrɯ}{}{ⓔrɤɣrɯ} 
\classe{vi} \paradigme{dir}{pɯ-}
\begin{définition}\pfra{traiter une douleur en appliquant un objet chaud}\end{définition}
\begin{définition}\pcmn{用热的东西治关节炎}\end{définition}
\begin{exemple}\pjya{χtɕoŋ kɯ-tu nɯnɯ, kɯɕpaz ɯ-tʂɤm tɯ-χpɯm ɲɯ́-wɣ-mar tɕe, chɯ́-wɣ-ɣɤmpja tɕe kɤ-rɤɣrɯ kɤ-ti ɲɯ-ŋu}\hspace{5pt}\pcmn{关节炎患者在膝盖上涂旱獭油发热,这是治疗关节炎的方法}\end{exemple}
\begin{sous-entrée}{zrɤɣrɯ}{ⓔrɤɣrɯⓝzrɤɣrɯ} 
\classe{vt} 
\begin{définition}\pfra{appliquer un objet chaud sur une articulation}\end{définition}
\begin{définition}\pcmn{用热的东西敷身体的某个关节治关节炎}\end{définition}
\begin{exemple}\pjya{a-χpɯm pɯ-z-rɤɣrɯ-t-a}\hspace{5pt}\pcmn{我用热的东西敷了膝盖}\end{exemple}\end{sous-entrée}

\end{entrée}

\begin{entrée}{rɤji}{}{ⓔrɤji} 
\classe{vi}  
\grammaire{apass} \paradigme{dir}{lɤ-}\paradigme{dir}{pɯ-}
\begin{définition}\pfra{planter, semer}\end{définition}
\begin{définition}\pcmn{播种}\end{définition}
\begin{exemple}\pjya{aʑo pɯ-rɤji-a}\hspace{5pt}\pcmn{我播种了}\end{exemple}\relationsémantique{参考}{\lien{ⓔji}{ji}}\end{entrée}

\begin{entrée}{rɤjla}{}{ⓔrɤjla} 
\classe{n} 
\begin{définition}\pfra{habit d'homme en lin}\end{définition}
\begin{définition}\pcmn{布制成的男装}\end{définition}\étymologie{ras + lwa.ba}\end{entrée}

\begin{entrée}{rɤjndoʁ}{}{ⓔrɤjndoʁ} 
\classe{n} 
\begin{définition}\pfra{navet}\end{définition}
\begin{définition}\pcmn{芜菁根}\end{définition}\relationsémantique{参考}{\lien{ⓔɕaʁwɯ}{ɕaʁwɯ}}\relationsémantique{参考}{\lien{ⓔlaβzɣi}{laβzɣi}}\relationsémantique{参考}{\lien{ⓔkamda}{kamda}}\relationsémantique{参考}{\lien{ⓔrgawɯ}{rgawɯ}}\relationsémantique{参考}{\lien{ⓔrasti}{rasti}}\end{entrée}

\begin{entrée}{rɤjoʁβzɯr}{}{ⓔrɤjoʁβzɯr} 
\classe{vt}  
\grammaire{comp} \paradigme{dir}{tɤ-}
\begin{définition}\pfra{débarrasser, ranger une pièce}\end{définition}
\begin{définition}\pcmn{收拾;弄整齐}\end{définition}
\begin{exemple}\pjya{laχtɕha tɤ-rɤjoʁβzɯr}\hspace{5pt}\pcmn{你收拾一下东西}\end{exemple}
\begin{exemple}\pjya{nɯ fse a-mɤ-pɯ-ɤnɯta, tɤ-rɤjoʁβzɯr}\hspace{5pt}\pcmn{东西不能这样放着,你收拾一下}\end{exemple}\relationsémantique{参考}{\lien{ⓔjoʁ}{joʁ}}\relationsémantique{参考}{\lien{ⓔβzɯr}{βzɯr}}\relationsémantique{参考}{\lien{ⓔjoʁβzɯr}{joʁβzɯr}}\end{entrée}

\begin{entrée}{rɤjroʁ}{}{ⓔrɤjroʁ} 
\classe{vi} \paradigme{dir}{\_}\sens{1}
\begin{définition}\pfra{laissant de longues traces}\end{définition}
\begin{définition}\pcmn{留下长条的痕迹}\end{définition}\sens{2}
\begin{définition}\pfra{ayant des rayures}\end{définition}
\begin{définition}\pcmn{有纹路}\end{définition}\sens{3}\paradigme{dir}{pɯ-}
\begin{définition}\pfra{(avoir de la morve) qui pend au nez}\end{définition}
\begin{définition}\pcmn{挂着(鼻涕)}\end{définition}
\begin{exemple}\pjya{ɯ-ɕnaβ ra pjɯ-rɤjroʁ ʑo}\hspace{5pt}\pcmn{他挂着鼻涕}\end{exemple}
\begin{sous-entrée}{zrɤjroʁ}{ⓔrɤjroʁⓢ3ⓝzrɤjroʁ} 
\classe{vt} 
\begin{définition}\pfra{laisser une trace}\end{définition}
\begin{définition}\pcmn{留下纹路}\end{définition}\relationsémantique{参考}{\lien{ⓔtɯ-jroʁ}{tɯ-jroʁ}}\end{sous-entrée}

\end{entrée}

\begin{entrée}{rɤjtshi}{}{ⓔrɤjtshi}\relationsémantique{参考}{\lien{ⓔjtshi}{jtshi}}\end{entrée}

\begin{entrée}{rɤjɯɣ}{}{ⓔrɤjɯɣ} 
\classe{vt} \paradigme{dir}{lɤ-}
\begin{définition}\pfra{faire de la ficelle en roulant dans les mains}\end{définition}
\begin{définition}\pcmn{搓线(用手)}\end{définition}
\begin{exemple}\pjya{tɯ-ŋgru lu-kɤ-rɤjɯɣ}\hspace{5pt}\pcmn{搓成线的牛筋}\end{exemple}\relationsémantique{同义词}{\lien{ⓔpɣo}{pɣo}}\relationsémantique{同义词}{\lien{ⓔrɯm}{rɯm}}\end{entrée}

\begin{entrée}{rɤjwaʁ}{}{ⓔrɤjwaʁ} 
\classe{vs} \paradigme{dir}{nɯ-}\paradigme{dir}{tɤ-}
\begin{définition}\pfra{pousser des feuilles}\end{définition}
\begin{définition}\pcmn{长出叶子}\end{définition}
\begin{exemple}\pjya{χɕitka jɤ-ɣe tɕe, sɯku ɲɯ-rɤjwaʁ ɲɯ-ŋu}\hspace{5pt}\pcmn{到了春天,树长出叶子}\end{exemple}\relationsémantique{参考}{\lien{ⓔtɤ-jwaʁ}{tɤ-jwaʁ}}\relationsémantique{参考}{\lien{ⓔɣɤjwaʁ}{ɣɤjwaʁ}}\end{entrée}

\begin{entrée}{rɤɟar}{}{ⓔrɤɟar}\relationsémantique{参考}{\lien{ⓔɟar}{ɟar}}\end{entrée}

\begin{entrée}{rɤɟom}{}{ⓔrɤɟom} 
\classe{vt}  
\grammaire{denom} \paradigme{dir}{\_}
\begin{définition}\pfra{mesurer}\end{définition}
\begin{définition}\pcmn{一排一排地量}\end{définition}
\begin{exemple}\pjya{tɯmbri nɯ-rɤɟom-a}\hspace{5pt}\pcmn{我把绳子一排一排地量了一下}\end{exemple}
\begin{exemple}\pjya{si nɯ-rɤɟom-a}\hspace{5pt}\pcmn{我把木头一排一排地量了一下}\end{exemple}\relationsémantique{参考}{\lien{ⓔtɯ-ɟom}{tɯ-ɟom}}\end{entrée}

\begin{entrée}{rɤkha}{}{ⓔrɤkha} 
\classe{vi} \paradigme{dir}{tɤ-}
\begin{définition}\pfra{construire une maison}\end{définition}
\begin{définition}\pcmn{修房子}\end{définition}
\begin{exemple}\pjya{aʑo tu-nɯ-rɤkha-a ŋu}\hspace{5pt}\pcmn{我修(自己的)房子}\end{exemple}\relationsémantique{参考}{\lien{ⓔkha}{kha}}\end{entrée}

\begin{entrée}{rɤkhɯkhrɤt}{}{ⓔrɤkhɯkhrɤt}\relationsémantique{参考}{\lien{ⓔkhrɤtⓗ1}{khrɤt₁}}\end{entrée}

\begin{entrée}{rɤkrɤz}{}{ⓔrɤkrɤz} 
\classe{vi}  
\grammaire{denom} \paradigme{dir}{tɤ-}
\begin{définition}\pfra{discuter}\end{définition}
\begin{définition}\pcmn{商量}\end{définition}
\begin{exemple}\pjya{tɤ-rɤkrɤz-tɕi}\hspace{5pt}\pcmn{我们俩商量过了}\end{exemple}
\begin{exemple}\pjya{khro tɤ-rɤkrɤz-nɯ}\hspace{5pt}\pcmn{他们商量了很久}\end{exemple}\relationsémantique{参考}{\lien{ⓔtɯkrɤz}{tɯkrɤz}}\relationsémantique{参考}{\lien{ⓔnɯkrɤz}{nɯkrɤz}}\étymologie{gros}\end{entrée}

\begin{entrée}{rɤkro}{}{ⓔrɤkro}\relationsémantique{参考}{\lien{ⓔkro}{kro}}\end{entrée}

\begin{entrée}{rɤkrɯ}{}{ⓔrɤkrɯ} 
\classe{vt} \paradigme{dir}{pɯ-}\paradigme{dir}{thɯ-}
\begin{définition}\pfra{couper}\end{définition}
\begin{définition}\pcmn{切}\end{définition}
\begin{exemple}\pjya{si thɯ-rɤkrɯ-t-a}\hspace{5pt}\pcmn{我切了木头}\end{exemple}
\begin{exemple}\pjya{yangyu pɯ-rɤkrɯ-t-a}\hspace{5pt}\pcmn{我切了洋芋}\end{exemple}
\begin{exemple}\pjya{tɤ-mthɯm pɯ-rɤkrɯ-t-a}\hspace{5pt}\pcmn{我切了肉}\end{exemple}\relationsémantique{参考}{\lien{ⓔkɯkrɯ}{kɯkrɯ}}
\begin{sous-entrée}{zrɤkrɯ}{ⓔrɤkrɯⓝzrɤkrɯ} 
\classe{vt}  
\grammaire{caus} 
\begin{définition}\pfra{couper avec}\end{définition}
\begin{définition}\pcmn{用……切}\end{définition}
\begin{exemple}\pjya{ɯ-ɕɣa kɯ-tu nɯ kɯ pjɯ́-wɣ-z-rɤkrɯ kɤ-ti ɕti}\hspace{5pt}\pcmn{有刀刃的东西都可以用来切}\end{exemple}\end{sous-entrée}

\end{entrée}

\begin{entrée}{rɤkɯkrɯ}{}{ⓔrɤkɯkrɯ} 
\classe{vt} \paradigme{dir}{pɯ-}
\begin{définition}\pfra{découper en morceaux}\end{définition}
\begin{définition}\pcmn{切成几块}\end{définition}\relationsémantique{参考}{\lien{ⓔkɯkrɯ}{kɯkrɯ}}\relationsémantique{参考}{\lien{ⓔrɤkrɯ}{rɤkrɯ}}\end{entrée}

\begin{entrée}{rɤlaj}{}{ⓔrɤlaj} 
\classe{vt} \paradigme{dir}{pɯ-}
\begin{définition}\pfra{pétrir la pâte}\end{définition}
\begin{définition}\pcmn{揉面;挼}\end{définition}
\begin{exemple}\pjya{tɤjlu pɯ-rɤlaj-a}\hspace{5pt}\pcmn{我揉了面}\end{exemple}
\begin{exemple}\pjya{tɤrcoʁ pɯ-rɤlaj-a}\hspace{5pt}\pcmn{我揉了泥土}\end{exemple}
\begin{exemple}\pjya{tɤjlu pjɯ́-wɣ-rɤlaj tɕe mɯm}\hspace{5pt}\pcmn{面要揉才好吃}\end{exemple}\end{entrée}

\begin{entrée}{rɤlɤt}{}{ⓔrɤlɤt} 
\classe{vi} 
\begin{définition}\pfra{mouler, fondre (le métal)}\end{définition}
\begin{définition}\pcmn{铸造}\end{définition}\relationsémantique{参考}{\lien{}{lɤt}}\end{entrée}

\begin{entrée}{rɤli}{}{ⓔrɤli} 
\classe{vt} \paradigme{dir}{tɤ-}
\begin{définition}\pfra{dédommager}\end{définition}
\begin{définition}\pcmn{赔偿}\end{définition}
\begin{exemple}\pjya{a-laχtɕha ɲɤ-tɯ-βde-t, tɤ-rɤli}\hspace{5pt}\pcmn{你把我的东西弄丢了,你要赔!}\end{exemple}
\begin{exemple}\pjya{ɯ-laχtɕha ɲɤ-nɯβde-t-a, tɤ-rɤli-t-a}\hspace{5pt}\pcmn{我把他的东西弄丢了,我给他赔了}\end{exemple}\relationsémantique{同义词}{\lien{ⓔnɯmbe}{nɯmbe}}\end{entrée}

\begin{entrée}{rɤluj}{}{ⓔrɤluj} 
\classe{vs} \paradigme{dir}{tɤ-}
\begin{définition}\pfra{pleurer sans cesse (enfant)}\end{définition}
\begin{définition}\pcmn{小孩子不停的哭喊;撒娇}\end{définition}\end{entrée}

\begin{entrée}{rɤlkɯɣ}{}{ⓔrɤlkɯɣ} 
\classe{vt} \paradigme{dir}{tɤ-}
\begin{définition}\pfra{enrouler en cercle}\end{définition}
\begin{définition}\pcmn{卷成一圈}\end{définition}
\begin{exemple}\pjya{tɯmbri tɤ-rɤlkɯɣ-a}\hspace{5pt}\pcmn{我把绳子卷成一圈一圈}\end{exemple}\relationsémantique{参考}{\lien{ⓔtɯ-tɤlkɯɣ}{tɯ-tɤlkɯɣ}}\relationsémantique{同义词}{\lien{ⓔaɣɯrkɯrkɯⓝzɣɯrkɯrkɯ}{zɣɯrkɯrkɯ}}\relationsémantique{同义词}{\lien{ⓔrɤstɯm}{rɤstɯm}}\end{entrée}

\begin{entrée}{rɤloʁ}{}{ⓔrɤloʁ} 
\classe{vi}  
\grammaire{denom} \paradigme{dir}{kɤ-}
\begin{définition}\pfra{faire un nid}\end{définition}
\begin{définition}\pcmn{打窝}\end{définition}
\begin{exemple}\pjya{qajdo ɲɯ-rɤloʁ}\hspace{5pt}\pcmn{乌鸦在做巢}\end{exemple}
\begin{exemple}\pjya{paʁ ɲɯ-rɤloʁ}\hspace{5pt}\pcmn{猪在做窝}\end{exemple}\relationsémantique{参考}{\lien{ⓔtɤ-loʁⓗ1}{tɤ-loʁ₁}}\end{entrée}

\begin{entrée}{rɤma}{}{ⓔrɤma} 
\classe{vi} \paradigme{dir}{tɤ-}\paradigme{dir}{pɯ-}
\begin{définition}\pfra{travailler}\end{définition}
\begin{définition}\pcmn{劳动}\end{définition}
\begin{exemple}\pjya{jisŋi ɕ-pɯ-rɤma-a}\hspace{5pt}\pcmn{我今天去劳动了}\end{exemple}\relationsémantique{参考}{\lien{ⓔta-ma}{ta-ma}}\relationsémantique{参考}{\lien{ⓔnɤma}{nɤma}}
\begin{sous-entrée}{zrɤma}{ⓔrɤmaⓝzrɤma} 
\classe{vt}  
\grammaire{caus} 
\begin{définition}\pfra{faire travailler}\end{définition}
\begin{définition}\pcmn{让……工作}\end{définition}\end{sous-entrée}

\end{entrée}

\begin{entrée}{rɤmat}{}{ⓔrɤmat} 
\classe{vi}  
\grammaire{denom} \paradigme{dir}{thɯ-}
\begin{définition}\pfra{faire des fruits}\end{définition}
\begin{définition}\pcmn{结果子}\end{définition}
\begin{exemple}\pjya{jiɕqha nɯ cho-rɤmat}\hspace{5pt}\pcmn{那个结了果}\end{exemple}\relationsémantique{参考}{\lien{ⓔɯ-mat}{ɯ-mat}}\end{entrée}

\begin{entrée}{rɤmbi}{}{ⓔrɤmbi} 
\classe{vi}  
\grammaire{apass} \paradigme{dir}{nɯ-}
\begin{définition}\pfra{donner à quelqu'un}\end{définition}
\begin{définition}\pcmn{给别人}\end{définition}
\begin{exemple}\pjya{@yangyu nɯ-rɤmbi-a}\hspace{5pt}\pcmn{我给了洋芋}\end{exemple}
\begin{exemple}\pjya{stoʁ nɯ-rɤmbi-a}\hspace{5pt}\pcmn{我给了胡豆}\end{exemple}
\begin{exemple}\pjya{nɯ-rɤmbi-j ma khɯnɤmu ɲɯ-ɕti tɕe a-mɤ-thɯ-rɤpɯ ma mɯ́j-saχɕɯn}\hspace{5pt}\pcmn{我们把它拿去送人了。因为是母狗,生崽的话很不卫生。}\end{exemple}\relationsémantique{参考}{\lien{ⓔmbi}{mbi}}\end{entrée}

\begin{entrée}{rɤmboʁɲɟi}{}{ⓔrɤmboʁɲɟi} 
\classe{vt} \paradigme{dir}{nɯ-}
\begin{définition}\pfra{écraser en petits morceaux}\end{définition}
\begin{définition}\pcmn{弄碎(搓成碎片)}\end{définition}\relationsémantique{参考}{\lien{ⓔrɤɲɟiɲɟi}{rɤɲɟiɲɟi}}\relationsémantique{参考}{\lien{ⓔrɤɲɟoʁɲɟi}{rɤɲɟoʁɲɟi}}\end{entrée}

\begin{entrée}{rɤmbɯmbri}{}{ⓔrɤmbɯmbri} 
\classe{vt} \paradigme{dir}{tɤ-}
\begin{définition}\pfra{perdre/jeter au fil du chemin}\end{définition}
\begin{définition}\pcmn{一边走路一边撒下}\end{définition}
\begin{exemple}\pjya{tɤ-fkɯm pjɤ-spoʁ tɕe, ɯ-ŋgɯ kɯ-tu nɯra to-tɯ-rɤmbɯmbri-t}\hspace{5pt}\pcmn{袋子有洞,所以把里面的东西撒得一路都是}\end{exemple}\relationsémantique{参考}{\lien{ⓔarɤmbɯmbri}{arɤmbɯmbri}}\end{entrée}

\begin{entrée}{rɤmdzɯt}{}{ⓔrɤmdzɯt} 
\classe{vi} \paradigme{dir}{tɤ-}\sens{1}
\begin{définition}\pfra{décider}\end{définition}
\begin{définition}\pcmn{决定,……说了算}\end{définition}
\begin{exemple}\pjya{nɤʑo tɤ-rɤmdzɯt}\hspace{5pt}\pcmn{你说了算}\end{exemple}
\begin{exemple}\pjya{znde pjɯ́-wɣ-phɯt tɕe ma-pɯ́-wɣ-phɯt, nɤʑo tɯ-rɤmdzɯt}\hspace{5pt}\pcmn{你决定要不要拆石墙}\end{exemple}\sens{2}
\begin{définition}\pfra{dépendre de}\end{définition}
\begin{définition}\pcmn{取决于}\end{définition}
\begin{exemple}\pjya{nɯ kɤ-nɤma pe mɤ-pe rɤmdzɯt}\hspace{5pt}\pcmn{看工作做得好不好}\end{exemple}\relationsémantique{同义词}{\lien{ⓔzrɤtɕha}{zrɤtɕha}}\relationsémantique{参考}{\lien{ⓔmdzɯt}{mdzɯt}}\end{entrée}

\begin{entrée}{rɤmgrɯn}{}{ⓔrɤmgrɯn}\relationsémantique{参考}{\lien{ⓔmgrɯn}{mgrɯn}}\end{entrée}

\begin{entrée}{rɤmnɯ}{}{ⓔrɤmnɯ} 
\classe{vi} \paradigme{dir}{nɯ-}
\begin{définition}\pfra{germer (arbre)}\end{définition}
\begin{définition}\pcmn{发芽(树)}\end{définition}\end{entrée}

\begin{entrée}{rɤmɲo}{}{ⓔrɤmɲo}\relationsémantique{参考}{\lien{ⓔmɲoⓗ1}{mɲo₁}}\end{entrée}

\begin{entrée}{rɤmpɕɤr}{}{ⓔrɤmpɕɤr} 
\classe{vi} \paradigme{dir}{tɤ-}\paradigme{dir}{tɤ-}
\begin{définition}\pfra{se maquiller}\end{définition}
\begin{définition}\pcmn{打扮(装饰)}\end{définition}
\begin{définition}\pfra{maquiller, décorer}\end{définition}
\begin{définition}\pcmn{装饰;装扮}\end{définition}
\begin{exemple}\pjya{jisŋi to-tɯ-rɤmpɕɤr}\hspace{5pt}\pcmn{你今天打扮了}\end{exemple}
\begin{exemple}\pjya{tɤ-rɤmpɕar-a}\hspace{5pt}\pcmn{我打扮了}\end{exemple}
\begin{exemple}\pjya{stɯnmɯ tɤ-mda tɕe, kɯ-rɯstɯnmɯ tɤ-tɕɯ tɕheme nɯni tú-wɣ-zrɤmpɕɤr ra}\hspace{5pt}\pcmn{办婚礼的时候要打扮新郎和新娘}\end{exemple}\relationsémantique{参考}{\lien{ⓔmpɕɤr}{mpɕɤr}}
\begin{sous-entrée}{zrɤmpɕɤr}{ⓔrɤmpɕɤrⓝzrɤmpɕɤr} 
\classe{vt} \end{sous-entrée}

\end{entrée}

\begin{entrée}{rɤmphrɯm}{}{ⓔrɤmphrɯm} 
\classe{vt} \paradigme{dir}{\_}
\begin{définition}\pfra{aligner en rangées}\end{définition}
\begin{définition}\pcmn{排整齐}\end{définition}
\begin{exemple}\pjya{kɯmɕku kɤ-ji tɕe, lothi chɯ́-wɣ-rɤmphrɯm tɕe, tɯ-rdoʁ ɯ-thɤcu nɯ tɕu ntsɯ tɯ-rdoʁ kɯ-fse chɯ́-wɣ-ji tɕe tɕe chɯ́-wɣ-rɤmphrɯm ɲɯ-ŋu}\hspace{5pt}\pcmn{种大蒜的时候,要一个接着一个地排下来}\end{exemple}\relationsémantique{同义词}{\lien{ⓔaʑɯrjaⓝsɤʑɯrja}{sɤʑɯrja}}\relationsémantique{参考}{\lien{ⓔtɯ-tɤmphrɯm}{tɯ-tɤmphrɯm}}\end{entrée}

\begin{entrée}{rɤmphɯr}{}{ⓔrɤmphɯr}\relationsémantique{参考}{\lien{ⓔmphɯr}{mphɯr}}\end{entrée}

\begin{entrée}{rɤmprɤt}{}{ⓔrɤmprɤt} 
\classe{vt} \paradigme{dir}{nɯ-}
\begin{définition}\pfra{interrompre}\end{définition}
\begin{définition}\pcmn{中断}\end{définition}
\begin{exemple}\pjya{ɯʑo kɯ ɯ-ma ɲɤ-rɤmprɤt}\hspace{5pt}\pcmn{他中断了工作}\end{exemple}
\begin{exemple}\pjya{tɯ-mɯ ka-lɤt tɕe kha kɤ-βzu nɯ-rɤmprɤt-i pɯ-ra}\hspace{5pt}\pcmn{因为下雨,我们中断了修房子的工程}\end{exemple}\relationsémantique{参考}{\lien{ⓔprɤt}{prɤt}}\end{entrée}

\begin{entrée}{rɤmrɯmrɤmrɯm}{}{ⓔrɤmrɯmrɤmrɯm} 
\classe{idph.2} 
\begin{définition}\pfra{(dire) les uns après les autres}\end{définition}
\begin{définition}\pcmn{接二连三(说)}\end{définition}
\begin{exemple}\pjya{rɤmrɯmrɤmrɯm ʑo to-nɯ-ti-nɯ}\hspace{5pt}\pcmn{他们接二连三地说出了自己的想法}\end{exemple}\end{entrée}

\begin{entrée}{rɤmɯthu}{}{ⓔrɤmɯthu} 
\classe{vi} \paradigme{dir}{tɤ-}\paradigme{dir}{nɯ-}
\begin{définition}\pfra{demander partout}\end{définition}
\begin{définition}\pcmn{到处问}\end{définition}
\begin{exemple}\pjya{aʑo tɤ-rɤmɯthu-a}\hspace{5pt}\pcmn{我到处问了}\end{exemple}\relationsémantique{同义词}{\lien{ⓔnɤthɯthu}{nɤthɯthu}}\relationsémantique{参考}{\lien{ⓔthuⓗ1}{thu₁}}\end{entrée}

\begin{entrée}{rɤndaŋ}{}{ⓔrɤndaŋ} 
\classe{vi} \paradigme{dir}{pɯ-}\paradigme{dir}{thɯ-}
\begin{définition}\pfra{réfléchir}\end{définition}
\begin{définition}\pcmn{思考,考虑}\end{définition}
\begin{exemple}\pjya{ta-ma kɤ-nɤma tɕe, koŋla pjɯ-kɯ-rɤndaŋ ra}\hspace{5pt}\pcmn{工作的时候需要思考}\end{exemple}\relationsémantique{参考}{\lien{ⓔɯ-ndaŋ,lɤt}{ɯ-ndaŋ,lɤt}}\end{entrée}

\begin{entrée}{rɤndɯn}{}{ⓔrɤndɯn}\relationsémantique{参考}{\lien{ⓔndɯn}{ndɯn}}\end{entrée}

\begin{entrée}{rɤndzraʁ}{}{ⓔrɤndzraʁ} 
\classe{vt}  
\grammaire{caus} \paradigme{dir}{tɤ-}\paradigme{dir}{nɯ-}\paradigme{dir}{nɯ-}
\begin{définition}\pfra{pétrir}\end{définition}
\begin{définition}\pcmn{用手捏}\end{définition}
\begin{définition}\pfra{pétrir avec}\end{définition}
\begin{définition}\pcmn{用……捏}\end{définition}
\begin{exemple}\pjya{rɟɤɣi tɤ-rɤndzraʁ-a}\hspace{5pt}\pcmn{我把糌粑捏成一坨了}\end{exemple}
\begin{exemple}\pjya{tɤrcoʁ nɯ-rɤndzraʁ-a}\hspace{5pt}\pcmn{我把泥土捏成一坨了}\end{exemple}
\begin{exemple}\pjya{tɯ-pu nɯ-rɤndzraʁ}\hspace{5pt}\pcmn{我把血肠的馅儿捏下去了}\end{exemple}
\begin{exemple}\pjya{tɯrme ɲɤ-rɤndzraʁ}\hspace{5pt}\pcmn{他把人捏成一坨了(打得很惨了)}\end{exemple}
\begin{sous-entrée}{zrɤndzraʁ}{ⓔrɤndzraʁⓝzrɤndzraʁ} 
\classe{vt} \end{sous-entrée}

\end{entrée}

\begin{entrée}{rɤndzri}{}{ⓔrɤndzri} 
\classe{vt} \paradigme{dir}{kɤ-}
\begin{définition}\pfra{tordre la paille pour la faire sécher}\end{définition}
\begin{définition}\pcmn{拧(干草)}\end{définition}
\begin{exemple}\pjya{tɯɣro kɤ-rɤndzri-t-a}\hspace{5pt}\pcmn{我把干草拧成一绞了}\end{exemple}\relationsémantique{参考}{\lien{ⓔndzri}{ndzri}}\relationsémantique{参考}{\lien{ⓔtɯ-tɤndzri}{tɯ-tɤndzri}}\end{entrée}

\begin{entrée}{rɤnŋa}{}{ⓔrɤnŋa} 
\classe{vi}  
\grammaire{apass} \paradigme{dir}{pɯ-}
\begin{définition}\pfra{devoir de l'argent}\end{définition}
\begin{définition}\pcmn{欠账}\end{définition}
\begin{exemple}\pjya{ɯʑɤɣ, ɯ-phe pɯ-rɤnŋa-a}\hspace{5pt}\pcmn{我以前欠他的钱}\end{exemple}
\begin{exemple}\pjya{aʑo nɤ-ɕki rɤnŋa-a}\hspace{5pt}\pcmn{我欠你的钱}\end{exemple}\relationsémantique{参考}{\lien{ⓔŋa}{ŋa}}\relationsémantique{参考}{\lien{ⓔtɯ-nŋa}{tɯ-nŋa}}\end{entrée}

\begin{entrée}{rɤntɕha}{}{ⓔrɤntɕha}\relationsémantique{参考}{\lien{ⓔntɕha}{ntɕha}}\end{entrée}

\begin{entrée}{rɤntɕhom}{}{ⓔrɤntɕhom} 
\classe{vi} \paradigme{dir}{tɤ-}
\begin{définition}\pfra{effectuer une danse rituelle}\end{définition}
\begin{définition}\pcmn{跳神}\end{définition}
\begin{exemple}\pjya{rgɯnba ɲɯ-rɤntɕhom-nɯ}\hspace{5pt}\pcmn{庙里在跳神}\end{exemple}\étymologie{ⁿtɕʰams}\end{entrée}

\begin{entrée}{rɤntshom}{}{ⓔrɤntshom} 
\classe{vi} \paradigme{dir}{kɤ-}
\begin{définition}\pfra{faire une retraite}\end{définition}
\begin{définition}\pcmn{闭关修行}\end{définition}
\begin{exemple}\pjya{βlama ɲɯ-rɤntshom}\hspace{5pt}\pcmn{喇嘛在修行}\end{exemple}
\begin{exemple}\pjya{βlama ko-rɤntshom}\hspace{5pt}\pcmn{喇嘛修行了}\end{exemple}\étymologie{bsɲen.mtsʰams}\end{entrée}

\begin{entrée}{rɤɲɟiɲɟi}{}{ⓔrɤɲɟiɲɟi} 
\classe{vt} \paradigme{dir}{nɯ-}
\begin{définition}\pfra{écraser en petits morceaux}\end{définition}
\begin{définition}\pcmn{弄碎(搓成碎片)}\end{définition}\relationsémantique{参考}{\lien{ⓔrɤmboʁɲɟi}{rɤmboʁɲɟi}}\relationsémantique{参考}{\lien{ⓔrɤɲɟoʁɲɟi}{rɤɲɟoʁɲɟi}}\end{entrée}

\begin{entrée}{rɤɲɟoʁɲɟi}{}{ⓔrɤɲɟoʁɲɟi} 
\classe{vt} \paradigme{dir}{pɯ-}
\begin{définition}\pfra{écraser en petits morceaux}\end{définition}
\begin{définition}\pcmn{弄碎(搓成碎片)}\end{définition}
\begin{exemple}\pjya{@yangyu pɯ-rɤɲɟoʁɲɟi-t-a}\hspace{5pt}\pcmn{我把土豆弄碎了}\end{exemple}\relationsémantique{参考}{\lien{ⓔrɤɲɟiɲɟi}{rɤɲɟiɲɟi}}\relationsémantique{参考}{\lien{ⓔrɤmboʁɲɟi}{rɤmboʁɲɟi}}\end{entrée}

\begin{entrée}{rɤŋgat}{}{ⓔrɤŋgat} 
\classe{vi} \paradigme{dir}{tɤ-}
\begin{définition}\pfra{se préparer à, être sur le point de}\end{définition}
\begin{définition}\pcmn{准备}\end{définition}
\begin{exemple}\pjya{kɯ-ɕe tɤ-rɤŋgat-a}\hspace{5pt}\pcmn{我准备出发}\end{exemple}
\begin{exemple}\pjya{tɯ-mɯ kɯ-lɤt ɲɯ-rɤŋgat}\hspace{5pt}\pcmn{快要下雨了}\end{exemple}
\begin{exemple}\pjya{kɯ-mbɯt ɲɯ-rɤŋgat}\hspace{5pt}\pcmn{快要垮了}\end{exemple}\relationsémantique{同义词}{\lien{ⓔmɲoⓗ1ⓝʑɣɤmɲo}{ʑɣɤmɲo}}\end{entrée}

\begin{entrée}{rɤŋgɯm}{}{ⓔrɤŋgɯm} 
\classe{vi}  
\grammaire{denom} \paradigme{dir}{thɯ-}
\begin{définition}\pfra{pondre}\end{définition}
\begin{définition}\pcmn{下蛋}\end{définition}
\begin{exemple}\pjya{pɣa chɤ-rɤŋgɯm}\hspace{5pt}\pcmn{鸟下了蛋}\end{exemple}
\begin{exemple}\pjya{kumpɣa ɲɯ-rɤŋgɯm}\hspace{5pt}\pcmn{鸟在下蛋}\end{exemple}\relationsémantique{参考}{\lien{ⓔtɤ-ŋgɯm}{tɤ-ŋgɯm}}\end{entrée}

\begin{entrée}{rɤŋom}{}{ⓔrɤŋom} 
\classe{n} 
\begin{définition}\pfra{énervement}\end{définition}
\begin{définition}\pcmn{气愤}\end{définition}
\begin{exemple}\pjya{rɤŋom kɯ pɯwɯ pjɤ-sat}\hspace{5pt}\pcmn{气得把驴子弄死了}\end{exemple}\relationsémantique{参考}{\lien{ⓔnɯrɤŋom}{nɯrɤŋom}}\end{entrée}

\begin{entrée}{rɤɴqra}{}{ⓔrɤɴqra} 
\classe{vt}  
\grammaire{denom} \paradigme{dir}{nɯ-}
\begin{définition}\pfra{(faire de façon) incomplète}\end{définition}
\begin{définition}\pcmn{(做得)不完整}\end{définition}
\begin{exemple}\pjya{jiɕqha pɯ-mdoʁmdi ri, pɯ-rɤɴqra-t-a}\hspace{5pt}\pcmn{刚才是完整的,我弄了个缺口(例如馍馍咬了一口)}\end{exemple}
\begin{exemple}\pjya{a-qajɣi pɯ-rɤɴqra-t-a}\hspace{5pt}\pcmn{我没有把馍馍吃完}\end{exemple}
\begin{exemple}\pjya{kɯrtsɤɣ kɯ paʁ to-ndza tɕe, chɤ-rɤɴqra}\hspace{5pt}\pcmn{豹子吃了猪,吃得不完整}\end{exemple}
\begin{exemple}\pjya{jɯfɕɯr χpi kɤ-fɕɤt mɯ-pɯ-sthɯt-a tɕe, nɯ-rɤɴqra-t-a}\hspace{5pt}\pcmn{昨天我没有把故事讲完,讲得不完整}\end{exemple}\relationsémantique{参考}{\lien{ⓔɯ-ɴqra}{ɯ-ɴqra}}\end{entrée}

\begin{entrée}{rɤpɕaʁ}{}{ⓔrɤpɕaʁ} 
\classe{vi}  
\grammaire{denom} \paradigme{dir}{lɤ-}
\begin{définition}\pfra{se prosterner}\end{définition}
\begin{définition}\pcmn{跪下磕头}\end{définition}
\begin{exemple}\pjya{lɤ-rɤpɕaʁa-a (=pɕaʁ lɤ-βzu-t-a)}\hspace{5pt}\pcmn{我磕了头}\end{exemple}\relationsémantique{参考}{\lien{ⓔpɕaʁ}{pɕaʁ}}\end{entrée}

\begin{entrée}{rɤpɣaʁ}{}{ⓔrɤpɣaʁ}\relationsémantique{参考}{\lien{ⓔpɣaʁ}{pɣaʁ}}\end{entrée}

\begin{entrée}{rɤpɣi}{}{ⓔrɤpɣi} 
\classe{vt} \paradigme{dir}{pɯ-}\paradigme{dir}{lɤ-}
\begin{définition}\pfra{mélanger la farine et l'eau}\end{définition}
\begin{définition}\pcmn{和面}\end{définition}
\begin{exemple}\pjya{qajɣi βze-a pɯ-ŋu tɕe, tɤjlu pɯ-rɤpɣi-t-a}\hspace{5pt}\pcmn{我准备做馍馍,就和了面}\end{exemple}\end{entrée}

\begin{entrée}{rɤphɯ}{}{ⓔrɤphɯ} 
\classe{vt} 
\begin{définition}\pfra{donner un prix}\end{définition}
\begin{définition}\pcmn{定价格}\end{définition}
\begin{exemple}\pjya{ki laχtɕha ki kɤ-rɤphɯ-t-a}\hspace{5pt}\pcmn{我定了这个东西的价格}\end{exemple}\relationsémantique{参考}{\lien{ⓔʑɣɤrɤphɯ}{ʑɣɤrɤphɯ}}\relationsémantique{参考}{\lien{ⓔɯ-phɯ}{ɯ-phɯ}}\end{entrée}

\begin{entrée}{rɤpjɤt}{}{ⓔrɤpjɤt}\relationsémantique{参考}{\lien{ⓔpjɤt}{pjɤt}}\end{entrée}

\begin{entrée}{rɤpjɤz}{}{ⓔrɤpjɤz} 
\classe{vt}  
\grammaire{denom} \paradigme{dir}{thɯ-}
\begin{définition}\pfra{tresser (les cheveux, fils)}\end{définition}
\begin{définition}\pcmn{编(头发,线)}\end{définition}
\begin{exemple}\pjya{nɤ-ku thɯ-rɤpjɤz}\hspace{5pt}\pcmn{你编辫子吧}\end{exemple}\relationsémantique{参考}{\lien{ⓔtɤpjɤz}{tɤpjɤz}}\end{entrée}

\begin{entrée}{rɤpɯ}{}{ⓔrɤpɯ} 
\classe{vi} \paradigme{dir}{thɯ-}
\begin{définition}\pfra{mettre bas (animaux)}\end{définition}
\begin{définition}\pcmn{生崽子(动物)}\end{définition}
\begin{exemple}\pjya{nɯŋa thɯ-rɤpɯ}\hspace{5pt}\pcmn{奶牛生了崽子}\end{exemple}
\begin{exemple}\pjya{paʁ thɯ-rɤpɯ}\hspace{5pt}\pcmn{猪生了崽子}\end{exemple}\relationsémantique{参考}{\lien{ⓔtɤ-pɯ}{tɤ-pɯ}}
\begin{sous-entrée}{zrɤpɯ}{ⓔrɤpɯⓝzrɤpɯ} 
\classe{vt}  
\grammaire{caus} 
\begin{définition}\pfra{faire avoir des petits (animaux)}\end{définition}
\begin{définition}\pcmn{让……生崽}\end{définition}\end{sous-entrée}

\end{entrée}

\begin{entrée}{rɤqur}{}{ⓔrɤqur} 
\classe{vi} \paradigme{dir}{tɤ-}
\begin{définition}\pfra{ramasser}\end{définition}
\begin{définition}\pcmn{收藏}\end{définition}
\begin{exemple}\pjya{laχtɕha tɤ-rɤqur-a}\hspace{5pt}\pcmn{我把东西收起来了}\end{exemple}
\begin{exemple}\pjya{laχtɕha kɤ-ntɕhoz mɤ-kɯ-ra nɯ ra tɤ-rɤqur-a}\hspace{5pt}\pcmn{我把不用的东西收藏起来了}\end{exemple}
\begin{exemple}\pjya{laχtɕha kɯ-ra nɯ ra a-pɯ-ɤta, mɤ-kɯ-ra nɯ ra tɤ-rɤqur}\hspace{5pt}\pcmn{需要的东西方在那里,不需要的东西收藏好}\end{exemple}
\begin{exemple}\pjya{a-ŋga nɯ thamtɕɤt mɯ́j-ra tɕe tɤ-rɤqur-a}\hspace{5pt}\pcmn{我不需要那么多衣服,所以就把它收藏起来了}\end{exemple}\end{entrée}

\begin{entrée}{rɤru}{}{ⓔrɤru} 
\classe{vi} \paradigme{dir}{tɤ-}\sens{1}
\begin{définition}\pfra{se lever}\end{définition}
\begin{définition}\pcmn{起床;起来}\end{définition}
\begin{exemple}\pjya{jɯfɕo tɤ-rɤru-a}\hspace{5pt}\pcmn{我今天早上起来了}\end{exemple}
\begin{exemple}\pjya{tʂu (tʂɤχcɤl) tɤ-rɤru ma ɲɯ-tɯ-saʁdɯɣ}\hspace{5pt}\pcmn{你起来,你挡到(我的路)}\end{exemple}
\begin{exemple}\pjya{jiɕqha nɯ nɯ-nɤkhɤzŋga-t-a ri, maka mɯ́j-rɤru}\hspace{5pt}\pcmn{我叫了,但是他根本不起床}\end{exemple}\sens{2}\paradigme{dir}{tɤ-}
\begin{définition}\pfra{fermenter (vin)}\end{définition}
\begin{définition}\pcmn{发酵(酒)}\end{définition}
\begin{définition}\pfra{faire lever}\end{définition}
\begin{définition}\pcmn{让……起床}\end{définition}
\begin{exemple}\pjya{cha to-rɤru (=ɲɤ-xtsu)}\hspace{5pt}\pcmn{酒发酵了}\end{exemple}
\begin{exemple}\pjya{tɤ-zrɤru-t-a}\hspace{5pt}\pcmn{我让他起床了}\end{exemple}
\begin{exemple}\pjya{tɤndʐo kɯ a-ŋgo to-zrɤru}\hspace{5pt}\pcmn{冷的天气令我生病了}\end{exemple}
\begin{sous-entrée}{zrɤru}{ⓔrɤruⓢ2ⓝzrɤru} 
\classe{vt} \end{sous-entrée}

\end{entrée}

\begin{entrée}{rɤraʁzɯz}{}{ⓔrɤraʁzɯz}\relationsémantique{参考}{\lien{ⓔraʁrɯz}{raʁrɯz}}\end{entrée}

\begin{entrée}{rɤrɤt}{}{ⓔrɤrɤt}\relationsémantique{参考}{\lien{ⓔrɤt}{rɤt}}\end{entrée}

\begin{entrée}{rɤrcoʁ}{}{ⓔrɤrcoʁ} 
\classe{vi} \paradigme{dir}{pɯ-}
\begin{définition}\pfra{mélanger de l’eau et de la terre}\end{définition}
\begin{définition}\pcmn{和泥}\end{définition}
\begin{exemple}\pjya{jisŋi pɯ-rɤrcoʁa (=tɤrcoʁ pɯ-βzu-t-a)}\hspace{5pt}\pcmn{我今天和了泥}\end{exemple}
\begin{exemple}\pjya{kha ɲɯ-ɤsɯ-βzu-nɯ tɕe, pɯ-rɤrcoʁ-a}\hspace{5pt}\pcmn{他们在修房子,所以我就和了泥}\end{exemple}
\begin{exemple}\pjya{ʑala ɲɯ-ɤsɯ-lɤt tɕe, pɯ-rɤrcoʁ-a}\hspace{5pt}\pcmn{他在糊墙,所以我就和了泥}\end{exemple}\relationsémantique{参考}{\lien{ⓔtɤrcoʁ}{tɤrcoʁ}}\relationsémantique{参考}{\lien{ⓔɣɤrcoʁ}{ɣɤrcoʁ}}\end{entrée}

\begin{entrée}{rɤrɟit}{}{ⓔrɤrɟit} 
\classe{vi}  
\grammaire{denom} \paradigme{dir}{thɯ-}\paradigme{dir}{thɯ-}
\begin{définition}\pfra{donner naissance à un enfant}\end{définition}
\begin{définition}\pcmn{生孩子}\end{définition}
\begin{définition}\pfra{faire donner naissance à un enfant}\end{définition}
\begin{définition}\pcmn{使……生孩子}\end{définition}
\begin{exemple}\pjya{chɤ-rɤrɟit (=ɯ-rɟit to-tu)}\hspace{5pt}\pcmn{她生了孩子}\end{exemple}\relationsémantique{参考}{\lien{ⓔtɤ-rɟit}{tɤ-rɟit}}
\begin{sous-entrée}{zrɤrɟit}{ⓔrɤrɟitⓝzrɤrɟit} 
\classe{vt}  
\grammaire{caus} \end{sous-entrée}

\end{entrée}

\begin{entrée}{rɤrka}{}{ⓔrɤrka} 
\classe{vi} \paradigme{dir}{thɯ-}
\begin{définition}\pfra{faire des jumeaux}\end{définition}
\begin{définition}\pcmn{生双胞胎}\end{définition}
\begin{exemple}\pjya{tshɤt chɤ-rɤrka}\hspace{5pt}\pcmn{山羊生了双胞胎}\end{exemple}\relationsémantique{参考}{\lien{ⓔtɤrkaⓗ2}{tɤrka₂}}\end{entrée}

\begin{entrée}{rɤrma}{}{ⓔrɤrma} 
\classe{vi} \paradigme{dir}{kɤ-}
\begin{définition}\pfra{habiter}\end{définition}
\begin{définition}\pcmn{居住}\end{définition}
\begin{exemple}\pjya{jiʑo chengdu kɤ-rɤrma-j}\hspace{5pt}\pcmn{我们住在成都了}\end{exemple}\relationsémantique{参考}{\lien{ⓔrma}{rma}}\relationsémantique{参考}{\lien{ⓔtɯrma}{tɯrma}}\end{entrée}

\begin{entrée}{rɤrmbɣo}{}{ⓔrɤrmbɣo} 
\classe{vi} \paradigme{dir}{lɤ-}\paradigme{dir}{lɤ-}
\begin{définition}\pfra{s'accumuler (eau)}\end{définition}
\begin{définition}\pcmn{积水}\end{définition}
\begin{définition}\pfra{retenir (accumuler) de l'eau}\end{définition}
\begin{définition}\pcmn{积水(大范围)}\end{définition}
\begin{exemple}\pjya{tɯ-ci lú-wɣ-zrɤrmbɣo ɲɯ-ra}\hspace{5pt}\pcmn{要积水}\end{exemple}
\begin{sous-entrée}{zrɤrmbɣo}{ⓔrɤrmbɣoⓝzrɤrmbɣo} 
\classe{vt} \end{sous-entrée}

\end{entrée}

\begin{entrée}{rɤroʁ}{}{ⓔrɤroʁ}\relationsémantique{参考}{\lien{ⓔroʁ}{roʁ}}\end{entrée}

\begin{entrée}{rɤroʁrɯz}{}{ⓔrɤroʁrɯz}\relationsémantique{参考}{\lien{ⓔraʁrɯz}{raʁrɯz}}\end{entrée}

\begin{entrée}{rɤrti}{}{ⓔrɤrti} 
\classe{vi} \paradigme{dir}{thɯ-}
\begin{définition}\pfra{mettre bas (cheval)}\end{définition}
\begin{définition}\pcmn{生崽(马)}\end{définition}
\begin{exemple}\pjya{mbro chɤ-rɤrti}\hspace{5pt}\pcmn{马生崽了}\end{exemple}\relationsémantique{同义词}{\lien{ⓔrɤpɯ}{rɤpɯ}}\relationsémantique{同义词}{\lien{}{ɬoʁ}}\relationsémantique{同义词}{\lien{ⓔrɤrɟit}{rɤrɟit}}\relationsémantique{参考}{\lien{ⓔɯ-rti}{ɯ-rti}}\end{entrée}

\begin{entrée}{rɤrtsɤɣ}{}{ⓔrɤrtsɤɣ} 
\classe{vs} \paradigme{dir}{tɤ-}
\begin{définition}\pfra{avoir la tige qui pousse}\end{définition}
\begin{définition}\pcmn{拔节}\end{définition}
\begin{exemple}\pjya{tɤɕi to-rɤrtsɤɣ}\hspace{5pt}\pcmn{青稞拔节了}\end{exemple}\relationsémantique{参考}{\lien{}{tɤrtsɤɣ}}\end{entrée}

\begin{entrée}{rɤrtshɯm}{}{ⓔrɤrtshɯm} 
\classe{vi} \paradigme{dir}{nɯ-}\paradigme{dir}{nɯ-}
\begin{définition}\pfra{ne pas finir}\end{définition}
\begin{définition}\pcmn{没有做完}\end{définition}
\begin{définition}\pfra{ne pas être fini}\end{définition}
\begin{définition}\pcmn{没有完成}\end{définition}
\begin{exemple}\pjya{a-ma na-rɤrtshɯm}\hspace{5pt}\pcmn{我的工作没有做完}\end{exemple}
\begin{exemple}\pjya{ɯ-rju nɯ ɲɤ-rɤrtshɯm}\hspace{5pt}\pcmn{他话没有说完}\end{exemple}
\begin{exemple}\pjya{ta-ma ɲɤ-rɤrtshɯm}\hspace{5pt}\pcmn{工作没有做完}\end{exemple}
\begin{exemple}\pjya{kɤ-nɤma nɯ kɤ-tshɯt mɯ-pɯ-ŋgrɯ tɕe, nɯ-rɤrtshɯm-i pɯ-ra}\hspace{5pt}\pcmn{工作没有能完成,所以我们只好半途而废了}\end{exemple}
\begin{exemple}\pjya{ɲɤ-k-ɤrɤrtshɯm-ci}\hspace{5pt}\pcmn{没有做完}\end{exemple}
\begin{exemple}\pjya{kɤ-nɤma kɤ-tshɯt mɯ-pjɤ-khɯ tɕe ɲɤ-k-ɤrɤrtshɯm-ci}\hspace{5pt}\pcmn{工作没有能做完,所以没有完成}\end{exemple}\relationsémantique{参考}{\lien{ⓔɯ-rtshɯm}{ɯ-rtshɯm}}
\begin{sous-entrée}{arɤrtshɯm}{ⓔrɤrtshɯmⓝarɤrtshɯm} 
\classe{vi}  
\grammaire{pass} \end{sous-entrée}

\end{entrée}

\begin{entrée}{rɤrzɯɣ/\variante{rɤrzɯrzɯɣ}}{}{ⓔrɤrzɯɣ} 
\classe{vt} \paradigme{dir}{pɯ-}
\begin{définition}\pfra{couper en sections}\end{définition}
\begin{définition}\pcmn{锯成一段一段、一节一节}\end{définition}
\begin{exemple}\pjya{si pɯ-rɤrzɯɣ-a}\hspace{5pt}\pcmn{我把木头锯成几段了}\end{exemple}
\begin{exemple}\pjya{tɯmbri pɯ-rɤrzɯɣ-a}\hspace{5pt}\pcmn{我把绳子剪成几段了}\end{exemple}\relationsémantique{同义词}{\lien{ⓔrɤɣdɤt}{rɤɣdɤt}}\relationsémantique{参考}{\lien{ⓔtɯ-rzɯɣ}{tɯ-rzɯɣ}}\end{entrée}

\begin{entrée}{rɤscɤt}{}{ⓔrɤscɤt}\relationsémantique{参考}{\lien{ⓔscɤt}{scɤt}}\end{entrée}

\begin{entrée}{rɤscoz}{}{ⓔrɤscoz} 
\classe{vi}  
\grammaire{denom} \paradigme{dir}{tɤ-}
\begin{définition}\pfra{écrire}\end{définition}
\begin{définition}\pcmn{写出来}\end{définition}
\begin{exemple}\pjya{ɲɯ-rɤscoz (tɤ-scoz ɲɯ-ɤsɯ-rɤt)}\hspace{5pt}\pcmn{他在写信}\end{exemple}
\begin{exemple}\pjya{jaχpa rɯmu tɤ-tɯ-nɯrɤscoz ɕti}\hspace{5pt}\pcmn{这是你签的字(这是你自己一手造成的)}\end{exemple}\relationsémantique{参考}{\lien{ⓔtɤ-scoz}{tɤ-scoz}}\end{entrée}

\begin{entrée}{rɤskɤr}{}{ⓔrɤskɤr}\relationsémantique{参考}{\lien{ⓔskɤr}{skɤr}}\end{entrée}

\begin{entrée}{rɤsloʁ}{}{ⓔrɤsloʁ}\relationsémantique{参考}{\lien{ⓔsloʁ}{sloʁ}}\end{entrée}

\begin{entrée}{rɤspɤr}{}{ⓔrɤspɤr}\relationsémantique{参考}{\lien{ⓔspɤr}{spɤr}}\end{entrée}

\begin{entrée}{rɤspra}{}{ⓔrɤspra} 
\classe{vt}  
\grammaire{denom} \paradigme{dir}{kɤ-}\paradigme{dir}{nɯ-}
\begin{définition}\pfra{prendre par poignées}\end{définition}
\begin{définition}\pcmn{一把一把地拿}\end{définition}
\begin{exemple}\pjya{sɯjno nɯ-rɤspra-t-a}\hspace{5pt}\pcmn{我一把一把地拿了草}\end{exemple}\relationsémantique{参考}{\lien{ⓔtɯ-spra}{tɯ-spra}}\end{entrée}

\begin{entrée}{rɤspɯ}{}{ⓔrɤspɯ} 
\classe{vi}  
\grammaire{denom} \paradigme{dir}{tɤ-}
\begin{définition}\pfra{laisser couler du pus}\end{définition}
\begin{définition}\pcmn{化脓}\end{définition}
\begin{exemple}\pjya{a-jaʁ to-rɤspɯ}\hspace{5pt}\pcmn{我的手化脓了}\end{exemple}\relationsémantique{参考}{\lien{ⓔtɤ-spɯ}{tɤ-spɯ}}\end{entrée}

\begin{entrée}{rɤsqa}{}{ⓔrɤsqa} 
\classe{n} 
\begin{définition}\pfra{navet cuit}\end{définition}
\begin{définition}\pcmn{煮熟了的圆根}\end{définition}\relationsémantique{参考}{\lien{ⓔrasti}{rasti}}\end{entrée}

\begin{entrée}{rɤstu}{}{ⓔrɤstu} 
\classe{vs} \paradigme{dir}{tɤ-}
\begin{définition}\pfra{honnête}\end{définition}
\begin{définition}\pcmn{老实}\end{définition}
\begin{exemple}\pjya{kɯ-rɤstu ci ɲɯ-ŋu, ɯ-stu tu-ti ɲɯ-ɕti}\hspace{5pt}\pcmn{他是个老实人,他说真话}\end{exemple}\relationsémantique{参考}{\lien{ⓔɯ-stuⓗ1}{ɯ-stu₁}}\end{entrée}

\begin{entrée}{rɤsta}{}{ⓔrɤsta} 
\classe{vi} \paradigme{dir}{kɤ-}
\begin{définition}\pfra{immobile, fixé, rester à un endroit}\end{définition}
\begin{définition}\pcmn{固定;长期;待在一个地方不动;定型}\end{définition}
\begin{exemple}\pjya{ɯ-kɯ-rɤma kɯ-rɤsta ʑo ŋu}\hspace{5pt}\pcmn{他长期给他做事}\end{exemple}
\begin{exemple}\pjya{nɤʑo nɯtɕu kɤ-rɤsta}\hspace{5pt}\pcmn{你待在那里不要动}\end{exemple}
\begin{sous-entrée}{zrɤsta}{ⓔrɤstaⓝzrɤsta} 
\classe{vt} 
\begin{définition}\pfra{fixer}\end{définition}
\begin{définition}\pcmn{固定}\end{définition}\end{sous-entrée}

\end{entrée}

\begin{entrée}{rɤstɯm}{}{ⓔrɤstɯm} 
\classe{vt} \paradigme{dir}{tɤ-}
\begin{définition}\pfra{enrouler une corde autour de son bras pour la ranger}\end{définition}
\begin{définition}\pcmn{把凌乱的绳子收拢整齐}\end{définition}
\begin{exemple}\pjya{tɯmbri tɤ-rɤstɯm-a}\hspace{5pt}\pcmn{我把绳子收拢整齐}\end{exemple}\relationsémantique{同义词}{\lien{ⓔrɤlkɯɣ}{rɤlkɯɣ}}\relationsémantique{参考}{\lien{ⓔstɯm}{stɯm}}\end{entrée}

\begin{entrée}{rɤt}{}{ⓔrɤt} 
\classe{vt} \paradigme{dir}{pɯ-}\sens{1}
\begin{définition}\pfra{écrire}\end{définition}
\begin{définition}\pcmn{写}\end{définition}
\begin{exemple}\pjya{tɤ-scoz ci pɯ-rat-a}\hspace{5pt}\pcmn{我写了一封信}\end{exemple}\sens{2}
\begin{définition}\pfra{dessiner}\end{définition}
\begin{définition}\pcmn{画}\end{définition}
\begin{sous-entrée}{rɤrɤt}{ⓔrɤtⓢ2ⓝrɤrɤt} 
\classe{vi}  
\grammaire{apass} 
\begin{définition}\pfra{écrire}\end{définition}
\begin{définition}\pcmn{写字}\end{définition}\end{sous-entrée}

\begin{sous-entrée}{sɯrɤt}{ⓔrɤtⓢ2ⓝsɯrɤt} 
\classe{vt} \sens{1}
\begin{définition}\pfra{écrire/dessiner avec}\end{définition}
\begin{définition}\pcmn{用……写}\end{définition}\end{sous-entrée}

\sens{2}
\begin{définition}\pfra{faire écrire/dessiner}\end{définition}
\begin{définition}\pcmn{请……写……}\end{définition}
\begin{sous-entrée}{zrɤrɤt}{ⓔrɤtⓢ2ⓝzrɤrɤt} 
\classe{vt}  
\grammaire{apass}
\grammaire{caus} 
\begin{définition}\pfra{faire écrire, dessiner}\end{définition}
\begin{définition}\pcmn{请……写字}\end{définition}\end{sous-entrée}

\begin{sous-entrée}{arɤt}{ⓔrɤtⓢ2ⓝarɤt} 
\classe{vi}  
\grammaire{pass} 
\begin{définition}\pfra{être écrit}\end{définition}
\begin{définition}\pcmn{写着}\end{définition}
\begin{exemple}\pjya{ɯ-taʁ a-rmi arɤt}\hspace{5pt}\pcmn{上面写着我的名字}\end{exemple}\end{sous-entrée}

\end{entrée}

\begin{entrée}{rɤtɕaʁ}{}{ⓔrɤtɕaʁ} 
\classe{vt} \paradigme{dir}{pɯ-}
\begin{définition}\pfra{fouler du pied}\end{définition}
\begin{définition}\pcmn{踩}\end{définition}
\begin{exemple}\pjya{@cai ma-pɯ-tɯ-rɤtɕaʁ}\hspace{5pt}\pcmn{你别踩菜}\end{exemple}
\begin{exemple}\pjya{a-mi ma-pɯ-tɯ-rɤtɕaʁ}\hspace{5pt}\pcmn{你别踩我的脚}\end{exemple}
\begin{exemple}\pjya{mbro pɯ́-wɣ-rɤtɕaʁ-a}\hspace{5pt}\pcmn{马踩到我了}\end{exemple}
\begin{sous-entrée}{zrɤtɕaʁ}{ⓔrɤtɕaʁⓝzrɤtɕaʁ}
\begin{exemple}\pjya{wuma ʑo tɤ-mbɣom tɕe ɯ-thoʁ mɯ-pjɤ-zrɤtɕaʁ ʑo jo-rɟɯɣ}\hspace{5pt}\pcmn{他很急,脚不着地地跑了过去}\end{exemple}\end{sous-entrée}

\end{entrée}

\begin{entrée}{rɤtɕɤβ}{}{ⓔrɤtɕɤβ}\relationsémantique{参考}{\lien{ⓔtɕɤβ}{tɕɤβ}}\end{entrée}

\begin{entrée}{rɤtɕhaʁ}{}{ⓔrɤtɕhaʁ} 
\classe{vt} \paradigme{dir}{tɤ-}
\begin{définition}\pfra{attacher en enroulant}\end{définition}
\begin{définition}\pcmn{一把一把地捆起来}\end{définition}\end{entrée}

\begin{entrée}{rɤtɕɯmtɕaʁ}{}{ⓔrɤtɕɯmtɕaʁ} 
\classe{vt} \paradigme{dir}{pɯ-}
\begin{définition}\pfra{piétiner et écraser}\end{définition}
\begin{définition}\pcmn{乱踩(植物)}\end{définition}
\begin{exemple}\pjya{tɯji ɯ-ŋgɯ ma-jɤ-tɯ-ɕe-nɯ ma, tɤ-rɤku tɯ-rɤtɕɯmtɕaʁ-nɯ, tɕe tu-wxti mɤ-cha}\hspace{5pt}\pcmn{你们不要到地里去,会踩到庄稼,以后就长不大}\end{exemple}\relationsémantique{参考}{\lien{ⓔrɤtɕaʁ}{rɤtɕaʁ}}\end{entrée}

\begin{entrée}{rɤtɣa}{}{ⓔrɤtɣa} 
\classe{vt}  
\grammaire{denom} \paradigme{dir}{nɯ-}\paradigme{dir}{thɯ-}
\begin{définition}\pfra{mesurer avec deux doigts}\end{définition}
\begin{définition}\pcmn{用大拇指和食指量尺寸}\end{définition}
\begin{exemple}\pjya{thɯ-rɤtɣa-t-a}\hspace{5pt}\pcmn{我量了木头}\end{exemple}
\begin{exemple}\pjya{tɯmbri nɯ-rɤtɣa-t-a}\hspace{5pt}\pcmn{我量了绳子}\end{exemple}\relationsémantique{同义词}{\lien{ⓔrɤɟom}{rɤɟom}}\relationsémantique{参考}{\lien{ⓔtɯ-tɣaⓗ1}{tɯ-tɣa₁}}\end{entrée}

\begin{entrée}{rɤthu}{}{ⓔrɤthu}\relationsémantique{参考}{\lien{ⓔthuⓗ1}{thu₁}}\end{entrée}

\begin{entrée}{rɤtho}{}{ⓔrɤtho} 
\classe{vi} \paradigme{dir}{lɤ-}\paradigme{dir}{tɤ-}
\begin{définition}\pfra{pousser des bourgeons}\end{définition}
\begin{définition}\pcmn{长蓓蕾}\end{définition}\relationsémantique{参考}{\lien{ⓔɯ-tho}{ɯ-tho}}\end{entrée}

\begin{entrée}{rɤthuthe}{}{ⓔrɤthuthe} 
\classe{vi} \paradigme{dir}{nɯ-}
\begin{définition}\pfra{demander la permission}\end{définition}
\begin{définition}\pcmn{征求……的同意}\end{définition}
\begin{exemple}\pjya{a-ɕki ɣɯ-nɯ-rɤthuthe}\hspace{5pt}\pcmn{我征求了我的同意}\end{exemple}
\begin{exemple}\pjya{mɤ-kɤ-rɤthuthe kɯ jo-ɣi}\hspace{5pt}\pcmn{他没有征求同意就来了}\end{exemple}\relationsémantique{参考}{\lien{ⓔthuⓗ2}{thu}}\end{entrée}

\begin{entrée}{rɤtsɣe}{}{ⓔrɤtsɣe} 
\classe{vi}  
\grammaire{apass} \paradigme{dir}{tɤ-}\paradigme{dir}{nɯ-}
\begin{définition}\pfra{vendre}\end{définition}
\begin{définition}\pcmn{卖}\end{définition}
\begin{exemple}\pjya{tɤ-rɤtsɣe-tɕi}\hspace{5pt}\pcmn{我们俩卖了东西}\end{exemple}
\begin{exemple}\pjya{jiɕqha nɯ ɲɯ-rɤtsɣe}\hspace{5pt}\pcmn{那个人在卖东西}\end{exemple}\relationsémantique{参考}{\lien{ⓔntsɣe}{ntsɣe}}\end{entrée}

\begin{entrée}{rɤtshɤt}{}{ⓔrɤtshɤt} 
\classe{vt} \paradigme{dir}{tɤ-}
\begin{définition}\pfra{essayer, comparer}\end{définition}
\begin{définition}\pcmn{试一下,比一下}\end{définition}
\begin{exemple}\pjya{tɤ-rɤtshat-a}\hspace{5pt}\pcmn{我试了一下}\end{exemple}
\begin{exemple}\pjya{ɯʑo kɯ tɯ-ŋga ta-rɤtshɤt}\hspace{5pt}\pcmn{他试了衣服}\end{exemple}\relationsémantique{参考}{\lien{ⓔtshɤtⓗ1}{tshɤt₁}}\end{entrée}

\begin{entrée}{rɤtshɯɣ}{}{ⓔrɤtshɯɣ} 
\classe{vi} \paradigme{dir}{tɤ-}
\begin{définition}\pfra{faire avec modération}\end{définition}
\begin{définition}\pcmn{小心做,适当地做}\end{définition}
\begin{exemple}\pjya{cha kɤ-tshi tu-rɤtshɯɣ-a ɕti}\hspace{5pt}\pcmn{我小心不要喝太多酒}\end{exemple}\end{entrée}

\begin{entrée}{rɤtʂɯβ}{}{ⓔrɤtʂɯβ}\relationsémantique{参考}{\lien{ⓔtʂɯβ}{tʂɯβ}}\end{entrée}

\begin{entrée}{rɤwaŋ}{}{ⓔrɤwaŋ} 
\classe{n} 
\begin{définition}\pfra{responsabilité}\end{définition}
\begin{définition}\pcmn{责任}\end{définition}
\begin{exemple}\pjya{ɯʑo to-ngo ri, aʑo a-rɤwaŋ pɯ-me}\hspace{5pt}\pcmn{虽然他生病了,但是我没有责任}\end{exemple}\étymologie{raŋ.dbaŋ}\end{entrée}

\begin{entrée}{rɤwum}{}{ⓔrɤwum} 
\classe{vt} \paradigme{dir}{tɤ-}
\begin{définition}\pfra{ranger, rassembler (des objets dispersés)}\end{définition}
\begin{définition}\pcmn{收拾;收拢}\end{définition}
\begin{exemple}\pjya{laχtɕha tɤ-rɤwum}\hspace{5pt}\pcmn{你收拾一下东西}\end{exemple}
\begin{exemple}\pjya{tɯ-ŋga tɤ-rɤwum}\hspace{5pt}\pcmn{你收拾一下衣服}\end{exemple}
\begin{exemple}\pjya{fsapaʁ tɤ-rɤwum}\hspace{5pt}\pcmn{你把牲畜赶回来}\end{exemple}
\begin{exemple}\pjya{ɯ-ndo tɕe kɤ-rɤwum mɤ-kɯ-sɤcha ʑo ɲɯ-βze ŋu}\hspace{5pt}\pcmn{到最后就不可收拾}\end{exemple}\relationsémantique{参考}{\lien{ⓔwum}{wum}}\relationsémantique{参考}{\lien{ⓔstɯm}{stɯm}}\end{entrée}

\begin{entrée}{rɤχpɯn}{}{ⓔrɤχpɯn} 
\classe{vi}  
\grammaire{denom} \paradigme{dir}{lɤ-}
\begin{définition}\pfra{devenir moine}\end{définition}
\begin{définition}\pcmn{当和尚}\end{définition}
\begin{exemple}\pjya{kɤ-rɤχpɯn mɯ́j-nɤlɤn}\hspace{5pt}\pcmn{他不答应当和尚}\end{exemple}\relationsémantique{参考}{\lien{ⓔχpɯn}{χpɯn}}\relationsémantique{参考}{\lien{ⓔnɯχpɯn}{nɯχpɯn}}\end{entrée}

\begin{entrée}{rɤzboʁ}{}{ⓔrɤzboʁ} 
\classe{vt}  
\grammaire{denom} \paradigme{dir}{tɤ-}
\begin{définition}\pfra{prendre par poignées}\end{définition}
\begin{définition}\pcmn{一把一把地拿}\end{définition}
\begin{exemple}\pjya{sɯjno tɤ-rɤzboʁ-a}\hspace{5pt}\pcmn{我一把一把地拿了草}\end{exemple}\relationsémantique{同义词}{\lien{ⓔrɤspra}{rɤspra}}\relationsémantique{参考}{\lien{ⓔtɯ-zboʁ}{tɯ-zboʁ}}\end{entrée}

\begin{entrée}{rɤzda}{}{ⓔrɤzda} 
\classe{vt} \paradigme{dir}{tɤ-}
\begin{définition}\pfra{saluer (avant le départ)}\end{définition}
\begin{définition}\pcmn{打招呼}\end{définition}
\begin{exemple}\pjya{ɕ-ta-rɤzda}\hspace{5pt}\pcmn{他向他打了招呼}\end{exemple}
\begin{exemple}\pjya{tɤ-tɯ-rɤŋgat tɕe, ɯʑo ɕ-tɤ-rɤzde je}\hspace{5pt}\pcmn{你要出发的时候,给他打个招呼}\end{exemple}
\begin{exemple}\pjya{tɯ-ɕe tɤ-mda tɕe, a-ɣɯ-tɤ-kɯ-rɤzda-a je}\hspace{5pt}\pcmn{当你要走的时候,来跟我打一声招呼}\end{exemple}\relationsémantique{参考}{\lien{ⓔnɤzda}{nɤzda}}\relationsémantique{参考}{\lien{ⓔɣɤzda}{ɣɤzda}}\relationsémantique{参考}{\lien{ⓔtɯ-zda}{tɯ-zda}}\end{entrée}

\begin{entrée}{rɤzga}{}{ⓔrɤzga} 
\classe{vi}  
\grammaire{denom} \paradigme{dir}{kɤ-}\sens{1}
\begin{définition}\pfra{faire du miel}\end{définition}
\begin{définition}\pcmn{酿蜜}\end{définition}\sens{2}
\begin{définition}\pfra{butiner}\end{définition}
\begin{définition}\pcmn{采蜜}\end{définition}
\begin{exemple}\pjya{ɣʑo ɲɯ-rɤzga}\hspace{5pt}\pcmn{蜜蜂在酿蜜}\end{exemple}
\begin{exemple}\pjya{jiɕqha tɯrme nɯ kɯ-rɯkɯŋu kɯ ɣʑo kɯ-rɤzga ʑo ɲɯ-fse}\hspace{5pt}\pcmn{这个人把所有的财物都带回家,跟蜜蜂一样}\end{exemple}\relationsémantique{参考}{\lien{ⓔzgaⓗ2}{zga₂}}\end{entrée}

\begin{entrée}{rɤzgɤr}{}{ⓔrɤzgɤr} 
\classe{vi}  
\grammaire{denom} \paradigme{dir}{thɯ-}\sens{1}
\begin{définition}\pfra{planter une tente}\end{définition}
\begin{définition}\pcmn{搭帐篷}\end{définition}
\begin{exemple}\pjya{staχpɯ cho-rɤzgɤr}\hspace{5pt}\pcmn{豌豆到处蔓延(在其他地方“搭帐篷”)}\end{exemple}\sens{2}
\begin{définition}\pfra{réparer une tente}\end{définition}
\begin{définition}\pcmn{缝帐篷}\end{définition}
\begin{exemple}\pjya{iʑora kɯre ku-rɤzgɤr-i}\hspace{5pt}\pcmn{我们在缝帐篷}\end{exemple}\relationsémantique{参考}{\lien{ⓔzgɤr}{zgɤr}}\end{entrée}

\begin{entrée}{rɤznde}{}{ⓔrɤznde} 
\classe{vi}  
\grammaire{apass} \paradigme{dir}{tɤ-}
\begin{définition}\pfra{construire en empilant}\end{définition}
\begin{définition}\pcmn{垒起来}\end{définition}
\begin{exemple}\pjya{tɤ-rɤznde-a (=znde tɤ-βzu-t-a)}\hspace{5pt}\pcmn{我修了墙}\end{exemple}\relationsémantique{参考}{\lien{ⓔzndeⓗ1}{znde₁}}\relationsémantique{参考}{\lien{ⓔzndeⓗ2}{znde₂}}\end{entrée}

\begin{entrée}{rɤʑa}{}{ⓔrɤʑa} 
\classe{vs} \paradigme{dir}{tɤ-}
\begin{définition}\pfra{gratter}\end{définition}
\begin{définition}\pcmn{痒}\end{définition}
\begin{exemple}\pjya{ɯ-mgɯr ɲɯ-rɤʑa}\hspace{5pt}\pcmn{他背部很痒}\end{exemple}
\begin{exemple}\pjya{βɣɤrtshi kɯ tu-kɯ-ndza tɕe ɯ-sta ɲɯ-rɤʑa}\hspace{5pt}\pcmn{被蚊子咬就很痒}\end{exemple}\end{entrée}

\begin{entrée}{rɤʑi/\variante{rɤʑit}}{}{ⓔrɤʑi} 
\classe{vi} \paradigme{dir}{kɤ-}\sens{1}
\begin{définition}\pfra{rester}\end{définition}
\begin{définition}\pcmn{留下}\end{définition}
\begin{exemple}\pjya{nɯtɕu ko-rɤʑi}\hspace{5pt}\pcmn{他留在那里了}\end{exemple}
\begin{exemple}\pjya{kɤ-rɤʑit-a}\hspace{5pt}\pcmn{我呆了}\end{exemple}
\begin{exemple}\pjya{ʁnɯ-sla pɯ-rɤʑi-nɯ}\hspace{5pt}\pcmn{他们呆了两个月}\end{exemple}\sens{2}
\begin{définition}\pfra{se trouver à un certain endroit}\end{définition}
\begin{définition}\pcmn{在某个地方}\end{définition}
\begin{exemple}\pjya{ŋotɕu ku-tɯ-rɤʑi?}\hspace{5pt}\pcmn{你在哪里?}\end{exemple}
\begin{sous-entrée}{ɯ-taʁ,rɤʑi}{ⓔrɤʑiⓢ2ⓝɯ-taʁ,rɤʑi}
\begin{définition}\pfra{dépendre de ... pour vivre}\end{définition}
\begin{définition}\pcmn{靠……维持生活}\end{définition}
\begin{exemple}\pjya{lu-rɤji, paʁ pjɯ-χse, nɯnɯ ɯ-taʁ ku-rɤʑi ɕti kɯmaʁ ɯ-phoʁ kɯ-tu me}\hspace{5pt}\pcmn{他靠种地和喂猪维持生活,没有其它收入来源}\end{exemple}\end{sous-entrée}

\begin{sous-entrée}{nɯrɤʑi}{ⓔrɤʑiⓢ2ⓝnɯrɤʑi} 
\grammaire{autoben} 
\begin{définition}\pfra{se reposer chez soi}\end{définition}
\begin{définition}\pcmn{在家休息}\end{définition}
\begin{exemple}\pjya{jisŋi kɤ-nɯrɤʑi}\hspace{5pt}\pcmn{今天呆在家里吧}\end{exemple}
\begin{exemple}\pjya{kɤ-nɯrɤʑi-a}\hspace{5pt}\pcmn{我呆在家里了}\end{exemple}
\begin{exemple}\pjya{ʁnɯ-sŋi, χsɯ-sŋi jamar ma mɤ-rɤʑi}\hspace{5pt}\pcmn{他只待两三天(就回来)}\end{exemple}\relationsémantique{参考}{\lien{ⓔzrɤʑi}{zrɤʑi}}\end{sous-entrée}

\end{entrée}

\begin{entrée}{rbɤrbɤt}{}{ⓔrbɤrbɤt} 
\classe{idph.2} 
\begin{définition}\pfra{ordonné}\end{définition}
\begin{définition}\pcmn{形容排列整齐的样子}\end{définition}
\begin{exemple}\pjya{tɯrme ra rbɤrbɤt ʑo ɲɯ-ndzur-nɯ}\hspace{5pt}\pcmn{人站得很整齐}\end{exemple}
\begin{exemple}\pjya{khɯtsa ɲɯ-ɤʑɯrja rbɤrbɤt ʑo}\hspace{5pt}\pcmn{碗摆得很整齐}\end{exemple}\end{entrée}

\begin{entrée}{rboʁrboʁ}{}{ⓔrboʁrboʁ} 
\classe{idph.2} 
\begin{définition}\pfra{en grappe}\end{définition}
\begin{définition}\pcmn{形容果子等集中、不分散的样子}\end{définition}
\begin{exemple}\pjya{@putao ɣɯ ɯ-mat nɯ rboʁrboʁ kɯ-pa ʑo ko-tshoʁ}\hspace{5pt}\pcmn{葡萄结得又多又密}\end{exemple}\end{entrée}

\begin{entrée}{rbɯɣrbɯɣ}{}{ⓔrbɯɣrbɯɣ} 
\classe{idph.2} 
\begin{définition}\pfra{l'un a la suite de l'autre, très nombreux (graines dans une cosse)}\end{définition}
\begin{définition}\pcmn{形容果子很多,一个接着一个}\end{définition}
\begin{exemple}\pjya{staχpɯcɤβ rbɯɣrbɯɣ ʑo chɤ-bzu}\hspace{5pt}\pcmn{豌豆长了很多果子}\end{exemple}\end{entrée}

\begin{entrée}{rca}{}{ⓔrca} 
\classe{interj} 
\begin{définition}\pfra{expression d'une surprise, (oui, bien sûr)}\end{définition}
\begin{définition}\pcmn{表示惊叹语气}\end{définition}
\begin{exemple}\pjya{wo, ɲɯ-ŋu rca!}\hspace{5pt}\pcmn{对,是的!}\end{exemple}\end{entrée}

\begin{entrée}{rcánɯ}{}{ⓔrcánɯ} 
\classe{adv} 
\begin{définition}\pfra{exprime la surprise; utilisé avec l'irréel, exprime qu'il n'y a pas d'espoir pour que l'action se réalise}\end{définition}
\begin{définition}\pcmn{表示惊叹语气的助词(和非实然式合用时表示“没有希望实现”)}\end{définition}
\begin{exemple}\pjya{nɯ rcánɯ ɲɯ-pe ko}\hspace{5pt}\pcmn{这倒是很好}\end{exemple}
\begin{exemple}\pjya{ɯʑo rcánɯ tɯ-tʂɯβ ri chɯ-βze, ɕoŋβzu ri ɲɯ-βze, smɤnba ri tu-βze, mɤ-spe ʑo me}\hspace{5pt}\pcmn{他又搞裁缝又做木工还要行医,什么都会}\end{exemple}
\begin{exemple}\pjya{nɤʑo rcánɯ ci lɤ-tɯ-nɤrʑaʁ}\hspace{5pt}\pcmn{你倒是呆了很久}\end{exemple}\end{entrée}

\begin{entrée}{rcaŋ}{}{ⓔrcaŋ} 
\classe{idph.1} 
\begin{définition}\pfra{tomber sur le dos}\end{définition}
\begin{définition}\pcmn{仰着跌倒的样子}\end{définition}
\begin{exemple}\pjya{rcaŋ ʑo kɤ-rŋgɯ}\hspace{5pt}\pcmn{他仰着躺下了}\end{exemple}
\begin{exemple}\pjya{rcaŋ ʑo pɯ-ndʐaβa}\hspace{5pt}\pcmn{我仰着跌倒了}\end{exemple}
\begin{sous-entrée}{ɣɤrcaŋlaŋ}{ⓔrcaŋⓝɣɤrcaŋlaŋ} 
\classe{vi} 
\begin{exemple}\pjya{ɲɯ-ɣɤrcaŋlaŋ ntsɯ}\hspace{5pt}\pcmn{他脚不停地在动,坐不住}\end{exemple}\end{sous-entrée}

\end{entrée}

\begin{entrée}{rcaŋpɕaʁ}{}{ⓔrcaŋpɕaʁ} 
\classe{n} 
\begin{définition}\pfra{prosternations le long d'une route jusqu'à un lieu saint}\end{définition}
\begin{définition}\pcmn{朝圣;磕长头}\end{définition}
\begin{exemple}\pjya{ɬasa rcaŋpɕaʁ to-tsɯm}\hspace{5pt}\pcmn{他去拉萨朝圣}\end{exemple}\relationsémantique{参考}{\lien{ⓔrɯrcaŋpɕaʁ}{rɯrcaŋpɕaʁ}}\étymologie{skʲaŋ.pʰʲag}\end{entrée}

\begin{entrée}{rcaχtoŋ}{}{ⓔrcaχtoŋ} 
\classe{intj} 
\begin{définition}\pfra{mange de la merde !}\end{définition}
\begin{définition}\pcmn{一句脏话}\end{définition}\étymologie{skʲag.gtoŋ}\end{entrée}

\begin{entrée}{rcɤlpa}{}{ⓔrcɤlpa} 
\classe{n} 
\begin{définition}\pfra{estomac de bovidé}\end{définition}
\begin{définition}\pcmn{牛胃}\end{définition}\étymologie{rkʲal.pa}\end{entrée}

\begin{entrée}{rcɤmbeŋga}{}{ⓔrcɤmbeŋga} 
\classe{n} 
\begin{définition}\pfra{personne qui porte une veste usée}\end{définition}
\begin{définition}\pcmn{穿破旧皮袄的人}\end{définition}\relationsémantique{参考}{\lien{ⓔtɯ-rcɤmbe}{tɯ-rcɤmbe}}\relationsémantique{参考}{\lien{ⓔtɯ-rcu}{tɯ-rcu}}\relationsémantique{参考}{\lien{ⓔtɤ-mbe}{tɤ-mbe}}\end{entrée}

\begin{entrée}{rcharcha/\variante{rcarca}}{}{ⓔrcharcha} 
\classe{idph.2} 
\begin{définition}\pfra{souriant}\end{définition}
\begin{définition}\pcmn{笑嘻嘻}\end{définition}
\begin{exemple}\pjya{tɤ-pɤtso rcharcha ɲɯ-nɤre}\hspace{5pt}\pcmn{小孩子是笑嘻嘻的}\end{exemple}\end{entrée}

\begin{entrée}{rchɤrchɤt}{}{ⓔrchɤrchɤt} 
\classe{idph.2} 
\begin{définition}\pfra{cliquetis}\end{définition}
\begin{définition}\pcmn{敲玻璃、铁皮的叮当声}\end{définition}
\begin{sous-entrée}{ɣɤrchɤrchɤt}{ⓔrchɤrchɤtⓝɣɤrchɤrchɤt} 
\classe{vi} \end{sous-entrée}

\begin{sous-entrée}{sɤrchɤrchɤt}{ⓔrchɤrchɤtⓝsɤrchɤrchɤt} 
\classe{vt} 
\begin{exemple}\pjya{tɤ-pɤtso kɯ ɯ-kɯmtɕhɯ ɲɯ-ɤz-nɤkhɯkhrɯt ɲɯ-sɤrchɤrchɤt}\hspace{5pt}\pcmn{小孩子把玩具拖着,发出叮当声}\end{exemple}
\begin{exemple}\pjya{tɤmɯmɯm ɲɯ-sɤrchɤrchɤt}\hspace{5pt}\pcmn{铃铛发出叮当声}\end{exemple}\relationsémantique{参考}{\lien{ⓔrkhɯβrkhɯβ}{rkhɯβrkhɯβ}}\end{sous-entrée}

\end{entrée}

\begin{entrée}{rchɯɣnɤlɯɣ}{}{ⓔrchɯɣnɤlɯɣ}\relationsémantique{参考}{\lien{ⓔrchɯɣrchɯɣ}{rchɯɣrchɯɣ}}\end{entrée}

\begin{entrée}{rchɯɣrchɯɣ}{}{ⓔrchɯɣrchɯɣ} 
\classe{idph.2} \sens{1}
\begin{définition}\pfra{ridé}\end{définition}
\begin{définition}\pcmn{形容皱纹很多的样子}\end{définition}
\begin{exemple}\pjya{ɯ-rŋa tɤrʑɯɣ rchɯɣrchɯɣ ʑo to-βzu}\hspace{5pt}\pcmn{他脸上长了很多皱纹}\end{exemple}\sens{2}\paradigme{dir}{nɯ-}\paradigme{dir}{tɤ-}\paradigme{dir}{tɤ-}
\begin{définition}\pfra{beaucoup de ... ensemble}\end{définition}
\begin{définition}\pcmn{形容很多(人;动物)在一起}\end{définition}
\begin{définition}\pfra{dire d'une même voix}\end{définition}
\begin{définition}\pcmn{异口同声}\end{définition}
\begin{définition}\pfra{entrechoquer en faisant du bruit (objets durs)}\end{définition}
\begin{définition}\pcmn{使硬的东西相碰发出声音}\end{définition}
\begin{définition}\pfra{s'entrechoquer en faisant du bruit (objets durs)}\end{définition}
\begin{définition}\pcmn{硬的东西相碰发出声音}\end{définition}
\begin{exemple}\pjya{rchɯɣrchɯɣ ʑo ku-tɯ-rɤʑi-nɯ ma mɯ́j-tɯ-rɤma-nɯ}\hspace{5pt}\pcmn{你们在这里这么多人,什么都不做}\end{exemple}
\begin{exemple}\pjya{ɲɤ-nɯrchɯrchɯɣ-ndʑi ʑo tɕe "aʑo ɲɯ-ɣɤŋgi" ɲɯ-ti-ndʑi}\hspace{5pt}\pcmn{她们俩异口同声地说我是对的}\end{exemple}
\begin{exemple}\pjya{ɕɤrɯ ra ɲɯ-sɤrchɯɣlɯɣ-nɯ ntsɯ}\hspace{5pt}\pcmn{不停地令骨头相撞发出声音}\end{exemple}
\begin{exemple}\pjya{jiɕqha mkhɯrlu nɯ ɲɯ-ɣɤrchɯɣrchɯɣ}\hspace{5pt}\pcmn{很多转经筒同时转动发出声音}\end{exemple}\relationsémantique{参考}{\lien{ⓔɣɤrqhɯβrqhɯβ}{ɣɤrqhɯβrqhɯβ}}
\begin{sous-entrée}{ɣɤrchɯɣlɯɣ}{ⓔrchɯɣrchɯɣⓢ2ⓝɣɤrchɯɣlɯɣ} 
\classe{vi} 
\begin{définition}\pfra{rigoler ensemble}\end{définition}
\begin{définition}\pcmn{嬉皮笑脸;说说笑笑}\end{définition}
\begin{exemple}\pjya{jiɕqha tɯrme ni ɲɯ-ɣɤrchɯɣlɯɣ-ndʑi}\hspace{5pt}\pcmn{这两个人在说说笑笑}\end{exemple}\end{sous-entrée}

\begin{sous-entrée}{nɯrchɯrchɯɣ}{ⓔrchɯɣrchɯɣⓢ2ⓝnɯrchɯrchɯɣ} 
\classe{vt} \end{sous-entrée}

\begin{sous-entrée}{sɤrchɯɣlɯɣ}{ⓔrchɯɣrchɯɣⓢ2ⓝsɤrchɯɣlɯɣ} 
\classe{vt} \end{sous-entrée}

\begin{sous-entrée}{ɣɤrchɯɣrchɯɣ}{ⓔrchɯɣrchɯɣⓢ2ⓝɣɤrchɯɣrchɯɣ} 
\classe{vt} \end{sous-entrée}

\end{entrée}

\begin{entrée}{rcirci/\variante{rchirchi}}{}{ⓔrcirci} 
\classe{idph.2} 
\begin{définition}\pfra{être souriant}\end{définition}
\begin{définition}\pcmn{形容笑得很高兴的模样}\end{définition}
\begin{exemple}\pjya{rchirchi ɲɯ-nɤre}\hspace{5pt}\pcmn{他笑得很高兴}\end{exemple}
\begin{exemple}\pjya{ɯʑo ɲɯ-rga tɕe rcirci ʑo ɲɯ-ʑɣɤstu}\hspace{5pt}\pcmn{因为他很高兴,笑得牙齿都露出来了}\end{exemple}\relationsémantique{参考}{\lien{ⓔrcharcha}{rcharcha}}\relationsémantique{参考}{\lien{ⓔmɯɣmɯɣ}{mɯɣmɯɣ}}\end{entrée}

\begin{entrée}{rcɯɣrcɯɣ}{}{ⓔrcɯɣrcɯɣ} 
\classe{idph.2} 
\begin{définition}\pfra{rire sans s'arrêter}\end{définition}
\begin{définition}\pcmn{形容笑个不停的样子}\end{définition}
\begin{exemple}\pjya{mɤ-tɯ-ɣɤrcɯɣrcɯɣ}\hspace{5pt}\pcmn{你不要笑个不停(女孩子一起说说笑笑时这么说)}\end{exemple}\end{entrée}

\begin{entrée}{rɕɯβrɕɯβ}{}{ⓔrɕɯβrɕɯβ} 
\classe{idph.2} 
\begin{définition}\pfra{rugueux}\end{définition}
\begin{définition}\pcmn{形容粗糙(像砂纸一样),快要脱皮的样子}\end{définition}
\begin{exemple}\pjya{tɯrgi ɯ-rqhu rɕɯβrɕɯβ ʑo ɲɯ-pa}\hspace{5pt}\pcmn{杉树的树皮很粗糙,快要脱皮的样子}\end{exemple}\relationsémantique{参考}{\lien{ⓔɣɤrɕɯrɕɯβⓢ2ⓝsɤrɕɯβrɕɯβ}{sɤrɕɯβrɕɯβ}}\end{entrée}

\begin{entrée}{rda}{}{ⓔrda} 
\classe{n} 
\begin{définition}\pfra{signe}\end{définition}
\begin{définition}\pcmn{信号;兆头}\end{définition}
\begin{exemple}\pjya{rda to-lɤt (to-ʑmbri)}\hspace{5pt}\pcmn{他敲钟了}\end{exemple}
\begin{exemple}\pjya{ʁmaʁmi ra kɤ-rɤru tɤ-mda tɕe ɯ-rda tu-lɤt-nɯ ŋgrɤl}\hspace{5pt}\pcmn{军人起床的时间到了就吹起床号}\end{exemple}\étymologie{brda}\end{entrée}

\begin{entrée}{rdardɯl}{}{ⓔrdardɯl} 
\classe{n}  
\grammaire{n.rdpl} 
\begin{définition}\pfra{poussières}\end{définition}
\begin{définition}\pcmn{灰尘}\end{définition}\relationsémantique{参考}{\lien{ⓔrdɯl}{rdɯl}}\end{entrée}

\begin{entrée}{rdɤβ}{}{ⓔrdɤβ} 
\classe{vi} \paradigme{dir}{thɯ-}
\begin{définition}\pfra{échouer en affaire}\end{définition}
\begin{définition}\pcmn{做亏本生意}\end{définition}
\begin{exemple}\pjya{thɯ-rdaβ-a, thɯ-rdɤβ}\hspace{5pt}\pcmn{我亏本,他亏本}\end{exemple}
\begin{exemple}\pjya{jɤxtshi tɯtsɣe thɯ-rdaβ-a}\hspace{5pt}\pcmn{我这一次做了亏本生意}\end{exemple}\étymologie{rdab.tɕʰal}\end{entrée}

\begin{entrée}{rdɤβzu}{}{ⓔrdɤβzu} 
\classe{n} 
\begin{définition}\pfra{maçon}\end{définition}
\begin{définition}\pcmn{石匠}\end{définition}
\begin{exemple}\pjya{rdɤβzu ŋu-a}\hspace{5pt}\pcmn{我是石匠}\end{exemple}\relationsémantique{参考}{\lien{ⓔrɯrdɤβzu}{rɯrdɤβzu}}\étymologie{rdo.bzu}\end{entrée}

\begin{entrée}{rdɤl}{}{ⓔrdɤl} 
\classe{vi}
\classe{vt} \paradigme{dir}{kɤ-}\paradigme{dir}{kɤ-}
\begin{définition}\pfra{aller trop loin}\end{définition}
\begin{définition}\pcmn{走过头}\end{définition}
\begin{exemple}\pjya{nɯtɕu ku-rɤʑi pɯ-ra ri, kɤ-rdɤl}\hspace{5pt}\pcmn{他本应该在这里停,但是他走过头了}\end{exemple}
\begin{exemple}\pjya{ɯʑo ɕɯŋgɯ ko-rdal-a}\hspace{5pt}\pcmn{我不小心走过他了}\end{exemple}
\begin{exemple}\pjya{aʑo kɤ-nɯʑɯβ ko-rdal-a}\hspace{5pt}\pcmn{我睡过头了}\end{exemple}
\begin{exemple}\pjya{tɯtshot ko-rdɤl}\hspace{5pt}\pcmn{时间已经超了}\end{exemple}
\begin{sous-entrée}{sɯrdɤl}{ⓔrdɤlⓝsɯrdɤl}\end{sous-entrée}

\begin{exemple}\pjya{tɯtshot ko-sɯrdɤl-tɕi}\hspace{5pt}\pcmn{我们已经超了时间了}\end{exemple}\end{entrée}

\begin{entrée}{rdɤmbɯm}{}{ⓔrdɤmbɯm} 
\classe{n} 
\begin{définition}\pfra{tas de pierre, symbole bouddhique}\end{définition}
\begin{définition}\pcmn{马尼堆;敖包}\end{définition}
\begin{exemple}\pjya{rdɤmbɯm nɯ rdɤstaʁ ʁɟa ʑo kɤ-kɤ-rmbɯ ŋu tɕeri kú-wɣ-sɯ-ɤrtɯm tɕe ɯ-pɕi nɯ znde tú-wɣ-βzu ra. nɯ nɯ tɯ-ɟom ro ro jamar kɯ-mbro tɕe nɯ ɯ-taʁ nɯ tɕu nɯ sɤz kɯ-xtɕi tsa znde kú-wɣ-sɯ-ɤrtɯm ra tɕe nɯ ɯ-taʁ tɕe rdɤstaʁ kú-wɣ-rmbɯ tɕe rdɤstaʁ kɯ-wɣrum ɯ-qapi a-pɯ-dɤn tɕe pe. tɕe nɯ tɯrme tsuku kɯ rdɤmbɯm tu-ti-nɯ tsuku kɯ tshoko tu-ti-nɯ ŋu.}\hspace{5pt}\pcmn{敖包全部是用石头堆成的,但是要把它弄成圆形,在外面砌墙,要砌得一庹多高,在(石堆)上面再砌小一点的圆墙。在上面堆上石头,白石头多一点就好。有的人把敖包叫作\lien{ⓔrdɤmbɯm}{rdɤmbɯm},有的叫\lien{}{tshoko}。}\end{exemple}\étymologie{rdo.ⁿbum}\end{entrée}

\begin{entrée}{rdɤrkɤz}{}{ⓔrdɤrkɤz} 
\classe{n} 
\begin{définition}\pfra{gravure sur pierre}\end{définition}
\begin{définition}\pcmn{石刻}\end{définition}\étymologie{rdo.brkos}\end{entrée}

\begin{entrée}{rdɤstaʁ}{}{ⓔrdɤstaʁ} 
\classe{n} 
\begin{définition}\pfra{pierre}\end{définition}
\begin{définition}\pcmn{石头}\end{définition}\relationsémantique{参考}{\lien{ⓔnɯrdɤstaʁ}{nɯrdɤstaʁ}}\étymologie{rdo}\end{entrée}

\begin{entrée}{rdom}{}{ⓔrdom} 
\classe{vi} \paradigme{dir}{nɯ-}
\begin{définition}\pfra{vagabonder}\end{définition}
\begin{définition}\pcmn{流浪(带贬义)}\end{définition}
\begin{exemple}\pjya{ɲɯ-rdom}\hspace{5pt}\pcmn{他在流浪}\end{exemple}
\begin{exemple}\pjya{kɯ-nɤphɯphɯ aʁɤndɯndɤt ɲɯ-rdom}\hspace{5pt}\pcmn{乞丐到处流浪}\end{exemple}\end{entrée}

\begin{entrée}{rdoŋ}{}{ⓔrdoŋ} 
\classe{vt} \paradigme{dir}{pɯ-}
\begin{définition}\pfra{battre}\end{définition}
\begin{définition}\pcmn{打; 夯土墙}\end{définition}
\begin{exemple}\pjya{nɤ-stu tɤ-fse ma ta-rdoŋ}\hspace{5pt}\pcmn{你表现好一点,不然就打你}\end{exemple}
\begin{exemple}\pjya{caŋ pɯ-rdoŋ-a}\hspace{5pt}\pcmn{我捶了墙}\end{exemple}
\begin{exemple}\pjya{caŋ pa-rdoŋ}\hspace{5pt}\pcmn{他捶了墙}\end{exemple}\relationsémantique{同义词}{\lien{ⓔʁndɯ}{ʁndɯ}}\étymologie{rduŋ}\end{entrée}

\begin{entrée}{rdɯl}{}{ⓔrdɯl} 
\classe{n} 
\begin{définition}\pfra{poussière}\end{définition}
\begin{définition}\pcmn{尘土}\end{définition}\étymologie{rdul}\end{entrée}

\begin{entrée}{rdzardza}{}{ⓔrdzardza} 
\classe{idph.2} \sens{1}
\begin{définition}\pfra{qui n'accepte pas les critiques}\end{définition}
\begin{définition}\pcmn{形容爱顶嘴的样子}\end{définition}
\begin{exemple}\pjya{a-phe kɯ-nɯkhɤja ci rdzardza ɲɯ-tɯ-ŋu}\hspace{5pt}\pcmn{你跟我顶嘴}\end{exemple}
\begin{exemple}\pjya{nɤki tɤ-pɤtso nɯ taʁndo mɤ-tso tɕe rdzardza ʑo tu-ʑɣɤstu ɕti}\hspace{5pt}\pcmn{那个孩子不听话,总是爱顶嘴}\end{exemple}\sens{2}
\begin{définition}\pfra{plein de vie (petite plante)}\end{définition}
\begin{définition}\pcmn{形容神气勃勃的样子(小植物)}\end{définition}
\begin{sous-entrée}{rdzanɤrdza}{ⓔrdzardzaⓢ2ⓝrdzanɤrdza} 
\classe{idph.3} 
\begin{définition}\pfra{qui se battent (animaux)}\end{définition}
\begin{définition}\pcmn{形容互相打架的样子}\end{définition}
\begin{exemple}\pjya{paχtsa ni ɲɯ-ɤlɯlɤt-ndʑi rdzanɤrdza ɲɯ-ʑɣɤstu-ndʑi}\hspace{5pt}\pcmn{两个猪仔在打架}\end{exemple}
\begin{exemple}\pjya{ki khɯna ki rdzanɤrdza kɯ-ŋɤn (kɯ-ftsɤn) ci ɲɯ-ŋu}\hspace{5pt}\pcmn{这条狗很凶}\end{exemple}\end{sous-entrée}

\end{entrée}

\begin{entrée}{rdzari}{}{ⓔrdzari} 
\classe{n} 
\begin{définition}\pfra{sommet de la montagne}\end{définition}
\begin{définition}\pcmn{最高山上}\end{définition}\étymologie{rdza.ri}\end{entrée}

\begin{entrée}{rdzɯrdzi}{}{ⓔrdzɯrdzi} 
\classe{idph.2} 
\begin{définition}\pfra{pleine d'épines (plante)}\end{définition}
\begin{définition}\pcmn{有刺(植物)}\end{définition}
\begin{exemple}\pjya{zɲɟa rdzɯrdzi ɲɯ-pa}\hspace{5pt}\pcmn{黄刺泡儿有刺}\end{exemple}
\begin{exemple}\pjya{rtɕhɯʁju rdzɯrdzi ɲɯ-pa}\hspace{5pt}\pcmn{毛虫有毛}\end{exemple}\relationsémantique{参考}{\lien{ⓔrzɯrzi}{rzɯrzi}}\end{entrée}

\begin{entrée}{rga}{₁₂}{ⓔrgaⓗ1ⓗ2} 
\classe{vs}
\classe{vi-t} \paradigme{dir}{nɯ-}\paradigme{dir}{nɯ-}
\begin{définition}\pfra{être content}\end{définition}
\begin{définition}\pcmn{高兴}\end{définition}
\begin{définition}\pfra{aimer}\end{définition}
\begin{définition}\pcmn{喜欢}\end{définition}
\begin{exemple}\pjya{aʑo ɲɯ-rga-a}\hspace{5pt}\pcmn{(听到这个消息)我很高兴}\end{exemple}
\begin{exemple}\pjya{nɯ ma kɤ-nɯrga mɯ́j-khɯ ʑo ɲɯ-rga-a}\hspace{5pt}\pcmn{我无比得高兴}\end{exemple}
\begin{sous-entrée}{rga}{ⓔrgaⓗ1ⓝrga}\end{sous-entrée}

\begin{exemple}\pjya{kɤ-rɯɕmi mɯ́j-rga}\hspace{5pt}\pcmn{他不爱说话}\end{exemple}\relationsémantique{参考}{\lien{ⓔnɯrga}{nɯrga}}\relationsémantique{参考}{\lien{ⓔɕɯrga}{ɕɯrga}}\relationsémantique{参考}{\lien{ⓔnɯchɤrga}{nɯchɤrga}}\relationsémantique{参考}{\lien{ⓔnɯɕmɯrga}{nɯɕmɯrga}}\étymologie{dga}\end{entrée}

\begin{entrée}{rgali}{}{ⓔrgali} 
\classe{n} 
\begin{définition}\pfra{génisse}\end{définition}
\begin{définition}\pcmn{小奶牛}\end{définition}\end{entrée}

\begin{entrée}{rgargɯn}{}{ⓔrgargɯn} 
\classe{n}  
\grammaire{n.rdpl} 
\begin{définition}\pfra{personne âgée}\end{définition}
\begin{définition}\pcmn{老年人}\end{définition}\étymologie{dgan}\end{entrée}

\begin{entrée}{rgawa}{}{ⓔrgawa} 
\classe{n} 
\begin{définition}\pfra{cérémonie pour les morts}\end{définition}
\begin{définition}\pcmn{为死人办的仪式}\end{définition}\étymologie{dga.ba}\end{entrée}

\begin{entrée}{rgawɯ}{}{ⓔrgawɯ} 
\classe{n} 
\begin{définition}\pfra{navet séché}\end{définition}
\begin{définition}\pcmn{干了的芜菁根}\end{définition}\relationsémantique{参考}{\lien{ⓔrɤjndoʁ}{rɤjndoʁ}}\end{entrée}

\begin{entrée}{rgɤl}{}{ⓔrgɤl} 
\classe{idph.1} 
\begin{définition}\pfra{tou d'un coup}\end{définition}
\begin{définition}\pcmn{突然}\end{définition}
\begin{exemple}\pjya{rgɤl ʑo pɯ-ndʐaβ}\hspace{5pt}\pcmn{他突然间摔下去了}\end{exemple}\end{entrée}

\begin{entrée}{rgɤm}{}{ⓔrgɤm} 
\classe{n} 
\begin{définition}\pfra{boîte}\end{définition}
\begin{définition}\pcmn{箱子}\end{définition}\étymologie{sgam}\end{entrée}

\begin{entrée}{rgɤmpɯ}{}{ⓔrgɤmpɯ} 
\classe{n} 
\begin{définition}\pfra{petite boîte}\end{définition}
\begin{définition}\pcmn{小盒子}\end{définition}\relationsémantique{参考}{\lien{ⓔtɤ-pɯ}{tɤ-pɯ}}\end{entrée}

\begin{entrée}{rgɤnmɯ}{}{ⓔrgɤnmɯ} 
\classe{n} 
\begin{définition}\pfra{vieillarde}\end{définition}
\begin{définition}\pcmn{老太太}\end{définition}\étymologie{rgan.mo}\end{entrée}

\begin{entrée}{rgɤtpu}{}{ⓔrgɤtpu} 
\classe{n} 
\begin{définition}\pfra{vieillard}\end{définition}
\begin{définition}\pcmn{老人}\end{définition}\étymologie{rgad.po}\end{entrée}

\begin{entrée}{rgɤz}{}{ⓔrgɤz} 
\classe{vs} \paradigme{dir}{thɯ-}
\begin{définition}\pfra{être vieux, vieillir}\end{définition}
\begin{définition}\pcmn{年老}\end{définition}
\begin{exemple}\pjya{jiɕqha rgɤtpu nɯ cho-rgɤz}\hspace{5pt}\pcmn{那位老年人变老了}\end{exemple}
\begin{exemple}\pjya{nɯŋa do chɤ-rgɤz}\hspace{5pt}\pcmn{老奶牛变老了}\end{exemple}
\begin{exemple}\pjya{kha to-rgɤz}\hspace{5pt}\pcmn{房子老了}\end{exemple}
\begin{sous-entrée}{ɣɤrgɤz}{ⓔrgɤzⓝɣɤrgɤz} 
\classe{vs} 
\begin{définition}\pfra{qui vieillit facilement}\end{définition}
\begin{définition}\pcmn{容易衰老}\end{définition}
\begin{exemple}\pjya{tɤtho nɯ mɤ-ɣɤrgɤz ma anɯrŋi nɤ anɯrŋi ɕti, pjɯ-nɯɕɯrɲɟo mɤ-ŋgrɤl, ɯ-mdoʁ ɲɯ-nɤsci mɤ-ŋgrɤl}\hspace{5pt}\pcmn{松树不容易衰老,一直都是绿的,不变色}\end{exemple}\end{sous-entrée}

\étymologie{rgas}\end{entrée}

\begin{entrée}{rgonma}{}{ⓔrgonma} 
\classe{n} 
\begin{définition}\pfra{jument}\end{définition}
\begin{définition}\pcmn{母马}\end{définition}\étymologie{rgod.ma}\end{entrée}

\begin{entrée}{rgoʁphɤr}{}{ⓔrgoʁphɤr} 
\classe{n} 
\begin{définition}\pfra{bol en bois ayant un couvercle}\end{définition}
\begin{définition}\pcmn{有盖子的木碗}\end{définition}\end{entrée}

\begin{entrée}{rgot}{}{ⓔrgot} 
\classe{vs} 
\begin{définition}\pfra{robuste}\end{définition}
\begin{définition}\pcmn{身强力壮;不好对付}\end{définition}\étymologie{rgod}\end{entrée}

\begin{entrée}{rgɯdɯ}{}{ⓔrgɯdɯ} 
\classe{n} 
\begin{définition}\pfra{léproserie}\end{définition}
\begin{définition}\pcmn{麻风病医院}\end{définition}\end{entrée}

\begin{entrée}{rgɯkhra}{}{ⓔrgɯkhra} 
\classe{n} 
\begin{définition}\pfra{vache à pois}\end{définition}
\begin{définition}\pcmn{黑白相间;红白相间的牛}\end{définition}\étymologie{kʰra}\end{entrée}

\begin{entrée}{rgɯnba}{}{ⓔrgɯnba} 
\classe{n} 
\begin{définition}\pfra{temple}\end{définition}
\begin{définition}\pcmn{庙}\end{définition}\étymologie{dgon.pa}\end{entrée}

\begin{entrée}{rgɯni}{}{ⓔrgɯni} 
\classe{n} 
\begin{définition}\pfra{vache rousse}\end{définition}
\begin{définition}\pcmn{红毛牛}\end{définition}\end{entrée}

\begin{entrée}{rgɯnmdɯt}{}{ⓔrgɯnmdɯt} 
\classe{n} 
\begin{définition}\pfra{groupe de neuf nœuds (sur un khatag ou avec un fil normal)}\end{définition}
\begin{définition}\pcmn{在哈达上打九个结}\end{définition}\end{entrée}

\begin{entrée}{rgɯnsa}{}{ⓔrgɯnsa} 
\classe{n} 
\begin{définition}\pfra{pâturage d'hiver}\end{définition}
\begin{définition}\pcmn{冬天的牧场}\end{définition}\étymologie{dgun.sa}\end{entrée}

\begin{entrée}{rgɯntɕɯn}{}{ⓔrgɯntɕɯn} 
\classe{n} 
\begin{définition}\pfra{grand monastère}\end{définition}
\begin{définition}\pcmn{大庙}\end{définition}\end{entrée}

\begin{entrée}{rgɯpa}{}{ⓔrgɯpa} 
\classe{n} 
\begin{définition}\pfra{neuvième mois}\end{définition}
\begin{définition}\pcmn{九月}\end{définition}\étymologie{dgu.pa}\end{entrée}

\begin{entrée}{rgɯskɯ}{}{ⓔrgɯskɯ} 
\classe{n} 
\begin{définition}\pfra{vache dont la tête, le ventre et le haut du dos sont blancs, les membres et les flancs noirs}\end{définition}
\begin{définition}\pcmn{头、肚子和背梁白色,四肢和两侧黑色的牛}\end{définition}\end{entrée}

\begin{entrée}{rɣɤβrɣɤβ}{}{ⓔrɣɤβrɣɤβ} 
\classe{idph.2} 
\begin{définition}\pfra{transparent}\end{définition}
\begin{définition}\pcmn{形容透明而光亮的样子}\end{définition}
\begin{exemple}\pjya{tɤŋgɤr rɣɤβrɣɤβ ʑo ɲɯ-pa}\hspace{5pt}\pcmn{猪膘显得很透明}\end{exemple}
\begin{exemple}\pjya{ɯ-qom rɣɤβrɣɤβ ʑo to-stu}\hspace{5pt}\pcmn{(小孩子)快要眼泪哗哗的样子}\end{exemple}
\begin{sous-entrée}{ɣɤrɣɤβrɣɤβ}{ⓔrɣɤβrɣɤβⓝɣɤrɣɤβrɣɤβ} 
\classe{vi} 
\begin{exemple}\pjya{tɤŋgɤr ɲɯ-ɣɤrɣɤβrɣɤβ}\hspace{5pt}\pcmn{猪膘显得很透明}\end{exemple}\end{sous-entrée}

\begin{sous-entrée}{sɤrɣɤβrɣɤβ}{ⓔrɣɤβrɣɤβⓝsɤrɣɤβrɣɤβ} 
\classe{vt} 
\begin{exemple}\pjya{tɤ-pɤtso kɯ ɯ-qom ɲɯ-sɤrɣɤβrɣɤβ ʑo}\hspace{5pt}\pcmn{小孩子快要眼泪哗哗的样子}\end{exemple}\end{sous-entrée}

\end{entrée}

\begin{entrée}{rɣurɣu}{}{ⓔrɣurɣu} 
\classe{idph.2} 
\begin{définition}\pfra{semblable à une ampoule}\end{définition}
\begin{définition}\pcmn{形容物体像水泡的样子}\end{définition}
\begin{exemple}\pjya{qaɕpa ɯ-ŋgɯm rɣurɣu ʑo ɲɯ-pa}\hspace{5pt}\pcmn{青蛙卵又圆又透明}\end{exemple}
\begin{exemple}\pjya{a-jaʁ cɯmbɤrom rɣurɣu ʑo to-rku}\hspace{5pt}\pcmn{我手上长了水泡}\end{exemple}\end{entrée}

\begin{entrée}{ri}{}{ⓔri} 
\classe{sfp} 
\begin{définition}\pfra{et ... alors?}\end{définition}
\begin{définition}\pcmn{呢}\end{définition}
\begin{exemple}\pjya{a-wa ri?}\hspace{5pt}\pcmn{我们爸爸呢?}\end{exemple}\end{entrée}

\begin{entrée}{ri}{₁}{ⓔriⓗ1} 
\classe{postp} 
\begin{définition}\pfra{locatif}\end{définition}
\begin{définition}\pcmn{在}\end{définition}\end{entrée}

\begin{entrée}{ri}{₂}{ⓔriⓗ2} 
\classe{cnj} 
\begin{définition}\pfra{mais}\end{définition}
\begin{définition}\pcmn{但是}\end{définition}\end{entrée}

\begin{entrée}{ri}{₃}{ⓔriⓗ3} 
\classe{vi} \paradigme{dir}{nɯ-}
\begin{définition}\pfra{rester}\end{définition}
\begin{définition}\pcmn{剩下}\end{définition}
\begin{exemple}\pjya{ɯʑo nɯ-ri}\hspace{5pt}\pcmn{剩下了他}\end{exemple}
\begin{exemple}\pjya{ɯ-ro ɲɤ-ri}\hspace{5pt}\pcmn{有剩余的}\end{exemple}
\begin{exemple}\pjya{a-ʑɯβ ko-ri (=pjɤ-rtaʁ)}\hspace{5pt}\pcmn{我睡够了}\end{exemple}\relationsémantique{参考}{\lien{ⓔβri}{βri}}\relationsémantique{参考}{\lien{ⓔtɯ-sroʁ,ri}{tɯ-sroʁ,ri}}\relationsémantique{参考}{\lien{ⓔnɯʑɯβri}{nɯʑɯβri}}\end{entrée}

\begin{entrée}{riβdaʁ}{}{ⓔriβdaʁ} 
\classe{n} 
\begin{définition}\pfra{divinité des montagnes}\end{définition}
\begin{définition}\pcmn{山神}\end{définition}\étymologie{ri.bdag}\end{entrée}

\begin{entrée}{rirɤβ}{}{ⓔrirɤβ} 
\classe{n} 
\begin{définition}\pfra{haute montagne}\end{définition}
\begin{définition}\pcmn{高山}\end{définition}\étymologie{ri.rab}\end{entrée}

\begin{entrée}{rjaŋrjaŋ}{}{ⓔrjaŋrjaŋ} 
\classe{idph.2} 
\begin{définition}\pfra{long et cylindrique}\end{définition}
\begin{définition}\pcmn{长,圆柱形的}\end{définition}
\begin{sous-entrée}{mɤlɤrjaŋ}{ⓔrjaŋrjaŋⓝmɤlɤrjaŋ}
\begin{exemple}\pjya{stukɤr mɤlɤrjaŋ pjɤ-phɯt}\hspace{5pt}\pcmn{他砍了很长(的一棵树做)檩子}\end{exemple}\relationsémantique{参考}{\lien{ⓔrjoʁrjoʁ}{rjoʁrjoʁ}}\end{sous-entrée}

\end{entrée}

\begin{entrée}{rjɤrjɤt}{}{ⓔrjɤrjɤt} 
\classe{idph.2} \sens{1}
\begin{définition}\pfra{long et fin}\end{définition}
\begin{définition}\pcmn{细长}\end{définition}
\begin{exemple}\pjya{rjɤrjɤt ci ɲɯ-ŋu}\hspace{5pt}\pcmn{是细长细长的}\end{exemple}
\begin{exemple}\pjya{romɲa tsuku rjɤrjɤt pjɤ-phɯt-nɯ}\hspace{5pt}\pcmn{他们砍了许多(树,做成)小梁}\end{exemple}\sens{2}
\begin{définition}\pfra{rester peu de temps}\end{définition}
\begin{définition}\pcmn{耽误很短时间}\end{définition}
\begin{exemple}\pjya{rjɤrjɤt ci ɕ-to-ti}\hspace{5pt}\pcmn{快快当当地去说了}\end{exemple}
\begin{exemple}\pjya{rjɤrjɤt ci tɯpri ʑ-ɲɤ-nɯmnɤt}\hspace{5pt}\pcmn{快快当当地传达了口信}\end{exemple}
\begin{exemple}\pjya{rjɤrjɤt ci ɕe-a ŋu}\hspace{5pt}\pcmn{我只去耽误一会就回来}\end{exemple}
\begin{exemple}\pjya{tɯpri ɯ-kɯ-nɯmnɤt ci rjɤrjɤt jɤ-ari-a}\hspace{5pt}\pcmn{我快快当当地传达了口信}\end{exemple}
\begin{sous-entrée}{rjɤnɤrjɤt}{ⓔrjɤrjɤtⓢ2ⓝrjɤnɤrjɤt} 
\classe{idph.3} 
\begin{exemple}\pjya{qapri ci rjɤrjɤt nɤ rjɤrjɤt thɯ-ari}\hspace{5pt}\pcmn{蛇细长细长地过去了}\end{exemple}\end{sous-entrée}

\begin{sous-entrée}{rjɤnɤlɤt}{ⓔrjɤrjɤtⓢ2ⓝrjɤnɤlɤt}
\begin{définition}\pfra{long et fin, aux mouvements agiles}\end{définition}
\begin{définition}\pcmn{又细又长,动作灵活}\end{définition}
\begin{exemple}\pjya{βʑɯ nɯ kha zɯ rjɤnɤlɤt ʑo tu-ŋke ŋu}\hspace{5pt}\pcmn{老鼠,身子和尾巴又细又长,在房间里走动}\end{exemple}\end{sous-entrée}

\begin{sous-entrée}{stɤrjɤt}{ⓔrjɤrjɤtⓢ2ⓝstɤrjɤt}
\begin{définition}\pfra{long et fin, agile}\end{définition}
\begin{définition}\pcmn{又细又长;很灵活}\end{définition}
\begin{exemple}\pjya{βʑɯ stɤrjɤt ʑo nɯ-ɕqhlɤt}\hspace{5pt}\pcmn{老鼠一下子就消失了}\end{exemple}\end{sous-entrée}

\begin{sous-entrée}{ɣɤrjɤlɤt}{ⓔrjɤrjɤtⓢ2ⓝɣɤrjɤlɤt} 
\classe{vi} 
\begin{définition}\pfra{frétiller}\end{définition}
\begin{définition}\pcmn{颤动}\end{définition}
\begin{exemple}\pjya{qaɟɯ ɲɯ-ɣɤrjɤlɤt}\hspace{5pt}\pcmn{鱼在颤动}\end{exemple}\end{sous-entrée}

\begin{sous-entrée}{sɤrjɤrjɤt}{ⓔrjɤrjɤtⓢ2ⓝsɤrjɤrjɤt} 
\classe{vt} 
\begin{définition}\pfra{faire frétiller, faire vibrer}\end{définition}
\begin{définition}\pcmn{使……颤动}\end{définition}\end{sous-entrée}

\end{entrée}

\begin{entrée}{rjoʁrjoʁ}{}{ⓔrjoʁrjoʁ} 
\classe{idph.2} 
\begin{définition}\pfra{cylindrique}\end{définition}
\begin{définition}\pcmn{形容圆柱形}\end{définition}\end{entrée}

\begin{entrée}{rɟa}{}{ⓔrɟa} 
\classe{n} 
\begin{définition}\pfra{pantholops hodgsoni}\end{définition}
\begin{définition}\pcmn{藏羚}\end{définition}\étymologie{rgʲa}\end{entrée}

\begin{entrée}{rɟama}{}{ⓔrɟama} 
\classe{n} 
\begin{définition}\pfra{balance}\end{définition}
\begin{définition}\pcmn{称}\end{définition}\étymologie{rgʲa.ma}\end{entrée}

\begin{entrée}{rɟamar}{}{ⓔrɟamar} 
\classe{n} 
\begin{définition}\pfra{bovidé de couleur noire dont le haut du dos et le bout des oreilles sont marrons clairs}\end{définition}
\begin{définition}\pcmn{全身黑色,耳尖和背梁棕色的牛}\end{définition}\end{entrée}

\begin{entrée}{rɟambrɯɣ}{}{ⓔrɟambrɯɣ} 
\classe{n} 
\begin{définition}\pfra{espèce de chien dont le corps est noir et les yeux sont entourés de rouge}\end{définition}
\begin{définition}\pcmn{四眼狗}\end{définition}\étymologie{rgʲa.ⁿbrug}\end{entrée}

\begin{entrée}{rɟamtshu}{}{ⓔrɟamtshu} 
\classe{n} 
\begin{définition}\pfra{mer}\end{définition}
\begin{définition}\pcmn{海}\end{définition}\étymologie{rgʲa.mtsʰo}\end{entrée}

\begin{entrée}{rɟanaʁtɕaʁri}{}{ⓔrɟanaʁtɕaʁri} 
\classe{n} 
\begin{définition}\pfra{la grande muraille de chine}\end{définition}
\begin{définition}\pcmn{万里长城;一种佛教图纹}\end{définition}\étymologie{rgʲa.nag ltɕags.ri}\end{entrée}

\begin{entrée}{rɟandzi}{}{ⓔrɟandzi} 
\classe{n} 
\begin{définition}\pfra{bovidé de couleur noire avec une tache blanche sur le front}\end{définition}
\begin{définition}\pcmn{全身黑色的牛,在额头上带有白色}\end{définition}\end{entrée}

\begin{entrée}{rɟaŋ}{}{ⓔrɟaŋ} 
\classe{n} 
\begin{définition}\pfra{lointain}\end{définition}
\begin{définition}\pcmn{长远}\end{définition}
\begin{exemple}\pjya{rɟaŋ ɣɯ ɯ-sɯso kɤ-lɤt ra}\hspace{5pt}\pcmn{要有长远打算}\end{exemple}
\begin{exemple}\pjya{rɟaŋ ʑo z-jo-fskɤr}\hspace{5pt}\pcmn{他绕了个大圈}\end{exemple}\relationsémantique{参考}{\lien{ⓔrɯrɟaŋrɟɤz}{rɯrɟaŋrɟɤz}}\relationsémantique{参考}{\lien{ⓔnɯrɟaŋ}{nɯrɟaŋ}}\étymologie{rgʲaŋ}\end{entrée}

\begin{entrée}{rɟaŋsoʁ}{}{ⓔrɟaŋsoʁ} 
\classe{n} 
\begin{définition}\pfra{scie}\end{définition}
\begin{définition}\pcmn{锯}\end{définition}\étymologie{sog.le}\end{entrée}

\begin{entrée}{rɟara}{}{ⓔrɟara} 
\classe{n} 
\begin{définition}\pfra{cour}\end{définition}
\begin{définition}\pcmn{院子}\end{définition}\relationsémantique{同义词}{\lien{ⓔtɕhaʁla}{tɕhaʁla}}\end{entrée}

\begin{entrée}{rɟaʁ}{}{ⓔrɟaʁ} 
\classe{vi} \paradigme{dir}{pɯ-}\paradigme{dir}{kɤ-}
\begin{définition}\pfra{danser}\end{définition}
\begin{définition}\pcmn{跳舞}\end{définition}
\begin{exemple}\pjya{jiɕqha nɯ ɲɯ-rɟaʁ-nɯ}\hspace{5pt}\pcmn{他们在跳舞}\end{exemple}
\begin{exemple}\pjya{pjɯ-rɟaʁ-a ŋgrɤl, aj tɯrɟaʁ rga-a}\hspace{5pt}\pcmn{我喜欢跳舞}\end{exemple}\relationsémantique{参考}{\lien{ⓔtɯrɟaʁ}{tɯrɟaʁ}}\end{entrée}

\begin{entrée}{rɟaspa}{}{ⓔrɟaspa} 
\classe{adv} 
\begin{définition}\pfra{à peu près, plutôt}\end{définition}
\begin{définition}\pcmn{差不多;比较}\end{définition}
\begin{exemple}\pjya{nɤ-tɕhomba rɟaspa to-mna tɕe ɲɯ-pe}\hspace{5pt}\pcmn{你的感冒差不多好了}\end{exemple}\relationsémantique{同义词}{\lien{ⓔnɤkɤro}{nɤkɤro}}\end{entrée}

\begin{entrée}{rɟawu}{}{ⓔrɟawu} 
\classe{n} 
\begin{définition}\pfra{barbe}\end{définition}
\begin{définition}\pcmn{连鬓胡}\end{définition}\étymologie{rgʲa.bo}\end{entrée}

\begin{entrée}{rɟɤβlun}{}{ⓔrɟɤβlun} 
\classe{n} 
\begin{définition}\pfra{ministre}\end{définition}
\begin{définition}\pcmn{宰相;大臣}\end{définition}\étymologie{rgʲa.blon}\end{entrée}

\begin{entrée}{rɟɤɕi}{}{ⓔrɟɤɕi} 
\classe{n} 
\begin{définition}\pfra{embauchoir}\end{définition}
\begin{définition}\pcmn{楦头}\end{définition}\end{entrée}

\begin{entrée}{rɟɤdoʁ}{}{ⓔrɟɤdoʁ} 
\classe{n} 
\begin{définition}\pfra{fibres de chanvre}\end{définition}
\begin{définition}\pcmn{大麻皮(用来织麻布的纬线和经线,编绳子)}\end{définition}\end{entrée}

\begin{entrée}{rɟɤdɯm}{}{ⓔrɟɤdɯm} 
\classe{n} 
\begin{définition}\pfra{grumeaux (dans la pâte de farine)}\end{définition}
\begin{définition}\pcmn{没有搅均匀的面坨坨}\end{définition}
\begin{exemple}\pjya{tɤjlu pɯ-tɯ-lɤt tɕe koŋla nɯ-ɕmi tɕe, rɟɤdɯm a-mɤ-nɯ-βze}\hspace{5pt}\pcmn{你和面的时候,要好好搅拌不要有坨坨}\end{exemple}\end{entrée}

\begin{entrée}{rɟɤɣi}{}{ⓔrɟɤɣi} 
\classe{n} 
\begin{définition}\pfra{tsampa}\end{définition}
\begin{définition}\pcmn{糌粑的一种吃法}\end{définition}
\begin{exemple}\pjya{rɟɤɣi pɯ-nɯ-lat-a}\hspace{5pt}\pcmn{我倒了糌粑}\end{exemple}
\begin{exemple}\pjya{rɟɤɣi tɤ-nɯ-βzu-t-a}\hspace{5pt}\pcmn{我挼了糌粑}\end{exemple}
\begin{exemple}\pjya{rɟɤɣi nɯ khɯtsa ɯ-ŋgɯ rtsɤmtɕhɯ pjɯ́-wɣ-lɤt tɕe ɯ-taʁ ta-mar tɯ-snaʁ pjɯ́-wɣ-lɤt tɕe ɯ-taʁ tɯ-ɣndʑɤr khɯtsa tú-wɣ-sɯ-mtshɤt tɕe tú-wɣ-ɕmi tɕe tú-wɣ-rɤlaj tɕe tú-wɣ-ndza ŋu.}\hspace{5pt}\pcmn{\lien{ⓔrɟɤɣi}{rɟɤɣi}就是在碗里倒上糌粑水,然后放上一小块酥油,再放上满碗的糌粑然后搅匀,挼好就可以吃了。}\end{exemple}\end{entrée}

\begin{entrée}{rɟɤkɤr/\variante{rɟɤkɤrji}}{}{ⓔrɟɤkɤr} 
\classe{n} 
\begin{définition}\pfra{Inde}\end{définition}
\begin{définition}\pcmn{印度}\end{définition}
\begin{exemple}\pjya{rɟɤkɤr zɯ βlama ra kɯ srɯnmɯ ra tu-nɯkon-nɯ ɲɯ-ŋu}\hspace{5pt}\pcmn{在印度,喇嘛们把全世界的妖精管制在那里。}\end{exemple}\étymologie{rgʲa dkar}\end{entrée}

\begin{entrée}{rɟɤlkhɤβ}{}{ⓔrɟɤlkhɤβ} 
\classe{n} 
\begin{définition}\pfra{pays}\end{définition}
\begin{définition}\pcmn{国家}\end{définition}\étymologie{rgʲal.kʰab}\end{entrée}

\begin{entrée}{rɟɤlpu}{}{ⓔrɟɤlpu} 
\classe{n} 
\begin{définition}\pfra{roi}\end{définition}
\begin{définition}\pcmn{土司}\end{définition}\étymologie{rgʲal.po}\end{entrée}

\begin{entrée}{rɟɤlsa}{}{ⓔrɟɤlsa} 
\classe{n} 
\begin{définition}\pfra{palais}\end{définition}
\begin{définition}\pcmn{宫殿}\end{définition}
\begin{exemple}\pjya{tɕɯχtsi rɟɤlsa}\hspace{5pt}\pcmn{卓克基官寨}\end{exemple}\étymologie{rgʲal.sa}\end{entrée}

\begin{entrée}{rɟɤntɕa}{}{ⓔrɟɤntɕa} 
\classe{n} 
\begin{définition}\pfra{bijoux, décoration}\end{définition}
\begin{définition}\pcmn{装饰品}\end{définition}\étymologie{rgʲan.tɕʰa}\end{entrée}

\begin{entrée}{rɟɤŋgɤɣ}{}{ⓔrɟɤŋgɤɣ} 
\classe{n} 
\begin{définition}\pfra{poutre horizontale}\end{définition}
\begin{définition}\pcmn{横梁}\end{définition}
\begin{exemple}\pjya{rɟɤŋgɤɣ nɯ rɟɯɣ cho nɯ kɯ-naχtɕɯɣ nɯ}\hspace{5pt}\pcmn{\lien{ⓔrɟɤŋgɤɣ}{rɟɤŋgɤɣ}跟\lien{ⓔrɟɯɣⓗ2}{rɟɯɣ}一样}\end{exemple}\end{entrée}

\begin{entrée}{rɟɤpɕɤt}{}{ⓔrɟɤpɕɤt} 
\classe{n} 
\begin{définition}\pfra{demi livre}\end{définition}
\begin{définition}\pcmn{半斤}\end{définition}\étymologie{*rgʲa.pʰʲed}\end{entrée}

\begin{entrée}{rɟɤskɤt}{}{ⓔrɟɤskɤt} 
\classe{n} 
\begin{définition}\pfra{escalier en bois}\end{définition}
\begin{définition}\pcmn{板梯}\end{définition}\étymologie{rgʲa.skas}\end{entrée}

\begin{entrée}{rɟɤskhi}{}{ⓔrɟɤskhi} 
\classe{n} 
\begin{définition}\pfra{vannerie}\end{définition}
\begin{définition}\pcmn{簸箕}\end{définition}\end{entrée}

\begin{entrée}{rɟɤthaʁ}{}{ⓔrɟɤthaʁ} 
\classe{n} 
\begin{définition}\pfra{verrou}\end{définition}
\begin{définition}\pcmn{插销}\end{définition}\end{entrée}

\begin{entrée}{rɟɤthɤβ}{}{ⓔrɟɤthɤβ} 
\classe{n} 
\begin{définition}\pfra{four chinois}\end{définition}
\begin{définition}\pcmn{炉子}\end{définition}\étymologie{rgʲa.tʰab}\end{entrée}

\begin{entrée}{rɟɤtpa}{}{ⓔrɟɤtpa} 
\classe{n} 
\begin{définition}\pfra{huitième mois}\end{définition}
\begin{définition}\pcmn{八月}\end{définition}\étymologie{brgʲad.pa}\end{entrée}

\begin{entrée}{rɟɤtsha}{}{ⓔrɟɤtsha} 
\classe{n} 
\begin{définition}\pfra{plaque de sel}\end{définition}
\begin{définition}\pcmn{盐的大块}\end{définition}
\begin{exemple}\pjya{jiʑo pɯ-xtɕi-j tɕe, rɟɤtsha ɯ-ntɕhɯr ntsɯ tu-nɯntsɯɣ-i pɯ-ŋu}\hspace{5pt}\pcmn{我们小时候一直舔盐块}\end{exemple}\étymologie{rgʲa.tsʰʷa}\end{entrée}

\begin{entrée}{rɟɤxtsa}{}{ⓔrɟɤxtsa} 
\classe{n} 
\begin{définition}\pfra{botte à semelle épaisse}\end{définition}
\begin{définition}\pcmn{鞋底又厚又硬的靴子}\end{définition}\end{entrée}

\begin{entrée}{rɟɤz}{}{ⓔrɟɤz} 
\classe{vs} 
\begin{définition}\pfra{être développé}\end{définition}
\begin{définition}\pcmn{发达}\end{définition}
\begin{exemple}\pjya{nɤki nɯ ɯ-ɕa wuma ʑo kɯ-rɟɤz ci ɲɯ-ŋu}\hspace{5pt}\pcmn{那个人肌肉很发达}\end{exemple}
\begin{exemple}\pjya{ɯ-sɯm rɟɤz (=ɯ-ro jom; ɯ-mɲaʁsta jom)}\hspace{5pt}\pcmn{他心胸宽阔}\end{exemple}\end{entrée}

\begin{entrée}{rɟitpa}{}{ⓔrɟitpa} 
\classe{n} 
\begin{définition}\pfra{lignée, famille}\end{définition}
\begin{définition}\pcmn{家族}\end{définition}\relationsémantique{参考}{\lien{ⓔtɤ-rɟit}{tɤ-rɟit}}\étymologie{rgʲud.pa}\end{entrée}

\begin{entrée}{rɟum}{}{ⓔrɟum} 
\classe{vs} \paradigme{dir}{nɯ-}
\begin{définition}\pfra{large}\end{définition}
\begin{définition}\pcmn{宽(布,纸)}\end{définition}
\begin{exemple}\pjya{ki ɯ-spa ɲɯ-rɟum}\hspace{5pt}\pcmn{很宽}\end{exemple}\relationsémantique{反义词}{\lien{ⓔtɕɤr}{tɕɤr}}\relationsémantique{参考}{\lien{ⓔarɟumtɕɤr}{arɟumtɕɤr}}\end{entrée}

\begin{entrée}{rɟɯfsoʁ}{}{ⓔrɟɯfsoʁ} 
\classe{n} 
\begin{définition}\pfra{fait de gagner de l'argent}\end{définition}
\begin{définition}\pcmn{挣钱}\end{définition}\relationsémantique{参考}{\lien{ⓔtɯ-rɟɯ}{tɯ-rɟɯ}}\relationsémantique{参考}{\lien{ⓔfsoʁⓗ1}{fsoʁ₁}}\relationsémantique{参考}{\lien{ⓔɣɯrɟɯfsoʁ}{ɣɯrɟɯfsoʁ}}\end{entrée}

\begin{entrée}{rɟɯɣ}{₂}{ⓔrɟɯɣⓗ2} 
\classe{n} 
\begin{définition}\pfra{poutre horizontale}\end{définition}
\begin{définition}\pcmn{横梁}\end{définition}
\begin{exemple}\pjya{tɤ-jtsi ɯ-χto ɯ-taʁ kɯ-rɤsta ɣɯ tɤpjaʁ nɯ rɟɯɣ rmi}\hspace{5pt}\pcmn{固定在柱子的杈子上面的横梁叫\lien{ⓔrɟɯɣⓗ2}{rɟɯɣ}}\end{exemple}\end{entrée}

\begin{entrée}{rɟɯɣ}{₁}{ⓔrɟɯɣⓗ1} 
\classe{vi} \paradigme{dir}{\_}
\begin{définition}\pfra{courir}\end{définition}
\begin{définition}\pcmn{跑}\end{définition}
\begin{exemple}\pjya{kɤ-rɟɯɣ-a, nɯ-rɟɯɣ-a}\hspace{5pt}\pcmn{我跑了}\end{exemple}\relationsémantique{参考}{\lien{ⓔnɤrɟɯrɟɯɣ}{nɤrɟɯrɟɯɣ}}
\begin{sous-entrée}{sɯrɟɯɣ}{ⓔrɟɯɣⓗ1ⓝsɯrɟɯɣ} 
\classe{vt} \sens{1}
\begin{définition}\pfra{faire courir}\end{définition}
\begin{définition}\pcmn{使……跑}\end{définition}\end{sous-entrée}

\sens{2}
\begin{définition}\pfra{courir avec, apporter en courant}\end{définition}
\begin{définition}\pcmn{带着……跑;跑步拿来}\end{définition}
\begin{exemple}\pjya{andi ɲɯ-mbɣom tɕe, ki sɤcɯ ki ɲɯ-sɯrɟɯɣ-a ɲɯ-ntshi}\hspace{5pt}\pcmn{他们在那边有急事,我只好跑着把钥匙送过去}\end{exemple}\sens{3}
\begin{définition}\pfra{courir au moyen de}\end{définition}
\begin{définition}\pcmn{用……跑}\end{définition}
\begin{exemple}\pjya{rɯdaʁ kɯ ɯ-mi kɯβde-ldʑa ju-sɯrɟɯɣ ɲɯ-ɕti}\hspace{5pt}\pcmn{动物用四只脚跑}\end{exemple}\étymologie{rgʲug}\end{entrée}

\begin{entrée}{rɟɯma}{}{ⓔrɟɯma} 
\classe{n} 
\begin{définition}\pfra{vis}\end{définition}
\begin{définition}\pcmn{螺丝}\end{définition}
\begin{exemple}\pjya{rɟɯma tɤ-sprat-a}\hspace{5pt}\pcmn{我拧了螺丝}\end{exemple}\end{entrée}

\begin{entrée}{rɟɯmtsɯ}{}{ⓔrɟɯmtsɯ} 
\classe{n} 
\begin{définition}\pfra{pincette en bambou}\end{définition}
\begin{définition}\pcmn{竹子制成的夹子}\end{définition}\relationsémantique{同义词}{\lien{ⓔɟɯmɢom}{ɟɯmɢom}}\end{entrée}

\begin{entrée}{rɟɯnaŋlaŋspjɤt}{}{ⓔrɟɯnaŋlaŋspjɤt} 
\classe{n} 
\begin{définition}\pfra{richesses}\end{définition}
\begin{définition}\pcmn{财富}\end{définition}
\begin{exemple}\pjya{rɟɯnaŋlaŋspjɤt pjɯ-kɤ-khɯ}\hspace{5pt}\pcmn{保佑我们能顺利地创造财富}\end{exemple}\étymologie{rgʲu.naŋ.loŋ.spʲod}\end{entrée}

\begin{entrée}{rɟɯrŋom}{}{ⓔrɟɯrŋom} 
\classe{n} 
\begin{définition}\pfra{convoitise des richesses}\end{définition}
\begin{définition}\pcmn{贪财}\end{définition}
\begin{exemple}\pjya{rɟɯrŋom ma-tɯ-βze}\hspace{5pt}\pcmn{不要贪财}\end{exemple}\relationsémantique{参考}{\lien{ⓔtɯ-rɟɯ}{tɯ-rɟɯ}}\relationsémantique{参考}{\lien{ⓔsŋom}{sŋom}}\relationsémantique{参考}{\lien{ⓔnɯrɟɯrŋom}{nɯrɟɯrŋom}}\end{entrée}

\begin{entrée}{rɟɯstɤβ}{}{ⓔrɟɯstɤβ} 
\classe{n} 
\begin{définition}\pfra{capacité à gagner de l'argent}\end{définition}
\begin{définition}\pcmn{赚钱的本事,财力}\end{définition}\étymologie{rgʲu.stobs}\end{entrée}

\begin{entrée}{rɟɯtɕɯn}{}{ⓔrɟɯtɕɯn} 
\classe{n} 
\begin{définition}\pfra{résor}\end{définition}
\begin{définition}\pcmn{宝物,贵重物品}\end{définition}\étymologie{rgʲu.tɕʰen}\end{entrée}

\begin{entrée}{rku}{}{ⓔrku} 
\classe{vt}
\classe{vt} \paradigme{dir}{tɤ-}\paradigme{dir}{\_}\sens{1}
\begin{définition}\pfra{mettre dans}\end{définition}
\begin{définition}\pcmn{装进}\end{définition}
\begin{exemple}\pjya{laχtɕha khɯɣɲɟɯ ɯ-ŋgɯ nɯ-rku-t-a}\hspace{5pt}\pcmn{我把东西装进窗子里了}\end{exemple}
\begin{exemple}\pjya{ɯ-ŋgɯ ɲɤ-rku}\hspace{5pt}\pcmn{他放在里面了}\end{exemple}
\begin{exemple}\pjya{kɤ-rku xtɕhɯt}\hspace{5pt}\pcmn{装得下}\end{exemple}
\begin{exemple}\pjya{mɤʑɯ tú-wɣ-rku tɕhɯt}\hspace{5pt}\pcmn{还装得下}\end{exemple}
\begin{exemple}\pjya{tɤɕi lʁa ɯ-ŋgɯ tɤ-rku-t-a}\hspace{5pt}\pcmn{我把青稞装到口袋里}\end{exemple}
\begin{exemple}\pjya{tɤ-fkɯm ɯ-ŋgɯ thɯ-rku-t-a}\hspace{5pt}\pcmn{我装进口袋里了}\end{exemple}
\begin{exemple}\pjya{kɯɕnom ɯ-rdoʁ chɤ-rku}\hspace{5pt}\pcmn{青稞穗的颗粒结满了}\end{exemple}\sens{2}\paradigme{dir}{pɯ-}
\begin{définition}\pfra{verser}\end{définition}
\begin{définition}\pcmn{倒进}\end{définition}
\begin{exemple}\pjya{tʂha pɯ-rku-t-a}\hspace{5pt}\pcmn{我倒了茶}\end{exemple}\sens{3}\paradigme{dir}{kɤ-}\paradigme{dir}{pɯ-}
\begin{définition}\pfra{traiter une luxation}\end{définition}
\begin{définition}\pcmn{治(扭伤了的关节)}\end{définition}
\begin{définition}\pfra{oppresser}\end{définition}
\begin{définition}\pcmn{征服;压在自己下面}\end{définition}
\begin{exemple}\pjya{ɯ-jaʁ ɲɤ-nɯ-ɬoʁ tɕe, kɤ-rku-t-a}\hspace{5pt}\pcmn{他的手扭伤了,我给他治了}\end{exemple}
\begin{exemple}\pjya{a-pa pjɯ-nɯrke-a ra}\hspace{5pt}\pcmn{我一定要征服他们}\end{exemple}\relationsémantique{Component 2}{\lien{ⓔrku}{rku}}
\begin{sous-entrée}{ɯ-pa,nɯrku}{ⓔrkuⓢ3ⓝɯ-pa,nɯrku}\end{sous-entrée}

\begin{sous-entrée}{tɯ-ku,rku}{ⓔrkuⓢ3ⓝtɯ-ku,rku} 
\classe{np} 
\begin{définition}\pfra{s'occuper de, participer à}\end{définition}
\begin{définition}\pcmn{多管闲事;参与}\end{définition}
\begin{exemple}\pjya{nɤ-ku sɤ-rku kɯ-me, nɯra a-mɤ-thɯ-tɯ-nɯkon}\hspace{5pt}\pcmn{没有你的事,你不要管}\end{exemple}
\begin{exemple}\pjya{nɤ-ku ma-kɤ-tɯ-nɯ-rke}\hspace{5pt}\pcmn{你不要管闲事}\end{exemple}
\begin{exemple}\pjya{a-ku mɯ-kɤ-nɯrku-t-a}\hspace{5pt}\pcmn{我没有参与}\end{exemple}\relationsémantique{Component 1}{\lien{ⓔtɯ-ku}{tɯ-ku}}\end{sous-entrée}

\begin{sous-entrée}{nɯrku}{ⓔrkuⓝnɯrku} 
\grammaire{autoben} 
\begin{définition}\pfra{porter}\end{définition}
\begin{définition}\pcmn{戴}\end{définition}
\begin{exemple}\pjya{tɯtshot lu-nɯrke-a}\hspace{5pt}\pcmn{我戴手表}\end{exemple}\end{sous-entrée}

\begin{sous-entrée}{sɤrkɯrku}{ⓔrkuⓝsɤrkɯrku} 
\classe{vt} \paradigme{dir}{pɯ-}
\begin{définition}\pfra{ranger}\end{définition}
\begin{définition}\pcmn{收拾好;放好}\end{définition}
\begin{exemple}\pjya{laχtɕha pɯ-sɤrkɯrku-t-a}\hspace{5pt}\pcmn{我把东西收拾好了}\end{exemple}\end{sous-entrée}

\begin{sous-entrée}{ʑɣɤsɯrku}{ⓔrkuⓝʑɣɤsɯrku} 
\classe{vi}  
\grammaire{refl}
\grammaire{caus} 
\begin{définition}\pfra{se mettre dans}\end{définition}
\begin{définition}\pcmn{让自己加入}\end{définition}\end{sous-entrée}

\begin{sous-entrée}{ʑɣɤrku}{ⓔrkuⓝʑɣɤrku} 
\classe{vi}  
\grammaire{refl}
\grammaire{caus} 
\begin{définition}\pfra{se mettre dans}\end{définition}
\begin{définition}\pcmn{把自己装进}\end{définition}
\begin{exemple}\pjya{aʑo lʁa ɯ-ŋgɯ thɯ-ʑɣɤrku-a}\hspace{5pt}\pcmn{我把自己装进口袋里了}\end{exemple}\relationsémantique{同义词}{\lien{ⓔsɤriⓝʑɣɤsɤri}{ʑɣɤsɤri}}\end{sous-entrée}

\begin{sous-entrée}{nɤrkɯrku}{ⓔrkuⓝnɤrkɯrku} 
\classe{vt} 
\begin{définition}\pfra{mettre n'importe où}\end{définition}
\begin{définition}\pcmn{装来装去,到处塞}\end{définition}\relationsémantique{参考}{\lien{ⓔarku}{arku}}\end{sous-entrée}

\end{entrée}

\begin{entrée}{rkaŋ}{}{ⓔrkaŋ} 
\classe{vs} 
\begin{définition}\pfra{vigoureux}\end{définition}
\begin{définition}\pcmn{硬朗}\end{définition}
\begin{exemple}\pjya{a-mu kɯɕnɯsqaprɤɣ ʑo thɯ-azɣɯt ri, wuma ʑo rkaŋ}\hspace{5pt}\pcmn{虽然我母亲已经76岁,但是她很能干}\end{exemple}
\begin{exemple}\pjya{ki tɕheme ki mɯ-ɲɤ-rkaŋ}\hspace{5pt}\pcmn{这个女子怀孕了}\end{exemple}\end{entrée}

\begin{entrée}{rkɤdɯt}{}{ⓔrkɤdɯt} 
\classe{n} 
\begin{définition}\pfra{clarinette}\end{définition}
\begin{définition}\pcmn{唢呐}\end{définition}\étymologie{rkaŋ.??}\end{entrée}

\begin{entrée}{rkɤl}{}{ⓔrkɤl} 
\classe{vs} \paradigme{dir}{tɤ-}
\begin{définition}\pfra{en sécurité (endroit)}\end{définition}
\begin{définition}\pcmn{安全;防危险(地方)、不容易受到攻击}\end{définition}
\begin{exemple}\pjya{kɯki sɤtɕha ki, ɯ-rkɯ thamtɕɤt praʁ kɯ ku-fskɤr ɲɯ-ɕti tɕe, ɲɯ-rkɤl}\hspace{5pt}\pcmn{这个地方周围都是悬崖,很安全}\end{exemple}
\begin{exemple}\pjya{kha pjɯ́-wɣ-sɤtsa tɕe kɯ-rkɤl kɤ-nɯmga ŋu}\hspace{5pt}\pcmn{锁了房子的门是为了安全}\end{exemple}\end{entrée}

\begin{entrée}{rkɤsnom}{}{ⓔrkɤsnom} 
\classe{n} 
\begin{définition}\pfra{pantalon}\end{définition}
\begin{définition}\pcmn{裤子}\end{définition}\étymologie{rkaŋ.snam}\end{entrée}

\begin{entrée}{rkɤtu}{}{ⓔrkɤtu} 
\classe{n} 
\begin{définition}\pfra{marteau de bois}\end{définition}
\begin{définition}\pcmn{木槌(用来敲打麻织品)}\end{définition}\end{entrée}

\begin{entrée}{rkɤz}{}{ⓔrkɤz} 
\classe{vt} \paradigme{dir}{pɯ-}
\begin{définition}\pfra{graver, sculpter}\end{définition}
\begin{définition}\pcmn{雕刻}\end{définition}
\begin{exemple}\pjya{parɕaŋ pɯ-rkaz-a}\hspace{5pt}\pcmn{我刻了印版}\end{exemple}
\begin{exemple}\pjya{parɕaŋ pa-rkɤz}\hspace{5pt}\pcmn{他刻了印版}\end{exemple}
\begin{sous-entrée}{rɤrkɤz}{ⓔrkɤzⓝrɤrkɤz} 
\grammaire{apass} 
\begin{définition}\pfra{graver un xylographe}\end{définition}
\begin{définition}\pcmn{刻印版}\end{définition}
\begin{exemple}\pjya{kɯ-rɤrkɤz}\hspace{5pt}\pcmn{刻印版的人}\end{exemple}\end{sous-entrée}

\étymologie{rkos}\end{entrée}

\begin{entrée}{rkhɤrkhɤt}{}{ⓔrkhɤrkhɤt} 
\classe{idph.2} 
\begin{définition}\pfra{bruit de frappement léger}\end{définition}
\begin{définition}\pcmn{形容轻轻的敲击声}\end{définition}
\begin{exemple}\pjya{tɕheme nɯ ɯ-xtsa ɯ-qa kɯ-ɤmtɕoʁ kɯ-mbro to-ŋga tɕe tu-ŋke tɕe rkhɤnɤrkhɤt ɲɯ-ti}\hspace{5pt}\pcmn{那位女孩子穿着高跟鞋,走路的时候就有咚咚咚的声音}\end{exemple}\end{entrée}

\begin{entrée}{rkhe}{}{ⓔrkhe} 
\classe{vt} \paradigme{dir}{pɯ-}
\begin{définition}\pfra{graver}\end{définition}
\begin{définition}\pcmn{(一节一节地)刻}\end{définition}
\begin{exemple}\pjya{tɕoχtsi pɯ-rkhe-t-a}\hspace{5pt}\pcmn{我刻了桌子}\end{exemple}
\begin{exemple}\pjya{ɲɯ-rʑi tɕe a-jaʁ pjɤ-rkhe}\hspace{5pt}\pcmn{东西很重,我手上留了个印}\end{exemple}
\begin{exemple}\pjya{pjɤ-tɯ-rkhe-t}\hspace{5pt}\pcmn{你刻了}\end{exemple}\end{entrée}

\begin{entrée}{rkhoŋnɤrkhoŋ}{}{ⓔrkhoŋnɤrkhoŋ} 
\classe{idph.3} 
\begin{définition}\pfra{bruit d'un pierre qui se cogne contre du bois}\end{définition}
\begin{définition}\pcmn{形容石头撞击木头的声音}\end{définition}
\begin{exemple}\pjya{tɤrɤm ɯ-taʁ rdɤstaʁ tú-wɣ-lɤt tɕe, rkhoŋnɤrkhoŋ tu-ti ŋu}\hspace{5pt}\pcmn{在木板上扔石头的时候就会发出砰砰声}\end{exemple}\relationsémantique{同义词}{\lien{ⓔphoŋnɤphoŋ}{phoŋnɤphoŋ}}\end{entrée}

\begin{entrée}{rkhɯβrkhɯβ/\variante{rkhɯrkhɯp}}{}{ⓔrkhɯβrkhɯβ} 
\classe{idph.2} 
\begin{définition}\pfra{bruit de coup sur une planche de bois}\end{définition}
\begin{définition}\pcmn{敲木板的声音}\end{définition}\relationsémantique{同义词}{\lien{ⓔrchɤrchɤt}{rchɤrchɤt}}\relationsémantique{参考}{\lien{ⓔnɤrkhɯrkhɯβ}{nɤrkhɯrkhɯβ}}\end{entrée}

\begin{entrée}{rko}{}{ⓔrko} 
\classe{vs} \paradigme{dir}{thɯ-}\paradigme{dir}{nɯ-}\paradigme{dir}{tɤ-}\sens{1}
\begin{définition}\pfra{dur}\end{définition}
\begin{définition}\pcmn{硬}\end{définition}
\begin{exemple}\pjya{nɤrŋi ɯ-kɤcɯɣ chɤ-rko}\hspace{5pt}\pcmn{婴儿的囟门闭合了}\end{exemple}\sens{2}
\begin{définition}\pfra{obstiné}\end{définition}
\begin{définition}\pcmn{顽固}\end{définition}
\begin{exemple}\pjya{ɲɯ-rko}\hspace{5pt}\pcmn{很硬/他很顽固}\end{exemple}\relationsémantique{参考}{\lien{ⓔnɯrkorlɯt}{nɯrkorlɯt}}\end{entrée}

\begin{entrée}{rkoŋɟɤl}{}{ⓔrkoŋɟɤl} 
\classe{n} 
\begin{définition}\pfra{démon à un pied}\end{définition}
\begin{définition}\pcmn{独脚鬼}\end{définition}\étymologie{rkaŋ.rgʲal}\end{entrée}

\begin{entrée}{rkoŋtoŋ}{}{ⓔrkoŋtoŋ} 
\classe{n} \sens{1}
\begin{définition}\pfra{fémur}\end{définition}
\begin{définition}\pcmn{胫骨}\end{définition}\sens{2}
\begin{définition}\pfra{trompette en fémur humain}\end{définition}
\begin{définition}\pcmn{胫骨号筒}\end{définition}\étymologie{rkaŋ.duŋ}\end{entrée}

\begin{entrée}{rkorsa}{}{ⓔrkorsa} 
\classe{n} 
\begin{définition}\pfra{toilette}\end{définition}
\begin{définition}\pcmn{厕所}\end{définition}\end{entrée}

\begin{entrée}{rkɯn}{}{ⓔrkɯn} 
\classe{vs} \paradigme{dir}{nɯ-}
\begin{définition}\pfra{peu}\end{définition}
\begin{définition}\pcmn{少}\end{définition}
\begin{exemple}\pjya{nɤ-mɤ-kɤ-tso to-rkɯn}\hspace{5pt}\pcmn{你不懂的东西变少了}\end{exemple}
\begin{sous-entrée}{nɤrkɯn}{ⓔrkɯnⓝnɤrkɯn} 
\classe{vt} 
\begin{définition}\pfra{trouver trop peu, manquer de}\end{définition}
\begin{définition}\pcmn{觉得少,缺}\end{définition}
\begin{exemple}\pjya{a-kɤ-ndza kɤ-tshi ra pɯ-nɤrkɯn-a pɯ-ra}\hspace{5pt}\pcmn{我以前吃喝都觉得欠缺}\end{exemple}\relationsémantique{参考}{\lien{ⓔɣɤrkɯn}{ɣɤrkɯn}}\end{sous-entrée}

\étymologie{dkon}\end{entrée}

\begin{entrée}{rkɯwɯ}{}{ⓔrkɯwɯ} 
\classe{n} 
\begin{définition}\pfra{lampe à beurre}\end{définition}
\begin{définition}\pcmn{酥油灯}\end{définition}\end{entrée}

\begin{entrée}{rla}{}{ⓔrla} 
\classe{vt} \paradigme{dir}{nɯ-}
\begin{définition}\pfra{détacher, défaire (nœud)}\end{définition}
\begin{définition}\pcmn{解开}\end{définition}
\begin{exemple}\pjya{tɤ-mtɯ nɯ-rla-t-a}\hspace{5pt}\pcmn{我解开了结}\end{exemple}
\begin{exemple}\pjya{tɤ-mtɯ na-rla}\hspace{5pt}\pcmn{他解开了结}\end{exemple}
\begin{exemple}\pjya{nɤ-xtsa nɯ-rle}\hspace{5pt}\pcmn{你解开鞋子吧}\end{exemple}
\begin{sous-entrée}{arla}{ⓔrlaⓝarla} 
\classe{vi}  
\grammaire{pass} 
\begin{exemple}\pjya{tɤ-mtsɯ arla}\hspace{5pt}\pcmn{结是解开的}\end{exemple}\end{sous-entrée}

\end{entrée}

\begin{entrée}{rlaŋrlaŋ}{}{ⓔrlaŋrlaŋ} 
\classe{idph.2} 
\begin{définition}\pfra{rond}\end{définition}
\begin{définition}\pcmn{圆形(又圆又大)}\end{définition}
\begin{exemple}\pjya{tɤphɯ ɯ-tɯ-wxti kɯ rlaŋrlaŋ ʑo ɲɯ-pa}\hspace{5pt}\pcmn{土巴又圆又大}\end{exemple}
\begin{exemple}\pjya{@yangyu kɯwxtɯwxti rlaŋrlaŋ ʑo thɯ-nɯɬoʁ}\hspace{5pt}\pcmn{(人家挖出来的时候)洋芋又大又圆}\end{exemple}
\begin{exemple}\pjya{sla to-kɯ-ɤrtɯm ci rlaŋrlaŋ}\hspace{5pt}\pcmn{圆圆的月亮}\end{exemple}
\begin{sous-entrée}{rlaŋnɤrlaŋ}{ⓔrlaŋrlaŋⓝrlaŋnɤrlaŋ} 
\classe{idph.3} 
\begin{exemple}\pjya{rŋgɯ rlaŋnɤlaŋ pɯ-ndʐaβ}\hspace{5pt}\pcmn{圆形的大石包滚下去了}\end{exemple}\relationsémantique{参考}{\lien{ⓔslaŋslaŋ}{slaŋslaŋ}}\relationsémantique{参考}{\lien{ⓔɕlaŋɕlaŋ}{ɕlaŋɕlaŋ}}\relationsémantique{参考}{\lien{ⓔclaŋclaŋ}{claŋclaŋ}}\relationsémantique{参考}{\lien{ⓔrloŋrloŋ}{rloŋrloŋ}}\relationsémantique{参考}{\lien{ⓔrloʁrloʁ}{rloʁrloʁ}}\end{sous-entrée}

\end{entrée}

\begin{entrée}{rlaʁ}{}{ⓔrlaʁ} 
\classe{vi} \paradigme{dir}{nɯ-}
\begin{définition}\pfra{disparaître}\end{définition}
\begin{définition}\pcmn{失踪}\end{définition}
\begin{exemple}\pjya{a-taqaβ ɲɤ-rlaʁ}\hspace{5pt}\pcmn{我的针不见了}\end{exemple}
\begin{exemple}\pjya{a-mbrɯtɕɯ ɲɤ-rlaʁ}\hspace{5pt}\pcmn{我的刀不见了}\end{exemple}
\begin{exemple}\pjya{a-laχtɕha ɲɤ-rlaʁ}\hspace{5pt}\pcmn{我的东西不见了}\end{exemple}
\begin{exemple}\pjya{kɯ-rlaʁ kha}\hspace{5pt}\pcmn{被遗弃的房子}\end{exemple}\relationsémantique{参考}{\lien{ⓔɣɤrlaʁ}{ɣɤrlaʁ}}\étymologie{brlag}\end{entrée}

\begin{entrée}{rlaʁrlaʁ}{}{ⓔrlaʁrlaʁ} 
\classe{idph.2} 
\begin{définition}\pfra{rond et dur}\end{définition}
\begin{définition}\pcmn{形容圆又硬的样子}\end{définition}
\begin{exemple}\pjya{tɤ-fkɯm ɯ-ŋgɯ tɯsqar chɤ-sɯmtshɤt rlaʁrlaʁ ʑo ɲɯ-pa}\hspace{5pt}\pcmn{袋子里的糌粑装得很满,涨鼓鼓的}\end{exemple}\end{entrée}

\begin{entrée}{rloŋrloŋ}{}{ⓔrloŋrloŋ} 
\classe{idph.2} 
\begin{définition}\pfra{sphérique}\end{définition}
\begin{définition}\pcmn{形容球形(比较大)的样子}\end{définition}
\begin{exemple}\pjya{jla ŋgorli rloŋrloŋ ɲɯ-pa}\hspace{5pt}\pcmn{无角犏牛显得圆圆的}\end{exemple}\relationsémantique{参考}{\lien{ⓔrloʁrloʁ}{rloʁrloʁ}}\relationsémantique{参考}{\lien{ⓔrlaŋrlaŋ}{rlaŋrlaŋ}}\relationsémantique{参考}{\lien{ⓔrwoʁrwoʁ}{rwoʁrwoʁ}}\end{entrée}

\begin{entrée}{rloŋrta}{}{ⓔrloŋrta} 
\classe{n} \sens{1}
\begin{définition}\pfra{rlung rta}\end{définition}
\begin{définition}\pcmn{经幡}\end{définition}\sens{2}
\begin{définition}\pfra{chance}\end{définition}
\begin{définition}\pcmn{运气}\end{définition}
\begin{exemple}\pjya{ɯ-rloŋrta ɲɯ-taʁ}\hspace{5pt}\pcmn{他运气好}\end{exemple}\étymologie{rluŋ.rta}\end{entrée}

\begin{entrée}{rloʁrloʁ}{}{ⓔrloʁrloʁ} 
\classe{idph.2} 
\begin{définition}\pfra{sphérique}\end{définition}
\begin{définition}\pcmn{球形(比较小)}\end{définition}
\begin{exemple}\pjya{ɯ-ku rloʁrloʁ ʑo ɲɯ-pa}\hspace{5pt}\pcmn{他的头是圆形的}\end{exemple}
\begin{sous-entrée}{rloʁnɤrloʁ}{ⓔrloʁrloʁⓝrloʁnɤrloʁ} 
\classe{idph.3} 
\begin{définition}\pfra{qui a une tête ronde}\end{définition}
\begin{définition}\pcmn{头部圆圆的,动作又灵活又可爱}\end{définition}
\begin{exemple}\pjya{tɤ-pɤtso rloʁrloʁ nɤ rloʁrloʁ ɲɯ-ɤnɯɣro}\hspace{5pt}\pcmn{小孩子在玩}\end{exemple}
\begin{exemple}\pjya{tɤ-pɤtso chɤ-wxti tɕe rloʁnɤrloʁ ʑo tu-ŋke to-cha}\hspace{5pt}\pcmn{小孩子大了就能走路了}\end{exemple}\relationsémantique{参考}{\lien{ⓔrloŋrloŋ}{rloŋrloŋ}}\relationsémantique{参考}{\lien{ⓔrwoʁrwoʁ}{rwoʁrwoʁ}}\end{sous-entrée}

\end{entrée}

\begin{entrée}{rlɯm}{}{ⓔrlɯm} 
\classe{idph.1} 
\begin{définition}\pfra{complètement}\end{définition}
\begin{définition}\pcmn{全部}\end{définition}
\begin{exemple}\pjya{nɤ-tʂha rlɯm kɤ-tshi}\hspace{5pt}\pcmn{你把你的茶全部喝完}\end{exemple}
\begin{exemple}\pjya{jiʑo rlɯm kɯ kɤ-tshi-j ɕti}\hspace{5pt}\pcmn{我们都喝了}\end{exemple}\end{entrée}

\begin{entrée}{rma}{}{ⓔrma} 
\classe{vi} \paradigme{dir}{kɤ-}
\begin{définition}\pfra{habiter chez quelqu'un}\end{définition}
\begin{définition}\pcmn{留宿}\end{définition}
\begin{exemple}\pjya{jɯɣmɯr kutɕu kɤ-nɯ-rma}\hspace{5pt}\pcmn{你今天晚上在这里留宿吧}\end{exemple}
\begin{exemple}\pjya{jɯɣmɯr mbarkhom kɤ-rma}\hspace{5pt}\pcmn{你今天晚上在马尔康留宿吧}\end{exemple}
\begin{exemple}\pjya{ku-nɯ-rma ɲɯ-sɯsɤm}\hspace{5pt}\pcmn{他想在这里留宿}\end{exemple}\relationsémantique{同义词}{\lien{ⓔnɯkho}{nɯkho}}\relationsémantique{参考}{\lien{ⓔsɯrma}{sɯrma}}\end{entrée}

\begin{entrée}{rmɤβja}{}{ⓔrmɤβja} 
\classe{n} 
\begin{définition}\pfra{paon}\end{définition}
\begin{définition}\pcmn{孔雀}\end{définition}\étymologie{rma.bʲa}\end{entrée}

\begin{entrée}{rmɤβrmɤβ}{}{ⓔrmɤβrmɤβ} 
\classe{idph.2} 
\begin{définition}\pfra{une couche fine}\end{définition}
\begin{définition}\pcmn{形容薄薄的一层,不完全透明的样子}\end{définition}
\begin{exemple}\pjya{jisŋi zdɯm ci rmɤβrmɤβ ɣɤʑu}\hspace{5pt}\pcmn{今天云很薄,不完全透明}\end{exemple}
\begin{exemple}\pjya{tɯ-ci ɲɤ-nɤrʑaʁ tɕe ɯ-taʁ ɯ-ɕom kɯ-fse ci rmɤβrmɤβ ko-ta, tɕe kɤ-tshi mɯ-ɲɤ-sna}\hspace{5pt}\pcmn{水放久了就会在表面上形成薄薄的一层,不能再喝了}\end{exemple}\relationsémantique{同义词}{\lien{ⓔʂmɤβʂmɤβ}{ʂmɤβʂmɤβ}}\end{entrée}

\begin{entrée}{rmɤmbe}{}{ⓔrmɤmbe} 
\classe{n} 
\begin{définition}\pfra{mue (mammifère)}\end{définition}
\begin{définition}\pcmn{脱毛;换毛}\end{définition}\relationsémantique{参考}{\lien{ⓔtɤ-rme}{tɤ-rme}}\relationsémantique{参考}{\lien{ⓔmbe}{mbe}}\relationsémantique{参考}{\lien{ⓔnɯrmɤmbe}{nɯrmɤmbe}}\end{entrée}

\begin{entrée}{rmbatɕɯβ}{}{ⓔrmbatɕɯβ} 
\classe{n} 
\begin{définition}\pfra{espèce de plante}\end{définition}
\begin{définition}\pcmn{【灰灰菜】}\end{définition}
\begin{exemple}\pjya{rmbatɕɯβ nɯ sɯjno ci ŋu, tɯ-ɟom jamar tu-mbro cha, ɲɯ-ɤɣɯrtɯ-rtaʁ cha, ɯ-rɣi wuma ʑo dɤn, ɯ-jwaʁ ɯ-qhu chu nɯ kɯ-pɣi tu, kɯ-ɤɣɯrnɯɕɯr tu, tɯ-ɣndʑɤr tɤ-kɤ-mar kɯ-fse tu, pha ɯ-phoŋbu nɯ kɯ-pɣi ŋu, kɤ-ndza sna, paʁ wuma ʑo rga}\hspace{5pt}\pcmn{灰灰菜是一种植物,可以长一米多高,可以长出很多枝桠,结的种子很多,叶子的背面有的是灰色的,也有的是淡红色的,好像上面涂了一层面粉,全身是灰色的,人可以吃,猪特别喜欢。}\end{exemple}\end{entrée}

\begin{entrée}{rmbɯ}{}{ⓔrmbɯ} 
\classe{vt} \paradigme{dir}{tɤ-}
\begin{définition}\pfra{amasser}\end{définition}
\begin{définition}\pcmn{堆起来(不整齐)}\end{définition}
\begin{exemple}\pjya{tɯjpu ta-rmbɯ}\hspace{5pt}\pcmn{他把粮食堆起来了}\end{exemple}
\begin{exemple}\pjya{rdɤstaʁ tɤ-rmbɯ-t-a}\hspace{5pt}\pcmn{我把石头堆起来了}\end{exemple}
\begin{exemple}\pjya{tɯ-ɣli tɤ-rmbɯ-t-a}\hspace{5pt}\pcmn{我把肥料堆起来了}\end{exemple}
\begin{exemple}\pjya{tɤjpa to-rmbɯ}\hspace{5pt}\pcmn{她把雪堆起来了}\end{exemple}\relationsémantique{参考}{\lien{ⓔamɯrmbɯ}{amɯrmbɯ}}\relationsémantique{参考}{\lien{ⓔtɯ-tɤrmbɯ}{tɯ-tɤrmbɯ}}\end{entrée}

\begin{entrée}{rmi}{}{ⓔrmi} 
\classe{vi} \paradigme{dir}{tɤ-}
\begin{définition}\pfra{s’appeler}\end{définition}
\begin{définition}\pcmn{名字叫}\end{définition}
\begin{exemple}\pjya{jiɕqha nɯ nɯ ɲɯ-rmi}\hspace{5pt}\pcmn{他叫这个}\end{exemple}
\begin{exemple}\pjya{nɤʑo tɕhi tɯ-rmi?}\hspace{5pt}\pcmn{你叫什么名字}\end{exemple}
\begin{exemple}\pjya{aʑo χpɤltɕin rmi-a, nɤʑo @xiangbolin ɲɯ-tɯ-rmi ɣe}\hspace{5pt}\pcmn{我叫柏尔青,你叫向柏霖}\end{exemple}\relationsémantique{参考}{\lien{ⓔtɤ-rmi}{tɤ-rmi}}\end{entrée}

\begin{entrée}{rmɯrmi}{}{ⓔrmɯrmi} 
\classe{adv} 
\begin{définition}\pfra{tous sans exception}\end{définition}
\begin{définition}\pcmn{每一个,一个都不漏;每一种}\end{définition}
\begin{exemple}\pjya{kɯmdza rmɯrmi nɯ jo-ɣi-nɯ}\hspace{5pt}\pcmn{每一个亲戚都到了}\end{exemple}\end{entrée}

\begin{entrée}{rnaʁ}{}{ⓔrnaʁ} 
\classe{vs} \paradigme{dir}{pɯ-}
\begin{définition}\pfra{profond}\end{définition}
\begin{définition}\pcmn{深}\end{définition}
\begin{exemple}\pjya{jɯ-xtu ɯ-tɯ-rnaʁ nɯ}\hspace{5pt}\pcmn{我们很饿(肚子很深)}\end{exemple}\end{entrée}

\begin{entrée}{rnɤβʑi}{}{ⓔrnɤβʑi} 
\classe{n} \sens{1}
\begin{définition}\pfra{chapeau à quatre bords}\end{définition}
\begin{définition}\pcmn{圆盔耳帽}\end{définition}\sens{2}
\begin{définition}\pfra{casserole en fer}\end{définition}
\begin{définition}\pcmn{生铁锅,有四个把子}\end{définition}\étymologie{rna.bʑi}\end{entrée}

\begin{entrée}{rnɤjɯ}{}{ⓔrnɤjɯ} 
\classe{n} 
\begin{définition}\pfra{boucle d'oreille}\end{définition}
\begin{définition}\pcmn{耳环}\end{définition}\étymologie{rna.ju}\end{entrée}

\begin{entrée}{rnɤlu}{}{ⓔrnɤlu} 
\classe{n} 
\begin{définition}\pfra{sans oreille}\end{définition}
\begin{définition}\pcmn{缺耳朵的}\end{définition}\end{entrée}

\begin{entrée}{rnɤrɯ}{}{ⓔrnɤrɯ} 
\classe{n} 
\begin{définition}\pfra{casserole en fer}\end{définition}
\begin{définition}\pcmn{生铁锅,有两个把子}\end{définition}\étymologie{rna.ru?}\end{entrée}

\begin{entrée}{rndu}{}{ⓔrndu} 
\classe{vt} \paradigme{dir}{pɯ-}\sens{1}
\begin{définition}\pfra{obtenir}\end{définition}
\begin{définition}\pcmn{拿到}\end{définition}\sens{2}
\begin{définition}\pfra{trouver}\end{définition}
\begin{définition}\pcmn{找到}\end{définition}
\begin{exemple}\pjya{tɤjmɤɣ pjɤ-rndu}\hspace{5pt}\pcmn{他找到蘑菇了}\end{exemple}
\begin{exemple}\pjya{laχtɕha kɤ-χtɯ pjɤ-rndu}\hspace{5pt}\pcmn{买到东西了}\end{exemple}
\begin{exemple}\pjya{pɯ-rndu-tɕi}\hspace{5pt}\pcmn{我们俩找到了}\end{exemple}
\begin{exemple}\pjya{kɤ-rndu sɤznɤ pɯ-rnde-t-a}\hspace{5pt}\pcmn{不但没有得到好处,反而遭殃了}\end{exemple}\end{entrée}

\begin{entrée}{rnde}{}{ⓔrnde} 
\classe{vt} \paradigme{dir}{pɯ-}
\begin{définition}\pfra{subir un désastre (ne s'emploie pas seul)}\end{définition}
\begin{définition}\pcmn{吃亏;遭殃}\end{définition}
\begin{exemple}\pjya{kɤ-rndu sɤznɤ pɯ-rnde-t-a}\hspace{5pt}\pcmn{我不但没有得到好处,反而遭殃了}\end{exemple}
\begin{exemple}\pjya{kɤ-rndu sɤznɤ kɤ-rnde}\hspace{5pt}\pcmn{不但没有得到好处,反而遭殃了}\end{exemple}\end{entrée}

\begin{entrée}{rndi}{}{ⓔrndi} 
\classe{vi} \paradigme{dir}{tɤ-}
\begin{définition}\pfra{sage, qui ne s'enfuit pas à chaque occasion}\end{définition}
\begin{définition}\pcmn{不爱到处乱跑,听话}\end{définition}
\begin{exemple}\pjya{ki fsapaʁ ki ɲɯ-rndi}\hspace{5pt}\pcmn{这个动物不爱到处乱跑}\end{exemple}
\begin{exemple}\pjya{ɕɯŋgɯ staʁ to-rndi}\hspace{5pt}\pcmn{以前很调皮,爱到处乱跑,现在不调皮了}\end{exemple}\end{entrée}

\begin{entrée}{rndzɤkɤŋe}{}{ⓔrndzɤkɤŋe} 
\classe{n} 
\begin{définition}\pfra{ombre de la montagne}\end{définition}
\begin{définition}\pcmn{太阳落山的时候山上的阴影}\end{définition}
\begin{exemple}\pjya{rndzɤkɤŋe tɤ-anɯri}\end{exemple}\end{entrée}

\begin{entrée}{rɲaŋ}{}{ⓔrɲaŋ} 
\classe{vs} \paradigme{dir}{nɯ-}
\begin{définition}\pfra{ancien}\end{définition}
\begin{définition}\pcmn{陈旧}\end{définition}
\begin{exemple}\pjya{tamar nɯ ʑaʑa ɲɯ-rɲaŋ ɕti}\hspace{5pt}\pcmn{酥油很快就会变味}\end{exemple}\étymologie{rɲiŋ}\end{entrée}

\begin{entrée}{rɲɟaʁlo}{}{ⓔrɲɟaʁlo} 
\classe{n} 
\begin{définition}\pfra{bâton qui sert à caler la porte}\end{définition}
\begin{définition}\pcmn{门闩}\end{définition}
\begin{exemple}\pjya{rɲɟaʁlo ɯ-thaʁ pjɯ́-wɣ-lɤt tɕe, kɤ-cɯ mɤ-khɯ}\hspace{5pt}\pcmn{插上插销,门就不能打开了}\end{exemple}
\begin{exemple}\pjya{rɲɟaʁlo nɯ-lat-a}\hspace{5pt}\pcmn{我闩了门。}\end{exemple}
\begin{exemple}\pjya{rɲɟaʁlo-ɣɲɟɯ}\hspace{5pt}\pcmn{门闩的洞}\end{exemple}\relationsémantique{参考}{\lien{ⓔrɟɤthaʁ}{rɟɤthaʁ}}\end{entrée}

\begin{entrée}{rɲɟi}{}{ⓔrɲɟi} 
\classe{vs} \paradigme{dir}{thɯ-}
\begin{définition}\pfra{long}\end{définition}
\begin{définition}\pcmn{长}\end{définition}
\begin{exemple}\pjya{tɯmbri ɲɯ-rɲɟi}\hspace{5pt}\pcmn{绳子很长}\end{exemple}
\begin{exemple}\pjya{qha kɯrɯ-ŋga nɯ ɲɯ-rɲɟi}\hspace{5pt}\pcmn{这件藏装很长}\end{exemple}\relationsémantique{参考}{\lien{ⓔɣɤrɲɟiⓝɣɤrɲɟi}{ɣɤrɲɟi}}\relationsémantique{反义词}{\lien{ⓔxtɯtⓗ2}{xtɯt}}\relationsémantique{同义词}{\lien{ⓔzri}{zri}}\relationsémantique{参考}{\lien{ⓔxtɯrɲɟi}{xtɯrɲɟi}}\end{entrée}

\begin{entrée}{rɲo}{}{ⓔrɲo} 
\classe{vt} \sens{1}\paradigme{dir}{tɤ-}
\begin{définition}\pfra{essayer, goûter}\end{définition}
\begin{définition}\pcmn{尝试}\end{définition}
\begin{exemple}\pjya{tɤjko kɤ-ndza tɤ-rɲo-t-a ri, pjɤ-tɕur}\hspace{5pt}\pcmn{我尝过酸菜,很酸}\end{exemple}
\begin{exemple}\pjya{ɯ-ɲɯ-mɯm kɯ tu-rɲam-a ɲɯ-ra}\hspace{5pt}\pcmn{我要尝一下好不好吃}\end{exemple}\sens{2}\paradigme{dir}{pɯ-}
\begin{définition}\pfra{faire l'expérience de, avoir déjà}\end{définition}
\begin{définition}\pcmn{体会;曾经……有过}\end{définition}
\begin{exemple}\pjya{pjɯ́-wɣ-rɲo mɤɕtʂa mɤ-kɯ-tso}\hspace{5pt}\pcmn{自己亲身体会之前不能了解}\end{exemple}
\begin{exemple}\pjya{tɕhi pɯ-nɯ-ŋɯ-ŋu pjɯ́-wɣ-rɲo ra}\hspace{5pt}\pcmn{无论什么事情都要亲身体会}\end{exemple}
\begin{exemple}\pjya{tɤjko kɤ-ndza pɯ-rɲo-t-a}\hspace{5pt}\pcmn{我曾经吃过酸菜}\end{exemple}
\begin{exemple}\pjya{kɤ-ɕe pɯ-rɲo-t-a}\hspace{5pt}\pcmn{我去过}\end{exemple}\end{entrée}

\begin{entrée}{rɲɯɣrɲɯɣ}{}{ⓔrɲɯɣrɲɯɣ} 
\classe{idph.2} 
\begin{définition}\pfra{long, fin et flexible}\end{définition}
\begin{définition}\pcmn{形容又细又长又柔软,很没有精神的样子}\end{définition}
\begin{exemple}\pjya{khɯɣɲɟɯ zɯ, laʁjɯɣ rɲɯɣrɲɯɣ ɲɤ-tɕɤt}\hspace{5pt}\pcmn{他把从窗户(往外面)探出来根木棒}\end{exemple}
\begin{exemple}\pjya{jiɕqha nɯ rɲɯɣrɲɯɣ ɲɤ-nɯ-ɬoʁ}\hspace{5pt}\pcmn{(细长的东西)出来了}\end{exemple}
\begin{sous-entrée}{sɤrɲɯɣrɲɯɣ}{ⓔrɲɯɣrɲɯɣⓝsɤrɲɯɣrɲɯɣ} 
\classe{vt} 
\begin{exemple}\pjya{qapri kɯ ɯ-mdʑu ɲɯ-sɤrɲɯɣrɲɯɣ}\hspace{5pt}\pcmn{蛇把舌头伸出来}\end{exemple}\end{sous-entrée}

\end{entrée}

\begin{entrée}{rɲɯl}{}{ⓔrɲɯl} 
\classe{vi} \paradigme{dir}{pɯ-}\sens{1}
\begin{définition}\pfra{fâner}\end{définition}
\begin{définition}\pcmn{凋谢}\end{définition}
\begin{exemple}\pjya{mɯntoʁ pjɤ-rɲɯl}\hspace{5pt}\pcmn{花凋谢了}\end{exemple}\sens{2}
\begin{définition}\pfra{se délabrer (maison)}\end{définition}
\begin{définition}\pcmn{塌下来了}\end{définition}
\begin{exemple}\pjya{kha pjɤ-rɲɯl (=pjɤ-mbɯt)}\hspace{5pt}\pcmn{房子塌下来了}\end{exemple}\relationsémantique{同义词}{\lien{ⓔmbɯt}{mbɯt}}\sens{3}
\begin{définition}\pfra{avoir complètement pourri}\end{définition}
\begin{définition}\pcmn{完全腐烂掉了}\end{définition}
\begin{exemple}\pjya{pjɤ-tsɣi tɕe pjɤ-rɲɯl}\hspace{5pt}\pcmn{腐烂了}\end{exemple}
\begin{exemple}\pjya{rɯdaʁ pjɤ-si tɕe pjɤ-rɲɯl}\hspace{5pt}\pcmn{动物死了然后就腐烂了}\end{exemple}\relationsémantique{同义词}{\lien{}{zɯɣ₂}}\relationsémantique{参考}{\lien{ⓔjaⓗ2}{ja₂}}\étymologie{rɲid}\end{entrée}

\begin{entrée}{rŋu}{}{ⓔrŋu} 
\classe{vl} \paradigme{dir}{tɤ-}\paradigme{dir}{thɯ-}
\begin{définition}\pfra{frire (le blé)}\end{définition}
\begin{définition}\pcmn{干炒(麦子)}\end{définition}
\begin{exemple}\pjya{tɤɕi tɤ-rŋu-t-a}\hspace{5pt}\pcmn{我炒了青稞}\end{exemple}
\begin{exemple}\pjya{tɤɕi chɯ́-wɣ-rŋu tɕe nɯ kóʁmɯz nɤ tɯsqar ɲɯ-βze ɕti}\hspace{5pt}\pcmn{炒了青稞就可以做糌粑}\end{exemple}\étymologie{rŋo}\end{entrée}

\begin{entrée}{rŋama}{}{ⓔrŋama} 
\classe{n} 
\begin{définition}\pfra{(porter à) complétion}\end{définition}
\begin{définition}\pcmn{(做)到底,(做)得彻底}\end{définition}
\begin{exemple}\pjya{ɯʑo kɯ kɤ-nɤma ra rŋama mɤ-kɯ-ɬoʁ ɲɯ-βde ɲɯ-ɕti}\hspace{5pt}\pcmn{我做事做到一半就放弃}\end{exemple}
\begin{sous-entrée}{rŋama,tɕɤt}{ⓔrŋamaⓝrŋama,tɕɤt}\paradigme{dir}{thɯ-}
\begin{définition}\pfra{faire complètement, porter à complétion}\end{définition}
\begin{définition}\pcmn{做到底}\end{définition}
\begin{exemple}\pjya{rŋama mɯ-chɤ-tɯ-tɕɤt}\hspace{5pt}\pcmn{你没有把事情做得彻底}\end{exemple}\end{sous-entrée}

\étymologie{rŋa.ma}\end{entrée}

\begin{entrée}{rŋamoŋ}{}{ⓔrŋamoŋ} 
\classe{n} 
\begin{définition}\pfra{chameau}\end{définition}
\begin{définition}\pcmn{骆驼}\end{définition}
\begin{exemple}\pjya{rŋamoŋ raŋzga}\hspace{5pt}\pcmn{骆驼(固有的)鞍子}\end{exemple}\étymologie{rŋa.moŋ}\end{entrée}

\begin{entrée}{rŋapa}{}{ⓔrŋapa} 
\classe{n} 
\begin{définition}\pfra{cinquième mois}\end{définition}
\begin{définition}\pcmn{五月}\end{définition}\étymologie{lŋa.pa}\end{entrée}

\begin{entrée}{rŋawa}{}{ⓔrŋawa} 
\classe{n}  
\grammaire{n.lieu} 
\begin{définition}\pfra{Rngaba}\end{définition}
\begin{définition}\pcmn{阿坝}\end{définition}\end{entrée}

\begin{entrée}{rŋɤβrŋɤβ}{}{ⓔrŋɤβrŋɤβ} 
\classe{idph.2} 
\begin{définition}\pfra{haut et fin}\end{définition}
\begin{définition}\pcmn{形容细而高的样子}\end{définition}
\begin{exemple}\pjya{kumpɣa kɯ ɯ-ku rŋɤβrŋɤβ ʑo to-joʁ}\hspace{5pt}\pcmn{鸡把头伸得很高(东看西看)}\end{exemple}
\begin{sous-entrée}{rŋɤβnɤrŋɤβ}{ⓔrŋɤβrŋɤβⓝrŋɤβnɤrŋɤβ} 
\classe{idph.3} \end{sous-entrée}

\end{entrée}

\begin{entrée}{rŋɤfsoʁ}{}{ⓔrŋɤfsoʁ} 
\classe{n} 
\begin{définition}\pfra{vache dont la tête est blanche}\end{définition}
\begin{définition}\pcmn{白头牛}\end{définition}\end{entrée}

\begin{entrée}{rŋɤɣndʑɯr}{}{ⓔrŋɤɣndʑɯr} 
\classe{vi} \paradigme{dir}{thɯ-}
\begin{définition}\pfra{faire frire de la tsampa et moudre des grains d'orge}\end{définition}
\begin{définition}\pcmn{又炒糌粑又磨面}\end{définition}\relationsémantique{参考}{\lien{ⓔrŋu}{rŋu}}\relationsémantique{参考}{\lien{ⓔɣndʑɯr}{ɣndʑɯr}}\relationsémantique{参考}{\lien{ⓔsɤrŋɤɣndʑɯr}{sɤrŋɤɣndʑɯr}}\end{entrée}

\begin{entrée}{rŋɤmboʁ}{}{ⓔrŋɤmboʁ} 
\classe{n} 
\begin{définition}\pfra{grains d'orge grillés}\end{définition}
\begin{définition}\pcmn{青稞爆花}\end{définition}\end{entrée}

\begin{entrée}{rŋɤrŋɤt}{}{ⓔrŋɤrŋɤt} 
\classe{idph.2} 
\begin{définition}\pfra{imposant}\end{définition}
\begin{définition}\pcmn{形容雄伟;又高又宽的样子}\end{définition}
\begin{exemple}\pjya{praʁ rŋɤrŋɤt ʑo ɲɯ-pa}\hspace{5pt}\pcmn{悬崖又高又宽}\end{exemple}
\begin{sous-entrée}{mɤlɤrŋɤt}{ⓔrŋɤrŋɤtⓝmɤlɤrŋɤt} 
\classe{idph.6} 
\begin{exemple}\pjya{@wenchuan zgo ra mɤlɤrŋɤt ɲɯ-ŋu}\hspace{5pt}\pcmn{汶川的山又高又宽}\end{exemple}\end{sous-entrée}

\end{entrée}

\begin{entrée}{rŋɤʁjoʁ}{}{ⓔrŋɤʁjoʁ} 
\classe{n} 
\begin{définition}\pfra{bâton courbé avec lequel on frappe le tambour}\end{définition}
\begin{définition}\pcmn{用来打鼓的棍子}\end{définition}\étymologie{rŋa.gjog}\end{entrée}

\begin{entrée}{ruŋgu/\variante{rɯŋgu}}{}{ⓔruŋgu} 
\classe{n} 
\begin{définition}\pfra{pâturage}\end{définition}
\begin{définition}\pcmn{牧场}\end{définition}\end{entrée}

\begin{entrée}{rŋgɤβ}{}{ⓔrŋgɤβ} 
\classe{vt} \paradigme{dir}{tɤ-}
\begin{définition}\pfra{attacher}\end{définition}
\begin{définition}\pcmn{捆绑}\end{définition}
\begin{exemple}\pjya{tɯrme ka-ndo-nɯ tɕe ɯ-jaʁ tu-rŋgɤβ-nɯ ŋu}\hspace{5pt}\pcmn{他们抓人的时候就把他的手捆起来}\end{exemple}
\begin{exemple}\pjya{aʑo kɯ ɯ-jaʁ tɤ-rŋgaβ-a}\hspace{5pt}\pcmn{我捆了他的手}\end{exemple}\relationsémantique{同义词}{\lien{ⓔzgroʁⓗ1}{zgroʁ₁}}\relationsémantique{参考}{\lien{ⓔtɯ-mthɤrɴɢɤβ}{tɯ-mthɤrɴɢɤβ}}\relationsémantique{参考}{\lien{}{ɯ-jaqhɤrŋɤβ}}\end{entrée}

\begin{entrée}{rŋgɤm}{}{ⓔrŋgɤm} 
\classe{n} 
\begin{définition}\pfra{morceau dur}\end{définition}
\begin{définition}\pcmn{硬块;固体}\end{définition}
\begin{exemple}\pjya{χɕɤlkara ɯ-rŋgɤm ɲɯ-ŋu ma ɯ-ɣndʑɤr ɲɯ-maʁ}\hspace{5pt}\pcmn{冰糖是由硬块组成的,不是粉状的。}\end{exemple}\end{entrée}

\begin{entrée}{rŋgɯ}{₂}{ⓔrŋgɯⓗ2} 
\classe{n} 
\begin{définition}\pfra{gros rocher}\end{définition}
\begin{définition}\pcmn{岩石}\end{définition}\end{entrée}

\begin{entrée}{rŋgɯ}{₁}{ⓔrŋgɯⓗ1} 
\classe{vi} \sens{1}\paradigme{dir}{kɤ-}\paradigme{dir}{lɤ-}
\begin{définition}\pfra{s'allonger}\end{définition}
\begin{définition}\pcmn{躺}\end{définition}
\begin{exemple}\pjya{nɯŋa ko-rŋgɯ}\hspace{5pt}\pcmn{牛躺下了}\end{exemple}
\begin{exemple}\pjya{jla ko-rŋgɯ}\hspace{5pt}\pcmn{犏牛躺下了}\end{exemple}
\begin{exemple}\pjya{tɯrme kɤ-nɯ-rŋgɯ}\hspace{5pt}\pcmn{人躺下了}\end{exemple}
\begin{exemple}\pjya{kɤ-nɯ-rŋgɯ-j}\hspace{5pt}\pcmn{我们躺下了}\end{exemple}
\begin{exemple}\pjya{ɯ-thoʁ lɤ-rŋgɯ}\hspace{5pt}\pcmn{他在地上躺下了}\end{exemple}\sens{2}\paradigme{dir}{kɤ-}
\begin{définition}\pfra{dormir}\end{définition}
\begin{définition}\pcmn{睡觉}\end{définition}\relationsémantique{参考}{\lien{ⓔnɯkhɤrŋgɯ}{nɯkhɤrŋgɯ}}\end{entrée}

\begin{entrée}{rŋi}{}{ⓔrŋi} 
\classe{vs} 
\begin{définition}\pfra{être encore rouges (braises)}\end{définition}
\begin{définition}\pcmn{(火种)还在燃烧,还没有熄灭,还有复燃的可能}\end{définition}
\begin{exemple}\pjya{smi ɲɯ-rŋi}\hspace{5pt}\pcmn{火种还没有熄灭}\end{exemple}\end{entrée}

\begin{entrée}{rŋo}{}{ⓔrŋo} 
\classe{vt} \paradigme{dir}{nɯ-}
\begin{définition}\pfra{emprunter (un objet)}\end{définition}
\begin{définition}\pcmn{向别人借(能归还原物)}\end{définition}
\begin{exemple}\pjya{nɤ-ɕki sɲɯɣjɯ nɯ-rŋo-t-a}\hspace{5pt}\pcmn{我向你借了一只笔}\end{exemple}\relationsémantique{参考}{\lien{ⓔɕɯrŋo}{ɕɯrŋo}}\end{entrée}

\begin{entrée}{rŋɯl}{}{ⓔrŋɯl} 
\classe{n} 
\begin{définition}\pfra{argent}\end{définition}
\begin{définition}\pcmn{银子}\end{définition}\relationsémantique{参考}{\lien{ⓔaɣɯrŋɯl}{aɣɯrŋɯl}}\étymologie{dŋul}\end{entrée}

\begin{entrée}{rŋɯlkhoz}{}{ⓔrŋɯlkhoz} 
\classe{n} 
\begin{définition}\pfra{sac pour mettre de l'argent}\end{définition}
\begin{définition}\pcmn{装银子的口袋}\end{définition}\end{entrée}

\begin{entrée}{rŋɯzrŋɯz/\variante{rŋɯzŋɯz}}{}{ⓔrŋɯzrŋɯz} 
\classe{idph.2} 
\begin{définition}\pfra{osseux, maigrichon}\end{définition}
\begin{définition}\pcmn{过瘦}\end{définition}
\begin{exemple}\pjya{fsapaʁ mɤ-kɯ-mthu ci rŋɯzrŋɯz ɲɯ-ŋu}\hspace{5pt}\pcmn{那个牲畜又弱又瘦}\end{exemple}
\begin{exemple}\pjya{tɯrme mɤ-kɯ-mthu ci rŋɯzrŋɯz ɲɯ-ŋu}\hspace{5pt}\pcmn{那个人又弱又瘦}\end{exemple}
\begin{sous-entrée}{rŋɯznɤrŋɯz}{ⓔrŋɯzrŋɯzⓝrŋɯznɤrŋɯz} 
\classe{idph.2} 
\begin{exemple}\pjya{rŋɯznɤrŋɯz ɲɯ-tɯ-nɤŋkɯŋke}\hspace{5pt}\pcmn{你那么瘦,在那里走来走去(骂人的话)}\end{exemple}\end{sous-entrée}

\end{entrée}

\begin{entrée}{ro}{}{ⓔro} 
\classe{vs} \sens{1}\paradigme{dir}{nɯ-}
\begin{définition}\pfra{en trop}\end{définition}
\begin{définition}\pcmn{多余的}\end{définition}
\begin{exemple}\pjya{ki tɯ-ŋga ki ɲɯ-ro, kɤ-ɕɣɤz ɲɯ-ra}\hspace{5pt}\pcmn{这件衣服是多余的,要还给人家(多拿了属于别人的衣服)}\end{exemple}
\begin{exemple}\pjya{ki ɯ-phɯ ki kɤ-kho ɲɤ-ro tɕe ɲɯ-ta-fsɯɣ}\hspace{5pt}\pcmn{你多给了钱,我找一下零钱给你}\end{exemple}\sens{2}\paradigme{dir}{tɤ-}
\begin{définition}\pfra{qui ressort}\end{définition}
\begin{définition}\pcmn{高出;凸出来}\end{définition}
\begin{exemple}\pjya{tu-ro ʑɣɤrʑɣɤr ʑo ɲɯ-ŋu}\hspace{5pt}\pcmn{有(一两根)凸出来}\end{exemple}\end{entrée}

\begin{entrée}{roko}{}{ⓔroko} 
\classe{n} 
\begin{définition}\pfra{type de métal, ressemble au laiton}\end{définition}
\begin{définition}\pcmn{金属的一种,类似于黄铜}\end{définition}
\begin{exemple}\pjya{roko tɕhoma}\hspace{5pt}\pcmn{铜的皮带扣子}\end{exemple}\end{entrée}

\begin{entrée}{rom}{}{ⓔrom} 
\classe{vs} \paradigme{dir}{tɤ-}\paradigme{dir}{nɯ-}\paradigme{dir}{tɤ-}
\begin{définition}\pfra{séché}\end{définition}
\begin{définition}\pcmn{晒干的}\end{définition}
\begin{définition}\pfra{sécher}\end{définition}
\begin{définition}\pcmn{晒干}\end{définition}
\begin{exemple}\pjya{si to-rom}\hspace{5pt}\pcmn{木料干了}\end{exemple}
\begin{exemple}\pjya{paʁndza tɤ-sɯɣrom-a}\hspace{5pt}\pcmn{我把猪食晒干了}\end{exemple}
\begin{exemple}\pjya{sɯjno tɤ-sɯɣrom-a}\hspace{5pt}\pcmn{我把草晒干了}\end{exemple}
\begin{exemple}\pjya{sɯjno ɲɤ-sɯɣrom}\hspace{5pt}\pcmn{他把草晒干了}\end{exemple}
\begin{exemple}\pjya{lɤpɯɣ ɲɤ-sɯɣrom}\hspace{5pt}\pcmn{他把萝卜晒干了}\end{exemple}\relationsémantique{参考}{\lien{ⓔɯ-ɣrom}{ɯ-ɣrom}}
\begin{sous-entrée}{sɯɣrom}{ⓔromⓝsɯɣrom} 
\classe{vt} \end{sous-entrée}

\end{entrée}

\begin{entrée}{romɲa}{}{ⓔromɲa} 
\classe{n} 
\begin{définition}\pfra{poutre}\end{définition}
\begin{définition}\pcmn{小梁}\end{définition}\end{entrée}

\begin{entrée}{roŋba}{}{ⓔroŋba} 
\classe{n} 
\begin{définition}\pfra{locuteurs du rgyalrong oriental}\end{définition}
\begin{définition}\pcmn{讲四土话的藏族}\end{définition}
\begin{exemple}\pjya{ɣnɤsqi-xpa ɕɯŋgɯ zɯ roŋba-skɤt pjɤ-βzjoz}\hspace{5pt}\pcmn{他20年前学了四土话}\end{exemple}\étymologie{roŋ.pa}\end{entrée}

\begin{entrée}{roŋwa}{}{ⓔroŋwa} 
\classe{n} 
\begin{définition}\pfra{agriculteurs}\end{définition}
\begin{définition}\pcmn{农民}\end{définition}\étymologie{roŋ.ba}\end{entrée}

\begin{entrée}{roŋzga}{}{ⓔroŋzga} 
\classe{n} 
\begin{définition}\pfra{bosse (chameau)}\end{définition}
\begin{définition}\pcmn{峰(骆驼)}\end{définition}
\begin{exemple}\pjya{rŋamoŋ roŋzga}\end{exemple}\end{entrée}

\begin{entrée}{rorʁe}{}{ⓔrorʁe} 
\classe{n} 
\begin{définition}\pcmn{走缘边小柱头之间的穿杆}\end{définition}
\begin{exemple}\pjya{jɤɣɤt laχtsɯ cho mɤro ɣɯ ɯ-kɯ-spoʁ ɯ-ŋgɯ jɯ-kɤ-rʁe laʁjɯɣ nɯ rorʁe rmi}\hspace{5pt}\pcmn{走缘柱头和梁架的洞里穿过去的木棒叫\lien{ⓔrorʁe}{rorʁe}}\end{exemple}\end{entrée}

\begin{entrée}{roʁ}{}{ⓔroʁ} 
\classe{vt} \paradigme{dir}{kɤ-}\paradigme{dir}{thɯ-}\sens{1}
\begin{définition}\pfra{graver}\end{définition}
\begin{définition}\pcmn{雕刻}\end{définition}
\begin{exemple}\pjya{ɕoŋβzu kɯ ji-lɤtɕhom tha-roʁ}\hspace{5pt}\pcmn{木匠刻了我们的大奶桶}\end{exemple}\sens{2}\paradigme{dir}{tɤ-}
\begin{définition}\pfra{acculer (chasseur)}\end{définition}
\begin{définition}\pcmn{追到最险要的地方不让动物逃走(猎人)}\end{définition}
\begin{définition}\pfra{graver}\end{définition}
\begin{définition}\pcmn{刻}\end{définition}
\begin{sous-entrée}{rɤroʁ}{ⓔroʁⓢ2ⓝrɤroʁ} 
\classe{vi}  
\grammaire{apass} \end{sous-entrée}

\end{entrée}

\begin{entrée}{roʁrɯz}{}{ⓔroʁrɯz} 
\classe{n} 
\begin{définition}\pfra{changement et balayage}\end{définition}
\begin{définition}\pcmn{收拾东西和扫地}\end{définition}
\begin{exemple}\pjya{kha tɕe sɤskɯsku ʑo roʁrɯz kɤ-βzu ra}\hspace{5pt}\pcmn{家里每天早上都要打扫和收拾东西}\end{exemple}\relationsémantique{参考}{\lien{ⓔraʁrɯzⓝrɤroʁrɯz}{rɤroʁrɯz}}\relationsémantique{参考}{\lien{ⓔraʁrɯz}{raʁrɯz}}\end{entrée}

\begin{entrée}{rpu}{}{ⓔrpu} 
\classe{vl}
\classe{vt}  
\grammaire{refl} \paradigme{dir}{kɤ-}\paradigme{dir}{kɤ-}\paradigme{dir}{kɤ-}
\begin{définition}\pfra{cogner}\end{définition}
\begin{définition}\pcmn{撞;碰撞}\end{définition}
\begin{définition}\pfra{se cogner}\end{définition}
\begin{définition}\pcmn{撞}\end{définition}
\begin{exemple}\pjya{ɯ-taʁ kɤ-rpu-a}\hspace{5pt}\pcmn{我撞了他}\end{exemple}
\begin{exemple}\pjya{nɤ-ku ko-tɯ-nɯrpu-t tɕe pjɤ-tɯ-phɤβ pjɤ-ra}\hspace{5pt}\pcmn{你撞了头,你本来应该低头}\end{exemple}
\begin{exemple}\pjya{alo jiʑo ji-kha nɤ-ku ko-tɯ-nɯrpu-t khi loβtɕi}\hspace{5pt}\pcmn{(听说)你在我们干木鸟的家里撞了头,对吧?}\end{exemple}
\begin{exemple}\pjya{a-mi kɤ-nɯrpu-t-a}\hspace{5pt}\pcmn{我撞到脚了}\end{exemple}
\begin{sous-entrée}{nɯrpu}{ⓔrpuⓝnɯrpu} 
\classe{vt}  
\grammaire{autoben} \end{sous-entrée}

\begin{sous-entrée}{ʑɣɤrpu}{ⓔrpuⓝʑɣɤrpu} 
\classe{vi} \end{sous-entrée}

\begin{définition}\pfra{se cogner soi-même}\end{définition}
\begin{définition}\pcmn{撞到自己}\end{définition}\relationsémantique{Component 2}{\lien{}{lɤt}}
\begin{sous-entrée}{ɯ-rpu,lɤt}{ⓔrpuⓝɯ-rpu,lɤt} 
\classe{np} 
\begin{définition}\pfra{frapper avec ...}\end{définition}
\begin{définition}\pcmn{用……打}\end{définition}\relationsémantique{Component 1}{\lien{}{ɯ-rpu}}\end{sous-entrée}

\begin{exemple}\pjya{a-taʁ taʁndzɤr-rpu ta-lɤt}\hspace{5pt}\pcmn{他用喂猪槽打了我}\end{exemple}
\begin{exemple}\pjya{a-taʁ khɯtsa-rpu ta-lɤt}\hspace{5pt}\pcmn{他用碗打了我}\end{exemple}\end{entrée}

\begin{entrée}{rpɤŋgɯ}{}{ⓔrpɤŋgɯ} 
\classe{n}  
\grammaire{n.lieu} 
\begin{définition}\pfra{l'un des hameaux de Kamnyu}\end{définition}
\begin{définition}\pcmn{干木鸟的大队之一}\end{définition}\end{entrée}

\begin{entrée}{rpɣo}{}{ⓔrpɣo} 
\classe{n} 
\begin{définition}\pfra{en haut de la montagne}\end{définition}
\begin{définition}\pcmn{高山上}\end{définition}\end{entrée}

\begin{entrée}{rpɣorɤku}{}{ⓔrpɣorɤku} 
\classe{n} 
\begin{définition}\pfra{cultures de haute montagne}\end{définition}
\begin{définition}\pcmn{高山作物(胡豆、豌豆、青稞、小麦、原根、莜麦)}\end{définition}\relationsémantique{参考}{\lien{ⓔrpɣo}{rpɣo}}\relationsémantique{参考}{\lien{ⓔtɤ-rɤku}{tɤ-rɤku}}\end{entrée}

\begin{entrée}{rpjɯ}{}{ⓔrpjɯ} 
\classe{vs} \paradigme{dir}{tɤ-}\paradigme{dir}{tɤ-}
\begin{définition}\pfra{abîmé (lait)}\end{définition}
\begin{définition}\pcmn{变质(牛奶)}\end{définition}
\begin{définition}\pfra{laisser s'abîmer (laità}\end{définition}
\begin{définition}\pcmn{让(牛奶)变质}\end{définition}
\begin{exemple}\pjya{tɤ-lu to-rpjɯ}\hspace{5pt}\pcmn{牛奶坏了}\end{exemple}
\begin{exemple}\pjya{tɤ-lu a-mɤ-tɤ-tɯ-sɯrpji ma nɤja}\hspace{5pt}\pcmn{你不要让牛奶变质,不然太可惜}\end{exemple}
\begin{sous-entrée}{sɯrpjɯ}{ⓔrpjɯⓝsɯrpjɯ} 
\classe{vt} \end{sous-entrée}

\end{entrée}

\begin{entrée}{rpɯ}{}{ⓔrpɯ} 
\classe{vs} \paradigme{dir}{tɤ-}\paradigme{dir}{thɯ-}
\begin{définition}\pfra{sale (cheveux)}\end{définition}
\begin{définition}\pcmn{头发又脏又长}\end{définition}
\begin{exemple}\pjya{ɯ-ku chɤ-rpɯ}\hspace{5pt}\pcmn{他的头发脏了}\end{exemple}\end{entrée}

\begin{entrée}{rqaco}{}{ⓔrqaco} 
\classe{n}  
\grammaire{n.lieu} 
\begin{définition}\pfra{Rqakyo (village de Gdongbrgyad)}\end{définition}
\begin{définition}\pcmn{尕脚村}\end{définition}\end{entrée}

\begin{entrée}{rqɤnrqɤn}{}{ⓔrqɤnrqɤn} 
\classe{idph.2} 
\begin{définition}\pfra{doré}\end{définition}
\begin{définition}\pcmn{形容红中带有金黄的晚霞的颜色}\end{définition}
\begin{exemple}\pjya{prɤɲi ɲɯ-ɣɯrni ʑo rqɤnrqɤn}\hspace{5pt}\pcmn{晚霞是红色的}\end{exemple}
\begin{exemple}\pjya{ɯ-mɲaʁ ɕɤwɤr to-βzu tɕe ɲɯ-ɣɯrni ʑo rqɤnrqɤn}\hspace{5pt}\pcmn{他眼睛有结膜炎,非常红}\end{exemple}\end{entrée}

\begin{entrée}{rqhaŋrqhaŋ}{}{ⓔrqhaŋrqhaŋ} 
\classe{idph.2} 
\begin{définition}\pfra{grand et mince}\end{définition}
\begin{définition}\pcmn{形容大而瘦的样子}\end{définition}
\begin{exemple}\pjya{si pjɤ-rom tɕe rqhaŋrqhaŋ ʑo ɲɯ-pa}\hspace{5pt}\pcmn{树干了,显得又大又瘦}\end{exemple}\end{entrée}

\begin{entrée}{rqhɤrqhɤt}{}{ⓔrqhɤrqhɤt} 
\classe{idph.2} 
\begin{définition}\pfra{qui vient de sortir (champignon, plante)}\end{définition}
\begin{définition}\pcmn{形容菌子、草等又新鲜又结实的样子}\end{définition}\end{entrée}

\begin{entrée}{rqhoʁ}{}{ⓔrqhoʁ} 
\classe{idph.1} 
\begin{définition}\pfra{coup de fusil}\end{définition}
\begin{définition}\pcmn{打枪的声音}\end{définition}
\begin{exemple}\pjya{ɕɤmɯɣdɯ rqhoʁ ʑo ta-lɤt}\hspace{5pt}\pcmn{他啪一声地射了枪}\end{exemple}\relationsémantique{参考}{\lien{ⓔɣɤrqhoʁrqhoʁ}{ɣɤrqhoʁrqhoʁ}}\end{entrée}

\begin{entrée}{rqoʁ}{}{ⓔrqoʁ} 
\classe{vt} \sens{1}\paradigme{dir}{kɤ-}
\begin{définition}\pfra{prendre dans les bras}\end{définition}
\begin{définition}\pcmn{抱;搂}\end{définition}
\begin{exemple}\pjya{kɤ́-wɣ-rqoʁ-a}\hspace{5pt}\pcmn{他抱了我}\end{exemple}
\begin{exemple}\pjya{ko-rqoʁ}\hspace{5pt}\pcmn{抱了他}\end{exemple}\sens{2}\paradigme{dir}{tɤ-}
\begin{définition}\pfra{prendre dans les bras et relever}\end{définition}
\begin{définition}\pcmn{抱起来}\end{définition}
\begin{exemple}\pjya{ta-rqoʁ}\hspace{5pt}\pcmn{他把他抱起来了}\end{exemple}\relationsémantique{参考}{\lien{ⓔtɯ-rqoʁ}{tɯ-rqoʁ}}\relationsémantique{参考}{\lien{ⓔarqɯrqoʁ}{arqɯrqoʁ}}\end{entrée}

\begin{entrée}{rʁu}{}{ⓔrʁu} 
\classe{vs} \sens{1}\paradigme{dir}{pɯ-}
\begin{définition}\pfra{s'évaporer}\end{définition}
\begin{définition}\pcmn{蒸发掉}\end{définition}
\begin{exemple}\pjya{tɯ-ci tú-wɣ-sɤla qhe pjɯ-rʁu ɕti}\hspace{5pt}\pcmn{烧开水的时候水会蒸发掉}\end{exemple}\sens{2}\paradigme{dir}{nɯ-}
\begin{définition}\pfra{se tarir (lait d'une vache)}\end{définition}
\begin{définition}\pcmn{干(奶水)}\end{définition}
\begin{exemple}\pjya{nɯŋa nɯ-rɤpɯ ɯ-qhu χsɯ-sla jamar tɕe ɯ-lu ɲɯ-rʁu ɕti}\hspace{5pt}\pcmn{奶牛生崽三个月后,奶水就干了}\end{exemple}\end{entrée}

\begin{entrée}{rʁɤβrʁɤβ}{}{ⓔrʁɤβrʁɤβ} 
\classe{idph.2} \sens{1}
\begin{définition}\pfra{rugueux}\end{définition}
\begin{définition}\pcmn{形容粗糙的样子}\end{définition}\sens{2}
\begin{définition}\pfra{séparées, pas ramassées ensemble (feuilles)}\end{définition}
\begin{définition}\pcmn{不连贯;分散(叶子)}\end{définition}
\begin{exemple}\pjya{zdɯɬa ɣɯ ɯ-jwaʁ nɯ kú-wɣ-rtoʁ qhe mɤ-andzoʁjoʁ kɯ rʁɤβrʁɤβ ʑo pa, tɕeri ɲɯ́-wɣ-nɤmɯma tɕe mnu ma mɤ-rʁom}\hspace{5pt}\pcmn{芍药花的叶子看起来不连贯、零碎,但是摸起来很光滑,不粗糙}\end{exemple}\end{entrée}

\begin{entrée}{rʁɤrʁɤt}{}{ⓔrʁɤrʁɤt} 
\classe{idph.2} 
\begin{définition}\pfra{tenant qqch très fort}\end{définition}
\begin{définition}\pcmn{形容抓得很紧、抱得很紧的样子}\end{définition}
\begin{exemple}\pjya{qajɯ nɯ sɯjno ɯ-ku rʁɤrʁɤt ʑo ku-ɴqoʁ ŋu}\hspace{5pt}\pcmn{虫子紧紧地抓住植物}\end{exemple}\end{entrée}

\begin{entrée}{rʁe}{}{ⓔrʁe} 
\classe{vt} \paradigme{dir}{nɯ-}\paradigme{dir}{thɯ-}
\begin{définition}\pfra{enfiler, passer à travers}\end{définition}
\begin{définition}\pcmn{穿}\end{définition}
\begin{exemple}\pjya{ɕnɤloʁ na-rʁe}\hspace{5pt}\pcmn{我穿了牛鼻圈}\end{exemple}
\begin{exemple}\pjya{zndɤrchɤβ na-rʁe}\hspace{5pt}\pcmn{他(把东西)穿进墙缝了}\end{exemple}
\begin{exemple}\pjya{taqaβrna na-rʁe (nɯ-rʁe-t-a)}\hspace{5pt}\pcmn{他穿了针(我穿了针)}\end{exemple}
\begin{exemple}\pjya{ɯ-jaʁ ɯ-pɤloʁ ɯ-ŋgɯ tha-rʁe}\hspace{5pt}\pcmn{他把手穿进袖子里了}\end{exemple}
\begin{exemple}\pjya{rkɤsnom ɯ-mi tha-rʁe}\hspace{5pt}\pcmn{他把脚穿进裤子里了}\end{exemple}\relationsémantique{参考}{\lien{ⓔnɯrʁe}{nɯrʁe}}\end{entrée}

\begin{entrée}{rʁom}{}{ⓔrʁom} 
\classe{vs} \paradigme{dir}{tɤ-}\paradigme{dir}{thɯ-}
\begin{définition}\pfra{rugueux}\end{définition}
\begin{définition}\pcmn{粗糙}\end{définition}
\begin{exemple}\pjya{tɯ-ŋga ɲɯ-rʁom}\hspace{5pt}\pcmn{衣服很粗糙}\end{exemple}
\begin{exemple}\pjya{nɯ ɯ-ɣmbɤrme ɲɯ-rʁom}\hspace{5pt}\pcmn{他的胡子很粗糙}\end{exemple}\relationsémantique{反义词}{\lien{ⓔmnu}{mnu}}\end{entrée}

\begin{entrée}{rʁoʁrʁoʁ}{}{ⓔrʁoʁrʁoʁ} 
\classe{idph.2} \sens{1}
\begin{définition}\pfra{frisé (cheveu)}\end{définition}
\begin{définition}\pcmn{卷(头发)}\end{définition}\sens{2}
\begin{définition}\pfra{peu agile}\end{définition}
\begin{définition}\pcmn{不灵活}\end{définition}
\begin{exemple}\pjya{ɯ-ku rʁoʁrʁoʁ ʑo ɲɯ-pa}\end{exemple}
\begin{exemple}\pjya{ɯ-ku kɯ-ɤrʁɯrʁu ci rʁoʁrʁoʁ ɲɯ-ŋu}\hspace{5pt}\pcmn{他的头发是卷卷的}\end{exemple}
\begin{exemple}\pjya{jiɕqha tɯrme nɯ mɤ-kɯ-ɤɕpala ci rʁoʁrʁoʁ ɲɯ-ɕti}\hspace{5pt}\pcmn{那个人动作不灵活}\end{exemple}\end{entrée}

\begin{entrée}{rʁɯβrʁɯβ}{}{ⓔrʁɯβrʁɯβ} 
\classe{idph.2} \sens{1}
\begin{définition}\pfra{qui porte beaucoup de fruit}\end{définition}
\begin{définition}\pcmn{结的果子很多}\end{définition}\sens{2}
\begin{définition}\pfra{devenu rugueux après avoir été séché}\end{définition}
\begin{définition}\pcmn{由于干燥而变得很粗糙}\end{définition}
\begin{exemple}\pjya{jiɕqha si nɯ rʁɯβrʁɯβ kɯ-ɤɣɯmat ci ɲɯ-ŋu}\hspace{5pt}\pcmn{那棵树结的果子很多}\end{exemple}
\begin{exemple}\pjya{ɯ-mat rʁɯβrʁɯβ ʑo ɲɯ-pa}\hspace{5pt}\pcmn{果子很多}\end{exemple}
\begin{exemple}\pjya{si ɯ-rtaʁ rʁɯβrʁɯβ ɲɯ-pa}\hspace{5pt}\pcmn{树枝很多}\end{exemple}
\begin{exemple}\pjya{sɤtɕha to-khrɯ rʁɯβrʁɯβ ʑo}\hspace{5pt}\pcmn{地变得很干}\end{exemple}
\begin{sous-entrée}{rʁɯβnɤrʁɯβ}{ⓔrʁɯβrʁɯβⓢ2ⓝrʁɯβnɤrʁɯβ} 
\classe{idph.3} 
\begin{définition}\pfra{pressé}\end{définition}
\begin{définition}\pcmn{急躁}\end{définition}
\begin{exemple}\pjya{jiɕqha tɯrme rcánɯ rʁɯβnɤrʁɯβ kɯ-znɤʁamɟa ci ɲɯ-ŋu}\hspace{5pt}\pcmn{那个人做事很急躁}\end{exemple}
\begin{exemple}\pjya{rʁɯβnɤrʁɯβ ɲɯ-ɤsɯ-ndza}\hspace{5pt}\pcmn{他在大声地吃(又脆又干的东西)}\end{exemple}\relationsémantique{参考}{\lien{ⓔsɤrʁɯrʁɯβ}{sɤrʁɯrʁɯβ}}\end{sous-entrée}

\end{entrée}

\begin{entrée}{rʁɯm}{}{ⓔrʁɯm} 
\classe{idph.1} 
\begin{définition}\pfra{se recroqueviller d'un seul coup}\end{définition}
\begin{définition}\pcmn{突然间卷起来;突然收拢起来}\end{définition}
\begin{exemple}\pjya{qarma mtshalu ɣɯ ɯ-mat, tɯ-jaʁ a-nɯ-ɤtɯɣ qhe, rʁɯm ʑo ɲɯ-ti tɕe, ɯ-rɣi nɯ pjɯ-nɯɬoʁ, ɯ-rqhu nɯ lu-orʁɯrʁu ŋu}\hspace{5pt}\pcmn{荨麻的果子碰到手就会突然间收拢起来,种子很快落地}\end{exemple}
\begin{exemple}\pjya{tɤŋkɯ chɯ́-wɣ-nɯ-pu tɕe tɤ-sɤɕke tɕe, rʁɯm ʑo tu-ti tɕe ku-owɯwum ŋu}\hspace{5pt}\pcmn{烤猪皮的时候,烧烫了就会马上卷起来,收拢起来}\end{exemple}\end{entrée}

\begin{entrée}{rsoŋrsoŋ}{}{ⓔrsoŋrsoŋ} 
\classe{idph.2} 
\begin{définition}\pfra{très poilu}\end{définition}
\begin{définition}\pcmn{形容毛茸茸}\end{définition}
\begin{exemple}\pjya{ɯ-mtɕhirme rsoŋrsoŋ ʑo ɲɯ-pa}\hspace{5pt}\pcmn{他的胡须是毛茸茸的}\end{exemple}\end{entrée}

\begin{entrée}{rsɯβrsɯβ/\variante{rsɯprsɯp}}{}{ⓔrsɯβrsɯβ} 
\classe{idph.2} 
\begin{définition}\pfra{poilu}\end{définition}
\begin{définition}\pcmn{毛多的}\end{définition}
\begin{exemple}\pjya{kɤɣɯrme ci rsɯprsɯp ɲ-ɯŋu}\hspace{5pt}\pcmn{毛很多}\end{exemple}
\begin{exemple}\pjya{ɯ-mi ɯ-rme rsɯprsɯp ʑo ɲɯ-pa}\hspace{5pt}\pcmn{他脚上的毛很多}\end{exemple}
\begin{exemple}\pjya{sɤtɕha ra rsɯprsɯp ʑo ɲɯ-pa}\hspace{5pt}\pcmn{地上草很多}\end{exemple}
\begin{sous-entrée}{rsɯβnɤrsɯβ}{ⓔrsɯβrsɯβⓝrsɯβnɤrsɯβ}
\begin{définition}\pfra{bruit de feuilles mortes}\end{définition}
\begin{définition}\pcmn{树叶沙沙响的声音}\end{définition}
\begin{exemple}\pjya{sɯŋgɯ rsɯpnɤrsɯp ɲɯ-ŋke}\hspace{5pt}\pcmn{他在森林里走动,发出沙沙声}\end{exemple}
\begin{exemple}\pjya{soʁma ɯ-ŋgɯ rsɯpnɤrsɯp kɤ-ari}\hspace{5pt}\pcmn{他在干草里去了,发出沙沙声}\end{exemple}\relationsémantique{参考}{\lien{ⓔɣɤrsɯβrsɯβ}{ɣɤrsɯβrsɯβ}}\relationsémantique{参考}{\lien{ⓔsɯβsɯβ}{sɯβsɯβ}}\end{sous-entrée}

\end{entrée}

\begin{entrée}{rtaβrɤn}{}{ⓔrtaβrɤn} 
\classe{n} 
\begin{définition}\pfra{cheval castré}\end{définition}
\begin{définition}\pcmn{骟马}\end{définition}\étymologie{rta}\end{entrée}

\begin{entrée}{rtakhaŋ}{}{ⓔrtakhaŋ} 
\classe{n} 
\begin{définition}\pfra{écurie}\end{définition}
\begin{définition}\pcmn{马厩}\end{définition}\relationsémantique{同义词}{\lien{ⓔmbrosta}{mbrosta}}\end{entrée}

\begin{entrée}{rtalu}{}{ⓔrtalu} 
\classe{n} 
\begin{définition}\pfra{année du cheval}\end{définition}
\begin{définition}\pcmn{马年}\end{définition}\étymologie{rta.lo}\end{entrée}

\begin{entrée}{rtamu}{}{ⓔrtamu} 
\classe{n} 
\begin{définition}\pfra{jument}\end{définition}
\begin{définition}\pcmn{母马}\end{définition}\relationsémantique{同义词}{\lien{ⓔrgonma}{rgonma}}\end{entrée}

\begin{entrée}{rtamdɯt}{}{ⓔrtamdɯt} 
\classe{n} 
\begin{définition}\pfra{un type de nœud}\end{définition}
\begin{définition}\pcmn{一种打结的方式}\end{définition}\étymologie{rta.mdud}\end{entrée}

\begin{entrée}{rtaphu}{}{ⓔrtaphu} 
\classe{n} 
\begin{définition}\pfra{étalon}\end{définition}
\begin{définition}\pcmn{公马}\end{définition}\relationsémantique{参考}{\lien{ⓔχsɤβ}{χsɤβ}}\end{entrée}

\begin{entrée}{rtaʁ}{}{ⓔrtaʁ} 
\classe{vs} \paradigme{dir}{tɤ-}
\begin{définition}\pfra{assez}\end{définition}
\begin{définition}\pcmn{足够}\end{définition}
\begin{exemple}\pjya{kɤ-ndza ɲɯ-rtaʁ}\hspace{5pt}\pcmn{够吃}\end{exemple}
\begin{exemple}\pjya{kɤ-ŋga ɲɯ-rtaʁ}\hspace{5pt}\pcmn{够穿}\end{exemple}
\begin{exemple}\pjya{ɯʑo kɯ tɯ-rtaʁ kɯ-me ʑo ɲɯ́-wɣ-rɯɣne-a ɲɯ-ŋu}\hspace{5pt}\pcmn{他无端地埋怨我}\end{exemple}
\begin{sous-entrée}{artaʁlaʁ}{ⓔrtaʁⓝartaʁlaʁ} 
\classe{vs} 
\begin{définition}\pfra{suffisant}\end{définition}
\begin{définition}\pcmn{足够}\end{définition}
\begin{exemple}\pjya{kɤ-mbi ɯ-spa ɲɯ-rkɯn tɕe kɯ-ɤrtaʁlaʁ maŋe}\hspace{5pt}\pcmn{给的东西太少,不够分给大家}\end{exemple}\relationsémantique{参考}{\lien{ⓔɣɤrtaʁ}{ɣɤrtaʁ}}\end{sous-entrée}

\end{entrée}

\begin{entrée}{rtɤβ}{}{ⓔrtɤβ} 
\classe{vt} \sens{1}\paradigme{dir}{pɯ-}
\begin{définition}\pfra{frapper}\end{définition}
\begin{définition}\pcmn{打(用细长的东西)}\end{définition}
\begin{exemple}\pjya{laʁjɯɣ pa-rtɤβ}\hspace{5pt}\pcmn{他用棍子打了}\end{exemple}
\begin{exemple}\pjya{tɤ-pɤtso mɯ́j-khɯ tɕe, tɤtar pɯ-rtaβ-a}\hspace{5pt}\pcmn{小孩子不听话,我用了一根木条打他}\end{exemple}\sens{2}\paradigme{dir}{nɯ-}\paradigme{dir}{kɤ-}
\begin{définition}\pfra{attacher}\end{définition}
\begin{définition}\pcmn{缠线;拴(鞋带);围起来}\end{définition}
\begin{exemple}\pjya{mthɯxtɕɤr na-rtɤβ}\hspace{5pt}\pcmn{他拴了腰带}\end{exemple}
\begin{exemple}\pjya{xtsɤxtɕɤr na-nɯ-rtɤβ}\hspace{5pt}\pcmn{他系了鞋带}\end{exemple}\relationsémantique{参考}{\lien{ⓔɯ-rtɯrtɤβ}{ɯ-rtɯrtɤβ}}\end{entrée}

\begin{entrée}{rtɤdaʁ}{}{ⓔrtɤdaʁ} 
\classe{n} 
\begin{définition}\pfra{corde en poil}\end{définition}
\begin{définition}\pcmn{牛毛搓成的绳子}\end{définition}\end{entrée}

\begin{entrée}{rtɤltɕaʁ}{}{ⓔrtɤltɕaʁ} 
\classe{n} 
\begin{définition}\pfra{fouet de cheval}\end{définition}
\begin{définition}\pcmn{打马的鞭子}\end{définition}\étymologie{rta.ltɕag}\end{entrée}

\begin{entrée}{rtɤmkɯχsɤl}{}{ⓔrtɤmkɯχsɤl} 
\classe{adv} 
\begin{définition}\pfra{directement}\end{définition}
\begin{définition}\pcmn{干脆}\end{définition}
\begin{exemple}\pjya{kɤ-ɕe mɯ́j-tɯ-sɯsɤm ɕti qhe rtɤmkɯχsɤl kɤ-nɯ-rɤʑi}\hspace{5pt}\pcmn{你既然不想去,就干脆留在这里}\end{exemple}\end{entrée}

\begin{entrée}{rtɕhɯɣrtɕhɯɣ}{}{ⓔrtɕhɯɣrtɕhɯɣ} 
\classe{idph.2} 
\begin{définition}\pfra{qui a un problème}\end{définition}
\begin{définition}\pcmn{形容有毛病,让人看不顺眼的样子}\end{définition}
\begin{exemple}\pjya{ki tɯrme ki rtɕɯɣrtɕɯɣ ʑo ɯ-tshɯɣa mɯ́j-βdi}\hspace{5pt}\pcmn{这个人样子不顺眼}\end{exemple}
\begin{exemple}\pjya{ki laχtɕha ki rtɕɯɣrtɕɯɣ, sna maŋe}\hspace{5pt}\pcmn{这个东西有毛病,不能要}\end{exemple}\end{entrée}

\begin{entrée}{rtɕhɯʁjɯ}{}{ⓔrtɕhɯʁjɯ} 
\classe{n} 
\begin{définition}\pfra{chenille}\end{définition}
\begin{définition}\pcmn{毛虫}\end{définition}\end{entrée}

\begin{entrée}{rtoʁ}{}{ⓔrtoʁ} 
\classe{vt} \sens{1}\paradigme{dir}{tɤ-}
\begin{définition}\pfra{regarder}\end{définition}
\begin{définition}\pcmn{看}\end{définition}
\begin{exemple}\pjya{laχtɕha tɤ-rtoʁ-a}\hspace{5pt}\pcmn{我看了东西}\end{exemple}
\begin{exemple}\pjya{jɯɣi ta-rtoʁ}\hspace{5pt}\pcmn{他看了书}\end{exemple}\sens{2}\paradigme{dir}{tɤ-}
\begin{définition}\pfra{vérifier}\end{définition}
\begin{définition}\pcmn{查看}\end{définition}
\begin{exemple}\pjya{smɤnba kɯ tɤ́-wɣ-rto-ʁa}\hspace{5pt}\pcmn{医生给我看病了}\end{exemple}\sens{3}\paradigme{dir}{kɤ-}
\begin{définition}\pfra{observer}\end{définition}
\begin{définition}\pcmn{观察}\end{définition}
\begin{exemple}\pjya{ɯʑo kɤ-rtoʁ-a ri, wuma ʑo ɲɯ-stu}\hspace{5pt}\pcmn{经过我的观察,我认为他是很努力的人}\end{exemple}\relationsémantique{同义词}{\lien{ⓔχpjɤt}{χpjɤt}}\relationsémantique{参考}{\lien{ⓔnɯsɯrtoʁ}{nɯsɯrtoʁ}}
\begin{sous-entrée}{sɯrtoʁ}{ⓔrtoʁⓢ3ⓝsɯrtoʁ} 
\classe{vt}  
\grammaire{caus} 
\begin{exemple}\pjya{smɤnba kɯ tɤ́-wɣ-sɯrtoʁ-a}\hspace{5pt}\pcmn{他请了医生给我看病}\end{exemple}\end{sous-entrée}

\begin{sous-entrée}{ʑɣɤsɯrtoʁ}{ⓔrtoʁⓢ3ⓝʑɣɤsɯrtoʁ} 
\classe{vi}  
\grammaire{refl} 
\begin{exemple}\pjya{smɤnba ɯ-phe kɯ-ʑɣɤsɯrtoʁ lɤ-ari}\hspace{5pt}\pcmn{他去看病了}\end{exemple}
\begin{exemple}\pjya{ɕ-tɤ-ʑɣɤsɯrtoʁ}\hspace{5pt}\pcmn{你去看病吧}\end{exemple}
\begin{exemple}\pjya{smɤnba a-tɤ-tɯ-ʑɣɤsɯrtoʁ}\hspace{5pt}\pcmn{你找医生看你的病吧}\end{exemple}
\begin{exemple}\pjya{smɤnba ci kɯ-ʑɣɤsɯrtoʁ jɤ-ari-a}\hspace{5pt}\pcmn{我去看医生了}\end{exemple}\end{sous-entrée}

\begin{sous-entrée}{nɤrtɯrtoʁ}{ⓔrtoʁⓢ3ⓝnɤrtɯrtoʁ} 
\classe{vt} 
\begin{définition}\pfra{regarder dans tous les sens}\end{définition}
\begin{définition}\pcmn{看来看去}\end{définition}\end{sous-entrée}

\begin{sous-entrée}{artɯrtoʁ}{ⓔrtoʁⓢ3ⓝartɯrtoʁ} 
\classe{vi}  
\grammaire{recip} 
\begin{définition}\pfra{se regarder les uns les autres}\end{définition}
\begin{définition}\pcmn{互相看}\end{définition}\end{sous-entrée}

\étymologie{rtogs}\end{entrée}

\begin{entrée}{rtoʁldɤn}{}{ⓔrtoʁldɤn} 
\classe{n} 
\begin{définition}\pfra{sage}\end{définition}
\begin{définition}\pcmn{得道者}\end{définition}\étymologie{rtogs.ldan}\end{entrée}

\begin{entrée}{rtoʁldɤn mɯntoʁ}{}{ⓔrtoʁldɤn mɯntoʁ} 
\classe{n} 
\begin{définition}\pfra{type de fleur}\end{définition}
\begin{définition}\pcmn{和尚花}\end{définition}\end{entrée}

\begin{entrée}{rtsa}{}{ⓔrtsa} 
\classe{vt} \paradigme{dir}{nɯ-}
\begin{définition}\pfra{enlever les organes sexuels des animaux femelles}\end{définition}
\begin{définition}\pcmn{取出雌性动物的胎盘或子宫}\end{définition}
\begin{exemple}\pjya{paʁ na-rtsa}\hspace{5pt}\pcmn{他阉割了母猪}\end{exemple}\end{entrée}

\begin{entrée}{rtsaka}{}{ⓔrtsaka} 
\classe{n} 
\begin{définition}\pfra{herbe verte}\end{définition}
\begin{définition}\pcmn{青草}\end{définition}\end{entrée}

\begin{entrée}{rtsaʁjɯɣ}{}{ⓔrtsaʁjɯɣ} 
\classe{n} 
\begin{définition}\pfra{bâton pour frapper les contrevenants à l'ordre dans le monastère}\end{définition}
\begin{définition}\pcmn{宗教活动时,维持纪律的和尚用来打人的棍子}\end{définition}\étymologie{dbʲug}\end{entrée}

\begin{entrée}{rtsatɯɣ}{}{ⓔrtsatɯɣ} 
\classe{n} 
\begin{définition}\pfra{herbe non identifiée qui rend malade le bétail qui l'absorbe}\end{définition}
\begin{définition}\pcmn{有毒的草,学名不明,牲畜吃了就生病}\end{définition}\étymologie{rtswa.dug}\end{entrée}

\begin{entrée}{rtsawa}{}{ⓔrtsawa} 
\classe{n}
\classe{vt} 
\begin{définition}\pfra{importance}\end{définition}
\begin{définition}\pcmn{重要性}\end{définition}
\begin{exemple}\pjya{ɯ-rtsawa ɲɯ-wxti}\hspace{5pt}\pcmn{很重要}\end{exemple}\relationsémantique{Component 2}{\lien{ⓔndo}{ndo}}
\begin{sous-entrée}{ɯ-rtsawa,ndo}{ⓔrtsawaⓝɯ-rtsawa,ndo} 
\classe{np} 
\begin{définition}\pfra{contrôler}\end{définition}
\begin{définition}\pcmn{掌握}\end{définition}
\begin{exemple}\pjya{jiʑo rɟɤlkhɤβ ɣɯ ji-rtsawa ɯ-kɯ-ndo nɯ @gongchandang ŋu}\hspace{5pt}\pcmn{共产党是掌握我们中国的}\end{exemple}\relationsémantique{Component 1}{\lien{}{ɯ-rtsawa}}\end{sous-entrée}

\étymologie{rtsa.ba}\end{entrée}

\begin{entrée}{rtsɤmkɯɣ}{}{ⓔrtsɤmkɯɣ} 
\classe{n} 
\begin{définition}\pfra{sac à rtsampa}\end{définition}
\begin{définition}\pcmn{糌粑口袋}\end{définition}\étymologie{rtsam.kʰug}\end{entrée}

\begin{entrée}{rtsɤmtɕhɯ}{}{ⓔrtsɤmtɕhɯ} 
\classe{n} 
\begin{définition}\pfra{eau que l'on met dans le bol pendant que l'on mange de la tsampa}\end{définition}
\begin{définition}\pcmn{挼糌粑时倒进碗里的水}\end{définition}\end{entrée}

\begin{entrée}{rtsɤxtɕɤr}{}{ⓔrtsɤxtɕɤr} 
\classe{n} 
\begin{définition}\pfra{bande colorée}\end{définition}
\begin{définition}\pcmn{花带子(小孩子带的)}\end{définition}
\begin{exemple}\pjya{rtsɤxtɕɤr nɯ-nɯrtaβ-a}\hspace{5pt}\pcmn{我带了花带子}\end{exemple}\end{entrée}

\begin{entrée}{rtshartsha}{}{ⓔrtshartsha} 
\classe{idph.2} 
\begin{définition}\pfra{un peu rugueux}\end{définition}
\begin{définition}\pcmn{形容略粗糙的样子}\end{définition}
\begin{exemple}\pjya{(qaɕparaz) khro mɤ-mpɕu, rʁom tsa rtshartsha}\hspace{5pt}\pcmn{(那种草)不光滑,有点粗糙}\end{exemple}\end{entrée}

\begin{entrée}{rtshɤrtshɤt}{}{ⓔrtshɤrtshɤt} 
\classe{idph.2} 
\begin{définition}\pfra{fin et résistant (feuille)}\end{définition}
\begin{définition}\pcmn{形容叶子、纸等薄而结实,不易折破的样子}\end{définition}\end{entrée}

\begin{entrée}{rtshom}{}{ⓔrtshom} 
\classe{vi} 
\begin{définition}\pfra{avoir une fente (seau en bois)}\end{définition}
\begin{définition}\pcmn{木板间的隙缝中出现裂口(木桶变干之后)}\end{définition}
\begin{exemple}\pjya{zɯm ɲɤ-rtshom tɕe ɲɤ-ri}\hspace{5pt}\pcmn{木桶有了裂口就在漏水}\end{exemple}
\begin{exemple}\pjya{zɯm tɯ-ɕoʁ tɯ-ɕoʁ tɤ-kɤ-sprɤt ɲɯ-ŋu tɕe, a-tɤ-zbaʁ tɕe ɲɯ-rtshom ɲɯ-ŋu, tɯ-ci tɤ-me tɕe ɲɯ-rtshom ɲɯ-ŋu}\hspace{5pt}\pcmn{木桶是由一条一条的木板条组成的,只要变干它就会出现裂口}\end{exemple}\end{entrée}

\begin{entrée}{rtshɯβrtshɯβ}{}{ⓔrtshɯβrtshɯβ} 
\classe{idph.2} 
\begin{définition}\pfra{grossier et piquant (surface)}\end{définition}
\begin{définition}\pcmn{形容平面表面粗糙的样子}\end{définition}\end{entrée}

\begin{entrée}{rtshɯrtshi}{}{ⓔrtshɯrtshi} 
\classe{idph.2} 
\begin{définition}\pfra{râpeux}\end{définition}
\begin{définition}\pcmn{形容物体摸起来粗糙刮手的样子}\end{définition}\end{entrée}

\begin{entrée}{rtsi}{}{ⓔrtsi} 
\classe{vt} \paradigme{dir}{tɤ-}\sens{1}
\begin{définition}\pfra{calculer}\end{définition}
\begin{définition}\pcmn{算}\end{définition}
\begin{exemple}\pjya{tɤ-rtsi-t-a}\hspace{5pt}\pcmn{我算了}\end{exemple}
\begin{exemple}\pjya{ji-nɯŋa thɤstɯɣ ɣɤʑu kɯ tɤ-rtsi-t-a}\hspace{5pt}\pcmn{我数了一下我们家的牛有多少头}\end{exemple}\sens{2}
\begin{définition}\pfra{considérer comme}\end{définition}
\begin{définition}\pcmn{当成}\end{définition}
\begin{exemple}\pjya{ki kɤ-rtsi kɯ-tu me nɤ}\hspace{5pt}\pcmn{这不算什么}\end{exemple}\relationsémantique{同义词}{\lien{ⓔχsɤrⓗ1}{χsɤr₁}}\relationsémantique{参考}{\lien{ⓔkɤrtsi}{kɤrtsi}}
\begin{sous-entrée}{sɤrtsi}{ⓔrtsiⓝsɤrtsi} 
\classe{vt} 
\begin{définition}\pfra{considérer comme}\end{définition}
\begin{définition}\pcmn{视为}\end{définition}
\begin{exemple}\pjya{mkhɤrmaŋ ra ɯ-rɟit ʑo tu-nɯ-sɤrtsi pjɤ-ŋu (=tu-nɯ-ste)}\hspace{5pt}\pcmn{他把群众当成自己的子女一样}\end{exemple}
\begin{exemple}\pjya{aʑo a-tɕɯ tu-ta-nɯ-sɤrtsi ŋu}\hspace{5pt}\pcmn{我把你当儿子一样看待}\end{exemple}\end{sous-entrée}

\begin{sous-entrée}{ʑɣɤrtsi}{ⓔrtsiⓝʑɣɤrtsi} 
\classe{vs}  
\grammaire{refl} \sens{1}
\begin{définition}\pfra{se considérer comme}\end{définition}
\begin{définition}\pcmn{自称;把自己视为}\end{définition}\relationsémantique{参考}{\lien{ⓔʑɣɤpaⓗ2}{ʑɣɤpa₂}}\end{sous-entrée}

\sens{2}
\begin{définition}\pfra{se compter parmi}\end{définition}
\begin{définition}\pcmn{把自己算在里面}\end{définition}
\begin{exemple}\pjya{ɯʑo to-nɯ-ʑɣɤrtsi}\hspace{5pt}\pcmn{他没有把自己算在里面}\end{exemple}\étymologie{rtsi}\end{entrée}

\begin{entrée}{rtsiaʁ}{}{ⓔrtsiaʁ} 
\classe{vs} 
\begin{définition}\pfra{escarpé et sinueux (chemin)}\end{définition}
\begin{définition}\pcmn{陡峭;难走(路)}\end{définition}
\begin{exemple}\pjya{tʂu ɲɯ-rtsiaʁ}\hspace{5pt}\pcmn{路很陡峭}\end{exemple}\end{entrée}

\begin{entrée}{rtsimu}{}{ⓔrtsimu} 
\classe{n} 
\begin{définition}\pfra{façon dont poussent les branches (arbre)}\end{définition}
\begin{définition}\pcmn{(树木的)长势}\end{définition}
\begin{exemple}\pjya{ki si ki ɯ-rtsimu ɲɯ-βdi}\hspace{5pt}\pcmn{这棵树长势很美}\end{exemple}\end{entrée}

\begin{entrée}{rtsot}{}{ⓔrtsot} 
\classe{n} 
\begin{définition}\pfra{vengeance}\end{définition}
\begin{définition}\pcmn{报仇}\end{définition}
\begin{exemple}\pjya{maka rtsot tu-βze-a ɲɯ-ntshi}\hspace{5pt}\pcmn{我要报仇}\end{exemple}\étymologie{rtsod}\end{entrée}

\begin{entrée}{rtsɯβ}{}{ⓔrtsɯβ} 
\classe{vs}
\classe{vi} 
\begin{définition}\pfra{qui contient beaucoup de gros grains}\end{définition}
\begin{définition}\pcmn{粗粮多}\end{définition}
\begin{exemple}\pjya{tɤjlu ɲɯ-rtsɯβ}\hspace{5pt}\pcmn{面粉加的粗粮多}\end{exemple}\relationsémantique{Component 2}{\lien{ⓔrtsɯβ}{rtsɯβ}}
\begin{sous-entrée}{ɯ-lu,rtsɯβ}{ⓔrtsɯβⓝɯ-lu,rtsɯβ} 
\classe{np} 
\begin{définition}\pfra{avoir un cycle astrologique complet}\end{définition}
\begin{définition}\pcmn{是……的本命年}\end{définition}
\begin{exemple}\pjya{ɣɯjpa ɯ-lu rtsɯβ}\hspace{5pt}\pcmn{今年是他的本命年}\end{exemple}
\begin{exemple}\pjya{nɤ-lu sɤ-rtsɯβ}\hspace{5pt}\pcmn{你的本命年}\end{exemple}\relationsémantique{Component 1}{\lien{}{ɯ-lu}}\end{sous-entrée}

\end{entrée}

\begin{entrée}{rtsɯɕaŋlaŋmtɕɤt}{}{ⓔrtsɯɕaŋlaŋmtɕɤt} 
\classe{n} 
\begin{définition}\pfra{toutes les plantes}\end{définition}
\begin{définition}\pcmn{所有的草木}\end{définition}\étymologie{rtsi.ɕiŋ.tham.tɕɤt}\end{entrée}

\begin{entrée}{rtsɯɣ}{}{ⓔrtsɯɣ} 
\classe{vt} \paradigme{dir}{\_}
\begin{définition}\pfra{empiler}\end{définition}
\begin{définition}\pcmn{堆起来}\end{définition}
\begin{exemple}\pjya{si pɯ-rtsɯɣ-a}\hspace{5pt}\pcmn{我把木头堆起来了}\end{exemple}
\begin{exemple}\pjya{tɤɕi pɯ-rtsɯɣ-a}\hspace{5pt}\pcmn{我把青稞堆起来了}\end{exemple}\étymologie{rtsigs}\end{entrée}

\begin{entrée}{rtsɯpɣaʁ}{}{ⓔrtsɯpɣaʁ} 
\classe{n} 
\begin{définition}\pfra{labourage après la récolte}\end{définition}
\begin{définition}\pcmn{庄稼收割了以后重新翻地}\end{définition}
\begin{exemple}\pjya{rtsɯpɣaʁ lɤ-lɤt-i}\hspace{5pt}\pcmn{我们翻了地}\end{exemple}\relationsémantique{参考}{\lien{ⓔrɯrtsɯpɣaʁ}{rɯrtsɯpɣaʁ}}\relationsémantique{参考}{\lien{ⓔnɯrtsɯpɣaʁ}{nɯrtsɯpɣaʁ}}\end{entrée}

\begin{entrée}{rtsɯtpa}{}{ⓔrtsɯtpa} 
\classe{n} 
\begin{définition}\pfra{poils épais}\end{définition}
\begin{définition}\pcmn{粗毛}\end{définition}\end{entrée}

\begin{entrée}{rtsɯtʂɯɣ}{}{ⓔrtsɯtʂɯɣ} 
\classe{n} 
\begin{définition}\pfra{compte}\end{définition}
\begin{définition}\pcmn{帐}\end{définition}
\begin{exemple}\pjya{nɤ-rtsɯtʂɯɣ te-a ra, ma-ta-ta}\hspace{5pt}\pcmn{我要跟你算帐,我不会放过你的}\end{exemple}\étymologie{rtsi.sgrig}\end{entrée}

\begin{entrée}{rtsɯz}{}{ⓔrtsɯz} 
\classe{vt} \paradigme{dir}{nɯ-}\paradigme{dir}{tɤ-}
\begin{définition}\pfra{calculer}\end{définition}
\begin{définition}\pcmn{算}\end{définition}
\begin{exemple}\pjya{thɤstɯɣ ɲɯ-ɤmɯβɟɤt-i nɯ-rtsɯz-a}\hspace{5pt}\pcmn{我算了一下我们每个人可以分多少}\end{exemple}
\begin{exemple}\pjya{fsapaʁ tɤ-rtsɯz-a}\hspace{5pt}\pcmn{我数了一下牲畜}\end{exemple}\relationsémantique{同义词}{\lien{ⓔχsɤrⓗ1}{χsɤr₁}}\relationsémantique{参考}{\lien{ⓔrtsi}{rtsi}}\end{entrée}

\begin{entrée}{rɯ}{₂}{ⓔrɯⓗ2} 
\classe{n} 
\begin{définition}\pfra{lieu d'habitation temporaire dans la montagne}\end{définition}
\begin{définition}\pcmn{山上、牧草上暂住的地方(帐篷里)}\end{définition}
\begin{exemple}\pjya{rɯ ɲɯ-scat-a}\hspace{5pt}\pcmn{我在搬帐篷}\end{exemple}\étymologie{ri}\end{entrée}

\begin{entrée}{rɯ}{₁}{ⓔrɯⓗ1} 
\classe{vs} 
\begin{définition}\pfra{épais (liquide qui contient beaucoup de matière grasse)}\end{définition}
\begin{définition}\pcmn{浓(液体里的牛奶或者油)}\end{définition}
\begin{exemple}\pjya{tɤ-lu kɯ-rɯ to-lɤt}\hspace{5pt}\pcmn{他倒了很多牛奶(茶里的牛奶很浓)}\end{exemple}
\begin{exemple}\pjya{tɤ-lu mɯ́j-rɯ tɕe qhluqhlu ʑo ɲɯ-pa}\hspace{5pt}\pcmn{牛奶不浓,(茶里)只有一点点白色}\end{exemple}\end{entrée}

\begin{entrée}{rɯβ}{}{ⓔrɯβ} 
\classe{vs} \paradigme{dir}{kɤ-}
\begin{définition}\pfra{impraticable}\end{définition}
\begin{définition}\pcmn{不好走(杂草、灌木丛生)}\end{définition}
\begin{exemple}\pjya{sɯŋgɯ ɲɯ-rɯβ}\hspace{5pt}\pcmn{森林不好走}\end{exemple}\end{entrée}

\begin{entrée}{rɯβluβra}{}{ⓔrɯβluβra} 
\classe{vs}  
\grammaire{denom} 
\begin{définition}\pfra{qui donne de bons conseils}\end{définition}
\begin{définition}\pcmn{善于出主意}\end{définition}
\begin{exemple}\pjya{ɯʑo kɯ-rɯβluβra ci ɲɯ-ŋu}\hspace{5pt}\pcmn{这个人善于出主意}\end{exemple}\relationsémantique{参考}{\lien{ⓔβluβra}{βluβra}}\end{entrée}

\begin{entrée}{rɯβnɤrɯβ}{}{ⓔrɯβnɤrɯβ} 
\classe{idph.3} 
\begin{définition}\pfra{qui coule sans s'arrêter goute à goute}\end{définition}
\begin{définition}\pcmn{不停地漏出来;不停地滴出来}\end{définition}
\begin{exemple}\pjya{tɤ-se rɯβrɯβ nɤ rɯβrɯβ ɲɯ-nɯɬoʁ}\hspace{5pt}\pcmn{血一滴一滴地流出来}\end{exemple}
\begin{exemple}\pjya{tɯ-ci rɯβrɯβ nɤ rɯβrɯβ ɲɯ-nɯftsaʁ}\hspace{5pt}\pcmn{一滴一滴地漏水}\end{exemple}
\begin{sous-entrée}{rɯwɯwi}{ⓔrɯβnɤrɯβⓝrɯwɯwi} 
\classe{idph.7} 
\begin{exemple}\pjya{tɯ-ci rɯwɯwi ʑo pɯ-ɣe}\hspace{5pt}\pcmn{水慢慢地往下流}\end{exemple}\end{sous-entrée}

\begin{sous-entrée}{rɯwɯrawi}{ⓔrɯβnɤrɯβⓝrɯwɯrawi} 
\classe{idph.8} 
\begin{définition}\pfra{confus}\end{définition}
\begin{définition}\pcmn{心情烦乱}\end{définition}
\begin{exemple}\pjya{ɯ-kɤ-nɯzdɯɣ ɲɯ-dɤn tɕe, ɯ-sɯm rɯwɯrawi ɲɯ-xtsu}\hspace{5pt}\pcmn{他担心的事情很多,心烦意乱}\end{exemple}
\begin{exemple}\pjya{ɯ-sɯm rɯwɯrawi ʑo ɲɯ-βze}\hspace{5pt}\pcmn{心情很烦乱}\end{exemple}\relationsémantique{参考}{\lien{ⓔɣɤrɯβrɯβ}{ɣɤrɯβrɯβ}}\end{sous-entrée}

\end{entrée}

\begin{entrée}{rɯcɤβŋgɤβ}{}{ⓔrɯcɤβŋgɤβ} 
\classe{vs} \paradigme{dir}{tɤ-}
\begin{définition}\pfra{être orgueilleux}\end{définition}
\begin{définition}\pcmn{骄傲}\end{définition}
\begin{exemple}\pjya{ma-tɯ-rɯcɤβŋgɤβ}\hspace{5pt}\pcmn{你不要骄傲}\end{exemple}
\begin{exemple}\pjya{tɤ-rɯcɤβŋgaβ-a}\hspace{5pt}\pcmn{我在他面前骄傲了一下}\end{exemple}
\begin{exemple}\pjya{ɯʑo kɤ-rɯcɤŋgɤβ mɤ-spe}\hspace{5pt}\pcmn{他不会骄傲自大}\end{exemple}\étymologie{fn:骄傲}\end{entrée}

\begin{entrée}{rɯcɯnmu}{}{ⓔrɯcɯnmu} 
\classe{vi} \paradigme{dir}{tɤ-}
\begin{définition}\pfra{répandre des rumeurs}\end{définition}
\begin{définition}\pcmn{挑拨离间}\end{définition}
\begin{exemple}\pjya{a-mɤ-tɯ-rɯcɯnmu}\hspace{5pt}\pcmn{你不要挑拨离间}\end{exemple}
\begin{exemple}\pjya{jiɕqha kɯ-rɯcɯnmu ci ɲɯ-ŋu}\hspace{5pt}\pcmn{这个人爱挑拨离间}\end{exemple}\relationsémantique{参考}{\lien{ⓔcɯnmu}{cɯnmu}}\end{entrée}

\begin{entrée}{rɯɕaŋchi}{}{ⓔrɯɕaŋchi} 
\classe{vi} \paradigme{dir}{tɤ-}
\begin{définition}\pfra{aimer se maquiller et porter des habits luxueux}\end{définition}
\begin{définition}\pcmn{喜欢打扮,穿豪华的衣服,爱美}\end{définition}
\begin{exemple}\pjya{iɕqha tɕheme nɯ ɲɯ-rɯɕaŋchi}\hspace{5pt}\pcmn{这个女子喜欢打扮}\end{exemple}
\begin{exemple}\pjya{ɕɯŋgɯ mɯ-pɯ-rɯɕaŋchi, tham to-rɯɕaŋchi}\hspace{5pt}\pcmn{他以前不打扮,现在就最爱打扮了}\end{exemple}\end{entrée}

\begin{entrée}{rɯɕɤtsha}{}{ⓔrɯɕɤtsha} 
\classe{vi} \paradigme{dir}{tɤ-}
\begin{définition}\pfra{faire attention}\end{définition}
\begin{définition}\pcmn{小心}\end{définition}
\begin{exemple}\pjya{tɤ-rɯɕɤtsha ma ɲɯ-sɤɣʑɯr}\hspace{5pt}\pcmn{你小心,很危险}\end{exemple}\relationsémantique{同义词}{\lien{ⓔrɯndzaŋspa}{rɯndzaŋspa}}\end{entrée}

\begin{entrée}{rɯɕmi}{}{ⓔrɯɕmi} 
\classe{vi}  
\grammaire{caus} \paradigme{dir}{tɤ-}\paradigme{dir}{tɤ-}
\begin{définition}\pfra{parler}\end{définition}
\begin{définition}\pcmn{讲}\end{définition}
\begin{exemple}\pjya{jiɕqha nɯ ɲɯ-rɯɕmi}\hspace{5pt}\pcmn{那个人在说话}\end{exemple}
\begin{exemple}\pjya{qajdo to-rɯɕmi}\hspace{5pt}\pcmn{乌鸦说话了}\end{exemple}
\begin{exemple}\pjya{li ci tɤti ma jiɕqha tu-rɯɕmi tɕe mɯ-kɤ-tso-a}\hspace{5pt}\pcmn{你再讲一次,刚才他在讲话,我没有听清楚}\end{exemple}
\begin{exemple}\pjya{tɤ-rɯɕmi jɤɣ}\hspace{5pt}\pcmn{你可以讲}\end{exemple}
\begin{sous-entrée}{zrɯɕmi}{ⓔrɯɕmiⓝzrɯɕmi} 
\classe{vt} \end{sous-entrée}

\begin{définition}\pfra{faire parler}\end{définition}
\begin{définition}\pcmn{令……说话}\end{définition}
\begin{sous-entrée}{anɯrɯɕmɯɕmi}{ⓔrɯɕmiⓝanɯrɯɕmɯɕmi} 
\classe{vi}  
\grammaire{recip} 
\begin{définition}\pfra{se parler l'un à l'autre}\end{définition}
\begin{définition}\pcmn{互相讲话}\end{définition}\end{sous-entrée}

\end{entrée}

\begin{entrée}{rɯɕmɯlaʁ}{}{ⓔrɯɕmɯlaʁ} 
\classe{vi} 
\begin{définition}\pfra{parler}\end{définition}
\begin{définition}\pcmn{讲话}\end{définition}
\begin{exemple}\pjya{aʑo kɤ-rɯɕmɯlaʁ mɤ-cha-a wo}\hspace{5pt}\pcmn{我不善于给别人打招呼}\end{exemple}
\begin{exemple}\pjya{kɯm ɯ-pɕi nɯtɕu kɯ-rɯɕmɯlaʁ kɯ-fse ci ɣɤʑu}\hspace{5pt}\pcmn{门后面好像有人在讲话}\end{exemple}\end{entrée}

\begin{entrée}{rɯɕmɯχtɤm}{}{ⓔrɯɕmɯχtɤm} 
\classe{vi} \paradigme{dir}{pɯ-}
\begin{définition}\pfra{dire des balivernes}\end{définition}
\begin{définition}\pcmn{啰嗦,说废话}\end{définition}
\begin{exemple}\pjya{ma-tɯ-rɯɕmɯχtɤm kɯ nɤ-ma nɯ ʑ-nɯ-nɤme}\hspace{5pt}\pcmn{不要说废话,去干你的事}\end{exemple}\relationsémantique{参考}{\lien{ⓔrɯɕmi}{rɯɕmi}}\end{entrée}

\begin{entrée}{rɯɕoŋβzu}{}{ⓔrɯɕoŋβzu} 
\classe{vi} \paradigme{dir}{nɯ-}
\begin{définition}\pfra{faire des travaux du bois}\end{définition}
\begin{définition}\pcmn{做木工}\end{définition}
\begin{exemple}\pjya{ʑara ɣɯ ku-rɯɕoŋβzu-a}\hspace{5pt}\pcmn{我在他们家里做木工}\end{exemple}\relationsémantique{参考}{\lien{ⓔɕoŋβzu}{ɕoŋβzu}}\end{entrée}

\begin{entrée}{rɯdaʁ}{}{ⓔrɯdaʁ} 
\classe{n} 
\begin{définition}\pfra{bête sauvage}\end{définition}
\begin{définition}\pcmn{野兽}\end{définition}\étymologie{ri.dʷags}\end{entrée}

\begin{entrée}{rɯfsɤri}{}{ⓔrɯfsɤri} 
\classe{vt} \paradigme{dir}{lɤ-}
\begin{définition}\pfra{filer pour faire une ficelle}\end{définition}
\begin{définition}\pcmn{搓成线}\end{définition}
\begin{exemple}\pjya{tasa cho mphɯli lɤ-rɯfsɤri-t-a}\hspace{5pt}\pcmn{我把大麻和亚麻搓成线}\end{exemple}\relationsémantique{参考}{\lien{ⓔtɤ-fsɤri}{tɤ-fsɤri}}\end{entrée}

\begin{entrée}{rɯftɕaka}{}{ⓔrɯftɕaka} 
\classe{vi}  
\grammaire{denom} \paradigme{dir}{tɤ-}
\begin{définition}\pfra{préparer}\end{définition}
\begin{définition}\pcmn{准备}\end{définition}
\begin{exemple}\pjya{kɯ-ɕe ɲɯ-rɯftɕaka}\hspace{5pt}\pcmn{他准备去}\end{exemple}\relationsémantique{参考}{\lien{ⓔftɕaka}{ftɕaka}}\relationsémantique{参考}{\lien{ⓔnɯftɕaka}{nɯftɕaka}}\relationsémantique{参考}{\lien{ⓔsɤftɕaka}{sɤftɕaka}}\end{entrée}

\begin{entrée}{rɯftɕɤfkɤt}{}{ⓔrɯftɕɤfkɤt} 
\classe{vi}  
\grammaire{denom} \paradigme{dir}{tɤ-}
\begin{définition}\pfra{donner son avis, donner des suggestions}\end{définition}
\begin{définition}\pcmn{给别人出主意(自作多情)}\end{définition}
\begin{exemple}\pjya{jiɕqha nɯ ɲɯ-rɯftɕɤfkɤt}\hspace{5pt}\pcmn{那个人在给别人出主意}\end{exemple}\relationsémantique{参考}{\lien{ⓔftɕɤfkɤt}{ftɕɤfkɤt}}\end{entrée}

\begin{entrée}{rɯɣ}{}{ⓔrɯɣ} 
\classe{vs} 
\begin{définition}\pfra{précieux}\end{définition}
\begin{définition}\pcmn{贵重}\end{définition}\end{entrée}

\begin{entrée}{rɯɣnɤn}{}{ⓔrɯɣnɤn} 
\classe{vi} \paradigme{dir}{nɯ-}\paradigme{dir}{nɯ-}
\begin{définition}\pfra{s'opposer, chercher à causer des ennuis}\end{définition}
\begin{définition}\pcmn{作对;找茬}\end{définition}
\begin{définition}\pfra{plaisanter}\end{définition}
\begin{définition}\pcmn{开玩笑,逗着玩}\end{définition}
\begin{exemple}\pjya{a-ɕki ɲɯ-rɯɣnɤn}\hspace{5pt}\pcmn{他跟我作对}\end{exemple}
\begin{exemple}\pjya{ma-nɯ-tɯ-rɯɣnɤn}\hspace{5pt}\pcmn{你不要(跟他)作对}\end{exemple}
\begin{exemple}\pjya{ma-nɯ-kɯ-zrɯɣnan-a}\hspace{5pt}\pcmn{你不要跟我开玩笑}\end{exemple}
\begin{exemple}\pjya{nɯ́-wɣ-zrɯɣnan-a}\hspace{5pt}\pcmn{他跟我开了玩笑}\end{exemple}
\begin{sous-entrée}{zrɯɣnɤn}{ⓔrɯɣnɤnⓝzrɯɣnɤn} 
\classe{vt} \end{sous-entrée}

\end{entrée}

\begin{entrée}{rɯɣne}{}{ⓔrɯɣne} 
\classe{vt} \paradigme{dir}{nɯ-}
\begin{définition}\pfra{critiquer}\end{définition}
\begin{définition}\pcmn{责怪}\end{définition}
\begin{exemple}\pjya{nɯ-rɯɣne-t-a}\hspace{5pt}\pcmn{我骂了他}\end{exemple}
\begin{exemple}\pjya{nɯ́-wɣ-rɯɣne-a}\hspace{5pt}\pcmn{他骂了我}\end{exemple}
\begin{exemple}\pjya{pɯ-az-rɯɣne}\hspace{5pt}\pcmn{(以前)他在骂他}\end{exemple}
\begin{exemple}\pjya{aj pɯ-ɣɤtɕa-a, ma-nɯ-kɯ-rɯɣne-a}\hspace{5pt}\pcmn{我错了,你不要骂我}\end{exemple}\end{entrée}

\begin{entrée}{rɯjɤɣɤt}{}{ⓔrɯjɤɣɤt} 
\classe{vi}  
\grammaire{denom} \paradigme{dir}{pɯ-}
\begin{définition}\pfra{aller aux toilettes}\end{définition}
\begin{définition}\pcmn{上厕所}\end{définition}\relationsémantique{参考}{\lien{ⓔjɤɣɤt}{jɤɣɤt}}\end{entrée}

\begin{entrée}{rɯɟuli}{}{ⓔrɯɟuli} 
\classe{vi}  
\grammaire{denom} 
\begin{définition}\pfra{jouer de la flûte}\end{définition}
\begin{définition}\pcmn{吹笛子}\end{définition}\relationsémantique{参考}{\lien{ⓔɟuli}{ɟuli}}\end{entrée}

\begin{entrée}{rɯkɤtɯm}{}{ⓔrɯkɤtɯm} 
\classe{vi}  
\grammaire{denom} \paradigme{dir}{tɤ-}
\begin{définition}\pfra{enrouler de fil l'appareil pour tisser, enrouler un fil en boule}\end{définition}
\begin{définition}\pcmn{左右缠绕,把线缠成球形}\end{définition}
\begin{exemple}\pjya{tɤ-ri tɤ-rɯkɤtɯm}\hspace{5pt}\pcmn{你缠线吧}\end{exemple}\end{entrée}

\begin{entrée}{rɯkhɤcɤl}{}{ⓔrɯkhɤcɤl} 
\classe{vi}  
\grammaire{denom} \paradigme{dir}{pɯ-}
\begin{définition}\pfra{bavarder}\end{définition}
\begin{définition}\pcmn{聊天}\end{définition}
\begin{exemple}\pjya{jiɕqha nɯ ɲɯ-rɯkhɤcɤl}\hspace{5pt}\pcmn{那个人在聊天}\end{exemple}
\begin{exemple}\pjya{pɯ-rɯkhɤcɤl-i}\hspace{5pt}\pcmn{我们聊天了}\end{exemple}\end{entrée}

\begin{entrée}{rɯkhɤrlɤn}{}{ⓔrɯkhɤrlɤn} 
\classe{vi}  
\grammaire{denom} \paradigme{dir}{tɤ-}
\begin{définition}\pfra{construire une maison}\end{définition}
\begin{définition}\pcmn{修房子}\end{définition}
\begin{exemple}\pjya{roŋwa thɯ-mɤɕi rɯkhɤrlɤn}\hspace{5pt}\pcmn{农民富有了就修房子}\end{exemple}\relationsémantique{参考}{\lien{ⓔkhɤrlɤn}{khɤrlɤn}}\end{entrée}

\begin{entrée}{rɯkhon}{}{ⓔrɯkhon} 
\classe{vi}  
\grammaire{denom} \paradigme{dir}{tɤ-}
\begin{définition}\pfra{préparer au cas où}\end{définition}
\begin{définition}\pcmn{预备}\end{définition}
\begin{exemple}\pjya{ɯ-qhu kɤ-nɤma nɯnɯ tham tɕe tu-kɯ-rɯkhon ra}\hspace{5pt}\pcmn{以后的工作要现在预备好(以防万一)}\end{exemple}\relationsémantique{参考}{\lien{ⓔɯ-khon}{ɯ-khon}}\end{entrée}

\begin{entrée}{rɯkhramba}{}{ⓔrɯkhramba} 
\classe{vi}  
\grammaire{denom} \paradigme{dir}{tɤ-}
\begin{définition}\pfra{mentir}\end{définition}
\begin{définition}\pcmn{撒谎}\end{définition}
\begin{exemple}\pjya{jiɕqha nɯ rɯkhramba}\hspace{5pt}\pcmn{那个人在撒谎}\end{exemple}
\begin{exemple}\pjya{mɤ-kɯ-rɯkhramba ci ŋu}\hspace{5pt}\pcmn{他是一个不撒谎的人}\end{exemple}
\begin{exemple}\pjya{nɯkhramba}\end{exemple}\end{entrée}

\begin{entrée}{rɯkɯɕnom}{}{ⓔrɯkɯɕnom} 
\classe{vi}  
\grammaire{denom} \paradigme{dir}{tɤ-}
\begin{définition}\pfra{monter en épi}\end{définition}
\begin{définition}\pcmn{抽穗}\end{définition}\relationsémantique{参考}{\lien{ⓔkɯɕnom}{kɯɕnom}}\end{entrée}

\begin{entrée}{rɯkɯmaʁ}{}{ⓔrɯkɯmaʁ} 
\classe{vi}  
\grammaire{denom} \paradigme{dir}{thɯ-}
\begin{définition}\pfra{maladroit}\end{définition}
\begin{définition}\pcmn{笨拙,经常损坏东西}\end{définition}
\begin{exemple}\pjya{jiɕqha nɯ ɲɯ-rɯkɯmaʁ}\hspace{5pt}\pcmn{那个人动作笨拙}\end{exemple}\relationsémantique{参考}{\lien{ⓔmaʁⓗ1}{maʁ₁}}\relationsémantique{参考}{\lien{ⓔnɯkɯmaʁ}{nɯkɯmaʁ}}\end{entrée}

\begin{entrée}{rɯkɯŋu}{}{ⓔrɯkɯŋu} 
\classe{vs}  
\grammaire{denom} \paradigme{dir}{tɤ-}
\begin{définition}\pfra{attentionné envers sa famille}\end{définition}
\begin{définition}\pcmn{关心家庭}\end{définition}
\begin{exemple}\pjya{tɯrme kɯ-rɯkɯŋu ci ɲɯ-ŋu}\hspace{5pt}\pcmn{他是个顾家的人}\end{exemple}
\begin{exemple}\pjya{jiɕqha tɯrme ɲɯ-rɯkɯŋu tɕe, laχtɕha wuma ɲɯ-ɤsɯ-χtɯ}\hspace{5pt}\pcmn{那个人很顾家,买很多东西回家}\end{exemple}\relationsémantique{反义词}{\lien{ⓔrɯkɯmaʁ}{rɯkɯmaʁ}}\relationsémantique{参考}{\lien{ⓔŋu}{ŋu}}\end{entrée}

\begin{entrée}{rɯlajɯ}{}{ⓔrɯlajɯ} 
\classe{vi}  
\grammaire{denom} \paradigme{dir}{thɯ-}
\begin{définition}\pfra{chanter un chant de montagne}\end{définition}
\begin{définition}\pcmn{唱山歌}\end{définition}
\begin{exemple}\pjya{chɤ-rɯlajɯ}\hspace{5pt}\pcmn{他唱了山歌}\end{exemple}\relationsémantique{参考}{\lien{ⓔlajɯ}{lajɯ}}\end{entrée}

\begin{entrée}{rɯlaʁjɤt}{}{ⓔrɯlaʁjɤt} 
\classe{vi}  
\grammaire{denom} 
\begin{définition}\pfra{faire du travail manuel}\end{définition}
\begin{définition}\pcmn{做手工活}\end{définition}\relationsémantique{参考}{\lien{ⓔlaʁjɤt}{laʁjɤt}}\end{entrée}

\begin{entrée}{rɯlɯ}{}{ⓔrɯlɯ} 
\classe{n} 
\begin{définition}\pfra{boulette}\end{définition}
\begin{définition}\pcmn{小丸;小团;小球}\end{définition}\end{entrée}

\begin{entrée}{rɯmu}{}{ⓔrɯmu} 
\classe{n} 
\begin{définition}\pfra{motif}\end{définition}
\begin{définition}\pcmn{纹路}\end{définition}
\begin{exemple}\pjya{jaχpa rɯmu}\hspace{5pt}\pcmn{手纹}\end{exemple}\étymologie{ri.mo}\end{entrée}

\begin{entrée}{rɯm}{}{ⓔrɯm} 
\classe{vt}  
\grammaire{denom} \paradigme{dir}{lɤ-}
\begin{définition}\pfra{faire de la ficelle en roulant dans les mains (sens inverse des aiguilles d'une montre)}\end{définition}
\begin{définition}\pcmn{搓线(逆时针方向)}\end{définition}
\begin{exemple}\pjya{tɤ-ri lɤ-rɯm-a}\hspace{5pt}\pcmn{我搓了线}\end{exemple}\relationsémantique{同义词}{\lien{ⓔpɣo}{pɣo}}\relationsémantique{同义词}{\lien{ⓔrɤjɯɣ}{rɤjɯɣ}}\end{entrée}

\begin{entrée}{rɯmani}{}{ⓔrɯmani} 
\classe{vi}  
\grammaire{denom} \paradigme{dir}{nɯ-}
\begin{définition}\pfra{réciter les mantras}\end{définition}
\begin{définition}\pcmn{念玛尼}\end{définition}
\begin{exemple}\pjya{rgɤrgɯn ra ɲɯ-rɯmani-nɯ}\hspace{5pt}\pcmn{老年人们在念玛尼}\end{exemple}\end{entrée}

\begin{entrée}{rɯmba}{}{ⓔrɯmba} 
\classe{n} 
\begin{définition}\pfra{espèce}\end{définition}
\begin{définition}\pcmn{种类}\end{définition}\étymologie{rim.pa}\end{entrée}

\begin{entrée}{rɯmboʁkhɯr}{}{ⓔrɯmboʁkhɯr} 
\classe{vt}  
\grammaire{denom}
\grammaire{denom} \paradigme{dir}{thɯ-}
\begin{définition}\pfra{envelopper avec un tissu rectangulaire}\end{définition}
\begin{définition}\pcmn{用正方形的布包起来}\end{définition}
\begin{exemple}\pjya{cho-rɯmboʁkhɯr}\hspace{5pt}\pcmn{他把它包起来了}\end{exemple}\relationsémantique{参考}{\lien{ⓔmboʁkhɯr}{mboʁkhɯr}}\end{entrée}

\begin{entrée}{rɯmpɕɯmɤr}{}{ⓔrɯmpɕɯmɤr} 
\classe{vi}  
\grammaire{denom} \paradigme{dir}{tɤ-}
\begin{définition}\pfra{célébrer}\end{définition}
\begin{définition}\pcmn{庆祝}\end{définition}
\begin{exemple}\pjya{tɤ-rɯmpɕɯmɤr-i}\hspace{5pt}\pcmn{我们庆祝了}\end{exemple}\end{entrée}

\begin{entrée}{rɯmphrɯmɯ}{}{ⓔrɯmphrɯmɯ} 
\classe{vi}  
\grammaire{denom} \paradigme{dir}{pɯ-}\sens{1}
\begin{définition}\pfra{prédire l'avenir}\end{définition}
\begin{définition}\pcmn{算命}\end{définition}\sens{2}
\begin{définition}\pfra{se faire les griffes (chat)}\end{définition}
\begin{définition}\pcmn{抓(猫)}\end{définition}
\begin{exemple}\pjya{lɯlu ɲɯ-rɯmphrɯmɯ}\hspace{5pt}\pcmn{猫喜欢乱抓}\end{exemple}
\begin{sous-entrée}{zrɯmphrɯmɯ}{ⓔrɯmphrɯmɯⓢ2ⓝzrɯmphrɯmɯ} 
\classe{vt} 
\begin{définition}\pfra{faire regarder l'avenir}\end{définition}
\begin{définition}\pcmn{请别人算命}\end{définition}
\begin{exemple}\pjya{nɯ-sqar-a tɕe pɯ-zrɯmphrɯmɯ-t-a}\hspace{5pt}\pcmn{我请了他算命}\end{exemple}\relationsémantique{参考}{\lien{ⓔmphrɯmɯ}{mphrɯmɯ}}\end{sous-entrée}

\end{entrée}

\begin{entrée}{rɯmɯntoʁ}{}{ⓔrɯmɯntoʁ} 
\classe{vi}  
\grammaire{denom} \paradigme{dir}{nɯ-}
\begin{définition}\pfra{fleurir}\end{définition}
\begin{définition}\pcmn{开花}\end{définition}
\begin{exemple}\pjya{khɯjŋga ɲɤ-rɯmɯntoʁ}\hspace{5pt}\pcmn{杜鹃花开了}\end{exemple}
\begin{exemple}\pjya{pɤjka ɲɤ-rɯmɯntoʁ}\hspace{5pt}\pcmn{白瓜开花了}\end{exemple}\relationsémantique{参考}{\lien{ⓔmɯntoʁ}{mɯntoʁ}}\end{entrée}

\begin{entrée}{rɯndzaŋspa}{}{ⓔrɯndzaŋspa} 
\classe{vi}  
\grammaire{denom} \paradigme{dir}{tɤ-}
\begin{définition}\pfra{faire attention}\end{définition}
\begin{définition}\pcmn{小心}\end{définition}
\begin{exemple}\pjya{tɤ-rɯndzaŋspa ma kɯ-mɯrkɯ tɯ-ɕlɯɣ}\hspace{5pt}\pcmn{你小心一点,不然会被偷东西的}\end{exemple}\étymologie{mdzaŋs.pa}\end{entrée}

\begin{entrée}{rɯndzɤqhɤjɯ}{}{ⓔrɯndzɤqhɤjɯ} 
\classe{vi} \paradigme{dir}{tɤ-}\paradigme{dir}{tɤ-}
\begin{définition}\pfra{manger derrière le dos des autres}\end{définition}
\begin{définition}\pcmn{瞒着别人偷吃}\end{définition}
\begin{définition}\pfra{manger derrière le dos de ...}\end{définition}
\begin{définition}\pcmn{瞒着……偷吃}\end{définition}
\begin{exemple}\pjya{jiɕqha ɲɯ-rɯndzɤqhɤjɯ}\hspace{5pt}\pcmn{那个人瞒着别人偷吃东西}\end{exemple}
\begin{exemple}\pjya{tɤ-rɯndzɤqhɤjɯ-tɕi}\hspace{5pt}\pcmn{我们俩偷吃了东西}\end{exemple}
\begin{exemple}\pjya{ma-tɤ-tɯ-rɯndzɤqhɤjɯ}\hspace{5pt}\pcmn{你不要瞒着别人偷吃!}\end{exemple}
\begin{exemple}\pjya{tɤ-ta-nɯndzɤqhɤjɯ}\hspace{5pt}\pcmn{我瞒着你偷吃东西了}\end{exemple}\relationsémantique{参考}{\lien{ⓔndzɤqhɤjɯ}{ndzɤqhɤjɯ}}
\begin{sous-entrée}{nɯndzɤqhɤjɯ}{ⓔrɯndzɤqhɤjɯⓝnɯndzɤqhɤjɯ} 
\classe{vt} \end{sous-entrée}

\end{entrée}

\begin{entrée}{rɯndzɤtshi}{}{ⓔrɯndzɤtshi} 
\classe{vi}  
\grammaire{denom} \paradigme{dir}{tɤ-}\paradigme{dir}{thɯ-}\paradigme{dir}{tɤ-}
\begin{définition}\pfra{prendre un repas}\end{définition}
\begin{définition}\pcmn{吃一顿}\end{définition}
\begin{définition}\pfra{donner un repas}\end{définition}
\begin{définition}\pcmn{请别人吃一顿}\end{définition}
\begin{exemple}\pjya{thɯ-rɯndzɤtshi}\hspace{5pt}\pcmn{你吃饭吧!}\end{exemple}
\begin{exemple}\pjya{tu-rɯndzɤtshi ɯ-ɲɯ́-cha?}\hspace{5pt}\pcmn{他能不能吃饭?(说一个病人)}\end{exemple}\relationsémantique{参考}{\lien{ⓔndzɤtshiⓗ2}{ndzɤtshi₂}}
\begin{sous-entrée}{zrɯndzɤtshi}{ⓔrɯndzɤtshiⓝzrɯndzɤtshi} 
\classe{vt} \end{sous-entrée}

\end{entrée}

\begin{entrée}{rɯɲɟele}{}{ⓔrɯɲɟele} 
\classe{vi} \paradigme{dir}{thɯ-}
\begin{définition}\pfra{tendre les jambes}\end{définition}
\begin{définition}\pcmn{伸脚}\end{définition}
\begin{exemple}\pjya{thɯ-rɯɲɟele-a}\hspace{5pt}\pcmn{我伸了脚}\end{exemple}
\begin{exemple}\pjya{sɤrɲɟɤle}\end{exemple}\end{entrée}

\begin{entrée}{rɯŋgɤlwoʁ}{}{ⓔrɯŋgɤlwoʁ} 
\classe{vi}  
\grammaire{incorp} 
\begin{définition}\pfra{gaspiller, éparpiller}\end{définition}
\begin{définition}\pcmn{乱撒;浪费}\end{définition}
\begin{exemple}\pjya{ma-pɯ-tɯ-rɯŋgɤlwoʁ}\hspace{5pt}\pcmn{你不要乱撒}\end{exemple}\relationsémantique{同义词}{\lien{ⓔrɯtɕhɯχtɤr}{rɯtɕhɯχtɤr}}\relationsémantique{同义词}{\lien{ⓔlwoʁ}{lwoʁ}}\end{entrée}

\begin{entrée}{rɯŋgoŋpu}{}{ⓔrɯŋgoŋpu} 
\classe{vi} \paradigme{dir}{tɤ-}
\begin{définition}\pfra{provoquer des désastres}\end{définition}
\begin{définition}\pcmn{惹祸;破坏东西}\end{définition}
\begin{exemple}\pjya{jisŋi tɤ-rɯŋgoŋpu-a}\hspace{5pt}\pcmn{我今天破坏了东西}\end{exemple}\end{entrée}

\begin{entrée}{rɯŋundʑu}{}{ⓔrɯŋundʑu} 
\classe{vi} \paradigme{dir}{tɤ-}
\begin{définition}\pfra{chercher à s'attirer les faveurs des gens}\end{définition}
\begin{définition}\pcmn{讨好;跟别人说好话}\end{définition}
\begin{exemple}\pjya{ɯ-phe tɤ-rɯŋundʑu-a}\hspace{5pt}\pcmn{我跟他说了好话}\end{exemple}\relationsémantique{参考}{\lien{ⓔnɯŋundʑu}{nɯŋundʑu}}\end{entrée}

\begin{entrée}{rɯŋɯŋɤn}{}{ⓔrɯŋɯŋɤn} 
\classe{vi} \paradigme{dir}{tɤ-}
\begin{définition}\pfra{causer des dégâts}\end{définition}
\begin{définition}\pcmn{搞破坏;捣乱}\end{définition}
\begin{exemple}\pjya{tɤ-rɟit kɯ-rɯŋɯŋɤn ci ɲɯ-ŋu}\hspace{5pt}\pcmn{他是一个(爱)搞破坏的孩子}\end{exemple}
\begin{exemple}\pjya{tɕhɯthɤn chɤ-ɣi tɕe to-rɯŋɯŋɤn}\hspace{5pt}\pcmn{洪水来了,搞了破坏}\end{exemple}\end{entrée}

\begin{entrée}{rɯphɯrɤm}{}{ⓔrɯphɯrɤm}\relationsémantique{参考}{\lien{ⓔnɯphɯrɤm}{nɯphɯrɤm}}\end{entrée}

\begin{entrée}{rɯphɯrlaʁ}{}{ⓔrɯphɯrlaʁ} 
\classe{vi} \paradigme{dir}{thɯ-}
\begin{définition}\pfra{ruiner}\end{définition}
\begin{définition}\pcmn{倾家荡产;破坏}\end{définition}
\begin{exemple}\pjya{nɯ-kha thamtɕɤt cho-phɯt, nɯ-laχtɕha chɤ-sɤrɕo tɕe chɤ-rɯphɯrlaʁ}\hspace{5pt}\pcmn{他拆了房子,浪费了他们的东西,把财产花光了}\end{exemple}\étymologie{ⁿpʰro.brlag}\end{entrée}

\begin{entrée}{rɯpjɤβlaʁ}{}{ⓔrɯpjɤβlaʁ} 
\classe{vs} \paradigme{dir}{tɤ-}
\begin{définition}\pfra{rusé}\end{définition}
\begin{définition}\pcmn{狡猾}\end{définition}
\begin{exemple}\pjya{jiɕqha nɯ ɲɯ-rɯpjɤβlaʁ}\hspace{5pt}\pcmn{那个人很狡猾}\end{exemple}\end{entrée}

\begin{entrée}{rɯpjɤŋkhɤr}{}{ⓔrɯpjɤŋkhɤr} 
\classe{vi} \paradigme{dir}{tɤ-}
\begin{définition}\pfra{tourner dans le ciel (oiseau)}\end{définition}
\begin{définition}\pcmn{盘旋}\end{définition}
\begin{exemple}\pjya{qaliaʁ ɲɯ-rɯpjɤŋkhɤr}\hspace{5pt}\pcmn{老鹰在盘旋}\end{exemple}\étymologie{ⁿkʰor}\end{entrée}

\begin{entrée}{rɯqajɯ}{}{ⓔrɯqajɯ} 
\classe{vi}  
\grammaire{denom} \paradigme{dir}{nɯ-}
\begin{définition}\pfra{avoir des vers}\end{définition}
\begin{définition}\pcmn{生蛆}\end{définition}
\begin{exemple}\pjya{tɤ-mthɯm ɲɤ-rɯqajɯ}\hspace{5pt}\pcmn{肉生蛆了}\end{exemple}\relationsémantique{参考}{\lien{ⓔqajɯ}{qajɯ}}\relationsémantique{参考}{\lien{ⓔnɯqajɯ}{nɯqajɯ}}\end{entrée}

\begin{entrée}{rɯqartsɤβ}{}{ⓔrɯqartsɤβ} 
\classe{vi}  
\grammaire{denom}
\grammaire{denom} \paradigme{dir}{kɤ-}
\begin{définition}\pfra{récolter}\end{définition}
\begin{définition}\pcmn{收割}\end{définition}
\begin{exemple}\pjya{kɤ-rɯqartsɤβ-i}\hspace{5pt}\pcmn{我们收割了}\end{exemple}
\begin{exemple}\pjya{ɲɯ-rɯqartsɤβ-nɯ}\hspace{5pt}\pcmn{他们在收割}\end{exemple}\relationsémantique{参考}{\lien{ⓔqartsɤβ}{qartsɤβ}}\end{entrée}

\begin{entrée}{rɯqhaχɕu}{}{ⓔrɯqhaχɕu} 
\classe{vi} \paradigme{dir}{tɤ-}\paradigme{dir}{tɤ-}
\begin{définition}\pfra{se vanter}\end{définition}
\begin{définition}\pcmn{炫耀}\end{définition}
\begin{définition}\pfra{se vanter de}\end{définition}
\begin{définition}\pcmn{炫耀(某种东西)}\end{définition}
\begin{exemple}\pjya{jiɕqha nɯ ɲɯ-rɯqhaχɕu}\hspace{5pt}\pcmn{那个人在夸耀自己}\end{exemple}
\begin{exemple}\pjya{laχtɕha to-nɯχtɯ tɕe ɲɯ-rɯqhaχɕu}\hspace{5pt}\pcmn{他买了东西就炫耀}\end{exemple}
\begin{exemple}\pjya{ma-tɯ-rɯqhaχɕu ntsɯ}\hspace{5pt}\pcmn{你不要总是夸耀自己}\end{exemple}
\begin{exemple}\pjya{``nɤʑo tɯ-rɯqhaχɕu ntsɯ sɤznɤ, nɤ-rɟɯ nɯra mɤ-tɯ-rɯre kɯ" toti}\hspace{5pt}\pcmn{他对算命先生说:“比起夸耀自己(很会算命),你还不如看好你的财产”(算命先生被贼偷了东西的故事)}\end{exemple}
\begin{exemple}\pjya{ɯ-ŋga ci to-nɯ-χtɯ, nɯ tu-nɯqhaχɕe ɲɯ-ŋu}\hspace{5pt}\pcmn{他买了一件衣服,现在一直在炫耀}\end{exemple}\relationsémantique{参考}{\lien{ⓔqhaχɕu}{qhaχɕu}}\relationsémantique{参考}{\lien{ⓔχɕu}{χɕu}}\relationsémantique{参考}{\lien{ⓔznaχɕɯχɕu}{znaχɕɯχɕu}}
\begin{sous-entrée}{nɯqhaχɕu}{ⓔrɯqhaχɕuⓝnɯqhaχɕu} 
\classe{vt} \end{sous-entrée}

\end{entrée}

\begin{entrée}{rɯru}{}{ⓔrɯru} 
\classe{vt} \paradigme{dir}{nɯ-}
\begin{définition}\pfra{garder, surveiller}\end{définition}
\begin{définition}\pcmn{守卫;看守}\end{définition}
\begin{exemple}\pjya{fsapaʁ nɯ-rɯre}\hspace{5pt}\pcmn{我看一下牲畜吧}\end{exemple}
\begin{exemple}\pjya{smi nɯ-rɯre}\hspace{5pt}\pcmn{你看火吧}\end{exemple}\relationsémantique{参考}{\lien{ⓔruⓗ1}{ru₁}}\relationsémantique{同义词}{\lien{ⓔnɤmdzɯ}{nɤmdzɯ}}\end{entrée}

\begin{entrée}{rɯra}{}{ⓔrɯra} 
\classe{vi} \paradigme{dir}{\_}
\begin{définition}\pfra{aller voir}\end{définition}
\begin{définition}\pcmn{探望}\end{définition}
\begin{exemple}\pjya{ɯʑo ɲɯ-ngo tɕe z-jɤ-rɯra-a (=z-jɤ-rtoʁ-a)}\hspace{5pt}\pcmn{他生病,我就去看他了}\end{exemple}\end{entrée}

\begin{entrée}{rɯrawa}{}{ⓔrɯrawa} 
\classe{vi} \paradigme{dir}{nɯ-}
\begin{définition}\pfra{exiger des autres}\end{définition}
\begin{définition}\pcmn{要求别人为自己付出}\end{définition}
\begin{exemple}\pjya{ɯʑo a-ɕki rŋɯl kɤ-mbi ɲɯ-nɯrawa}\hspace{5pt}\pcmn{他要求我给他钱}\end{exemple}
\begin{exemple}\pjya{ɯʑo a-ɕki tu-kɤ-qur ntsɯ ɲɯ-nɯrawa ŋu}\hspace{5pt}\pcmn{他总是要求我帮他}\end{exemple}\étymologie{re.ba}\end{entrée}

\begin{entrée}{rɯrɤt}{}{ⓔrɯrɤt} 
\classe{vt} \paradigme{dir}{tɤ-}
\begin{définition}\pfra{décider des soutras à réciter pour quelqu'un}\end{définition}
\begin{définition}\pcmn{喇嘛规定(给别人)念经}\end{définition}
\begin{exemple}\pjya{nɯ-rpi kɯ-ɴqɯ-ɴqa ʑo to-rɯrɤt-nɯ}\hspace{5pt}\pcmn{(喇嘛们)规定给他们念隆重的佛经}\end{exemple}\relationsémantique{同义词}{\lien{ⓔkhrɤtⓗ2}{khrɤt₂}}\end{entrée}

\begin{entrée}{rɯrcaŋpɕaʁ}{}{ⓔrɯrcaŋpɕaʁ} 
\classe{vi}  
\grammaire{denom} 
\begin{définition}\pfra{se prosterner jusqu'à un lieu saint tout le long de la route}\end{définition}
\begin{définition}\pcmn{磕长头(到观音桥)}\end{définition}\relationsémantique{参考}{\lien{ⓔrcaŋpɕaʁ}{rcaŋpɕaʁ}}\étymologie{brkʲaŋs.pʰʲag}\end{entrée}

\begin{entrée}{rɯrdɤβzu}{}{ⓔrɯrdɤβzu} 
\classe{vi} 
\begin{définition}\pfra{faire de la maçonnerie}\end{définition}
\begin{définition}\pcmn{做石工}\end{définition}
\begin{exemple}\pjya{ʑara ɣɯ ku-rɯrdɤβzu-a}\hspace{5pt}\pcmn{我在给他们做石工}\end{exemple}\relationsémantique{参考}{\lien{ⓔrdɤβzu}{rdɤβzu}}\end{entrée}

\begin{entrée}{rɯrgɤm}{}{ⓔrɯrgɤm} 
\classe{n} 
\begin{définition}\pfra{cercueil}\end{définition}
\begin{définition}\pcmn{棺材}\end{définition}\étymologie{rus.sgam?}\end{entrée}

\begin{entrée}{rɯri}{}{ⓔrɯri}\relationsémantique{参考}{\lien{ⓔraŋri}{raŋri}}\end{entrée}

\begin{entrée}{rɯrɟa}{}{ⓔrɯrɟa} 
\classe{vt}  
\grammaire{denom} \paradigme{dir}{pɯ-}
\begin{définition}\pfra{maudire}\end{définition}
\begin{définition}\pcmn{咒骂,诅咒}\end{définition}
\begin{exemple}\pjya{nɤʑo kɯ pɯ-kɯ-rɯrɟa-a}\hspace{5pt}\pcmn{你咒骂我了}\end{exemple}\relationsémantique{参考}{\lien{ⓔɯ-rɟa}{ɯ-rɟa}}\end{entrée}

\begin{entrée}{rɯrɟaŋrɟɤz}{}{ⓔrɯrɟaŋrɟɤz} 
\classe{vi} 
\begin{définition}\pfra{perdre du temps}\end{définition}
\begin{définition}\pcmn{拖延时间}\end{définition}
\begin{exemple}\pjya{ma-tɯ-rɯrɟaŋrɟɤz kɯ tɤ-mbɣom ma kɤ-nɤma kɤ-sthɯt mɤ-tsu}\hspace{5pt}\pcmn{别拖延时间,抓紧时间,不然的话这件事情做不完}\end{exemple}\end{entrée}

\begin{entrée}{rɯrɟɯfsoʁ}{}{ⓔrɯrɟɯfsoʁ}\relationsémantique{参考}{\lien{ⓔɣɯrɟɯfsoʁ}{ɣɯrɟɯfsoʁ}}\end{entrée}

\begin{entrée}{rɯrtsi}{}{ⓔrɯrtsi} 
\classe{n} \sens{1}
\begin{définition}\pfra{montagne}\end{définition}
\begin{définition}\pcmn{高山}\end{définition}\sens{2}
\begin{définition}\pfra{dieu de la montagne}\end{définition}
\begin{définition}\pcmn{山神}\end{définition}\étymologie{ri.rtse}\end{entrée}

\begin{entrée}{rɯrtsɯpɣaʁ}{}{ⓔrɯrtsɯpɣaʁ} 
\classe{vi}  
\grammaire{denom} \paradigme{dir}{lɤ-}
\begin{définition}\pfra{retourner la terre après la récolte}\end{définition}
\begin{définition}\pcmn{庄稼收割了以后重新翻地}\end{définition}\relationsémantique{参考}{\lien{ⓔrtsɯpɣaʁ}{rtsɯpɣaʁ}}\relationsémantique{参考}{\lien{ⓔnɯrtsɯpɣaʁ}{nɯrtsɯpɣaʁ}}\end{entrée}

\begin{entrée}{rɯrtsɯtʂɯɣ}{}{ⓔrɯrtsɯtʂɯɣ} 
\classe{vi} \paradigme{dir}{nɯ-}\paradigme{dir}{thɯ-}
\begin{définition}\pfra{faire les comptes}\end{définition}
\begin{définition}\pcmn{算帐}\end{définition}
\begin{exemple}\pjya{thɯ-rɯrtsɯtʂɯɣ-a / rtsɯtʂɯɣ thɯ-ta-t-a}\hspace{5pt}\pcmn{我算账了}\end{exemple}\étymologie{rtsi.sgrig}\end{entrée}

\begin{entrée}{rɯʁdɯʁdɯɣ}{}{ⓔrɯʁdɯʁdɯɣ} 
\classe{vs} \paradigme{dir}{tɤ-}\paradigme{case}{ɣɯ}
\begin{définition}\pfra{gêner}\end{définition}
\begin{définition}\pcmn{妨碍别人做事}\end{définition}
\begin{exemple}\pjya{tɯrme ɣɯ ɲɯ-rɯʁdɯʁdɯɣ}\hspace{5pt}\pcmn{他妨碍别人的工作}\end{exemple}
\begin{exemple}\pjya{aʑɯɣ ɲɯ-rɯʁdɯʁdɯɣ}\hspace{5pt}\pcmn{他妨碍我的工作}\end{exemple}\end{entrée}

\begin{entrée}{rɯʁdɯxpa}{}{ⓔrɯʁdɯxpa} 
\classe{vs}  
\grammaire{denom} \paradigme{dir}{tɤ-}
\begin{définition}\pfra{empêcher}\end{définition}
\begin{définition}\pcmn{妨碍}\end{définition}
\begin{exemple}\pjya{ɲɯ-rɯʁdɯxpa}\hspace{5pt}\pcmn{他在妨碍人}\end{exemple}
\begin{exemple}\pjya{phɤnba kɤ-βzu mɤ-kɯ-cha ci pɯ-ŋu kɯnɤ, ma-tɤ-kɯ-rɯʁdɯxpa ra}\hspace{5pt}\pcmn{没有能力帮别人的话,也不可以妨碍别人}\end{exemple}\relationsémantique{参考}{\lien{ⓔrɯʁdɯʁdɯɣ}{rɯʁdɯʁdɯɣ}}\relationsémantique{参考}{\lien{ⓔʁdɯxpa}{ʁdɯxpa}}\end{entrée}

\begin{entrée}{rɯʁgiwa}{}{ⓔrɯʁgiwa} 
\classe{vi} \paradigme{dir}{tɤ-}
\begin{définition}\pfra{faire lire des soutras pour les morts}\end{définition}
\begin{définition}\pcmn{请人念经}\end{définition}\relationsémantique{参考}{\lien{ⓔʁgiwa}{ʁgiwa}}\end{entrée}

\begin{entrée}{rɯʁlɤwɯr}{}{ⓔrɯʁlɤwɯr} 
\classe{vi} \paradigme{dir}{tɤ-}
\begin{définition}\pfra{soudain}\end{définition}
\begin{définition}\pcmn{突然}\end{définition}
\begin{exemple}\pjya{@wenchuan waɟɯ to-rɯʁlɤwɯr ɕti}\hspace{5pt}\pcmn{汶川大地震发生得很突然}\end{exemple}\étymologie{glo.bur}\end{entrée}

\begin{entrée}{rɯscɯscit}{}{ⓔrɯscɯscit} 
\classe{vs} 
\begin{définition}\pfra{oisif}\end{définition}
\begin{définition}\pcmn{清闲;安逸}\end{définition}
\begin{exemple}\pjya{ɯ-tɯ-rɯscɯscit nɯ!}\hspace{5pt}\pcmn{他真清闲}\end{exemple}\end{entrée}

\begin{entrée}{rɯskɤrwa}{}{ⓔrɯskɤrwa} 
\classe{vi}  
\grammaire{denom} \paradigme{dir}{kɤ-}
\begin{définition}\pfra{faire tourner les moulins à prière}\end{définition}
\begin{définition}\pcmn{转经}\end{définition}
\begin{exemple}\pjya{kɯ-rɯskɤrwa jɤ-ari-a}\hspace{5pt}\pcmn{我去转经了}\end{exemple}
\begin{exemple}\pjya{ɕ-kɤ-rɯskɤrwa-a}\hspace{5pt}\pcmn{我去转经了}\end{exemple}
\begin{exemple}\pjya{aʑo χsɯ-tɤxɯr kɤ-rɯskɤrwa-a}\hspace{5pt}\pcmn{我转经转了三周}\end{exemple}\relationsémantique{参考}{\lien{ⓔskɤrwa}{skɤrwa}}\end{entrée}

\begin{entrée}{rɯsɲaŋne}{}{ⓔrɯsɲaŋne} 
\classe{vi} \paradigme{dir}{pɯ-}
\begin{définition}\pfra{jeûner}\end{définition}
\begin{définition}\pcmn{念哑巴经(禁食斋) 我念了哑巴经}\end{définition}\relationsémantique{参考}{\lien{ⓔsɲaŋne}{sɲaŋne}}\relationsémantique{参考}{\lien{ⓔnɯsɲaŋne}{nɯsɲaŋne}}\end{entrée}

\begin{entrée}{rɯspa}{}{ⓔrɯspa} 
\classe{n} 
\begin{définition}\pfra{génie}\end{définition}
\begin{définition}\pcmn{天才}\end{définition}\étymologie{rigs.pa}\end{entrée}

\begin{entrée}{rɯstɯnmɯ}{}{ⓔrɯstɯnmɯ} 
\classe{vi}  
\grammaire{caus} \paradigme{dir}{tɤ-}
\begin{définition}\pfra{se marier}\end{définition}
\begin{définition}\pcmn{结婚}\end{définition}\paradigme{dir}{tɤ-}
\begin{exemple}\pjya{ji-me tɤ-rɯstɯnmɯ}\hspace{5pt}\pcmn{我们的女儿结了婚}\end{exemple}
\begin{exemple}\pjya{ji-tɕɯ tɤ-rɯstɯnmɯ}\hspace{5pt}\pcmn{我们的儿子结了婚}\end{exemple}
\begin{exemple}\pjya{kɯ-rɯstɯnmɯ ɣɤʑu}\hspace{5pt}\pcmn{有人在结婚}\end{exemple}
\begin{exemple}\pjya{ɕ-tɤ-rɯstɯnmɯ-a}\hspace{5pt}\pcmn{我去结婚了}\end{exemple}
\begin{exemple}\pjya{kɯ-rɯstɯnmɯ tɤrca ju-ɕe-a ŋu}\hspace{5pt}\pcmn{我去参加婚礼}\end{exemple}
\begin{sous-entrée}{zrɯstɯnmɯ}{ⓔrɯstɯnmɯⓝzrɯstɯnmɯ} 
\classe{vt} \end{sous-entrée}

\begin{définition}\pfra{marier, faire se marier}\end{définition}
\begin{définition}\pcmn{使……结婚}\end{définition}\étymologie{ston.mo}\end{entrée}

\begin{entrée}{rɯsɯso}{}{ⓔrɯsɯso} 
\classe{vi}  
\grammaire{denom} \sens{1}\paradigme{dir}{thɯ-}
\begin{définition}\pfra{réfléchir}\end{définition}
\begin{définition}\pcmn{想}\end{définition}\sens{2}
\begin{définition}\pfra{se souvenir}\end{définition}
\begin{définition}\pcmn{回忆}\end{définition}
\begin{exemple}\pjya{ɲɯ-rɯsɯso}\hspace{5pt}\pcmn{他在想}\end{exemple}\sens{3}\paradigme{dir}{pɯ-}
\begin{définition}\pfra{comparer}\end{définition}
\begin{définition}\pcmn{比较}\end{définition}
\begin{exemple}\pjya{nɤʑo cho pjɯ-kɯ-rɯsɯso tɕe, aʑo kɯ a-laz ɲɯ-sna}\hspace{5pt}\pcmn{跟你比较的话,我的命要好一些}\end{exemple}\relationsémantique{参考}{\lien{ⓔsɯso}{sɯso}}\end{entrée}

\begin{entrée}{rɯtɤmtɯ}{}{ⓔrɯtɤmtɯ} 
\classe{vt}  
\grammaire{denom} \paradigme{dir}{thɯ-}
\begin{définition}\pfra{faire un nœud}\end{définition}
\begin{définition}\pcmn{打结}\end{définition}
\begin{exemple}\pjya{tɤ-ri ɯ-ndo thɯ-rɯtɤmtɯ-t-a}\hspace{5pt}\pcmn{我在线的一头打了个结}\end{exemple}\relationsémantique{参考}{\lien{ⓔtɤ-mtɯ}{tɤ-mtɯ}}\end{entrée}

\begin{entrée}{rɯtɕɤmɯ}{}{ⓔrɯtɕɤmɯ} 
\classe{vi}  
\grammaire{denom} \paradigme{dir}{lɤ-}
\begin{définition}\pfra{devenir none}\end{définition}
\begin{définition}\pcmn{当尼姑}\end{définition}\relationsémantique{参考}{\lien{ⓔtɕɤmɯ}{tɕɤmɯ}}\relationsémantique{参考}{\lien{ⓔnɯtɕɤmɯ}{nɯtɕɤmɯ}}\end{entrée}

\begin{entrée}{rɯtɕhɤβ}{}{ⓔrɯtɕhɤβ} 
\classe{n} 
\begin{définition}\pfra{endroit où il n'y a que des rochers et pas d'herbe}\end{définition}
\begin{définition}\pcmn{高山上只有岩石没有草的地方}\end{définition}\end{entrée}

\begin{entrée}{rɯtɕhɤfɕɤt}{}{ⓔrɯtɕhɤfɕɤt} 
\classe{vi} 
\begin{définition}\pfra{participer à un débat philosophique}\end{définition}
\begin{définition}\pcmn{辩经}\end{définition}
\begin{exemple}\pjya{χpɯn ra ɲɯ-rɯtɕhɤfɕɤt-nɯ}\hspace{5pt}\pcmn{和尚们在辩经}\end{exemple}\relationsémantique{参考}{\lien{ⓔtɕhɤfɕɤt}{tɕhɤfɕɤt}}\end{entrée}

\begin{entrée}{rɯtɕhɯtɕhi}{}{ⓔrɯtɕhɯtɕhi} 
\classe{vi} 
\begin{définition}\pfra{chicaner}\end{définition}
\begin{définition}\pcmn{讲究,计较}\end{définition}
\begin{exemple}\pjya{nɤ-kɤ-qha ɯ-tɯ-dɤn nɯ, a-mɤ-tɯ-rɯtɕhɯtɕhi}\hspace{5pt}\pcmn{你不喜欢的东西太多了,不要这么计较}\end{exemple}\end{entrée}

\begin{entrée}{rɯtɕhɯχtɤr}{}{ⓔrɯtɕhɯχtɤr} 
\classe{vi} \paradigme{dir}{thɯ-}
\begin{définition}\pfra{éparpiller, gaspiller}\end{définition}
\begin{définition}\pcmn{乱撒,浪费}\end{définition}
\begin{exemple}\pjya{ma-pɯ-tɯ-rɯtɕhɯχtɤr}\hspace{5pt}\pcmn{你不要浪费}\end{exemple}\relationsémantique{参考}{\lien{ⓔrɯŋgɤlwoʁ}{rɯŋgɤlwoʁ}}\relationsémantique{参考}{\lien{ⓔphɤtɕhɯχtɤr}{phɤtɕhɯχtɤr}}\end{entrée}

\begin{entrée}{rɯtɕi}{}{ⓔrɯtɕi} 
\classe{cnj} 
\begin{définition}\pfra{mais}\end{définition}
\begin{définition}\pcmn{但是}\end{définition}\end{entrée}

\begin{entrée}{rɯtshoŋpa}{}{ⓔrɯtshoŋpa} 
\classe{vi} \paradigme{dir}{pɯ-}
\begin{définition}\pfra{faire du commerce}\end{définition}
\begin{définition}\pcmn{做生意}\end{définition}
\begin{exemple}\pjya{pjɤ-rɯtshoŋpa}\hspace{5pt}\pcmn{他以前做生意(现在不做了)}\end{exemple}\étymologie{tsʰoŋ.pa}\end{entrée}

\begin{entrée}{rɯtʂa}{}{ⓔrɯtʂa} 
\classe{n} 
\begin{définition}\pfra{envie}\end{définition}
\begin{définition}\pcmn{妒忌}\end{définition}
\begin{exemple}\pjya{rɯtʂa ʁo kɯ-pe ci maʁ}\hspace{5pt}\pcmn{妒忌是不好的}\end{exemple}\relationsémantique{参考}{\lien{ⓔnɯrɯtʂa}{nɯrɯtʂa}}\end{entrée}

\begin{entrée}{rɯtɯsqa}{}{ⓔrɯtɯsqa} 
\classe{vi}  
\grammaire{denom} \paradigme{dir}{lɤ-}\paradigme{dir}{kɤ-}
\begin{définition}\pfra{manger du gruau de blé}\end{définition}
\begin{définition}\pcmn{吃麦子粥}\end{définition}
\begin{exemple}\pjya{lɤ-rɯtɯsqa-j}\hspace{5pt}\pcmn{我们吃了麦子粥}\end{exemple}
\begin{exemple}\pjya{ʑara kɤ-rɯtɯsqa rga-nɯ}\hspace{5pt}\pcmn{他们喜欢吃麦子粥}\end{exemple}\relationsémantique{参考}{\lien{ⓔtɯsqa}{tɯsqa}}\end{entrée}

\begin{entrée}{rɯtɯwɯ}{}{ⓔrɯtɯwɯ} 
\classe{vs} \paradigme{dir}{tɤ-}
\begin{définition}\pfra{être sur le point de pousser des épis (de l'orge)}\end{définition}
\begin{définition}\pcmn{快要抽穗(青稞)}\end{définition}
\begin{exemple}\pjya{tɤɕi to-rɯtɯwɯ}\hspace{5pt}\pcmn{青稞快要抽穗}\end{exemple}\end{entrée}

\begin{entrée}{rɯxpa}{}{ⓔrɯxpa} 
\classe{n} 
\begin{définition}\pfra{mémoire}\end{définition}
\begin{définition}\pcmn{记性}\end{définition}\étymologie{rig.pa}\end{entrée}

\begin{entrée}{rɯxtuxti}{}{ⓔrɯxtuxti} 
\classe{vt} \paradigme{dir}{tɤ-}
\begin{définition}\pfra{respecter}\end{définition}
\begin{définition}\pcmn{尊重;抬高;奉承}\end{définition}
\begin{exemple}\pjya{tu-ta-rɯxtuxti ŋu nɤ!}\hspace{5pt}\pcmn{我尊重你}\end{exemple}
\begin{sous-entrée}{zrɯxtuxti}{ⓔrɯxtuxtiⓝzrɯxtuxti}
\begin{exemple}\pjya{tu-zrɯxtuxti-a zgɤt}\hspace{5pt}\pcmn{我应该尊重他}\end{exemple}\end{sous-entrée}

\begin{sous-entrée}{ʑɣɤrɯxtuxti}{ⓔrɯxtuxtiⓝʑɣɤrɯxtuxti} 
\classe{vi}  
\grammaire{refl} 
\begin{définition}\pfra{être vaniteux}\end{définition}
\begin{définition}\pcmn{自大}\end{définition}\relationsémantique{参考}{\lien{ⓔwxti}{wxti}}\end{sous-entrée}

\end{entrée}

\begin{entrée}{rɯχamba}{}{ⓔrɯχamba} 
\classe{vs} 
\begin{définition}\pfra{être présomptueux}\end{définition}
\begin{définition}\pcmn{骄傲;自大}\end{définition}
\begin{sous-entrée}{znɯχamba}{ⓔrɯχambaⓝznɯχamba} 
\classe{vt} 
\begin{définition}\pfra{agir de façon présomptueuse}\end{définition}
\begin{définition}\pcmn{骄傲;自大}\end{définition}
\begin{exemple}\pjya{nɤ-kɤ-znɯχamba me}\hspace{5pt}\pcmn{你没有什么理由骄傲}\end{exemple}\end{sous-entrée}

\end{entrée}

\begin{entrée}{rɯχɕɯχɕɤβ}{}{ⓔrɯχɕɯχɕɤβ} 
\classe{vi} \paradigme{dir}{tɤ-}
\begin{définition}\pfra{parler de façon exagérée}\end{définition}
\begin{définition}\pcmn{(讲话)夸张}\end{définition}
\begin{exemple}\pjya{nɤʑo ʁo tɯ-rɯχɕɯχɕɤβ ntsɯ ɕti nɤ}\hspace{5pt}\pcmn{你讲话总是很夸张}\end{exemple}\relationsémantique{参考}{\lien{ⓔχɕɤβ}{χɕɤβ}}\end{entrée}

\begin{entrée}{rɯχparɤβ}{}{ⓔrɯχparɤβ} 
\classe{vt} \paradigme{dir}{tɤ-}
\begin{définition}\pfra{se vanter}\end{définition}
\begin{définition}\pcmn{卖弄,夸耀自己}\end{définition}\relationsémantique{同义词}{\lien{ⓔznɤchacha}{znɤchacha}}\end{entrée}

\begin{entrée}{rɯχtɕɯrɯ}{}{ⓔrɯχtɕɯrɯ} 
\classe{vs} 
\begin{définition}\pfra{tout nu}\end{définition}
\begin{définition}\pcmn{裸体;光着身子}\end{définition}\relationsémantique{参考}{\lien{ⓔχtɕɯrɯpa}{χtɕɯrɯpa}}\étymologie{gtɕer.bu}\end{entrée}

\begin{entrée}{rɯχtsɯχtso}{}{ⓔrɯχtsɯχtso} 
\classe{vi} 
\begin{définition}\pfra{aimer la propreté}\end{définition}
\begin{définition}\pcmn{爱干净(有洁癖)}\end{définition}
\begin{exemple}\pjya{ɲɯ-tɯ-rɯχtsɯχtso}\hspace{5pt}\pcmn{你爱干净}\end{exemple}
\begin{exemple}\pjya{mɯ́j-tɯ-rɯχtsɯχtso}\hspace{5pt}\pcmn{你不爱干净}\end{exemple}
\begin{exemple}\pjya{ɲɯ-tɯ-rɯχtsɯχtso tɕe, aʑo a-@beibei mɯ́j-tɯ-ntɕhoz (tɯrme ɯ-@beibei mɯ́j-tɯ-ntɕhoz)}\hspace{5pt}\pcmn{你爱干净,所以不愿意用我的杯子(你不用别人的杯子)}\end{exemple}\relationsémantique{参考}{\lien{ⓔχtso}{χtso}}\end{entrée}

\begin{entrée}{rɯz}{}{ⓔrɯz} 
\classe{vs} \paradigme{dir}{pɯ-}
\begin{définition}\pfra{vrai}\end{définition}
\begin{définition}\pcmn{真实}\end{définition}
\begin{exemple}\pjya{tɯ-mɯ ɲɯ-ɤsɯ-lɤt ɲɯ-ti-nɯ, ɕ-thɯ-nɤrɯra-a ri, ɲɯ-rɯz}\hspace{5pt}\pcmn{我听说外面在下雨,我去看了一下是真的}\end{exemple}\étymologie{rigs}\end{entrée}

\begin{entrée}{rɯzdɯzdɯɣ}{}{ⓔrɯzdɯzdɯɣ} 
\classe{vi} \paradigme{dir}{tɤ-}
\begin{définition}\pfra{raconter ses malheurs}\end{définition}
\begin{définition}\pcmn{诉苦}\end{définition}
\begin{exemple}\pjya{ɯʑo a-phe ɲɯ-rɯzdɯzdɯɣ}\hspace{5pt}\pcmn{他向我诉苦}\end{exemple}
\begin{exemple}\pjya{ɯʑo pjɤ-nɤɴqa tɕe pɯ-rɯzdɯzdɯɣ ntsɯ}\hspace{5pt}\pcmn{因为觉得辛苦,他不停地诉苦}\end{exemple}\end{entrée}

\begin{entrée}{rwa}{}{ⓔrwa} 
\classe{n} 
\begin{définition}\pfra{tente de nomade en poil de yak}\end{définition}
\begin{définition}\pcmn{牧民的帐篷(毛制成的)}\end{définition}\étymologie{sbra.ba}\end{entrée}

\begin{entrée}{rwɤt}{}{ⓔrwɤt} 
\classe{vt} \paradigme{dir}{pɯ-}\paradigme{dir}{lɤ-}
\begin{définition}\pfra{creuser}\end{définition}
\begin{définition}\pcmn{挖}\end{définition}
\begin{exemple}\pjya{ŋgɤm lɤ-rwat-a}\hspace{5pt}\pcmn{我挖了土墙}\end{exemple}
\begin{sous-entrée}{sɯrwɤt}{ⓔrwɤtⓝsɯrwɤt}
\begin{exemple}\pjya{qaʁ kɯ sɤtɕha pɯ-sɯrwat-a}\hspace{5pt}\pcmn{我用锄头挖了地}\end{exemple}\relationsémantique{同义词}{\lien{ⓔlɣa}{lɣa}}\end{sous-entrée}

\end{entrée}

\begin{entrée}{rwoʁrwoʁ}{}{ⓔrwoʁrwoʁ} 
\classe{idph.2} 
\begin{définition}\pfra{plein de petites boules ou de petits morceaux de même taille}\end{définition}
\begin{définition}\pcmn{很多小球或大小一致的小块}\end{définition}
\begin{exemple}\pjya{staχpɯ rwoʁrwoʁ ʑo ɲɯ-pa}\hspace{5pt}\pcmn{豌豆是圆的}\end{exemple}
\begin{exemple}\pjya{tɤ-pɤtso kɯxtɕɯxtɕi rwoʁrwoʁ ɲɯ-ɤʑɯrja-nɯ}\hspace{5pt}\pcmn{小孩子在排队}\end{exemple}\relationsémantique{参考}{\lien{ⓔrloʁrloʁ}{rloʁrloʁ}}\relationsémantique{参考}{\lien{ⓔrloŋrloŋ}{rloŋrloŋ}}\relationsémantique{参考}{\lien{ⓔrlaŋrlaŋ}{rlaŋrlaŋ}}\end{entrée}

\begin{entrée}{rzɤβrzɤβ}{}{ⓔrzɤβrzɤβ} 
\classe{idph.2} 
\begin{définition}\pfra{flou}\end{définition}
\begin{définition}\pcmn{模模糊糊,不清楚}\end{définition}
\begin{exemple}\pjya{χɕɤlmɯɣ a-pɯ-me tɕe, tɤscoz ra rzɤβrzɤβ ʑo ɲɯ-pa tɕe mɯ́j-sɯχsal-a}\hspace{5pt}\pcmn{没有眼镜的话,文字模模糊糊,我根本看不清楚}\end{exemple}
\begin{exemple}\pjya{zdɯm lɤ-kɯ-ɣe ʑo rzɤβrzɤβ ɲɯ-fse}\hspace{5pt}\pcmn{好像起了雾一样,模模糊糊地看不清}\end{exemple}
\begin{sous-entrée}{ɣɤrzɤβrzɤβ}{ⓔrzɤβrzɤβⓝɣɤrzɤβrzɤβ}
\begin{définition}\pfra{être flou}\end{définition}
\begin{définition}\pcmn{模模糊糊,不清楚}\end{définition}
\begin{exemple}\pjya{a-mɲaʁndo ɲɯ-ɣɤrzɤβrzɤβ tɕe mɯ́j-sɯχsal-a}\hspace{5pt}\pcmn{我眼边很模糊,看不清楚}\end{exemple}\end{sous-entrée}

\end{entrée}

\begin{entrée}{rzoŋ}{}{ⓔrzoŋ} 
\classe{vt} \paradigme{dir}{thɯ-}
\begin{définition}\pfra{mettre dedans}\end{définition}
\begin{définition}\pcmn{往里装}\end{définition}
\begin{exemple}\pjya{thɯ-rzoŋ-a}\hspace{5pt}\pcmn{我装了}\end{exemple}
\begin{exemple}\pjya{tha-rzoŋ}\hspace{5pt}\pcmn{他装了}\end{exemple}
\begin{exemple}\pjya{ɕɤmɯɣdɯ thɯ-rzoŋ}\hspace{5pt}\pcmn{你给枪装子弹吧}\end{exemple}\étymologie{rdzoŋ}\end{entrée}

\begin{entrée}{rzoŋlu}{}{ⓔrzoŋlu} 
\classe{idph} 
\begin{définition}\pfra{très occupé}\end{définition}
\begin{définition}\pcmn{忙得不可开交}\end{définition}
\begin{exemple}\pjya{a-ma ɯ-tɯ-dɤn kɯ rzoŋlu ʑo ɲɯ-xtsu}\hspace{5pt}\pcmn{我事情很多,忙得不可开交}\end{exemple}\end{entrée}

\begin{entrée}{rzoŋwa}{}{ⓔrzoŋwa} 
\classe{n} 
\begin{définition}\pfra{dot}\end{définition}
\begin{définition}\pcmn{嫁妆}\end{définition}\end{entrée}

\begin{entrée}{rzoʁ}{}{ⓔrzoʁ} 
\classe{vi} 
\begin{définition}\pfra{pousser (complète)}\end{définition}
\begin{définition}\pcmn{(完全)长出了}\end{définition}
\begin{exemple}\pjya{lɯtoʁ ɲɤ-rzoʁ}\hspace{5pt}\pcmn{(春天的时候),草木复苏}\end{exemple}
\begin{exemple}\pjya{a-ʁɲɤlwa nɯ-rzoʁ}\hspace{5pt}\pcmn{我受了很多苦}\end{exemple}
\begin{exemple}\pjya{tɤ-rɤku ra ɲɤ-rzoʁ}\hspace{5pt}\pcmn{庄稼全部生长起来了}\end{exemple}
\begin{exemple}\pjya{ta-ma ɲɯ-rzoʁ ʑo ɕti}\hspace{5pt}\pcmn{很多事情同一个时间堆在一起,无从下手}\end{exemple}\étymologie{rdzogs}\end{entrée}

\begin{entrée}{rzɯɴɢaʁ}{}{ⓔrzɯɴɢaʁ} 
\classe{n}  
\grammaire{n.lieu} 
\begin{définition}\pfra{Rdzong.'gag}\end{définition}
\begin{définition}\pcmn{松岗乡}\end{définition}\end{entrée}

\begin{entrée}{rzɯrzi/\variante{rdzɯrdzi}}{}{ⓔrzɯrzi} 
\classe{idph.2} \sens{1}
\begin{définition}\pfra{inquiet}\end{définition}
\begin{définition}\pcmn{形容担心,放心不下的状态}\end{définition}
\begin{exemple}\pjya{rdzɯrdzi ɲɯ-mu-a}\hspace{5pt}\pcmn{我害怕}\end{exemple}
\begin{exemple}\pjya{rzɯrzi ɲɯ-mu-a}\hspace{5pt}\pcmn{我害怕}\end{exemple}\sens{2}
\begin{définition}\pfra{frais (temps)}\end{définition}
\begin{définition}\pcmn{冷飕飕}\end{définition}
\begin{exemple}\pjya{jisŋi tɯ-mɯ rzɯrzi ci ɲɯ-ŋu, ɲɯ-ɣɤndʐo}\hspace{5pt}\pcmn{今天冷飕飕的}\end{exemple}\relationsémantique{参考}{\lien{ⓔrdzɯrdzi}{rdzɯrdzi}}\end{entrée}

\begin{entrée}{rʑaʁ}{}{ⓔrʑaʁ} 
\classe{vi} \paradigme{dir}{tɤ-}
\begin{définition}\pfra{se passer un certain nombre de jours}\end{définition}
\begin{définition}\pcmn{过几天}\end{définition}
\begin{exemple}\pjya{χsɯm to-rʑaʁ}\hspace{5pt}\pcmn{过了三天}\end{exemple}\relationsémantique{参考}{\lien{}{tɤ-rʑaʁ₁}}\end{entrée}

\begin{entrée}{rʑaʁtɕhɤt}{}{ⓔrʑaʁtɕhɤt} 
\classe{n} 
\begin{définition}\pfra{limite de temps}\end{définition}
\begin{définition}\pcmn{期限}\end{définition}\end{entrée}

\begin{entrée}{rʑi}{}{ⓔrʑi} 
\classe{vs} \paradigme{dir}{tɤ-}
\begin{définition}\pfra{lourd}\end{définition}
\begin{définition}\pcmn{重}\end{définition}
\begin{exemple}\pjya{ɯ-fkur ɲɯ-rʑi}\hspace{5pt}\pcmn{他背的东西很重}\end{exemple}
\begin{exemple}\pjya{rdɤstaʁ ɲɯ-rʑi}\hspace{5pt}\pcmn{石头很重}\end{exemple}\relationsémantique{反义词}{\lien{ⓔʑoⓗ1}{ʑo₁}}
\begin{sous-entrée}{ɣɤrʑi}{ⓔrʑiⓝɣɤrʑi} 
\classe{vt}  
\grammaire{caus} 
\begin{définition}\pfra{alourdir}\end{définition}
\begin{définition}\pcmn{加重}\end{définition}\end{sous-entrée}

\end{entrée}

\begin{entrée}{rʑɯɣrʑɯɣ}{}{ⓔrʑɯɣrʑɯɣ} 
\classe{idph.2} 
\begin{définition}\pfra{gros et lisse}\end{définition}
\begin{définition}\pcmn{形容横放在那里,又粗又光滑的样子}\end{définition}
\begin{exemple}\pjya{lɤpɯɣ ɯ-tɯ-wxti kɯ rʑɯɣrʑɯɣ ʑo ɲɯ-pa}\hspace{5pt}\pcmn{萝卜放在那里又粗又光滑}\end{exemple}
\begin{sous-entrée}{rʑɯɣnɤlɯɣ}{ⓔrʑɯɣrʑɯɣⓝrʑɯɣnɤlɯɣ}
\begin{définition}\pfra{qui rampe en se tortillant}\end{définition}
\begin{définition}\pcmn{形容蜿蜒爬行的样子}\end{définition}
\begin{exemple}\pjya{qapri nɯ rʑɯɣnɤlɯɣ ʑo jɤ-ari}\hspace{5pt}\pcmn{蛇蜿蜒爬行的样子}\end{exemple}\end{sous-entrée}

\begin{sous-entrée}{ɣɤrʑɯɣlɯɣ}{ⓔrʑɯɣrʑɯɣⓝɣɤrʑɯɣlɯɣ} 
\classe{vi} 
\begin{définition}\pfra{serpenter, ramper en se tortillant}\end{définition}
\begin{définition}\pcmn{蜿蜒爬行}\end{définition}
\begin{définition}\pfra{qapri ɲɯ-ɣɤrʑɯɣlɯɣ ʑo thɯ-ɣe}\end{définition}
\begin{définition}\pcmn{蛇爬下来了}\end{définition}\end{sous-entrée}

\end{entrée}

\newpage\caractère{ʁ}

\begin{entrée}{ʁarphɤβ}{}{ⓔʁarphɤβ} 
\classe{n} 
\begin{définition}\pfra{battement d'ailes}\end{définition}
\begin{définition}\pcmn{拍翅膀}\end{définition}
\begin{exemple}\pjya{qaliaʁ kɯ ʁarphɤβ pjɤ-lɤt}\hspace{5pt}\pcmn{老鹰拍了翅膀}\end{exemple}\relationsémantique{参考}{\lien{ⓔtɤ-ʁar}{tɤ-ʁar}}\relationsémantique{参考}{\lien{ⓔnɤʁarphɤβ}{nɤʁarphɤβ}}\relationsémantique{参考}{\lien{ⓔɣɤrphɤrphɤβ}{ɣɤrphɤrphɤβ}}\end{entrée}

\begin{entrée}{ʁaʁ}{}{ⓔʁaʁ} 
\classe{vi} \paradigme{dir}{nɯ-}\sens{1}
\begin{définition}\pfra{éclore}\end{définition}
\begin{définition}\pcmn{孵出来}\end{définition}
\begin{exemple}\pjya{tɤ-ŋgɯm ɲɤ-ʁaʁ}\hspace{5pt}\pcmn{蛋孵出来了}\end{exemple}\sens{2}\paradigme{dir}{nɯ-}
\begin{définition}\pfra{s'épanouir (fleur)}\end{définition}
\begin{définition}\pcmn{开花}\end{définition}
\begin{définition}\pfra{faire éclore}\end{définition}
\begin{définition}\pcmn{使孵出来;使开花}\end{définition}
\begin{exemple}\pjya{mɯntoʁ ɲo-ʁaʁ}\hspace{5pt}\pcmn{花开了}\end{exemple}
\begin{exemple}\pjya{kumpɣa ɯ-ŋgɯm nɯ-sɯɣʁaʁ-a}\hspace{5pt}\pcmn{我令鸡蛋孵化(用人工的手段)}\end{exemple}
\begin{sous-entrée}{sɯɣʁaʁ/\variante{sɯʁaʁ}}{ⓔʁaʁⓢ2ⓝsɯɣʁaʁ} 
\classe{vt} \end{sous-entrée}

\end{entrée}

\begin{entrée}{ʁatɯl}{}{ⓔʁatɯl} 
\classe{n} 
\begin{définition}\pfra{habit en peau de renard}\end{définition}
\begin{définition}\pcmn{狐狸皮皮袄}\end{définition}\étymologie{wa.dol}\end{entrée}

\begin{entrée}{ʁaz}{}{ⓔʁaz}\relationsémantique{参考}{\lien{ⓔʁaznɤ}{ʁaznɤ}}\end{entrée}

\begin{entrée}{ʁaznɤ}{}{ⓔʁaznɤ} 
\classe{adv} 
\begin{définition}\pfra{profiter de l'occasion}\end{définition}
\begin{définition}\pcmn{趁……的机会}\end{définition}
\begin{exemple}\pjya{tɕiʑo ʁna ɣɤʑu-tɕi ʁaznɤ nɯkrɤz-tɕi}\hspace{5pt}\pcmn{要趁我们俩都在的时候商量}\end{exemple}
\begin{exemple}\pjya{nɤ-tɤ-lu ɲɯ-sɤɕke ʁaznɤ kɤ-tshi, tɕe a-mɤ-nɯmɯɕtaʁ}\hspace{5pt}\pcmn{牛奶要趁热喝,不要放冷了}\end{exemple}
\begin{exemple}\pjya{sɤɕke ʁaz tɤ-ndze}\hspace{5pt}\pcmn{趁热吃吧}\end{exemple}\end{entrée}

\begin{entrée}{ʁbɤβʁbɤβ}{}{ⓔʁbɤβʁbɤβ} 
\classe{idph.2} 
\begin{définition}\pfra{épais et gros}\end{définition}
\begin{définition}\pcmn{形容厚而大的样子}\end{définition}
\begin{exemple}\pjya{jiɕqha tɯrme nɯ ɯ-rŋa ra ɲɯ-tshu ʑo ʁbɤʁbɤβ}\hspace{5pt}\pcmn{这个人的脸又粗又胖}\end{exemple}\end{entrée}

\begin{entrée}{ʁdɤnba}{}{ⓔʁdɤnba} 
\classe{n} 
\begin{définition}\pfra{capital}\end{définition}
\begin{définition}\pcmn{本钱}\end{définition}
\begin{exemple}\pjya{kɯki nɤ-tɯtsɣe ɯ-ʁdɤnba a-pɯ-ŋu}\hspace{5pt}\pcmn{把这个用来做你做生意的本钱}\end{exemple}\end{entrée}

\begin{entrée}{ʁdɤʁdɤt}{}{ⓔʁdɤʁdɤt} 
\classe{idph.2} 
\begin{définition}\pfra{rectangulaire, solide}\end{définition}
\begin{définition}\pcmn{正方形、结实的样子}\end{définition}
\begin{exemple}\pjya{rgɤm ʁdɤrdɤt ʑo ɲɯ-pa}\hspace{5pt}\pcmn{盒子又小又是正方形的,很结实的样子}\end{exemple}\relationsémantique{同义词}{\lien{ⓔɣdɤɣdɤt}{ɣdɤɣdɤt}}\end{entrée}

\begin{entrée}{ʁdɯβʁdɯβ}{}{ⓔʁdɯβʁdɯβ} 
\classe{idph.2} 
\begin{définition}\pfra{rectangulaire}\end{définition}
\begin{définition}\pcmn{形容四四方方的形状(手上拿得起的东西)}\end{définition}
\begin{exemple}\pjya{@larou ʁdɯβʁdɯβ ci nɯ́-wɣ-mbi-a}\hspace{5pt}\pcmn{他把腊肉给我了}\end{exemple}
\begin{exemple}\pjya{tɤ-ŋgɤr ci ʁdɯβʁdɯβ pjɤ-phɯt}\hspace{5pt}\pcmn{他割下一大块(四四方方的)猪膘}\end{exemple}
\begin{exemple}\pjya{ɯ-phoŋbu ʁdɯβʁdɯβ ʑo ɲɯ-pa}\hspace{5pt}\pcmn{他身子矮胖}\end{exemple}\end{entrée}

\begin{entrée}{ʁdɯɣ}{₂}{ⓔʁdɯɣⓗ2} 
\classe{n} 
\begin{définition}\pfra{parasol (réservé aux sprul sku)}\end{définition}
\begin{définition}\pcmn{华盖(活佛专用)}\end{définition}\étymologie{gdug}\end{entrée}

\begin{entrée}{ʁdɯɣ}{₁}{ⓔʁdɯɣⓗ1} 
\classe{vi} \paradigme{dir}{tɤ-}
\begin{définition}\pfra{grave}\end{définition}
\begin{définition}\pcmn{有害}\end{définition}
\begin{exemple}\pjya{nɯnɯ kɤ-ʁdɯɣ me}\hspace{5pt}\pcmn{没有关系}\end{exemple}
\begin{exemple}\pjya{tɕhomba mɯ́j-ʁdɯɣ}\hspace{5pt}\pcmn{感冒不严重}\end{exemple}
\begin{exemple}\pjya{nɯ mɤ-ʁdɯɣ}\hspace{5pt}\pcmn{那个没有关系}\end{exemple}
\begin{exemple}\pjya{smɤn a-tɤ-ndze tɕe mɯ́j-ʁdɯɣ}\hspace{5pt}\pcmn{吃了药就没有关系}\end{exemple}\end{entrée}

\begin{entrée}{ʁdɯn}{}{ⓔʁdɯn} 
\classe{n} 
\begin{définition}\pfra{mal}\end{définition}
\begin{définition}\pcmn{邪气}\end{définition}\étymologie{gdon}\end{entrée}

\begin{entrée}{ʁdɯrɟɤt}{}{ⓔʁdɯrɟɤt} 
\classe{n}  
\grammaire{n.lieu} 
\begin{définition}\pfra{Gdongbrgyad}\end{définition}
\begin{définition}\pcmn{龙尔甲乡}\end{définition}\end{entrée}

\begin{entrée}{ʁdɯrtoʁ/\variante{ʁdɯrto}}{}{ⓔʁdɯrtoʁ} 
\classe{n} 
\begin{définition}\pfra{espace entre les poutres du plafond et les poutres du toit}\end{définition}
\begin{définition}\pcmn{横梁和房背之间的平板}\end{définition}
\begin{exemple}\pjya{ʁdɯrtoʁ nɯ komɤl ɯ-taʁ stukɤr ɯ-pa tɤrɤm kɯ-jaʁ ku-kɯ-ɕe nɯ ŋu}\hspace{5pt}\pcmn{\lien{ⓔʁdɯrtoʁ}{ʁdɯrtoʁ} 是在横梁之上,大梁之下横过去的厚木板}\end{exemple}\end{entrée}

\begin{entrée}{ʁdɯrtsa}{}{ⓔʁdɯrtsa} 
\classe{n} 
\begin{définition}\pfra{amadou}\end{définition}
\begin{définition}\pcmn{火绒(把小麻皮捶绒)}\end{définition}\end{entrée}

\begin{entrée}{ʁdɯskɤr}{}{ⓔʁdɯskɤr} 
\classe{n} 
\begin{définition}\pfra{drapeau à prière}\end{définition}
\begin{définition}\pcmn{玛尼旗}\end{définition}\relationsémantique{同义词}{\lien{}{tartɕin}}\end{entrée}

\begin{entrée}{ʁdɯxpa}{}{ⓔʁdɯxpa} 
\classe{n} 
\begin{définition}\pfra{empêchement, gêne}\end{définition}
\begin{définition}\pcmn{障碍;坏处}\end{définition}
\begin{exemple}\pjya{kɤ-rɤma tɤ-ŋu tɕe, tɯʑo kɯ-rɯsɯso mɤ-βdi ma, tɯ-zda ra nɯ-ndaŋ kɯnɤ pjɯ́-wɣ-lɤt ra ma, nɯ-ʁdɯxpa ɣɯ-βzu mɤ-βdi}\hspace{5pt}\pcmn{工作的时候,不要只按照自己的想法来,要考虑到别人,不要妨碍别人}\end{exemple}
\begin{exemple}\pjya{a-ʁdɯxpa ta-βzu}\hspace{5pt}\pcmn{他妨碍我了}\end{exemple}\relationsémantique{参考}{\lien{ⓔrɯʁdɯxpa}{rɯʁdɯxpa}}\relationsémantique{反义词}{\lien{ⓔphɤnba}{phɤnba}}\end{entrée}

\begin{entrée}{ʁe}{}{ⓔʁe} 
\classe{n} 
\begin{définition}\pfra{gauche}\end{définition}
\begin{définition}\pcmn{左边}\end{définition}\end{entrée}

\begin{entrée}{ʁejlu}{}{ⓔʁejlu} 
\classe{n} 
\begin{définition}\pfra{gauche}\end{définition}
\begin{définition}\pcmn{左边}\end{définition}
\begin{exemple}\pjya{ʁejlu tɤ-lat-a, tɤ-ntɕhoz-a}\hspace{5pt}\pcmn{我用了左手}\end{exemple}\relationsémantique{参考}{\lien{ⓔsɯʁejlu}{sɯʁejlu}}\end{entrée}

\begin{entrée}{ʁejlɤɕkɤr}{}{ⓔʁejlɤɕkɤr} 
\classe{n} 
\begin{définition}\pfra{gaucher}\end{définition}
\begin{définition}\pcmn{左撇子(贬义)}\end{définition}\end{entrée}

\begin{entrée}{ʁgɤskɯ}{}{ⓔʁgɤskɯ} 
\classe{n} 
\begin{définition}\pfra{surveillant dans le monastère}\end{définition}
\begin{définition}\pcmn{负责惩罚违律行为的和尚}\end{définition}\étymologie{*dge.sku}\end{entrée}

\begin{entrée}{ʁgɤsloŋ}{}{ⓔʁgɤsloŋ} 
\classe{n} 
\begin{définition}\pfra{bhiksu}\end{définition}
\begin{définition}\pcmn{比丘}\end{définition}\étymologie{dge.sloŋ}\end{entrée}

\begin{entrée}{ʁgiwa}{}{ⓔʁgiwa} 
\classe{n} 
\begin{définition}\pfra{récitation de mantras}\end{définition}
\begin{définition}\pcmn{念经}\end{définition}
\begin{exemple}\pjya{ʁgiwa tɤ-βzu-j}\hspace{5pt}\pcmn{我们请人念经了}\end{exemple}\relationsémantique{参考}{\lien{ⓔrɯʁgiwa}{rɯʁgiwa}}\étymologie{dge.ba}\end{entrée}

\begin{entrée}{ʁgra}{}{ⓔʁgra} 
\classe{n} 
\begin{définition}\pfra{ennemi}\end{définition}
\begin{définition}\pcmn{敌人}\end{définition}
\begin{exemple}\pjya{tɯ-ʁgra nɯ tɯ-mɲaʁrme ma a-pɯ-me ra, tɯ-βzaŋsa nɯ tɯ-kɤrme jamar a-pɯ-dɤn ra}\hspace{5pt}\pcmn{敌人要少得像眉毛上的毛,朋友要多得像头上的头发}\end{exemple}\relationsémantique{反义词}{\lien{ⓔtɯ-ɣɲi}{tɯ-ɣɲi}}\relationsémantique{参考}{\lien{ⓔʁgraja}{ʁgraja}}\end{entrée}

\begin{entrée}{ʁgraja}{}{ⓔʁgraja} 
\classe{n} 
\begin{définition}\pfra{ennemi}\end{définition}
\begin{définition}\pcmn{仇人}\end{définition}\étymologie{dgra.ja}\end{entrée}

\begin{entrée}{ʁgusloŋ}{}{ⓔʁgusloŋ} 
\classe{n} 
\begin{définition}\pfra{type de moine}\end{définition}
\begin{définition}\pcmn{和尚的一种}\end{définition}\end{entrée}

\begin{entrée}{ʁju}{₁}{ⓔʁjuⓗ1} 
\classe{n} 
\begin{définition}\pfra{saucisson au bœuf}\end{définition}
\begin{définition}\pcmn{牛肉香肠}\end{définition}\end{entrée}

\begin{entrée}{ʁju}{₂}{ⓔʁjuⓗ2} 
\classe{n} 
\begin{définition}\pfra{turquoise}\end{définition}
\begin{définition}\pcmn{碧玉;绿松石}\end{définition}\relationsémantique{同义词}{\lien{ⓔmti}{mti}}\étymologie{gju}\end{entrée}

\begin{entrée}{ʁja}{}{ⓔʁja} 
\classe{n} 
\begin{définition}\pfra{vert-de-gris}\end{définition}
\begin{définition}\pcmn{铜锈}\end{définition}\étymologie{gja}\end{entrée}

\begin{entrée}{ʁjaŋ}{}{ⓔʁjaŋ} 
\classe{n} 
\begin{définition}\pfra{bon présage, bonne fortune}\end{définition}
\begin{définition}\pcmn{吉祥}\end{définition}
\begin{exemple}\pjya{kha ɯ-ŋgɯ ʁjaŋ a-pɯ-tu ra}\hspace{5pt}\pcmn{一个家庭需要吉祥}\end{exemple}\étymologie{gjaŋ}\end{entrée}

\begin{entrée}{ʁjaŋsɯ}{}{ⓔʁjaŋsɯ} 
\classe{n} 
\begin{définition}\pfra{feutre}\end{définition}
\begin{définition}\pcmn{毡子}\end{définition}
\begin{exemple}\pjya{ʁjaŋsɯ kɤ-sɯta ra}\hspace{5pt}\pcmn{要擀毡子}\end{exemple}\end{entrée}

\begin{entrée}{ʁjaŋtʂoŋ}{}{ⓔʁjaŋtʂoŋ} 
\classe{n} 
\begin{définition}\pfra{svastika}\end{définition}
\begin{définition}\pcmn{卍字}\end{définition}\étymologie{gjung.druŋ}\end{entrée}

\begin{entrée}{ʁjɤr}{}{ⓔʁjɤr} 
\classe{vs} \paradigme{dir}{tɤ-}
\begin{définition}\pfra{pousser dru}\end{définition}
\begin{définition}\pcmn{茂盛}\end{définition}
\begin{exemple}\pjya{ɯ-muj ɲɯ-ʁjɤr}\hspace{5pt}\pcmn{它的羽毛长得很密}\end{exemple}\end{entrée}

\begin{entrée}{ʁjɤrsa}{}{ⓔʁjɤrsa} 
\classe{n} 
\begin{définition}\pfra{pâturage d'été}\end{définition}
\begin{définition}\pcmn{夏天的牧场}\end{définition}\étymologie{dbʲar.sa}\end{entrée}

\begin{entrée}{ʁjɤʁjɤβ}{}{ⓔʁjɤʁjɤβ} 
\classe{idph.2} 
\begin{définition}\pfra{peu foncée (couleur)}\end{définition}
\begin{définition}\pcmn{形容颜色不深}\end{définition}\relationsémantique{参考}{\lien{ⓔʁjɯʁji}{ʁjɯʁji}}\end{entrée}

\begin{entrée}{ʁjit}{}{ⓔʁjit} 
\classe{vt} \paradigme{dir}{tɤ-}
\begin{définition}\pfra{se souvenir, manquer à}\end{définition}
\begin{définition}\pcmn{想起;想念}\end{définition}
\begin{exemple}\pjya{jiɕqha nɯ kɯ tɤ́-wɣ-ʁjit-a}\hspace{5pt}\pcmn{他想起我了}\end{exemple}
\begin{exemple}\pjya{ɯ-rʑaβ ɲɯ-ʁjit}\hspace{5pt}\pcmn{他想起他的妻子}\end{exemple}
\begin{exemple}\pjya{ɯʑo pɯ-kɤ-βzu ra tɤ-ʁjit-a}\hspace{5pt}\pcmn{我想起他所做的事情}\end{exemple}
\begin{exemple}\pjya{aʑo tɤ-ta-ʁjit}\hspace{5pt}\pcmn{我想起你了}\end{exemple}
\begin{sous-entrée}{ɯ-sɤʁjɯʁjit}{ⓔʁjitⓝɯ-sɤʁjɯʁjit}
\begin{définition}\pfra{rappeller}\end{définition}
\begin{définition}\pcmn{提醒}\end{définition}
\begin{exemple}\pjya{a-sɤʁjɯʁjit tu-tɯ-βze ɲɯ-ŋu}\hspace{5pt}\pcmn{你在提醒我}\end{exemple}\end{sous-entrée}

\end{entrée}

\begin{entrée}{ʁjitpa}{}{ⓔʁjitpa} 
\classe{n} 
\begin{définition}\pfra{assurance}\end{définition}
\begin{définition}\pcmn{心里清楚}\end{définition}
\begin{exemple}\pjya{tɯ-ti mɤ-ra ma a-ʁjitpa tu}\hspace{5pt}\pcmn{不用你说,我心里很清楚}\end{exemple}
\begin{exemple}\pjya{kɯki sɤtɕha ki nɤ-ʁjitpa ɯ́-tu?}\hspace{5pt}\pcmn{你熟悉这个地方吗}\end{exemple}\end{entrée}

\begin{entrée}{ʁjoʁ}{}{ⓔʁjoʁ} 
\classe{n} 
\begin{définition}\pfra{serviteur}\end{définition}
\begin{définition}\pcmn{仆人}\end{définition}
\begin{sous-entrée}{tɤ-ʁjoʁ}{ⓔʁjoʁⓝtɤ-ʁjoʁ} 
\classe{np} 
\begin{définition}\pfra{serviteur de}\end{définition}
\begin{définition}\pcmn{仆人}\end{définition}\end{sous-entrée}

\étymologie{gjog}\end{entrée}

\begin{entrée}{ʁjɯβkɯ}{}{ⓔʁjɯβkɯ} 
\classe{adv} 
\begin{définition}\pfra{qui ressemble un peu}\end{définition}
\begin{définition}\pcmn{仿佛像}\end{définition}
\begin{exemple}\pjya{ʁjɯβkɯ ɯ-mu tsa ɲɯ-fse}\hspace{5pt}\pcmn{他有点像他母亲}\end{exemple}\end{entrée}

\begin{entrée}{ʁjɯβtshɤt}{}{ⓔʁjɯβtshɤt} 
\classe{n} 
\begin{définition}\pfra{estimation}\end{définition}
\begin{définition}\pcmn{估计}\end{définition}
\begin{exemple}\pjya{fso tɯjpu fsusqi-tɯrpa wɣɯ́-kho ɲɯ-ra ri, ʁjɯβtshɤt ci tɤ-rku-t-a}\hspace{5pt}\pcmn{明天要交三十斤粮食,我大概估计了一下就装了}\end{exemple}
\begin{exemple}\pjya{ɯ-tɯ-zri @liangmi ɲɯ-ra tɕe ʁjɯβtshɤt ci pɯ-ʁndzar-a}\hspace{5pt}\pcmn{需要两米长(的木料),我大概估计了一下就锯了}\end{exemple}\relationsémantique{参考}{\lien{ⓔnɯʁjɯβtshɤt}{nɯʁjɯβtshɤt}}\end{entrée}

\begin{entrée}{ʁjɯmbrɯɣ}{}{ⓔʁjɯmbrɯɣ} 
\classe{n} 
\begin{définition}\pfra{dragon}\end{définition}
\begin{définition}\pcmn{龙}\end{définition}\étymologie{ⁿbrug}\end{entrée}

\begin{entrée}{ʁjɯmbrɯɣma}{}{ⓔʁjɯmbrɯɣma} 
\classe{n} 
\begin{définition}\pfra{bol}\end{définition}
\begin{définition}\pcmn{瓷碗,画着龙形状的图案}\end{définition}\étymologie{ⁿbrug.ma}\end{entrée}

\begin{entrée}{ʁjɯʁji}{}{ⓔʁjɯʁji} 
\classe{idph.2} \sens{1}
\begin{définition}\pfra{peu foncé}\end{définition}
\begin{définition}\pcmn{形容颜色不深}\end{définition}\sens{2}
\begin{définition}\pfra{se sentant un peu mal}\end{définition}
\begin{définition}\pcmn{形容有点不舒服的感觉}\end{définition}
\begin{exemple}\pjya{a-ʑi ci ʁjɯʁji ɲɯ-loʁ}\hspace{5pt}\pcmn{我有点想吐}\end{exemple}\sens{3}
\begin{définition}\pfra{air de chien battu}\end{définition}
\begin{définition}\pcmn{形容受委屈的样子}\end{définition}
\begin{exemple}\pjya{khɯna kɯ ɯ-jme pjɤ-ɕɯɴqoʁ tɕe ʁjɯʁji ʑo to-ʑɣɤstu}\hspace{5pt}\pcmn{狗夹起尾巴做出了一副很委屈的样子}\end{exemple}
\begin{exemple}\pjya{ɯ-wa kɯ tó-wɣ-nɤmqe tɕe ʁjɯʁji ʑo to-ʑɣɤstu}\hspace{5pt}\pcmn{他被父亲骂了以后做出了一副受委屈的样子}\end{exemple}\relationsémantique{参考}{\lien{ⓔʁjɤʁjɤβ}{ʁjɤʁjɤβ}}\end{entrée}

\begin{entrée}{ʁɟa}{}{ⓔʁɟa} 
\classe{adv} \sens{1}
\begin{définition}\pfra{seulement}\end{définition}
\begin{définition}\pcmn{光是}\end{définition}
\begin{exemple}\pjya{jima ʁɟa ʑo lu-ji-nɯ}\hspace{5pt}\pcmn{他们种的都是玉米}\end{exemple}\sens{2}
\begin{définition}\pfra{tout le temps}\end{définition}
\begin{définition}\pcmn{一直;总是}\end{définition}
\begin{exemple}\pjya{@kaihui kɯ-fse tɤ-ra tɕe, kɤ-ŋke ʁɟa ʑo ju-kɯ-ɕe pɯ-ra}\hspace{5pt}\pcmn{(以前交通不方便),要开会的时候,一直是走路去的}\end{exemple}\relationsémantique{参考}{\lien{ⓔaʁɟa}{aʁɟa}}\étymologie{gja.ma}\end{entrée}

\begin{entrée}{ʁɟo}{}{ⓔʁɟo} 
\classe{vt} \paradigme{dir}{nɯ-}
\begin{définition}\pfra{rincer}\end{définition}
\begin{définition}\pcmn{冲水}\end{définition}
\begin{exemple}\pjya{khɯtsa nɯ-ʁɟɤm}\hspace{5pt}\pcmn{你把碗冲一下}\end{exemple}
\begin{exemple}\pjya{khɯtsa na-ʁɟo}\hspace{5pt}\pcmn{他把碗冲了一下}\end{exemple}
\begin{exemple}\pjya{tɯ-ŋga na-ʁɟo}\hspace{5pt}\pcmn{他把衣服冲一下}\end{exemple}
\begin{exemple}\pjya{nɯ-ʁɟo-t-a}\hspace{5pt}\pcmn{我冲了}\end{exemple}
\begin{exemple}\pjya{tɯthɯ na-ʁɟo}\hspace{5pt}\pcmn{他把锅子冲了一下}\end{exemple}\relationsémantique{参考}{\lien{ⓔɯ-ʁɟoʁɟe}{ɯ-ʁɟoʁɟe}}\end{entrée}

\begin{entrée}{ʁɟɯβʁɟɯβ}{}{ⓔʁɟɯβʁɟɯβ} 
\classe{idph.2} 
\begin{définition}\pfra{gros}\end{définition}
\begin{définition}\pcmn{形容人全身都胖的样子}\end{définition}
\begin{exemple}\pjya{ɯ-tɯ-tshu kɯ ʁɟɯβʁɟɯβ ʑo ɲɯ-pa}\hspace{5pt}\pcmn{他全身都很胖}\end{exemple}\end{entrée}

\begin{entrée}{ʁɟɯʁɟri}{}{ⓔʁɟɯʁɟri} 
\classe{idph.2} \sens{1}
\begin{définition}\pfra{gras et mou}\end{définition}
\begin{définition}\pcmn{形容胖而软的样子}\end{définition}\sens{2}
\begin{définition}\pfra{humide}\end{définition}
\begin{définition}\pcmn{形容潮湿的样子}\end{définition}\relationsémantique{参考}{\lien{ⓔχcɯχcri}{χcɯχcri}}\end{entrée}

\begin{entrée}{ʁlaŋlu}{}{ⓔʁlaŋlu} 
\classe{n} 
\begin{définition}\pfra{année du bœuf}\end{définition}
\begin{définition}\pcmn{牛年}\end{définition}\étymologie{glaŋ.lo}\end{entrée}

\begin{entrée}{ʁlɤwɯr}{}{ⓔʁlɤwɯr} 
\classe{adv} 
\begin{définition}\pfra{soudain}\end{définition}
\begin{définition}\pcmn{突然}\end{définition}
\begin{exemple}\pjya{ʁlɤwɯr to-rɤŋgat}\hspace{5pt}\pcmn{他突然出发了}\end{exemple}
\begin{exemple}\pjya{ʁlɤwɯr to-ngo}\hspace{5pt}\pcmn{他突然生病了}\end{exemple}\relationsémantique{参考}{\lien{ⓔnɯʁlɤwɯr}{nɯʁlɤwɯr}}\étymologie{glo.bur}\end{entrée}

\begin{entrée}{ʁlɯβʁlɯβ}{}{ⓔʁlɯβʁlɯβ} 
\classe{idph.2} 
\begin{définition}\pfra{concave}\end{définition}
\begin{définition}\pcmn{凹进去}\end{définition}
\begin{exemple}\pjya{khɯsta kɯ-rnaʁ tsa ci ʁlɯβʁlɯβ ɲɯ-ŋu}\hspace{5pt}\pcmn{碗有点深,是凹进去的}\end{exemple}
\begin{exemple}\pjya{scoʁ ʁlɯβʁlɯβ kɯ-pa}\hspace{5pt}\pcmn{凹进去(很深)的水瓢}\end{exemple}
\begin{exemple}\pjya{praʁ ɯ-pa rɯdaʁ ci ku-rŋgɯ ɲɯ-ŋu ma ɯ-sta ʁlɯβʁlɯβ ɣɤʑu}\hspace{5pt}\pcmn{悬崖下有个动物的窝,地是凹进去的}\end{exemple}\relationsémantique{参考}{\lien{}{aχchowɤlu}}\relationsémantique{参考}{\lien{ⓔaqhoβlu}{aqhoβlu}}\end{entrée}

\begin{entrée}{ʁlɯm}{}{ⓔʁlɯm} 
\classe{n} 
\begin{définition}\pfra{orge qui vient juste de fermenter}\end{définition}
\begin{définition}\pcmn{刚刚发酵的青稞}\end{définition}\étymologie{glum}\end{entrée}

\begin{entrée}{ʁlɯmbɯɣ}{}{ⓔʁlɯmbɯɣ} 
\classe{n} 
\begin{définition}\pfra{estimation}\end{définition}
\begin{définition}\pcmn{估计}\end{définition}
\begin{exemple}\pjya{ʁlɯmbɯɣ ci tɤ-βzu-t-a}\hspace{5pt}\pcmn{我估计了一下}\end{exemple}\relationsémantique{同义词}{\lien{ⓔʁjɯβtshɤt}{ʁjɯβtshɤt}}\relationsémantique{参考}{\lien{ⓔnɯʁlɯmbɯɣ}{nɯʁlɯmbɯɣ}}\end{entrée}

\begin{entrée}{ʁlɯmci}{}{ⓔʁlɯmci} 
\classe{n} 
\begin{définition}\pfra{tchang qui vient juste de fermenter}\end{définition}
\begin{définition}\pcmn{刚刚发酵的青稞酒}\end{définition}\étymologie{glum}\end{entrée}

\begin{entrée}{ʁlɯmsɯsi}{}{ⓔʁlɯmsɯsi} 
\classe{n} 
\begin{définition}\pfra{Polygonatum sibiricum}\end{définition}
\begin{définition}\pcmn{黄精}\end{définition}
\begin{exemple}\pjya{ʁlɯmsɯsi nɯ sɯjno ci ŋu. tɯ-ji mŋu ndo cho si kɯ-xtɕi ɯ-ŋgɯ ra tu-ɬoʁ ŋu, ɯ-zrɤm nɯ tɕazga fse, ɯ-ru kɯ-ɣɯrni tsa ŋu, ɯ-jwaʁ kɯ-tɕɤr kɯ-ɤmtɕoʁ tsa ŋu, ndɯβ. ɯ-mat nɯ ɯ-ru tu-nɯɴqhe tɕe, tu-oʑɯrja ŋu. ɯ-ru tɯ-ldʑa ma me, ɯ-mat thɯ-tɯt tɕe, chɯ-ɣɯrni ŋu. tú-wɣ-ndza tɕe chi.}\hspace{5pt}\pcmn{黄精是一种植物。生长在地边和灌木林里,根部像姜一样,茎是红色的,叶子窄而尖,很小。果实是顺着茎往上长。只有一根茎,果实成熟后是红色的,可以吃,很甜。}\end{exemple}\end{entrée}

\begin{entrée}{ʁlɯn}{}{ⓔʁlɯn} 
\classe{vs} \paradigme{dir}{tɤ-}
\begin{définition}\pfra{insensible aux malheurs}\end{définition}
\begin{définition}\pcmn{承受得住磨难,不容易动摇}\end{définition}\relationsémantique{参考}{\lien{ⓔʁlɯnba}{ʁlɯnba}}\étymologie{glen.pa}\end{entrée}

\begin{entrée}{ʁlɯnba}{}{ⓔʁlɯnba} 
\classe{n} 
\begin{définition}\pfra{insensible aux malheurs}\end{définition}
\begin{définition}\pcmn{承受得住磨难,不容易动摇}\end{définition}
\begin{exemple}\pjya{nɤki tɯrme ki ʁlɯnba ci ɕti tɕe mɤ-naʁdɯɣ}\hspace{5pt}\pcmn{他是个承受得住磨难的人,他不会动摇的}\end{exemple}\relationsémantique{参考}{\lien{ⓔʁlɯn}{ʁlɯn}}\étymologie{glen.pa}\end{entrée}

\begin{entrée}{ʁma}{}{ⓔʁma} 
\classe{vs} \paradigme{dir}{pɯ-}
\begin{définition}\pfra{trop bas (coup de fusil)}\end{définition}
\begin{définition}\pcmn{打枪瞄低}\end{définition}
\begin{exemple}\pjya{ɕɤmɯɣdɯ ɯ-tɯ-ʁma nɯ}\hspace{5pt}\pcmn{那杆枪打得很低}\end{exemple}\relationsémantique{反义词}{\lien{ⓔmthu}{mthu}}\étymologie{dma}\end{entrée}

\begin{entrée}{ʁmaʁ}{}{ⓔʁmaʁ} 
\classe{n} 
\begin{définition}\pfra{armée}\end{définition}
\begin{définition}\pcmn{军队}\end{définition}
\begin{exemple}\pjya{pɣɤtɕɯ ʁmaʁ}\hspace{5pt}\pcmn{鸟群}\end{exemple}\end{entrée}

\begin{entrée}{ʁmaʁdɤr}{}{ⓔʁmaʁdɤr} 
\classe{n} 
\begin{définition}\pfra{drapeau}\end{définition}
\begin{définition}\pcmn{旗}\end{définition}\étymologie{dmag.dar}\end{entrée}

\begin{entrée}{ʁmaʁmi}{}{ⓔʁmaʁmi} 
\classe{n} 
\begin{définition}\pfra{militaire}\end{définition}
\begin{définition}\pcmn{士兵}\end{définition}
\begin{exemple}\pjya{ʁmaʁmi pɯ-kɯ-ɬoʁ}\hspace{5pt}\pcmn{退役的士兵}\end{exemple}\relationsémantique{参考}{\lien{ⓔnɯʁmaʁmi}{nɯʁmaʁmi}}\étymologie{dmag.mi}\end{entrée}

\begin{entrée}{ʁmazgrɯβ}{}{ⓔʁmazgrɯβ} 
\classe{n} 
\begin{définition}\pfra{cicatrice}\end{définition}
\begin{définition}\pcmn{伤疤}\end{définition}\end{entrée}

\begin{entrée}{ʁmɤrɲɯɣ}{}{ⓔʁmɤrɲɯɣ} 
\classe{n} 
\begin{définition}\pfra{moustique}\end{définition}
\begin{définition}\pcmn{蚊子}\end{définition}
\begin{exemple}\pjya{ʁmɤrɲɯɣ nɯ qajɯ ɯ-ʁar kɯ-tu ci ŋu, ɲɯ-nɯqambɯmbjom cha, ɯ-mtɕhi kɯ-ɤmtɕɯ-mtɕoʁ ci ŋu, ɯ-mtɕhi ɯ-taʁ taqaβ kɯ-fse kɯ-xtshɯ-xtshɯm, tɤ-rme jamar ʑo kɯ-xtshɯm tu, ɯ-mɤlɤjaʁ kɯtʂɤ-ldʑa tu, kɯ-rɲɟɯ-rɲɟi kɯ-xtshɯ-xtshɯm ʑo tu, pha ɯ-phoŋbu nɯ kɯ-ɲaʁ ŋu, wuma ʑo sɤmtsɯɣ tɕe aɣɯtɯɣ tɕe, tɯ-ndʐi kɤ-mtsɯɣ tɕe, kɤ-mtsɯɣ ɯ-rkɯ nɯ ɯ-kho kɯ-jom ʑo tu-z-nɯɣmbɤβ cha tɕe wuma ʑo rɤʑa tɕe pɯ́-wɣ-qraʁ kɯnɤ ɣɯ-rɤβraʁ ra.}\hspace{5pt}\pcmn{蚊子是带有翅膀的昆虫,会飞。嘴很尖,嘴有一种针,细得像毛一样。有六只脚,又细又长,全身都是黑色的。蜇人,有毒。蜇皮肤时,蜇过的地方周围会有较大面积的皮肤肿起来,很痒,而且抠破了还继续发痒。}\end{exemple}\end{entrée}

\begin{entrée}{ʁmɤrsɤr}{}{ⓔʁmɤrsɤr} 
\classe{n} 
\begin{définition}\pfra{doré}\end{définition}
\begin{définition}\pcmn{金黄色}\end{définition}\end{entrée}

\begin{entrée}{ʁmɤrsmɯɣ}{}{ⓔʁmɤrsmɯɣ} 
\classe{n} 
\begin{définition}\pfra{bordeau}\end{définition}
\begin{définition}\pcmn{紫色}\end{définition}\relationsémantique{参考}{\lien{ⓔarɯʁmɤrsmɯɣ}{arɯʁmɤrsmɯɣ}}\étymologie{dmar.smug}\end{entrée}

\begin{entrée}{ʁmɤrɯɣ}{}{ⓔʁmɤrɯɣ} 
\classe{n} 
\begin{définition}\pfra{casserole en cuivre}\end{définition}
\begin{définition}\pcmn{红铜锅,用来熬茶}\end{définition}\étymologie{dmar.rigs.?}\end{entrée}

\begin{entrée}{ʁmbɣi}{}{ⓔʁmbɣi} 
\classe{n} 
\begin{définition}\pfra{soleil}\end{définition}
\begin{définition}\pcmn{太阳}\end{définition}\relationsémantique{参考}{\lien{ⓔʁmbɣɯzɯn}{ʁmbɣɯzɯn}}\end{entrée}

\begin{entrée}{ʁmbɣɯzɯn}{}{ⓔʁmbɣɯzɯn} 
\classe{n} 
\begin{définition}\pfra{éclipse de soleil}\end{définition}
\begin{définition}\pcmn{日蚀}\end{définition}\relationsémantique{参考}{\lien{ⓔʁmbɣi}{ʁmbɣi}}\end{entrée}

\begin{entrée}{ʁmbroŋ/\variante{mbroŋ}}{}{ⓔʁmbroŋ} 
\classe{n} 
\begin{définition}\pfra{yak sauvage}\end{définition}
\begin{définition}\pcmn{野牦牛}\end{définition}\étymologie{ⁿbroŋ}\end{entrée}

\begin{entrée}{ʁmɯɣ}{}{ⓔʁmɯɣ} 
\classe{vt} \paradigme{dir}{tɤ-}\paradigme{dir}{tɤ-}
\begin{définition}\pfra{réfléchir à un plan}\end{définition}
\begin{définition}\pcmn{计划;计谋;怀疑}\end{définition}
\begin{définition}\pfra{décider par soi-même}\end{définition}
\begin{définition}\pcmn{自己决定}\end{définition}
\begin{exemple}\pjya{nɯ ɯ-mɤ-kɯ-ŋu ci ɲɯ-ʁmɯɣ}\hspace{5pt}\pcmn{他怀疑是不是}\end{exemple}
\begin{exemple}\pjya{kɯki a-pɯ-jɤɣ tɕe, tu-nɯna-a ku-ʁmɯɣ-a}\hspace{5pt}\pcmn{这件事结束的时候,我打算休息}\end{exemple}
\begin{exemple}\pjya{fsaqhe tɕe mbarkhom kɤ-ɕe ku-ʁmɯɣ-a}\end{exemple}
\begin{exemple}\pjya{mbarkhom ju-ɕe-a ku-ʁmɯɣ-a}\hspace{5pt}\pcmn{我打算明年去马尔康}\end{exemple}
\begin{exemple}\pjya{nɤ-qajɣi tu-kɤ-χtɯ tɤ-ʁmɯɣ-a ri, pɤjkhu mɯ-ko-smi}\hspace{5pt}\pcmn{我本来想你买馍馍(给你吃)但是没有熟}\end{exemple}
\begin{exemple}\pjya{jisŋi kɯ-nɯɕe ku-ʑɣɤʁmɯɣ-a (kɤ-nɯɕe ku-ʑɣɤʁmɯɣ-a)}\hspace{5pt}\pcmn{我决定今天回去}\end{exemple}
\begin{sous-entrée}{ʑɣɤʁmɯɣ}{ⓔʁmɯɣⓝʑɣɤʁmɯɣ} 
\classe{vt} \end{sous-entrée}

\end{entrée}

\begin{entrée}{ʁmɯrcɯ}{}{ⓔʁmɯrcɯ} 
\classe{n} 
\begin{définition}\pfra{grive (garrulax maximus)}\end{définition}
\begin{définition}\pcmn{大噪鹛【画眉鸟】}\end{définition}
\begin{exemple}\pjya{ʁmɯrcɯ nɯ khro mɤ-wxti, kha ɯ-rkɯ sɯŋgɯ ra ku-rɤʑi rga, ɯ-mdoʁ kɯ-pɣi ɯ-ŋgɯz kɯnɤ ldʑaŋkɯ kɯ-ɤrɤɕɯɕrɤz tu, ɯ-jme nɯ kɯ-ɲaʁ tɕe ɯ-ku kɯ-wɣrum tu, nɯ-nɯqambɯmbjom tɕe ɯ-ʁar ɯ-rca ɯ-jme kɯnɤ ɲɯ-sqhiar ŋu, tɤ-rɤku kɤ-ndza wuma rga, qajɯ nɯ ra kɤ-ndza rga tɕe qambalɯla ta-ndza tɕe ɯ-phoŋbu tu-ndze tɕe ɯ-ʁar nɯ ra pjɯ-βde ŋu. qambalɯla nɯ tɤtʂu ɯ-ɕki kɤ-ɣi rga tɕe ʁmɯrcɯ kɯ tɤtʂu ɯ-pa nɯ ra qambalɯla ɯ-ʁar pjɯ-prɤm ʑo ŋgrɤl.}\hspace{5pt}\pcmn{画眉鸟长得不大,喜欢住在房子周围的树林里,身体是灰色的,上面有灰色的纹路,尾巴是黑色,顶端有白点。翅膀展开飞翔时,尾巴也跟着展开。喜欢吃庄稼、虫子。当它吃蝴蝶的时候,只吃身子,不吃翅膀,把翅膀扔在地上。因为蝴蝶喜欢靠近灯光,画眉鸟就会在灯下的地上留下很多蝴蝶翅膀。}\end{exemple}\end{entrée}

\begin{entrée}{ʁmɯrɲɯɣ}{}{ⓔʁmɯrɲɯɣ} 
\classe{n} 
\begin{définition}\pfra{grive (pomatorhinus erythrocnemis)}\end{définition}
\begin{définition}\pcmn{斑胸钩嘴鹛}\end{définition}\end{entrée}

\begin{entrée}{ʁmɯrqaʁ}{}{ⓔʁmɯrqaʁ} 
\classe{n} 
\begin{définition}\pfra{grive (garrulax ocellatus)}\end{définition}
\begin{définition}\pcmn{眼纹噪鹛【呱呱鸡】}\end{définition}\end{entrée}

\begin{entrée}{ʁmɯrtsɯ}{}{ⓔʁmɯrtsɯ} 
\classe{n} 
\begin{définition}\pfra{espèce d'arbrisseau}\end{définition}
\begin{définition}\pcmn{灌木的一种}\end{définition}
\begin{exemple}\pjya{ʁmɯrtsɯ nɯ si kɯ-mbɤr ci ŋu, ɯʑo sti tu-ɬoʁ tɕe, kɯ-ndɯβ kɯ-dɤn tsa tɯtɯrca tu-ɬoʁ ŋu, ɯ-rkɯ si kɯ-mbro tsa a-pɯ-tu tɕe, ɯ-taʁ ku-rtɤβ nɤ ku-rtɤβ tɕe tu-ɕe nɤ tu-ɕe ɯ-kɤχcɤl mɤɕtʂa tu-ɕe ŋu. ɣɯjpa ɲɯ-rɯmɯntoʁ tɕe fsaqhe tɕe ɯ-mat ku-tshoʁ ŋu, ɯ-mat ku-tɯ-tshoʁ nɯ ɣɯrni thɯ-tɯt tɕe chɯ-ɲaʁ ŋu, kɤ-ndza sna tɕe chi. ɯ-jwaʁ kɯ-ɤrtɯm kɯ-zri tsa ŋu. ɯ-ru ɯ-mdoʁ nɯ kɯ-qarŋe tsa ŋu.}\hspace{5pt}\pcmn{\lien{ⓔʁmɯrtsɯ}{ʁmɯrtsɯ} 是比较矮的树,当它单独生长时,长得细而密的,一丛一丛的,当它靠近其他较高的树时,它就会缠着向上爬,一直爬到树梢。今年开花明年结果,果实刚结时是红色的,成熟时是黑色的,可以吃,是甜的,叶子是椭圆形的,树干是黄色的。}\end{exemple}\end{entrée}

\begin{entrée}{ʁnamʑa}{}{ⓔʁnamʑa} 
\classe{n} 
\begin{définition}\pfra{diadème}\end{définition}
\begin{définition}\pcmn{天冠}\end{définition}\étymologie{gnam.ʑwa}\end{entrée}

\begin{entrée}{ʁnaʁna}{}{ⓔʁnaʁna} 
\classe{n} 
\begin{définition}\pfra{les deux}\end{définition}
\begin{définition}\pcmn{两个都}\end{définition}
\begin{exemple}\pjya{tɕiʑo ʁnaʁna ʑo tɕi-ʑɯβ ɲɯ-ɣi}\hspace{5pt}\pcmn{我们俩都有瞌睡}\end{exemple}\end{entrée}

\begin{entrée}{ʁnɤmchi}{}{ⓔʁnɤmchi} 
\classe{n} 
\begin{définition}\pfra{grande ourse}\end{définition}
\begin{définition}\pcmn{北斗}\end{définition}\étymologie{gnam.kʰʲi}\end{entrée}

\begin{entrée}{ʁnɤmjaŋ}{}{ⓔʁnɤmjaŋ} 
\classe{n} 
\begin{définition}\pfra{plafond}\end{définition}
\begin{définition}\pcmn{天花板}\end{définition}\étymologie{gnam.jaŋs}\end{entrée}

\begin{entrée}{ʁnɤt}{}{ⓔʁnɤt} 
\classe{vs} \paradigme{dir}{tɤ-}\paradigme{dir}{nɯ-}
\begin{définition}\pfra{nuisible}\end{définition}
\begin{définition}\pcmn{有害}\end{définition}
\begin{définition}\pfra{être gêné de}\end{définition}
\begin{définition}\pcmn{不好意思,过意不去}\end{définition}
\begin{exemple}\pjya{nɤʑo nɤ-χti ɯ-taʁ ɲɯ-tɯ-ʁnɤt}\hspace{5pt}\pcmn{你欺负他}\end{exemple}
\begin{exemple}\pjya{ɯʑo kɯ kɤ-sɯzdɯɣ ɲɯ-naʁnɤt}\hspace{5pt}\pcmn{他不好意思麻烦别人}\end{exemple}
\begin{sous-entrée}{naʁnɤt}{ⓔʁnɤtⓝnaʁnɤt} 
\classe{vt} \end{sous-entrée}

\begin{sous-entrée}{ɣɤʁnɤt}{ⓔʁnɤtⓝɣɤʁnɤt} 
\classe{vs} 
\begin{définition}\pfra{s'abîmer facilement}\end{définition}
\begin{définition}\pcmn{容易损坏}\end{définition}
\begin{exemple}\pjya{tɯ-rnom ɯ-ɕɤrɯ nɯ wuma ʑo ɣɤʁnɤt, pjɯ-kɯ-ndʐaβ, kú-wɣ-nɯ-rpu nɯ ra tɕe tu-ʁnɤt mbat}\hspace{5pt}\pcmn{肋骨容易裂伤}\end{exemple}\end{sous-entrée}

\relationsémantique{参考}{\lien{ⓔsaʁnɤt}{saʁnɤt}}\étymologie{gnod}\end{entrée}

\begin{entrée}{ʁndɤr}{}{ⓔʁndɤr} 
\classe{vi}  
\grammaire{acaus} \paradigme{dir}{pɯ-}
\begin{définition}\pfra{se disperser}\end{définition}
\begin{définition}\pcmn{散开}\end{définition}
\begin{exemple}\pjya{tɯjpu pjɤ-ʁndɤr}\hspace{5pt}\pcmn{面粉撒得到处都是}\end{exemple}
\begin{exemple}\pjya{laχtɕha pjɤ-ʁndɤr}\hspace{5pt}\pcmn{东西撒得到处都是}\end{exemple}
\begin{exemple}\pjya{fsapaʁ ra pjɤ-ʁndɤr-nɯ}\hspace{5pt}\pcmn{牲畜分散了}\end{exemple}
\begin{exemple}\pjya{ɯ-sɯm ra pjɤ-ʁndɤr ʑo (=ɯ-ku ra pjɤ-mɯɕtaʁ ʑo)}\hspace{5pt}\pcmn{他被吓到了}\end{exemple}\relationsémantique{参考}{\lien{ⓔχtɤr}{χtɤr}}\étymologie{gtor}\end{entrée}

\begin{entrée}{ʁndɯ}{}{ⓔʁndɯ} 
\classe{vt} \paradigme{dir}{tɤ-}\paradigme{dir}{pɯ-}\paradigme{dir}{tɤ-}
\begin{définition}\pfra{battre, frapper}\end{définition}
\begin{définition}\pcmn{打}\end{définition}
\begin{définition}\pfra{frapper (des gens)}\end{définition}
\begin{définition}\pcmn{打人}\end{définition}
\begin{exemple}\pjya{nɤ-stu tɤ-fse ma ta-ʁndɯ}\hspace{5pt}\pcmn{你表现好一点,不然就打你}\end{exemple}\relationsémantique{同义词}{\lien{ⓔrdoŋ}{rdoŋ}}
\begin{sous-entrée}{saʁndɯ}{ⓔʁndɯⓝsaʁndɯ} 
\classe{vi}  
\grammaire{apass} \end{sous-entrée}

\end{entrée}

\begin{entrée}{ʁndzɤr}{}{ⓔʁndzɤr} 
\classe{vt} \paradigme{dir}{pɯ-}\paradigme{dir}{nɯ-}\paradigme{dir}{\_}\paradigme{dir}{pɯ-}
\begin{définition}\pfra{couper avec des ciseaux, couper un arbre, scier}\end{définition}
\begin{définition}\pcmn{剪;砍树;锯断}\end{définition}
\begin{définition}\pfra{couper avec}\end{définition}
\begin{définition}\pcmn{用……来剪}\end{définition}
\begin{exemple}\pjya{ɕoŋtɕa pa-ʁndzɤr}\hspace{5pt}\pcmn{他把木料锯断了}\end{exemple}
\begin{exemple}\pjya{si pa-ʁndzɤr}\hspace{5pt}\pcmn{他把木料锯断了}\end{exemple}
\begin{exemple}\pjya{tɤ-ri pa-ʁndzɤr}\hspace{5pt}\pcmn{他把线剪断了}\end{exemple}
\begin{exemple}\pjya{rɟaŋsoʁ kɯ si pɯ-sɯʁndzar-a}\hspace{5pt}\pcmn{我用锯子把木料锯断了}\end{exemple}
\begin{exemple}\pjya{tsɯntu kɯ pɯ-sɯʁndzar-a}\hspace{5pt}\pcmn{我用剪刀剪了}\end{exemple}
\begin{sous-entrée}{sɯʁndzɤr}{ⓔʁndzɤrⓝsɯʁndzɤr}\end{sous-entrée}

\end{entrée}

\begin{entrée}{ʁnɯ}{}{ⓔʁnɯ} 
\classe{vi} \paradigme{dir}{tɤ-}\paradigme{dir}{tɤ-}
\begin{définition}\pfra{supposer}\end{définition}
\begin{définition}\pcmn{起疑心}\end{définition}
\begin{définition}\pfra{causer la suspicion}\end{définition}
\begin{définition}\pcmn{令人起疑心}\end{définition}
\begin{exemple}\pjya{tɤ-ʁnɯ-a}\hspace{5pt}\pcmn{我怀疑了}\end{exemple}
\begin{exemple}\pjya{ɯʑo kɯ-ti maŋe nɤri, aʑo tɤ-ʁnɯ-a ɕti ma}\hspace{5pt}\pcmn{他倒没有说,是我自己怀疑的}\end{exemple}
\begin{exemple}\pjya{ɯ-re wuma ʑo ɲɯ-ɬoʁ tɕe, tɤ-ʁnɯ-a ma ɯ-tshɯɣa mɯ́j-βdi}\hspace{5pt}\pcmn{他很想笑的样子,所以我就有点怀疑了}\end{exemple}
\begin{exemple}\pjya{kɤ-ti pjɯ-sthɯt maŋe tɕe, tɤ́-wɣ-sɯʁnɯ-a}\hspace{5pt}\pcmn{他说总是没完没了(一会这样说,一会那样说),令我起了疑心}\end{exemple}
\begin{sous-entrée}{sɯʁnɯ}{ⓔʁnɯⓝsɯʁnɯ} 
\classe{vt}  
\grammaire{caus} \end{sous-entrée}

\end{entrée}

\begin{entrée}{ʁnɯz}{}{ⓔʁnɯz} 
\classe{num} 
\begin{définition}\pfra{deux}\end{définition}
\begin{définition}\pcmn{二}\end{définition}\end{entrée}

\begin{entrée}{ʁnɯzmɯz}{}{ⓔʁnɯzmɯz} 
\classe{n} 
\begin{définition}\pfra{deux sortes (expression toute faite, emprunt)}\end{définition}
\begin{définition}\pcmn{两种}\end{définition}\end{entrée}

\begin{entrée}{ʁɲɤlwa}{}{ⓔʁɲɤlwa} 
\classe{n} 
\begin{définition}\pfra{enfer}\end{définition}
\begin{définition}\pcmn{阴间}\end{définition}
\begin{exemple}\pjya{ʁɲɤlwa sɤtɕha jɤ-ari}\hspace{5pt}\pcmn{他去了阴间(去世了)}\end{exemple}
\begin{exemple}\pjya{a-ʁɲɤlwa ʑo nɯ-rzoʁ}\hspace{5pt}\pcmn{我受了很多苦}\end{exemple}\relationsémantique{参考}{\lien{ⓔnɯʁɲɤlwa}{nɯʁɲɤlwa}}\étymologie{dmʲal.ba}\end{entrée}

\begin{entrée}{ʁɲɤrpa}{}{ⓔʁɲɤrpa} 
\classe{n} 
\begin{définition}\pfra{intendant du monastère}\end{définition}
\begin{définition}\pcmn{寺庙的管家}\end{définition}\étymologie{gɲer.pa}\end{entrée}

\begin{entrée}{ʁɲɟiʁɲɟi}{}{ⓔʁɲɟiʁɲɟi} 
\classe{idph.2} 
\begin{définition}\pfra{énorme}\end{définition}
\begin{définition}\pcmn{形容粗壮高大的样子}\end{définition}
\begin{définition}\pcmn{他长的又粗壮又高大}\end{définition}
\begin{exemple}\pjya{ʁɲɟiʁɲɟi ʑo ɲɯ-pa}\end{exemple}\relationsémantique{参考}{\lien{ⓔʁɲɟliʁɲɟli}{ʁɲɟliʁɲɟli}}
\begin{sous-entrée}{ʁɲɟinɤʁɲɟi}{ⓔʁɲɟiʁɲɟiⓝʁɲɟinɤʁɲɟi} 
\classe{idph.3} \end{sous-entrée}

\end{entrée}

\begin{entrée}{ʁɲɟliʁɲɟli}{}{ⓔʁɲɟliʁɲɟli} 
\classe{idph.2} 
\begin{définition}\pfra{énorme}\end{définition}
\begin{définition}\pcmn{肥壮;高大}\end{définition}
\begin{exemple}\pjya{loŋbutɕhi ʁɲɟliʁɲɟli ʑo ɲɯ-pa}\hspace{5pt}\pcmn{大象又肥壮又高大}\end{exemple}
\begin{exemple}\pjya{ɯʑo chɤ-tshu tɕe ʁɲɟliʁɲɟli ʑo chɤ-pa}\hspace{5pt}\pcmn{他变胖了,身体变得很肥壮}\end{exemple}\relationsémantique{参考}{\lien{ⓔʁɲɟiʁɲɟi}{ʁɲɟiʁɲɟi}}\end{entrée}

\begin{entrée}{ʁɲɯɣnaʁ}{}{ⓔʁɲɯɣnaʁ} 
\classe{n} 
\begin{définition}\pfra{bovidé à tête blanche, mais dont les yeux sont cerclés de noir}\end{définition}
\begin{définition}\pcmn{头白而眼圈黑的牛}\end{définition}\étymologie{mig.nag}\end{entrée}

\begin{entrée}{ʁɲɯɣra}{}{ⓔʁɲɯɣra} 
\classe{n} 
\begin{définition}\pfra{masque pour se couvrir les yeux}\end{définition}
\begin{définition}\pcmn{眼罩}\end{définition}\étymologie{mig.ra}\end{entrée}

\begin{entrée}{ʁɲɯm}{}{ⓔʁɲɯm} 
\classe{vt} \paradigme{dir}{tɤ-}
\begin{définition}\pfra{avoir des soupçons à l'égard de}\end{définition}
\begin{définition}\pcmn{怀疑,起疑心}\end{définition}
\begin{exemple}\pjya{ɯʑo kɯ tɤ́-wɣ-ʁɲɯm-a}\hspace{5pt}\pcmn{他对我起了疑心}\end{exemple}\end{entrée}

\begin{entrée}{ʁɲɯspa}{}{ⓔʁɲɯspa} 
\classe{n} 
\begin{définition}\pfra{deuxième mois}\end{définition}
\begin{définition}\pcmn{二月}\end{définition}\étymologie{gɲis.pa}\end{entrée}

\begin{entrée}{ʁŋaftɕɤt}{}{ⓔʁŋaftɕɤt} 
\classe{n} 
\begin{définition}\pfra{dernier jour du jeune}\end{définition}
\begin{définition}\pcmn{禁食斋的最后一天}\end{définition}
\begin{exemple}\pjya{sɲaŋne kɯ-maqhu ɯ-sŋi nɯ ʁŋaftɕɤt tu-kɯ-ti ɲɯ-ŋu}\hspace{5pt}\pcmn{禁食斋的最后一天叫做}\end{exemple}\end{entrée}

\begin{entrée}{ʁo}{}{ⓔʁo} 
\classe{adv} 
\begin{définition}\pfra{adversatif}\end{définition}
\begin{définition}\pcmn{倒}\end{définition}\end{entrée}

\begin{entrée}{ʁoŋmdzɤt}{}{ⓔʁoŋmdzɤt} 
\classe{n} 
\begin{définition}\pfra{moine en charge de la récitation des soutras}\end{définition}
\begin{définition}\pcmn{掌管念经活到的和尚}\end{définition}\end{entrée}

\begin{entrée}{ʁoŋtɕhaʁ}{}{ⓔʁoŋtɕhaʁ} 
\classe{n} 
\begin{définition}\pfra{menace}\end{définition}
\begin{définition}\pcmn{威胁;镇压}\end{définition}
\begin{exemple}\pjya{nɯ-ʁoŋtɕhaʁ sɤ-lɤt ci tú-wɣ-βzu ɲɯ-ra}\hspace{5pt}\pcmn{要用威胁的手段镇压他们}\end{exemple}\end{entrée}

\begin{entrée}{ʁrɯlu}{}{ⓔʁrɯlu} 
\classe{n} 
\begin{définition}\pfra{sans corne}\end{définition}
\begin{définition}\pcmn{无角的}\end{définition}\relationsémantique{参考}{\lien{ⓔta-ʁrɯ}{ta-ʁrɯ}}\end{entrée}

\begin{entrée}{ʁrɯrpu}{}{ⓔʁrɯrpu} 
\classe{n} 
\begin{définition}\pfra{coup de corne (pas avec la pointe)}\end{définition}
\begin{définition}\pcmn{用角打}\end{définition}
\begin{exemple}\pjya{jla kɯ a-taʁ ʁrɯrpu ta-lɤt}\hspace{5pt}\pcmn{犏牛用角打了我}\end{exemple}\relationsémantique{参考}{\lien{ⓔta-ʁrɯ}{ta-ʁrɯ}}\relationsémantique{参考}{\lien{ⓔrpu}{rpu}}\relationsémantique{参考}{\lien{}{nɯʁrɯrpu}}\end{entrée}

\begin{entrée}{ʁzɤβ}{}{ⓔʁzɤβ} 
\classe{vi} \paradigme{dir}{tɤ-}
\begin{définition}\pfra{attentif, soigneux}\end{définition}
\begin{définition}\pcmn{细心;细致}\end{définition}
\begin{exemple}\pjya{ɯ-ma ɲɯ-ʁzɤβ}\hspace{5pt}\pcmn{他做事很细致}\end{exemple}
\begin{exemple}\pjya{ɯ-βraʁ kɤ-ʁzaβ-a}\hspace{5pt}\pcmn{我拴好了}\end{exemple}
\begin{exemple}\pjya{kɯm ɯ-ɣɤβdi kɤ-ʁzaβ-a}\hspace{5pt}\pcmn{我把门顶住好了}\end{exemple}
\begin{exemple}\pjya{tɤ-rɤku kɤ-saχsi nɯ-ʁzaβ-a}\hspace{5pt}\pcmn{我把粮食跟石头泥巴分开好了}\end{exemple}
\begin{exemple}\pjya{smɤɣ ɯ-saʁ nɯ-ʁzaβ-a}\hspace{5pt}\pcmn{我把羊毛撕好了}\end{exemple}
\begin{exemple}\pjya{ɯ-pa kɯ-ʁzɤβ ci ɲɯ-ŋu}\hspace{5pt}\pcmn{他是个做事做得很细致的人}\end{exemple}
\begin{sous-entrée}{sɯʁzɤβ}{ⓔʁzɤβⓝsɯʁzɤβ} 
\classe{vt} 
\begin{exemple}\pjya{nɤ-kɯ-mŋɤm koŋla kɤ-rtoʁ a-pɯ-tɯ-sɯ-ʁzɤβ}\end{exemple}\end{sous-entrée}

\étymologie{gzab}\end{entrée}

\begin{entrée}{ʁzɤmi}{}{ⓔʁzɤmi} 
\classe{n} 
\begin{définition}\pfra{couple}\end{définition}
\begin{définition}\pcmn{夫妻}\end{définition}\étymologie{bza.mi}\end{entrée}

\begin{entrée}{ʁzɤn}{}{ⓔʁzɤn} 
\classe{n} 
\begin{définition}\pfra{kyasha}\end{définition}
\begin{définition}\pcmn{袈裟}\end{définition}\étymologie{gzan}\end{entrée}

\begin{entrée}{ʁzɤr}{}{ⓔʁzɤr} 
\classe{n} 
\begin{définition}\pfra{difficulté à respirer}\end{définition}
\begin{définition}\pcmn{呼吸困难}\end{définition}
\begin{exemple}\pjya{ʁzɤr nɯ tu-kɯ-ɤɕqhe ri mŋɤm, tu-kɯ-ɤtɕhɯz ri mŋɤm tɯ-sŋɯro kɯ-wxti lú-wɣ-tɕɤt ri mŋɤm tɯ-ŋgo kɯ-ŋɤn ci ŋu}\hspace{5pt}\pcmn{\lien{ⓔʁzɤr}{ʁzɤr}是一种严重的病,(如果得了这个病)你咳嗽时,打喷嚏时,出大气时都感到痛。}\end{exemple}\end{entrée}

\begin{entrée}{ʁzi}{}{ⓔʁzi} 
\classe{vs} \paradigme{dir}{thɯ-}
\begin{définition}\pfra{nécessaire}\end{définition}
\begin{définition}\pcmn{需要}\end{définition}
\begin{exemple}\pjya{ki ɯʑo ɣɯ ɲɯ-ʁzi}\hspace{5pt}\pcmn{他需要这个}\end{exemple}
\begin{exemple}\pjya{ɯʑo kɯ pa-nɤpe ri ɕɤxɕo tɕe cho-ʁzi tɕe ra ɲɯ-ti}\hspace{5pt}\pcmn{他原来不喜欢,但是现在就需要了,就说要}\end{exemple}\relationsémantique{参考}{\lien{ⓔnaʁzi}{naʁzi}}\end{entrée}

\begin{entrée}{ʁzraŋʁzraŋ}{}{ⓔʁzraŋʁzraŋ} 
\classe{idph.2} 
\begin{définition}\pfra{ébouriffé}\end{définition}
\begin{définition}\pcmn{乱蓬蓬}\end{définition}
\begin{exemple}\pjya{ɯ-kɤrme ʁzraŋʁzraŋ ʑo ɲɯ-pa}\hspace{5pt}\pcmn{他头发是乱蓬蓬的}\end{exemple}\end{entrée}

\begin{entrée}{ʁzɯɣ}{}{ⓔʁzɯɣ} 
\classe{n} 
\begin{définition}\pfra{apparence}\end{définition}
\begin{définition}\pcmn{相貌}\end{définition}\étymologie{gzugs}\end{entrée}

\begin{entrée}{ʁzɯlu}{}{ⓔʁzɯlu} 
\classe{n} 
\begin{définition}\pfra{maladie de l'œil}\end{définition}
\begin{définition}\pcmn{白眼病}\end{définition}
\begin{exemple}\pjya{tɯrme ʁzɯlu nɯ mbro ɯ-phɯ ci ŋu, mbro ʁzɯlu nɯ khɯna ɯ-phɯ ci ŋu, khɯna ʁzɯlu nɯ mbro ɯ-phɯ ci ŋu}\hspace{5pt}\pcmn{患白眼病的人有马的价格,患白眼病的马只有狗的价格,患白眼病的狗有马的价格}\end{exemple}\end{entrée}

\begin{entrée}{ʁʑɯnɯ}{}{ⓔʁʑɯnɯ} 
\classe{n} 
\begin{définition}\pfra{jeune homme}\end{définition}
\begin{définition}\pcmn{小伙子,青年}\end{définition}\relationsémantique{参考}{\lien{ⓔnɯʁʑɯnɯ}{nɯʁʑɯnɯ}}\étymologie{gʑon.nu}\end{entrée}

\newpage\caractère{s}

\begin{entrée}{sa}{}{ⓔsa} 
\classe{vs}
\classe{vt} \paradigme{dir}{nɯ-}\paradigme{dir}{tɤ-}
\begin{définition}\pfra{s'émousser}\end{définition}
\begin{définition}\pcmn{磨损;变钝}\end{définition}
\begin{exemple}\pjya{mbrɯtɕɯ to-sa}\hspace{5pt}\pcmn{刀变钝了}\end{exemple}
\begin{exemple}\pjya{qraʁ ɲɤ-sa}\hspace{5pt}\pcmn{铧变钝了}\end{exemple}\relationsémantique{同义词}{\lien{ⓔzɤt}{zɤt}}
\begin{sous-entrée}{ɣɤsa}{ⓔsaⓝɣɤsa} 
\classe{vs} \sens{1}
\begin{définition}\pfra{s'émousser facilement}\end{définition}
\begin{définition}\pcmn{容易变钝}\end{définition}\end{sous-entrée}

\sens{2}\paradigme{dir}{nɯ-}\paradigme{dir}{tɤ-}
\begin{définition}\pfra{qui diminue vite}\end{définition}
\begin{définition}\pcmn{减少得快(食物;衣服)}\end{définition}
\begin{exemple}\pjya{tɯ-mgo ɲɯ-ɣɤsa}\hspace{5pt}\pcmn{饭很快吃完(不经吃)}\end{exemple}
\begin{exemple}\pjya{nɤ-ɕoʁɕoʁ ɲɯ-ɣɤsa}\hspace{5pt}\pcmn{你纸巾用得很快}\end{exemple}\relationsémantique{同义词}{\lien{ⓔarɕoⓝɣɤrɕo}{ɣɤrɕo}}
\begin{sous-entrée}{sɯxsa}{ⓔsaⓢ2ⓝsɯxsa}\end{sous-entrée}

\begin{définition}\pfra{émousser}\end{définition}
\begin{définition}\pcmn{弄钝}\end{définition}
\begin{exemple}\pjya{mbrɯtɕɯ nɯ-sɯxsa-t-a}\hspace{5pt}\pcmn{我把刀弄钝了}\end{exemple}\end{entrée}

\begin{entrée}{saɕɯ}{}{ⓔsaɕɯ} 
\classe{n} 
\begin{définition}\pfra{mélèze}\end{définition}
\begin{définition}\pcmn{落叶松}\end{définition}
\begin{exemple}\pjya{saɕɯ nɯ ɯ-tshɯɣa tɯrgi ɲɯ-fse, ɯ-mdoʁ nɯ ra ɕɤɣ ɲɯ-fse, ɯ-ru nɯ tɯrgi ɲɯ-fse, tɕe tɯrgi cho ɕɤɣ ni pjɯ-nɯɕɯrɲɟo-ndʑi mɤ-cha, qartsɯmɤftɕar arŋi-ndʑi. saɕɯ pjɯ-nɯɕɯrɲɟo tɕe, ɯ-jwaʁ pjɤ-ŋgra ɕti, tɕe tɯrme ra kɯ saɕɯ nɯ tɯrgi ɣɯ ɯ-ftsa ŋu ri tɯrgi sɤz kɯ-mbro tu-ɬoʁ ŋu tɕe, nɯ mɤ-kɯ-tʂaŋ ɯ-ndʐa kɯ pjɯ-nɯɕɯrɲɟo ŋu tu-ti-nɯ ŋu.}\hspace{5pt}\pcmn{落叶松的形状像杉树,但是颜色和柏树相同。树干像杉树的一样。杉树和柏树叶子秋天不会变色,不分冬夏都是绿的,落叶松叶子秋天变色,然后落叶。人们说落叶松虽然是杉树的侄子但生长的地区海拔比杉树生长的地区高,因为这样不合理,所以它叶子秋天变色。}\end{exemple}
\begin{exemple}\pjya{saɕɯ cho tɯrgi nɯ ni kɤndʑɯrpɯftsa ɲɯ-ŋu-ndʑi. ndʑi-sta nɯnɯ, ɯ-rpɯ ɯ-sta saɕɯ lo-ɕe tɕe, qartsɯ tɕe kɯ-nɯɕɯrɲɟo ŋu. tɯrgi cho-maŋthi tɕe núndʐa nɯnɯ tɯrgi mɤ-nɯɕɯrɲɟo tu-kɯ-ti ɲɯ-ŋu.}\hspace{5pt}\pcmn{落叶松和杉树是舅甥的关系,落叶松去了舅舅的位子所以到了秋天变色,杉树到了下游所以不变色}\end{exemple}\end{entrée}

\begin{entrée}{sakaβ}{}{ⓔsakaβ} 
\classe{n} 
\begin{définition}\pfra{puits}\end{définition}
\begin{définition}\pcmn{井}\end{définition}\relationsémantique{参考}{\lien{ⓔkaβ}{kaβ}}\end{entrée}

\begin{entrée}{sakhar}{}{ⓔsakhar} 
\classe{n} 
\begin{définition}\pfra{porcherie}\end{définition}
\begin{définition}\pcmn{猪圈}\end{définition}\end{entrée}

\begin{entrée}{salaboŋboŋ}{}{ⓔsalaboŋboŋ} 
\classe{n} 
\begin{définition}\pfra{vesse-de-loup}\end{définition}
\begin{définition}\pcmn{马勃}\end{définition}
\begin{exemple}\pjya{salaboŋboŋ nɯ stɤmku ri tu-ɬoʁ ŋu tɕe, kɯ-wɣrum ŋu kɯ-ɤrtɯm rloʁ rloʁ ŋu ɯ-ru me, wuma ʑo mpɯ, nɯ-rom tɕe pjɯ́-wɣ-rɤtɕaʁ tɕe rdɯl ɲɯ-nɯɬoʁ ŋu.}\hspace{5pt}\pcmn{马勃长在草地上,全白色,呈球形,没有杆,很软,干了以后踩上去就会扬起灰尘(孢子粉)来。}\end{exemple}\relationsémantique{同义词}{\lien{ⓔɬɤndʐithamaka}{ɬɤndʐithamaka}}\end{entrée}

\begin{entrée}{salakɯndzur}{}{ⓔsalakɯndzur} 
\classe{n} 
\begin{définition}\pfra{galipette}\end{définition}
\begin{définition}\pcmn{(翻)筋斗}\end{définition}
\begin{exemple}\pjya{a-tɕɯ kɯ salakɯndzur ta-βzu}\hspace{5pt}\pcmn{我儿子翻了筋斗}\end{exemple}\end{entrée}

\begin{entrée}{salaʁdɯʁdɯɣ}{}{ⓔsalaʁdɯʁdɯɣ} 
\classe{n} 
\begin{définition}\pfra{(personne, chose) qui gêne}\end{définition}
\begin{définition}\pcmn{碍事的(人;东西)}\end{définition}\relationsémantique{参考}{\lien{ⓔsala,zrɯ}{sala,zrɯ}}\end{entrée}

\begin{entrée}{sala,zrɯ}{}{ⓔsala,zrɯ} 
\classe{n}
\classe{vt} \paradigme{dir}{kɤ-}
\begin{définition}\pfra{prendre de la place, gêner}\end{définition}
\begin{définition}\pcmn{占(不该占的)地方}\end{définition}\relationsémantique{Component 1}{\lien{}{sala}}\relationsémantique{Component 2}{\lien{ⓔzrɯⓗ1}{zrɯ}}
\begin{exemple}\pjya{nɯtɕu tɤ-rɤru ma sala ɲɯ-tɯ-zri ɲɯ-ŋu}\hspace{5pt}\pcmn{起来,你在那里很碍事}\end{exemple}
\begin{exemple}\pjya{sala kɯ-zrɯ ma-tɯ-βze}\hspace{5pt}\pcmn{别占地方}\end{exemple}\relationsémantique{参考}{\lien{ⓔsalaʁdɯʁdɯɣ}{salaʁdɯʁdɯɣ}}\end{entrée}

\begin{entrée}{sali}{}{ⓔsali} 
\classe{n} 
\begin{définition}\pfra{arbalète}\end{définition}
\begin{définition}\pcmn{弩弓}\end{définition}\end{entrée}

\begin{entrée}{saloŋ}{}{ⓔsaloŋ} 
\classe{adv} 
\begin{définition}\pfra{partout}\end{définition}
\begin{définition}\pcmn{到处}\end{définition}
\begin{définition}\pcmn{他走遍了所有的地方}\end{définition}
\begin{exemple}\pjya{saloŋ ri ʑo ɕ-to-khɤt}\end{exemple}\relationsémantique{参考}{\lien{ⓔlaloŋ}{laloŋ}}\end{entrée}

\begin{entrée}{san}{}{ⓔsan} 
\classe{n} 
\begin{définition}\pfra{parapluie}\end{définition}
\begin{définition}\pcmn{伞}\end{définition}\end{entrée}

\begin{entrée}{saŋdi}{}{ⓔsaŋdi} 
\classe{n} 
\begin{définition}\pfra{place des serviteurs, à l'ouest}\end{définition}
\begin{définition}\pcmn{下等人坐的地方(往西方)}\end{définition}\end{entrée}

\begin{entrée}{saŋrɟɤz}{}{ⓔsaŋrɟɤz} 
\classe{n} 
\begin{définition}\pfra{bouddha}\end{définition}
\begin{définition}\pcmn{佛,神仙}\end{définition}\étymologie{saŋs.rgʲas}\end{entrée}

\begin{entrée}{saqru}{}{ⓔsaqru}\relationsémantique{参考}{\lien{ⓔqru}{qru}}\end{entrée}

\begin{entrée}{sar}{}{ⓔsar} 
\classe{vt} \paradigme{dir}{pɯ-}
\begin{définition}\pfra{filtrer}\end{définition}
\begin{définition}\pcmn{滤}\end{définition}
\begin{exemple}\pjya{mbrɤz pɯ-sar}\hspace{5pt}\pcmn{你把米过滤一下}\end{exemple}
\begin{exemple}\pjya{tɤ-lu pɯ-sar}\hspace{5pt}\pcmn{你把牛奶过滤一下}\end{exemple}\end{entrée}

\begin{entrée}{sarndzu}{}{ⓔsarndzu} 
\classe{n}  
\grammaire{n.lieu} 
\begin{définition}\pfra{Gsar.rdzong}\end{définition}
\begin{définition}\pcmn{沙尔宗乡}\end{définition}\end{entrée}

\begin{entrée}{sarsi}{}{ⓔsarsi} 
\classe{n} 
\begin{définition}\pfra{abricot}\end{définition}
\begin{définition}\pcmn{杏}\end{définition}
\begin{exemple}\pjya{sarsi nɯ zgo kɯ-mbɤr tu-ɬoʁ ŋu, si mbro ɯ-rtaʁ dɤn, ɯ-βri nɯ kɯ-pɣi ŋu, ɯ-jwaʁ nɯ qaɕti ɯ-jwaʁ cho naχtɕɯɣ, ɯ-mɯntoʁ kɯ-ɣɯrni ɯ-ŋgɯz kɯnɤ kɯ-wɣrum tsa ŋu. ɯ-jwaʁ ɲɯ-lɤt ɕɯŋgɯ ɯ-mɯntoʁ ɲɯ-lɤt ŋu. ɯ-mat thɯ-aβzu tɕe, qaɕti tsa fse ri ndɯβ. thɯ-tɯt tɕe, ɯ-phaʁ ntsi ɣɯrni, ɯ-phaʁ ntsi qarŋe, tú-wɣ-ndza qaɕti sɤz mɯm ma chi tɕe, kɤ́-rqhɯ-rqhu kɤ-ndza sna. sarsi nɯ kha ɯ-rkɯ pɯ-kɤ-nɯ-ji tɕi tu, ɕɯŋgɯ zɯ ɯʑo tu-kɯ-nɯ-ɬoʁ tɕi tu.}\hspace{5pt}\pcmn{杏树生长在下半山上,长得很高,枝桠多,树皮是灰色的,叶子和桃树的叶子一样,花是粉红色的。在长叶子之前就开花。结果像桃子,但小一些。成熟,半边是黄色,半边是红色的。吃起来比桃子好吃,因为很甜,可以连皮一起吃。杏树,有的种在房子旁边,也有自己生长在野外的。}\end{exemple}\relationsémantique{参考}{\lien{ⓔnɯsarsi}{nɯsarsi}}\end{entrée}

\begin{entrée}{sarwɯ}{}{ⓔsarwɯ} 
\classe{n} 
\begin{définition}\pfra{élément du métier à tisser}\end{définition}
\begin{définition}\pcmn{搓杆(纺锤的木棒)}\end{définition}\end{entrée}

\begin{entrée}{saʁ}{}{ⓔsaʁ} 
\classe{vt} \paradigme{dir}{nɯ-}
\begin{définition}\pfra{séparer des fils emmêlés}\end{définition}
\begin{définition}\pcmn{撕开}\end{définition}
\begin{exemple}\pjya{smɤɣ nɯ-saʁ-a}\hspace{5pt}\pcmn{我撕开了羊毛}\end{exemple}
\begin{exemple}\pjya{tɤ-rme nɯ-saʁ-a}\hspace{5pt}\pcmn{我撕开了毛}\end{exemple}\end{entrée}

\begin{entrée}{saʁdɤt}{}{ⓔsaʁdɤt} 
\classe{vs}  
\grammaire{deexp} \paradigme{dir}{tɤ-}
\begin{définition}\pfra{glissant}\end{définition}
\begin{définition}\pcmn{滑(路)}\end{définition}
\begin{exemple}\pjya{a-pɯ-zbaʁ tɕe mɤ-saʁdɤt, tɯ-mɯ a-pɯ-lɤt tɕe saʁdɤt}\hspace{5pt}\pcmn{地干就不滑,下雨的话就滑}\end{exemple}
\begin{exemple}\pjya{tɯ-mɯ pjɤ-lɤt tɕe ɲɯ-saʁdɤt}\hspace{5pt}\pcmn{下雨了,地很滑}\end{exemple}\relationsémantique{参考}{\lien{ⓔaʁdɤt}{aʁdɤt}}\relationsémantique{同义词}{\lien{ⓔsɤŋgio}{sɤŋgio}}\end{entrée}

\begin{entrée}{saʁdɯɣ}{}{ⓔsaʁdɯɣ} 
\classe{vs}  
\grammaire{deexp} \paradigme{dir}{tɤ-}
\begin{définition}\pfra{ennuyer, empêcher}\end{définition}
\begin{définition}\pcmn{干扰;防碍}\end{définition}
\begin{exemple}\pjya{tɤ-rɤru ma ɲɯ-tɯ-saʁdɯɣ}\hspace{5pt}\pcmn{你起来,你在那里碍事}\end{exemple}\end{entrée}

\begin{entrée}{saʁjɤr}{}{ⓔsaʁjɤr}\relationsémantique{参考}{\lien{ⓔaʁjɤr}{aʁjɤr}}\end{entrée}

\begin{entrée}{saʁjɯβ}{}{ⓔsaʁjɯβ} 
\classe{vt} \paradigme{dir}{tɤ-}
\begin{définition}\pfra{cacher}\end{définition}
\begin{définition}\pcmn{遮住,遮掩}\end{définition}
\begin{exemple}\pjya{a-mɤ-pɯ́-wɣ-mto nɯ-sɯso-t-a tɕe tɤ-saʁjɯβ-a}\hspace{5pt}\pcmn{为了不让别人发现,我把它遮掩了}\end{exemple}\relationsémantique{同义词}{\lien{ⓔsqaβjɯβ}{sqaβjɯβ}}\relationsémantique{参考}{\lien{}{naʁjɯɣ}}\relationsémantique{参考}{\lien{ⓔta-ʁjɯβ}{ta-ʁjɯβ}}\end{entrée}

\begin{entrée}{saʁɟa}{}{ⓔsaʁɟa}\relationsémantique{参考}{\lien{ⓔaʁɟa}{aʁɟa}}\end{entrée}

\begin{entrée}{saʁlɤt}{}{ⓔsaʁlɤt} 
\classe{vt} \paradigme{dir}{lɤ-}\paradigme{dir}{pɯ-}
\begin{définition}\pfra{vanner}\end{définition}
\begin{définition}\pcmn{扬场}\end{définition}
\begin{exemple}\pjya{tɤ-rɤku pɯ-saʁlat-a (=qale pɯ-lat-a)}\hspace{5pt}\pcmn{我扬了粮食}\end{exemple}
\begin{exemple}\pjya{stoʁ lɤ-saʁlat-a}\hspace{5pt}\pcmn{我扬了胡豆}\end{exemple}\end{entrée}

\begin{entrée}{saʁnɤt}{}{ⓔsaʁnɤt} 
\classe{vs} \paradigme{dir}{thɯ-}\sens{1}
\begin{définition}\pfra{faire mal}\end{définition}
\begin{définition}\pcmn{伤到}\end{définition}
\begin{exemple}\pjya{a-ɕa ɯ-taʁ ɲɯ-saʁnɤt}\hspace{5pt}\pcmn{伤到我了}\end{exemple}\sens{2}
\begin{définition}\pfra{faire du mal}\end{définition}
\begin{définition}\pcmn{伤害}\end{définition}
\begin{exemple}\pjya{a-taʁ pɯ-saʁnɤt}\hspace{5pt}\pcmn{伤害我了}\end{exemple}\relationsémantique{参考}{\lien{ⓔʁnɤt}{ʁnɤt}}\étymologie{gnod}\end{entrée}

\begin{entrée}{saʁre}{}{ⓔsaʁre} 
\classe{vs} \paradigme{dir}{tɤ-}
\begin{définition}\pfra{austère, impressionnant}\end{définition}
\begin{définition}\pcmn{严肃,庄重}\end{définition}\relationsémantique{参考}{\lien{ⓔɣɤʁre}{ɣɤʁre}}\relationsémantique{参考}{\lien{}{ɯ-rʁe}}\end{entrée}

\begin{entrée}{saʁrɯm}{}{ⓔsaʁrɯm} 
\classe{vt}  
\grammaire{denom} \paradigme{dir}{tɤ-}
\begin{définition}\pfra{cacher, couvrir}\end{définition}
\begin{définition}\pcmn{遮住}\end{définition}
\begin{exemple}\pjya{nɤ-tɤŋe tɤ-saʁrɯm-a}\hspace{5pt}\pcmn{我帮你遮了太阳}\end{exemple}
\begin{exemple}\pjya{tɤ-rte tɤ-ŋge tɕe tɤŋe tɤ-saʁrɯm}\hspace{5pt}\pcmn{你戴上帽子,遮住太阳}\end{exemple}
\begin{exemple}\pjya{ɯ-kɯ-saʁrɯm ɲo-me tɕe to-sɤmto}\hspace{5pt}\pcmn{遮住它的东西没有了,别人就看得见它了}\end{exemple}\relationsémantique{参考}{\lien{ⓔta-ʁrɯm}{ta-ʁrɯm}}\end{entrée}

\begin{entrée}{sat}{}{ⓔsat} 
\classe{vt}  
\grammaire{caus}
\grammaire{caus}
\grammaire{refl} \paradigme{dir}{pɯ-}\paradigme{dir}{pɯ-}
\begin{définition}\pfra{tuer}\end{définition}
\begin{définition}\pcmn{杀}\end{définition}
\begin{sous-entrée}{rɤsat}{ⓔsatⓝrɤsat} 
\classe{vi}  
\grammaire{apass} 
\begin{définition}\pfra{tuer des animaux}\end{définition}
\begin{définition}\pcmn{杀动物}\end{définition}\relationsémantique{参考}{\lien{}{sɤsat₁}}\end{sous-entrée}

\begin{sous-entrée}{sɯsat}{ⓔsatⓝsɯsat} 
\classe{vt} \end{sous-entrée}

\sens{1}
\begin{définition}\pfra{faire tuer}\end{définition}
\begin{définition}\pcmn{使人杀(另一个人)}\end{définition}\sens{2}\paradigme{dir}{pɯ-}
\begin{définition}\pfra{tuer avec}\end{définition}
\begin{définition}\pcmn{用……杀}\end{définition}
\begin{sous-entrée}{ʑɣɤsɯsat}{ⓔsatⓢ2ⓝʑɣɤsɯsat} 
\classe{vi} \end{sous-entrée}

\begin{définition}\pfra{se faire tuer}\end{définition}
\begin{définition}\pcmn{遭人杀害}\end{définition}
\begin{sous-entrée}{asɯsat}{ⓔsatⓝasɯsat} 
\classe{vi}  
\grammaire{refl} 
\begin{définition}\pfra{se tuer les uns les autres}\end{définition}
\begin{définition}\pcmn{互相残杀}\end{définition}
\begin{exemple}\pjya{pjɤ-k-ɤsɯsat-ndʑi-ci}\hspace{5pt}\pcmn{他们俩互相残杀了}\end{exemple}\end{sous-entrée}

\end{entrée}

\begin{entrée}{saχaʁ}{}{ⓔsaχaʁ} 
\classe{vs} \paradigme{dir}{nɯ-}
\begin{définition}\pfra{être extrême}\end{définition}
\begin{définition}\pcmn{极度的}\end{définition}\relationsémantique{参考}{\lien{ⓔnaχaʁ}{naχaʁ}}\end{entrée}

\begin{entrée}{saχɕɤra}{}{ⓔsaχɕɤra} 
\classe{vt} \paradigme{dir}{nɯ-}
\begin{définition}\pfra{étendre}\end{définition}
\begin{définition}\pcmn{铺平;弄得平整;陈展}\end{définition}
\begin{exemple}\pjya{tɯ-sta ɯ-taʁ @pugai nɯ-saχɕɤre}\hspace{5pt}\pcmn{你把铺盖在床上铺平}\end{exemple}\end{entrée}

\begin{entrée}{saχɕɯβ}{}{ⓔsaχɕɯβ} 
\classe{vt}  
\grammaire{caus} \paradigme{dir}{tɤ-}
\begin{définition}\pfra{enrouler les brins (d'une corde)}\end{définition}
\begin{définition}\pcmn{把几股绳子 合并拧在一起、弄成双股}\end{définition}
\begin{exemple}\pjya{tɯmbri tɤ-saχɕɯβ-i}\hspace{5pt}\pcmn{我们把绳子拧在一起了}\end{exemple}
\begin{exemple}\pjya{tɯmbri χsɯ-ldʑa ʑo to-saχɕɯβ tɕe ɲɯ-ngɯt}\hspace{5pt}\pcmn{他把三股绳子拧在一起了,这样很结实。}\end{exemple}
\begin{exemple}\pjya{stukɤr thɯ-saχɕɯβ-i}\hspace{5pt}\pcmn{我们把梁弄成双根}\end{exemple}\relationsémantique{参考}{\lien{ⓔaχɕɯβ}{aχɕɯβ}}\relationsémantique{参考}{\lien{ⓔnɯsaχɕɯβ}{nɯsaχɕɯβ}}\end{entrée}

\begin{entrée}{saχɕɯn}{}{ⓔsaχɕɯn} 
\classe{vs} \paradigme{dir}{nɯ-}
\begin{définition}\pfra{frais, propre, hygiénique}\end{définition}
\begin{définition}\pcmn{新鲜;卫生}\end{définition}
\begin{exemple}\pjya{tɤ-mthɯm mɯ́j-saχɕɯn, ɲɯ-saχɕɯn}\hspace{5pt}\pcmn{肉不新鲜,很新鲜}\end{exemple}
\begin{exemple}\pjya{qaɟy mɯ́j-saχɕɯn}\hspace{5pt}\pcmn{鱼不新鲜}\end{exemple}\relationsémantique{参考}{\lien{ⓔnaχɕɯn}{naχɕɯn}}\end{entrée}

\begin{entrée}{saχpaʁ}{}{ⓔsaχpaʁ} 
\classe{vt} \paradigme{dir}{tɤ-}
\begin{définition}\pfra{respecter}\end{définition}
\begin{définition}\pcmn{尊重}\end{définition}
\begin{exemple}\pjya{jiɕqha nɯ ɲɯ-ɣɤʁre tɕe, kɤ-saχpaʁ ɲɯ-sna}\hspace{5pt}\pcmn{那个人是个受尊重的人,要尊重他}\end{exemple}
\begin{exemple}\pjya{aʑo ɣɯ-saχpaʁ-a}\hspace{5pt}\pcmn{他尊重我}\end{exemple}
\begin{sous-entrée}{sɤsaχpaʁ}{ⓔsaχpaʁⓝsɤsaχpaʁ} 
\classe{vs}  
\grammaire{deexp} 
\begin{définition}\pfra{être respecté}\end{définition}
\begin{définition}\pcmn{受尊重}\end{définition}
\begin{exemple}\pjya{jiɕqha nɯ ɲɯ-sɤsaχpaʁ}\hspace{5pt}\pcmn{那个人受人尊重}\end{exemple}\end{sous-entrée}

\begin{sous-entrée}{asaχpɯχpaʁ}{ⓔsaχpaʁⓝasaχpɯχpaʁ} 
\classe{vi}  
\grammaire{refl} 
\begin{définition}\pfra{se respecter les uns les autres}\end{définition}
\begin{définition}\pcmn{互相尊重}\end{définition}
\begin{exemple}\pjya{tɕiʑo ni ʑɤŋgɯz asaχpɯχpaʁ-tɕi}\hspace{5pt}\pcmn{我们俩互相尊重}\end{exemple}\end{sous-entrée}

\end{entrée}

\begin{entrée}{saχsɤl}{₁₂}{ⓔsaχsɤlⓗ1ⓗ2} 
\classe{vs}
\classe{vt} \paradigme{dir}{tɤ-}
\begin{définition}\pfra{clair, évident}\end{définition}
\begin{définition}\pcmn{明显}\end{définition}
\begin{exemple}\pjya{to-saχsɤl}\hspace{5pt}\pcmn{变得很明显}\end{exemple}
\begin{sous-entrée}{saχsɤl/\variante{sɯsaχsɤl}}{ⓔsaχsɤlⓗ1ⓝsaχsɤl}\end{sous-entrée}

\begin{définition}\pfra{révéler}\end{définition}
\begin{définition}\pcmn{透露出,让……显现出来}\end{définition}
\begin{exemple}\pjya{kɯ-ŋu ɯ-kɯ-ŋu nɯ kɤ-sɯsaχsɤl ra}\hspace{5pt}\pcmn{要把真相透露出去}\end{exemple}\end{entrée}

\begin{entrée}{saχsi}{}{ⓔsaχsi} 
\classe{vt} \paradigme{dir}{nɯ-}\paradigme{dir}{tɤ-}
\begin{définition}\pfra{faire complètement}\end{définition}
\begin{définition}\pcmn{做得彻底}\end{définition}
\begin{exemple}\pjya{tɤ-rɤku nɯ-saχsi-t-a}\hspace{5pt}\pcmn{我把粮食跟石头泥巴分开了}\end{exemple}
\begin{exemple}\pjya{nɤ-khɯtsa tɤ-saχsi}\hspace{5pt}\pcmn{你把碗里的东西吃完!}\end{exemple}
\begin{exemple}\pjya{ɕ-kɤ-nɯzʁe tɤ-saχsi}\hspace{5pt}\pcmn{你把东西搬过去,不要漏掉一些(你给我搬得干净彻底)}\end{exemple}
\begin{exemple}\pjya{mɯ-ɲɤ-saχsi}\hspace{5pt}\pcmn{没有做得彻底}\end{exemple}
\begin{exemple}\pjya{kɤntɕhaʁ kɯ-rɤma ɲɯ-dɤn, kɤ-saχsi mɯ́jkhɯ}\hspace{5pt}\pcmn{街上清道夫很多,但是不能清洁干净}\end{exemple}\relationsémantique{参考}{\lien{ⓔaχsi}{aχsi}}\end{entrée}

\begin{entrée}{saχsom}{}{ⓔsaχsom}\relationsémantique{参考}{\lien{ⓔaχsom}{aχsom}}\end{entrée}

\begin{entrée}{saχsɯ}{₁}{ⓔsaχsɯⓗ1} 
\classe{n} 
\begin{définition}\pfra{repas de midi}\end{définition}
\begin{définition}\pcmn{中午饭,午饭}\end{définition}
\begin{exemple}\pjya{nɤʑo nɤ-saχsɯ ɯ-pɯ́-tu?}\hspace{5pt}\pcmn{你吃中午饭了没有}\end{exemple}
\begin{exemple}\pjya{pɤjkhu a-saχsɯ pɯ-me}\hspace{5pt}\pcmn{还没有吃中午餐}\end{exemple}\end{entrée}

\begin{entrée}{saχsɯ}{₂}{ⓔsaχsɯⓗ2} 
\classe{n} 
\begin{définition}\pfra{brosse pour laver la vaisselle}\end{définition}
\begin{définition}\pcmn{耍把}\end{définition}\end{entrée}

\begin{entrée}{saχsɯko}{}{ⓔsaχsɯko}\relationsémantique{参考}{\lien{ⓔaχsɯko}{aχsɯko}}\end{entrée}

\begin{entrée}{saχthɯm}{}{ⓔsaχthɯm} 
\classe{vt} \paradigme{dir}{tɤ-}
\begin{définition}\pfra{puiser avec}\end{définition}
\begin{définition}\pcmn{用……舀(水、颗粒)}\end{définition}
\begin{exemple}\pjya{khɯtsa ta-saχthɯm}\hspace{5pt}\pcmn{他用碗舀了}\end{exemple}
\begin{exemple}\pjya{scoʁ ta-saχthɯm}\hspace{5pt}\pcmn{他用瓢子舀了}\end{exemple}\end{entrée}

\begin{entrée}{saχti}{}{ⓔsaχti} 
\classe{vs} 
\begin{définition}\pfra{aimable}\end{définition}
\begin{définition}\pcmn{很好相处,合得来}\end{définition}\relationsémantique{同义词}{\lien{ⓔsɤzda}{sɤzda}}\relationsémantique{参考}{\lien{ⓔtɯ-χti}{tɯ-χti}}\relationsémantique{参考}{\lien{ⓔnaχti}{naχti}}\end{entrée}

\begin{entrée}{sɤβdaʁ}{}{ⓔsɤβdaʁ} 
\classe{n} 
\begin{définition}\pfra{divinité locale}\end{définition}
\begin{définition}\pcmn{地方神}\end{définition}\étymologie{sa.bdag}\end{entrée}

\begin{entrée}{sɤβdɤβde}{}{ⓔsɤβdɤβde}\relationsémantique{参考}{\lien{ⓔaβdɤβde}{aβdɤβde}}\end{entrée}

\begin{entrée}{sɤβlo}{}{ⓔsɤβlo} 
\classe{vt} \paradigme{dir}{nɯ-}
\begin{définition}\pfra{s'occuper de (à propos des enfants)}\end{définition}
\begin{définition}\pcmn{带;看(孩子)}\end{définition}
\begin{exemple}\pjya{aʑo a-ɣe ku-sɤβlam-a}\hspace{5pt}\pcmn{我在带我的孙子}\end{exemple}\relationsémantique{参考}{\lien{ⓔnɤpɤβdaʁ}{nɤpɤβdaʁ}}\end{entrée}

\begin{entrée}{sɤβlɯβlɯɣ}{}{ⓔsɤβlɯβlɯɣ}\relationsémantique{参考}{\lien{ⓔɣɤβlɯβlɯɣ}{ɣɤβlɯβlɯɣ}}\end{entrée}

\begin{entrée}{sɤβri}{}{ⓔsɤβri}\relationsémantique{参考}{\lien{ⓔβri}{βri}}\end{entrée}

\begin{entrée}{sɤβzu}{}{ⓔsɤβzu} 
\classe{vt} \sens{1}\paradigme{dir}{tɤ-}\paradigme{dir}{nɯ-}
\begin{définition}\pfra{préparer, faire une sorte que..., rendre...}\end{définition}
\begin{définition}\pcmn{准备,把...做成,让……变成}\end{définition}
\begin{exemple}\pjya{kɤ-ŋga to-sɤβzu}\hspace{5pt}\pcmn{他把衣服准备好了,可以穿了}\end{exemple}
\begin{exemple}\pjya{kɤ-ndza to-sɤβzu}\hspace{5pt}\pcmn{他把食物准备好了,可以吃了}\end{exemple}
\begin{exemple}\pjya{tɤɕi kɤ-rŋu tɤ-sɤβzu-t-a}\hspace{5pt}\pcmn{我把青稞准备好了,可以炒了}\end{exemple}
\begin{exemple}\pjya{tɤɕi tɯsqar nɯ-sɤβzu-t-a}\hspace{5pt}\pcmn{把青稞作成糌粑了}\end{exemple}
\begin{exemple}\pjya{tɤ-pɤtso kɤ-ntɕhoz kɯ-sna nɯ-sɤβzu-t-a}\hspace{5pt}\pcmn{我把孩子培养成有用的人了}\end{exemple}
\begin{exemple}\pjya{tɤ-mu nɯ kɯ paʁ kɤ-ntɕha to-sɤβzu}\hspace{5pt}\pcmn{那位大娘把猪喂肥了,可以宰了}\end{exemple}
\begin{exemple}\pjya{jiʑora kɯnɤ tu-kɤ-ndza kɯ-me ɲɯ-tɯ-sɤβze ɲɯ-ŋu}\hspace{5pt}\pcmn{你令我们没有东西吃}\end{exemple}
\begin{exemple}\pjya{jiʑora kɤ-nɯʑɯβ mɤ-kɯ-khɯ tu-tɯ-sɤβze ɲɯ-ŋu}\hspace{5pt}\pcmn{你令我们无法睡觉}\end{exemple}\relationsémantique{参考}{\lien{ⓔβzuⓗ1}{βzu₁}}\relationsémantique{参考}{\lien{ⓔaβzu}{aβzu}}\end{entrée}

\begin{entrée}{sɤβzdoʁβzdɯ}{}{ⓔsɤβzdoʁβzdɯ} 
\classe{vt} \paradigme{dir}{tɤ-}
\begin{définition}\pfra{rassembler}\end{définition}
\begin{définition}\pcmn{聚集}\end{définition}
\begin{exemple}\pjya{laχtɕha ra tɤ-sɤβzdoʁβzdɯ-t-a}\hspace{5pt}\pcmn{我把东西聚在一起了}\end{exemple}
\begin{exemple}\pjya{fsapaʁ ra tɤ-sɤβzdoʁβzdɯ-t-a}\hspace{5pt}\pcmn{我把牲畜聚在一起了}\end{exemple}\relationsémantique{参考}{\lien{ⓔaβzdoʁβzdɯ}{aβzdoʁβzdɯ}}\end{entrée}

\begin{entrée}{sɤβzi}{}{ⓔsɤβzi}\relationsémantique{参考}{\lien{ⓔβzi}{βzi}}\end{entrée}

\begin{entrée}{sɤβzɯβzu}{}{ⓔsɤβzɯβzu} 
\classe{vt} \paradigme{dir}{tɤ-}
\begin{définition}\pfra{se débrouiller, broder un peu une histoire}\end{définition}
\begin{définition}\pcmn{想办法;把故事的一些情节编一点}\end{définition}
\begin{exemple}\pjya{tɤ-thu-t-a, ɯ-kɯ-spa maŋe tɕe, aʑo tɤ-nɯsɤβzɯβzu-t-a}\hspace{5pt}\pcmn{我问了,没人知道,我就自己编了一些}\end{exemple}\relationsémantique{参考}{\lien{ⓔsɤβzu}{sɤβzu}}\end{entrée}

\begin{entrée}{sɤcɤrlu}{}{ⓔsɤcɤrlu}\relationsémantique{参考}{\lien{ⓔacɤrlu}{acɤrlu}}\end{entrée}

\begin{entrée}{sɤcha}{}{ⓔsɤcha} 
\classe{vs}  
\grammaire{deexp} \paradigme{dir}{tɤ-}
\begin{définition}\pfra{être possible}\end{définition}
\begin{définition}\pcmn{可能}\end{définition}
\begin{exemple}\pjya{ku-kɯ-ɕe ɲɯ-sɤcha}\hspace{5pt}\pcmn{(这个地方)可以去}\end{exemple}
\begin{exemple}\pjya{kɤ-fkɯr ɲɯ-sɤcha}\hspace{5pt}\pcmn{(这个东西)背得了}\end{exemple}
\begin{exemple}\pjya{zgo ɯ-tɯ-mbro nɯ, kɤ-ɕe mɤ-sɤcha}\hspace{5pt}\pcmn{山高得没人能上去}\end{exemple}
\begin{exemple}\pjya{tɤɕi tɯ-lʁa kɤ-fkur sɤcha}\hspace{5pt}\pcmn{一般来说,一袋青稞,人家背得走}\end{exemple}
\begin{exemple}\pjya{kɯ-ɤntɤm kɤ-nɤrɟɯɣrɟɯɣ sɤcha, tɤton kɤ-rɟɯɣ mɤ-sɤcha}\hspace{5pt}\pcmn{在平路跑步还是受得了,在山坡跑步就受不了了}\end{exemple}
\begin{exemple}\pjya{tú-wɣ-fkur sɤcha}\end{exemple}\relationsémantique{参考}{\lien{ⓔchaⓗ1}{cha₁}}\end{entrée}

\begin{entrée}{sɤchɯchɯβ}{}{ⓔsɤchɯchɯβ} 
\classe{vt} \paradigme{dir}{tɤ-}
\begin{définition}\pfra{foncer sans se préoccuper de rien (cheval), manger à toute vitesse}\end{définition}
\begin{définition}\pcmn{不顾一切地闯过去(马)、吃得很急}\end{définition}
\begin{exemple}\pjya{tɤ-sɤchɯchɯβ ʑo tɕe, jɤ-ɕe}\hspace{5pt}\pcmn{你快点吃就走}\end{exemple}
\begin{exemple}\pjya{tɤ-sɤchɯchɯβ-a lɤ-ari-a}\hspace{5pt}\pcmn{我不顾一切地冲上去了}\end{exemple}\end{entrée}

\begin{entrée}{sɤchɯrʁu}{}{ⓔsɤchɯrʁu}\relationsémantique{参考}{\lien{ⓔachɯrʁu}{achɯrʁu}}\end{entrée}

\begin{entrée}{sɤci}{}{ⓔsɤci}\relationsémantique{参考}{\lien{ⓔaci}{aci}}\end{entrée}

\begin{entrée}{sɤcɯ}{}{ⓔsɤcɯ} 
\classe{n} 
\begin{définition}\pfra{clé}\end{définition}
\begin{définition}\pcmn{钥匙}\end{définition}\relationsémantique{参考}{\lien{ⓔcɯⓗ1}{cɯ₁}}\end{entrée}

\begin{entrée}{sɤcɯqhlɯβ}{}{ⓔsɤcɯqhlɯβ}\relationsémantique{参考}{\lien{ⓔɣɤcɯqhlɯβ}{ɣɤcɯqhlɯβ}}\end{entrée}

\begin{entrée}{sɤɕaβ}{}{ⓔsɤɕaβ}\relationsémantique{参考}{\lien{ⓔɕaβⓗ1}{ɕaβ₁}}\end{entrée}

\begin{entrée}{sɤɕar}{}{ⓔsɤɕar}\relationsémantique{参考}{\lien{ⓔɕar}{ɕar}}\end{entrée}

\begin{entrée}{sɤɕɤt}{}{ⓔsɤɕɤt} 
\classe{vt} \paradigme{dir}{thɯ-}\paradigme{dir}{pɯ-}
\begin{définition}\pfra{peigner}\end{définition}
\begin{définition}\pcmn{梳}\end{définition}
\begin{exemple}\pjya{nɤ-ku thɯ-sɤɕɤt}\hspace{5pt}\pcmn{你梳一下头}\end{exemple}
\begin{exemple}\pjya{ɯ-ku ɲɯ-ɤʁzrɤwolu, pɯ-sɤɕat-a}\hspace{5pt}\pcmn{他头发很乱,所以我给他梳了头}\end{exemple}\relationsémantique{参考}{\lien{ⓔtɤɕɤt}{tɤɕɤt}}\étymologie{ɕad}\end{entrée}

\begin{entrée}{sɤɕkɤɣɕkɤɣ}{}{ⓔsɤɕkɤɣɕkɤɣ}\relationsémantique{参考}{\lien{ⓔɣɤɕkɤɣɕkɤɣ}{ɣɤɕkɤɣɕkɤɣ}}\end{entrée}

\begin{entrée}{sɤɕke}{₁}{ⓔsɤɕkeⓗ1} 
\classe{vs} \paradigme{dir}{tɤ-}
\begin{définition}\pfra{brûlant}\end{définition}
\begin{définition}\pcmn{烫}\end{définition}
\begin{exemple}\pjya{ki kɯ-sɤɕke sthɯci ɲɯ-maʁ}\hspace{5pt}\pcmn{没有那么烫}\end{exemple}
\begin{exemple}\pjya{kɯ-sɤɕke tɤ-ndze !}\hspace{5pt}\pcmn{趁热吃!}\end{exemple}\relationsémantique{参考}{\lien{ⓔʑɣɤsɤɕke}{ʑɣɤsɤɕke}}
\begin{sous-entrée}{nɤsɤɕke}{ⓔsɤɕkeⓗ1ⓝnɤsɤɕke} 
\classe{vt} 
\begin{définition}\pfra{trouver brûlant}\end{définition}
\begin{définition}\pcmn{觉得烫}\end{définition}
\begin{exemple}\pjya{ɲɯ-nɤsɤɕke-a}\hspace{5pt}\pcmn{我觉得很烫}\end{exemple}\end{sous-entrée}

\end{entrée}

\begin{entrée}{sɤɕke}{₂}{ⓔsɤɕkeⓗ2} 
\classe{vt} \paradigme{dir}{kɤ-}
\begin{définition}\pfra{brûler}\end{définition}
\begin{définition}\pcmn{烤焦}\end{définition}
\begin{exemple}\pjya{qajɣi ko-sɤɕke-t-a}\hspace{5pt}\pcmn{我不小心把馍馍烤焦了}\end{exemple}\relationsémantique{参考}{\lien{ⓔɕke}{ɕke}}\end{entrée}

\begin{entrée}{sɤɕoχɕi}{}{ⓔsɤɕoχɕi}\relationsémantique{参考}{\lien{ⓔaɕoχɕi}{aɕoχɕi}}\end{entrée}

\begin{entrée}{sɤɕphɤɣɕphɤɣ}{}{ⓔsɤɕphɤɣɕphɤɣ} 
\classe{vt} 
\begin{définition}\pfra{faire claquer}\end{définition}
\begin{définition}\pcmn{令(衣服)啪啪响}\end{définition}
\begin{exemple}\pjya{qale kɯ tɯ-ŋga ɲɯ-sɤɕphɤɣɕphɤɣ ʑo}\hspace{5pt}\pcmn{风吹,令衣服啪啪地响}\end{exemple}\relationsémantique{参考}{\lien{ⓔɕphɤɣnɤɕphɤɣ}{ɕphɤɣnɤɕphɤɣ}}\end{entrée}

\begin{entrée}{sɤɕprɯm}{}{ⓔsɤɕprɯm}\relationsémantique{参考}{\lien{ⓔaɕprɯm}{aɕprɯm}}\end{entrée}

\begin{entrée}{sɤɕpɯɕpa}{}{ⓔsɤɕpɯɕpa}\relationsémantique{参考}{\lien{ⓔaɕpɯɕpa}{aɕpɯɕpa}}\end{entrée}

\begin{entrée}{sɤɕqa}{}{ⓔsɤɕqa} 
\classe{vs} \paradigme{dir}{tɤ-}
\begin{définition}\pfra{être supportable}\end{définition}
\begin{définition}\pcmn{受得了}\end{définition}
\begin{exemple}\pjya{a-χpɯm ɲɯ-mŋɤm ri, mɤ-kɯ-sɤɕqa maŋe}\hspace{5pt}\pcmn{我的膝盖痛但是没有什么受不了的}\end{exemple}
\begin{exemple}\pjya{ɕɤxɕo rcanɯ, tɯ-mɯ mɯ́j-lɤt tɕe, ɯ-tɯ-sɤɕke kɯ mɯ́j-sɤɕqa ʑo}\hspace{5pt}\pcmn{这几天没有下雨,天气热得使人受不了}\end{exemple}\relationsémantique{参考}{\lien{ⓔnɤɕqa}{nɤɕqa}}\end{entrée}

\begin{entrée}{sɤɕqali}{}{ⓔsɤɕqali}\relationsémantique{参考}{\lien{ⓔɣɤɕqali}{ɣɤɕqali}}\end{entrée}

\begin{entrée}{sɤɕqhe}{}{ⓔsɤɕqhe}\relationsémantique{参考}{\lien{ⓔaɕqhe}{aɕqhe}}\end{entrée}

\begin{entrée}{sɤɕtar}{}{ⓔsɤɕtar}\relationsémantique{参考}{\lien{ⓔnɯɕtar}{nɯɕtar}}\end{entrée}

\begin{entrée}{sɤɕtɕɯɣ}{}{ⓔsɤɕtɕɯɣ} 
\classe{n} 
\begin{définition}\pfra{lanière pour porter les enfants sur le dos}\end{définition}
\begin{définition}\pcmn{背孩子的带子}\end{définition}\end{entrée}

\begin{entrée}{sɤɕte}{}{ⓔsɤɕte}\relationsémantique{参考}{\lien{ⓔɕte}{ɕte}}\end{entrée}

\begin{entrée}{sɤɕtʂaŋlaŋ}{}{ⓔsɤɕtʂaŋlaŋ}\relationsémantique{参考}{\lien{ⓔɕtʂaŋɕtʂaŋ}{ɕtʂaŋɕtʂaŋ}}\end{entrée}

\begin{entrée}{sɤɕtʂɯlɯɣ}{}{ⓔsɤɕtʂɯlɯɣ}\relationsémantique{参考}{\lien{ⓔɕtʂɯɣɕtʂɯɣ}{ɕtʂɯɣɕtʂɯɣ}}\end{entrée}

\begin{entrée}{sɤɕɯmɕɯm}{}{ⓔsɤɕɯmɕɯm}\relationsémantique{参考}{\lien{ⓔɕɯmɕɯm}{ɕɯmɕɯm}}\end{entrée}

\begin{entrée}{sɤdɤmɲi}{}{ⓔsɤdɤmɲi}\relationsémantique{参考}{\lien{ⓔmɲi}{mɲi}}\end{entrée}

\begin{entrée}{sɤdoŋdoŋ}{}{ⓔsɤdoŋdoŋ}\relationsémantique{参考}{\lien{ⓔɣɤdoŋdoŋ}{ɣɤdoŋdoŋ}}\end{entrée}

\begin{entrée}{sɤdrɤt}{}{ⓔsɤdrɤt}\relationsémantique{参考}{\lien{ⓔadrɤt}{adrɤt}}\end{entrée}

\begin{entrée}{sɤdɯxpa}{}{ⓔsɤdɯxpa}\relationsémantique{参考}{\lien{ⓔsɤzdɯxpa}{sɤzdɯxpa}}\end{entrée}

\begin{entrée}{sɤdzɯlɯt}{}{ⓔsɤdzɯlɯt} 
\classe{vt} \paradigme{dir}{nɯ-}\paradigme{dir}{tɤ-}
\begin{définition}\pfra{agiter}\end{définition}
\begin{définition}\pcmn{使扭动}\end{définition}\relationsémantique{参考}{\lien{ⓔɣɤdzɯlɯt}{ɣɤdzɯlɯt}}\end{entrée}

\begin{entrée}{sɤdʑɯɣdʑɯɣ}{}{ⓔsɤdʑɯɣdʑɯɣ}
\begin{sous-entrée}{znɤdʑɯɣdʑɯɣ}{ⓔsɤdʑɯɣdʑɯɣⓝznɤdʑɯɣdʑɯɣ} 
\classe{vi} \paradigme{dir}{tɤ-}
\begin{définition}\pfra{donner un petit coup (avec un bâton)}\end{définition}
\begin{définition}\pcmn{(用棍子)戳一下}\end{définition}
\begin{exemple}\pjya{tɤɲi kɯ ɲɤ-znɤdʑɯɣdʑɯɣ}\end{exemple}
\begin{exemple}\pjya{laχtɕha ko-znɤdʑɯɣdʑɯɣ}\hspace{5pt}\pcmn{他用拐棍戳了那个东西}\end{exemple}\end{sous-entrée}

\end{entrée}

\begin{entrée}{sɤdʐaŋlaŋ}{}{ⓔsɤdʐaŋlaŋ} 
\classe{vt} \paradigme{dir}{tɤ-}
\begin{définition}\pfra{balancer}\end{définition}
\begin{définition}\pcmn{甩来甩去}\end{définition}
\begin{exemple}\pjya{tɤ-sɤdʐaŋlaŋ}\hspace{5pt}\pcmn{他(把这个东西)甩来甩去了}\end{exemple}\relationsémantique{同义词}{\lien{ⓔɕtʂaŋɕtʂaŋⓝsɤɕtʂaŋlaŋ}{sɤɕtʂaŋlaŋ}}\end{entrée}

\begin{entrée}{sɤfɕu}{}{ⓔsɤfɕu}\relationsémantique{参考}{\lien{ⓔafɕu}{afɕu}}\end{entrée}

\begin{entrée}{sɤfɕɤra}{}{ⓔsɤfɕɤra} 
\classe{vt} \sens{1}\paradigme{dir}{tɤ-}
\begin{définition}\pfra{discuter}\end{définition}
\begin{définition}\pcmn{讨论;议论}\end{définition}
\begin{exemple}\pjya{ɲɤ-sɤfɕɤra-nɯ}\hspace{5pt}\pcmn{他们商量了}\end{exemple}
\begin{exemple}\pjya{tɤ-sɤfɕɤra-j}\hspace{5pt}\pcmn{我们商量了}\end{exemple}
\begin{exemple}\pjya{kɤ-nɤma tɕhi tu-kɤ-stu nɯra tɕiʑo tɤ-sɤfɕɤra-tɕi}\hspace{5pt}\pcmn{我们商量了怎么做}\end{exemple}
\begin{exemple}\pjya{jɤ-ɣi-nɯ tɕe sɤfɕɤra-j}\hspace{5pt}\pcmn{你们来,我们讨论一下}\end{exemple}\sens{2}\paradigme{dir}{nɯ-}
\begin{définition}\pfra{dévoiler (une information non publique)}\end{définition}
\begin{définition}\pcmn{传出去}\end{définition}
\begin{exemple}\pjya{ma-nɯ́-wɣ-sɤfɕɤra}\hspace{5pt}\pcmn{不要公开}\end{exemple}\end{entrée}

\begin{entrée}{sɤfɕi}{}{ⓔsɤfɕi} 
\classe{vi} \paradigme{dir}{nɯ-}
\begin{définition}\pfra{être enceinte (après le septième mois de grossesse)}\end{définition}
\begin{définition}\pcmn{怀孕(快要生;怀孕七月以后)}\end{définition}
\begin{exemple}\pjya{jiɕqha tɕheme ɲɤ-sɤfɕi, ɯ-xtu jɤrjɤr ɲɤ-pa}\hspace{5pt}\pcmn{那个女子怀孕了,肚子变得很沉}\end{exemple}\end{entrée}

\begin{entrée}{sɤfka}{}{ⓔsɤfka}\relationsémantique{参考}{\lien{ⓔfkaⓗ2}{fka}}\end{entrée}

\begin{entrée}{sɤfsu}{}{ⓔsɤfsu} 
\classe{vt} \sens{1}\paradigme{dir}{tɤ-}
\begin{définition}\pfra{comparer la taille}\end{définition}
\begin{définition}\pcmn{比较长短}\end{définition}
\begin{exemple}\pjya{a-jaʁndzu tɤ-sɤfsu-t-a ri mɯ-ɲɯ-ɤfsɯfsu}\hspace{5pt}\pcmn{我比了一下手指的长短,结果长短不一}\end{exemple}
\begin{exemple}\pjya{xtɯrɲɟi ta-sɤfsu}\hspace{5pt}\pcmn{他比了一下长短}\end{exemple}
\begin{exemple}\pjya{tɕi-mthɯxtɕɤr tɤ-sɤfsu-t-a}\hspace{5pt}\pcmn{我比了我们俩的腰带有多长}\end{exemple}\sens{2}\paradigme{dir}{nɯ-}
\begin{définition}\pfra{uniformiser la taille}\end{définition}
\begin{définition}\pcmn{弄得一样长}\end{définition}
\begin{exemple}\pjya{χtsiɯ kɯ tɯjpu tɤ́-wɣ-sɯ-ɕtʂo tɕe, ɯ-mŋu nɯ raŋri ʑo ɲɯ́-wɣ-sɤfsu ra}\hspace{5pt}\pcmn{用瓢子量粮食的时候,要把每一个瓢的口抹平(确定每一瓢相等)}\end{exemple}\relationsémantique{参考}{\lien{ⓔɯ-fsu}{ɯ-fsu}}\relationsémantique{参考}{\lien{ⓔsɤfsuja}{sɤfsuja}}\relationsémantique{参考}{\lien{ⓔafsɯfsu}{afsɯfsu}}\end{entrée}

\begin{entrée}{sɤfse}{}{ⓔsɤfse} 
\classe{adv} 
\begin{définition}\pfra{comme}\end{définition}
\begin{définition}\pcmn{很像}\end{définition}
\begin{exemple}\pjya{nɤʑo tɤ-pɤtso sɤfse kɯ kɯ-chi ɲɯ-tɯ-rga}\hspace{5pt}\pcmn{你很像小孩子,喜欢吃糖}\end{exemple}
\begin{exemple}\pjya{nɤʑo aʑo sɤfse kɯ tɤndʐo mɯ́j-tɯ-schi}\hspace{5pt}\pcmn{你很像我,很怕冷}\end{exemple}\relationsémantique{参考}{\lien{ⓔfseⓗ1}{fse₁}}\end{entrée}

\begin{entrée}{sɤfsuja}{}{ⓔsɤfsuja} 
\classe{vt} \paradigme{dir}{tɤ-}
\begin{définition}\pfra{comparer la taille}\end{définition}
\begin{définition}\pcmn{比较长短}\end{définition}
\begin{exemple}\pjya{tɕi-mthɯxtɕɤr tɤ-sɤfsuja-t-a}\hspace{5pt}\pcmn{我比了我们俩的腰带有多长}\end{exemple}
\begin{exemple}\pjya{ki tɯmbri ni ŋotɕu nɯ ɲɯ-rɲɟi kɯ tu-sɤfsuje-a}\hspace{5pt}\pcmn{我把这两根绳子比较一下,看哪一根长}\end{exemple}\relationsémantique{同义词}{\lien{ⓔsɤfsu}{sɤfsu}}\end{entrée}

\begin{entrée}{sɤfstɯn}{}{ⓔsɤfstɯn}\relationsémantique{参考}{\lien{ⓔfstɯn}{fstɯn}}\end{entrée}

\begin{entrée}{sɤfsɯfse}{}{ⓔsɤfsɯfse}\relationsémantique{参考}{\lien{ⓔfseⓗ2}{fse₂}}\end{entrée}

\begin{entrée}{sɤftɕaka}{}{ⓔsɤftɕaka} 
\classe{vt} \paradigme{dir}{tɤ-}
\begin{définition}\pfra{préparer}\end{définition}
\begin{définition}\pcmn{准备;收拾}\end{définition}
\begin{exemple}\pjya{to-sɤftɕaka}\hspace{5pt}\pcmn{他准备了东西}\end{exemple}
\begin{exemple}\pjya{laχtɕha tɤ-sɤftɕaka-t-a}\hspace{5pt}\pcmn{我准备了东西}\end{exemple}
\begin{exemple}\pjya{kɤ-ndza tɤ-sɤftɕaka-t-a}\hspace{5pt}\pcmn{我准备了食物}\end{exemple}
\begin{exemple}\pjya{tʂha lɤ-sɤftɕaka-t-a}\hspace{5pt}\pcmn{我准备了茶}\end{exemple}\relationsémantique{参考}{\lien{ⓔftɕaka}{ftɕaka}}\end{entrée}

\begin{entrée}{sɤftɕaʁ}{}{ⓔsɤftɕaʁ}\relationsémantique{参考}{\lien{ⓔftɕaʁ}{ftɕaʁ}}\end{entrée}

\begin{entrée}{sɤftɕɤl}{}{ⓔsɤftɕɤl}\relationsémantique{参考}{\lien{ⓔftɕɤl}{ftɕɤl}}\end{entrée}

\begin{entrée}{sɤglɤglɤɣ}{}{ⓔsɤglɤglɤɣ} 
\classe{vi} \paradigme{dir}{tɤ-}
\begin{définition}\pfra{frapper en faisant du bruit}\end{définition}
\begin{définition}\pcmn{敲得很响;震动}\end{définition}\relationsémantique{参考}{\lien{ⓔɣɤglɤglɤɣ}{ɣɤglɤglɤɣ}}\relationsémantique{参考}{\lien{ⓔglɤɣglɤɣ}{glɤɣglɤɣ}}\end{entrée}

\begin{entrée}{sɤgrɤl}{}{ⓔsɤgrɤl} 
\classe{n} 
\begin{définition}\pfra{limite}\end{définition}
\begin{définition}\pcmn{界限}\end{définition}\étymologie{gral}\end{entrée}

\begin{entrée}{sɤɣa}{}{ⓔsɤɣa} 
\classe{vs} \paradigme{dir}{tɤ-}
\begin{définition}\pfra{endroit, chemin non dangereux}\end{définition}
\begin{définition}\pcmn{安全;平坦的路;地方}\end{définition}
\begin{exemple}\pjya{tʂu ɲɯ-sɤɣa}\hspace{5pt}\pcmn{路很安全}\end{exemple}
\begin{exemple}\pjya{stɤmku ɲɯ-sɤɣa}\hspace{5pt}\pcmn{草坪很平坦}\end{exemple}\relationsémantique{参考}{\lien{ⓔnɤɣa}{nɤɣa}}\end{entrée}

\begin{entrée}{sɤɣɤmɯ}{}{ⓔsɤɣɤmɯ}\relationsémantique{参考}{\lien{ⓔɣɤmɯ}{ɣɤmɯ}}\end{entrée}

\begin{entrée}{sɤɣdoŋɣdoŋ}{}{ⓔsɤɣdoŋɣdoŋ}\relationsémantique{参考}{\lien{ⓔɣdoŋnɤɣdoŋ}{ɣdoŋnɤɣdoŋ}}\end{entrée}

\begin{entrée}{sɤɣdɯɣ}{}{ⓔsɤɣdɯɣ} 
\classe{vs} \paradigme{dir}{tɤ-}
\begin{définition}\pfra{être désagréable}\end{définition}
\begin{définition}\pcmn{讨厌; 不舒服}\end{définition}
\begin{exemple}\pjya{a-mgɯr ɯ-qhu ma-tɯ-ɣɤjɤβjɤβ ɲɯ-tɯ-sɤɣdɯɣ}\hspace{5pt}\pcmn{你不要乱摸我的背部,你很讨厌}\end{exemple}
\begin{exemple}\pjya{a-phe ma-tɯ-ɣɤsɯɣsɯɣ, ɲɯ-sɤɣdɯɣ}\hspace{5pt}\pcmn{你不要在我身边乱动,很讨厌}\end{exemple}
\begin{exemple}\pjya{ɲɯ-nɯtɕhomba-a, ɲɯ-sɤɣdɯɣ}\hspace{5pt}\pcmn{我感冒了,很不舒服}\end{exemple}
\begin{exemple}\pjya{a-mgɯr ɯ-qhu ɲɯ-sɤɣdɯɣ}\hspace{5pt}\pcmn{我背上不舒服}\end{exemple}
\begin{exemple}\pjya{a-sɯm ɲɯ-sɤɣdɯɣ}\hspace{5pt}\pcmn{我心里不舒服}\end{exemple}
\begin{exemple}\pjya{a-qhoχpa ɲɯ-sɤɣdɯɣ}\hspace{5pt}\pcmn{我心里不舒服}\end{exemple}
\begin{exemple}\pjya{a-xtu ɲɯ-sɤɣdɯɣ}\hspace{5pt}\pcmn{我肚子不舒服}\end{exemple}\relationsémantique{参考}{\lien{ⓔnɤsɤɣdɯɣ}{nɤsɤɣdɯɣ}}\end{entrée}

\begin{entrée}{sɤɣmu}{}{ⓔsɤɣmu} 
\classe{vs}  
\grammaire{deexp} \paradigme{dir}{tɤ-}\paradigme{dir}{thɯ-}
\begin{définition}\pfra{terrifiant}\end{définition}
\begin{définition}\pcmn{恐怖;可怕}\end{définition}
\begin{exemple}\pjya{jiɕqha kɯ-sɤɣmu ci ɲɯ-ŋu}\hspace{5pt}\pcmn{那个很恐怖}\end{exemple}
\begin{exemple}\pjya{jla ɲɯ-sɤtɕhɯ tɕe ɲɯ-sɤɣmu}\hspace{5pt}\pcmn{犏牛顶人很可怕}\end{exemple}
\begin{exemple}\pjya{khɯna ɲɯ-sɤmtsɯɣ tɕe ɲɯ-sɤɣmu}\hspace{5pt}\pcmn{狗咬人很可怕}\end{exemple}
\begin{exemple}\pjya{cho-sɤɣmu}\hspace{5pt}\pcmn{变得很恐怖}\end{exemple}\relationsémantique{参考}{\lien{ⓔmuⓗ1}{mu₁}}\end{entrée}

\begin{entrée}{sɤɣɲat}{}{ⓔsɤɣɲat} 
\classe{vs}  
\grammaire{deexp} 
\begin{définition}\pfra{fatigant}\end{définition}
\begin{définition}\pcmn{令人很累}\end{définition}
\begin{exemple}\pjya{ki kɤ-nɤma ki wuma ɲɯ-sɤɣɲat}\hspace{5pt}\pcmn{这种工作很累人}\end{exemple}\relationsémantique{参考}{\lien{ⓔɲat}{ɲat}}\end{entrée}

\begin{entrée}{sɤɣur}{}{ⓔsɤɣur} 
\classe{vt} \paradigme{dir}{tɤ-}\paradigme{dir}{kɤ-}\paradigme{dir}{pɯ-}
\begin{définition}\pfra{couvrir de tous les côtés (mais pas le dessus), bloquer}\end{définition}
\begin{définition}\pcmn{遮拦四周(但没有遮住上面)、挡住去路}\end{définition}
\begin{exemple}\pjya{fsapaʁ tɤ-sɤɣur-a (=kɤ-ja-t-a)}\hspace{5pt}\pcmn{我把牲畜围起来了}\end{exemple}
\begin{exemple}\pjya{sɤxɕe me ma kɤ-sɤɣur-a}\hspace{5pt}\pcmn{没有去路,因为被我挡住了}\end{exemple}
\begin{exemple}\pjya{kumpɣa ɕɯ-nɤru ɲɯ-ŋu tɕe, (tɯjpu) tɤ-sɤɣur-a}\hspace{5pt}\pcmn{鸡要去偷吃粮食,我就把(粮食)围起来了}\end{exemple}
\begin{exemple}\pjya{rdɤstaʁ kɯ smi kɤ-sɤɣur-i}\hspace{5pt}\pcmn{我用石头把火围起来了}\end{exemple}\relationsémantique{参考}{\lien{ⓔtɤ-ɣur}{tɤ-ɣur}}\end{entrée}

\begin{entrée}{sɤɣɯrɣɯr}{}{ⓔsɤɣɯrɣɯr}\relationsémantique{参考}{\lien{ⓔɣɯrɣɯrⓝɣɤɣɯrɣɯr}{ɣɤɣɯrɣɯr}}\end{entrée}

\begin{entrée}{sɤɣʑɯr}{}{ⓔsɤɣʑɯr} 
\classe{vs} \paradigme{dir}{tɤ-}
\begin{définition}\pfra{dangereux}\end{définition}
\begin{définition}\pcmn{危险}\end{définition}\end{entrée}

\begin{entrée}{sɤja}{}{ⓔsɤja} 
\classe{vt} \paradigme{dir}{nɯ-}
\begin{définition}\pfra{rendre}\end{définition}
\begin{définition}\pcmn{还东西}\end{définition}
\begin{exemple}\pjya{laʁtɕha ɲɤ-nɯβde-t-a tɕe nɯ́-wɣ-sɤja-a}\hspace{5pt}\pcmn{我把东西弄丢了,他送还给我了}\end{exemple}
\begin{exemple}\pjya{ɲɤ-nɯβde tɕe nɯ-sɤja-t-a}\hspace{5pt}\pcmn{他把东西弄丢了,我送还给他了}\end{exemple}
\begin{exemple}\pjya{nɤ-laʁtɕha nɯ-ta-sɤja}\hspace{5pt}\pcmn{我已经把你的东西还给你了}\end{exemple}\end{entrée}

\begin{entrée}{sɤjɤr}{}{ⓔsɤjɤr}\relationsémantique{参考}{\lien{ⓔajɤr}{ajɤr}}\end{entrée}

\begin{entrée}{sɤjɤrjɤr}{}{ⓔsɤjɤrjɤr}\relationsémantique{参考}{\lien{ⓔjɤrjɤr}{jɤrjɤr}}\end{entrée}

\begin{entrée}{sɤjɤzjɯ}{}{ⓔsɤjɤzjɯ}\relationsémantique{参考}{\lien{ⓔajɤzjɯ}{ajɤzjɯ}}\end{entrée}

\begin{entrée}{sɤjku}{}{ⓔsɤjku} 
\classe{n} 
\begin{définition}\pfra{bouleau}\end{définition}
\begin{définition}\pcmn{白桦树}\end{définition}
\begin{exemple}\pjya{sɤjku nɯ ɯ-jwaʁ mbraj cho naχtɕɯɣ, ɯ-rqhu nɯ wuma ʑo jaʁ cho ngɯt tɕe ɲchɣaʁ rmi, kɯ-wɣrum ŋu, si wuma ʑo mbro, ɯ-si nɯ ngɯt}\hspace{5pt}\pcmn{白桦树叶子长得和红桦树一样,树皮又厚又结实,叫\lien{ⓔɲchɣaʁ}{ɲchɣaʁ},树皮的外层是白色的,是高大的树种,木质很结实。}\end{exemple}\end{entrée}

\begin{entrée}{sɤjlɤβ}{}{ⓔsɤjlɤβ} 
\classe{n} 
\begin{définition}\pfra{la vapeur du sol}\end{définition}
\begin{définition}\pcmn{土地上的蒸汽}\end{définition}\relationsémantique{参考}{\lien{ⓔtɤjlɤβ}{tɤjlɤβ}}\end{entrée}

\begin{entrée}{sɤjloʁ}{}{ⓔsɤjloʁ} 
\classe{vs} \paradigme{dir}{nɯ-}\sens{1}
\begin{définition}\pfra{laid}\end{définition}
\begin{définition}\pcmn{丑陋;难看}\end{définition}\sens{2}
\begin{définition}\pfra{dégoutant}\end{définition}
\begin{définition}\pcmn{难吃}\end{définition}
\begin{sous-entrée}{nɤsɤjloʁ}{ⓔsɤjloʁⓢ2ⓝnɤsɤjloʁ} 
\classe{vt}  
\grammaire{trop} 
\begin{définition}\pfra{trouver laid, trouver dégoutant}\end{définition}
\begin{définition}\pcmn{觉得难看,觉得难吃}\end{définition}
\begin{exemple}\pjya{nɤki ɯ-mdoʁ nɯ ɲɯ-nɤsɤjloʁ-a}\hspace{5pt}\pcmn{我觉得这个颜色很难看}\end{exemple}\end{sous-entrée}

\end{entrée}

\begin{entrée}{sɤjndɤt}{}{ⓔsɤjndɤt} 
\classe{vs} \paradigme{dir}{thɯ-}
\begin{définition}\pfra{mignon, sage (enfant)}\end{définition}
\begin{définition}\pcmn{可爱;乖}\end{définition}
\begin{exemple}\pjya{laχtɕha ɲɯ-sɤjndɤt}\hspace{5pt}\pcmn{东西很可爱}\end{exemple}
\begin{exemple}\pjya{tɕheme ɲɯ-sɤjndɤt}\hspace{5pt}\pcmn{女孩子很可爱}\end{exemple}
\begin{exemple}\pjya{tɯrme ɲɯ-sɤjndɤt}\hspace{5pt}\pcmn{人很可爱}\end{exemple}\relationsémantique{参考}{\lien{ⓔnɤjndɤt}{nɤjndɤt}}
\begin{sous-entrée}{nɤsɤjndɤt}{ⓔsɤjndɤtⓝnɤsɤjndɤt} 
\classe{vt}  
\grammaire{trop} 
\begin{définition}\pfra{trouver mignon}\end{définition}
\begin{définition}\pcmn{觉得可爱}\end{définition}\relationsémantique{同义词}{\lien{ⓔnɤjndɤt}{nɤjndɤt}}\end{sous-entrée}

\end{entrée}

\begin{entrée}{sɤjoʁjoʁ}{}{ⓔsɤjoʁjoʁ} 
\classe{vt} \paradigme{dir}{tɤ-}
\begin{définition}\pfra{lever un peu, ranger}\end{définition}
\begin{définition}\pcmn{弄高一点,收拾东西}\end{définition}
\begin{exemple}\pjya{si tɤ-sɤjoʁjoʁ-a}\hspace{5pt}\pcmn{我把木料弄上去一点了}\end{exemple}
\begin{exemple}\pjya{thɤfka ɯ-ŋgɯ si nɯ ra tɤ-sɤjoʁjoʁ tɕe nɯt}\hspace{5pt}\pcmn{你把炉子里的柴弄上去一点就会燃烧}\end{exemple}\relationsémantique{参考}{\lien{ⓔjoʁ}{joʁ}}\end{entrée}

\begin{entrée}{sɤjqu}{}{ⓔsɤjqu}\relationsémantique{参考}{\lien{ⓔjqu}{jqu}}\end{entrée}

\begin{entrée}{sɤjʁu}{}{ⓔsɤjʁu}\relationsémantique{参考}{\lien{ⓔajʁu}{ajʁu}}\end{entrée}

\begin{entrée}{sɤjtshi}{}{ⓔsɤjtshi}\relationsémantique{参考}{\lien{ⓔjtshi}{jtshi}}\end{entrée}

\begin{entrée}{sɤjtɯ}{}{ⓔsɤjtɯ}\relationsémantique{参考}{\lien{ⓔajtɯ}{ajtɯ}}\end{entrée}

\begin{entrée}{sɤjwɤrlɤr}{}{ⓔsɤjwɤrlɤr}\relationsémantique{参考}{\lien{ⓔɣɤjwɤrlɤr}{ɣɤjwɤrlɤr}}\end{entrée}

\begin{entrée}{sɤɟɯɣlɯɣ}{}{ⓔsɤɟɯɣlɯɣ} 
\classe{vt} \paradigme{dir}{tɤ-}
\begin{définition}\pfra{hausser (les épaules)}\end{définition}
\begin{définition}\pcmn{耸(肩)}\end{définition}
\begin{exemple}\pjya{ɯ-rpaʁ ra to-sɤɟɯɣlɯɣ}\hspace{5pt}\pcmn{他耸了一下肩}\end{exemple}\end{entrée}

\begin{entrée}{sɤɟɯɟrɯɣ}{}{ⓔsɤɟɯɟrɯɣ}\relationsémantique{参考}{\lien{ⓔɣɤɟɯɟrɯɣ}{ɣɤɟɯɟrɯɣ}}\end{entrée}

\begin{entrée}{sɤkɤβjɤβ}{}{ⓔsɤkɤβjɤβ}\relationsémantique{参考}{\lien{ⓔɣɤkɤβjɤβ}{ɣɤkɤβjɤβ}}\end{entrée}

\begin{entrée}{sɤkɤlɤt}{}{ⓔsɤkɤlɤt}\relationsémantique{参考}{\lien{ⓔakɤlɤt}{akɤlɤt}}\end{entrée}

\begin{entrée}{sɤkɤsci}{}{ⓔsɤkɤsci} 
\classe{vt} \paradigme{dir}{tɤ-}\paradigme{dir}{pɯ-}\paradigme{dir}{thɯ-}
\begin{définition}\pfra{changer}\end{définition}
\begin{définition}\pcmn{换;调换}\end{définition}
\begin{exemple}\pjya{tɕi-ŋga tɤ-nɯ-sɤkɤsci-tɕi}\hspace{5pt}\pcmn{我们俩调换了衣服}\end{exemple}
\begin{exemple}\pjya{ɯ-ŋga tshɯrɟɯn chɯ-sɤkɤsci ŋu}\hspace{5pt}\pcmn{他经常换衣服}\end{exemple}\relationsémantique{同义词}{\lien{ⓔsɤscɯndu}{sɤscɯndu}}\relationsémantique{参考}{\lien{ⓔnɤsci}{nɤsci}}\end{entrée}

\begin{entrée}{sɤkɤtɕɤβ}{}{ⓔsɤkɤtɕɤβ}\relationsémantique{参考}{\lien{ⓔakɤtɕɤβ}{akɤtɕɤβ}}\end{entrée}

\begin{entrée}{sɤkhar}{}{ⓔsɤkhar} 
\classe{vt} \paradigme{dir}{kɤ-}\paradigme{dir}{thɯ-}
\begin{définition}\pfra{enfermer}\end{définition}
\begin{définition}\pcmn{用……围起来}\end{définition}
\begin{exemple}\pjya{si kɤ-sɤkhar-i}\hspace{5pt}\pcmn{我们用树叉围成圆圈了}\end{exemple}\relationsémantique{参考}{\lien{ⓔakhar}{akhar}}\relationsémantique{参考}{\lien{ⓔnɤkhar}{nɤkhar}}\étymologie{ⁿkʰor}\end{entrée}

\begin{entrée}{sɤkhra}{}{ⓔsɤkhra} 
\classe{vt} \paradigme{dir}{pɯ-}\paradigme{dir}{kɤ-}
\begin{définition}\pfra{colorer de toutes sortes de couleurs}\end{définition}
\begin{définition}\pcmn{配好各种颜色}\end{définition}
\begin{exemple}\pjya{pɯ-sɤkhra-t-a}\hspace{5pt}\pcmn{我画了很多颜色}\end{exemple}\relationsémantique{参考}{\lien{ⓔakhra}{akhra}}\relationsémantique{参考}{\lien{ⓔkhra}{khra}}\end{entrée}

\begin{entrée}{sɤkhɯ}{}{ⓔsɤkhɯ} 
\classe{vt} \paradigme{dir}{tɤ-}
\begin{définition}\pfra{fumer}\end{définition}
\begin{définition}\pcmn{熏}\end{définition}
\begin{exemple}\pjya{rgɯnba kɤ-sɤkhɯ tɤ-tsɯm}\hspace{5pt}\pcmn{你把求烟的东送上去}\end{exemple}
\begin{exemple}\pjya{rgɯnba kɤ-sɤkhɯ ɯ-spa tɤ-tsɯm-a}\hspace{5pt}\pcmn{我把供神求烟的东西送上去里了}\end{exemple}
\begin{exemple}\pjya{tɤ-mthɯm tɤ-sɤkhɯ-t-a}\hspace{5pt}\pcmn{我熏了肉}\end{exemple}\relationsémantique{参考}{\lien{ⓔtɤ-khɯ}{tɤ-khɯ}}\relationsémantique{参考}{\lien{ⓔɣɤkhɯⓗ1}{ɣɤkhɯ₁}}\relationsémantique{参考}{\lien{ⓔnɤkhɯ}{nɤkhɯ}}\end{entrée}

\begin{entrée}{sɤkhɯkhɯɣ}{}{ⓔsɤkhɯkhɯɣ} 
\classe{vt} \paradigme{dir}{kɤ-}
\begin{définition}\pfra{boire à toute vitesse}\end{définition}
\begin{définition}\pcmn{急着喝}\end{définition}
\begin{exemple}\pjya{tɯ-ci ka-sɤkhɯkhɯɣ}\hspace{5pt}\pcmn{他急着喝了水}\end{exemple}
\begin{exemple}\pjya{cha ka-sɤkhɯkhɯɣ}\hspace{5pt}\pcmn{他急着喝了酒}\end{exemple}
\begin{exemple}\pjya{tɯcɯrqɯ ka-sɤkhɯkhɯɣ}\hspace{5pt}\pcmn{他急着喝了冷水}\end{exemple}
\begin{exemple}\pjya{ma-kɤ-tɯ-sɤkhɯkhɯɣ}\hspace{5pt}\pcmn{你不要急着喝}\end{exemple}\end{entrée}

\begin{entrée}{sɤla}{}{ⓔsɤla}\relationsémantique{参考}{\lien{ⓔala}{ala}}\end{entrée}

\begin{entrée}{sɤlaŋphɤn}{}{ⓔsɤlaŋphɤn} 
\classe{n} 
\begin{définition}\pfra{bassine}\end{définition}
\begin{définition}\pcmn{盆子}\end{définition}\étymologie{fn:洗脸盆}\end{entrée}

\begin{entrée}{sɤlɤɣɯ}{}{ⓔsɤlɤɣɯ} 
\classe{vt} \paradigme{dir}{nɯ-}\paradigme{dir}{kɤ-}
\begin{définition}\pfra{connecter, rattacher}\end{définition}
\begin{définition}\pcmn{连接(断了的东西)}\end{définition}
\begin{exemple}\pjya{tɤ-fsɤri ka-sɤlɤɣɯ}\hspace{5pt}\pcmn{他把麻绳接上了}\end{exemple}
\begin{exemple}\pjya{tɯ-mbri ka-sɤlɤɣɯ}\hspace{5pt}\pcmn{他把绳子接上了}\end{exemple}\relationsémantique{同义词}{\lien{ⓔamthoʁmthɯtⓝsɤmthoʁmthɯt}{sɤmthoʁmthɯt}}\relationsémantique{参考}{\lien{ⓔalɤɣɯ}{alɤɣɯ}}
\begin{sous-entrée}{nɯɣɯsɤlɤɣɯ}{ⓔsɤlɤɣɯⓝnɯɣɯsɤlɤɣɯ} 
\classe{vs} 
\begin{définition}\pfra{difficile à connecter (parole)}\end{définition}
\begin{définition}\pcmn{不容易连贯}\end{définition}
\begin{exemple}\pjya{ɯ-rju mɯ́j-nɯɣɯsɤlɤɣɯ}\hspace{5pt}\pcmn{他的话不容易连起来}\end{exemple}\end{sous-entrée}

\end{entrée}

\begin{entrée}{sɤlɤt}{}{ⓔsɤlɤt} 
\classe{vi} \paradigme{dir}{\_}
\begin{définition}\pfra{raccompagner}\end{définition}
\begin{définition}\pcmn{送行}\end{définition}
\begin{exemple}\pjya{kɯ-sɤlɤt jɤ-ari}\hspace{5pt}\pcmn{他去送行了}\end{exemple}
\begin{exemple}\pjya{ɕ-kɤ-sɤlɤt, ʑ-nɯ-sɤlɤt}\hspace{5pt}\pcmn{他去送行了}\end{exemple}\relationsémantique{同义词}{\lien{ⓔsɤsco}{sɤsco}}\end{entrée}

\begin{entrée}{sɤljɤljɤt}{}{ⓔsɤljɤljɤt} 
\classe{vt} \paradigme{dir}{nɯ-}
\begin{définition}\pfra{remuer (queue)}\end{définition}
\begin{définition}\pcmn{摇(尾巴)}\end{définition}
\begin{exemple}\pjya{khɯna kɯ ɯ-jme ɲɯ-ɤsɯ-sɤljɤljɤt}\hspace{5pt}\pcmn{狗在摇尾巴}\end{exemple}\end{entrée}

\begin{entrée}{sɤlothi}{}{ⓔsɤlothi}\relationsémantique{参考}{\lien{ⓔalothi}{alothi}}\end{entrée}

\begin{entrée}{sɤlpɯm}{}{ⓔsɤlpɯm}\relationsémantique{参考}{\lien{ⓔalpɯm}{alpɯm}}\end{entrée}

\begin{entrée}{sɤlqɤlqɤt}{}{ⓔsɤlqɤlqɤt} 
\classe{vt} \paradigme{dir}{tɤ-}
\begin{définition}\pfra{agiter légèrement (des ailes)}\end{définition}
\begin{définition}\pcmn{轻轻地扇动(翅膀)}\end{définition}\end{entrée}

\begin{entrée}{sɤltɕhɤltɕhɤt}{}{ⓔsɤltɕhɤltɕhɤt}\relationsémantique{参考}{\lien{ⓔltɕhɤltɕhɤt}{ltɕhɤltɕhɤt}}\end{entrée}

\begin{entrée}{sɤltɕhɯɣlɯɣ}{}{ⓔsɤltɕhɯɣlɯɣ}\relationsémantique{参考}{\lien{ⓔltɕhɯɣltɕhɯɣ}{ltɕhɯɣltɕhɯɣ}}\end{entrée}

\begin{entrée}{sɤlthɤlthɤβ}{}{ⓔsɤlthɤlthɤβ} 
\classe{vt} \paradigme{dir}{pɯ-}
\begin{définition}\pfra{cligner de l'œil}\end{définition}
\begin{définition}\pcmn{眨眼}\end{définition}\relationsémantique{参考}{\lien{}{thɤβ₁}}\end{entrée}

\begin{entrée}{sɤltshɤltshɤt}{}{ⓔsɤltshɤltshɤt} 
\classe{vt} \paradigme{dir}{nɯ-}
\begin{définition}\pfra{agiter légèrement}\end{définition}
\begin{définition}\pcmn{轻轻地摇动(有毛、有絮絮的东西)}\end{définition}
\begin{exemple}\pjya{ɲɯ-sɤltshɤltshɤt}\hspace{5pt}\pcmn{他在摇动}\end{exemple}
\begin{exemple}\pjya{ɯʑo kɯ na-sɤltshɤltshɤt}\hspace{5pt}\pcmn{他摇动了}\end{exemple}
\begin{exemple}\pjya{si nɯ ɲɯ-sɤltshɤltshɤt}\hspace{5pt}\pcmn{他在摇树}\end{exemple}
\begin{exemple}\pjya{rɟɤskhi ɲɯ-sɤltshɤltshɤt}\hspace{5pt}\pcmn{他在抖动簸箕}\end{exemple}
\begin{exemple}\pjya{ɲɯ-saʁlɤt tɕe ɲɯ-sɤltshɤltshɤt ntsɯ}\hspace{5pt}\pcmn{他扬粮食,把(簸箕)抖来抖去}\end{exemple}
\begin{sous-entrée}{ɣɤltshɤltshɤt}{ⓔsɤltshɤltshɤtⓝɣɤltshɤltshɤt} 
\classe{vi} 
\begin{définition}\pfra{trembler, se secouer}\end{définition}
\begin{définition}\pcmn{发抖}\end{définition}
\begin{exemple}\pjya{ɯ-tɯ-nɤndʐo kɯ ɲɯ-ɣɤltshɤltshɤt ʑo}\hspace{5pt}\pcmn{小孩子冷得发抖}\end{exemple}\end{sous-entrée}

\end{entrée}

\begin{entrée}{sɤltshɯltshɯɣ}{}{ⓔsɤltshɯltshɯɣ} 
\classe{vt} \paradigme{dir}{nɯ-}
\begin{définition}\pfra{bercer}\end{définition}
\begin{définition}\pcmn{摇(小孩子)}\end{définition}
\begin{exemple}\pjya{nɯ-sɤltshɯltshɯɣ-a tɕe tɕe ɲɯ-rga tɕe mɯ́j-ɣɤwu}\hspace{5pt}\pcmn{我把他摇了一下,他高兴就不哭了}\end{exemple}\end{entrée}

\begin{entrée}{sɤlɯrlɯr}{}{ⓔsɤlɯrlɯr}\relationsémantique{参考}{\lien{ⓔlɯrlɯr}{lɯrlɯr}}\end{entrée}

\begin{entrée}{sɤlɯzlɯz}{}{ⓔsɤlɯzlɯz}\relationsémantique{参考}{\lien{ⓔɣɤlɯzlɯz}{ɣɤlɯzlɯz}}\end{entrée}

\begin{entrée}{sɤlwɤlwɤt}{}{ⓔsɤlwɤlwɤt} 
\classe{vt} \paradigme{dir}{tɤ-}\paradigme{dir}{nɯ-}
\begin{définition}\pfra{agiter}\end{définition}
\begin{définition}\pcmn{挥动}\end{définition}
\begin{exemple}\pjya{ɯ-jaʁ ta-sɤlwɤlwɤt}\hspace{5pt}\pcmn{他挥了手}\end{exemple}
\begin{exemple}\pjya{ɯ-rna ɲɯ-sɤlwɤlwɤt tɕe βɣɤza ɲɯ-ɤsɯ-no}\hspace{5pt}\pcmn{它在动耳朵赶蚊子}\end{exemple}
\begin{exemple}\pjya{qale ta-βzu tɕe, rloŋrta ɲɯ-sɤlwɤlwɤt}\hspace{5pt}\pcmn{风吹,使玛尼旗飘动}\end{exemple}\relationsémantique{参考}{\lien{ⓔɣɤlwɤlwɤt}{ɣɤlwɤlwɤt}}\end{entrée}

\begin{entrée}{sɤɬɯt}{}{ⓔsɤɬɯt}\relationsémantique{参考}{\lien{ⓔaɬɯt}{aɬɯt}}\end{entrée}

\begin{entrée}{sɤmbɤldʑɤm}{}{ⓔsɤmbɤldʑɤm}\relationsémantique{参考}{\lien{ⓔambɤldʑɤm}{ambɤldʑɤm}}\end{entrée}

\begin{entrée}{sɤmbi}{}{ⓔsɤmbi} 
\classe{vl} \paradigme{dir}{nɯ-}
\begin{définition}\pfra{réclamer à qqn}\end{définition}
\begin{définition}\pcmn{向别人去要}\end{définition}
\begin{exemple}\pjya{laχtɕha nɯ-sɤmbi}\hspace{5pt}\pcmn{你向他要东西}\end{exemple}
\begin{exemple}\pjya{ɕ-pɯ-sɤmbi-j}\hspace{5pt}\pcmn{我们去要了}\end{exemple}
\begin{exemple}\pjya{aʑo tɤ-ŋgɯm ci nɯ-sɤmbi-a}\hspace{5pt}\pcmn{我要了一个鸡蛋}\end{exemple}
\begin{exemple}\pjya{aʑo api ɯ-ɕki kɯmtɕhɯ ci nɯ-sɤmbi-a}\hspace{5pt}\pcmn{我向哥哥要了玩具}\end{exemple}
\begin{exemple}\pjya{nɯ-sɤmbi-t-a}\end{exemple}\relationsémantique{参考}{\lien{ⓔmbi}{mbi}}\end{entrée}

\begin{entrée}{sɤmbrɤqɤt}{}{ⓔsɤmbrɤqɤt} 
\classe{vt} \paradigme{dir}{nɯ-}
\begin{définition}\pfra{différencier}\end{définition}
\begin{définition}\pcmn{分辨}\end{définition}
\begin{exemple}\pjya{stoʁ staχpɯ na-sɤmbrɤqɤt}\hspace{5pt}\pcmn{他分辨了胡豆和豌豆}\end{exemple}
\begin{exemple}\pjya{laχtɕha na-sɤmbrɤqɤt}\hspace{5pt}\pcmn{他区分了东西}\end{exemple}
\begin{exemple}\pjya{mbraj cho tɯrgi na-sɤmbrɤqɤt}\hspace{5pt}\pcmn{他分辨了红桦树和杉树}\end{exemple}
\begin{exemple}\pjya{kɯ-ŋu kɯ-maʁ ɲɯ-sɤmbrɤqɤt}\hspace{5pt}\pcmn{他分辨真的和假的}\end{exemple}
\begin{exemple}\pjya{tɯ-rju tɯ-ŋka ɯ-ŋgɯ nɯ tɕu kɤ-sɤmbrɤqɤt tu}\hspace{5pt}\pcmn{一句话有几种意思}\end{exemple}
\begin{exemple}\pjya{tɕe qale kɯ tɤɕi ɯ-rdoʁ cho ɯ-βɣi nɯ ra ɲɤ-sɤmbrɤqɤt ɕti}\hspace{5pt}\pcmn{风把青稞颗粒和糠秕分开}\end{exemple}\relationsémantique{参考}{\lien{ⓔambrɤqɤt}{ambrɤqɤt}}\end{entrée}

\begin{entrée}{sɤmbrɯ}{}{ⓔsɤmbrɯ} 
\classe{vi}  
\grammaire{denom} \paradigme{dir}{tɤ-}
\begin{définition}\pfra{s’énerver}\end{définition}
\begin{définition}\pcmn{生气}\end{définition}
\begin{exemple}\pjya{aʑo tɤ-sɤmbrɯ-a}\hspace{5pt}\pcmn{我生气了}\end{exemple}
\begin{exemple}\pjya{ɲɯ-sɤmbrɯ}\hspace{5pt}\pcmn{他在生气}\end{exemple}\relationsémantique{参考}{\lien{ⓔsɤzmbrɯ}{sɤzmbrɯ}}\relationsémantique{参考}{\lien{ⓔnɤmbrɯ}{nɤmbrɯ}}\relationsémantique{参考}{\lien{ⓔɣɤsɤmbrɯ}{ɣɤsɤmbrɯ}}\relationsémantique{参考}{\lien{ⓔtɤ-mbrɯ}{tɤ-mbrɯ}}\end{entrée}

\begin{entrée}{sɤmbrɯŋgɯ}{}{ⓔsɤmbrɯŋgɯ} 
\classe{vs}  
\grammaire{incorp} \paradigme{dir}{tɤ-}
\begin{définition}\pfra{être détestable}\end{définition}
\begin{définition}\pcmn{讨厌}\end{définition}
\begin{exemple}\pjya{tɯrme kɯ-sɤmbrɯŋgɯ ci ɲɯ-ŋu}\hspace{5pt}\pcmn{他是一个讨厌的人}\end{exemple}
\begin{exemple}\pjya{kɤ-sɤmbrɯŋgɯ ɲɯ-tɯ-χɕu}\hspace{5pt}\pcmn{你最会叫人生气}\end{exemple}\relationsémantique{参考}{\lien{ⓔsɤmbrɯ}{sɤmbrɯ}}\relationsémantique{参考}{\lien{ⓔtɤ-mbrɯ,ŋgɯ}{tɤ-mbrɯ,ŋgɯ}}\end{entrée}

\begin{entrée}{sɤmdzɯ}{}{ⓔsɤmdzɯ}\relationsémantique{参考}{\lien{ⓔamdzɯ}{amdzɯ}}\end{entrée}

\begin{entrée}{sɤmgri}{}{ⓔsɤmgri} 
\classe{vt} \paradigme{dir}{nɯ-}
\begin{définition}\pfra{rendre claire (l’eau)}\end{définition}
\begin{définition}\pcmn{澄清}\end{définition}
\begin{exemple}\pjya{tɯ-ci ɲɯ́-wɣ-sɤmgri tɕe kɤ-tshi sna}\hspace{5pt}\pcmn{水澄清了就可以喝}\end{exemple}
\begin{exemple}\pjya{mɯ-nɯ-kɤ-sɤmgri nɯ kɤ-tshi mɤ-sna}\hspace{5pt}\pcmn{没有澄清的不能喝}\end{exemple}\end{entrée}

\begin{entrée}{sɤmgro}{}{ⓔsɤmgro} 
\classe{vs} \paradigme{dir}{tɤ-}
\begin{définition}\pfra{qui donne envie}\end{définition}
\begin{définition}\pcmn{令人有……的欲望}\end{définition}
\begin{exemple}\pjya{aʑo kɤ-mbi mɯ́j-sɤmgro-a wo}\hspace{5pt}\pcmn{别人把东西给我,我就不领情}\end{exemple}
\begin{exemple}\pjya{kɤ-ndza ɲɯ-sɤmgro}\hspace{5pt}\pcmn{令人有想吃的感觉}\end{exemple}\relationsémantique{反义词}{\lien{ⓔsɤŋɤβ}{sɤŋɤβ}}\relationsémantique{参考}{\lien{}{nɤmgro}}\end{entrée}

\begin{entrée}{sɤmgrɯn}{}{ⓔsɤmgrɯn}\relationsémantique{参考}{\lien{ⓔmgrɯn}{mgrɯn}}\end{entrée}

\begin{entrée}{sɤmɲɤm}{}{ⓔsɤmɲɤm}\relationsémantique{参考}{\lien{ⓔamɲɤm}{amɲɤm}}\end{entrée}

\begin{entrée}{sɤmɲo}{}{ⓔsɤmɲo} 
\classe{vs} 
\begin{définition}\pfra{magnifique, splendide (spectacle, paysage)}\end{définition}
\begin{définition}\pcmn{精彩(表演);值得观看}\end{définition}
\begin{exemple}\pjya{jisŋi stɯnmɯ wuma pɯ-sɤmɲo}\hspace{5pt}\pcmn{今天的婚礼很精彩}\end{exemple}
\begin{exemple}\pjya{tɕetu zgoku tu-kɯ-ɕe ɲɯ-kɯ-nɤrɯra tɕe wuma ʑo sɤmɲo}\hspace{5pt}\pcmn{山上的风景很值得观看}\end{exemple}\relationsémantique{参考}{\lien{ⓔnɤmɲo}{nɤmɲo}}\end{entrée}

\begin{entrée}{sɤmŋaʁ}{}{ⓔsɤmŋaʁ}\relationsémantique{参考}{\lien{ⓔamŋaʁ}{amŋaʁ}}\end{entrée}

\begin{entrée}{sɤmŋo}{}{ⓔsɤmŋo} 
\classe{n}  
\grammaire{n.lieu} 
\begin{définition}\pfra{Somang}\end{définition}
\begin{définition}\pcmn{梭磨河}\end{définition}\end{entrée}

\begin{entrée}{sɤmŋɯr}{}{ⓔsɤmŋɯr} 
\classe{vs} \paradigme{dir}{nɯ-}
\begin{définition}\pfra{goût huileux écœurant}\end{définition}
\begin{définition}\pcmn{腻(食物)}\end{définition}
\begin{exemple}\pjya{kɤndza ɲɯ-sɤmŋɯr}\hspace{5pt}\pcmn{食物很腻}\end{exemple}
\begin{exemple}\pjya{tɤ-mthɯm ɲɯ-sɤmŋɯr}\hspace{5pt}\pcmn{肉很腻}\end{exemple}\end{entrée}

\begin{entrée}{sɤmokhɯtsa}{}{ⓔsɤmokhɯtsa} 
\classe{n} 
\begin{définition}\pfra{bol en bois}\end{définition}
\begin{définition}\pcmn{木碗}\end{définition}\end{entrée}

\begin{entrée}{sɤmtɕhoʁ}{}{ⓔsɤmtɕhoʁ} 
\classe{vt}  
\grammaire{caus} \paradigme{dir}{nɯ-}\paradigme{dir}{tɤ-}
\begin{définition}\pfra{mettre en ordre}\end{définition}
\begin{définition}\pcmn{摆整齐;使均匀}\end{définition}
\begin{exemple}\pjya{ndʑu nɯ-sɤmtɕhoʁ-a}\hspace{5pt}\pcmn{我把筷子弄整齐了}\end{exemple}
\begin{exemple}\pjya{nɤki tɯ-ŋga ra nɯ-sɤmtɕhoʁ}\hspace{5pt}\pcmn{把衣服收拾整齐}\end{exemple}
\begin{exemple}\pjya{jisŋi ji-ma nɯ ra sɤmtɕhoʁ-a ra}\hspace{5pt}\pcmn{我要整理今天的工作}\end{exemple}\relationsémantique{参考}{\lien{ⓔamtɕhoʁ}{amtɕhoʁ}}\end{entrée}

\begin{entrée}{sɤmtɕhɯβ}{}{ⓔsɤmtɕhɯβ}\relationsémantique{参考}{\lien{ⓔmtɕhɯβ}{mtɕhɯβ}}\end{entrée}

\begin{entrée}{sɤmthoʁmthɯt}{}{ⓔsɤmthoʁmthɯt} 
\classe{vt} \paradigme{dir}{lɤ-}
\begin{définition}\pfra{relier}\end{définition}
\begin{définition}\pcmn{连接}\end{définition}
\begin{exemple}\pjya{tɤfsɤri lɤ-sɤmthoʁmthɯt-a}\hspace{5pt}\pcmn{我把麻绳接起来了}\end{exemple}
\begin{exemple}\pjya{tɯmbri lɤ-sɤmthoʁmthɯt-a}\hspace{5pt}\pcmn{我把绳子接起来了}\end{exemple}\relationsémantique{同义词}{\lien{ⓔsɤlɤɣɯ}{sɤlɤɣɯ}}\relationsémantique{参考}{\lien{ⓔamthoʁmthɯt}{amthoʁmthɯt}}\end{entrée}

\begin{entrée}{sɤmto}{}{ⓔsɤmto} 
\classe{vs}  
\grammaire{deexp} \paradigme{dir}{tɤ-}
\begin{définition}\pfra{visible}\end{définition}
\begin{définition}\pcmn{看得见的}\end{définition}
\begin{exemple}\pjya{ɯ-phoŋbu lonba ɲɯ-sɤmto}\hspace{5pt}\pcmn{他的身体完全看得到}\end{exemple}
\begin{exemple}\pjya{ɯ-qiɯ ɲɯ-sɤmto}\hspace{5pt}\pcmn{看得到一半}\end{exemple}
\begin{exemple}\pjya{nɤ-ku ɲɯ-sɤmto}\hspace{5pt}\pcmn{看得见你的头}\end{exemple}
\begin{exemple}\pjya{kutɕu ku-kɯ-rɤʑi tɕe, kha sɤmto}\hspace{5pt}\pcmn{在这里看得见房子}\end{exemple}\end{entrée}

\begin{entrée}{sɤmtshɤm}{}{ⓔsɤmtshɤm} 
\classe{vs} \paradigme{dir}{tɤ-}
\begin{définition}\pfra{facile à entendre}\end{définition}
\begin{définition}\pcmn{听得见}\end{définition}
\begin{exemple}\pjya{aʑo tu-ti-a ɯ-ɲɯ́-sɤmtshɤm}\hspace{5pt}\pcmn{我说的话听得见吗?}\end{exemple}
\begin{exemple}\pjya{nɤʑɯɣ ɯ-ɲɯ́-sɤmtshɤm?}\hspace{5pt}\pcmn{你听得到吗?}\end{exemple}\relationsémantique{参考}{\lien{ⓔmtshɤm}{mtshɤm}}\end{entrée}

\begin{entrée}{sɤmtshɤr}{}{ⓔsɤmtshɤr} 
\classe{vs} \paradigme{dir}{tɤ-}
\begin{définition}\pfra{étrange}\end{définition}
\begin{définition}\pcmn{奇怪}\end{définition}
\begin{exemple}\pjya{alo tʂu nɯ kɯ-sɤmtshɤr pjɤ-mbɯt}\hspace{5pt}\pcmn{上游,路塌方得很吓人}\end{exemple}\relationsémantique{参考}{\lien{ⓔnɤmtshɤr}{nɤmtshɤr}}\end{entrée}

\begin{entrée}{sɤmtshi}{}{ⓔsɤmtshi}\relationsémantique{参考}{\lien{ⓔmtshi}{mtshi}}\end{entrée}

\begin{entrée}{sɤmtsɯɣ}{}{ⓔsɤmtsɯɣ}\relationsémantique{参考}{\lien{ⓔmtsɯɣ}{mtsɯɣ}}\end{entrée}

\begin{entrée}{sɤmtsɯr}{}{ⓔsɤmtsɯr} 
\classe{vs}  
\grammaire{deexp} 
\begin{définition}\pfra{famine (y avoir une)}\end{définition}
\begin{définition}\pcmn{饥荒}\end{définition}
\begin{exemple}\pjya{ɯ-tɯ-sɤmtsɯr saχaʁ}\hspace{5pt}\pcmn{有饥荒}\end{exemple}\relationsémantique{参考}{\lien{ⓔmtsɯr}{mtsɯr}}\end{entrée}

\begin{entrée}{sɤmɯβde}{}{ⓔsɤmɯβde}\relationsémantique{参考}{\lien{ⓔβde}{βde}}\end{entrée}

\begin{entrée}{sɤmɯmto}{}{ⓔsɤmɯmto}\relationsémantique{参考}{\lien{ⓔamɯmto}{amɯmto}}\end{entrée}

\begin{entrée}{sɤmɯrpu}{}{ⓔsɤmɯrpu}\relationsémantique{参考}{\lien{ⓔamɯrpu}{amɯrpu}}\end{entrée}

\begin{entrée}{sɤmɯrtsɯɣ}{}{ⓔsɤmɯrtsɯɣ}\relationsémantique{参考}{\lien{ⓔmɯrtsɯɣ}{mɯrtsɯɣ}}\end{entrée}

\begin{entrée}{sɤmɯsthaβ}{}{ⓔsɤmɯsthaβ}\relationsémantique{参考}{\lien{ⓔamɯsthaβ}{amɯsthaβ}}\end{entrée}

\begin{entrée}{sɤmɯsɯz}{}{ⓔsɤmɯsɯz}\relationsémantique{参考}{\lien{ⓔamɯsɯz}{amɯsɯz}}\end{entrée}

\begin{entrée}{sɤmɯtso}{}{ⓔsɤmɯtso}\relationsémantique{参考}{\lien{ⓔamɯtso}{amɯtso}}\end{entrée}

\begin{entrée}{sɤmɯtɯɣ}{}{ⓔsɤmɯtɯɣ}\relationsémantique{参考}{\lien{ⓔamɯtɯɣ}{amɯtɯɣ}}\end{entrée}

\begin{entrée}{sɤmɯzɣɯt}{}{ⓔsɤmɯzɣɯt}\relationsémantique{参考}{\lien{ⓔamɯzɣɯt}{amɯzɣɯt}}\end{entrée}

\begin{entrée}{sɤnaʁdɤz}{}{ⓔsɤnaʁdɤz}\relationsémantique{参考}{\lien{ⓔnaʁdɤz}{naʁdɤz}}\end{entrée}

\begin{entrée}{sɤnaχsoz}{}{ⓔsɤnaχsoz}\relationsémantique{参考}{\lien{ⓔnaχsoz}{naχsoz}}\end{entrée}

\begin{entrée}{sɤnɤjkɯz}{}{ⓔsɤnɤjkɯz}\relationsémantique{参考}{\lien{ⓔnɤjkɯz}{nɤjkɯz}}\end{entrée}

\begin{entrée}{sɤnɤkhe}{}{ⓔsɤnɤkhe}\relationsémantique{参考}{\lien{ⓔnɤkhe}{nɤkhe}}\end{entrée}

\begin{entrée}{sɤnɤmpɕɤr}{}{ⓔsɤnɤmpɕɤr}\relationsémantique{参考}{\lien{ⓔnɤmpɕɤr}{nɤmpɕɤr}}\end{entrée}

\begin{entrée}{sɤnɤmtsioʁ}{}{ⓔsɤnɤmtsioʁ}\relationsémantique{参考}{\lien{ⓔnɤmtsioʁ}{nɤmtsioʁ}}\end{entrée}

\begin{entrée}{sɤnɤntshɣɤz}{}{ⓔsɤnɤntshɣɤz}\relationsémantique{参考}{\lien{ⓔnɤntshɣɤz}{nɤntshɣɤz}}\end{entrée}

\begin{entrée}{sɤnɤre}{}{ⓔsɤnɤre}\relationsémantique{参考}{\lien{ⓔnɤreⓗ1ⓢ2ⓝnɤre}{nɤre}}\end{entrée}

\begin{entrée}{sɤnɤsɤɣ}{}{ⓔsɤnɤsɤɣ}\relationsémantique{参考}{\lien{ⓔnɤsɤɣ}{nɤsɤɣ}}\end{entrée}

\begin{entrée}{sɤnɤsma}{}{ⓔsɤnɤsma}\relationsémantique{参考}{\lien{ⓔnɤsma}{nɤsma}}\end{entrée}

\begin{entrée}{sɤnɤstu}{}{ⓔsɤnɤstu}\relationsémantique{参考}{\lien{ⓔnɤstu}{nɤstu}}\end{entrée}

\begin{entrée}{sɤnɤz}{}{ⓔsɤnɤz}\relationsémantique{参考}{\lien{ⓔnɤz}{nɤz}}\end{entrée}

\begin{entrée}{sɤndu}{}{ⓔsɤndu} 
\classe{vt} \paradigme{dir}{tɤ-}
\begin{définition}\pfra{échanger}\end{définition}
\begin{définition}\pcmn{交换}\end{définition}
\begin{exemple}\pjya{tɯ-rɣi tɤ-sɤndu-j}\hspace{5pt}\pcmn{我们交换了种子}\end{exemple}\relationsémantique{同义词}{\lien{ⓔnɤsci}{nɤsci}}\relationsémantique{同义词}{\lien{ⓔsɤscɯndu}{sɤscɯndu}}\relationsémantique{参考}{\lien{ⓔantsɤndu}{antsɤndu}}\end{entrée}

\begin{entrée}{sɤndɤɣ}{}{ⓔsɤndɤɣ} 
\classe{vs}  
\grammaire{denom} \paradigme{dir}{tɤ-}
\begin{définition}\pfra{vénéneux}\end{définition}
\begin{définition}\pcmn{有毒性;导致中毒(食物)}\end{définition}
\begin{exemple}\pjya{tɤjmɤɣ ɲɯ-sɤndɤɣ}\hspace{5pt}\pcmn{蘑菇是有毒的}\end{exemple}\relationsémantique{参考}{\lien{ⓔtɤndɤɣ}{tɤndɤɣ}}\relationsémantique{参考}{\lien{ⓔnɤndɤɣⓝznɤndɤɣ}{znɤndɤɣ}}\end{entrée}

\begin{entrée}{sɤndɤr}{₂}{ⓔsɤndɤrⓗ2} 
\classe{n} 
\begin{définition}\pfra{dé à coudre}\end{définition}
\begin{définition}\pcmn{顶针}\end{définition}
\begin{exemple}\pjya{kɤ-rɤtʂɯβ ɯ-tshɯɣa nɯ kɯrɯ cho kupa ɣɯ ɯ-tɕhɤjlɯz mɤ-naχtɕɯɣ tɕe, taqaβ kɤ-ndo ɯ-tshɯɣa mɤ-naχtɕɯɣ tɕe, sɤndɤr kɤ-ntɕhoz ɯ-tshɯɣa kɯnɤ mɤ-naχtɕɯɣ tɕe, kupa ra kɯ sɤndɤr nɯ-jaʁndzu mɤpaχcɤl ɣɯ ɯ-qa zɯ lu-rʁe-nɯ ŋu ma taqaβ nɯ kɯ chɯ-sɯ-sthoʁ-nɯ ɲɯ-ra, kɯrɯ ra kɯ sɤndɤr nɯ ɯ-jaʁndzu maŋlo nɯ ɣɯ ɯ-kɤχcɤl nɯ tɕu tu-sɯ-ndo-j ŋu ma nɯ kɯ taqaβ sɯ-sthoʁ-i ra. tɕe kupa ra thɯ-rɤtʂɯβ-nɯ tɕe ɯ-ʁɤri lu-ɕe-nɯ ŋu, kɯrɯ ra chɯ-rɤtʂɯβ-i tɕe, ɯ-qhu chu chɯ-cit-i ŋu.}\hspace{5pt}\pcmn{藏族和汉族缝针的风俗不同,拿针的方式不一样,所以戴顶针的部位也不一样。汉族是把顶针戴在中指的根部,用这个部位来顶针,而我们藏族是把顶针戴在食指顶上,用这个部位来顶针。汉族缝布时是往前逢,我们藏族是向后缝。}\end{exemple}\end{entrée}

\begin{entrée}{sɤndɤr}{₁}{ⓔsɤndɤrⓗ1} 
\classe{vt} \paradigme{dir}{nɯ-}
\begin{définition}\pfra{faire mal en touchant une blessure}\end{définition}
\begin{définition}\pcmn{碰(伤口)}\end{définition}
\begin{exemple}\pjya{a-jaʁ na-sɤndɤr}\hspace{5pt}\pcmn{他碰了我的手,把我弄得很痛}\end{exemple}
\begin{exemple}\pjya{tɯ-ɣmaz nɯ-tɯ-sɤndɤr}\hspace{5pt}\pcmn{你碰了伤口}\end{exemple}
\begin{exemple}\pjya{a-tɯ-ɣmaz ma-nɯ-tɯ-sɤndɤr}\hspace{5pt}\pcmn{你不要碰我的伤口}\end{exemple}
\begin{exemple}\pjya{nɤ-jaʁ ɲɯ-mŋɤm tɕe, ma-nɯ-tɯ-sɤndɤr}\hspace{5pt}\pcmn{你的手既然很痛,不要碰它}\end{exemple}\relationsémantique{参考}{\lien{ⓔandɤr}{andɤr}}\end{entrée}

\begin{entrée}{sɤndɤrndɤr}{}{ⓔsɤndɤrndɤr} 
\classe{vt}  
\grammaire{deidph} \paradigme{dir}{tɤ-}\paradigme{dir}{nɯ-}
\begin{définition}\pfra{faire du bruit en vibrant fortement}\end{définition}
\begin{définition}\pcmn{震动得很响,发出声音}\end{définition}
\begin{exemple}\pjya{tɤrmbɣo ɲɯ-sɤndɤrndɤr}\hspace{5pt}\pcmn{打鼓打得很响}\end{exemple}\end{entrée}

\begin{entrée}{sɤndɣɤndɣɤt}{}{ⓔsɤndɣɤndɣɤt}\relationsémantique{参考}{\lien{ⓔɣɤndɣɤndɣɤt}{ɣɤndɣɤndɣɤt}}\end{entrée}

\begin{entrée}{sɤndɯja}{}{ⓔsɤndɯja}\relationsémantique{参考}{\lien{ⓔandɯja}{andɯja}}\end{entrée}

\begin{entrée}{sɤndɯndo}{}{ⓔsɤndɯndo}\relationsémantique{参考}{\lien{ⓔndo}{ndo}}\end{entrée}

\begin{entrée}{sɤndza}{}{ⓔsɤndza}\relationsémantique{参考}{\lien{ⓔndza}{ndza}}\end{entrée}

\begin{entrée}{sɤndzoʁjoʁ}{}{ⓔsɤndzoʁjoʁ}\relationsémantique{参考}{\lien{ⓔandzoʁjoʁ}{andzoʁjoʁ}}\end{entrée}

\begin{entrée}{sɤndzɯrndzɯr}{}{ⓔsɤndzɯrndzɯr}\relationsémantique{参考}{\lien{ⓔɣɤndzɯrndzɯr}{ɣɤndzɯrndzɯr}}\end{entrée}

\begin{entrée}{sɤndʑɤmstu}{}{ⓔsɤndʑɤmstu}\relationsémantique{参考}{\lien{ⓔandʑɤmstu}{andʑɤmstu}}\end{entrée}

\begin{entrée}{sɤndʑɯ}{}{ⓔsɤndʑɯ}\relationsémantique{参考}{\lien{ⓔndʑɯ}{ndʑɯ}}\end{entrée}

\begin{entrée}{sɤndʐaβ}{}{ⓔsɤndʐaβ} 
\classe{vs}  
\grammaire{deexp}
\grammaire{acaus} \paradigme{dir}{tɤ-}
\begin{définition}\pfra{endroit où il est courant de tomber}\end{définition}
\begin{définition}\pcmn{令人容易摔倒的地方}\end{définition}
\begin{exemple}\pjya{tʂu ɲɯ-ɣɤrɤβ ɲɯ-sɤndʐaβ}\hspace{5pt}\pcmn{路很陡峭,容易摔倒}\end{exemple}
\begin{exemple}\pjya{kutɕu sɤndʐaβ ŋgrɤl}\hspace{5pt}\pcmn{这里容易摔倒}\end{exemple}\relationsémantique{参考}{\lien{ⓔndʐaβ}{ndʐaβ}}\end{entrée}

\begin{entrée}{sɤntɕhoʁjɤr}{}{ⓔsɤntɕhoʁjɤr}\relationsémantique{参考}{\lien{ⓔantɕhoʁjɤr}{antɕhoʁjɤr}}\end{entrée}

\begin{entrée}{sɤntɕhoz}{}{ⓔsɤntɕhoz} 
\classe{n} 
\begin{définition}\pfra{utilité}\end{définition}
\begin{définition}\pcmn{用处}\end{définition}
\begin{exemple}\pjya{ki tɯ-rju tɯ-ŋka ki ɯ-sɤntɕhoz ɲɯ-dɤn}\hspace{5pt}\pcmn{这一句话很有用}\end{exemple}\end{entrée}

\begin{entrée}{sɤntɕhɯ}{}{ⓔsɤntɕhɯ}\relationsémantique{参考}{\lien{ⓔantɕhɯ}{antɕhɯ}}\end{entrée}

\begin{entrée}{sɤnthɣar}{}{ⓔsɤnthɣar}\relationsémantique{参考}{\lien{ⓔanthɣar}{anthɣar}}\end{entrée}

\begin{entrée}{sɤntsɤndu}{}{ⓔsɤntsɤndu} 
\classe{vt}  
\grammaire{caus} \paradigme{dir}{tɤ-}
\begin{définition}\pfra{échanger}\end{définition}
\begin{définition}\pcmn{交换}\end{définition}
\begin{exemple}\pjya{tɕi-ŋga to-nɯ-sɤntsɤndu-tɕi}\hspace{5pt}\pcmn{我们不小心把衣服交换了}\end{exemple}
\begin{exemple}\pjya{khɯtsa to-sɤntsɤndu-tɕi}\hspace{5pt}\pcmn{我们无意中把碗交换了}\end{exemple}
\begin{exemple}\pjya{ndʑi-rmi ɲɤ-sɤntsɤndu-t-a}\hspace{5pt}\pcmn{我说错了他们的名字}\end{exemple}\relationsémantique{参考}{\lien{ⓔantsɤndu}{antsɤndu}}\end{entrée}

\begin{entrée}{sɤntɯ}{}{ⓔsɤntɯ}\relationsémantique{参考}{\lien{ⓔantɯ}{antɯ}}\end{entrée}

\begin{entrée}{sɤnɯβlu}{}{ⓔsɤnɯβlu}\relationsémantique{参考}{\lien{ⓔnɯβlu}{nɯβlu}}\end{entrée}

\begin{entrée}{sɤnɯɕtar}{}{ⓔsɤnɯɕtar}\relationsémantique{参考}{\lien{ⓔnɯɕtar}{nɯɕtar}}\end{entrée}

\begin{entrée}{sɤnɯkhramba}{}{ⓔsɤnɯkhramba}\relationsémantique{参考}{\lien{ⓔnɯkhramba}{nɯkhramba}}\end{entrée}

\begin{entrée}{sɤnɯmpa}{}{ⓔsɤnɯmpa}\relationsémantique{参考}{\lien{ⓔnɯmpa}{nɯmpa}}\end{entrée}

\begin{entrée}{sɤnɯmtɕhu}{}{ⓔsɤnɯmtɕhu}\relationsémantique{参考}{\lien{ⓔnɯmtɕhu}{nɯmtɕhu}}\end{entrée}

\begin{entrée}{sɤnɯmthɯ}{}{ⓔsɤnɯmthɯ}\relationsémantique{参考}{\lien{}{nɯmthɯ}}\end{entrée}

\begin{entrée}{sɤnɯŋumit}{}{ⓔsɤnɯŋumit}\relationsémantique{参考}{\lien{ⓔnɯŋumit}{nɯŋumit}}\end{entrée}

\begin{entrée}{sɤnɯrga}{}{ⓔsɤnɯrga}\relationsémantique{参考}{\lien{ⓔnɯrga}{nɯrga}}\end{entrée}

\begin{entrée}{sɤnɯrtɕa}{}{ⓔsɤnɯrtɕa}\relationsémantique{参考}{\lien{ⓔnɯrtɕa}{nɯrtɕa}}\end{entrée}

\begin{entrée}{sɤnɯrɯtʂa}{}{ⓔsɤnɯrɯtʂa}\relationsémantique{参考}{\lien{ⓔnɯrɯtʂa}{nɯrɯtʂa}}\end{entrée}

\begin{entrée}{sɤnɯsnɯɲaʁ}{}{ⓔsɤnɯsnɯɲaʁ}\relationsémantique{参考}{\lien{ⓔnɯsnɯɲaʁ}{nɯsnɯɲaʁ}}\end{entrée}

\begin{entrée}{sɤnɯsɯkho}{}{ⓔsɤnɯsɯkho}\relationsémantique{参考}{\lien{ⓔnɯsɯkho}{nɯsɯkho}}\end{entrée}

\begin{entrée}{sɤnɯtɕetha}{}{ⓔsɤnɯtɕetha}\relationsémantique{参考}{\lien{ⓔnɯtɕetha}{nɯtɕetha}}\end{entrée}

\begin{entrée}{sɤnɯtɯtɕhɯ}{}{ⓔsɤnɯtɯtɕhɯ}\relationsémantique{参考}{\lien{ⓔnɯtɯtɕhɯ}{nɯtɯtɕhɯ}}\end{entrée}

\begin{entrée}{sɤnɯzdɯɣ}{}{ⓔsɤnɯzdɯɣ}\relationsémantique{参考}{\lien{ⓔnɯzdɯɣ}{nɯzdɯɣ}}\end{entrée}

\begin{entrée}{sɤnɯʑɤzdaŋ}{}{ⓔsɤnɯʑɤzdaŋ}\relationsémantique{参考}{\lien{ⓔnɯʑɤzdaŋ}{nɯʑɤzdaŋ}}\end{entrée}

\begin{entrée}{sɤɲaj}{}{ⓔsɤɲaj} 
\classe{vt}  
\grammaire{caus} \paradigme{dir}{\_}
\begin{définition}\pfra{accélérer un travail}\end{définition}
\begin{définition}\pcmn{赶紧把工作做完}\end{définition}
\begin{exemple}\pjya{tɯ-ntʂu la-sɤɲaj}\hspace{5pt}\pcmn{他赶紧锄草了}\end{exemple}
\begin{exemple}\pjya{kɤ-ntʂu lɤ-sɤɲaj-a}\hspace{5pt}\pcmn{我赶紧锄了草}\end{exemple}
\begin{exemple}\pjya{kɤ-nɯzʁe ka-sɤɲaj}\hspace{5pt}\pcmn{他赶紧搬了东西}\end{exemple}
\begin{exemple}\pjya{a-zda ra pɯ-nɯna-nɯ ɯ-tsi aʑo pɯ-nɯ-sɤɲaj-a (ʑ-lɤ-sɤɲaj-a)}\hspace{5pt}\pcmn{我的伙伴休息的时候,我趁机继续工作,早点把它做完}\end{exemple}
\begin{exemple}\pjya{kɤ-raχtɕi nɯ-sɤɲaj-a}\hspace{5pt}\pcmn{我赶紧地洗了}\end{exemple}
\begin{exemple}\pjya{kɤ-rɤrɤt pɯ-sɤɲaj-a}\hspace{5pt}\pcmn{我赶紧地写了}\end{exemple}\relationsémantique{参考}{\lien{ⓔaɲaj}{aɲaj}}\end{entrée}

\begin{entrée}{sɤɲcɣɤɲcɣɤt}{}{ⓔsɤɲcɣɤɲcɣɤt}\relationsémantique{参考}{\lien{ⓔɣɤɲcɣɤɲcɣɤt}{ɣɤɲcɣɤɲcɣɤt}}\end{entrée}

\begin{entrée}{sɤɲizɲiz}{}{ⓔsɤɲizɲiz}\relationsémantique{参考}{\lien{ⓔɣɤɲizɲiz}{ɣɤɲizɲiz}}\end{entrée}

\begin{entrée}{sɤɲɟu}{}{ⓔsɤɲɟu} 
\classe{n} 
\begin{définition}\pfra{herbe qu'on donne à manger à la vache pendant la traite}\end{définition}
\begin{définition}\pcmn{挤奶时,喂奶牛的草}\end{définition}\end{entrée}

\begin{entrée}{sɤɲɟɯrnor}{}{ⓔsɤɲɟɯrnor} 
\classe{vt} \paradigme{dir}{nɯ-}
\begin{définition}\pfra{se tromper}\end{définition}
\begin{définition}\pcmn{弄错}\end{définition}
\begin{exemple}\pjya{tʂu ma-nɯ-tɯ-sɤɲɟɯrnor}\hspace{5pt}\pcmn{你不要把路弄错了}\end{exemple}\relationsémantique{参考}{\lien{ⓔɲɟɯrnor}{ɲɟɯrnor}}\end{entrée}

\begin{entrée}{sɤŋu}{}{ⓔsɤŋu} 
\classe{n}  
\grammaire{n.lieu} 
\begin{définition}\pfra{Gdong.brgyad}\end{définition}
\begin{définition}\pcmn{龙尔甲}\end{définition}
\begin{exemple}\pjya{sɤŋupɯ}\hspace{5pt}\pcmn{龙尔甲人}\end{exemple}\end{entrée}

\begin{entrée}{sɤŋɤβ}{}{ⓔsɤŋɤβ} 
\classe{vs} \paradigme{dir}{nɯ-}\sens{1}
\begin{définition}\pfra{faible, en mauvaise santé}\end{définition}
\begin{définition}\pcmn{弱;不健康}\end{définition}
\begin{exemple}\pjya{tɤ-pɤtso ɲɯ-sɤŋɤβ, ɲɯ-xtɕi}\hspace{5pt}\pcmn{小孩子又小又弱}\end{exemple}\sens{2}
\begin{définition}\pfra{qui dérange, qui ne donne pas envie de}\end{définition}
\begin{définition}\pcmn{令人不愿意;不好意思}\end{définition}
\begin{exemple}\pjya{kɤ-ɕe ɲɯ-sɤŋɤβ}\hspace{5pt}\pcmn{不想去}\end{exemple}
\begin{exemple}\pjya{kɯ-rɤma jɤ-ari-a ri, ɲɤ-maqhu-a tɕe ɲɯ-sɤŋɤβ}\hspace{5pt}\pcmn{我去工作迟到了,有点不好意思}\end{exemple}
\begin{exemple}\pjya{kɤ-nɤma ɯ-tɯ-dɤn tɕe, kɤ-sɤʑa ɲɯ-sɤŋɤβ}\hspace{5pt}\pcmn{工作很多,不想开工}\end{exemple}
\begin{exemple}\pjya{rasti laji nɯ ra to-ɬoʁ ri ɲɯ-sɤŋɤβ ma ɯʑo ri ɲɯ-xtɕi, qajɯ kɯ ri to-ndza}\hspace{5pt}\pcmn{他种的大头菜虽然长出来了,但是令人不想要,因为又小又加上被虫吃了}\end{exemple}\sens{3}\paradigme{dir}{nɯ-}\paradigme{dir}{nɯ-}
\begin{définition}\pfra{à l'apparence peu avenante}\end{définition}
\begin{définition}\pcmn{丑陋,看起来不顺眼}\end{définition}
\begin{définition}\pfra{être gêné par, hésiter à}\end{définition}
\begin{définition}\pcmn{觉得不好意思}\end{définition}
\begin{définition}\pfra{gêner, embarrasser}\end{définition}
\begin{définition}\pcmn{令人不好意思}\end{définition}
\begin{exemple}\pjya{nɯ-nɤŋaβ-a}\hspace{5pt}\pcmn{我觉得不好意思了}\end{exemple}\relationsémantique{参考}{\lien{ⓔsɤŋɤβdi}{sɤŋɤβdi}}\relationsémantique{反义词}{\lien{ⓔsɤmgro}{sɤmgro}}
\begin{sous-entrée}{nɤŋɤβ}{ⓔsɤŋɤβⓢ3ⓝnɤŋɤβ} 
\classe{vt}  
\grammaire{trop} \end{sous-entrée}

\begin{sous-entrée}{znɤŋɤβ}{ⓔsɤŋɤβⓢ3ⓝznɤŋɤβ} 
\classe{vt} \end{sous-entrée}

\end{entrée}

\begin{entrée}{sɤŋɤβdi}{}{ⓔsɤŋɤβdi} 
\classe{n} 
\begin{définition}\pfra{mauvaise odeur}\end{définition}
\begin{définition}\pcmn{臭味;霉气}\end{définition}
\begin{exemple}\pjya{sɤŋɤβdi ʑo ɲɯ-mnɤm}\hspace{5pt}\pcmn{(房间里)有霉气}\end{exemple}\relationsémantique{参考}{\lien{ⓔtɤ-di}{tɤ-di}}\end{entrée}

\begin{entrée}{sɤŋgɤrɤt}{}{ⓔsɤŋgɤrɤt} 
\classe{n} 
\begin{définition}\pfra{ustensile de cuisine pour mélanger la tsampa lorsqu'on la fait frire}\end{définition}
\begin{définition}\pcmn{炒糌粑时,用来搅拌青稞的工具}\end{définition}\end{entrée}

\begin{entrée}{sɤŋgio}{}{ⓔsɤŋgio} 
\classe{vs}  
\grammaire{deexp}
\grammaire{acaus} \paradigme{dir}{tɤ-}
\begin{définition}\pfra{glissant}\end{définition}
\begin{définition}\pcmn{滑(路)}\end{définition}\relationsémantique{参考}{\lien{ⓔŋgio}{ŋgio}}\end{entrée}

\begin{entrée}{sɤŋo}{₁₂}{ⓔsɤŋoⓗ1ⓗ2} 
\classe{vi-t}
\classe{vt} \paradigme{dir}{nɯ-}\paradigme{dir}{kɤ-}
\begin{définition}\pfra{écouter, ressentir}\end{définition}
\begin{définition}\pcmn{听;感受到}\end{définition}
\begin{définition}\pfra{obéir}\end{définition}
\begin{définition}\pcmn{服从;听别人的劝告}\end{définition}
\begin{exemple}\pjya{kɤ-sɤŋo-t-a}\hspace{5pt}\pcmn{我听从他了}\end{exemple}
\begin{exemple}\pjya{wuma ɣɯ-sɤŋo-a}\hspace{5pt}\pcmn{他很听我的话}\end{exemple}
\begin{exemple}\pjya{@dianhua khro jɤ-lat-a ri, kɯ-sɤŋo maŋe}\hspace{5pt}\pcmn{我打了很多次电话,没有人接}\end{exemple}
\begin{exemple}\pjya{ŋgumdʑɯɣ ra kɯ ta-tɯt nɯ kɤ-sɤŋo-t-a}\hspace{5pt}\pcmn{他听从了领导的话}\end{exemple}
\begin{exemple}\pjya{a-wa kɯ ta-tɯt nɯ kɤ-sɤŋo-t-a}\hspace{5pt}\pcmn{我听从了我父亲的话}\end{exemple}\relationsémantique{参考}{\lien{ⓔmɤrnɤsɤŋo}{mɤrnɤsɤŋo}}\relationsémantique{参考}{\lien{ⓔsɤsɤŋo}{sɤsɤŋo}}
\begin{sous-entrée}{sɤŋo}{ⓔsɤŋoⓗ1ⓝsɤŋo}
\begin{exemple}\pjya{nɯ-sɤŋo-a}\hspace{5pt}\pcmn{我听了}\end{exemple}
\begin{exemple}\pjya{tɤ-ti jɤɣ ma kɯre ku-sɤŋo-a}\hspace{5pt}\pcmn{你讲吧,我在这里听着}\end{exemple}\end{sous-entrée}

\end{entrée}

\begin{entrée}{sɤŋoʁŋoʁ}{}{ⓔsɤŋoʁŋoʁ} 
\classe{vt} \paradigme{dir}{pɯ-}\paradigme{dir}{tɤ-}
\begin{définition}\pfra{hocher de la tête}\end{définition}
\begin{définition}\pcmn{点头}\end{définition}
\begin{exemple}\pjya{ɯ-ku ɲɯ-sɤŋoʁŋoʁ}\hspace{5pt}\pcmn{他在点头}\end{exemple}\relationsémantique{参考}{\lien{ⓔɣɤŋoʁ}{ɣɤŋoʁ}}\end{entrée}

\begin{entrée}{sɤŋɯr}{}{ⓔsɤŋɯr}\relationsémantique{参考}{\lien{ⓔnɤŋɯr}{nɤŋɯr}}\end{entrée}

\begin{entrée}{sɤpa}{}{ⓔsɤpa} 
\classe{vt} \paradigme{dir}{nɯ-}\paradigme{dir}{nɯ-}
\begin{définition}\pfra{transformer en}\end{définition}
\begin{définition}\pcmn{使其变成}\end{définition}
\begin{définition}\pfra{se transformer en}\end{définition}
\begin{définition}\pcmn{变成}\end{définition}
\begin{exemple}\pjya{mɤ-kɯ-mpɕɤr nɯ kɯ-mpɕɤr kɤ-sɤpa mɤ-khɯ}\hspace{5pt}\pcmn{不能把不漂亮的变成漂亮的}\end{exemple}\relationsémantique{同义词}{\lien{ⓔβzɟɯrⓢ2ⓝʑɣɤβzɟɯr}{ʑɣɤβzɟɯr}}
\begin{sous-entrée}{ʑɣɤsɤpa}{ⓔsɤpaⓝʑɣɤsɤpa} 
\classe{vi}  
\grammaire{refl} \end{sous-entrée}

\end{entrée}

\begin{entrée}{sɤpɤmbat}{}{ⓔsɤpɤmbat}\relationsémantique{参考}{\lien{ⓔapɤmbat}{apɤmbat}}\end{entrée}

\begin{entrée}{sɤpɕoʁ}{}{ⓔsɤpɕoʁ} 
\classe{n} 
\begin{définition}\pfra{endroit}\end{définition}
\begin{définition}\pcmn{地区}\end{définition}\étymologie{sa.pʰʲogs}\end{entrée}

\begin{entrée}{sɤpɕɯβjɤl}{}{ⓔsɤpɕɯβjɤl}\relationsémantique{参考}{\lien{ⓔapɕɯβjɤl}{apɕɯβjɤl}}\end{entrée}

\begin{entrée}{sɤpe}{}{ⓔsɤpe} 
\classe{vt} \paradigme{dir}{tɤ-}
\begin{définition}\pfra{faire bien}\end{définition}
\begin{définition}\pcmn{弄好}\end{définition}
\begin{exemple}\pjya{ta-sɤpe}\hspace{5pt}\pcmn{他弄好了}\end{exemple}
\begin{exemple}\pjya{kɤ-sɤpe ɯ-pɯ-tɯ-cha-nɯ ?}\hspace{5pt}\pcmn{你们弄好没有?}\end{exemple}
\begin{exemple}\pjya{nɤʑo sɤznɤ aʑo kɤ-nɤma sɤpe-a}\hspace{5pt}\pcmn{我工作得比你好}\end{exemple}\relationsémantique{参考}{\lien{ⓔpe}{pe}}\end{entrée}

\begin{entrée}{sɤpɣaʁsci}{}{ⓔsɤpɣaʁsci}\relationsémantique{参考}{\lien{ⓔapɣaʁsci}{apɣaʁsci}}\end{entrée}

\begin{entrée}{sɤphɤlɤjɤt}{}{ⓔsɤphɤlɤjɤt}\relationsémantique{参考}{\lien{ⓔaphɤlɤjɤt}{aphɤlɤjɤt}}\end{entrée}

\begin{entrée}{sɤphɤr}{}{ⓔsɤphɤr} 
\classe{vt} \paradigme{dir}{nɯ-}
\begin{définition}\pfra{secouer, épousseter, faire tomber d'un récipient}\end{définition}
\begin{définition}\pcmn{拍打,抖掉}\end{définition}
\begin{exemple}\pjya{nɤ-ŋga nɯ-sɤphɤr}\hspace{5pt}\pcmn{你抖一下衣服}\end{exemple}
\begin{exemple}\pjya{nɯ-sɤphɤr ma thɯci to-ndzoʁ, to-pɣi}\hspace{5pt}\pcmn{抖一下,(衣服上)粘上了一个什么东西}\end{exemple}
\begin{exemple}\pjya{tɤ-fkɯm, tɯ-ŋga tha-sɤphɤr}\hspace{5pt}\pcmn{他把口袋、衣服抖了一下}\end{exemple}
\begin{exemple}\pjya{nɯ-sɤphar-a, thɯ-sɤphar-a}\hspace{5pt}\pcmn{我抖了一下}\end{exemple}
\begin{exemple}\pjya{nɯ-sɤphar-a tɕe a-ŋga ɯ-taʁ rdɯl ra pɯ-mbɤr}\hspace{5pt}\pcmn{我把衣服抖了一下,上面的尘土掉下来了}\end{exemple}
\begin{exemple}\pjya{tɤjko ɯ-tɯ-tɕur kɯ tɯ-ku ɲɯ-kɯ-sɯ-sɤphɤr ɲɯ-ŋu}\hspace{5pt}\pcmn{酸菜发酸,酸到昏了头}\end{exemple}
\begin{exemple}\pjya{rtsɤmkɯɣ tha-sɤphɤr}\hspace{5pt}\pcmn{他把糌粑袋子抖了一下}\end{exemple}\relationsémantique{参考}{\lien{ⓔphɤr}{phɤr}}\relationsémantique{参考}{\lien{ⓔmbɤrⓗ2}{mbɤr₂}}\end{entrée}

\begin{entrée}{sɤphɯphrɯɣ}{}{ⓔsɤphɯphrɯɣ} 
\classe{vt} 
\begin{définition}\pfra{qui coule bruyament à grosses gouttes}\end{définition}
\begin{définition}\pcmn{形容答滴答滴地流出来,发出响声的样子}\end{définition}
\begin{exemple}\pjya{ɯ-qom to-sɤphɯphrɯɣ ʑo pjɤ-tɕɤt}\hspace{5pt}\pcmn{她热泪滚滚}\end{exemple}
\begin{exemple}\pjya{tɯ-mɯ ɲɯ-sɤphɯphrɯɣ ʑo ɲɯ-ɤsɯ-lɤt}\hspace{5pt}\pcmn{下大雨(雨点比较大)}\end{exemple}\end{entrée}

\begin{entrée}{sɤpjɤntɤm}{}{ⓔsɤpjɤntɤm}\relationsémantique{参考}{\lien{ⓔapjɤntɤm}{apjɤntɤm}}\end{entrée}

\begin{entrée}{sɤplaʁplaʁ}{}{ⓔsɤplaʁplaʁ}\relationsémantique{参考}{\lien{ⓔplaʁplaʁ}{plaʁplaʁ}}\end{entrée}

\begin{entrée}{sɤprɤtprɤt}{}{ⓔsɤprɤtprɤt} 
\classe{vt} \paradigme{dir}{nɯ-}\paradigme{dir}{tɤ-}
\begin{définition}\pfra{frapper le sol du pied}\end{définition}
\begin{définition}\pcmn{跺脚}\end{définition}
\begin{exemple}\pjya{ɯ-mi ɲɯ-sɤprɤtprɤt}\hspace{5pt}\pcmn{他在跺脚}\end{exemple}
\begin{exemple}\pjya{ɯ-mi ta-sɤprɤtprɤt}\hspace{5pt}\pcmn{他跺了脚}\end{exemple}\end{entrée}

\begin{entrée}{sɤpɯpa}{}{ⓔsɤpɯpa} 
\classe{vt} \paradigme{dir}{tɤ-}\paradigme{dir}{nɯ-}\sens{1}
\begin{définition}\pfra{rassembler et mettre en ordre, préparer}\end{définition}
\begin{définition}\pcmn{收集;准备好;使其慢慢发展(种植物、养动物)}\end{définition}
\begin{exemple}\pjya{laχtɕha tɤ-sɤpɯpa-t-a}\hspace{5pt}\pcmn{我把东西准备好了}\end{exemple}
\begin{exemple}\pjya{tɤ-sɤpɯpa-j}\hspace{5pt}\pcmn{我们准备好了}\end{exemple}
\begin{exemple}\pjya{to-sɤpɯpa}\hspace{5pt}\pcmn{他准备好了}\end{exemple}
\begin{exemple}\pjya{a-tɕɯ laχtɕha kɯ-tshoz ʑo tɤ-sɤpɯpa-t-a}\hspace{5pt}\pcmn{我给儿子把东西准备齐全}\end{exemple}\sens{2}
\begin{définition}\pfra{faire se développer}\end{définition}
\begin{définition}\pcmn{使其慢慢发展(种植物、养动物)}\end{définition}
\begin{exemple}\pjya{ki ji-nɯŋa-mbala kɯ-dɯ-dɤn kɯra ji-nɯŋa mu kɯ-ɤkhra ci pɯ-tu tɕe, nɯ kɯ nɯ-kɤ-sɤpɯpa ŋu}\hspace{5pt}\pcmn{我们有这么多黄牛和奶牛,完全是从我们过去的一头老花母牛发展起来的}\end{exemple}
\begin{exemple}\pjya{smɤn kɤntɕhɯ-ɣjɤn to-tɯ-χtɯ-t tɕe, khro to-tɯ-sɤpɯpa-t}\hspace{5pt}\pcmn{你买了很多次药,积累了很多}\end{exemple}\end{entrée}

\begin{entrée}{sɤpɯpri}{}{ⓔsɤpɯpri}\relationsémantique{参考}{\lien{ⓔapɯpri}{apɯpri}}\end{entrée}

\begin{entrée}{sɤqɤrle}{}{ⓔsɤqɤrle} 
\classe{vt}  
\grammaire{caus} \paradigme{dir}{nɯ-}
\begin{définition}\pfra{sélectionner, diviser, classer}\end{définition}
\begin{définition}\pcmn{分类;区分;分开}\end{définition}
\begin{exemple}\pjya{fsapaʁ nɯ-sɤqɤrle}\hspace{5pt}\pcmn{你把牲畜分开一下(例如,把大的跟小的分开)}\end{exemple}
\begin{exemple}\pjya{fsapaʁ nɯ-sɤqɤrle-t-a}\hspace{5pt}\pcmn{我(根据大小)分开了牲畜}\end{exemple}
\begin{exemple}\pjya{jɯfɕɯr nɯ-sɤqɤrle-t-a, jisŋi kɤ-sɤqɤrle mɤ-ra}\hspace{5pt}\pcmn{我昨天已经分开了,今天不用分了}\end{exemple}
\begin{exemple}\pjya{mbro jla nɯ-sɤqɤrle-t-a}\hspace{5pt}\pcmn{我把马和犏牛分开了}\end{exemple}
\begin{exemple}\pjya{tshɤt qaʑo nɯ-sɤqɤrle-j}\hspace{5pt}\pcmn{我们把山羊和绵羊分开了}\end{exemple}\relationsémantique{同义词}{\lien{ⓔqɤt}{qɤt}}\relationsémantique{参考}{\lien{ⓔqɤr}{qɤr}}\relationsémantique{参考}{\lien{ⓔaqɤrle}{aqɤrle}}\end{entrée}

\begin{entrée}{sɤqɤtsa}{}{ⓔsɤqɤtsa}\relationsémantique{参考}{\lien{ⓔaqɤtsa}{aqɤtsa}}\end{entrée}

\begin{entrée}{sɤqɤtʂha}{}{ⓔsɤqɤtʂha}\relationsémantique{参考}{\lien{ⓔaqɤtʂha}{aqɤtʂha}}\end{entrée}

\begin{entrée}{sɤqhe}{}{ⓔsɤqhe}\relationsémantique{参考}{\lien{ⓔaqhe}{aqhe}}\end{entrée}

\begin{entrée}{sɤqhlɤβlɤβ}{}{ⓔsɤqhlɤβlɤβ} 
\classe{vt} \paradigme{dir}{pɯ-}
\begin{définition}\pfra{émettre un bruit d'éclaboussures}\end{définition}
\begin{définition}\pcmn{(洗衣服的时候)发出溅水声}\end{définition}
\begin{exemple}\pjya{sɤlaŋphɤn ɯ-ŋgɯ tɯ-ŋga pɯ-χtɕi-t-a tɕe pɯ-sɤqhlɤβlaβ-a}\hspace{5pt}\pcmn{我在盆子里洗衣服的时候发出了很响的溅水声}\end{exemple}\end{entrée}

\begin{entrée}{sɤqhloŋloŋ}{}{ⓔsɤqhloŋloŋ}\relationsémantique{参考}{\lien{ⓔqhloŋ}{qhloŋ}}\end{entrée}

\begin{entrée}{sɤqhlɯβlɯβ}{}{ⓔsɤqhlɯβlɯβ} 
\classe{vt} \paradigme{dir}{tɤ-}
\begin{définition}\pfra{émettre du bruit en agitant l'eau}\end{définition}
\begin{définition}\pcmn{发出水晃动声}\end{définition}
\begin{exemple}\pjya{tɯ-ŋga nɯ́-wɣ-ʁɟo tɕe tɯ-ci tú-wɣ-sɤqhlɯβlɯβ ʑo ra}\hspace{5pt}\pcmn{冲衣服的时候就会发出水晃动的声音}\end{exemple}\end{entrée}

\begin{entrée}{sɤqhɯqha}{}{ⓔsɤqhɯqha} 
\classe{vs} \paradigme{dir}{tɤ-}\paradigme{dir}{thɯ-}
\begin{définition}\pfra{très actif}\end{définition}
\begin{définition}\pcmn{调皮;活泼(孩子)}\end{définition}
\begin{exemple}\pjya{ma-tɯ-sɤqhɯqha}\hspace{5pt}\pcmn{你不要调皮}\end{exemple}\end{entrée}

\begin{entrée}{sɤqur}{}{ⓔsɤqur}\relationsémantique{参考}{\lien{ⓔqur}{qur}}\end{entrée}

\begin{entrée}{sɤqrɤcha}{}{ⓔsɤqrɤcha} 
\classe{n} 
\begin{définition}\pfra{alcool pour recevoir les invités}\end{définition}
\begin{définition}\pcmn{款待人的酒}\end{définition}\relationsémantique{参考}{\lien{ⓔqru}{qru}}\relationsémantique{参考}{\lien{ⓔchaⓗ2}{cha}}\end{entrée}

\begin{entrée}{sɤr}{}{ⓔsɤr} 
\classe{vs} \paradigme{dir}{tɤ-}\paradigme{dir}{thɯ-}
\begin{définition}\pfra{frais}\end{définition}
\begin{définition}\pcmn{新鲜(食物)}\end{définition}
\begin{exemple}\pjya{ta-mar ɲɯ-sɤr}\hspace{5pt}\pcmn{酥油很新鲜}\end{exemple}
\begin{exemple}\pjya{tɤ-lu sɤr ʁaznɤ kɤ-tshi}\hspace{5pt}\pcmn{你要趁牛奶新鲜的时候喝}\end{exemple}\end{entrée}

\begin{entrée}{sɤraχtɕɤz}{}{ⓔsɤraχtɕɤz}\relationsémantique{参考}{\lien{ⓔraχtɕɤz}{raχtɕɤz}}\end{entrée}

\begin{entrée}{sɤrɤpa}{}{ⓔsɤrɤpa} 
\classe{vi}  
\grammaire{comp} \paradigme{dir}{tɤ-}
\begin{définition}\pfra{se moquer}\end{définition}
\begin{définition}\pcmn{取笑}\end{définition}
\begin{exemple}\pjya{ɯʑo ɲɯ-sɤrɤpa}\hspace{5pt}\pcmn{他取笑(别人)}\end{exemple}
\begin{exemple}\pjya{aj tɤ-sɤrɤpa-a}\hspace{5pt}\pcmn{我取笑(别人)了}\end{exemple}
\begin{exemple}\pjya{jiɕqha kɯ-sɤrɤpa ci ɲɯ-ŋu}\hspace{5pt}\pcmn{他是一个爱取笑人的人}\end{exemple}
\begin{exemple}\pjya{a-ɕki ma-tɯ-sɤrɤpa}\hspace{5pt}\pcmn{你不要跟我开玩笑}\end{exemple}\relationsémantique{参考}{\lien{ⓔsɤre}{sɤre}}\end{entrée}

\begin{entrée}{sɤrɤt}{}{ⓔsɤrɤt} 
\classe{vt} \paradigme{dir}{kɤ-}
\begin{définition}\pfra{dérouler un fil}\end{définition}
\begin{définition}\pcmn{卸下来(线)}\end{définition}
\begin{exemple}\pjya{tɤ-ri ka-sɤrɤt}\hspace{5pt}\pcmn{他把线卸下来了}\end{exemple}\end{entrée}

\begin{entrée}{sɤrchɤrchɤt}{}{ⓔsɤrchɤrchɤt}\relationsémantique{参考}{\lien{ⓔrchɤrchɤt}{rchɤrchɤt}}\end{entrée}

\begin{entrée}{sɤrchɯɣlɯɣ}{}{ⓔsɤrchɯɣlɯɣ}\relationsémantique{参考}{\lien{ⓔrchɯɣnɤlɯɣ}{rchɯɣnɤlɯɣ}}\end{entrée}

\begin{entrée}{sɤrchɯɣrchɯɣ}{}{ⓔsɤrchɯɣrchɯɣ}\relationsémantique{参考}{\lien{ⓔrchɯɣnɤlɯɣ}{rchɯɣnɤlɯɣ}}\end{entrée}

\begin{entrée}{sɤrɕo}{}{ⓔsɤrɕo}\relationsémantique{参考}{\lien{ⓔarɕo}{arɕo}}\end{entrée}

\begin{entrée}{sɤrɕɯβrɕɯβ}{}{ⓔsɤrɕɯβrɕɯβ}\relationsémantique{参考}{\lien{}{ɣɤrɕɯβrɕɯβ}}\end{entrée}

\begin{entrée}{sɤre}{}{ⓔsɤre} 
\classe{vs} \paradigme{dir}{tɤ-}\paradigme{dir}{thɯ-}\sens{1}
\begin{définition}\pfra{amusant}\end{définition}
\begin{définition}\pcmn{可笑}\end{définition}\sens{2}
\begin{définition}\pfra{excessif}\end{définition}
\begin{définition}\pcmn{过分}\end{définition}
\begin{exemple}\pjya{nɤ-tɯ-sɤre nɯ}\hspace{5pt}\pcmn{你好大的胆子}\end{exemple}\sens{3}
\begin{définition}\pfra{extrêmement}\end{définition}
\begin{définition}\pcmn{极度}\end{définition}
\begin{exemple}\pjya{a-xtu ɯ-tɯ-mŋɤm ɲɯ-sɤre ʑo}\hspace{5pt}\pcmn{我肚子非常痛}\end{exemple}
\begin{sous-entrée}{nɤsɤre}{ⓔsɤreⓢ3ⓝnɤsɤre} 
\classe{vt} 
\begin{définition}\pfra{trouver amusant}\end{définition}
\begin{définition}\pcmn{觉得好玩;觉得可笑}\end{définition}
\begin{exemple}\pjya{nɯ kɯ-fse tu-kɤ-ti nɯ aʑo ɲɯ-nɤsɤre-a}\hspace{5pt}\pcmn{我觉得那种说法很好笑}\end{exemple}\relationsémantique{参考}{\lien{ⓔnɤreⓗ1ⓢ2ⓝnɤre}{nɤre}}\relationsémantique{参考}{\lien{ⓔsɤrɤpa}{sɤrɤpa}}\relationsémantique{参考}{\lien{ⓔtɤ-re}{tɤ-re}}\end{sous-entrée}

\end{entrée}

\begin{entrée}{sɤrga}{}{ⓔsɤrga} 
\classe{vi}  
\grammaire{deexp} \paradigme{dir}{thɯ-}
\begin{définition}\pfra{aimable}\end{définition}
\begin{définition}\pcmn{可爱}\end{définition}
\begin{exemple}\pjya{jiɕqha nɯ ɲɯ-sɤrga}\hspace{5pt}\pcmn{那个(孩子)很可爱}\end{exemple}\relationsémantique{参考}{\lien{ⓔrgaⓗ1ⓝrga}{rga}}\relationsémantique{参考}{\lien{ⓔnɯrga}{nɯrga}}\end{entrée}

\begin{entrée}{sɤrɣɤβrɣɤβ}{}{ⓔsɤrɣɤβrɣɤβ}\relationsémantique{参考}{\lien{ⓔrɣɤβrɣɤβ}{rɣɤβrɣɤβ}}\end{entrée}

\begin{entrée}{sɤrɣi}{}{ⓔsɤrɣi}\relationsémantique{参考}{\lien{ⓔarɣi}{arɣi}}\end{entrée}

\begin{entrée}{sɤri}{}{ⓔsɤri} 
\classe{vt} \paradigme{dir}{tɤ-}\paradigme{dir}{tɤ-}
\begin{définition}\pfra{ajouter dans}\end{définition}
\begin{définition}\pcmn{加进;加入}\end{définition}
\begin{définition}\pfra{devenir membre, être sociable}\end{définition}
\begin{définition}\pcmn{加入;合群}\end{définition}
\begin{exemple}\pjya{mtɕhɤnmbrɯ ʑ-lɤ-sɤri-t-a}\hspace{5pt}\pcmn{我(把粮食)加进供给寺庙的供品里}\end{exemple}
\begin{exemple}\pjya{tɤlɤɕom pɯ-sɤri}\hspace{5pt}\pcmn{你把奶皮加进去}\end{exemple}
\begin{exemple}\pjya{khɯtsa ɯ-ŋgɯ kɯki ɯ-ro ki pɯ-sɤri-t-a}\hspace{5pt}\pcmn{我把剩下的加进了碗里}\end{exemple}
\begin{exemple}\pjya{kɯ-ʑɣɤsɤri ci ɲɯ-ŋu}\hspace{5pt}\pcmn{他是一个合群的人}\end{exemple}
\begin{exemple}\pjya{ndʑi-rca to-zɣɤsɤri}\hspace{5pt}\pcmn{他加入了他们俩的队伍}\end{exemple}
\begin{exemple}\pjya{aʑo tɤrca tɤ-ʑɣɤsɤri-a}\hspace{5pt}\pcmn{我加入其中了}\end{exemple}\relationsémantique{参考}{\lien{ⓔari}{ari}}\relationsémantique{同义词}{\lien{ⓔrkuⓝʑɣɤsɯrku}{ʑɣɤsɯrku}}
\begin{sous-entrée}{ʑɣɤsɤri}{ⓔsɤriⓝʑɣɤsɤri} 
\classe{vi}  
\grammaire{refl} \end{sous-entrée}

\end{entrée}

\begin{entrée}{sɤrju}{}{ⓔsɤrju}\relationsémantique{参考}{\lien{ⓔarju}{arju}}\end{entrée}

\begin{entrée}{sɤrɟɤsno}{}{ⓔsɤrɟɤsno} 
\classe{vt} \paradigme{dir}{thɯ-}
\begin{définition}\pfra{seller}\end{définition}
\begin{définition}\pcmn{套上马鞍}\end{définition}
\begin{exemple}\pjya{nɤʑo mbro ɕ-thɯ-sɤrɟɤsnɤm}\hspace{5pt}\pcmn{你给马套上马鞍}\end{exemple}\relationsémantique{参考}{\lien{ⓔtɤ-sno}{tɤ-sno}}\end{entrée}

\begin{entrée}{sɤrkhɯβrkhɯβ}{}{ⓔsɤrkhɯβrkhɯβ}\relationsémantique{参考}{\lien{ⓔnɤrkhɯrkhɯβ}{nɤrkhɯrkhɯβ}}\end{entrée}

\begin{entrée}{sɤrkɯrku}{}{ⓔsɤrkɯrku}\relationsémantique{参考}{\lien{ⓔrku}{rku}}\end{entrée}

\begin{entrée}{sɤrlɤɣrlɤɣ}{}{ⓔsɤrlɤɣrlɤɣ} 
\classe{vt} \paradigme{dir}{nɯ-}\paradigme{dir}{kɤ-}
\begin{définition}\pfra{secouer la tête}\end{définition}
\begin{définition}\pcmn{摇头}\end{définition}
\begin{exemple}\pjya{ɯ-ku ɲɯ-sɤrlɤɣrlɤɣ}\hspace{5pt}\pcmn{他在摇头}\end{exemple}
\begin{exemple}\pjya{nɯ-sɤrlɤɣrlɤɣ-a}\hspace{5pt}\pcmn{我摇头了}\end{exemple}\end{entrée}

\begin{entrée}{sɤrlɤn}{}{ⓔsɤrlɤn} 
\classe{n} 
\begin{définition}\pfra{trou humide}\end{définition}
\begin{définition}\pcmn{潮湿地}\end{définition}\étymologie{sa.rlon}\end{entrée}

\begin{entrée}{sɤrlɯrla}{}{ⓔsɤrlɯrla}\relationsémantique{参考}{\lien{ⓔarlɯrla}{arlɯrla}}\end{entrée}

\begin{entrée}{sɤrma}{}{ⓔsɤrma} 
\classe{adv} 
\begin{définition}\pfra{bonsoir}\end{définition}
\begin{définition}\pcmn{晚安}\end{définition}
\begin{exemple}\pjya{sɤrma-ndʑi je}\hspace{5pt}\pcmn{(你们俩)晚安!}\end{exemple}\end{entrée}

\begin{entrée}{sɤrmɤβrmɤβ}{}{ⓔsɤrmɤβrmɤβ}\relationsémantique{参考}{\lien{ⓔɣɤrmɤβrmɤβ}{ɣɤrmɤβrmɤβ}}\end{entrée}

\begin{entrée}{sɤrmbat}{}{ⓔsɤrmbat}\relationsémantique{参考}{\lien{ⓔarmbat}{armbat}}\end{entrée}

\begin{entrée}{sɤrmi}{}{ⓔsɤrmi} 
\classe{vt} \paradigme{dir}{tɤ-}\paradigme{dir}{tɤ-}
\begin{définition}\pfra{nommer}\end{définition}
\begin{définition}\pcmn{起名}\end{définition}
\begin{définition}\pfra{se choisir comme nom}\end{définition}
\begin{définition}\pcmn{给自己取名}\end{définition}
\begin{exemple}\pjya{kɯnɯβzaŋ to-sɤrmi-nɯ}\hspace{5pt}\pcmn{(他们叫她)她叫做根桑}\end{exemple}
\begin{exemple}\pjya{lɤβzaŋ to-sɤrmi-nɯ}\hspace{5pt}\pcmn{他们叫他洛桑}\end{exemple}\relationsémantique{参考}{\lien{ⓔrmi}{rmi}}\relationsémantique{参考}{\lien{ⓔtɤ-rmi}{tɤ-rmi}}
\begin{sous-entrée}{ʑɣɤsɤrmi}{ⓔsɤrmiⓝʑɣɤsɤrmi} 
\classe{vi} \end{sous-entrée}

\end{entrée}

\begin{entrée}{sɤrndzo}{}{ⓔsɤrndzo} 
\classe{n} 
\begin{définition}\pfra{baguette servant à séparer les fils (avant de tisser)}\end{définition}
\begin{définition}\pcmn{牵杆,用来牵线的工具}\end{définition}\end{entrée}

\begin{entrée}{sɤrɲɟɤle}{}{ⓔsɤrɲɟɤle}\relationsémantique{参考}{\lien{ⓔarɲɟɤle}{arɲɟɤle}}\end{entrée}

\begin{entrée}{sɤrɲɯɣrɲɯɣ}{}{ⓔsɤrɲɯɣrɲɯɣ}\relationsémantique{参考}{\lien{ⓔrɲɯɣrɲɯɣ}{rɲɯɣrɲɯɣ}}\end{entrée}

\begin{entrée}{sɤrŋɤɣndʑɯr}{}{ⓔsɤrŋɤɣndʑɯr} 
\classe{vt} \paradigme{dir}{nɯ-}
\begin{définition}\pfra{serrer les dents}\end{définition}
\begin{définition}\pcmn{咬牙,咬牙切齿}\end{définition}
\begin{exemple}\pjya{a-ɕɣa nɯ-sɤrŋɤɣndʑɯr-a}\hspace{5pt}\pcmn{我咬了牙}\end{exemple}
\begin{exemple}\pjya{nɤ-ɕɣa ɲɯ-tɯ-sɤrŋɤɣndʑɯr ʑo ɲɯ-ŋu}\hspace{5pt}\pcmn{你在咬牙}\end{exemple}\relationsémantique{参考}{\lien{ⓔrŋɤɣndʑɯr}{rŋɤɣndʑɯr}}\end{entrée}

\begin{entrée}{sɤrŋgɯŋga}{}{ⓔsɤrŋgɯŋga} 
\classe{n} 
\begin{définition}\pfra{couverture}\end{définition}
\begin{définition}\pcmn{被子;铺盖}\end{définition}\relationsémantique{参考}{\lien{ⓔtɯ-ŋga}{tɯ-ŋga}}\relationsémantique{参考}{\lien{ⓔrŋgɯⓗ2}{rŋgɯ}}\end{entrée}

\begin{entrée}{sɤrŋi}{}{ⓔsɤrŋi}\relationsémantique{参考}{\lien{ⓔarŋi}{arŋi}}\end{entrée}

\begin{entrée}{sɤrɴɢlɯm}{}{ⓔsɤrɴɢlɯm}\relationsémantique{参考}{\lien{ⓔarɴɢlɯm}{arɴɢlɯm}}\end{entrée}

\begin{entrée}{sɤro}{}{ⓔsɤro} 
\classe{vl} \paradigme{dir}{tɤ-}
\begin{définition}\pfra{poser sur un étalage}\end{définition}
\begin{définition}\pcmn{摆在架子上}\end{définition}
\begin{exemple}\pjya{ku-kɯ-tɣa tɕe, mɤro ɯ-taʁ tɤɕi qaj, stoʁ staχpɯ tú-wɣ-sɤro ŋu}\hspace{5pt}\pcmn{收割的时候,要把青稞、小麦、胡豆、豌豆都摆在架子上}\end{exemple}\end{entrée}

\begin{entrée}{sɤrphɤrphɤβ}{}{ⓔsɤrphɤrphɤβ}\relationsémantique{参考}{\lien{ⓔɣɤrphɤrphɤβ}{ɣɤrphɤrphɤβ}}\end{entrée}

\begin{entrée}{sɤrqhi}{}{ⓔsɤrqhi}\relationsémantique{参考}{\lien{ⓔarqhi}{arqhi}}\end{entrée}

\begin{entrée}{sɤrqhɯrqhɯβ}{}{ⓔsɤrqhɯrqhɯβ} 
\classe{vt} \paradigme{dir}{tɤ-}
\begin{définition}\pfra{bruit d'objets durs et secs qui s'entrechoquent}\end{définition}
\begin{définition}\pcmn{又干又硬的东西相撞发出声音}\end{définition}
\begin{exemple}\pjya{(paʁ kɯ) stoʁ ɲɯ-sɤrqhɯrqhɯβ ɲɯ-ɤsɯ-ndza}\hspace{5pt}\pcmn{(猪)在吃胡豆发出很多声音}\end{exemple}
\begin{exemple}\pjya{rdɤstaʁ ɲɯ-sɤrqhɯrqhɯβ pa-βde}\hspace{5pt}\pcmn{他把石头扔了,发出很多声音}\end{exemple}
\begin{exemple}\pjya{rdɤstaʁ ɲɯ-sɤrqhɯrqhɯβ ʑo tɯ-ci kɯ tha-tsɯm}\hspace{5pt}\pcmn{水把石头冲走了,发出很多声音}\end{exemple}\relationsémantique{参考}{\lien{ⓔɣɤrqhɯβrqhɯβ}{ɣɤrqhɯβrqhɯβ}}\end{entrée}

\begin{entrée}{sɤrʁɯrʁu}{}{ⓔsɤrʁɯrʁu}\relationsémantique{参考}{\lien{ⓔarʁɯrʁu}{arʁɯrʁu}}\end{entrée}

\begin{entrée}{sɤrʁɯrʁɯβ}{}{ⓔsɤrʁɯrʁɯβ} 
\classe{vt} \paradigme{dir}{tɤ-}
\begin{définition}\pfra{faire du bruit en grignotant une nourriture sèche à toute vitesse}\end{définition}
\begin{définition}\pcmn{吃又干又脆的东西时发出声音}\end{définition}
\begin{exemple}\pjya{tɤ-sɤrʁɯrʁɯβ-a tɤ-ndza-t-a}\hspace{5pt}\pcmn{我大声地吃了}\end{exemple}\relationsémantique{参考}{\lien{ⓔsɤrqhɯrqhɯβ}{sɤrqhɯrqhɯβ}}\relationsémantique{参考}{\lien{ⓔrʁɯβrʁɯβ}{rʁɯβrʁɯβ}}\end{entrée}

\begin{entrée}{sɤrtɕhɣaʁ}{}{ⓔsɤrtɕhɣaʁ} 
\classe{vt} \paradigme{dir}{nɯ-}
\begin{définition}\pfra{chicaner (à propos de quelquechose)}\end{définition}
\begin{définition}\pcmn{计较;找毛病(针对一件事)}\end{définition}
\begin{exemple}\pjya{tɯtɯrca ɕ-tu-rɯndzɤtshi-j pɯ-ŋu ri, ɯʑo kɯ na-sɤrtɕhɣaʁ}\hspace{5pt}\pcmn{我们本来要一起去吃饭,但是他在那里计较(没有跟我们一起)}\end{exemple}\relationsémantique{参考}{\lien{ⓔɣɤrtɕhɣaʁ}{ɣɤrtɕhɣaʁ}}\relationsémantique{同义词}{\lien{ⓔsɤtɕɯqaʁ}{sɤtɕɯqaʁ}}\relationsémantique{参考}{\lien{ⓔtɤ-rtɕhɣaʁ,tɕɤt}{tɤ-rtɕhɣaʁ,tɕɤt}}\end{entrée}

\begin{entrée}{sɤrtɕhoʁ}{}{ⓔsɤrtɕhoʁ}\relationsémantique{参考}{\lien{ⓔartɕhoʁ}{artɕhoʁ}}\end{entrée}

\begin{entrée}{sɤrtɕi}{}{ⓔsɤrtɕi}\relationsémantique{参考}{\lien{ⓔartɕi}{artɕi}}\end{entrée}

\begin{entrée}{sɤrtsi}{}{ⓔsɤrtsi}\relationsémantique{参考}{\lien{ⓔrtsi}{rtsi}}\end{entrée}

\begin{entrée}{sɤrtsɯrtso}{}{ⓔsɤrtsɯrtso} 
\classe{vt} \paradigme{dir}{tɤ-}
\begin{définition}\pfra{empiler}\end{définition}
\begin{définition}\pcmn{堆起来;堆整齐}\end{définition}
\begin{exemple}\pjya{laχtɕha ta-sɤrtsɯrtso}\hspace{5pt}\pcmn{他把东西堆起来了}\end{exemple}
\begin{exemple}\pjya{jɯɣi ta-sɤrtsɯrtso}\hspace{5pt}\pcmn{他把书堆起来了}\end{exemple}
\begin{exemple}\pjya{jɯɣi ra tɤ-sɤrtsɯrtso-t-a}\hspace{5pt}\pcmn{我把书堆起来了}\end{exemple}\relationsémantique{同义词}{\lien{ⓔrtsɯɣ}{rtsɯɣ}}\end{entrée}

\begin{entrée}{sɤrtɯmloʁ}{}{ⓔsɤrtɯmloʁ}\relationsémantique{参考}{\lien{ⓔartɯmloʁ}{artɯmloʁ}}\end{entrée}

\begin{entrée}{sɤrtɯrtɤβ}{}{ⓔsɤrtɯrtɤβ}\relationsémantique{参考}{\lien{ⓔartɯrtɤβ}{artɯrtɤβ}}\end{entrée}

\begin{entrée}{sɤrɯru}{}{ⓔsɤrɯru} 
\classe{vt} \paradigme{dir}{tɤ-}
\begin{définition}\pfra{comparer}\end{définition}
\begin{définition}\pcmn{比较,核对}\end{définition}
\begin{exemple}\pjya{tɤ-sɤrɯre}\hspace{5pt}\pcmn{你比较一下}\end{exemple}
\begin{exemple}\pjya{tɤ-sɤrɯru-t-a}\hspace{5pt}\pcmn{我比较了}\end{exemple}
\begin{exemple}\pjya{ɯʑo cho kɤ-sɤrɯru me}\hspace{5pt}\pcmn{你不要跟他比}\end{exemple}
\begin{exemple}\pjya{laχtɕha ʁnɯz ɯ-ɲɯ́-naxtɕɯɣ kɯ tú-wɣ-sɤrɯru}\hspace{5pt}\pcmn{比较一下两个东西是不是一样的}\end{exemple}
\begin{exemple}\pjya{ɕɯ ɣɯ kɯ ɲɯ-dɤn kɯ kɤ-sɤrɯru ɲɯ-ra}\hspace{5pt}\pcmn{要比较一下谁有最多}\end{exemple}
\begin{exemple}\pjya{tɕiʑo ni tú-wɣ-sɤrɯru tɕe, nɤʑo kɯ ɲɯ-tɯ-tshu}\hspace{5pt}\pcmn{我们俩做比较的话,你胖一些}\end{exemple}\end{entrée}

\begin{entrée}{sɤrwa}{}{ⓔsɤrwa} 
\classe{n} 
\begin{définition}\pfra{grêle}\end{définition}
\begin{définition}\pcmn{冰雹}\end{définition}
\begin{exemple}\pjya{jɯfɕɯndʐi sɤrwa chɤ-lɤt tɕe, tɤ-rɤku ra pjɤ-xtsɯ tɕe mɯ-to-sɤpe}\hspace{5pt}\pcmn{前天下了冰雹,破坏了庄稼}\end{exemple}\étymologie{ser.ba}\end{entrée}

\begin{entrée}{sɤrwɤrwɤt}{}{ⓔsɤrwɤrwɤt}\relationsémantique{参考}{\lien{ⓔɣɤrwɤrwɤt}{ɣɤrwɤrwɤt}}\end{entrée}

\begin{entrée}{sɤʁe}{}{ⓔsɤʁe}\relationsémantique{参考}{\lien{ⓔaʁe}{aʁe}}\end{entrée}

\begin{entrée}{sɤʁombi}{}{ⓔsɤʁombi} 
\classe{vs} 
\begin{définition}\pfra{être désespérant}\end{définition}
\begin{définition}\pcmn{令人没有希望}\end{définition}
\begin{exemple}\pjya{jiɕqha nɯ ɯ-kɯ-mŋɤm ɲɯ-thɯ kɤ-mna ɲɯ-sɤʁombi ɕti}\hspace{5pt}\pcmn{他的病很严重,没有痊愈的希望}\end{exemple}\relationsémantique{参考}{\lien{ⓔnɤʁombi}{nɤʁombi}}\end{entrée}

\begin{entrée}{sɤʁʑi}{}{ⓔsɤʁʑi} 
\classe{n} 
\begin{définition}\pfra{monde}\end{définition}
\begin{définition}\pcmn{世界}\end{définition}\étymologie{sa.gʑi}\end{entrée}

\begin{entrée}{sɤsaʁjɤr}{}{ⓔsɤsaʁjɤr}\relationsémantique{参考}{\lien{ⓔaʁjɤrⓝsaʁjɤr}{saʁjɤr}}\end{entrée}

\begin{entrée}{sɤsat}{₁₂}{ⓔsɤsatⓗ1ⓗ2} 
\classe{vi}
\classe{vs}  
\grammaire{apass} \paradigme{dir}{pɯ-}
\begin{définition}\pfra{tuer des gens}\end{définition}
\begin{définition}\pcmn{杀人}\end{définition}
\begin{exemple}\pjya{kɤ-sɤsat pjɤ-rɲo}\hspace{5pt}\pcmn{他曾经杀过人}\end{exemple}
\begin{sous-entrée}{sɤsat}{ⓔsɤsatⓗ1ⓝsɤsat}\end{sous-entrée}

\begin{définition}\pfra{mortel (arme)}\end{définition}
\begin{définition}\pcmn{杀伤力强}\end{définition}
\begin{exemple}\pjya{ki ɕɤmɯɣdɯ ki ɲɯ-sɤsat}\hspace{5pt}\pcmn{这把枪杀伤力强}\end{exemple}\relationsémantique{参考}{\lien{ⓔsat}{sat}}\end{entrée}

\begin{entrée}{sɤsaχpaʁ}{}{ⓔsɤsaχpaʁ}\relationsémantique{参考}{\lien{ⓔsaχpaʁ}{saχpaʁ}}\end{entrée}

\begin{entrée}{sɤsɤŋo}{}{ⓔsɤsɤŋo} 
\classe{vi}  
\grammaire{apass} \paradigme{dir}{tɤ-}
\begin{définition}\pfra{qui écoute les conseils, obéissant}\end{définition}
\begin{définition}\pcmn{听别人的劝告;听话}\end{définition}
\begin{exemple}\pjya{ɲɯ-sɤsɤŋo}\hspace{5pt}\pcmn{他听别人的意见}\end{exemple}\relationsémantique{参考}{\lien{}{sɤŋo₁}}\end{entrée}

\begin{entrée}{sɤschrɤβlɤβ}{}{ⓔsɤschrɤβlɤβ}\relationsémantique{参考}{\lien{ⓔɣɤchrɤβchrɤβ}{ɣɤchrɤβchrɤβ}}\end{entrée}

\begin{entrée}{sɤscit}{}{ⓔsɤscit} 
\classe{vs} \paradigme{dir}{thɯ-}
\begin{définition}\pfra{heureux, agréable (environnement, époque) amusant (personne)}\end{définition}
\begin{définition}\pcmn{幸福(环境、时代、生活)、好玩(人)}\end{définition}
\begin{exemple}\pjya{ji-tɯrma ɲɯ-sɤscit}\end{exemple}\relationsémantique{参考}{\lien{ⓔscit}{scit}}\relationsémantique{参考}{\lien{ⓔnɤsɤscit}{nɤsɤscit}}\étymologie{skʲid}\end{entrée}

\begin{entrée}{sɤsco}{}{ⓔsɤsco}\relationsémantique{参考}{\lien{ⓔsco}{sco}}\end{entrée}

\begin{entrée}{sɤscur}{}{ⓔsɤscur} 
\classe{n} 
\begin{définition}\pfra{lanière}\end{définition}
\begin{définition}\pcmn{背带}\end{définition}\end{entrée}

\begin{entrée}{sɤscɯndu}{}{ⓔsɤscɯndu} 
\classe{vt} \paradigme{dir}{tɤ-}\paradigme{dir}{pɯ-}
\begin{définition}\pfra{échanger}\end{définition}
\begin{définition}\pcmn{调换}\end{définition}
\begin{exemple}\pjya{tɕi-ŋga tɤ-nɯ-sɤscɯndu-tɕi}\hspace{5pt}\pcmn{我们调换了衣服}\end{exemple}
\begin{exemple}\pjya{tɯ-rju ɯ-qhu ɯ-ʁɤri ɲɤ-sɤscɯndu-t-a}\hspace{5pt}\pcmn{我不小心颠倒说话了}\end{exemple}\relationsémantique{同义词}{\lien{ⓔsɤkɤsci}{sɤkɤsci}}\relationsémantique{参考}{\lien{ⓔsɤndu}{sɤndu}}\end{entrée}

\begin{entrée}{sɤskɤt}{}{ⓔsɤskɤt}\relationsémantique{参考}{\lien{ⓔskɤt}{skɤt}}\end{entrée}

\begin{entrée}{sɤskɯsku}{}{ⓔsɤskɯsku} 
\classe{adv} 
\begin{définition}\pfra{tous les matins}\end{définition}
\begin{définition}\pcmn{每天早上}\end{définition}\end{entrée}

\begin{entrée}{sɤsma}{}{ⓔsɤsma}\relationsémantique{参考}{\lien{ⓔnɤsma}{nɤsma}}\end{entrée}

\begin{entrée}{sɤsŋom}{}{ⓔsɤsŋom}\relationsémantique{参考}{\lien{ⓔsŋom}{sŋom}}\end{entrée}

\begin{entrée}{sɤspa}{}{ⓔsɤspa}\relationsémantique{参考}{\lien{ⓔspa}{spa}}\end{entrée}

\begin{entrée}{sɤsphɯt}{}{ⓔsɤsphɯt}\relationsémantique{参考}{\lien{ⓔphɯtⓝsphɯt}{sphɯt}}\end{entrée}

\begin{entrée}{sɤsqɤr}{}{ⓔsɤsqɤr}\relationsémantique{参考}{\lien{ⓔsqɤr}{sqɤr}}\end{entrée}

\begin{entrée}{sɤsqra}{}{ⓔsɤsqra} 
\classe{n} 
\begin{définition}\pfra{limite}\end{définition}
\begin{définition}\pcmn{界限}\end{définition}\end{entrée}

\begin{entrée}{sɤstu}{₁}{ⓔsɤstuⓗ1} 
\classe{vs} 
\begin{définition}\pfra{inspirer confiance}\end{définition}
\begin{définition}\pcmn{令人相信}\end{définition}
\begin{exemple}\pjya{ɯ-phɯ chondɤre ɯ-@zhiliang nɯra kɯ-sɤstu tsa nɯ tɕu ɕ-tú-wɣ-χtɯ ra}\hspace{5pt}\pcmn{要在价格和质量都令人信服的地方买}\end{exemple}\relationsémantique{参考}{\lien{}{stu}}\end{entrée}

\begin{entrée}{sɤstu}{₂}{ⓔsɤstuⓗ2} 
\classe{vt} \sens{1}\paradigme{dir}{tɤ-}
\begin{définition}\pfra{soutenir, faire correctement un travail}\end{définition}
\begin{définition}\pcmn{端平;把任务作好}\end{définition}
\begin{exemple}\pjya{khɯsta tɤ-sɤste ma tɤ-lwoʁ}\hspace{5pt}\pcmn{你要把碗端平不然就会倒出来}\end{exemple}\sens{2}\paradigme{dir}{\_}
\begin{définition}\pfra{aller directement}\end{définition}
\begin{définition}\pcmn{直(走);直接……}\end{définition}
\begin{exemple}\pjya{lú-wɣ-sɤstu ʑo lu-kɯ-ɕe qhe lu-kɯ-zɣɯt ɕti}\hspace{5pt}\pcmn{直接往前就会到}\end{exemple}\relationsémantique{参考}{\lien{ⓔastu}{astu}}\relationsémantique{参考}{\lien{ⓔsɤstɤko}{sɤstɤko}}\end{entrée}

\begin{entrée}{sɤstɤko}{}{ⓔsɤstɤko} 
\classe{vt} \paradigme{dir}{thɯ-}\paradigme{dir}{\_}
\begin{définition}\pfra{tendre}\end{définition}
\begin{définition}\pcmn{伸直}\end{définition}
\begin{exemple}\pjya{ɯ-mi tha-sɤstɤko}\hspace{5pt}\pcmn{他伸了脚}\end{exemple}
\begin{exemple}\pjya{ɕom tha-sɤstɤko}\hspace{5pt}\pcmn{他把铁打成直条了}\end{exemple}
\begin{exemple}\pjya{nɤ-βri ra nɯ-sɤstɤkɤm}\hspace{5pt}\pcmn{你舒展一下筋骨吧}\end{exemple}\relationsémantique{参考}{\lien{ⓔastɤko}{astɤko}}
\begin{sous-entrée}{ʑɣɤsɤstɤko}{ⓔsɤstɤkoⓝʑɣɤsɤstɤko} 
\classe{vi}  
\grammaire{refl} 
\begin{définition}\pfra{tendre son corps}\end{définition}
\begin{définition}\pcmn{舒展筋骨}\end{définition}\relationsémantique{参考}{\lien{ⓔastu}{astu}}\relationsémantique{参考}{\lien{ⓔsɤstuⓗ2}{sɤstu₂}}\end{sous-entrée}

\end{entrée}

\begin{entrée}{sɤstoŋ}{}{ⓔsɤstoŋ} 
\classe{n} 
\begin{définition}\pfra{endroit inhabité}\end{définition}
\begin{définition}\pcmn{荒野}\end{définition}\étymologie{sa.stoŋ}\end{entrée}

\begin{entrée}{sɤsɯβzi}{}{ⓔsɤsɯβzi}\relationsémantique{参考}{\lien{ⓔβzi}{βzi}}\end{entrée}

\begin{entrée}{sɤsɯɣ}{}{ⓔsɤsɯɣ} 
\classe{vt}  
\grammaire{caus}
\grammaire{refl} \paradigme{dir}{kɤ-}\paradigme{dir}{tɤ-}\paradigme{dir}{tɤ-}
\begin{définition}\pfra{presser, serrer}\end{définition}
\begin{définition}\pcmn{挤}\end{définition}
\begin{exemple}\pjya{ɯ-mtɕhi ka-sɤsɯɣ}\hspace{5pt}\pcmn{他抿了嘴巴}\end{exemple}
\begin{exemple}\pjya{nɤ-jaʁ kɤ-sɤsɯɣ}\hspace{5pt}\pcmn{你把拳头握紧}\end{exemple}
\begin{exemple}\pjya{tɤ-mtɯ kɤ-sɤsɯɣ}\hspace{5pt}\pcmn{把结打紧}\end{exemple}
\begin{exemple}\pjya{a-mthɤɣ kɤ-sɤsɯɣ-a}\hspace{5pt}\pcmn{我把腰带束紧了}\end{exemple}
\begin{exemple}\pjya{a-fkur tɤ-sɤsɯɣ-a}\hspace{5pt}\pcmn{我把背子捆紧了}\end{exemple}
\begin{exemple}\pjya{tɤ-fkɯm ɯ-mŋu kɤ-sɤsɯɣ-a}\hspace{5pt}\pcmn{我把袋子的口收紧了}\end{exemple}
\begin{sous-entrée}{ʑɣɤsɤsɯɣ}{ⓔsɤsɯɣⓝʑɣɤsɤsɯɣ} 
\classe{vi} \end{sous-entrée}

\begin{définition}\pfra{faire le maximum}\end{définition}
\begin{définition}\pcmn{努力;抓紧时间}\end{définition}
\begin{exemple}\pjya{tɯ-βzjoz kɤ-ʑɣɤsɤsɯɣ ra}\hspace{5pt}\pcmn{要努力学习}\end{exemple}\relationsémantique{参考}{\lien{ⓔasɯɣ}{asɯɣ}}\end{entrée}

\begin{entrée}{sɤsɯɣli}{}{ⓔsɤsɯɣli}\relationsémantique{参考}{\lien{ⓔliⓗ3}{li₃}}\end{entrée}

\begin{entrée}{sɤsɯɣsɯɣ}{}{ⓔsɤsɯɣsɯɣ} 
\classe{vt} \paradigme{dir}{nɯ-}
\begin{définition}\pfra{toucher avec ...}\end{définition}
\begin{définition}\pcmn{用……轻轻地擦过、碰到}\end{définition}
\begin{exemple}\pjya{nɤ-ŋga nɯ znde ɯ-taʁ ma-nɯ-tɯ-sɤsɯɣsɯɣ ma sɤɴqhi}\hspace{5pt}\pcmn{你不要把衣服碰到墙上,很脏}\end{exemple}\end{entrée}

\begin{entrée}{sɤsɯxɕɤt}{}{ⓔsɤsɯxɕɤt}\relationsémantique{参考}{\lien{ⓔsɯxɕɤt}{sɯxɕɤt}}\end{entrée}

\begin{entrée}{sɤsɯz}{}{ⓔsɤsɯz} 
\classe{vs}  
\grammaire{deexp} 
\begin{définition}\pfra{être connu}\end{définition}
\begin{définition}\pcmn{(人们)知道的}\end{définition}
\begin{exemple}\pjya{ɯ-rmi ɲɯ-sɤsɯz}\hspace{5pt}\pcmn{人们知道他的名字}\end{exemple}\relationsémantique{参考}{\lien{ⓔsɯz}{sɯz}}\end{entrée}

\begin{entrée}{sɤsɯzdɯɣ}{}{ⓔsɤsɯzdɯɣ}\relationsémantique{参考}{\lien{ⓔsɯzdɯɣ}{sɯzdɯɣ}}\end{entrée}

\begin{entrée}{sɤʂɤʂɤt}{}{ⓔsɤʂɤʂɤt} 
\classe{vt} \paradigme{dir}{tɤ-}\paradigme{dir}{kɤ-}
\begin{définition}\pfra{lire / écrire de manière très fluide}\end{définition}
\begin{définition}\pcmn{念/写得很流利、吊羊毛吊得很顺手}\end{définition}
\begin{exemple}\pjya{thɯ-rɤrɤt ɲɯ-sɤʂɤʂɤt ʑo}\hspace{5pt}\pcmn{他写了,写得很流利}\end{exemple}
\begin{exemple}\pjya{kɤ-pɣo ta-sɤʂɤʂɤt ʑo}\hspace{5pt}\pcmn{他吊了羊毛,吊得很顺手}\end{exemple}\end{entrée}

\begin{entrée}{sɤʂχɯʂχɯβ}{}{ⓔsɤʂχɯʂχɯβ} 
\classe{vt} \paradigme{dir}{kɤ-}\paradigme{dir}{kɤ-}
\begin{définition}\pfra{siroter, froisser}\end{définition}
\begin{définition}\pcmn{(喝水时)发出啧啧声音,发出沙沙声}\end{définition}
\begin{définition}\pfra{émettre un bruit de froissement}\end{définition}
\begin{définition}\pcmn{发出沙沙声}\end{définition}
\begin{exemple}\pjya{tʂha kɤ-sɤʂχɯʂχɯβ}\hspace{5pt}\pcmn{喝茶时发出啧啧声}\end{exemple}
\begin{exemple}\pjya{ɕoʁɕoʁ ɲɯ-sɤʂχɯʂχɯβ}\hspace{5pt}\pcmn{他把纸弄皱发出沙沙声}\end{exemple}
\begin{exemple}\pjya{tɯ-ndʐi nɯ ɲɯ-ɣɤʂχɯʂχɯβ (kɤ-χtsɤβ mɯ-pjɤ-βdi)}\hspace{5pt}\pcmn{皮子发出沙沙声(表示没有揉好、很粗糙)}\end{exemple}\relationsémantique{参考}{\lien{ⓔɣɤʂχaβʂχaβ}{ɣɤʂχaβʂχaβ}}
\begin{sous-entrée}{ɣɤʂχɯʂχɯβ}{ⓔsɤʂχɯʂχɯβⓝɣɤʂχɯʂχɯβ} 
\classe{vi} \end{sous-entrée}

\end{entrée}

\begin{entrée}{sɤtaʁki}{}{ⓔsɤtaʁki}\relationsémantique{参考}{\lien{ⓔataʁki}{ataʁki}}\end{entrée}

\begin{entrée}{sɤtaʁtaʁ}{}{ⓔsɤtaʁtaʁ} 
\classe{vt} \paradigme{dir}{tɤ-}
\begin{définition}\pfra{amasser}\end{définition}
\begin{définition}\pcmn{堆迭}\end{définition}
\begin{exemple}\pjya{laχtɕha tɤ-sɤtaʁtaʁ}\hspace{5pt}\pcmn{你把东西堆起来}\end{exemple}
\begin{exemple}\pjya{tɯjpu tɤ-sɤtaʁtaʁ}\hspace{5pt}\pcmn{你把面堆起来}\end{exemple}
\begin{exemple}\pjya{jɯɣi tɤ-sɤtaʁtaʁ}\hspace{5pt}\pcmn{你把书堆起来}\end{exemple}\end{entrée}

\begin{entrée}{sɤtɤβ}{}{ⓔsɤtɤβ} 
\classe{n} 
\begin{définition}\pfra{aire à battre}\end{définition}
\begin{définition}\pcmn{打场}\end{définition}\end{entrée}

\begin{entrée}{sɤtɕaʁ}{}{ⓔsɤtɕaʁ}\relationsémantique{参考}{\lien{ⓔatɕaʁ}{atɕaʁ}}\end{entrée}

\begin{entrée}{sɤtɕaʁlaʁ}{}{ⓔsɤtɕaʁlaʁ}\relationsémantique{参考}{\lien{ⓔatɕaʁ}{atɕaʁ}}\end{entrée}

\begin{entrée}{sɤtɕɤβ}{}{ⓔsɤtɕɤβ}\relationsémantique{参考}{\lien{ⓔatɕɤβ}{atɕɤβ}}\end{entrée}

\begin{entrée}{sɤtɕɤt}{}{ⓔsɤtɕɤt} 
\classe{vt} \paradigme{dir}{thɯ-}
\begin{définition}\pfra{attiser le feu}\end{définition}
\begin{définition}\pcmn{把柴推进火里,把火拨旺}\end{définition}
\begin{exemple}\pjya{smi thɯ-sɤtɕɤt}\hspace{5pt}\pcmn{你把火拨旺吧}\end{exemple}
\begin{exemple}\pjya{smi kɤ-βlɯ ɲɯ-ra tɕe, nɤʑo smi thɯ-sɤtɕɤt ɲɯ-ntshi}\hspace{5pt}\pcmn{因为要烧火,请你把柴火推一下}\end{exemple}\end{entrée}

\begin{entrée}{sɤtɕɣɤrtɕɣɤr}{}{ⓔsɤtɕɣɤrtɕɣɤr}\relationsémantique{参考}{\lien{ⓔtɕɣɤrtɕɣɤr}{tɕɣɤrtɕɣɤr}}\end{entrée}

\begin{entrée}{sɤtɕha}{}{ⓔsɤtɕha} 
\classe{n} 
\begin{définition}\pfra{endroit}\end{définition}
\begin{définition}\pcmn{地方}\end{définition}\étymologie{sa.tɕʰa}\end{entrée}

\begin{entrée}{sɤtɕhɯ}{}{ⓔsɤtɕhɯ} 
\classe{vi}  
\grammaire{apass} \paradigme{dir}{tɤ-}
\begin{définition}\pfra{attaquer avec ses cornes}\end{définition}
\begin{définition}\pcmn{用角打人(牛)}\end{définition}
\begin{exemple}\pjya{mbala kɯ-sɤtɕhɯ}\hspace{5pt}\pcmn{顶人的公牛}\end{exemple}\relationsémantique{参考}{\lien{ⓔtɕhɯ}{tɕhɯ}}\end{entrée}

\begin{entrée}{sɤtɕhɯβtɕhɯβ}{}{ⓔsɤtɕhɯβtɕhɯβ} 
\classe{vt} \paradigme{dir}{pɯ-}\paradigme{dir}{tɤ-}
\begin{définition}\pfra{cligner des yeux}\end{définition}
\begin{définition}\pcmn{眨眼}\end{définition}
\begin{exemple}\pjya{ɯ-mɲaʁ ɲɯ-sɤtɕhɯβtɕhɯβ}\hspace{5pt}\pcmn{他在眨眼}\end{exemple}
\begin{exemple}\pjya{a-mɲaʁ tɤ-sɤtɕhɯβtɕhɯβ-a (pɯ-sɤtɕhɯβtɕhɯβ-a)}\hspace{5pt}\pcmn{我眨了眼}\end{exemple}\relationsémantique{参考}{\lien{ⓔsɤthɤβthɤβ}{sɤthɤβthɤβ}}\end{entrée}

\begin{entrée}{sɤtɕhɯtɕhɯ}{}{ⓔsɤtɕhɯtɕhɯ} 
\classe{vt} \paradigme{dir}{thɯ-}
\begin{définition}\pfra{se couvrir (de plusieurs couches de vêtements)}\end{définition}
\begin{définition}\pcmn{套衣服}\end{définition}
\begin{exemple}\pjya{tɯ-ŋga thɯ-sɤtɕhɯtɕhi}\hspace{5pt}\pcmn{你在上面套上衣服!}\end{exemple}
\begin{exemple}\pjya{tɯ-ŋga kɤntɕhɯ thɯ-sɤtɕhɯtɕhɯ-t-a}\hspace{5pt}\pcmn{我套了很多件衣服}\end{exemple}\end{entrée}

\begin{entrée}{sɤtɕhɯz}{}{ⓔsɤtɕhɯz}\relationsémantique{参考}{\lien{ⓔatɕhɯz}{atɕhɯz}}\end{entrée}

\begin{entrée}{sɤtɕɯɣtaʁ}{}{ⓔsɤtɕɯɣtaʁ}\relationsémantique{参考}{\lien{ⓔtaʁⓗ2}{taʁ₂}}\end{entrée}

\begin{entrée}{sɤtɕɯmthɯt}{}{ⓔsɤtɕɯmthɯt} 
\classe{vt} \paradigme{dir}{kɤ-}
\begin{définition}\pfra{mettre ensemble des objets séparés}\end{définition}
\begin{définition}\pcmn{把零散的东西(线、布片)拼成整体}\end{définition}
\begin{exemple}\pjya{ka-sɤtɕɯmthɯt}\hspace{5pt}\pcmn{他拼在一起了}\end{exemple}
\begin{exemple}\pjya{ɕomskrɯt kɤ-sɤtɕɯmthɯt-a}\hspace{5pt}\pcmn{我把铁丝拼在一起了}\end{exemple}
\begin{exemple}\pjya{tɯ-ŋga kɤ-sɤtɕɯmthɯt}\hspace{5pt}\pcmn{把(零碎的布片)拼在一起,做成一件衣服}\end{exemple}
\begin{exemple}\pjya{tɤ-ri kɤ-sɤtɕɯmthɯt}\hspace{5pt}\pcmn{把线拼在一起}\end{exemple}\relationsémantique{同义词}{\lien{ⓔsɤlɤɣɯ}{sɤlɤɣɯ}}\relationsémantique{同义词}{\lien{ⓔsɤthɤri}{sɤthɤri}}\end{entrée}

\begin{entrée}{sɤtɕɯqaʁ}{}{ⓔsɤtɕɯqaʁ} 
\classe{vt} \paradigme{dir}{nɯ-}\paradigme{dir}{pɯ-}
\begin{définition}\pfra{chicaner, s'opposer à}\end{définition}
\begin{définition}\pcmn{计较;反驳;找毛病;评理}\end{définition}
\begin{définition}\pfra{chicaner}\end{définition}
\begin{définition}\pcmn{计较;反驳}\end{définition}
\begin{exemple}\pjya{nɯ-sɤtɕɯqaʁ-a}\hspace{5pt}\pcmn{我跟他计较了}\end{exemple}
\begin{exemple}\pjya{mbro jla ɲɯ-sɤtɕɯqaʁ}\hspace{5pt}\pcmn{他计较马和犏牛的事情}\end{exemple}
\begin{exemple}\pjya{ɯ-rɟɯ ɲɯ-sɤtɕɯqaʁ}\hspace{5pt}\pcmn{他计较他的财产}\end{exemple}
\begin{exemple}\pjya{nɤki tɯrme ɯ-kɤ-sɤtɕɯqaʁ dɤn}\hspace{5pt}\pcmn{那个人很爱计较}\end{exemple}
\begin{exemple}\pjya{ki tɤ-kɤ-tɯt nɯ tɤ-ste, ma-nɯ-tɯ-sɤtɕɯqaʁ}\hspace{5pt}\pcmn{你要按照他说的去做,不要反驳}\end{exemple}
\begin{exemple}\pjya{nɤ-kɤ-sɤtɕɯqaʁ ʁɟa ʑo ɲɯ-ɕti}\hspace{5pt}\pcmn{你所说的话全部都是跟人家计较的}\end{exemple}
\begin{exemple}\pjya{pɯ-ɣɤtɕɯqaʁ-a}\hspace{5pt}\pcmn{我计较了}\end{exemple}\relationsémantique{同义词}{\lien{ⓔsɤrtɕhɣaʁ}{sɤrtɕhɣaʁ}}\relationsémantique{参考}{\lien{ⓔɣɤʁrɯqa}{ɣɤʁrɯqa}}
\begin{sous-entrée}{ɣɤtɕɯqaʁ}{ⓔsɤtɕɯqaʁⓝɣɤtɕɯqaʁ} 
\classe{vi} \end{sous-entrée}

\end{entrée}

\begin{entrée}{sɤtɕɯtɕit}{}{ⓔsɤtɕɯtɕit}\relationsémantique{参考}{\lien{ⓔatɕɯtɕit}{atɕɯtɕit}}\end{entrée}

\begin{entrée}{sɤtɕɯtʂi}{}{ⓔsɤtɕɯtʂi} 
\classe{vt} \paradigme{dir}{\_}
\begin{définition}\pfra{continuer, aller directement sans s'arrêter, en profiter pour faire quelque chose d'autre}\end{définition}
\begin{définition}\pcmn{继续;直接过去(可以停的地方没有停),顺便做另外一件事}\end{définition}
\begin{exemple}\pjya{tha-sɤtɕɯtʂi, la-sɤtɕɯtʂi}\hspace{5pt}\pcmn{他顺便带了}\end{exemple}
\begin{exemple}\pjya{tɯpri kɤ-ti tɤ-sɤtɕɯtʂi-t-a}\hspace{5pt}\pcmn{我顺便转告了口信}\end{exemple}
\begin{exemple}\pjya{ma-tɤ-tɯ-znɯne kɯ pɯ-sɤtɕɯtʂi}\hspace{5pt}\pcmn{你不要停下來,一定要做下去}\end{exemple}
\begin{exemple}\pjya{tɤ-nɯsɤtɕɯtʂi-nɯ jɤɣ ma aʑɯɣ mɤ-ʁdɯɣ}\hspace{5pt}\pcmn{你们继续(吃),不用管我}\end{exemple}
\begin{exemple}\pjya{kɤ-nɤma tɤ-sɤtɕɯtʂi jɤɣ}\hspace{5pt}\pcmn{你继续工作吧}\end{exemple}\end{entrée}

\begin{entrée}{sɤtɕɯxtʂot}{}{ⓔsɤtɕɯxtʂot}\relationsémantique{参考}{\lien{ⓔatɕɯxtʂot}{atɕɯxtʂot}}\end{entrée}

\begin{entrée}{sɤthɤβthɤβ}{}{ⓔsɤthɤβthɤβ} 
\classe{vt} 
\begin{définition}\pfra{cligner des yeux}\end{définition}
\begin{définition}\pcmn{眨眼(快)}\end{définition}
\begin{exemple}\pjya{ɯ-mɲaʁ ta-sɤthɤβthɤβ}\hspace{5pt}\pcmn{他眨了眼}\end{exemple}
\begin{sous-entrée}{ɣɤthɤβthɤβ}{ⓔsɤthɤβthɤβⓝɣɤthɤβthɤβ} 
\classe{vi} 
\begin{définition}\pfra{faire l'effronté derrière le dos}\end{définition}
\begin{définition}\pcmn{悄悄地顶嘴}\end{définition}
\begin{exemple}\pjya{ma-tɯ-ɣɤthɤβthɤβ}\hspace{5pt}\pcmn{你不要悄悄地顶嘴}\end{exemple}\relationsémantique{同义词}{\lien{ⓔsɤtɕhɯβtɕhɯβ}{sɤtɕhɯβtɕhɯβ}}\end{sous-entrée}

\end{entrée}

\begin{entrée}{sɤthɤri}{}{ⓔsɤthɤri} 
\classe{vt} \paradigme{dir}{nɯ-}
\begin{définition}\pfra{connecter, rattacher}\end{définition}
\begin{définition}\pcmn{连接}\end{définition}
\begin{exemple}\pjya{tɯmbri kɤ-sɤthɤri}\hspace{5pt}\pcmn{把(几根)绳子连接起来}\end{exemple}
\begin{exemple}\pjya{nɤki tɤ-ri nɯ nɯ-sɤthɤri-t-a}\hspace{5pt}\pcmn{我把那几根线连接在一起}\end{exemple}
\begin{exemple}\pjya{tɤ-pɤtso kɯ ji-xtsa ɲɯ-sɤthɤri}\hspace{5pt}\pcmn{小孩子把我们的鞋子系在一起}\end{exemple}\relationsémantique{参考}{\lien{ⓔathɤri}{athɤri}}\relationsémantique{同义词}{\lien{ⓔsɤtɕɯmthɯt}{sɤtɕɯmthɯt}}\relationsémantique{同义词}{\lien{ⓔsɤlɤɣɯ}{sɤlɤɣɯ}}\end{entrée}

\begin{entrée}{sɤthɣɤthɣɤt}{}{ⓔsɤthɣɤthɣɤt}\relationsémantique{参考}{\lien{ⓔɣɤthɣɤthɣɤt}{ɣɤthɣɤthɣɤt}}\end{entrée}

\begin{entrée}{sɤthoʁmphrɤt}{}{ⓔsɤthoʁmphrɤt} 
\classe{vt} \paradigme{dir}{tɤ-}
\begin{définition}\pfra{installer, mettre ensemble les pièces d'une machine, d'un habit de manière adéquate}\end{définition}
\begin{définition}\pcmn{把零件组装;令零件相吻合}\end{définition}
\begin{exemple}\pjya{tʂɤm ta-sɤthoʁmphrɤt}\hspace{5pt}\pcmn{他把板壁组装了}\end{exemple}
\begin{exemple}\pjya{mkhɯrlu ta-sɤthoʁmphrɤt}\hspace{5pt}\pcmn{他把机器组装了}\end{exemple}\relationsémantique{参考}{\lien{ⓔathoʁmphrɤt}{athoʁmphrɤt}}\end{entrée}

\begin{entrée}{sɤtsu}{}{ⓔsɤtsu} 
\classe{vs} 
\begin{définition}\pfra{que l'on a le temps de faire}\end{définition}
\begin{définition}\pcmn{来得及做的}\end{définition}
\begin{exemple}\pjya{fso tɕe ɕɯ-kɤ-nɤmɲo ɲɯ-sɤtso}\hspace{5pt}\pcmn{明天有时间去看(节目)}\end{exemple}\relationsémantique{参考}{\lien{ⓔtsu}{tsu}}\end{entrée}

\begin{entrée}{sɤtsa}{}{ⓔsɤtsa}\relationsémantique{参考}{\lien{ⓔatsa}{atsa}}\end{entrée}

\begin{entrée}{sɤtso}{}{ⓔsɤtso}\relationsémantique{参考}{\lien{ⓔtso}{tso}}\end{entrée}

\begin{entrée}{sɤtʂu}{}{ⓔsɤtʂu} 
\classe{vt} \paradigme{dir}{tɤ-}
\begin{définition}\pfra{illuminer avec une lampe}\end{définition}
\begin{définition}\pcmn{用灯照亮}\end{définition}
\begin{exemple}\pjya{ɣɟɯ kɯngɯ-rtsɤɣ tu ri kɤ-sɤtʂu khɯ, tɯrme kɯngɯ-tɣa ma me ri, kɤ-sɤtʂu mɤ-khɯ}\hspace{5pt}\pcmn{碉楼虽然有九层高可以用灯照亮,人虽然只有九拃高,但是不能用灯照亮(人心难测)}\end{exemple}\relationsémantique{参考}{\lien{ⓔtɤtʂu}{tɤtʂu}}\end{entrée}

\begin{entrée}{sɤtʂoʁloʁ}{}{ⓔsɤtʂoʁloʁ}\relationsémantique{参考}{\lien{ⓔatʂoʁloʁ}{atʂoʁloʁ}}\end{entrée}

\begin{entrée}{sɤtɯta}{}{ⓔsɤtɯta}\relationsémantique{参考}{\lien{ⓔatɯta}{atɯta}}\end{entrée}

\begin{entrée}{sɤwi/\variante{sɤwij}}{}{ⓔsɤwi} 
\classe{vt} \paradigme{dir}{pɯ-}\paradigme{dir}{kɤ-}
\begin{définition}\pfra{fermer (yeux)}\end{définition}
\begin{définition}\pcmn{闭上眼睛}\end{définition}
\begin{exemple}\pjya{ɯ-mɲaʁ ka-sɤwi}\hspace{5pt}\pcmn{他闭上眼睛了}\end{exemple}
\begin{exemple}\pjya{a-mɲaʁ kɤ-sɤwi-t-a ri a-ʑɯβ kɤ-sɯɣe mɯ́j-khɯ}\hspace{5pt}\pcmn{我虽然闭上眼睛,还是睡不着}\end{exemple}
\begin{exemple}\pjya{ɯ-mɲaʁ ra ko-sɤwi}\hspace{5pt}\pcmn{他闭上眼睛了}\end{exemple}\relationsémantique{参考}{\lien{ⓔawi}{awi}}\end{entrée}

\begin{entrée}{sɤwi}{}{ⓔsɤwi}\relationsémantique{参考}{\lien{ⓔawi}{awi}}\end{entrée}

\begin{entrée}{sɤwija}{}{ⓔsɤwija} 
\classe{vt} \paradigme{dir}{kɤ-}
\begin{définition}\pfra{enrouler autour du fuseau}\end{définition}
\begin{définition}\pcmn{缠在纺锤上}\end{définition}
\begin{exemple}\pjya{tɤ-ri ka-sɤwija}\hspace{5pt}\pcmn{他把线缠在纺锤上了}\end{exemple}\end{entrée}

\begin{entrée}{sɤwum}{}{ⓔsɤwum} 
\classe{vt} \paradigme{dir}{tɤ-}\paradigme{dir}{pɯ-}\paradigme{dir}{tɤ-}
\begin{définition}\pfra{ranger ensemble, fermer la bouche}\end{définition}
\begin{définition}\pcmn{收起來;闭嘴}\end{définition}
\begin{définition}\pfra{ranger ensemble}\end{définition}
\begin{définition}\pcmn{收拾在一起}\end{définition}
\begin{exemple}\pjya{nɤmtɕhi pɯ-sɤwum}\hspace{5pt}\pcmn{闭嘴!}\end{exemple}
\begin{exemple}\pjya{fsapaʁ ra tɤ-sɤwɯwum-a}\hspace{5pt}\pcmn{我把牲畜聚集在一起了}\end{exemple}
\begin{exemple}\pjya{tɤ-pɤtso ɯ-kɯmtɕhɯ ra pjɤ-ʁndɤr tɕe, pɯ-sɤwɯwum-a}\hspace{5pt}\pcmn{因为孩子的玩具散了,我把这些收集在一起}\end{exemple}
\begin{sous-entrée}{sɤwɯwum}{ⓔsɤwumⓝsɤwɯwum}\end{sous-entrée}

\end{entrée}

\begin{entrée}{sɤwɯwum}{}{ⓔsɤwɯwum}\relationsémantique{参考}{\lien{ⓔsɤwum}{sɤwum}}\end{entrée}

\begin{entrée}{sɤxoŋxoŋ}{}{ⓔsɤxoŋxoŋ}\relationsémantique{参考}{\lien{ⓔxoŋnɤxoŋ}{xoŋnɤxoŋ}}\end{entrée}

\begin{entrée}{sɤxphɤn}{}{ⓔsɤxphɤn} 
\classe{n} 
\begin{définition}\pfra{avantage}\end{définition}
\begin{définition}\pcmn{好处}\end{définition}
\begin{exemple}\pjya{tɕhi pjɯ́-wɣ-nɤβzjɯβzjoz ʑo ɯ-sɤxphɤn tu ɕti wo}\hspace{5pt}\pcmn{学什么都有用}\end{exemple}\end{entrée}

\begin{entrée}{sɤxtɕhɯxtɕhɯβ}{}{ⓔsɤxtɕhɯxtɕhɯβ} 
\classe{vt} \paradigme{dir}{tɤ-}
\begin{définition}\pfra{émettre un bruit de froissement (sac en plastique)}\end{définition}
\begin{définition}\pcmn{发出沙沙声(塑料袋)}\end{définition}
\begin{exemple}\pjya{a-rna ɯ-ŋgɯ βɣɤza ko-ɕe tɕe, ɲɯ-sɤxtɕhɯxtɕhɯβ ʑo}\hspace{5pt}\pcmn{苍蝇钻到我耳朵里,发出沙沙声}\end{exemple}
\begin{sous-entrée}{ɣɤxtɕhɯxtɕhɯβ}{ⓔsɤxtɕhɯxtɕhɯβⓝɣɤxtɕhɯxtɕhɯβ} 
\classe{vi} 
\begin{définition}\pfra{y avoir un bruit de froissement}\end{définition}
\begin{définition}\pcmn{沙沙地响}\end{définition}
\begin{exemple}\pjya{a-rna ɯ-ŋgɯ thɯci ko-ɕe, ɲɯ-ɣɤxtɕhɯxtɕhɯβ}\hspace{5pt}\pcmn{有东西钻到我耳朵里,发出沙沙声}\end{exemple}\end{sous-entrée}

\end{entrée}

\begin{entrée}{sɤxtɕɯɣ}{}{ⓔsɤxtɕɯɣ} 
\classe{n} 
\begin{définition}\pfra{lanière pour porter les enfants sur le dos}\end{définition}
\begin{définition}\pcmn{背小孩子的背带}\end{définition}\end{entrée}

\begin{entrée}{sɤxtɕɯxtɕi}{}{ⓔsɤxtɕɯxtɕi} 
\classe{adv} 
\begin{définition}\pfra{depuis tout petit}\end{définition}
\begin{définition}\pcmn{从小}\end{définition}
\begin{exemple}\pjya{sɤxtɕɯxtɕi kɯrɯ skɤt spe}\hspace{5pt}\pcmn{他从小都会讲藏语}\end{exemple}\end{entrée}

\begin{entrée}{sɤxɯβxɯβ}{}{ⓔsɤxɯβxɯβ}\relationsémantique{参考}{\lien{ⓔxɯβxɯβ}{xɯβxɯβ}}\end{entrée}

\begin{entrée}{sɤxɯxɯɣ}{}{ⓔsɤxɯxɯɣ} 
\classe{vt} \sens{1}\paradigme{dir}{nɯ-}
\begin{définition}\pfra{agiter}\end{définition}
\begin{définition}\pcmn{挥动}\end{définition}
\begin{exemple}\pjya{laʁjɯɣ nɯ a-ku ɯ-taʁ kɤ-sɤxɯxɯɣ-a}\hspace{5pt}\pcmn{我把棍子在我头上挥动了}\end{exemple}\sens{2}\paradigme{dir}{pɯ-}
\begin{définition}\pfra{souffler bruyamment}\end{définition}
\begin{définition}\pcmn{风吹,发出很紧的声音}\end{définition}
\begin{exemple}\pjya{qale ɲɯ-sɤxɯxɯɣ}\hspace{5pt}\pcmn{风在吹,发出声音}\end{exemple}\relationsémantique{参考}{\lien{ⓔɣɤxɯxɯɣ}{ɣɤxɯxɯɣ}}\end{entrée}

\begin{entrée}{sɤχa}{}{ⓔsɤχa}\relationsémantique{参考}{\lien{ⓔaχa}{aχa}}\end{entrée}

\begin{entrée}{sɤχsɯχsjɯβ}{}{ⓔsɤχsɯχsjɯβ} 
\classe{vt} \paradigme{dir}{tɤ-}
\begin{définition}\pfra{renifler}\end{définition}
\begin{définition}\pcmn{用鼻吸气,发出嘶嘶声}\end{définition}
\begin{exemple}\pjya{ɯ-ɕna to-sɤχsɯχsjɯβ (=χsjɯβnɤχsjɯβ to-stu)}\hspace{5pt}\pcmn{他用鼻吸了气}\end{exemple}\relationsémantique{参考}{\lien{ⓔχsjɯβnɤχsjɯβ}{χsjɯβnɤχsjɯβ}}\end{entrée}

\begin{entrée}{sɤz}{}{ⓔsɤz} 
\classe{postp} 
\begin{définition}\pfra{par rapport à}\end{définition}
\begin{définition}\pcmn{比}\end{définition}\relationsémantique{同义词}{\lien{ⓔstaʁ}{staʁ}}\relationsémantique{同义词}{\lien{ⓔsɤznɤ}{sɤznɤ}}\end{entrée}

\begin{entrée}{sɤzda}{}{ⓔsɤzda} 
\classe{vs} 
\begin{définition}\pfra{aimable}\end{définition}
\begin{définition}\pcmn{很好相处}\end{définition}\relationsémantique{同义词}{\lien{ⓔsaχti}{saχti}}\relationsémantique{参考}{\lien{ⓔnɤzda}{nɤzda}}\relationsémantique{参考}{\lien{ⓔɣɤzda}{ɣɤzda}}\end{entrée}

\begin{entrée}{sɤzdaʁ}{}{ⓔsɤzdaʁ}\relationsémantique{参考}{\lien{ⓔazdaʁ}{azdaʁ}}\end{entrée}

\begin{entrée}{sɤzdɯm}{}{ⓔsɤzdɯm} 
\classe{n} 
\begin{définition}\pfra{nuage de pluie}\end{définition}
\begin{définition}\pcmn{乌云(下雨之前的)}\end{définition}\relationsémantique{参考}{\lien{ⓔzdɯm}{zdɯm}}\end{entrée}

\begin{entrée}{sɤzdɯxpa/\variante{sɤdɯxpa}}{}{ⓔsɤzdɯxpa} 
\classe{vs} \paradigme{dir}{tɤ-}\paradigme{dir}{thɯ-}
\begin{définition}\pfra{pitoyable, pauvre}\end{définition}
\begin{définition}\pcmn{可怜}\end{définition}
\begin{exemple}\pjya{pɯ-sɤzdɯxpa}\hspace{5pt}\pcmn{他很可怜}\end{exemple}\relationsémantique{参考}{\lien{ⓔnɯzdɯxpa}{nɯzdɯxpa}}\end{entrée}

\begin{entrée}{sɤzɣɤkhe}{}{ⓔsɤzɣɤkhe}\relationsémantique{参考}{\lien{ⓔkhe}{khe}}\end{entrée}

\begin{entrée}{sɤzɣɤmɯ}{}{ⓔsɤzɣɤmɯ}\relationsémantique{参考}{\lien{ⓔɣɤmɯ}{ɣɤmɯ}}\end{entrée}

\begin{entrée}{sɤzɣɤxpra}{}{ⓔsɤzɣɤxpra}\relationsémantique{参考}{\lien{ⓔɣɤxpra}{ɣɤxpra}}\end{entrée}

\begin{entrée}{sɤzɣɯt}{}{ⓔsɤzɣɯt} 
\classe{vt}  
\grammaire{refl} \paradigme{dir}{\_}\paradigme{dir}{\_}
\begin{définition}\pfra{ramener}\end{définition}
\begin{définition}\pcmn{带到(目的地)}\end{définition}
\begin{exemple}\pjya{ta-sɤzɣɯt-nɯ}\hspace{5pt}\pcmn{他们带到了}\end{exemple}
\begin{exemple}\pjya{nɤ-mu ɣɯ ɯ-tɕɣom ɯ-nɯ́-tɯ-sɤzɣɯt}\hspace{5pt}\pcmn{你把花椒带给你母亲了吗?}\end{exemple}
\begin{sous-entrée}{ʑɣɤsɤzɣɯt}{ⓔsɤzɣɯtⓝʑɣɤsɤzɣɯt} 
\classe{vi} \end{sous-entrée}

\begin{définition}\pfra{se rendre à, auprès de}\end{définition}
\begin{définition}\pcmn{去到}\end{définition}\relationsémantique{参考}{\lien{ⓔzɣɯt}{zɣɯt}}\relationsémantique{参考}{\lien{ⓔɣɯt}{ɣɯt}}\end{entrée}

\begin{entrée}{sɤzjaŋlaŋ}{}{ⓔsɤzjaŋlaŋ}\relationsémantique{参考}{\lien{ⓔɣɤzjaŋlaŋ}{ɣɤzjaŋlaŋ}}\end{entrée}

\begin{entrée}{sɤzjaŋzjaŋ}{}{ⓔsɤzjaŋzjaŋ}\relationsémantique{参考}{\lien{ⓔɣɤzjaŋlaŋ}{ɣɤzjaŋlaŋ}}\end{entrée}

\begin{entrée}{sɤzjɤɣlɤɣ}{}{ⓔsɤzjɤɣlɤɣ}\relationsémantique{参考}{\lien{ⓔzjɤɣzjɤɣ}{zjɤɣzjɤɣ}}\end{entrée}

\begin{entrée}{sɤzmbrɯ}{}{ⓔsɤzmbrɯ} 
\classe{vt}  
\grammaire{caus} \paradigme{dir}{tɤ-}
\begin{définition}\pfra{énerver}\end{définition}
\begin{définition}\pcmn{惹人生气}\end{définition}
\begin{exemple}\pjya{tɤ-sɤzmbrɯ-t-a}\hspace{5pt}\pcmn{我惹他生气了}\end{exemple}
\begin{exemple}\pjya{tɤ-ta-sɤzmbrɯ}\hspace{5pt}\pcmn{我惹你生气了}\end{exemple}
\begin{exemple}\pjya{tɤ́-wɣ-sɤzmbrɯ-a}\hspace{5pt}\pcmn{他惹我生气了}\end{exemple}
\begin{exemple}\pjya{ma-tɤ-tɯ-sɤzmbri}\hspace{5pt}\pcmn{你不要惹他生气}\end{exemple}
\begin{exemple}\pjya{nɤʑo qhlɯ to-tɯ-sɤzmbrɯ-t}\hspace{5pt}\pcmn{你惹了水神}\end{exemple}\relationsémantique{参考}{\lien{ⓔsɤmbrɯ}{sɤmbrɯ}}\end{entrée}

\begin{entrée}{sɤznɤ}{}{ⓔsɤznɤ} 
\classe{adv} 
\begin{définition}\pfra{par rapport à}\end{définition}
\begin{définition}\pcmn{比}\end{définition}
\begin{exemple}\pjya{nɯ ɕɯŋgɯ sɤznɤ tɕi-χpi nɯra ɲɯ-tɯ-tso ɣe?}\hspace{5pt}\pcmn{你懂我们的故事懂得比以前多,对吧?}\end{exemple}
\begin{exemple}\pjya{nɯ sɤznɤ}\hspace{5pt}\pcmn{相反的、反而}\end{exemple}\relationsémantique{参考}{\lien{ⓔsɤz}{sɤz}}\end{entrée}

\begin{entrée}{sɤzoŋzoŋ}{}{ⓔsɤzoŋzoŋ}\relationsémantique{参考}{\lien{ⓔɣɤzoŋzoŋ}{ɣɤzoŋzoŋ}}\end{entrée}

\begin{entrée}{sɤzraʁ}{}{ⓔsɤzraʁ}\relationsémantique{参考}{\lien{ⓔnɤzraʁ}{nɤzraʁ}}\end{entrée}

\begin{entrée}{sɤʑa}{}{ⓔsɤʑa} 
\classe{vt} \paradigme{dir}{\_}
\begin{définition}\pfra{commencer}\end{définition}
\begin{définition}\pcmn{开始}\end{définition}
\begin{exemple}\pjya{tɕhomba ta-sɤʑa}\hspace{5pt}\pcmn{他开始感冒}\end{exemple}
\begin{exemple}\pjya{kɤ-rɤma ta-sɤʑa}\hspace{5pt}\pcmn{他开始工作了}\end{exemple}
\begin{exemple}\pjya{kɤ-nɤma mɯ-tɤ-tɯ-sɤʑa-t}\hspace{5pt}\pcmn{你没有开始工作}\end{exemple}
\begin{exemple}\pjya{ʑa tɤ-tɯ-sɤʑa-t}\hspace{5pt}\pcmn{你早就开始了}\end{exemple}
\begin{exemple}\pjya{kɤ-fɕɤt pɯ-sɤʑe}\hspace{5pt}\pcmn{请开始讲}\end{exemple}
\begin{exemple}\pjya{rɤɣo thɯ-sɤʑe}\hspace{5pt}\pcmn{请开始唱歌}\end{exemple}
\begin{exemple}\pjya{kutɕu tɕe ʁdɯrɟɤt sɤtɕha lu-sɤʑe ʑo ŋu}\hspace{5pt}\pcmn{从这里开始就是龙尔甲乡}\end{exemple}\relationsémantique{参考}{\lien{ⓔʑaⓗ1}{ʑa₁}}\relationsémantique{参考}{\lien{ⓔnɤmphruʑa}{nɤmphruʑa}}\end{entrée}

\begin{entrée}{sɤʑaŋ}{}{ⓔsɤʑaŋ} 
\classe{n} 
\begin{définition}\pfra{champs}\end{définition}
\begin{définition}\pcmn{田地}\end{définition}\étymologie{sa.ʑiŋ}\end{entrée}

\begin{entrée}{sɤʑdraŋlaŋ}{}{ⓔsɤʑdraŋlaŋ}\relationsémantique{参考}{\lien{ⓔʑdraŋʑdraŋ}{ʑdraŋʑdraŋ}}\end{entrée}

\begin{entrée}{sɤʑɣɤlɤt}{}{ⓔsɤʑɣɤlɤt}\relationsémantique{参考}{\lien{ⓔsɤʑɣɤʑɣɤt}{sɤʑɣɤʑɣɤt}}\end{entrée}

\begin{entrée}{sɤʑɣɤʑɣɤt/\variante{znɯʑɣɤʑɣɤt}}{}{ⓔsɤʑɣɤʑɣɤt} 
\classe{vt} 
\begin{définition}\pfra{brandir}\end{définition}
\begin{définition}\pcmn{举起,挥舞}\end{définition}
\begin{exemple}\pjya{mbrɯtɕɯ ɲɯ-sɤʑɣɤʑɣɤt ʑo tha-tsɯm}\hspace{5pt}\pcmn{他挥着刀跑下去了}\end{exemple}
\begin{sous-entrée}{sɤʑɣɤlɤt}{ⓔsɤʑɣɤʑɣɤtⓝsɤʑɣɤlɤt} 
\classe{vt} 
\begin{définition}\pfra{brandir et agiter dans tous les sens}\end{définition}
\begin{définition}\pcmn{乱挥乱舞}\end{définition}\end{sous-entrée}

\end{entrée}

\begin{entrée}{sɤʑɯloʁ}{}{ⓔsɤʑɯloʁ} 
\classe{vs}  
\grammaire{incorp} \paradigme{dir}{nɯ-}
\begin{définition}\pfra{dégouter}\end{définition}
\begin{définition}\pcmn{使人感到恶心}\end{définition}
\begin{exemple}\pjya{jiɕqha nɯ mɯ́j-saχɕɯn, ɲɯ-sɤʑɯloʁ}\hspace{5pt}\pcmn{不卫生,令人感到恶心}\end{exemple}
\begin{sous-entrée}{nɤsɤʑɯloʁ}{ⓔsɤʑɯloʁⓝnɤsɤʑɯloʁ} 
\classe{vt} 
\begin{définition}\pfra{trouver dégoutant}\end{définition}
\begin{définition}\pcmn{觉得恶心}\end{définition}\relationsémantique{参考}{\lien{ⓔnɤʑɯloʁ}{nɤʑɯloʁ}}\relationsémantique{参考}{\lien{ⓔtɯ-ʑi,loʁ}{tɯ-ʑi,loʁ}}\end{sous-entrée}

\end{entrée}

\begin{entrée}{sɤʑɯrja}{}{ⓔsɤʑɯrja}\relationsémantique{参考}{\lien{ⓔaʑɯrja}{aʑɯrja}}\end{entrée}

\begin{entrée}{sɤʑɯχtso}{}{ⓔsɤʑɯχtso}\relationsémantique{参考}{\lien{ⓔaʑɯχtso}{aʑɯχtso}}\end{entrée}

\begin{entrée}{scapa}{}{ⓔscapa} 
\classe{n} 
\begin{définition}\pfra{dague}\end{définition}
\begin{définition}\pcmn{刺刀}\end{définition}\end{entrée}

\begin{entrée}{scapafkɯm}{}{ⓔscapafkɯm} 
\classe{n} 
\begin{définition}\pfra{fourreau}\end{définition}
\begin{définition}\pcmn{刀鞘}\end{définition}\end{entrée}

\begin{entrée}{scaʁa}{}{ⓔscaʁa} 
\classe{n} 
\begin{définition}\pfra{pie, animal domestique dont le corps est noir et le milieu du corps blanc}\end{définition}
\begin{définition}\pcmn{喜鹊,全身黑色,腰白色的牲畜}\end{définition}\étymologie{skʲa.ka}\end{entrée}

\begin{entrée}{scɤmdʑɯɣ}{}{ⓔscɤmdʑɯɣ} 
\classe{n} 
\begin{définition}\pfra{protection}\end{définition}
\begin{définition}\pcmn{保佑}\end{définition}\étymologie{skʲabs.ⁿdʑug}\end{entrée}

\begin{entrée}{scɤt}{}{ⓔscɤt} 
\classe{vt} \paradigme{dir}{\_}\paradigme{dir}{\_}\paradigme{dir}{\_}
\begin{définition}\pfra{déplacer}\end{définition}
\begin{définition}\pcmn{摆动,搬}\end{définition}
\begin{définition}\pfra{déménager, déplacer}\end{définition}
\begin{définition}\pcmn{搬走}\end{définition}
\begin{définition}\pfra{déplacer partout}\end{définition}
\begin{définition}\pcmn{搬来搬去}\end{définition}
\begin{exemple}\pjya{pa-scɤt}\hspace{5pt}\pcmn{他搬了}\end{exemple}
\begin{exemple}\pjya{rɯ ɲɯ-scat-a}\hspace{5pt}\pcmn{我搬帐篷}\end{exemple}
\begin{exemple}\pjya{ɯ-sɤz-rɤʑi ɲo-scɤt}\hspace{5pt}\pcmn{他搬家了}\end{exemple}
\begin{exemple}\pjya{ɯ-khɯrthaŋ pjɤ-scɤt (=pjɤ-phɤβ)}\hspace{5pt}\pcmn{他被降职了}\end{exemple}
\begin{exemple}\pjya{ɯ-khɯrthaŋ to-scɤt}\hspace{5pt}\pcmn{他升职了}\end{exemple}
\begin{exemple}\pjya{pa-nɤscɯscɤt ntsɯ}\hspace{5pt}\pcmn{他不停地搬来搬去了}\end{exemple}
\begin{sous-entrée}{rɤscɤt}{ⓔscɤtⓝrɤscɤt} 
\classe{vi}  
\grammaire{apass} \end{sous-entrée}

\begin{sous-entrée}{nɤscɯscɤt}{ⓔscɤtⓝnɤscɯscɤt} 
\classe{vt}  
\grammaire{n.orient} \end{sous-entrée}

\end{entrée}

\begin{entrée}{schɤt}{}{ⓔschɤt} 
\classe{vi} \paradigme{dir}{pɯ-}
\begin{définition}\pfra{se retirer (eau)}\end{définition}
\begin{définition}\pcmn{下降(水)}\end{définition}
\begin{exemple}\pjya{tɯ-ci pjɤ-schɤt}\hspace{5pt}\pcmn{水下降了}\end{exemple}
\begin{sous-entrée}{sɯschɤt}{ⓔschɤtⓝsɯschɤt} 
\classe{vt}  
\grammaire{caus} \paradigme{dir}{pɯ-}
\begin{définition}\pfra{faire en sorte que l'eau se retire}\end{définition}
\begin{définition}\pcmn{使(水)下降}\end{définition}\end{sous-entrée}

\end{entrée}

\begin{entrée}{schi}{}{ⓔschi} 
\classe{vs} \paradigme{dir}{tɤ-}
\begin{définition}\pfra{supporter}\end{définition}
\begin{définition}\pcmn{经得住}\end{définition}
\begin{exemple}\pjya{aʑo tɤndʐo schi-a ma qaʑo ɕa tɤ-ndza-t-a}\hspace{5pt}\pcmn{我经得住冷,因为吃了羊肉}\end{exemple}
\begin{exemple}\pjya{ɯʑo tɤ-ŋɤm ɲɯ-schi}\hspace{5pt}\pcmn{他经得住痛}\end{exemple}
\begin{exemple}\pjya{tɤɕpaʁ schi-a}\hspace{5pt}\pcmn{我经得住渴}\end{exemple}
\begin{exemple}\pjya{tshɤdɯɣ mɤ-schi-a}\hspace{5pt}\pcmn{我经不起热的天气}\end{exemple}
\begin{exemple}\pjya{tɤndʐo mɯ́j-tɯ-schi}\hspace{5pt}\pcmn{你很怕冷}\end{exemple}\end{entrée}

\begin{entrée}{schɯɣschɯɣ}{}{ⓔschɯɣschɯɣ} 
\classe{idph.2} 
\begin{définition}\pfra{très froid}\end{définition}
\begin{définition}\pcmn{形容极冷}\end{définition}
\begin{exemple}\pjya{ɯ-jaʁ ɲɯ-mɯɕtaʁ schɯɣschɯɣ ʑo}\hspace{5pt}\pcmn{他的手冷冰冰的}\end{exemple}\end{entrée}

\begin{entrée}{sci}{₂}{ⓔsciⓗ2} 
\classe{n}  
\grammaire{n.lieu} 
\begin{définition}\pfra{l'un des hameaux de Gyutshapa}\end{définition}
\begin{définition}\pcmn{二茶村的大队之一}\end{définition}\end{entrée}

\begin{entrée}{sci}{₁}{ⓔsciⓗ1} 
\classe{vi} \paradigme{dir}{tɤ-}\paradigme{dir}{thɯ-}\sens{1}
\begin{définition}\pfra{naître}\end{définition}
\begin{définition}\pcmn{出生}\end{définition}
\begin{exemple}\pjya{ɯ-rɟit to-sci}\hspace{5pt}\pcmn{她的孩子出生了}\end{exemple}\sens{2}
\begin{définition}\pfra{grandir}\end{définition}
\begin{définition}\pcmn{生长}\end{définition}
\begin{exemple}\pjya{tɤ-pɤtso ra kɯ-ŋu kɤ-sɯxɕɤt ɯ-khɯkha a-thɯ-sci-nɯ (a-thɯ-ndzɤt-nɯ) tɕe pe}\hspace{5pt}\pcmn{在孩子生长的过程中要进行正面教育就会好}\end{exemple}\relationsémantique{同义词}{\lien{ⓔndzɤt}{ndzɤt}}
\begin{définition}\pfra{donner naissance}\end{définition}
\begin{définition}\pcmn{生(小孩子)}\end{définition}\relationsémantique{参考}{\lien{ⓔtɯ-sɤsci}{tɯ-sɤsci}}
\begin{sous-entrée}{sɯsci}{ⓔsciⓗ1ⓝsɯsci} 
\classe{vt}  
\grammaire{caus} \paradigme{dir}{tɤ-}\end{sous-entrée}

\étymologie{skʲe}\end{entrée}

\begin{entrée}{scinde}{}{ⓔscinde} 
\classe{n}  
\grammaire{n.lieu} 
\begin{définition}\pfra{l'un des hameaux de Gyutshapa}\end{définition}
\begin{définition}\pcmn{二茶村的大队之一}\end{définition}\end{entrée}

\begin{entrée}{scit}{}{ⓔscit} 
\classe{vs} \paradigme{dir}{tɤ-}\paradigme{dir}{thɯ-}
\begin{définition}\pfra{heureux}\end{définition}
\begin{définition}\pcmn{幸福}\end{définition}\relationsémantique{参考}{\lien{ⓔsɤscit}{sɤscit}}
\begin{sous-entrée}{sɯscit}{ⓔscitⓝsɯscit} 
\classe{vt}  
\grammaire{caus} 
\begin{définition}\pfra{rendre ... heureux}\end{définition}
\begin{définition}\pcmn{让……享福}\end{définition}\end{sous-entrée}

\étymologie{skʲid}\end{entrée}

\begin{entrée}{sciwa}{}{ⓔsciwa} 
\classe{n} 
\begin{définition}\pfra{vie, existence}\end{définition}
\begin{définition}\pcmn{生命}\end{définition}
\begin{exemple}\pjya{sciwa ɲo-nɤsci}\hspace{5pt}\pcmn{他换了一条生命,死了以后活过来了}\end{exemple}\étymologie{skʲe.ba}\end{entrée}

\begin{entrée}{sclaŋsclaŋ}{}{ⓔsclaŋsclaŋ} 
\classe{idph.2} 
\begin{définition}\pfra{chauve}\end{définition}
\begin{définition}\pcmn{形容光溜溜的样子}\end{définition}
\begin{exemple}\pjya{ɯ-ku sclaŋsclaŋ ɲɯ-pa}\hspace{5pt}\pcmn{他的头是光溜溜的}\end{exemple}
\begin{exemple}\pjya{ɯ-ku sclaŋsclaŋ chɤ-nɯ-sɯ-qrɤz}\hspace{5pt}\pcmn{他请人把头剃得光溜溜}\end{exemple}\end{entrée}

\begin{entrée}{scun}{}{ⓔscun} 
\classe{n} 
\begin{définition}\pfra{faute}\end{définition}
\begin{définition}\pcmn{过错}\end{définition}\étymologie{skʲon}\end{entrée}

\begin{entrée}{sco}{}{ⓔsco} 
\classe{vt} \paradigme{dir}{\_}
\begin{définition}\pfra{raccompagner}\end{définition}
\begin{définition}\pcmn{送行}\end{définition}
\begin{exemple}\pjya{ɕ-kɤ-sco-t-a}\hspace{5pt}\pcmn{我去送了他}\end{exemple}
\begin{exemple}\pjya{nɯ́-wɣ-sco-a}\hspace{5pt}\pcmn{他送了我}\end{exemple}\relationsémantique{参考}{\lien{ⓔnɤscɤlɤt}{nɤscɤlɤt}}
\begin{sous-entrée}{sɤsco}{ⓔscoⓝsɤsco} 
\classe{vi}  
\grammaire{apass} \end{sous-entrée}

\begin{sous-entrée}{ascɯsco}{ⓔscoⓝascɯsco} 
\classe{vi}  
\grammaire{refl} 
\begin{définition}\pfra{se raccompagner les uns les autres}\end{définition}
\begin{définition}\pcmn{互相送行}\end{définition}\end{sous-entrée}

\end{entrée}

\begin{entrée}{scoʁ}{}{ⓔscoʁ} 
\classe{n} 
\begin{définition}\pfra{louche}\end{définition}
\begin{définition}\pcmn{勺子;瓢}\end{définition}\étymologie{skʲogs}\end{entrée}

\begin{entrée}{scoʁɲaʁ}{}{ⓔscoʁɲaʁ} 
\classe{n} 
\begin{définition}\pfra{louche en fer}\end{définition}
\begin{définition}\pcmn{铁铸成的勺子}\end{définition}\end{entrée}

\begin{entrée}{scoʁrlɯ}{}{ⓔscoʁrlɯ} 
\classe{n} 
\begin{définition}\pfra{louche en cuivre}\end{définition}
\begin{définition}\pcmn{红铜铸成的勺子}\end{définition}\end{entrée}

\begin{entrée}{scoʁtshaʁ}{}{ⓔscoʁtshaʁ} 
\classe{n} 
\begin{définition}\pfra{filtre à thé}\end{définition}
\begin{définition}\pcmn{茶漏勺}\end{définition}
\begin{exemple}\pjya{scoʁtshaʁ nɯ tʂha tɤ́-wɣ-rku tɕe ɯ-sɤ-sɯxtshaʁ ŋu, tʂha ɯ-ŋgɯ tʂhɤzwa tu tɕe ɯ-tshaʁ pɕoʁ nɯ tɕu pjɯ́-wɣ-lɤt tɕe, tʂha ɯ-ci nɯ pjɯ-nɯɬoʁ ŋu ma ɯ-zwa nɯ scoʁ ɯ-ŋgɯ ku-nɯ-rɤʑi ŋu tɕe núndʐa scoʁtshaʁ rmi. ɯ-jɯ nɯ lú-wɣ-ltɤβ kɯ-khɯ ŋu, tɕe ndʐɯnbu kɤ-nɤndɯndo pe.}\hspace{5pt}\pcmn{茶漏勺是倒茶的时候用来滤茶的器具。因为茶里有茶叶等渣滓,朝着漏勺的方向倒过去,这样茶水会流出来,渣滓就会留在勺子里,所以这种勺子叫漏勺(滤器)。勺子把可以折起来,所以出远门时带着方便。}\end{exemple}\end{entrée}

\begin{entrée}{scraʁscraʁ}{}{ⓔscraʁscraʁ} 
\classe{idph.2} \paradigme{dir}{nɯ-}
\begin{définition}\pfra{tout petit}\end{définition}
\begin{définition}\pcmn{矮小,挨着地面}\end{définition}
\begin{définition}\pfra{parler sans arrêt sans se préoccuper des conventions sociales}\end{définition}
\begin{définition}\pcmn{不知安分守己,多嘴}\end{définition}
\begin{exemple}\pjya{jiɕqha nɯ ɲɯ-ɣɤscraʁscraʁ ntsɯ}\hspace{5pt}\pcmn{那个人总是多嘴}\end{exemple}
\begin{sous-entrée}{scraʁnɤscraʁ}{ⓔscraʁscraʁⓝscraʁnɤscraʁ} 
\classe{idph.3} 
\begin{exemple}\pjya{qaɕpa nɯ scraʁnɤscraʁ lu-ɕe pjɤ-ŋu}\hspace{5pt}\pcmn{青蛙一跳一跳地往上游去}\end{exemple}\end{sous-entrée}

\begin{sous-entrée}{ɣɤscraʁscraʁ}{ⓔscraʁscraʁⓝɣɤscraʁscraʁ} 
\classe{vi} \end{sous-entrée}

\begin{sous-entrée}{ɣɤscraʁlaʁ}{ⓔscraʁscraʁⓝɣɤscraʁlaʁ} 
\classe{vi} 
\begin{exemple}\pjya{ɯ-ku sɤrku kɯ-me ɲɯ-ɣɤscraʁlaʁ}\hspace{5pt}\pcmn{没有他的事,还是在多嘴}\end{exemple}\end{sous-entrée}

\end{entrée}

\begin{entrée}{scrɯscri}{}{ⓔscrɯscri} 
\classe{idph.2} 
\begin{définition}\pfra{liquide, dilué (boue)}\end{définition}
\begin{définition}\pcmn{形容(泥巴)稀,流体状}\end{définition}\relationsémantique{参考}{\lien{ⓔɲcrɯɣɲcrɯɣ}{ɲcrɯɣɲcrɯɣ}}\relationsémantique{参考}{\lien{ⓔχcrɯχcri}{χcrɯχcri}}\end{entrée}

\begin{entrée}{scɯʁzɯɣ}{}{ⓔscɯʁzɯɣ} 
\classe{n} 
\begin{définition}\pfra{apparence}\end{définition}
\begin{définition}\pcmn{外貌}\end{définition}\relationsémantique{参考}{\lien{ⓔscɯʁzɯɣ}{scɯʁzɯɣ}}\étymologie{skʲe.gzugs}\end{entrée}

\begin{entrée}{scuz}{}{ⓔscuz} 
\classe{n} 
\begin{définition}\pfra{faisan (Ithaginis cruentus)}\end{définition}
\begin{définition}\pcmn{血雉【松鸡子】}\end{définition}
\begin{exemple}\pjya{scuz nɯ pɣa ci ŋu, kɯmpɣa jamar wxti. ɯ-kɤχcɤl kɯ-xtɕɯ-xtɕi ɲaʁ, ɯ-phoŋbu kɯ-ɤrŋi ŋgɯz kɯ-pɣi tsa ŋu. ɯ-mi kɯ-qarŋe ŋu, ɯ-mtsioʁ kɯ-qarŋe ŋu, ɯ-jme ɯ-ku kɯ-rɟum tsa ŋu, ɯ-mɲaʁ ɯ-rkɯ qarma ɣɯ kɯ-fse kɯ-ɣɯrni a-fskɤr, sɯŋgɯ ʁɟa ʑo ku-rɤʑi ma tɯ-ji ɯ-ŋgɯ ɣi mɤ-ŋgrɤl, sɯmat cho qajɯ ʁɟa tu-ndze ŋu. tɤ-mbri tɕe, ``tɕur tɕur tɕur tɕur tɕur" tu-ti ŋu tɕe núndʐa ɯ-ɕa tɕur tu-kɯ-ti ɲɯ-ŋu.}\hspace{5pt}\pcmn{松鸡子是一种鸟,有鸡那么大。头顶上有黑点,身子蓝里带灰,脚是黄色的,嘴是黄色的,尾巴的顶端有点宽,眼睛周围有一圈红色的皮像马鸡的一样。它一直生活在森林里,从不下地,专门吃野果和虫子。叫起来时,它发出\lien{}{tɕur tɕur tɕur tɕur tɕur}的声音,所以人家说它的肉很酸。}\end{exemple}\end{entrée}

\begin{entrée}{sɣa}{}{ⓔsɣa} 
\classe{n} 
\begin{définition}\pfra{rouille}\end{définition}
\begin{définition}\pcmn{锈}\end{définition}\relationsémantique{参考}{\lien{ⓔnɯsɣa}{nɯsɣa}}\end{entrée}

\begin{entrée}{si}{₁}{ⓔsiⓗ1} 
\classe{n} 
\begin{définition}\pfra{arbre}\end{définition}
\begin{définition}\pcmn{树}\end{définition}\end{entrée}

\begin{entrée}{si}{₂}{ⓔsiⓗ2} 
\classe{vi} \paradigme{dir}{pɯ-}\paradigme{dir}{nɯ-}
\begin{définition}\pfra{mourir}\end{définition}
\begin{définition}\pcmn{死(人)}\end{définition}\end{entrée}

\begin{entrée}{sijmɤɣ}{}{ⓔsijmɤɣ} 
\classe{n} 
\begin{définition}\pfra{russule}\end{définition}
\begin{définition}\pcmn{【辣辣菌】}\end{définition}
\begin{exemple}\pjya{sijmɤɣ nɯ sɤjku ɯ-ŋgɯ tu-ɬoʁ ŋu, ɯ-tshɯɣa cho ɯ-mdoʁ nɯ jmɤɣni fse ri tú-wɣ-ndza tɕe mɤrtsaβ.}\hspace{5pt}\pcmn{辣辣菌生长在白桦树林里,形状和颜色和杉木菌一样,但吃起来很辣。}\end{exemple}\end{entrée}

\begin{entrée}{sindzɯ}{}{ⓔsindzɯ} 
\classe{n} 
\begin{définition}\pfra{testament}\end{définition}
\begin{définition}\pcmn{遗嘱}\end{définition}
\begin{exemple}\pjya{sindzɯ ci ra chɤ-rɤt.}\hspace{5pt}\pcmn{他写了遗嘱}\end{exemple}\relationsémantique{参考}{\lien{ⓔndzɯ}{ndzɯ}}\relationsémantique{参考}{\lien{ⓔsiⓗ1}{si₁}}\end{entrée}

\begin{entrée}{sjaŋnɤsjaŋ}{}{ⓔsjaŋnɤsjaŋ} 
\classe{idph.3} 
\begin{définition}\pfra{dodelinant de la tête}\end{définition}
\begin{définition}\pcmn{形容昂着头,头晃来晃去的样子}\end{définition}\end{entrée}

\begin{entrée}{sjoŋsjoŋ}{}{ⓔsjoŋsjoŋ} 
\classe{idph.2} 
\begin{définition}\pfra{gris}\end{définition}
\begin{définition}\pcmn{形容灰中带白的颜色}\end{définition}\end{entrée}

\begin{entrée}{sjɯŋsjɯŋ}{}{ⓔsjɯŋsjɯŋ} 
\classe{idph.2} 
\begin{définition}\pfra{blanc et haut}\end{définition}
\begin{définition}\pcmn{形容又高又白的样子}\end{définition}
\begin{exemple}\pjya{mtɕhortɯn ci sjɯŋsjɯŋ ʑo ɣɤʑu}\hspace{5pt}\pcmn{有一座白塔}\end{exemple}\end{entrée}

\begin{entrée}{sjɯrnɤsjɯr}{}{ⓔsjɯrnɤsjɯr} 
\classe{idph.3} 
\begin{définition}\pfra{qui émet de la fumée par intervalles réguliers}\end{définition}
\begin{définition}\pcmn{形容一次又一次地冒烟}\end{définition}
\begin{exemple}\pjya{thamaka sjɯrnɤsjɯr ɲɯ-ɤsɯ-sko}\hspace{5pt}\pcmn{他抽着烟,一次又一次地吐出烟来}\end{exemple}
\begin{sous-entrée}{sjɯrɯri}{ⓔsjɯrnɤsjɯrⓝsjɯrɯri} 
\classe{idph.7} 
\begin{définition}\pfra{qui s'élève lentement (fumée)}\end{définition}
\begin{définition}\pcmn{形容(烟)慢慢升起来的样子}\end{définition}
\begin{exemple}\pjya{ɯ-thamaka ɯ-khɯ sjɯrɯri ʑo tu-ɬoʁ ɲɯ-ŋu}\hspace{5pt}\pcmn{他抽的烟慢慢升起}\end{exemple}\end{sous-entrée}

\end{entrée}

\begin{entrée}{skɤɣ}{}{ⓔskɤɣ} 
\classe{vt} \paradigme{dir}{kɤ-}
\begin{définition}\pfra{engraisser}\end{définition}
\begin{définition}\pcmn{催肥}\end{définition}
\begin{exemple}\pjya{paʁ kɤ-skaɣ-a}\hspace{5pt}\pcmn{我把猪催肥了}\end{exemple}\end{entrée}

\begin{entrée}{skɤlɤn}{}{ⓔskɤlɤn} 
\classe{n} 
\begin{définition}\pfra{réponse}\end{définition}
\begin{définition}\pcmn{回句;回音}\end{définition}\étymologie{skad.len}\end{entrée}

\begin{entrée}{skɤlpa}{}{ⓔskɤlpa} 
\classe{n} 
\begin{définition}\pfra{monde}\end{définition}
\begin{définition}\pcmn{世界}\end{définition}\étymologie{bskal.pa}\end{entrée}

\begin{entrée}{skɤm}{₁}{ⓔskɤmⓗ1} 
\classe{n} 
\begin{définition}\pfra{bœuf à viande}\end{définition}
\begin{définition}\pcmn{菜牛}\end{définition}
\begin{exemple}\pjya{skɤmndʐi kɤ-qaʁ}\hspace{5pt}\pcmn{剥牛皮}\end{exemple}\étymologie{skom.po}\end{entrée}

\begin{entrée}{skɤm}{₂}{ⓔskɤmⓗ2} 
\classe{vi} \paradigme{dir}{pɯ-}
\begin{définition}\pfra{s'assécher (lac, étang)}\end{définition}
\begin{définition}\pcmn{干涸(湖、池塘)}\end{définition}
\begin{exemple}\pjya{ɕɯβloʁ pjɤ-skɤm}\hspace{5pt}\pcmn{池塘干涸了}\end{exemple}\étymologie{skam}\end{entrée}

\begin{entrée}{skɤmtɕhaŋ}{}{ⓔskɤmtɕhaŋ} 
\classe{n} 
\begin{définition}\pfra{tchang en jarre}\end{définition}
\begin{définition}\pcmn{青稞酒的一种做法}\end{définition}\relationsémantique{同义词}{\lien{ⓔchɤmda}{chɤmda}}\relationsémantique{同义词}{\lien{ⓔchɤci}{chɤci}}\end{entrée}

\begin{entrée}{skɤnɲɟɯr}{}{ⓔskɤnɲɟɯr} 
\classe{n} 
\begin{définition}\pfra{mélodie}\end{définition}
\begin{définition}\pcmn{曲调}\end{définition}
\begin{exemple}\pjya{nɯ rɤɣo nɯ ɣɯ ɯ-skɤnɲɟɯr ɲɯ-mpɕɤr}\hspace{5pt}\pcmn{那首歌的曲子很好听}\end{exemple}\étymologie{skad.ⁿgʲur}\end{entrée}

\begin{entrée}{skɤr}{}{ⓔskɤr} 
\classe{vt} \paradigme{dir}{tɤ-}
\begin{définition}\pfra{peser}\end{définition}
\begin{définition}\pcmn{称}\end{définition}\paradigme{dir}{tɤ-}
\begin{définition}\pfra{peser}\end{définition}
\begin{définition}\pcmn{称东西}\end{définition}
\begin{exemple}\pjya{tɤ-skar-a}\hspace{5pt}\pcmn{我称了}\end{exemple}
\begin{exemple}\pjya{ta-skɤr}\hspace{5pt}\pcmn{他称了}\end{exemple}
\begin{exemple}\pjya{laχtɕha tú-wɣ-skɤr}\hspace{5pt}\pcmn{称东西}\end{exemple}
\begin{exemple}\pjya{mbrɤz ta-skɤr}\hspace{5pt}\pcmn{他称了米}\end{exemple}
\begin{exemple}\pjya{tɤ-mthɯm ta-skɤr}\hspace{5pt}\pcmn{他称了肉}\end{exemple}
\begin{exemple}\pjya{tɯjpu tɤ-skar-a}\hspace{5pt}\pcmn{我称了粮食}\end{exemple}
\begin{sous-entrée}{rɤskɤr}{ⓔskɤrⓝrɤskɤr} 
\classe{vi}  
\grammaire{apass} \end{sous-entrée}

\étymologie{skar}\end{entrée}

\begin{entrée}{skɤrlɤm}{}{ⓔskɤrlɤm} 
\classe{n} 
\begin{définition}\pfra{chemin des moulins à prière}\end{définition}
\begin{définition}\pcmn{转经的路}\end{définition}\étymologie{bskor.lam}\end{entrée}

\begin{entrée}{skɤrma}{}{ⓔskɤrma} 
\classe{n} \sens{1}
\begin{définition}\pfra{règle}\end{définition}
\begin{définition}\pcmn{尺}\end{définition}\sens{2}
\begin{définition}\pfra{jour}\end{définition}
\begin{définition}\pcmn{日子}\end{définition}
\begin{exemple}\pjya{skɤrma kɯ-sna}\hspace{5pt}\pcmn{好日子}\end{exemple}\étymologie{skar.ma}\end{entrée}

\begin{entrée}{skɤrtɕɯn}{}{ⓔskɤrtɕɯn} 
\classe{n} 
\begin{définition}\pfra{vénus, étoile du soir}\end{définition}
\begin{définition}\pcmn{金星}\end{définition}\étymologie{skar.tɕʰen}\end{entrée}

\begin{entrée}{skɤrwa}{}{ⓔskɤrwa} 
\classe{n} 
\begin{définition}\pfra{faire tourner les moulins à prière}\end{définition}
\begin{définition}\pcmn{转经}\end{définition}
\begin{exemple}\pjya{skɤrwa ko-ɕe}\hspace{5pt}\pcmn{他去转经了}\end{exemple}
\begin{exemple}\pjya{skɤrwa tɯ-tɤxɯr kɤ-ari-a}\hspace{5pt}\pcmn{我转经转了一周}\end{exemple}\relationsémantique{参考}{\lien{ⓔrɯskɤrwa}{rɯskɤrwa}}\étymologie{skor.ba}\end{entrée}

\begin{entrée}{skɤt}{}{ⓔskɤt} 
\classe{vt}  
\grammaire{apass} \paradigme{dir}{nɯ-}\sens{1}
\begin{définition}\pfra{rendre un objet}\end{définition}
\begin{définition}\pcmn{退(东西)}\end{définition}
\begin{exemple}\pjya{laχtɕha na-skɤt}\hspace{5pt}\pcmn{他把东西退了}\end{exemple}
\begin{exemple}\pjya{jla na-skɤt}\hspace{5pt}\pcmn{他把犏牛退了}\end{exemple}\sens{2}\paradigme{dir}{nɯ-}
\begin{définition}\pfra{refuser}\end{définition}
\begin{définition}\pcmn{拒绝}\end{définition}
\begin{sous-entrée}{sɤskɤt}{ⓔskɤtⓢ2ⓝsɤskɤt} 
\classe{vi} \end{sous-entrée}

\begin{définition}\pfra{refuser}\end{définition}
\begin{définition}\pcmn{拒绝别人}\end{définition}
\begin{exemple}\pjya{nɯ-sɤskat-a}\hspace{5pt}\pcmn{我拒绝了}\end{exemple}
\begin{exemple}\pjya{kɤ-sɤskɤt mɤ-nɤtsa}\hspace{5pt}\pcmn{不好拒绝}\end{exemple}\end{entrée}

\begin{entrée}{sko}{}{ⓔsko} 
\classe{vt} \paradigme{dir}{pɯ-}
\begin{définition}\pfra{fumer}\end{définition}
\begin{définition}\pcmn{抽烟}\end{définition}
\begin{exemple}\pjya{thamakha pa-sko}\hspace{5pt}\pcmn{他抽了烟}\end{exemple}
\begin{exemple}\pjya{ma-pɯ-tɯ-sko-nɯ}\hspace{5pt}\pcmn{你们不要抽烟}\end{exemple}
\begin{exemple}\pjya{thamakha ma-pɯ-tɯ-skɤm ɯ́-jɤɣ?}\hspace{5pt}\pcmn{麻烦你不要抽烟}\end{exemple}
\begin{exemple}\pjya{thamakha kɤ-sko mɯ́j-pe ma ɲɯ́-ɣw-sɤɕqhe-a}\hspace{5pt}\pcmn{抽烟是不好的,会令我咳嗽}\end{exemple}\end{entrée}

\begin{entrée}{skraskra}{}{ⓔskraskra} 
\classe{idph.2} \sens{1}
\begin{définition}\pfra{gris et dépourvu de végétation}\end{définition}
\begin{définition}\pcmn{形容(地方)灰扑扑、光秃秃}\end{définition}
\begin{exemple}\pjya{sɤtɕha skraskra ʑo ɲɯ-pa, mɯ́j-sɤscit}\hspace{5pt}\pcmn{那个地方是灰扑扑、光秃秃的,不舒服}\end{exemple}\sens{2}
\begin{définition}\pfra{mal poli}\end{définition}
\begin{définition}\pcmn{形容调皮的样子}\end{définition}
\begin{exemple}\pjya{tɯrme ɯ-stu mɤ-kɯ-fse ci skraskra ɲɯ-ŋu}\hspace{5pt}\pcmn{那个人特别调皮}\end{exemple}\end{entrée}

\begin{entrée}{skrɯt}{}{ⓔskrɯt} 
\classe{n} 
\begin{définition}\pfra{fil très fin}\end{définition}
\begin{définition}\pcmn{丝}\end{définition}\relationsémantique{参考}{\lien{ⓔɕomskrɯt}{ɕomskrɯt}}\end{entrée}

\begin{entrée}{skɯ}{₂}{ⓔskɯⓗ2} 
\classe{n} 
\begin{définition}\pfra{statue de bouddha}\end{définition}
\begin{définition}\pcmn{佛像}\end{définition}\étymologie{sku}\end{entrée}

\begin{entrée}{skɯ}{₁}{ⓔskɯⓗ1} 
\classe{vt} \paradigme{dir}{pɯ-}
\begin{définition}\pfra{enterrer}\end{définition}
\begin{définition}\pcmn{埋}\end{définition}
\begin{exemple}\pjya{pjɤ-skɯ}\hspace{5pt}\pcmn{把他埋了}\end{exemple}
\begin{exemple}\pjya{fsapaʁ ɯ-ŋgo kɯ-tu pɯ-kɯ-si nɯ pjɯ́-wɣ-skɯ ŋu}\hspace{5pt}\pcmn{病死了的牲畜要埋(不能吃它的肉)}\end{exemple}\end{entrée}

\begin{entrée}{skɯβli}{}{ⓔskɯβli} 
\classe{n} 
\begin{définition}\pfra{cérémonie bouddhique}\end{définition}
\begin{définition}\pcmn{佛教仪式}\end{définition}\end{entrée}

\begin{entrée}{skɯrloʁ}{}{ⓔskɯrloʁ} 
\classe{n} 
\begin{définition}\pfra{type de pas d'aiguille}\end{définition}
\begin{définition}\pcmn{缝针的方法}\end{définition}\end{entrée}

\begin{entrée}{skɯrma}{}{ⓔskɯrma} 
\classe{n} 
\begin{définition}\pfra{cadeau}\end{définition}
\begin{définition}\pcmn{礼物(请别人带的)}\end{définition}
\begin{exemple}\pjya{kɯki a-mu kɯ nɤ-skɯrma jɤ́-wɣ-sɯɣɯt-a ŋu}\hspace{5pt}\pcmn{这是我母亲要我带给你的礼物}\end{exemple}\relationsémantique{同义词}{\lien{ⓔtɤ-pɤro}{tɤ-pɤro}}\relationsémantique{同义词}{\lien{ⓔtɤ-rkuz}{tɤ-rkuz}}\étymologie{skur.ma}\end{entrée}

\begin{entrée}{skɯχɕaʁ}{}{ⓔskɯχɕaʁ} 
\classe{n} 
\begin{définition}\pfra{feu (sprulsku)}\end{définition}
\begin{définition}\pcmn{圆寂了的活佛}\end{définition}\étymologie{sku.gɕegs}\end{entrée}

\begin{entrée}{sla}{}{ⓔsla} 
\classe{n} 
\begin{définition}\pfra{lune}\end{définition}
\begin{définition}\pcmn{月亮}\end{définition}
\begin{exemple}\pjya{sla qajɣi ɯ-rkɯ kɯ-fse ci ma ɲo-me}\hspace{5pt}\pcmn{月亮只有馍馍那么大}\end{exemple}\relationsémantique{参考}{\lien{ⓔslɤzɯn}{slɤzɯn}}\relationsémantique{参考}{\lien{ⓔtɯ-sla}{tɯ-sla}}\relationsémantique{参考}{\lien{ⓔslɤŋe}{slɤŋe}}\end{entrée}

\begin{entrée}{slama}{}{ⓔslama} 
\classe{n} 
\begin{définition}\pfra{étudiant}\end{définition}
\begin{définition}\pcmn{学生}\end{définition}
\begin{exemple}\pjya{nɤ-slama ɯ́-dɤn}\hspace{5pt}\pcmn{你的学生很多吗?}\end{exemple}
\begin{exemple}\pjya{slama tu-dɤn tɕe kɤ-sɤsɯxɕɤt mɤ-mbat ma}\hspace{5pt}\pcmn{学生很多的话,不容易教}\end{exemple}\étymologie{slob.ma}\end{entrée}

\begin{entrée}{slamaχti}{}{ⓔslamaχti} 
\classe{n} 
\begin{définition}\pfra{camarade de classe}\end{définition}
\begin{définition}\pcmn{同学}\end{définition}\relationsémantique{参考}{\lien{ⓔslama}{slama}}\relationsémantique{参考}{\lien{ⓔtɯ-χti}{tɯ-χti}}\end{entrée}

\begin{entrée}{slaŋslaŋ}{}{ⓔslaŋslaŋ} 
\classe{idph.2} 
\begin{définition}\pfra{blanc et rond}\end{définition}
\begin{définition}\pcmn{形容又白又圆(美观)的样子}\end{définition}
\begin{exemple}\pjya{sla slaŋslaŋ to-nɯɬoʁ}\hspace{5pt}\pcmn{月亮出来了,皎洁圆润}\end{exemple}
\begin{exemple}\pjya{ʑara nɯtɕu ʁʑɯnɯ ɲɤ-βzu tɕe slaŋslaŋ ʑo ɲɯ-pa, wuma ʑo ɲɯ-ɣɤχsrɯ}\hspace{5pt}\pcmn{他们的儿子现在长成了小伙子了,又白又胖,非常英俊}\end{exemple}
\begin{sous-entrée}{slaŋnɤslaŋ}{ⓔslaŋslaŋⓝslaŋnɤslaŋ} 
\classe{idph.3} 
\begin{exemple}\pjya{rŋgɯ slaŋnɤslaŋ pɯ-ndʐaβ}\hspace{5pt}\pcmn{圆形的大石包滚下去了}\end{exemple}\relationsémantique{参考}{\lien{ⓔrlaŋrlaŋ}{rlaŋrlaŋ}}\relationsémantique{参考}{\lien{ⓔɕlaŋɕlaŋ}{ɕlaŋɕlaŋ}}\relationsémantique{参考}{\lien{ⓔclaŋclaŋ}{claŋclaŋ}}\end{sous-entrée}

\end{entrée}

\begin{entrée}{slaʁ}{}{ⓔslaʁ} 
\classe{idph.1} 
\begin{définition}\pfra{d'un coup}\end{définition}
\begin{définition}\pcmn{一下子}\end{définition}
\begin{exemple}\pjya{slaʁ ta-ndza}\hspace{5pt}\pcmn{他一下子就吃了}\end{exemple}
\begin{exemple}\pjya{slaʁ thɯ-ɕqhlɤt}\hspace{5pt}\pcmn{他一下子就消失了}\end{exemple}
\begin{sous-entrée}{slaʁnɤslaʁ}{ⓔslaʁⓝslaʁnɤslaʁ} 
\classe{idph.3} 
\begin{définition}\pfra{en plusieurs coups}\end{définition}
\begin{définition}\pcmn{几下}\end{définition}
\begin{exemple}\pjya{ta-ma slaʁnɤslaʁ ɲɯ-ɤz-nɤma}\hspace{5pt}\pcmn{他几下子就做完了工作}\end{exemple}\end{sous-entrée}

\begin{sous-entrée}{phɯslaʁ}{ⓔslaʁⓝphɯslaʁ} 
\classe{idph.5} 
\begin{exemple}\pjya{pɣa phɯslaʁ ʑo thɯ-nɯqambɯmbjom}\hspace{5pt}\pcmn{鸟突然就飞走了}\end{exemple}
\begin{exemple}\pjya{phɯslaʁ ʑo nɯ-ʑɣɤɣɤme}\hspace{5pt}\pcmn{突然间消失了}\end{exemple}\relationsémantique{参考}{\lien{ⓔɕlaʁ}{ɕlaʁ}}\end{sous-entrée}

\end{entrée}

\begin{entrée}{slɤβkhaŋ}{}{ⓔslɤβkhaŋ} 
\classe{n} 
\begin{définition}\pfra{école}\end{définition}
\begin{définition}\pcmn{学校}\end{définition}\étymologie{slob.kʰaŋ}\end{entrée}

\begin{entrée}{slɤŋe}{}{ⓔslɤŋe} 
\classe{n} 
\begin{définition}\pfra{clair de lune}\end{définition}
\begin{définition}\pcmn{月光}\end{définition}\relationsémantique{参考}{\lien{ⓔtɤŋe}{tɤŋe}}\end{entrée}

\begin{entrée}{slɤrɯri}{}{ⓔslɤrɯri} 
\classe{adv} 
\begin{définition}\pfra{tous les mois}\end{définition}
\begin{définition}\pcmn{每个月}\end{définition}\end{entrée}

\begin{entrée}{slɤzɯn}{}{ⓔslɤzɯn} 
\classe{n} 
\begin{définition}\pfra{éclipse de lune}\end{définition}
\begin{définition}\pcmn{月蚀}\end{définition}\end{entrée}

\begin{entrée}{sloŋnɤsloŋ}{}{ⓔsloŋnɤsloŋ} 
\classe{idph.3} 
\begin{définition}\pfra{vagues déferlantes}\end{définition}
\begin{définition}\pcmn{形容波浪汹涌的样子}\end{définition}
\begin{exemple}\pjya{tɯ-ci chɤ-wxti tɕe sloŋnɤsloŋ ɲɯ-ʑɣɤstu}\hspace{5pt}\pcmn{河水暴涨,波浪汹涌}\end{exemple}\end{entrée}

\begin{entrée}{sloʁ}{}{ⓔsloʁ} 
\classe{vt}  
\grammaire{apass} \sens{1}\paradigme{dir}{nɯ-}
\begin{définition}\pfra{fouir (cochon)}\end{définition}
\begin{définition}\pcmn{用鼻子拱(猪)}\end{définition}
\begin{exemple}\pjya{phaʁrgot kɯ sɤtɕha ɲɤ-sloʁ}\hspace{5pt}\pcmn{野猪拱了土}\end{exemple}\sens{2}\paradigme{dir}{tɤ-}
\begin{définition}\pfra{déterrer}\end{définition}
\begin{définition}\pcmn{挖出来;掏出来}\end{définition}
\begin{exemple}\pjya{laχtɕha tɤ-sloʁ-a}\hspace{5pt}\pcmn{我把东西挖出来了}\end{exemple}\sens{3}\paradigme{dir}{nɯ-}
\begin{définition}\pfra{se frayer un chemin dans la foule}\end{définition}
\begin{définition}\pcmn{(从人群中)挤过去}\end{définition}
\begin{exemple}\pjya{tɯrme ra nɯ-sloʁ-a ʑo lɤ-nɯɬoʁ-a}\hspace{5pt}\pcmn{我从人群中挤出来}\end{exemple}\relationsémantique{同义词}{\lien{ⓔnɯɣlɯɣli}{nɯɣlɯɣli}}\sens{4}\paradigme{dir}{tɤ-}\paradigme{dir}{nɯ-}\paradigme{dir}{tɤ-}
\begin{définition}\pfra{fouiller}\end{définition}
\begin{définition}\pcmn{翻来覆去地找}\end{définition}
\begin{exemple}\pjya{rgɤm ɯ-ŋgɯ tɯ-ŋga nɯra to-sloʁ}\hspace{5pt}\pcmn{他翻箱子找衣服}\end{exemple}\relationsémantique{同义词}{\lien{ⓔpɣaʁ}{pɣaʁ}}
\begin{sous-entrée}{rɤsloʁ}{ⓔsloʁⓝrɤsloʁ} 
\classe{vi} \end{sous-entrée}

\begin{exemple}\pjya{paʁ ɲɤ-rɤsloʁ}\hspace{5pt}\pcmn{猪拱地了}\end{exemple}
\begin{sous-entrée}{sɯsloʁ}{ⓔsloʁⓝsɯsloʁ} 
\classe{vt} 
\begin{définition}\pfra{fouir avec}\end{définition}
\begin{définition}\pcmn{用……拱(猪)}\end{définition}
\begin{exemple}\pjya{ɯ-ɕna kɯ ɲɯ-sɯsloʁ tɕe pjɯ-lɣe ɲɯ-ɕti}\hspace{5pt}\pcmn{(野猪)用鼻子拱地}\end{exemple}\end{sous-entrée}

\end{entrée}

\begin{entrée}{sloχpɯn}{}{ⓔsloχpɯn} 
\classe{n} 
\begin{définition}\pfra{professeur}\end{définition}
\begin{définition}\pcmn{老师}\end{définition}\étymologie{slob.dpon}\end{entrée}

\begin{entrée}{slɯŋslɯŋ}{}{ⓔslɯŋslɯŋ} 
\classe{idph.2} 
\begin{définition}\pfra{lumineux et rond}\end{définition}
\begin{définition}\pcmn{形容又亮又圆的样子}\end{définition}
\begin{sous-entrée}{slɯŋɯŋi}{ⓔslɯŋslɯŋⓝslɯŋɯŋi} 
\classe{idph.9} 
\begin{exemple}\pjya{tɤŋe slɯŋɯŋi ʑo to-nɯɬoʁ}\hspace{5pt}\pcmn{太阳慢慢地放着光辉升起来了}\end{exemple}\end{sous-entrée}

\end{entrée}

\begin{entrée}{smar}{}{ⓔsmar} 
\classe{n} 
\begin{définition}\pfra{fleuve}\end{définition}
\begin{définition}\pcmn{河流}\end{définition}\end{entrée}

\begin{entrée}{smɤɣ}{}{ⓔsmɤɣ} 
\classe{n} 
\begin{définition}\pfra{laine}\end{définition}
\begin{définition}\pcmn{羊毛}\end{définition}\end{entrée}

\begin{entrée}{smɤɣri}{}{ⓔsmɤɣri} 
\classe{n} 
\begin{définition}\pfra{fil de laine}\end{définition}
\begin{définition}\pcmn{羊毛线}\end{définition}\relationsémantique{参考}{\lien{ⓔsmɤɣ}{smɤɣ}}\relationsémantique{参考}{\lien{ⓔtɤ-ri}{tɤ-ri}}\end{entrée}

\begin{entrée}{smɤn}{}{ⓔsmɤn} 
\classe{n} 
\begin{définition}\pfra{médicament}\end{définition}
\begin{définition}\pcmn{药}\end{définition}
\begin{exemple}\pjya{smɤn kɤ-ndza a-mɤ-nɯ-tɯ-jmɯt je!}\hspace{5pt}\pcmn{你不要忘记吃药}\end{exemple}
\begin{exemple}\pjya{smɤn tɕhi pɯ-nɯ-ŋɯ-ŋu ʑo nɤ ɯ-tshɤt kɤ-ndza ra}\hspace{5pt}\pcmn{无论什么药,都不能吃太多}\end{exemple}\étymologie{sman}\end{entrée}

\begin{entrée}{smɤnba}{}{ⓔsmɤnba} 
\classe{n} 
\begin{définition}\pfra{médecin}\end{définition}
\begin{définition}\pcmn{医生}\end{définition}\étymologie{sman.pa}\end{entrée}

\begin{entrée}{smɤnkhaŋ}{}{ⓔsmɤnkhaŋ} 
\classe{n} 
\begin{définition}\pfra{hôpital}\end{définition}
\begin{définition}\pcmn{医院}\end{définition}
\begin{exemple}\pjya{smɤnkhaŋ jɤ-tɯ-ari ?}\hspace{5pt}\pcmn{你去了医院吗?}\end{exemple}\end{entrée}

\begin{entrée}{smɤnrɯɣ}{}{ⓔsmɤnrɯɣ} 
\classe{n} 
\begin{définition}\pfra{médicament}\end{définition}
\begin{définition}\pcmn{药材}\end{définition}\étymologie{sman.rigs}\end{entrée}

\begin{entrée}{smɤʁjoʁ}{}{ⓔsmɤʁjoʁ} 
\classe{n} 
\begin{définition}\pfra{habit de moine}\end{définition}
\begin{définition}\pcmn{和尚服装的一种,穿在腰上}\end{définition}\relationsémantique{参考}{\lien{ⓔtɯ-smɤt}{tɯ-smɤt}}\end{entrée}

\begin{entrée}{smɤt}{}{ⓔsmɤt} 
\classe{vt} \paradigme{dir}{pɯ-}
\begin{définition}\pfra{rabaisser qqn}\end{définition}
\begin{définition}\pcmn{贬低}\end{définition}
\begin{exemple}\pjya{ɯʑo kɯ aʑo pɯ́-wɣ-smat-a}\hspace{5pt}\pcmn{他贬低了我}\end{exemple}\relationsémantique{参考}{\lien{ⓔnɯsmɤphɤβ}{nɯsmɤphɤβ}}\end{entrée}

\begin{entrée}{smi}{₂}{ⓔsmiⓗ2} 
\classe{n} 
\begin{définition}\pfra{feu}\end{définition}
\begin{définition}\pcmn{火}\end{définition}
\begin{exemple}\pjya{tɕetha mɤʑɯ tɯtshot kɯmŋu-skɤrma tɕe kɤ-lɤt ma, aj smi ci tu-βze-a ntshi ma ɲɯ-mɯɕtaʁ wo}\hspace{5pt}\pcmn{你过五分钟再打电话,我先烧一点火,因为很冷}\end{exemple}\end{entrée}

\begin{entrée}{smi}{₁}{ⓔsmiⓗ1} 
\classe{vs} \paradigme{dir}{kɤ-}
\begin{définition}\pfra{cuit (bouilli ou à la vapeur)}\end{définition}
\begin{définition}\pcmn{(煮、蒸)熟的}\end{définition}
\begin{exemple}\pjya{mbrɤz ko-smi}\hspace{5pt}\pcmn{饭熟了}\end{exemple}
\begin{exemple}\pjya{tɤ-mthɯm ko-smi}\hspace{5pt}\pcmn{肉熟了}\end{exemple}\relationsémantique{参考}{\lien{}{ɣɤsmi₁}}\end{entrée}

\begin{entrée}{smɯ}{}{ⓔsmɯ} 
\classe{n} 
\begin{définition}\pfra{vache}\end{définition}
\begin{définition}\pcmn{小奶牛}\end{définition}\end{entrée}

\begin{entrée}{smɯɣdɯm}{}{ⓔsmɯɣdɯm} 
\classe{n} 
\begin{définition}\pfra{bois de chauffage}\end{définition}
\begin{définition}\pcmn{柴}\end{définition}\end{entrée}

\begin{entrée}{smɯɣot}{}{ⓔsmɯɣot} 
\classe{n} 
\begin{définition}\pfra{lumière du feu}\end{définition}
\begin{définition}\pcmn{火的热光}\end{définition}
\begin{exemple}\pjya{smɯɣot ɲɯ-mpja}\hspace{5pt}\pcmn{火光发热}\end{exemple}\relationsémantique{参考}{\lien{ⓔsmiⓗ2}{smi₂}}\end{entrée}

\begin{entrée}{smɯɣsmɯɣ}{}{ⓔsmɯɣsmɯɣ} 
\classe{idph.2} 
\begin{définition}\pfra{vert vif}\end{définition}
\begin{définition}\pcmn{形容绿油油的色泽}\end{définition}
\begin{exemple}\pjya{tɯji nɯ smɯɣsmɯɣ ɲɯ-pa}\hspace{5pt}\pcmn{田是绿油油的}\end{exemple}\end{entrée}

\begin{entrée}{smɯlɤm}{}{ⓔsmɯlɤm} 
\classe{n} 
\begin{définition}\pfra{prière, espoir}\end{définition}
\begin{définition}\pcmn{愿望}\end{définition}
\begin{exemple}\pjya{a-pɯ-cha smɯlɤm}\hspace{5pt}\pcmn{希望能够成功!}\end{exemple}\étymologie{smon.lam}\end{entrée}

\begin{entrée}{smɯlju}{}{ⓔsmɯlju} 
\classe{n}  
\grammaire{n.lieu} 
\begin{définition}\pfra{Smeliou (village de Gdongbrgyad)}\end{définition}
\begin{définition}\pcmn{石木留村}\end{définition}\end{entrée}

\begin{entrée}{smɯmba}{}{ⓔsmɯmba} 
\classe{n} 
\begin{définition}\pfra{flamme}\end{définition}
\begin{définition}\pcmn{火焰}\end{définition}\relationsémantique{参考}{\lien{ⓔsmiⓗ2}{smi₂}}\end{entrée}

\begin{entrée}{smɯn}{}{ⓔsmɯn} 
\classe{vs} \paradigme{dir}{pɯ-}
\begin{définition}\pfra{porter des fruits}\end{définition}
\begin{définition}\pcmn{成熟}\end{définition}
\begin{exemple}\pjya{ɯ-mbrɤzɯ smɯn}\hspace{5pt}\pcmn{会有好结果}\end{exemple}\étymologie{smin}\end{entrée}

\begin{entrée}{smɯntʂɯɣ}{}{ⓔsmɯntʂɯɣ} 
\classe{n} 
\begin{définition}\pfra{pléiades}\end{définition}
\begin{définition}\pcmn{昴宿}\end{définition}\étymologie{smin.drug}\end{entrée}

\begin{entrée}{smɯŋgɯ}{}{ⓔsmɯŋgɯ} 
\classe{n} 
\begin{définition}\pfra{milieu du feu}\end{définition}
\begin{définition}\pcmn{火中间}\end{définition}
\begin{exemple}\pjya{smɯŋgɯ kɤ-ɕe mɤ-βdi ma sɤɕke}\hspace{5pt}\pcmn{不要走到火中间,会烫到}\end{exemple}\relationsémantique{参考}{\lien{ⓔsmiⓗ2}{smi₂}}\relationsémantique{参考}{\lien{ⓔɯ-ŋgɯ}{ɯ-ŋgɯ}}\end{entrée}

\begin{entrée}{smɯr}{}{ⓔsmɯr} 
\classe{n} 
\begin{définition}\pfra{morceau de glace flottant sur le fleuve}\end{définition}
\begin{définition}\pcmn{河流上的冰块}\end{définition}\end{entrée}

\begin{entrée}{smɯrqom}{}{ⓔsmɯrqom} 
\classe{n} 
\begin{définition}\pfra{espèce de plante}\end{définition}
\begin{définition}\pcmn{植物的一种}\end{définition}\relationsémantique{参考}{\lien{ⓔɬɤndʐitɤtsoʁ}{ɬɤndʐitɤtsoʁ}}\end{entrée}

\begin{entrée}{smɯʁjoʁ}{}{ⓔsmɯʁjoʁ} 
\classe{n} 
\begin{définition}\pfra{crochet de cheminée}\end{définition}
\begin{définition}\pcmn{火钩}\end{définition}\relationsémantique{同义词}{\lien{ⓔɕɤmiŋoʁ}{ɕɤmiŋoʁ}}\relationsémantique{参考}{\lien{ⓔsmiⓗ2}{smi₂}}\end{entrée}

\begin{entrée}{smɯʁrɤt}{}{ⓔsmɯʁrɤt} 
\classe{n} 
\begin{définition}\pfra{cendre}\end{définition}
\begin{définition}\pcmn{火炭}\end{définition}\end{entrée}

\begin{entrée}{smɯsmi}{}{ⓔsmɯsmi} 
\classe{n} 
\begin{définition}\pfra{bûche utilisée pour rallumer le feu}\end{définition}
\begin{définition}\pcmn{火种}\end{définition}\end{entrée}

\begin{entrée}{smɯtɕɣom}{}{ⓔsmɯtɕɣom} 
\classe{n} 
\begin{définition}\pfra{étincelle}\end{définition}
\begin{définition}\pcmn{火星}\end{définition}\end{entrée}

\begin{entrée}{sna}{}{ⓔsna} 
\classe{vs} \paradigme{dir}{tɤ-}\sens{1}
\begin{définition}\pfra{utilisable}\end{définition}
\begin{définition}\pcmn{质量好}\end{définition}
\begin{exemple}\pjya{laχtɕha ɲɯ-sna}\hspace{5pt}\pcmn{那个东西的质量很好}\end{exemple}
\begin{exemple}\pjya{ɕoŋtɕa ɲɯ-sna}\hspace{5pt}\pcmn{木料很好}\end{exemple}
\begin{exemple}\pjya{tɯ-rɟɯ ɲɯ-sna}\hspace{5pt}\pcmn{财物的质量好}\end{exemple}
\begin{exemple}\pjya{tʂha ɲɯ-sna}\hspace{5pt}\pcmn{茶很浓}\end{exemple}
\begin{exemple}\pjya{nɤʑo sna mataŋe}\hspace{5pt}\pcmn{你真没有用}\end{exemple}
\begin{exemple}\pjya{sna ɣɤtɤʑu nɤ, ɲɯ-tɯ-rkaŋ!}\hspace{5pt}\pcmn{你真能干!}\end{exemple}\sens{2}
\begin{définition}\pfra{généreux}\end{définition}
\begin{définition}\pcmn{大方;热情;善良}\end{définition}\end{entrée}

\begin{entrée}{snalŋaɕthɤβ}{}{ⓔsnalŋaɕthɤβ} 
\classe{n} 
\begin{définition}\pfra{lanière colorée (pantalon, chaussures)}\end{définition}
\begin{définition}\pcmn{彩色的带子(系衣服、鞋子)}\end{définition}\end{entrée}

\begin{entrée}{snama}{}{ⓔsnama} 
\classe{n} 
\begin{définition}\pfra{animal domestique capable de porter des charges}\end{définition}
\begin{définition}\pcmn{专门用来驮东西的牲畜}\end{définition}\end{entrée}

\begin{entrée}{snaŋwa}{}{ⓔsnaŋwa} 
\classe{n} 
\begin{définition}\pfra{état d'esprit}\end{définition}
\begin{définition}\pcmn{性情}\end{définition}
\begin{exemple}\pjya{snaŋwa ɲɯ-mthu}\hspace{5pt}\pcmn{自信心强}\end{exemple}\étymologie{snaŋ.ba}\end{entrée}

\begin{entrée}{snaʁtsa}{}{ⓔsnaʁtsa} 
\classe{n} 
\begin{définition}\pfra{encre}\end{définition}
\begin{définition}\pcmn{墨}\end{définition}\étymologie{snag.tsʰa}\end{entrée}

\begin{entrée}{snoŋwa}{}{ⓔsnoŋwa} 
\classe{n} 
\begin{définition}\pfra{confiance en soi}\end{définition}
\begin{définition}\pcmn{自信}\end{définition}
\begin{exemple}\pjya{ɯ-snoŋwa ɲɯ-mthu}\hspace{5pt}\pcmn{他很有自信,觉得自高自大}\end{exemple}\étymologie{snaŋ.ba}\end{entrée}

\begin{entrée}{snɯm}{}{ⓔsnɯm} 
\classe{n} 
\begin{définition}\pfra{huile contenu dans les poils; viande, huile}\end{définition}
\begin{définition}\pcmn{羊毛和牛毛中含着的油脂;肉;油的统称}\end{définition}\étymologie{snum}\end{entrée}

\begin{entrée}{snɯɲaʁ}{}{ⓔsnɯɲaʁ} 
\classe{n} 
\begin{définition}\pfra{fait de causer du tort}\end{définition}
\begin{définition}\pcmn{害人(的阴谋)}\end{définition}
\begin{exemple}\pjya{snɯɲaʁ ma-tɤ-tɯ-βze}\hspace{5pt}\pcmn{你不要害人}\end{exemple}\relationsémantique{参考}{\lien{ⓔtɯ-sni}{tɯ-sni}}\relationsémantique{参考}{\lien{ⓔɲaʁ}{ɲaʁ}}\relationsémantique{参考}{\lien{ⓔnɯsnɯɲaʁ}{nɯsnɯɲaʁ}}\end{entrée}

\begin{entrée}{sɲu}{}{ⓔsɲu} 
\classe{vi}  
\grammaire{caus} \paradigme{dir}{nɯ-}
\begin{définition}\pfra{être fou}\end{définition}
\begin{définition}\pcmn{疯}\end{définition}\paradigme{dir}{nɯ-}
\begin{exemple}\pjya{jiɕqha nɯ ɲɤ-sɲu}\hspace{5pt}\pcmn{那个人疯了}\end{exemple}
\begin{exemple}\pjya{kɯ-sɲu ci ɲɯ-ŋu}\hspace{5pt}\pcmn{他是疯子}\end{exemple}\relationsémantique{参考}{\lien{ⓔchɯsɲu}{chɯsɲu}}
\begin{sous-entrée}{sɯsɲu}{ⓔsɲuⓝsɯsɲu} 
\classe{vt} \end{sous-entrée}

\begin{définition}\pfra{rendre fou}\end{définition}
\begin{définition}\pcmn{令……疯}\end{définition}
\begin{sous-entrée}{nɤsɲɯsɲu}{ⓔsɲuⓝnɤsɲɯsɲu} 
\classe{vs} 
\begin{définition}\pfra{un peu fou (parfois normal, parfois fou)}\end{définition}
\begin{définition}\pcmn{疯疯癫癫}\end{définition}
\begin{exemple}\pjya{tɯrme kɯ-nɤsɲɯsɲu ci jɤ-ɣe}\hspace{5pt}\pcmn{来了一个疯疯癫癫的人}\end{exemple}\end{sous-entrée}

\étymologie{smʲo}\end{entrée}

\begin{entrée}{sɲaŋne}{}{ⓔsɲaŋne} 
\classe{n} 
\begin{définition}\pfra{upavâsa, fête du jeûne}\end{définition}
\begin{définition}\pcmn{哑巴经(禁食斋)}\end{définition}
\begin{exemple}\pjya{sɲaŋne kɤ-ndo-t-a}\hspace{5pt}\pcmn{我念了哑巴经}\end{exemple}\relationsémantique{参考}{\lien{ⓔrɯsɲaŋne}{rɯsɲaŋne}}\étymologie{smʲuŋ.gnas}\end{entrée}

\begin{entrée}{sɲaʁsɲaʁ}{}{ⓔsɲaʁsɲaʁ} 
\classe{idph.2} 
\begin{définition}\pfra{très pointu}\end{définition}
\begin{définition}\pcmn{形容很尖}\end{définition}
\begin{exemple}\pjya{ɲɯ-ɤmtɕoʁ ʑo sɲaʁsɲaʁ}\hspace{5pt}\pcmn{非常尖}\end{exemple}
\begin{exemple}\pjya{mbrɯtɕɯ ɯ-ku amtɕoʁ ʑo sɲaʁsɲaʁ}\hspace{5pt}\pcmn{刀子很尖}\end{exemple}\end{entrée}

\begin{entrée}{sɲɤβnɤlɤβ}{}{ⓔsɲɤβnɤlɤβ} 
\classe{idph.4} 
\begin{définition}\pfra{titubant}\end{définition}
\begin{définition}\pcmn{形容走路摇晃不稳}\end{définition}
\begin{exemple}\pjya{lo-βzi tɕe sɲɤβnɤlɤβ ʑo ɲɯ-ʑɣɤstu}\hspace{5pt}\pcmn{他喝醉了,走路东摇西摆的}\end{exemple}\end{entrée}

\begin{entrée}{sɲɤɣsɲɤɣ}{}{ⓔsɲɤɣsɲɤɣ} 
\classe{idph.2} \sens{1}
\begin{définition}\pfra{grand et gros (corps)}\end{définition}
\begin{définition}\pcmn{形容身体大而胖的样子}\end{définition}
\begin{exemple}\pjya{tɤ-pɤtso cho-wxti sɲɤɣsɲɤɣ ʑo}\hspace{5pt}\pcmn{小孩子长大了,变得又大又胖}\end{exemple}\sens{2}
\begin{définition}\pfra{bête}\end{définition}
\begin{définition}\pcmn{形容笨重或者不聪明的样子}\end{définition}
\begin{exemple}\pjya{sɲɤɣsɲɤɣ ʑo ma-tɤ-tɯ-ʑɣɤstu}\hspace{5pt}\pcmn{你要做出傻乎乎的样子}\end{exemple}\end{entrée}

\begin{entrée}{sɲɤt}{}{ⓔsɲɤt} 
\classe{n} 
\begin{définition}\pfra{harnais}\end{définition}
\begin{définition}\pcmn{后輶}\end{définition}\étymologie{rmed}\end{entrée}

\begin{entrée}{sɲikuku}{}{ⓔsɲikuku} 
\classe{adv} 
\begin{définition}\pfra{tous les jours}\end{définition}
\begin{définition}\pcmn{每天}\end{définition}\end{entrée}

\begin{entrée}{sɲoʁ}{}{ⓔsɲoʁ} 
\classe{vs} 
\begin{définition}\pfra{épaisse (huile)}\end{définition}
\begin{définition}\pcmn{稠(油)}\end{définition}
\begin{exemple}\pjya{pɤlɤtɕɯ wuma ʑo ɲɯ-sɲoʁ tɕe kɯ-dɤn kɤ-ndza mɯ́j-sɤcha}\hspace{5pt}\pcmn{酥油馍馍有多油,吃不下很多}\end{exemple}\end{entrée}

\begin{entrée}{sɲɯɣ}{}{ⓔsɲɯɣ} 
\classe{idph.1} 
\begin{définition}\pfra{douleur lancinante}\end{définition}
\begin{définition}\pcmn{形容痛得钻心}\end{définition}
\begin{exemple}\pjya{ɯ-tɯ-mŋɤm kɯ sɲɯɣ ʑo ɲɯ-ti}\hspace{5pt}\pcmn{他一下子就痛得钻心}\end{exemple}
\begin{sous-entrée}{sɲɯɣnɤsɲɯɣ}{ⓔsɲɯɣⓝsɲɯɣnɤsɲɯɣ} 
\classe{idph.3} 
\begin{exemple}\pjya{sɲɯɣnɤsɲɯɣ ʑo ɲɯ-ti ɲɯ-mŋɤm}\hspace{5pt}\pcmn{他痛得一阵又一阵}\end{exemple}\relationsémantique{参考}{\lien{ⓔɕɲɯɣ}{ɕɲɯɣ}}\end{sous-entrée}

\end{entrée}

\begin{entrée}{sɲɯɣjɯ}{}{ⓔsɲɯɣjɯ} 
\classe{n} 
\begin{définition}\pfra{pinceau}\end{définition}
\begin{définition}\pcmn{笔}\end{définition}\étymologie{smʲu.gu}\end{entrée}

\begin{entrée}{sɲɯŋgɯpala}{}{ⓔsɲɯŋgɯpala} 
\classe{adv} 
\begin{définition}\pfra{en plein jour}\end{définition}
\begin{définition}\pcmn{大白天}\end{définition}\end{entrée}

\begin{entrée}{sɲɯŋɯŋi}{}{ⓔsɲɯŋɯŋi} 
\classe{idph.7} 
\begin{définition}\pfra{s'élevant (fumée)}\end{définition}
\begin{définition}\pcmn{形容青烟慢慢升起的样子}\end{définition}
\begin{exemple}\pjya{tɤ-khɯ sɲɯŋɯŋi ʑo to-ɬoʁ}\hspace{5pt}\pcmn{青烟袅袅升起}\end{exemple}\end{entrée}

\begin{entrée}{sŋu}{}{ⓔsŋu} 
\classe{vt} \paradigme{dir}{tɤ-}
\begin{définition}\pfra{ordonner}\end{définition}
\begin{définition}\pcmn{嘱咐}\end{définition}
\begin{exemple}\pjya{sprɯskɯ ra kɯ a-ɕki ``nɯ a-tɤ-tɯ-fse ra" ta-sŋu-nɯ}\hspace{5pt}\pcmn{活佛们嘱咐我要这样做}\end{exemple}\end{entrée}

\begin{entrée}{sŋa}{}{ⓔsŋa} 
\classe{vi}  
\grammaire{caus} \paradigme{dir}{tɤ-}\paradigme{dir}{tɤ-}
\begin{définition}\pfra{revivre, revenir à soi}\end{définition}
\begin{définition}\pcmn{复活;苏醒}\end{définition}
\begin{exemple}\pjya{to-sŋa}\hspace{5pt}\pcmn{他苏醒了}\end{exemple}
\begin{sous-entrée}{sɯsŋa}{ⓔsŋaⓝsɯsŋa} 
\classe{vt} \end{sous-entrée}

\begin{définition}\pfra{faire revivre}\end{définition}
\begin{définition}\pcmn{令人复活、苏醒}\end{définition}\end{entrée}

\begin{entrée}{sŋarɤβ}{}{ⓔsŋarɤβ} 
\classe{n} 
\begin{définition}\pfra{autrefois}\end{définition}
\begin{définition}\pcmn{古时候}\end{définition}\étymologie{sŋa.rabs}\end{entrée}

\begin{entrée}{sŋarɯ}{}{ⓔsŋarɯ} 
\classe{n} 
\begin{définition}\pfra{avant de la selle}\end{définition}
\begin{définition}\pcmn{前鞍桥}\end{définition}\étymologie{sŋa.ru}\end{entrée}

\begin{entrée}{sŋaʁ}{₂}{ⓔsŋaʁⓗ2} 
\classe{n} 
\begin{définition}\pfra{sorcellerie}\end{définition}
\begin{définition}\pcmn{法术}\end{définition}
\begin{exemple}\pjya{ɯʑo sŋaʁ pjɤ-k-ɤʁe-ci}\hspace{5pt}\pcmn{他被施了法术}\end{exemple}
\begin{exemple}\pjya{sŋaʁ nɯra maka mɯ-pjɤ-mɟa}\hspace{5pt}\pcmn{法术对她根本无效}\end{exemple}\étymologie{sŋags}\end{entrée}

\begin{entrée}{sŋaʁ}{₁}{ⓔsŋaʁⓗ1} 
\classe{vt} \paradigme{dir}{tɤ-}
\begin{définition}\pfra{maudire, ensorceller}\end{définition}
\begin{définition}\pcmn{施法术;诅咒}\end{définition}\paradigme{dir}{tɤ-}\paradigme{dir}{tɤ-}
\begin{définition}\pfra{faire ensorceller}\end{définition}
\begin{définition}\pcmn{叫人用法术}\end{définition}
\begin{définition}\pfra{se faire ensorceller}\end{définition}
\begin{définition}\pcmn{招人给自己施法术}\end{définition}
\begin{exemple}\pjya{to-sŋaʁ (=tɯ-sŋaʁ to-ɕe)}\hspace{5pt}\pcmn{他咒了他}\end{exemple}
\begin{exemple}\pjya{tɤ́-wɣ-sŋaʁ-a}\hspace{5pt}\pcmn{他咒了我}\end{exemple}
\begin{sous-entrée}{sɯsŋaʁ}{ⓔsŋaʁⓗ1ⓝsɯsŋaʁ} 
\classe{vt} \end{sous-entrée}

\begin{sous-entrée}{ʑɣɤsɯsŋaʁ}{ⓔsŋaʁⓗ1ⓝʑɣɤsɯsŋaʁ} 
\classe{vi} \end{sous-entrée}

\étymologie{sŋags}\end{entrée}

\begin{entrée}{sŋaʁspa}{}{ⓔsŋaʁspa} 
\classe{n} 
\begin{définition}\pfra{sorcier}\end{définition}
\begin{définition}\pcmn{法师;巫师}\end{définition}\étymologie{sŋags.pa}\end{entrée}

\begin{entrée}{sŋɤsŋɤt}{}{ⓔsŋɤsŋɤt} 
\classe{idph.2} 
\begin{définition}\pfra{lourdaud}\end{définition}
\begin{définition}\pcmn{形容呆头呆脑的样子}\end{définition}\end{entrée}

\begin{entrée}{sŋi}{}{ⓔsŋi} 
\classe{n} 
\begin{définition}\pfra{journée}\end{définition}
\begin{définition}\pcmn{白天}\end{définition}\end{entrée}

\begin{entrée}{sŋiɕɤr}{}{ⓔsŋiɕɤr} 
\classe{n} 
\begin{définition}\pfra{jour et nuit}\end{définition}
\begin{définition}\pcmn{白天黑夜}\end{définition}\end{entrée}

\begin{entrée}{sŋo}{}{ⓔsŋo} 
\classe{n} 
\begin{définition}\pfra{rhododendron (dessous des feuilles orange)}\end{définition}
\begin{définition}\pcmn{羊角花(叶子背面黄色)}\end{définition}\end{entrée}

\begin{entrée}{sŋom}{}{ⓔsŋom} 
\classe{vi}
\classe{vs} \paradigme{dir}{nɯ-}
\begin{définition}\pfra{envier}\end{définition}
\begin{définition}\pcmn{贪;羡慕}\end{définition}
\begin{exemple}\pjya{tɯrɟaʁ ɯ-kɯ-nɤmɲo ju-ɕe-nɯ ɲɯ-ŋu tɕe, aʑo ɲɯ-sŋom-a.}\hspace{5pt}\pcmn{他们要去看舞蹈,我很羡慕(我去不成)}\end{exemple}\relationsémantique{参考}{\lien{ⓔnɯsŋom}{nɯsŋom}}\relationsémantique{参考}{\lien{ⓔrɟɯrŋom}{rɟɯrŋom}}\relationsémantique{参考}{\lien{ⓔnɯrɟɯrŋom}{nɯrɟɯrŋom}}
\begin{sous-entrée}{sɤsŋom}{ⓔsŋomⓝsɤsŋom} 
\grammaire{deexp} 
\begin{définition}\pfra{donner envie}\end{définition}
\begin{définition}\pcmn{令人羡慕}\end{définition}\end{sous-entrée}

\étymologie{rŋam}\end{entrée}

\begin{entrée}{sŋorma}{}{ⓔsŋorma} 
\classe{n} 
\begin{définition}\pfra{plantes qui poussent au printemps}\end{définition}
\begin{définition}\pcmn{春天发芽了的各种植物}\end{définition}\étymologie{sŋo}\end{entrée}

\begin{entrée}{sŋoʁmɤr}{}{ⓔsŋoʁmɤr} 
\classe{n} 
\begin{définition}\pfra{beurre sur lequel un moine a soufflé}\end{définition}
\begin{définition}\pcmn{和尚念经后,在上面吹过的酥油}\end{définition}\étymologie{mar}\end{entrée}

\begin{entrée}{sŋur}{}{ⓔsŋur} 
\classe{vi} \paradigme{dir}{tɤ-}
\begin{définition}\pfra{ronfler}\end{définition}
\begin{définition}\pcmn{打鼾}\end{définition}
\begin{exemple}\pjya{nɤ-tɯ-sŋur nɯ}\hspace{5pt}\pcmn{你鼾声打得很大声}\end{exemple}
\begin{exemple}\pjya{jɯfɕɯɕɤr nɤʑo pɯ-tɯ-sŋur}\hspace{5pt}\pcmn{你昨天晚上打了鼾}\end{exemple}
\begin{exemple}\pjya{to-tɯ-sŋur}\hspace{5pt}\pcmn{你原来不打鼾,现在就打鼾了}\end{exemple}
\begin{exemple}\pjya{aʑo kɤ-nɯʑɯβ-a tɕe, wuma ɲɯ-sŋur-a}\hspace{5pt}\pcmn{我睡的时候打鼾}\end{exemple}\étymologie{sŋur}\end{entrée}

\begin{entrée}{sŋɯqiɯ}{}{ⓔsŋɯqiɯ} 
\classe{n} 
\begin{définition}\pfra{une demi journée}\end{définition}
\begin{définition}\pcmn{半天}\end{définition}\end{entrée}

\begin{entrée}{sŋɯχcɤl}{}{ⓔsŋɯχcɤl} 
\classe{n} 
\begin{définition}\pfra{midi}\end{définition}
\begin{définition}\pcmn{中午}\end{définition}\end{entrée}

\begin{entrée}{sŋɯχcɤlpala}{}{ⓔsŋɯχcɤlpala} 
\classe{n} 
\begin{définition}\pfra{en plein jour}\end{définition}
\begin{définition}\pcmn{大白天}\end{définition}\end{entrée}

\begin{entrée}{so}{}{ⓔso} 
\classe{vs} \paradigme{dir}{tɤ-}\paradigme{dir}{pɯ-}\paradigme{dir}{thɯ-}
\begin{définition}\pfra{vide}\end{définition}
\begin{définition}\pcmn{空}\end{définition}
\begin{définition}\pfra{vider}\end{définition}
\begin{définition}\pcmn{掏空}\end{définition}
\begin{exemple}\pjya{ɲɯ-so}\hspace{5pt}\pcmn{是空的}\end{exemple}
\begin{exemple}\pjya{tɯji kɯ-so tu}\hspace{5pt}\pcmn{有空着的地(没有种庄稼的)}\end{exemple}
\begin{exemple}\pjya{kha kɯ-so ɕti}\hspace{5pt}\pcmn{房间是空的}\end{exemple}
\begin{exemple}\pjya{tɤ-fkɯm pɯ-sɯxsɤm}\hspace{5pt}\pcmn{你把口袋腾空}\end{exemple}\relationsémantique{参考}{\lien{ⓔɯ-xso}{ɯ-xso}}\relationsémantique{参考}{\lien{ⓔnɯxso}{nɯxso}}
\begin{sous-entrée}{sɯxso}{ⓔsoⓝsɯxso} 
\classe{vt} \end{sous-entrée}

\end{entrée}

\begin{entrée}{somo khɯtsa}{}{ⓔsomo khɯtsa} 
\classe{n} 
\begin{définition}\pfra{bol en bois}\end{définition}
\begin{définition}\pcmn{木碗}\end{définition}\end{entrée}

\begin{entrée}{soʁdɤr}{}{ⓔsoʁdɤr} 
\classe{n} 
\begin{définition}\pfra{rabot}\end{définition}
\begin{définition}\pcmn{锉刀}\end{définition}\end{entrée}

\begin{entrée}{soʁma}{}{ⓔsoʁma} 
\classe{n} 
\begin{définition}\pfra{paille de blé}\end{définition}
\begin{définition}\pcmn{麦秸}\end{définition}\étymologie{sog.ma}\end{entrée}

\begin{entrée}{soskɯsku/\variante{soskuku}}{}{ⓔsoskɯsku} 
\classe{adv} 
\begin{définition}\pfra{tous les matins}\end{définition}
\begin{définition}\pcmn{每天早上}\end{définition}\relationsémantique{参考}{\lien{ⓔsoz}{soz}}\end{entrée}

\begin{entrée}{soχpu}{}{ⓔsoχpu} 
\classe{n} 
\begin{définition}\pfra{mongol}\end{définition}
\begin{définition}\pcmn{蒙古人}\end{définition}\étymologie{sog.po}\end{entrée}

\begin{entrée}{soz}{}{ⓔsoz} 
\classe{adv} 
\begin{définition}\pfra{le matin}\end{définition}
\begin{définition}\pcmn{早上}\end{définition}\end{entrée}

\begin{entrée}{sozdɯmtɕi/\variante{sostɯmtɕi}}{}{ⓔsozdɯmtɕi} 
\classe{n} 
\begin{définition}\pfra{au point du jour}\end{définition}
\begin{définition}\pcmn{清早}\end{définition}\end{entrée}

\begin{entrée}{spa}{}{ⓔspa} 
\classe{vt} \paradigme{dir}{kɤ-}\paradigme{dir}{kɤ-}\paradigme{dir}{pɯ-}
\begin{définition}\pfra{pouvoir}\end{définition}
\begin{définition}\pcmn{会}\end{définition}
\begin{définition}\pfra{rendre capable}\end{définition}
\begin{définition}\pcmn{令别人学会}\end{définition}
\begin{exemple}\pjya{ɕoŋβzu ko-spa}\hspace{5pt}\pcmn{他会木工了}\end{exemple}
\begin{exemple}\pjya{kɤ-rɤrɤt ko-spa}\hspace{5pt}\pcmn{他会写字了}\end{exemple}
\begin{exemple}\pjya{ɯ-kɤ-spa ɲɯ-dɤn}\hspace{5pt}\pcmn{他会做的事情很多}\end{exemple}
\begin{exemple}\pjya{tɤ-rʑaʁ pɯ-zri ri, pɤjkhu mɯ́j-spe-a}\hspace{5pt}\pcmn{过了很久,我还是不会}\end{exemple}
\begin{exemple}\pjya{ɯʑo kɯ spe tɕeri, ɯʑo ɯ-ʁa khro maŋe wo}\hspace{5pt}\pcmn{她会做,但是她没有空}\end{exemple}
\begin{exemple}\pjya{"a-wa" tu-ti ɯ́-spe, "a-mu tu-ti" ra ɯ́-spe?}\hspace{5pt}\pcmn{(你儿子现在)会不会说“爸爸、妈妈”?}\end{exemple}
\begin{exemple}\pjya{mɯ-tɤ-spa-t-a ri me, pɯ-rɯɣnan-a ri me}\hspace{5pt}\pcmn{我没有做对不起(你)的事情,也没有跟你作对}\end{exemple}
\begin{exemple}\pjya{nɤʑɯɣ mɯ-ta-spa ri me, nɤʑɯɣ pɯ-rɯɣnɤn ri me}\hspace{5pt}\pcmn{他没有做对不起你的事情,也没有跟你作对}\end{exemple}
\begin{exemple}\pjya{kɤ-rɯkhɤcɤl a-kɤ-spe-a ɲɯ-ra}\hspace{5pt}\pcmn{我要学会(用藏语)讲话}\end{exemple}
\begin{exemple}\pjya{nɤʑo tɤ-rʑaʁ thɤstɯɣ jamar kɤ-βzjoz kɤ-tɯ-spa-t?}\hspace{5pt}\pcmn{你学了多久才学会了?}\end{exemple}
\begin{exemple}\pjya{ɯʑo kɤ-rɯcɤβŋgɤβ mɤ-spe}\hspace{5pt}\pcmn{他不会骄傲自大(没有那个性格)}\end{exemple}
\begin{exemple}\pjya{kɤ-sɯspa-t-a}\hspace{5pt}\pcmn{我令他学会了}\end{exemple}
\begin{exemple}\pjya{pɯ-nɯspɯspa-t-a ɕti}\hspace{5pt}\pcmn{我本来就会了}\end{exemple}
\begin{exemple}\pjya{@qiche kɤ-lɤt pa-nɯspɯspa ɕti}\hspace{5pt}\pcmn{他本来就会开车}\end{exemple}
\begin{sous-entrée}{sɯspa}{ⓔspaⓝsɯspa} 
\classe{vt}  
\grammaire{caus} \end{sous-entrée}

\begin{sous-entrée}{sɤspa}{ⓔspaⓝsɤspa} 
\classe{vs}  
\grammaire{deexp} 
\begin{définition}\pfra{que l'on connait}\end{définition}
\begin{définition}\pcmn{(人)会做的,知道的}\end{définition}
\begin{exemple}\pjya{mɤʑɯ ɯ-rmi mɤ-kɯ-sɤspa xcat ɕti}\hspace{5pt}\pcmn{还有很多不知道名字的}\end{exemple}\end{sous-entrée}

\begin{sous-entrée}{nɯspɯspa}{ⓔspaⓝnɯspɯspa} 
\classe{vt} \end{sous-entrée}

\end{entrée}

\begin{entrée}{spa,rka}{}{ⓔspa,rka} 
\classe{vt}
\classe{vt} \paradigme{dir}{tɤ-}
\begin{définition}\pfra{être innocent}\end{définition}
\begin{définition}\pcmn{无辜}\end{définition}
\begin{exemple}\pjya{mɤ-spe mɤ-rke kɯ-me}\hspace{5pt}\pcmn{无辜的人}\end{exemple}
\begin{exemple}\pjya{mkhɤrmaŋ mɤ-spa-nɯ mɤ-rka-nɯ kɯ-me}\hspace{5pt}\pcmn{无辜的老百姓}\end{exemple}\relationsémantique{Component 1}{\lien{ⓔspa}{spa}}\relationsémantique{Component 2}{\lien{}{rka}}\forme{1s}{mɤ-spe-a mɤ-rke-a me}\forme{2s}{mɤ-tɯ-spe mɤ-tɯ-rke me}\forme{2s}{mɤ-tɯ-sperke me}\forme{2s}{mɯ-tɤ-tɯ-spa-t-rka-t me}\end{entrée}

\begin{entrée}{spɤr}{}{ⓔspɤr} 
\classe{vt} \paradigme{dir}{\_}
\begin{définition}\pfra{changer de lieu de résidence, déménager}\end{définition}
\begin{définition}\pcmn{搬迁}\end{définition}
\begin{exemple}\pjya{rɯ na-spɤr-nɯ}\hspace{5pt}\pcmn{他们搬帐篷了}\end{exemple}
\begin{sous-entrée}{rɤspɤr}{ⓔspɤrⓝrɤspɤr} 
\classe{vi}  
\grammaire{apass} 
\begin{définition}\pfra{changer de lieu de résidence}\end{définition}
\begin{définition}\pcmn{搬迁}\end{définition}\relationsémantique{同义词}{\lien{ⓔcit}{cit}}\end{sous-entrée}

\end{entrée}

\begin{entrée}{spɤrmbɯt}{}{ⓔspɤrmbɯt} 
\classe{adv} 
\begin{définition}\pfra{partir tout d'un coup}\end{définition}
\begin{définition}\pcmn{突然就走}\end{définition}
\begin{exemple}\pjya{spɤrmbɯt jo-ɕe}\hspace{5pt}\pcmn{他突然就走了}\end{exemple}\relationsémantique{参考}{\lien{ⓔspɤr}{spɤr}}\end{entrée}

\begin{entrée}{spɤt}{}{ⓔspɤt} 
\classe{vi} \paradigme{dir}{\_}\paradigme{dir}{\_}
\begin{définition}\pfra{se déchirer, s'ouvrir jusqu'au bord (après qu'on ait tiré dessus)}\end{définition}
\begin{définition}\pcmn{(一拉就)扯断、拉破}\end{définition}
\begin{définition}\pfra{tirer en mordant ou en déchirant}\end{définition}
\begin{définition}\pcmn{扯断;一边拉一边咬断}\end{définition}
\begin{exemple}\pjya{tɯ-ŋga mɯ́j-ngɯt tɕe, jú-wɣ-rɤɕi tɕe ju-spɤt ɲɯ-ɕti}\hspace{5pt}\pcmn{衣服不结实,一扯就会拉破}\end{exemple}
\begin{exemple}\pjya{ɯ-mtsɯ pjɤ-spɤt}\hspace{5pt}\pcmn{扣子的眼子拉破了}\end{exemple}
\begin{exemple}\pjya{tɯ-ŋga pɯ-kɯ-ɴɢraʁ nɯ nɯ-sɯspat-a}\hspace{5pt}\pcmn{我把破烂的衣服扯断了}\end{exemple}
\begin{sous-entrée}{sɯspɤt}{ⓔspɤtⓝsɯspɤt} 
\classe{vt} \end{sous-entrée}

\end{entrée}

\begin{entrée}{spɣɤnthar/\variante{spɣɤnthɣar}}{}{ⓔspɣɤnthar} 
\classe{n} 
\begin{définition}\pfra{chiquenaude}\end{définition}
\begin{définition}\pcmn{弹手指}\end{définition}
\begin{exemple}\pjya{spɣɤnthɣar tɤ-lat-a}\hspace{5pt}\pcmn{我弹了手指}\end{exemple}\end{entrée}

\begin{entrée}{spɣi}{}{ⓔspɣi} 
\classe{n} 
\begin{définition}\pfra{grenier}\end{définition}
\begin{définition}\pcmn{用木头建造的仓库}\end{définition}\end{entrée}

\begin{entrée}{spɣɯthoʁ}{}{ⓔspɣɯthoʁ} 
\classe{n} 
\begin{définition}\pfra{sol du grenier}\end{définition}
\begin{définition}\pcmn{仓库的地面}\end{définition}\end{entrée}

\begin{entrée}{sphjar}{}{ⓔsphjar} 
\classe{vt} \paradigme{dir}{tɤ-}
\begin{définition}\pfra{étendre pour sécher}\end{définition}
\begin{définition}\pcmn{展开(晾干)}\end{définition}
\begin{exemple}\pjya{a-ŋga ɲɯ-ɤci tɕe tɤ-sphjar-a}\hspace{5pt}\pcmn{我衣服是湿的,所以我把它展开了(晾干)}\end{exemple}
\begin{exemple}\pjya{tɯ-ndʐi pɯ-tɯ-qar tɕe tɤ-sphjar}\hspace{5pt}\pcmn{你剥了皮就把它展开}\end{exemple}\relationsémantique{同义词}{\lien{ⓔsqhiar}{sqhiar}}\end{entrée}

\begin{entrée}{sphjaʁ}{}{ⓔsphjaʁ} 
\classe{vt} \paradigme{dir}{kɤ-}
\begin{définition}\pfra{mouiller, s’infiltrer complètement}\end{définition}
\begin{définition}\pcmn{浸透;烫透;冷透}\end{définition}
\begin{exemple}\pjya{tɯ-mɯ kɤ-lɤt tɕe, kɤ́-wɣ-sphjaʁ-a}\hspace{5pt}\pcmn{下雨了,我就被雨水淋湿了}\end{exemple}
\begin{exemple}\pjya{tɯ-ci kɯ kɤ́-wɣ-sphjaʁ-a}\hspace{5pt}\pcmn{我被淋湿了}\end{exemple}
\begin{sous-entrée}{sɯsphjaʁ}{ⓔsphjaʁⓝsɯsphjaʁ} 
\classe{vt}  
\grammaire{habil} 
\begin{définition}\pfra{traverser}\end{définition}
\begin{définition}\pcmn{穿透}\end{définition}
\begin{exemple}\pjya{tɤtʂu ɯ-tɯ-fsoʁ kɯ tɯ-ŋga ra ku-sɯsphjaʁ ɲɯ-ɕti}\hspace{5pt}\pcmn{灯光穿透衣服}\end{exemple}
\begin{exemple}\pjya{tɯ-rju kɤ-sɯsphjaʁ mɯ-pɯ-cha-a}\hspace{5pt}\pcmn{我说了一句,但他们没有理睬我}\end{exemple}\end{sous-entrée}

\end{entrée}

\begin{entrée}{sphɯt}{}{ⓔsphɯt}\relationsémantique{参考}{\lien{ⓔphɯt}{phɯt}}\end{entrée}

\begin{entrée}{spjaŋkɯ}{}{ⓔspjaŋkɯ} 
\classe{n} 
\begin{définition}\pfra{loup}\end{définition}
\begin{définition}\pcmn{狼}\end{définition}\étymologie{spʲaŋ.ki}\end{entrée}

\begin{entrée}{spjɤt}{}{ⓔspjɤt} 
\classe{vt} \paradigme{dir}{tɤ-}
\begin{définition}\pfra{montrer, faire une démonstration (de ses capacités)}\end{définition}
\begin{définition}\pcmn{展现出来,显示(自己的本领)}\end{définition}
\begin{exemple}\pjya{tɤ-spjat-a}\hspace{5pt}\pcmn{我显示了}\end{exemple}
\begin{exemple}\pjya{ɯ-kɤ-ro kɯ-tu ra ta-nɯ-spjɤt}\hspace{5pt}\pcmn{他把他拥有的东西展现出来了}\end{exemple}
\begin{exemple}\pjya{nɤ-kɤ-cha tɤ-nɯ-spjɤt}\hspace{5pt}\pcmn{你显示一下你的能力}\end{exemple}\étymologie{spʲod}\end{entrée}

\begin{entrée}{spjɤtɕha}{}{ⓔspjɤtɕha} 
\classe{n} 
\begin{définition}\pfra{action}\end{définition}
\begin{définition}\pcmn{行为}\end{définition}
\begin{exemple}\pjya{ki nɤʑo nɤ-spjɤtɕha ŋu}\hspace{5pt}\pcmn{这是你搞的鬼}\end{exemple}
\begin{exemple}\pjya{nɤ-spjɤtɕha ɯ-tɯ-ŋɤn}\hspace{5pt}\pcmn{你的行为很恶毒!}\end{exemple}\relationsémantique{参考}{\lien{ⓔnɯspjɤtɕha}{nɯspjɤtɕha}}\étymologie{spʲod.tɕʰa?}\end{entrée}

\begin{entrée}{spoŋ}{}{ⓔspoŋ} 
\classe{vi} \paradigme{dir}{\_}
\begin{définition}\pfra{se réfugier dans un autre pays}\end{définition}
\begin{définition}\pcmn{逃亡他乡}\end{définition}\end{entrée}

\begin{entrée}{spoŋspoz}{}{ⓔspoŋspoz} 
\classe{n} 
\begin{définition}\pfra{une plante}\end{définition}
\begin{définition}\pcmn{一种植物}\end{définition}
\begin{exemple}\pjya{spoŋspoz nɯ sɤtɕha kɯ-mbɤr tsa pɕoʁ tu-ɬoʁ ɲɯ-ŋu, ɯ-zrɤm ɲɯ-dɤn, ɲɯ-zri, ɯ-ku ɯ-jwaʁ nɯ ɲɯ-pɣi tɕe ɯ-rme kɯ-fse ɣɤʑu, ɯ-jwaʁ nɯ ɯ-qa ri ɲɤ-ɬoʁ tɕe ɯ-thoʁ pjɯ-ɤɲɟoʁ kɯ-fse ɲɯ-ŋu. ɯ-ru tu-ɬoʁ tɕe, ɯ-kɤχcɤl tɕe ɲɯ-rɯmɯntoʁ ɲɯ-ŋu. ɯ-mɯntoʁ hanɯni ɲɯ-ɣɯrni, ɯ-mɯntoʁ ɯ-rqhu kɤntɕhɯ-tɤlɤβ ɣɤʑu, ɯ-mɯntoʁ tɯ-rdoʁ ma maŋe, ɯ-rɣi ɣɤʑu ri tu-ɬoʁ mɯ́j-cha, ɯ-zrɤm ɯ-taʁ ɲɯ-mphɤl ɲɯ-ɕti. tɯ-xpa tu-ɬoʁ tɕe pjɯ-khrɯ ɲɯ-ɕti, ki sɯjno ki ɯ-di wuma ɲɯ-mɯm, smɤn ɲɯ-sna khi.}\hspace{5pt}\pcmn{\lien{ⓔspoŋspoz}{spoŋspoz} 生长在海拔低的地方(半山以下),根又多又长,苗、叶子的是灰色的,有毛。叶子长在根部上,好像贴在地面上,茎长出来以后,顶端上开花。花有点红,花萼有几层,只有一朵花,虽然有种子但是不能发芽,是靠它的根繁殖的。长了一年就会干枯。这种草香味很浓,据说可以入药。}\end{exemple}\end{entrée}

\begin{entrée}{spoŋsrɤm}{}{ⓔspoŋsrɤm} 
\classe{n} 
\begin{définition}\pfra{mammifère non identifié}\end{définition}
\begin{définition}\pcmn{一种动物}\end{définition}
\begin{exemple}\pjya{spoŋsrɤm nɯ rɯdaʁ ruŋgu kɯ-mbro ku-kɯ-rɤʑi ci ŋu tɕe ɯ-rme nɯ wuma ʑo mpɕɤr tɕe tɕhɯɕrɤm cho kɯ-naχtɕɯɣ ŋu}\hspace{5pt}\pcmn{\lien{ⓔspoŋsrɤm}{spoŋsrɤm}是生活在高山草地上的动物,它的毛长得很美,和水獭一样的贵重。}\end{exemple}\étymologie{spaŋ.sram}\end{entrée}

\begin{entrée}{sporɟɤlɯla}{}{ⓔsporɟɤlɯla} 
\classe{n} 
\begin{définition}\pfra{orvet}\end{définition}
\begin{définition}\pcmn{玻璃蛇}\end{définition}
\begin{exemple}\pjya{sporɟɤlɯla nɯ qapri kɯ-fse ci ŋu ri kɯ-xtɯt kɯ-xtshɯm ci ŋu. ɯ-βri ɯ-mdoʁ nɯ ra qapri fsɯ-fse ʑo fse, rdɤstaʁ ɯ-rchɤβ, praʁ ɯ-rchɤβ ra ku-rɤʑi ŋu. ɯ-βri nɯ mbju ʑo ku-sɤmtsɯɣ kɤ-mtshɤm me ri ɯ-kɯ-nɯɣmu wuma dɤn. kɯ-sɤmtshɤr ci tu tɕe, tɯ-jaχpa ɣɯ ɯ-βzɯr ``hu" tu-kɯ-ti tɕe ɯ-taʁ pjɯ́-wɣ-lɤt tɕe, tɯ-jaʁ pjɯ-tɯɣ ʑo ma mɤ-ra ma tɕe ɲɯ-mbrɤt ɕti. nɯ-mbrɤt tɕe, ɯ-rtshɯm nɯ pɤjkhu tu-nɤrɟɯrɟɯɣ ɕti.}\hspace{5pt}\pcmn{玻璃蛇像蛇一样,但比较短,也比较细。身子的颜色和蛇完全相同,生活石头缝和岩石缝里。身体有光泽。没有听说过咬人,但很多人怕它。奇怪的是,人们把手掌的边缘吹一下,然后打在它身上,只要轻轻地碰一下它就会断裂。断裂后的那一节还能继续跑。}\end{exemple}\end{entrée}

\begin{entrée}{spoʁ}{}{ⓔspoʁ} 
\classe{vi} \paradigme{dir}{nɯ-}
\begin{définition}\pfra{être troué}\end{définition}
\begin{définition}\pcmn{有洞}\end{définition}
\begin{exemple}\pjya{a-khɯtsa ɲɤ-spoʁ}\hspace{5pt}\pcmn{我的碗有洞}\end{exemple}\relationsémantique{参考}{\lien{ⓔkɯspoʁ}{kɯspoʁ}}\end{entrée}

\begin{entrée}{spoz}{}{ⓔspoz} 
\classe{n} 
\begin{définition}\pfra{encens}\end{définition}
\begin{définition}\pcmn{香}\end{définition}\étymologie{spos}\end{entrée}

\begin{entrée}{sprɤt}{}{ⓔsprɤt} 
\classe{vt}  
\grammaire{refl} \paradigme{dir}{tɤ-}\sens{1}
\begin{définition}\pfra{installer}\end{définition}
\begin{définition}\pcmn{安装}\end{définition}
\begin{exemple}\pjya{mkhɯrlu (jiqi) to-sprɤt}\hspace{5pt}\pcmn{他安装了机器}\end{exemple}
\begin{exemple}\pjya{ɕoŋβzu kɯ to-sprɤt}\hspace{5pt}\pcmn{木匠安装了}\end{exemple}\sens{2}
\begin{définition}\pfra{remettre entre les mains de}\end{définition}
\begin{définition}\pcmn{上缴}\end{définition}\paradigme{dir}{tɤ-}
\begin{exemple}\pjya{aʑo kɯ-mɯrkɯ pɯ-mto-t-a tɕe, ɯ-taʁ ra nɯ-jaʁ tɤ-sprat-a}\hspace{5pt}\pcmn{我看到小偷,移交给领导了}\end{exemple}
\begin{sous-entrée}{ʑɣɤsprɤt}{ⓔsprɤtⓝʑɣɤsprɤt} 
\classe{vi} \end{sous-entrée}

\begin{définition}\pfra{se livrer (à la justice)}\end{définition}
\begin{définition}\pcmn{把自己交给上级,自首}\end{définition}
\begin{exemple}\pjya{aʑo tɤ-ʑɣɤsprat-a}\hspace{5pt}\pcmn{我自首了}\end{exemple}\étymologie{sprod}\end{entrée}

\begin{entrée}{sprilu}{}{ⓔsprilu} 
\classe{n} 
\begin{définition}\pfra{année du singe}\end{définition}
\begin{définition}\pcmn{猴年}\end{définition}\étymologie{spreɦu.lo}\end{entrée}

\begin{entrée}{sprɯlpa}{}{ⓔsprɯlpa} 
\classe{n} 
\begin{définition}\pfra{magie}\end{définition}
\begin{définition}\pcmn{法术}\end{définition}\étymologie{sprul.pa}\end{entrée}

\begin{entrée}{sprɯskɯ}{}{ⓔsprɯskɯ} 
\classe{n} 
\begin{définition}\pfra{sprulsku}\end{définition}
\begin{définition}\pcmn{活佛}\end{définition}\étymologie{sprul.sku}\end{entrée}

\begin{entrée}{spɯ}{}{ⓔspɯ} 
\classe{vs} \paradigme{dir}{pɯ-}\paradigme{dir}{pɯ-}
\begin{définition}\pfra{sec}\end{définition}
\begin{définition}\pcmn{水分少;干(面、糌粑、稀泥)}\end{définition}
\begin{définition}\pfra{rendre sec}\end{définition}
\begin{définition}\pcmn{弄干}\end{définition}
\begin{exemple}\pjya{rɟɤɣi pjɤ-spɯ}\hspace{5pt}\pcmn{糌粑水分偏少,吃起来不要干燥}\end{exemple}
\begin{exemple}\pjya{a-rɟɤɣi ɯ-ŋgɯ tɯ-ci pjɤ-ɣɤrkɯn-a tɕe, pjɤ-ɣɤspɯ-t-a}\hspace{5pt}\pcmn{我减少了糌粑里的水分,令它很干}\end{exemple}\relationsémantique{反义词}{\lien{ⓔŋgri}{ŋgri}}
\begin{sous-entrée}{ɣɤspɯ}{ⓔspɯⓝɣɤspɯ} 
\classe{vt}  
\grammaire{caus} \end{sous-entrée}

\end{entrée}

\begin{entrée}{spɯrtɯm}{}{ⓔspɯrtɯm} 
\classe{n} 
\begin{définition}\pfra{une espèce de champignon}\end{définition}
\begin{définition}\pcmn{一种菌}\end{définition}
\begin{exemple}\pjya{spɯrtɯm nɯ tɯrgi ɯ-ŋgɯ tu-ɬoʁ ŋu, ɯ-mdoʁ nɯ kɯ-qandʐi ŋu, ɯ-tshɯɣa nɯ tɯrgi grɯβgrɯβ cho naχtɕɯɣ, kɯ-rko tsa ŋu ɯ-ru nɯ mɤ-ndoʁ, kɤ-ndza sna}\hspace{5pt}\pcmn{\lien{ⓔspɯrtɯm}{spɯrtɯm} 长在杉木林里,颜色是浅黑色,形状和杉木菌一样,比较硬,干而不脆,可以吃。}\end{exemple}\end{entrée}

\begin{entrée}{sqa}{}{ⓔsqa} 
\classe{vt} \paradigme{dir}{kɤ-}\paradigme{dir}{pɯ-}
\begin{définition}\pfra{cuire}\end{définition}
\begin{définition}\pcmn{炖;煮}\end{définition}
\begin{exemple}\pjya{tɤ-mthɯm ka-sqa}\hspace{5pt}\pcmn{他煮了肉}\end{exemple}
\begin{exemple}\pjya{paʁndza ka-sqa}\hspace{5pt}\pcmn{他煮了猪食}\end{exemple}
\begin{exemple}\pjya{kɤ-sqa-j}\hspace{5pt}\pcmn{我们煮了}\end{exemple}
\begin{exemple}\pjya{tɯ-mɯ kɯ kɤ́-wɣ-sqa ʑo}\hspace{5pt}\pcmn{被雨淋湿了}\end{exemple}\end{entrée}

\begin{entrée}{sqaβde}{}{ⓔsqaβde} 
\classe{num} 
\begin{définition}\pfra{quatorze}\end{définition}
\begin{définition}\pcmn{十四}\end{définition}\end{entrée}

\begin{entrée}{sqaβjɯβ}{}{ⓔsqaβjɯβ} 
\classe{vt} \paradigme{dir}{tɤ-}
\begin{définition}\pfra{cacher, couvrir}\end{définition}
\begin{définition}\pcmn{遮掩,遮住}\end{définition}
\begin{exemple}\pjya{a-mɲaʁ tɤ-sqaβjɯβ-a ma ɲɯ-nɤmbju}\hspace{5pt}\pcmn{因为很亮,我遮住了眼睛}\end{exemple}\relationsémantique{同义词}{\lien{ⓔsaʁjɯβ}{saʁjɯβ}}\end{entrée}

\begin{entrée}{sqaɕnɯz}{}{ⓔsqaɕnɯz} 
\classe{num} 
\begin{définition}\pfra{dix-sept}\end{définition}
\begin{définition}\pcmn{十七}\end{définition}\end{entrée}

\begin{entrée}{sqafsum}{}{ⓔsqafsum} 
\classe{num} 
\begin{définition}\pfra{treize}\end{définition}
\begin{définition}\pcmn{十三}\end{définition}\end{entrée}

\begin{entrée}{sqamnɯz}{}{ⓔsqamnɯz} 
\classe{num} 
\begin{définition}\pfra{douze}\end{définition}
\begin{définition}\pcmn{十二}\end{définition}\end{entrée}

\begin{entrée}{sqamŋu}{}{ⓔsqamŋu} 
\classe{num} 
\begin{définition}\pfra{quinze}\end{définition}
\begin{définition}\pcmn{十五}\end{définition}\end{entrée}

\begin{entrée}{sqandʐi}{}{ⓔsqandʐi}\relationsémantique{参考}{\lien{ⓔqandʐiⓗ1}{qandʐi₁}}\end{entrée}

\begin{entrée}{sqane}{}{ⓔsqane} 
\classe{vt} \paradigme{dir}{kɤ-}\paradigme{dir}{tɤ-}
\begin{définition}\pfra{couvrir, plonger dans l'obscurité, éteindre (lumière)}\end{définition}
\begin{définition}\pcmn{遮光;关灯}\end{définition}\relationsémantique{参考}{\lien{ⓔsqaʁjɯβ}{sqaʁjɯβ}}\relationsémantique{参考}{\lien{ⓔsqanɯ}{sqanɯ}}\end{entrée}

\begin{entrée}{sqangɯt}{}{ⓔsqangɯt} 
\classe{num} 
\begin{définition}\pfra{dix-neuf}\end{définition}
\begin{définition}\pcmn{十九}\end{définition}\end{entrée}

\begin{entrée}{sqanɯ}{}{ⓔsqanɯ} 
\classe{vt}  
\grammaire{caus} \paradigme{dir}{kɤ-}
\begin{définition}\pfra{plonger dans l'obscurité}\end{définition}
\begin{définition}\pcmn{遮光}\end{définition}
\begin{exemple}\pjya{qale ta-βzu tɕe, rdɯl kɯ ka-sqanɯ ʑo}\hspace{5pt}\pcmn{风吹得尘土满天飞扬,遮住了阳光}\end{exemple}
\begin{exemple}\pjya{tɤ-khɯ kɯ ka-sqanɯ ʑo}\hspace{5pt}\pcmn{浓烟遮住了阳光}\end{exemple}\relationsémantique{参考}{\lien{ⓔqanɯ}{qanɯ}}\end{entrée}

\begin{entrée}{sqaprɤɣ}{}{ⓔsqaprɤɣ} 
\classe{num} 
\begin{définition}\pfra{seize}\end{définition}
\begin{définition}\pcmn{十六}\end{définition}\end{entrée}

\begin{entrée}{sqaptɯɣ}{}{ⓔsqaptɯɣ} 
\classe{num} 
\begin{définition}\pfra{onze}\end{définition}
\begin{définition}\pcmn{十一}\end{définition}\end{entrée}

\begin{entrée}{sqapɯ}{}{ⓔsqapɯ}\relationsémantique{参考}{\lien{ⓔqapɯ}{qapɯ}}\end{entrée}

\begin{entrée}{sqarcat}{}{ⓔsqarcat} 
\classe{num} 
\begin{définition}\pfra{dix-huit}\end{définition}
\begin{définition}\pcmn{十八}\end{définition}\end{entrée}

\begin{entrée}{sqarcɯm}{}{ⓔsqarcɯm} 
\classe{vt} \paradigme{dir}{pɯ-}\paradigme{dir}{thɯ-}
\begin{définition}\pfra{froncer les sourcils}\end{définition}
\begin{définition}\pcmn{皱着眉毛,表现出不高兴的表情}\end{définition}
\begin{exemple}\pjya{ɯʑo kɯ ɯ-rŋa pjɤ-sqarcɯm}\hspace{5pt}\pcmn{他皱了眉头}\end{exemple}\relationsémantique{参考}{\lien{ⓔqarcɯm}{qarcɯm}}\end{entrée}

\begin{entrée}{sqarndɯm}{}{ⓔsqarndɯm}\relationsémantique{参考}{\lien{ⓔqarndɯm}{qarndɯm}}\end{entrée}

\begin{entrée}{sqaʁjɯβ}{}{ⓔsqaʁjɯβ} 
\classe{vt} \paradigme{dir}{tɤ-}
\begin{définition}\pfra{couvrir (un côté)}\end{définition}
\begin{définition}\pcmn{遮住(一面);挡住}\end{définition}
\begin{exemple}\pjya{tɤ́-wɣ-sqaʁjɯβ-a}\hspace{5pt}\pcmn{把我挡住了}\end{exemple}
\begin{exemple}\pjya{si kɯ tu-sqaʁjɯβ ɲɯ-ŋu}\hspace{5pt}\pcmn{被树遮住}\end{exemple}
\begin{exemple}\pjya{zgo kɯ tu-sqaʁjɯβ ɲɯ-ŋu}\hspace{5pt}\pcmn{被山遮住}\end{exemple}
\begin{exemple}\pjya{tɤ-kɯ-sqaʁjɯβ-a}\hspace{5pt}\pcmn{你把我挡住了}\end{exemple}
\begin{exemple}\pjya{tɤ-ta-sqaβjɯɣ}\hspace{5pt}\pcmn{我把你挡住了}\end{exemple}
\begin{exemple}\pjya{tɤŋe zdɯm kɯ to-sqaβjɯβ ɲɯ-ŋu, tɕe mɯ-ɲɤ-sɤmto}\hspace{5pt}\pcmn{云遮住了太阳,所以就看不见}\end{exemple}\relationsémantique{同义词}{\lien{ⓔsqaβjɯβ}{sqaβjɯβ}}\end{entrée}

\begin{entrée}{sqɤr}{}{ⓔsqɤr} 
\classe{vt} \paradigme{dir}{nɯ-}\paradigme{dir}{\_}\paradigme{dir}{tɤ-}
\begin{définition}\pfra{demander à quelqu'un de faire un travail}\end{définition}
\begin{définition}\pcmn{请人做事;求助}\end{définition}
\begin{définition}\pfra{demander à des gens de faire un travail}\end{définition}
\begin{définition}\pcmn{请人帮忙}\end{définition}
\begin{exemple}\pjya{nɯ-sqar-a}\hspace{5pt}\pcmn{我请了他}\end{exemple}
\begin{exemple}\pjya{nɯ́-wɣ-sqar-a}\hspace{5pt}\pcmn{他请了我}\end{exemple}
\begin{exemple}\pjya{a-kɯ-sqɤr me}\hspace{5pt}\pcmn{没有人请我}\end{exemple}
\begin{exemple}\pjya{tɯrɣi ɯ-kɯ\_-lɤt nɯ́-wɣ-sqar-a}\hspace{5pt}\pcmn{他请我撒种子}\end{exemple}
\begin{exemple}\pjya{ɕ-tɤ-sqar-a, ɕ-pɯ-sqar-a}\hspace{5pt}\pcmn{我去请他了(往上、往下)}\end{exemple}\relationsémantique{同义词}{\lien{ⓔftɕɤl}{ftɕɤl}}
\begin{sous-entrée}{asqɯsqɤr}{ⓔsqɤrⓝasqɯsqɤr} 
\classe{vi}  
\grammaire{recip} 
\begin{définition}\pfra{s'aider les uns les autres pour faire des travaux}\end{définition}
\begin{définition}\pcmn{互相帮忙做事}\end{définition}
\begin{exemple}\pjya{asqɯsqɤr-i}\hspace{5pt}\pcmn{我们互相帮忙(互相请)}\end{exemple}\end{sous-entrée}

\begin{sous-entrée}{sɤsqɤr}{ⓔsqɤrⓝsɤsqɤr} 
\classe{vi} \end{sous-entrée}

\end{entrée}

\begin{entrée}{sqhɤtɤjɯm}{}{ⓔsqhɤtɤjɯm} 
\classe{n} 
\begin{définition}\pfra{instruments de cuisine}\end{définition}
\begin{définition}\pcmn{三脚架和锅子;炊具}\end{définition}\relationsémantique{参考}{\lien{ⓔsqhi}{sqhi}}\end{entrée}

\begin{entrée}{sqhɤthɤlɤɣi}{}{ⓔsqhɤthɤlɤɣi} 
\classe{n} 
\begin{définition}\pfra{cendre}\end{définition}
\begin{définition}\pcmn{草木灰}\end{définition}\relationsémantique{同义词}{\lien{ⓔthɤfkɤlɤɣi}{thɤfkɤlɤɣi}}\relationsémantique{参考}{\lien{ⓔsqhi}{sqhi}}\end{entrée}

\begin{entrée}{sqhi}{}{ⓔsqhi} 
\classe{n} 
\begin{définition}\pfra{trépied}\end{définition}
\begin{définition}\pcmn{三脚架}\end{définition}\relationsémantique{参考}{\lien{ⓔsqhɤtɤjɯm}{sqhɤtɤjɯm}}\end{entrée}

\begin{entrée}{sqhiar}{}{ⓔsqhiar} 
\classe{vt} \paradigme{dir}{\_}
\begin{définition}\pfra{ouvrir, étendre}\end{définition}
\begin{définition}\pcmn{展开(布料、鸟的翅膀)}\end{définition}
\begin{exemple}\pjya{pɣa kɯ ɯ-ʁar nɯ na-sqhiar}\hspace{5pt}\pcmn{鸟展开了翅膀}\end{exemple}
\begin{exemple}\pjya{tɯ-ŋga tɤ-tɯ-ɕkho-t tɕe nɯ-sqhiar ra ma mɤ-zbaʁ}\hspace{5pt}\pcmn{你晒衣服时候,一定要把它展开不然就不会干}\end{exemple}\end{entrée}

\begin{entrée}{sqi}{}{ⓔsqi} 
\classe{num} 
\begin{définition}\pfra{dix}\end{définition}
\begin{définition}\pcmn{十}\end{définition}
\begin{exemple}\pjya{ɯ-sqɯ-xpa}\hspace{5pt}\pcmn{好几十年}\end{exemple}\end{entrée}

\begin{entrée}{sqlɯm}{}{ⓔsqlɯm} 
\classe{vi} \paradigme{dir}{pɯ-}\paradigme{dir}{kɤ-}
\begin{définition}\pfra{s'affaisser, avoir une crevasse}\end{définition}
\begin{définition}\pcmn{陷下去;塌下去}\end{définition}
\begin{exemple}\pjya{khɤxtu pɯ-sqlɯm}\hspace{5pt}\pcmn{房背陷下去了}\end{exemple}
\begin{exemple}\pjya{rnda pjɤ-sqlɯm}\hspace{5pt}\pcmn{楼层陷下去了}\end{exemple}
\begin{exemple}\pjya{sɤtɕha (tɯ-khɤl) pjɤ-sqlɯm}\hspace{5pt}\pcmn{(有一个)地方陷下去了}\end{exemple}\relationsémantique{同义词}{\lien{ⓔmbɯt}{mbɯt}}\relationsémantique{参考}{\lien{ⓔarɴɢlɯm}{arɴɢlɯm}}\end{entrée}

\begin{entrée}{srɤz}{}{ⓔsrɤz} 
\classe{n} 
\begin{définition}\pfra{prince}\end{définition}
\begin{définition}\pcmn{王子}\end{définition}\étymologie{sras}\end{entrée}

\begin{entrée}{sroχtɕɤn}{}{ⓔsroχtɕɤn} 
\classe{n} 
\begin{définition}\pfra{tuer des être vivants}\end{définition}
\begin{définition}\pcmn{杀生}\end{définition}
\begin{exemple}\pjya{sroχtɕɤn ma-pɯ-tɯ-lɤt}\hspace{5pt}\pcmn{你不要杀生}\end{exemple}\end{entrée}

\begin{entrée}{srɯβzɤn}{}{ⓔsrɯβzɤn} 
\classe{n} 
\begin{définition}\pfra{pièce de tissu placée entre deux autres morceaux de tissus au niveau de la couture}\end{définition}
\begin{définition}\pcmn{在两块布料的缝合处另外夹上一块布料}\end{définition}\end{entrée}

\begin{entrée}{srɯn}{₁}{ⓔsrɯnⓗ1} 
\classe{n} 
\begin{définition}\pfra{inflammation des sinus}\end{définition}
\begin{définition}\pcmn{鼻窦炎}\end{définition}\relationsémantique{参考}{\lien{ⓔsrɯsmɤn}{srɯsmɤn}}\end{entrée}

\begin{entrée}{srɯn}{₂}{ⓔsrɯnⓗ2} 
\classe{n} 
\begin{définition}\pfra{coton}\end{définition}
\begin{définition}\pcmn{棉花}\end{définition}\étymologie{srin}\end{entrée}

\begin{entrée}{srɯn}{₃}{ⓔsrɯnⓗ3} 
\classe{n} 
\begin{définition}\pfra{pellicules}\end{définition}
\begin{définition}\pcmn{头屑}\end{définition}
\begin{exemple}\pjya{ɯ-ku srɯn ɲɯ-dɤn}\hspace{5pt}\pcmn{他们头上很多头屑}\end{exemple}\end{entrée}

\begin{entrée}{srɯnbu/\variante{srɯtphu}}{}{ⓔsrɯnbu} 
\classe{n} 
\begin{définition}\pfra{râkshasa}\end{définition}
\begin{définition}\pcmn{妖精}\end{définition}\relationsémantique{参考}{\lien{ⓔsrɯnmɯ}{srɯnmɯ}}\end{entrée}

\begin{entrée}{srɯndɤr}{}{ⓔsrɯndɤr} 
\classe{n} 
\begin{définition}\pfra{acné}\end{définition}
\begin{définition}\pcmn{青春痘}\end{définition}
\begin{exemple}\pjya{nɤ-srɯndɤr ɲɤ-ɬoʁ}\hspace{5pt}\pcmn{你长了青春痘}\end{exemple}\relationsémantique{参考}{\lien{ⓔnɯsrɯɣndɤr}{nɯsrɯɣndɤr}}\end{entrée}

\begin{entrée}{srɯnloʁ}{}{ⓔsrɯnloʁ} 
\classe{n} \sens{1}
\begin{définition}\pfra{diadème en argent}\end{définition}
\begin{définition}\pcmn{头上的银制装饰品}\end{définition}\sens{2}
\begin{définition}\pfra{anneau}\end{définition}
\begin{définition}\pcmn{戒指}\end{définition}
\begin{exemple}\pjya{srɯnloʁ lɤ-nɯrʁe-t-a}\hspace{5pt}\pcmn{我戴了戒指}\end{exemple}
\begin{exemple}\pjya{srɯnloʁ-pɯ}\hspace{5pt}\pcmn{小戒指}\end{exemple}\end{entrée}

\begin{entrée}{srɯnmɯ}{}{ⓔsrɯnmɯ} 
\classe{n} 
\begin{définition}\pfra{râkshasî}\end{définition}
\begin{définition}\pcmn{妖精}\end{définition}\étymologie{srin.mo}\end{entrée}

\begin{entrée}{srɯsmɤn}{}{ⓔsrɯsmɤn} 
\classe{n} 
\begin{définition}\pfra{médicament contre l'inflammation des sinus}\end{définition}
\begin{définition}\pcmn{鼻窦炎的药(鼻烟)}\end{définition}\relationsémantique{参考}{\lien{ⓔsmɤn}{smɤn}}\end{entrée}

\begin{entrée}{srɯtphɯ}{}{ⓔsrɯtphɯ} 
\classe{n} 
\begin{définition}\pfra{râkshasa}\end{définition}
\begin{définition}\pcmn{男妖}\end{définition}\relationsémantique{参考}{\lien{ⓔsrɯnbu}{srɯnbu}}\end{entrée}

\begin{entrée}{stu}{₁}{ⓔstuⓗ1} 
\classe{vs}
\classe{vs}
\classe{vs} \paradigme{dir}{tɤ-}
\begin{définition}\pfra{assidu, travailleur}\end{définition}
\begin{définition}\pcmn{努力}\end{définition}
\begin{exemple}\pjya{kɤ-rɤma to-stu}\hspace{5pt}\pcmn{他工作很努力}\end{exemple}\sens{2}\paradigme{dir}{tɤ-}
\begin{définition}\pfra{faire attention à}\end{définition}
\begin{définition}\pcmn{注意安全}\end{définition}
\begin{définition}\pfra{assidu, travailleur}\end{définition}
\begin{définition}\pcmn{努力}\end{définition}
\begin{exemple}\pjya{smi tu-kɯ-stu ɲɯ-ra}\hspace{5pt}\pcmn{一定要注意安全用火}\end{exemple}
\begin{exemple}\pjya{stu-a mbat-a}\hspace{5pt}\pcmn{我会努力的}\end{exemple}
\begin{exemple}\pjya{tɤ-stu tɤ-mbat je}\hspace{5pt}\pcmn{你努力吧}\end{exemple}\relationsémantique{Component 1}{\lien{}{stu}}\relationsémantique{Component 2}{\lien{ⓔmbat}{mbat}}
\begin{sous-entrée}{stu,mbat}{ⓔstuⓗ1ⓢ2ⓝstu,mbat}\end{sous-entrée}

\begin{sous-entrée}{nɤstumbat}{ⓔstuⓗ1ⓝnɤstumbat} 
\classe{vt} 
\begin{définition}\pfra{trouver assidu}\end{définition}
\begin{définition}\pcmn{觉得努力}\end{définition}
\begin{exemple}\pjya{ɯʑo ɲɯ-nɤstumbat-a}\hspace{5pt}\pcmn{我觉得他很努力}\end{exemple}\end{sous-entrée}

\end{entrée}

\begin{entrée}{stu}{₂}{ⓔstuⓗ2} 
\classe{vi-t}  
\grammaire{caus} \paradigme{dir}{nɯ-}\paradigme{}{nɯ-}
\begin{définition}\pfra{croire (une parole)}\end{définition}
\begin{définition}\pcmn{相信}\end{définition}
\begin{exemple}\pjya{ɯ-tɯfɕɤt pɯ-βzu-t-a ri, ɯʑo ɲɯ-stu}\hspace{5pt}\pcmn{我告诉他了,他相信}\end{exemple}
\begin{exemple}\pjya{nɯ tɯ-tɕha nɯ jɤ-azɣɯt ri, ɯʑo ɲɯ-stu}\hspace{5pt}\pcmn{来了这个消息,他相信}\end{exemple}\relationsémantique{参考}{\lien{ⓔɯ-stuⓗ2}{ɯ-stu₂}}\relationsémantique{参考}{\lien{ⓔnɤstu}{nɤstu}}\relationsémantique{参考}{\lien{ⓔsɤstuⓗ1}{sɤstu₁}}\relationsémantique{参考}{\lien{ⓔnɤstuⓝsɤnɤstu}{sɤnɤstu}}
\begin{sous-entrée}{sɯstu}{ⓔstuⓗ2ⓝsɯstu} 
\classe{vt} \end{sous-entrée}

\begin{définition}\pfra{faire croire}\end{définition}
\begin{définition}\pcmn{令……相信}\end{définition}\end{entrée}

\begin{entrée}{stu}{₃}{ⓔstuⓗ3} 
\classe{vt} \paradigme{dir}{tɤ-}\paradigme{dir}{tɤ-}
\begin{définition}\pfra{faire d'une certaine manière}\end{définition}
\begin{définition}\pcmn{那样做}\end{définition}
\begin{définition}\pfra{considérer comme}\end{définition}
\begin{définition}\pcmn{当成}\end{définition}
\begin{exemple}\pjya{nɤʑo rɟɤɣi ɯ-ŋgɯ kɯ-chi pjɯ-tɯ-nɯ-lɤt ɲɯ-ŋu tɕe, aʑo kɯnɤ nɯ tɤ-stu-t-a}\hspace{5pt}\pcmn{你吃糌粑加糖,我也这样吃}\end{exemple}
\begin{exemple}\pjya{kɤ-stu ɲɯ-me}\hspace{5pt}\pcmn{没有办法}\end{exemple}
\begin{exemple}\pjya{aʑo rŋɯl ste-a me}\hspace{5pt}\pcmn{钱对我没有用}\end{exemple}
\begin{exemple}\pjya{kɯmaʁ to-stu-t-a tɕe ɲɯ-βzɟɯr-a ɲɯ-ntshi}\hspace{5pt}\pcmn{我弄错了,要纠正过来}\end{exemple}
\begin{exemple}\pjya{ɯ-rɟit ʑo tú-wɣ-nɯstu-a ŋu}\hspace{5pt}\pcmn{他把我当成自己的孩子一样}\end{exemple}\relationsémantique{参考}{\lien{ⓔrtsiⓝsɤrtsi}{sɤrtsi}}\relationsémantique{参考}{\lien{ⓔsɯpaⓗ2}{sɯpa}}
\begin{sous-entrée}{nɯstu}{ⓔstuⓗ3ⓝnɯstu} 
\classe{vt} \end{sous-entrée}

\end{entrée}

\begin{entrée}{sta}{}{ⓔsta} 
\classe{vi} \paradigme{dir}{thɯ-}\paradigme{dir}{thɯ-}
\begin{définition}\pfra{se réveiller}\end{définition}
\begin{définition}\pcmn{醒}\end{définition}
\begin{définition}\pfra{réveiller}\end{définition}
\begin{définition}\pcmn{弄醒}\end{définition}
\begin{exemple}\pjya{tɤ-pɤtso cho-sta}\hspace{5pt}\pcmn{小孩子醒了}\end{exemple}
\begin{exemple}\pjya{aj ʑa thɯ-sta-a}\hspace{5pt}\pcmn{我早就醒了}\end{exemple}
\begin{exemple}\pjya{a-tɯ-sta ɲɯ-maqhu}\hspace{5pt}\pcmn{我醒得很晚}\end{exemple}
\begin{exemple}\pjya{kɯm ɯ-zgra nɯ kɯ chó-wɣ-sɯsta}\hspace{5pt}\pcmn{门的声音把他弄醒了}\end{exemple}
\begin{sous-entrée}{sɯsta}{ⓔstaⓝsɯsta} 
\classe{vt}  
\grammaire{caus} \end{sous-entrée}

\end{entrée}

\begin{entrée}{staʁ}{}{ⓔstaʁ} 
\classe{postp} 
\begin{définition}\pfra{par rapport à}\end{définition}
\begin{définition}\pcmn{比}\end{définition}\relationsémantique{同义词}{\lien{ⓔsɤz}{sɤz}}\relationsémantique{同义词}{\lien{ⓔstaʁnɤ}{staʁnɤ}}\end{entrée}

\begin{entrée}{staʁɕɤr}{}{ⓔstaʁɕɤr} 
\classe{n} 
\begin{définition}\pfra{petit cochon dont la peau est bariolée de rose, de noir et de blanc}\end{définition}
\begin{définition}\pcmn{毛色黑白红相间的小猪}\end{définition}\end{entrée}

\begin{entrée}{staʁlu}{}{ⓔstaʁlu} 
\classe{n} 
\begin{définition}\pfra{année du tigre}\end{définition}
\begin{définition}\pcmn{虎年}\end{définition}\étymologie{stag.lo}\end{entrée}

\begin{entrée}{staʁnɤ}{}{ⓔstaʁnɤ} 
\classe{postp} 
\begin{définition}\pfra{par rapport à}\end{définition}
\begin{définition}\pcmn{比}\end{définition}
\begin{exemple}\pjya{nɯ staʁnɤ}\hspace{5pt}\pcmn{还不如}\end{exemple}\relationsémantique{同义词}{\lien{ⓔstaʁ}{staʁ}}\relationsémantique{参考}{\lien{ⓔsɤz}{sɤz}}\end{entrée}

\begin{entrée}{staʁrɟɤnma}{}{ⓔstaʁrɟɤnma} 
\classe{n} 
\begin{définition}\pfra{bol}\end{définition}
\begin{définition}\pcmn{瓷碗,画着佛像的图案}\end{définition}\étymologie{stag rgʲan.ma}\end{entrée}

\begin{entrée}{stat}{}{ⓔstat} 
\classe{vi} \paradigme{dir}{tɤ-}\paradigme{}{tɤ-}
\begin{définition}\pfra{s'arrêter}\end{définition}
\begin{définition}\pcmn{停止}\end{définition}
\begin{définition}\pfra{arrêter}\end{définition}
\begin{définition}\pcmn{使……停止}\end{définition}
\begin{exemple}\pjya{tɯ-mɯ to-stat (= tɯ-mɯ kɤ-lɤt to-znɯna)}\hspace{5pt}\pcmn{雨停了}\end{exemple}
\begin{exemple}\pjya{ɯ-ɕnɤse to-stat (= ɯ-ɕnɤse kɯ-ɬoʁ to-nɯna)}\hspace{5pt}\pcmn{他的鼻血止住了}\end{exemple}
\begin{exemple}\pjya{ɯ-ɕnɤse to-sɯstat pjɤ-cha}\hspace{5pt}\pcmn{他成功地把鼻血止住了}\end{exemple}
\begin{sous-entrée}{sɯstat}{ⓔstatⓝsɯstat} 
\classe{vt} \end{sous-entrée}

\end{entrée}

\begin{entrée}{staχpɯ}{}{ⓔstaχpɯ} 
\classe{n} 
\begin{définition}\pfra{haricot}\end{définition}
\begin{définition}\pcmn{豌豆}\end{définition}\end{entrée}

\begin{entrée}{staχpɯldzɣɤm}{}{ⓔstaχpɯldzɣɤm} 
\classe{n} 
\begin{définition}\pfra{paille de haricots}\end{définition}
\begin{définition}\pcmn{豌豆秸}\end{définition}\end{entrée}

\begin{entrée}{staχpɯqajɯ}{}{ⓔstaχpɯqajɯ} 
\classe{n} 
\begin{définition}\pfra{espèce de chenille}\end{définition}
\begin{définition}\pcmn{毛虫的一种}\end{définition}\end{entrée}

\begin{entrée}{staχpɯrɟɤskhi}{}{ⓔstaχpɯrɟɤskhi} 
\classe{n} 
\begin{définition}\pfra{une plante}\end{définition}
\begin{définition}\pcmn{植物的一种}\end{définition}
\begin{exemple}\pjya{staχpɯrɟɤskhi nɯ sɯjno kɯ-ʁjɤr ɯ-ŋgɯ tu-ɬoʁ ŋu, ɯ-jwaʁ kɯ-ɤrtɯ-rtɯm kɯ-xtɕɯ-xtɕi ŋu, ɯ-rme kɯ-fse tu, jaʁ tsa. ɯ-ru kɯ-xtɯ-xtɯt ma me, ɯ-mɯntoʁ staχpɯ mɯntoʁ fse tɕe kɯ-ɣɯrni ɲɯ-lɤt ŋu. ɯ-mat nɯ ɯ-cɤβ chɯ-βze ŋu. fsapaʁ ra kɯ tu-ndza-nɯ sna, tɯrme kɤ-ndza mɤ-sna.}\hspace{5pt}\pcmn{\lien{}{staχpɯ rɟɤskhi} 生长茂盛的草丛里,叶子圆圆的、小小的,上面有毛。叶子有点厚。只有短短的茎,花像豌豆的一样,是红色的,结的是荚果。牲畜可以吃,人不能吃。}\end{exemple}\end{entrée}

\begin{entrée}{stɤβtshɤt}{}{ⓔstɤβtshɤt} 
\classe{n} 
\begin{définition}\pfra{épreuves de force}\end{définition}
\begin{définition}\pcmn{比力气}\end{définition}\relationsémantique{参考}{\lien{ⓔnɯstɤβtshɤt}{nɯstɤβtshɤt}}\étymologie{stobs.tsʰad}\end{entrée}

\begin{entrée}{stɤɣdo}{}{ⓔstɤɣdo} 
\classe{n} 
\begin{définition}\pfra{enfant unique}\end{définition}
\begin{définition}\pcmn{独生子}\end{définition}\end{entrée}

\begin{entrée}{stɤjstɤj}{}{ⓔstɤjstɤj} 
\classe{idph.2} 
\begin{définition}\pfra{petit et trapu}\end{définition}
\begin{définition}\pcmn{形容又矮又圆的样子}\end{définition}
\begin{sous-entrée}{stɤjnɤlɤj}{ⓔstɤjstɤjⓝstɤjnɤlɤj}
\begin{exemple}\pjya{staχpɯ stɤjnɤlɤj ʑo ɲɯ-nɤmdɯmdar}\hspace{5pt}\pcmn{豌豆在弹来弹去}\end{exemple}\end{sous-entrée}

\begin{sous-entrée}{ɣɤstɤjlɤj}{ⓔstɤjstɤjⓝɣɤstɤjlɤj} 
\classe{vi} 
\begin{définition}\pfra{qui rebondit, qui saute}\end{définition}
\begin{définition}\pcmn{弹来弹去、跳来跳去(圆的、小的东西)}\end{définition}
\begin{exemple}\pjya{@piqiu ɲɯ-ɣɤstɤjlɤj ntsɯ}\hspace{5pt}\pcmn{皮球弹来弹去}\end{exemple}\relationsémantique{同义词}{\lien{ⓔstɯrstɯr}{stɯrstɯr}}\relationsémantique{同义词}{\lien{ⓔzdɯzdɯr}{zdɯzdɯr}}\relationsémantique{同义词}{\lien{ⓔɣɤstaŋlaŋ}{ɣɤstaŋlaŋ}}\end{sous-entrée}

\end{entrée}

\begin{entrée}{stɤm}{}{ⓔstɤm} 
\classe{vi} \paradigme{dir}{kɤ-}\paradigme{dir}{kɤ-}
\begin{définition}\pfra{se solidifier}\end{définition}
\begin{définition}\pcmn{凝固}\end{définition}
\begin{définition}\pfra{laisser se solidifier}\end{définition}
\begin{définition}\pcmn{使凝固}\end{définition}
\begin{exemple}\pjya{tɯkri kɤ-stɤm}\hspace{5pt}\pcmn{油凝固了}\end{exemple}
\begin{exemple}\pjya{sɤtɕha ko-stɤm}\hspace{5pt}\pcmn{地凝固了}\end{exemple}
\begin{exemple}\pjya{ta-mar thɯ-ftʂi-t-a tɕe kɤ-sɯstam-a}\hspace{5pt}\pcmn{我把酥油融化成液体,然后又让它凝固了}\end{exemple}
\begin{sous-entrée}{sɯstɤm}{ⓔstɤmⓝsɯstɤm} 
\classe{vt} \end{sous-entrée}

\end{entrée}

\begin{entrée}{stɤmku}{}{ⓔstɤmku} 
\classe{n} 
\begin{définition}\pfra{plaine}\end{définition}
\begin{définition}\pcmn{草坪}\end{définition}\end{entrée}

\begin{entrée}{stɤnga}{}{ⓔstɤnga} 
\classe{n} 
\begin{définition}\pfra{manteau}\end{définition}
\begin{définition}\pcmn{上衣}\end{définition}\relationsémantique{参考}{\lien{ⓔtɯ-stɤt}{tɯ-stɤt}}\relationsémantique{参考}{\lien{ⓔtɯ-ŋga}{tɯ-ŋga}}\end{entrée}

\begin{entrée}{stɤɴɢaʁ}{}{ⓔstɤɴɢaʁ} 
\classe{n} 
\begin{définition}\pfra{chemise de moine}\end{définition}
\begin{définition}\pcmn{和尚穿的上内衣}\end{définition}\étymologie{stod}\end{entrée}

\begin{entrée}{stɤrɟɯɣ}{}{ⓔstɤrɟɯɣ} 
\classe{idph} \paradigme{emphatic}{stɤrɟɯɣ jɤrɟɯɣ}
\begin{définition}\pfra{en courant}\end{définition}
\begin{définition}\pcmn{跑着}\end{définition}\relationsémantique{参考}{\lien{ⓔrɟɯɣⓗ1}{rɟɯɣ₁}}\relationsémantique{参考}{\lien{ⓔnɯstɤrɟɯɣ}{nɯstɤrɟɯɣ}}\end{entrée}

\begin{entrée}{stɤsmɤt}{}{ⓔstɤsmɤt} 
\classe{n} 
\begin{définition}\pfra{tête et queue}\end{définition}
\begin{définition}\pcmn{头尾}\end{définition}\étymologie{stod.smad}\end{entrée}

\begin{entrée}{stɤsqa}{}{ⓔstɤsqa} 
\classe{n} 
\begin{définition}\pfra{fève cuite}\end{définition}
\begin{définition}\pcmn{煮熟了的胡豆}\end{définition}\relationsémantique{参考}{\lien{ⓔstoʁ}{stoʁ}}\relationsémantique{参考}{\lien{ⓔsqa}{sqa}}\end{entrée}

\begin{entrée}{stɤt}{}{ⓔstɤt} 
\classe{vt} \paradigme{dir}{pɯ-}
\begin{définition}\pfra{attacher (bovidés)}\end{définition}
\begin{définition}\pcmn{拴在草茂盛的地方(犏牛、牛)}\end{définition}
\begin{exemple}\pjya{fsapaʁ pjɯ́-wɣ-stɤt}\hspace{5pt}\pcmn{拴牲畜}\end{exemple}
\begin{exemple}\pjya{jla pjɯ́-wɣ-stɤt}\hspace{5pt}\pcmn{拴牛}\end{exemple}\end{entrée}

\begin{entrée}{stɤtoŋ}{}{ⓔstɤtoŋ} 
\classe{n} 
\begin{définition}\pfra{manteau}\end{définition}
\begin{définition}\pcmn{上衣}\end{définition}\étymologie{stod.tʰuŋ}\end{entrée}

\begin{entrée}{stɤtpa}{}{ⓔstɤtpa} 
\classe{n}  
\grammaire{n.lieu} 
\begin{définition}\pfra{Stodpa}\end{définition}
\begin{définition}\pcmn{四大坝}\end{définition}
\begin{exemple}\pjya{stɤtpapɯ}\hspace{5pt}\pcmn{四大坝人}\end{exemple}\étymologie{stod.pa}\end{entrée}

\begin{entrée}{stɣɤrnɤstɣɤr}{}{ⓔstɣɤrnɤstɣɤr} 
\classe{idph.2} 
\begin{définition}\pfra{en bondissant}\end{définition}
\begin{définition}\pcmn{一跳一跳}\end{définition}\end{entrée}

\begin{entrée}{sthaβ}{}{ⓔsthaβ} 
\classe{vt} \sens{1}\paradigme{dir}{\_}
\begin{définition}\pfra{toucher}\end{définition}
\begin{définition}\pcmn{靠;碰}\end{définition}
\begin{exemple}\pjya{ɯ-taʁ ka-sthaβ}\hspace{5pt}\pcmn{他碰了一下(上面)}\end{exemple}
\begin{exemple}\pjya{kɤ-sthaβ-a}\hspace{5pt}\pcmn{我碰了一下}\end{exemple}
\begin{exemple}\pjya{ɯ-rŋa ɯ-taʁ a-jaʁ kɤ-sthaβ-a}\hspace{5pt}\pcmn{我用手碰了一下他的脸}\end{exemple}
\begin{exemple}\pjya{ɯʑo kɯ ɯ-jaʁ smi ɯ-taʁ ko-sthaβ tɕe pjɤ-sɤɕke}\hspace{5pt}\pcmn{他触摸到火了,很烫}\end{exemple}\sens{2}\paradigme{dir}{lɤ-}
\begin{définition}\pfra{mettre à chauffer sur le four}\end{définition}
\begin{définition}\pcmn{放在炉子上烤}\end{définition}
\begin{exemple}\pjya{tɯ-ci la-sthaβ}\hspace{5pt}\pcmn{他把水放在火上做热了}\end{exemple}\end{entrée}

\begin{entrée}{sthoŋsthoŋ}{}{ⓔsthoŋsthoŋ} 
\classe{idph.2} 
\begin{définition}\pfra{être déformé après avoir été trop rempli}\end{définition}
\begin{définition}\pcmn{形容口袋里的东西装得很满,变形了(指软的东西,如口袋、枕头棉花等等)}\end{définition}
\begin{exemple}\pjya{tɤ-mkɯm ɯ-rku mɯ-chɤ-βdi, sthoŋsthoŋ ʑo ɲɯ-pa}\hspace{5pt}\pcmn{枕头没有塞好,显得很满,不平整(显得很鼓鼓的)}\end{exemple}
\begin{sous-entrée}{sthoŋ}{ⓔsthoŋsthoŋⓝsthoŋ} 
\classe{idph.1} 
\begin{exemple}\pjya{tɯmbri kɯ pjɤ-k-ɤsɯxtɕɤr-ci ri ɲɤ-mbrɤt sthoŋ ʑo to-ti}\hspace{5pt}\pcmn{绳子系得太紧就突然断了}\end{exemple}\end{sous-entrée}

\begin{sous-entrée}{sthoŋnɤsthoŋ}{ⓔsthoŋsthoŋⓝsthoŋnɤsthoŋ} 
\classe{idph.3} 
\begin{exemple}\pjya{ɯ-skhrɯ mɯ́j-βdi tɕe, sthoŋnɤsthoŋ ʑo ɲɯ-ŋke}\hspace{5pt}\pcmn{她怀孕了,走路显得很臃肿}\end{exemple}\end{sous-entrée}

\begin{sous-entrée}{sthoŋnɤloŋ}{ⓔsthoŋsthoŋⓝsthoŋnɤloŋ}
\begin{exemple}\pjya{ɯ-mbrɯ ɯ-tɯ-ŋgɯ kɯ sthoŋnɤloŋ ʑo ɲɯ-rɤma}\hspace{5pt}\pcmn{他一边发脾气一边做事的样子}\end{exemple}\end{sous-entrée}

\begin{sous-entrée}{lɤmɤsthoŋ}{ⓔsthoŋsthoŋⓝlɤmɤsthoŋ}
\begin{exemple}\pjya{lɤmɤtshoŋ ci ʑo tu}\hspace{5pt}\pcmn{他膀大腰粗的}\end{exemple}\end{sous-entrée}

\begin{sous-entrée}{phɯsthoŋ}{ⓔsthoŋsthoŋⓝphɯsthoŋ}
\begin{exemple}\pjya{ɯ-re phɯsthoŋ ʑo ɲɤ-ɕlɯɣ}\hspace{5pt}\pcmn{他失声突然笑出来}\end{exemple}\end{sous-entrée}

\end{entrée}

\begin{entrée}{sthoʁ}{}{ⓔsthoʁ} 
\classe{vt} \sens{1}\paradigme{dir}{\_}
\begin{définition}\pfra{appuyer, pousser}\end{définition}
\begin{définition}\pcmn{按(用手)、推}\end{définition}
\begin{exemple}\pjya{kɯm thɯ-sthoʁ-a}\hspace{5pt}\pcmn{我把门推了一下(关门)}\end{exemple}
\begin{exemple}\pjya{kɯm kɤ-sthoʁ}\hspace{5pt}\pcmn{你关门吧}\end{exemple}\sens{2}\paradigme{dir}{pɯ-}
\begin{définition}\pfra{oppresser}\end{définition}
\begin{définition}\pcmn{压迫}\end{définition}
\begin{exemple}\pjya{rɟɤlpu kɯ mkhɤrmaŋ ra pjɯ-sthoʁ pjɤ-ŋu}\hspace{5pt}\pcmn{土司压迫老百姓}\end{exemple}\relationsémantique{参考}{\lien{ⓔnɯsthoʁ}{nɯsthoʁ}}\end{entrée}

\begin{entrée}{sthrɯβ}{}{ⓔsthrɯβ} 
\classe{idph.1} 
\begin{définition}\pfra{bruit de mucus qui sort du nez}\end{définition}
\begin{définition}\pcmn{鼻涕突然出来的声音}\end{définition}
\begin{exemple}\pjya{ɯ-ɕnaβ sthrɯβ ʑo thɯ-nɯɬoʁ tɕe ɲɯ-sɤjloʁ}\hspace{5pt}\pcmn{他的鼻涕噗的一声就出来了,很恶心}\end{exemple}
\begin{sous-entrée}{sthrɯβnɤsthrɯβ}{ⓔsthrɯβⓝsthrɯβnɤsthrɯβ} 
\classe{idph.3} \end{sous-entrée}

\begin{sous-entrée}{phɯsthrɯβ}{ⓔsthrɯβⓝphɯsthrɯβ} 
\classe{idph.7} \end{sous-entrée}

\begin{sous-entrée}{ɣɤsthɯsthrɯβ}{ⓔsthrɯβⓝɣɤsthɯsthrɯβ}\end{sous-entrée}

\end{entrée}

\begin{entrée}{sthɯβsthɯβ}{}{ⓔsthɯβsthɯβ} 
\classe{idph.2} 
\begin{définition}\pfra{sur le point d'apparaître}\end{définition}
\begin{définition}\pcmn{形容东西将露未露的样子}\end{définition}
\begin{exemple}\pjya{@baobao ɯ-ŋgɯ laχtɕha sthɯβsthɯβ ʑo ɲɯ-nɯxsɯ}\hspace{5pt}\pcmn{包里的东西露了一点出来}\end{exemple}
\begin{sous-entrée}{phɯsthɯβ}{ⓔsthɯβsthɯβⓝphɯsthɯβ} 
\classe{idph.7} 
\begin{définition}\pfra{en un instant}\end{définition}
\begin{définition}\pcmn{突然间}\end{définition}
\begin{exemple}\pjya{phɯsthɯβ ʑo ɲɤ-nɤre}\hspace{5pt}\pcmn{他突然间笑起来了}\end{exemple}\end{sous-entrée}

\end{entrée}

\begin{entrée}{sthɯci}{}{ⓔsthɯci} 
\classe{adv} 
\begin{définition}\pfra{autant}\end{définition}
\begin{définition}\pcmn{那么多}\end{définition}\end{entrée}

\begin{entrée}{sthɯt}{}{ⓔsthɯt} 
\classe{vt} \sens{1}\paradigme{dir}{\_}
\begin{définition}\pfra{finir}\end{définition}
\begin{définition}\pcmn{完}\end{définition}
\begin{exemple}\pjya{kɤ-nɤma ta-sthɯt}\hspace{5pt}\pcmn{他把工作做完了}\end{exemple}
\begin{exemple}\pjya{kɤ-nɤma nɯ-sthɯt-a}\hspace{5pt}\pcmn{我把工作做完了}\end{exemple}
\begin{exemple}\pjya{kɤ-βzjoz pa-sthɯt}\hspace{5pt}\pcmn{他学完了}\end{exemple}
\begin{exemple}\pjya{tɤ-scoz kɤ-rɤt pjɤ-sthɯt}\hspace{5pt}\pcmn{他把信写完了}\end{exemple}
\begin{exemple}\pjya{kɤ-ndza chɤ-sthɯt}\hspace{5pt}\pcmn{他吃完了}\end{exemple}
\begin{exemple}\pjya{kɤ-ntʂu la-nɯ-sthɯt}\hspace{5pt}\pcmn{他锄完草了}\end{exemple}\sens{2}\paradigme{dir}{pɯ-}
\begin{définition}\pfra{être fini, être perdu}\end{définition}
\begin{définition}\pcmn{完蛋了}\end{définition}
\begin{exemple}\pjya{pɯ-tɯ-nɯ-sthɯt}\hspace{5pt}\pcmn{你完蛋了!}\end{exemple}\end{entrée}

\begin{entrée}{sti}{₁}{ⓔstiⓗ1} 
\classe{vt} \paradigme{dir}{nɯ-}\paradigme{dir}{\_}
\begin{définition}\pfra{boucher}\end{définition}
\begin{définition}\pcmn{堵塞}\end{définition}
\begin{exemple}\pjya{kɯ-spoʁ ɲɯ́-wɣ-sti}\hspace{5pt}\pcmn{把洞堵住了}\end{exemple}
\begin{exemple}\pjya{nɤj nɤ-βra aj ʑ-nɯ-sti-t-a}\hspace{5pt}\pcmn{我替你去(做工)}\end{exemple}
\begin{exemple}\pjya{nɤ-mtɕhi lɤ-sti}\hspace{5pt}\pcmn{你捂住嘴巴吧!}\end{exemple}
\begin{exemple}\pjya{ɯ-mtɕhi kɤ-sti-t-a}\hspace{5pt}\pcmn{我捂住了他的嘴巴}\end{exemple}
\begin{exemple}\pjya{phoŋ pɯ-sti-t-a}\hspace{5pt}\pcmn{我把瓶子盖上}\end{exemple}
\begin{exemple}\pjya{tɤχsɤr ɯ-kɯ-sti ɕti-a ma koŋla a-kɤ-spa me}\hspace{5pt}\pcmn{我只是充数的,我什么也不会}\end{exemple}\relationsémantique{参考}{\lien{ⓔphoŋsti}{phoŋsti}}\relationsémantique{参考}{\lien{ⓔasti}{asti}}\end{entrée}

\begin{entrée}{sti}{₂}{ⓔstiⓗ2} 
\classe{vt} \paradigme{dir}{tɤ-}
\begin{définition}\pfra{enlever ce qui est en trop}\end{définition}
\begin{définition}\pcmn{把多余的东西减掉一些}\end{définition}
\begin{exemple}\pjya{ɯ-tɯ-mtshɤt to-tɕhom tɕe tú-wɣ-sti ɲɯ-ra}\hspace{5pt}\pcmn{太满了,要舀一点(水)出来(不然就会扑出来)}\end{exemple}
\begin{exemple}\pjya{ki tʂha ki ku-sti-a ɲɯ-ntshi-a ma ɯ-tɯ-mtshɤt to-tɕhom}\hspace{5pt}\pcmn{我喝了一口,不然这个太满了}\end{exemple}
\begin{exemple}\pjya{a-ŋga ɲɯ-sti-a ɲɯ-ntshi ma ɲɯ-sɤɕke}\hspace{5pt}\pcmn{我要脱一些衣服,太热了}\end{exemple}\relationsémantique{同义词}{\lien{ⓔnɯβʑit}{nɯβʑit}}\end{entrée}

\begin{entrée}{stiaŋnɤstiaŋ}{}{ⓔstiaŋnɤstiaŋ} 
\classe{idph.2} 
\begin{définition}\pfra{en bondissant}\end{définition}
\begin{définition}\pcmn{一跳一跳(蚱蜢)}\end{définition}\end{entrée}

\begin{entrée}{stukɤr}{}{ⓔstukɤr} 
\classe{n} 
\begin{définition}\pfra{poutre}\end{définition}
\begin{définition}\pcmn{梁}\end{définition}\end{entrée}

\begin{entrée}{stonka}{}{ⓔstonka} 
\classe{n} 
\begin{définition}\pfra{automne}\end{définition}
\begin{définition}\pcmn{秋天}\end{définition}\étymologie{ston.ka}\end{entrée}

\begin{entrée}{stoŋtsu}{}{ⓔstoŋtsu} 
\classe{num} 
\begin{définition}\pfra{mille}\end{définition}
\begin{définition}\pcmn{一千}\end{définition}
\begin{exemple}\pjya{ɯ-stoŋtsu}\hspace{5pt}\pcmn{几千个}\end{exemple}
\begin{exemple}\pjya{stoŋtu kɤ-χsɤr}\hspace{5pt}\pcmn{上千(个、次)}\end{exemple}\étymologie{stoŋ}\end{entrée}

\begin{entrée}{stoʁ}{}{ⓔstoʁ} 
\classe{n} 
\begin{définition}\pfra{pois}\end{définition}
\begin{définition}\pcmn{胡豆}\end{définition}\end{entrée}

\begin{entrée}{stoʁldzɣɤm}{}{ⓔstoʁldzɣɤm} 
\classe{n} 
\begin{définition}\pfra{paille de pois}\end{définition}
\begin{définition}\pcmn{胡豆秸}\end{définition}\end{entrée}

\begin{entrée}{stoʁmboʁ}{}{ⓔstoʁmboʁ} 
\classe{n} 
\begin{définition}\pfra{explosion (fusil)}\end{définition}
\begin{définition}\pcmn{爆炸(枪)}\end{définition}
\begin{exemple}\pjya{ɕɤmɯɣdɯ stoʁmboʁ ɲɤ-ɕe (nɯ-ari)}\hspace{5pt}\pcmn{枪不小心爆炸了}\end{exemple}\relationsémantique{参考}{\lien{ⓔamboʁ}{amboʁ}}\end{entrée}

\begin{entrée}{stoʁrŋu}{}{ⓔstoʁrŋu} 
\classe{n} 
\begin{définition}\pfra{fèves grillées}\end{définition}
\begin{définition}\pcmn{炒胡豆}\end{définition}\relationsémantique{参考}{\lien{ⓔstoʁ}{stoʁ}}\relationsémantique{参考}{\lien{ⓔrŋu}{rŋu}}\end{entrée}

\begin{entrée}{stoʁthɤβ}{}{ⓔstoʁthɤβ} 
\classe{n} 
\begin{définition}\pfra{culture parallèle}\end{définition}
\begin{définition}\pcmn{兼种(胡豆中种豌豆)}\end{définition}\end{entrée}

\begin{entrée}{stoʁtsa}{}{ⓔstoʁtsa} 
\classe{n} 
\begin{définition}\pfra{une plante}\end{définition}
\begin{définition}\pcmn{植物的一种}\end{définition}
\begin{exemple}\pjya{stoʁtsa nɯ li sɯjno ci ŋu, stɤmku cho tɯji ɯ-rkɯ ra tu-ɬoʁ ŋu, mɤ-mbro. ɯ-ru nɯ kɯ-ngɯ-ngɯt ŋu, kɯ-pɣi tsa ŋu, ɯ-jwaʁ ŋɯ arŋi, mba ri nɤrko, ɯ-mɯntoʁ kɯ-dɤn ʑo tɯtɯrca kɯ-ɤʑɯrja ɲɯ-lɤt ŋu. ɯ-mdoʁ aɣɯrnɯɕɯr. ɯ-tshɯɣa stoʁ ɯ-mɯntoʁ fse. ɯ-cɤβ chɯ-βze ŋu, thɯ-tɯt tɕe, ɯ-rdoʁ nɯ rko, tɕe stoʁtsa rmi tɕe, stoʁ ɯ-ftsa kɤ-ti ɲɯ-ŋu. fsapaʁ ra kɤ-ndza rga-nɯ, tɯrme kɤ-ndza mɤ-sna.}\hspace{5pt}\pcmn{\lien{ⓔstoʁtsa}{stoʁtsa}是一种草,生长在草地和田野边,长得不高。茎长得很结实,是灰色的,叶子是绿色的,薄而结实。花在一起排列着,是淡红色的,形状像胡豆的花一样。结的是荚果,成熟后,种子变硬,所以叫作 \lien{ⓔstoʁtsa}{stoʁtsa},就是“胡豆粒粒儿”的意思。牲畜喜欢吃,人不能吃。}\end{exemple}\end{entrée}

\begin{entrée}{stosqa}{}{ⓔstosqa} 
\classe{n} 
\begin{définition}\pfra{fève pas encore mûre}\end{définition}
\begin{définition}\pcmn{未熟的胡豆}\end{définition}\relationsémantique{参考}{\lien{ⓔstoʁ}{stoʁ}}\end{entrée}

\begin{entrée}{stɯm}{}{ⓔstɯm} 
\classe{vt} \paradigme{dir}{lɤ-}
\begin{définition}\pfra{ramasser (jambes)}\end{définition}
\begin{définition}\pcmn{收拢(手、脚)}\end{définition}
\begin{exemple}\pjya{a-mi lɤ-stɯm-a}\hspace{5pt}\pcmn{我把脚收拢了}\end{exemple}\relationsémantique{参考}{\lien{ⓔrɤstɯm}{rɤstɯm}}\end{entrée}

\begin{entrée}{stɯnmɯ}{}{ⓔstɯnmɯ} 
\classe{n} 
\begin{définition}\pfra{mariage}\end{définition}
\begin{définition}\pcmn{婚姻}\end{définition}\relationsémantique{参考}{\lien{ⓔrɯstɯnmɯ}{rɯstɯnmɯ}}\étymologie{ston.mo}\end{entrée}

\begin{entrée}{stɯrstɯr}{}{ⓔstɯrstɯr} 
\classe{idph.2} 
\begin{définition}\pfra{objet ronds et petits}\end{définition}
\begin{définition}\pcmn{形容圆形,很细小的东西(如珠子、豌豆等)}\end{définition}
\begin{sous-entrée}{ɣɤstɯrlɯr}{ⓔstɯrstɯrⓝɣɤstɯrlɯr} 
\classe{vi} 
\begin{définition}\pfra{sautiller, rebondir (petits objets ronds)}\end{définition}
\begin{définition}\pcmn{弹来弹去(豌豆、珠子等)}\end{définition}\relationsémantique{同义词}{\lien{ⓔɣɤzdɯzdɯr}{ɣɤzdɯzdɯr}}\relationsémantique{同义词}{\lien{ⓔɣɤstaŋlaŋ}{ɣɤstaŋlaŋ}}\relationsémantique{同义词}{\lien{ⓔstɤjstɤjⓝɣɤstɤjlɤj}{ɣɤstɤjlɤj}}\relationsémantique{同义词}{\lien{ⓔstɤjstɤj}{stɤjstɤj}}\end{sous-entrée}

\end{entrée}

\begin{entrée}{stɯsti}{}{ⓔstɯsti} 
\classe{adv} 
\begin{définition}\pfra{seul}\end{définition}
\begin{définition}\pcmn{独自一个人}\end{définition}\relationsémantique{参考}{\lien{ⓔɯʑo-sti}{ɯʑo-sti}}\end{entrée}

\begin{entrée}{stuxsi/\variante{stuksi}}{}{ⓔstuxsi} 
\classe{n} 
\begin{définition}\pfra{joug, attelage pour deux animaux}\end{définition}
\begin{définition}\pcmn{牛轭(双行)}\end{définition}\end{entrée}

\begin{entrée}{sɯ}{}{ⓔsɯ} 
\classe{vs} \paradigme{dir}{tɤ-}\paradigme{dir}{thɯ-}
\begin{définition}\pfra{riche, fructueux}\end{définition}
\begin{définition}\pcmn{茂盛;丰富}\end{définition}
\begin{exemple}\pjya{jiɕqha nɯ ɯ-ɲɤm ɲɯ-sɯ}\hspace{5pt}\pcmn{那个(动物)很壮}\end{exemple}
\begin{exemple}\pjya{tɤ-pɤtso ɯ-rgu ɲɯ-sɯ}\hspace{5pt}\pcmn{那个小孩子很有能力(不要小看他)}\end{exemple}
\begin{exemple}\pjya{ɣɯjpa taχpa ɲɯ-sɯ}\hspace{5pt}\pcmn{今年的庄稼很好}\end{exemple}\relationsémantique{参考}{\lien{ⓔɯ-rgu,sɯ}{ɯ-rgu,sɯ}}\relationsémantique{参考}{\lien{}{ɯ-ɲɤm,sɯ}}\relationsémantique{参考}{\lien{ⓔnɯɲɤmsɯ}{nɯɲɤmsɯ}}\end{entrée}

\begin{entrée}{sɯbɣi}{}{ⓔsɯbɣi} 
\classe{n} 
\begin{définition}\pfra{espèce d'arbrisseau}\end{définition}
\begin{définition}\pcmn{灌木的一种}\end{définition}
\begin{exemple}\pjya{sɯbɣi χsɯ-tɯphu tu, tɯ-tɯphu nɯ kɯ-xtshɯm kɯ-rɲɟi ŋu, tɯ-phɯ ɯ-ŋgɯ tɕe kɯ-dɯ-dɤn ʑo tu. ɯ-mat nɯ staχpɯ ɯ-mat kɯ-fse ŋu, ɯ-ru nɯ kuxtɕo kɤ-βzu sna, mɤʑɯ tɯ-tɯphu nɯ, sɯbɣi nɯ jpum tsa mbro tsa ɯ-mnɯ nɯ ɯ-spjɯŋ nɯ kɯ-wɣrum tɕe kɯ-mpɯ ŋu, ɯ-mɯntoʁ nɯ kɯ-wɣrum ɲɯ-lɤt tɕe mpɕɤr, ɯ-mat me, mɤʑɯ tɯ-tɯphu nɯ sɯŋgɯ kɯ-wxti ɯ-ŋgɯ tɯ-tɯphu ma me, xtɕi ri ɯ-ru nɯ ngɯt ɯ-mɯntoʁ cho ɯ-mat me. sɯbɣi χsɯ-tɯphu nɯ nɯ-rqhu kɯ-pɣi ŋu, tɕe núndʐa sɯbɣi rmi}\hspace{5pt}\pcmn{\lien{ⓔsɯbɣi}{sɯbɣi} 有三种,一种长得细长,一株长有许多根,果实像豌豆的果实,干了可以编背篼。另一种长得又粗又高,枝条主心是白色的,很软,能开出好看的白花,没有果实。还有一种长在高大的森林里,比较少见,虽然矮小但主干非常结实,既没有花,又没有果实。这三种\lien{ⓔsɯbɣi}{sɯbɣi}树皮都是灰色的,所以叫作\lien{ⓔsɯbɣi}{sɯbɣi}。}\end{exemple}\end{entrée}

\begin{entrée}{sɯβde}{}{ⓔsɯβde}\relationsémantique{参考}{\lien{ⓔβde}{βde}}\end{entrée}

\begin{entrée}{sɯβɣi}{}{ⓔsɯβɣi} 
\classe{n} 
\begin{définition}\pfra{sciure}\end{définition}
\begin{définition}\pcmn{锯末}\end{définition}\relationsémantique{参考}{\lien{ⓔtɯ-βɣi}{tɯ-βɣi}}\relationsémantique{参考}{\lien{ⓔsiⓗ1}{si₁}}\end{entrée}

\begin{entrée}{sɯβɣɯt}{}{ⓔsɯβɣɯt}\relationsémantique{参考}{\lien{ⓔβɣɯt}{βɣɯt}}\end{entrée}

\begin{entrée}{sɯβɟɤt}{}{ⓔsɯβɟɤt}\relationsémantique{参考}{\lien{ⓔβɟɤt}{βɟɤt}}\end{entrée}

\begin{entrée}{sɯβɟi}{}{ⓔsɯβɟi}\relationsémantique{参考}{\lien{ⓔβɟiⓗ1}{βɟi₁}}\end{entrée}

\begin{entrée}{sɯβʁa}{}{ⓔsɯβʁa}\relationsémantique{参考}{\lien{}{βra}}\end{entrée}

\begin{entrée}{sɯβsɯβ}{}{ⓔsɯβsɯβ} 
\classe{idph.2} 
\begin{définition}\pfra{recouvert de poils fins}\end{définition}
\begin{définition}\pcmn{形容毛茸茸的样子}\end{définition}
\begin{exemple}\pjya{ɯ-jaʁ sɯβsɯβ ʑo ɲɯ-pa}\hspace{5pt}\pcmn{他手上长满了毛}\end{exemple}
\begin{sous-entrée}{sɯβnɤsɯβ}{ⓔsɯβsɯβⓝsɯβnɤsɯβ}
\begin{définition}\pfra{douleur lancinante}\end{définition}
\begin{définition}\pcmn{形容一阵一阵地痛}\end{définition}
\begin{exemple}\pjya{a-tɯ-ɣmaz sɯβnɤsɯβ ɲɯ-mŋɤm}\hspace{5pt}\pcmn{我的伤口一阵一阵地痛}\end{exemple}\end{sous-entrée}

\begin{sous-entrée}{ɣɤsɯβsɯβ}{ⓔsɯβsɯβⓝɣɤsɯβsɯβ} 
\classe{vs} 
\begin{définition}\pfra{avoir une douleur lancinante}\end{définition}
\begin{définition}\pcmn{一阵一阵地痛}\end{définition}\relationsémantique{参考}{\lien{ⓔrsɯβrsɯβ}{rsɯβrsɯβ}}\end{sous-entrée}

\end{entrée}

\begin{entrée}{sɯβzu}{}{ⓔsɯβzu}\relationsémantique{参考}{\lien{ⓔβzuⓗ1}{βzu₁}}\end{entrée}

\begin{entrée}{sɯβzi}{}{ⓔsɯβzi}\relationsémantique{参考}{\lien{ⓔβzi}{βzi}}\end{entrée}

\begin{entrée}{sɯɕku}{}{ⓔsɯɕku} 
\classe{n} 
\begin{définition}\pfra{poireau}\end{définition}
\begin{définition}\pcmn{韭葱【扁担韭】}\end{définition}
\begin{exemple}\pjya{sɯɕku nɯ zgo khro mɤ-kɯ-mbro ɣɯ sɤjku cho mɲɤm ɯ-ŋgɯ tu-ɬoʁ ɲɯ-ŋu, ɯ-jwaʁ ɲɯ-pɣi tɕe ɲɯ-rʁom, kɯ-tɕɤr tɕe kɯ-rɲɟi ci ɲɯ-ŋu, ɯ-ru maŋe, ɯ-tho maŋe, tú-wɣ-ndza tɕe, ɕkɤphɤr cho ɲɯ-naχtɕɯɣ, ɲɯ-nɤkɤro.}\hspace{5pt}\pcmn{\lien{ⓔsɯɕku}{sɯɕku}生长在半山的白桦树和野白杨树的树林里,叶子是灰色的,粗糙,又窄又长,没有茎,没有花,吃起来和\lien{ⓔɕkɤphɤr}{ɕkɤphɤr}一样,还可以。}\end{exemple}\end{entrée}

\begin{entrée}{sɯɕke}{}{ⓔsɯɕke} 
\classe{vt} \paradigme{dir}{kɤ-}\paradigme{dir}{lɤ-}
\begin{définition}\pfra{faire brûler}\end{définition}
\begin{définition}\pcmn{烧焦}\end{définition}
\begin{exemple}\pjya{qajɣi lo-tɯ-sɯɕke-t}\hspace{5pt}\pcmn{你把馍馍烧焦了}\end{exemple}\relationsémantique{参考}{\lien{ⓔɕke}{ɕke}}\end{entrée}

\begin{entrée}{sɯɕlɯɣ}{}{ⓔsɯɕlɯɣ}\relationsémantique{参考}{\lien{ⓔɕlɯɣ}{ɕlɯɣ}}\end{entrée}

\begin{entrée}{sɯɕqhlɤt}{}{ⓔsɯɕqhlɤt}\relationsémantique{参考}{\lien{ⓔɕqhlɤt}{ɕqhlɤt}}\end{entrée}

\begin{entrée}{sɯɕtʂi}{}{ⓔsɯɕtʂi} 
\classe{vt}  
\grammaire{denom} \paradigme{dir}{tɤ-}
\begin{définition}\pfra{faire suer}\end{définition}
\begin{définition}\pcmn{令人流汗}\end{définition}
\begin{exemple}\pjya{ki ta-ma ɲɯ-ɴqa tɕe tɤ́-wɣ-sɯɕtʂi-a}\hspace{5pt}\pcmn{这个工作很辛苦,令我一身都是汗}\end{exemple}\relationsémantique{参考}{\lien{ⓔtɯ-ɕtʂi}{tɯ-ɕtʂi}}\end{entrée}

\begin{entrée}{sɯɕɯɣra}{}{ⓔsɯɕɯɣra} 
\classe{vt}  
\grammaire{denom} \paradigme{dir}{pɯ-}\paradigme{dir}{nɯ-}
\begin{définition}\pfra{tamiser}\end{définition}
\begin{définition}\pcmn{筛}\end{définition}
\begin{exemple}\pjya{ɲɤ-sɯɕɯɣra}\hspace{5pt}\pcmn{他筛了}\end{exemple}
\begin{exemple}\pjya{tɯjpu pjɤ-sɯɕɯɣ-ra}\hspace{5pt}\pcmn{他筛了粮食}\end{exemple}\relationsémantique{同义词}{\lien{ⓔsɯxtshaʁ}{sɯxtshaʁ}}\relationsémantique{参考}{\lien{ⓔɕɯɣra}{ɕɯɣra}}\end{entrée}

\begin{entrée}{sɯfɕɤl}{}{ⓔsɯfɕɤl}\relationsémantique{参考}{\lien{ⓔfɕɤl}{fɕɤl}}\end{entrée}

\begin{entrée}{sɯfkrɯz}{}{ⓔsɯfkrɯz}\relationsémantique{参考}{\lien{ⓔfkrɯz}{fkrɯz}}\end{entrée}

\begin{entrée}{sɯfsaŋ}{}{ⓔsɯfsaŋ} 
\classe{vt}  
\grammaire{refl} \paradigme{dir}{tɤ-}\paradigme{dir}{tɤ-}
\begin{définition}\pfra{faire des fumigations rituelles}\end{définition}
\begin{définition}\pcmn{求烟子}\end{définition}
\begin{exemple}\pjya{tɤfsaŋ ɯ-kɯ-sɯfsaŋ me, tɯ-ci ɯ-kɯ-χtɕi me}\hspace{5pt}\pcmn{柏树叶没有人为它求烟,水没有人洗}\end{exemple}
\begin{sous-entrée}{ʑɣɤsɯfsaŋ}{ⓔsɯfsaŋⓝʑɣɤsɯfsaŋ} 
\classe{vi} \end{sous-entrée}

\begin{définition}\pfra{faire des fumigations rituelles pour soi-même}\end{définition}
\begin{définition}\pcmn{为自己求烟子}\end{définition}
\begin{exemple}\pjya{tɤ-ʑɣɤsɯfsaŋ-a}\hspace{5pt}\pcmn{我为自己求了烟子}\end{exemple}\relationsémantique{参考}{\lien{ⓔfsaŋ}{fsaŋ}}\end{entrée}

\begin{entrée}{sɯfsoʁ}{}{ⓔsɯfsoʁ}\relationsémantique{参考}{\lien{ⓔfsoʁⓗ2}{fsoʁ₂}}\end{entrée}

\begin{entrée}{sɯftɕaʁ}{}{ⓔsɯftɕaʁ} 
\classe{vt} 
\begin{définition}\pfra{abîmer, salir}\end{définition}
\begin{définition}\pcmn{(把本来很干净的东西)弄脏}\end{définition}
\begin{exemple}\pjya{ɯ-thoʁ kɯ-ɤχsɯko nɯ, tɤɲɟoʁɲɟi pɯ-tɯ-χtɤr tɕe pɯ-tɯ-sɯftɕaʁ}\hspace{5pt}\pcmn{你在干净的地面上到处扔了垃圾,(把好端端的地面)弄脏了}\end{exemple}\relationsémantique{参考}{\lien{ⓔftɕaʁ}{ftɕaʁ}}\end{entrée}

\begin{entrée}{sɯftɕɯm}{}{ⓔsɯftɕɯm}\relationsémantique{参考}{\lien{ⓔftɕɯm}{ftɕɯm}}\end{entrée}

\begin{entrée}{sɯftshi}{}{ⓔsɯftshi}\relationsémantique{参考}{\lien{ⓔftshi}{ftshi}}\end{entrée}

\begin{entrée}{sɯɣdɯɣ}{}{ⓔsɯɣdɯɣ}\relationsémantique{参考}{\lien{ⓔdɯɣ}{dɯɣ}}\end{entrée}

\begin{entrée}{sɯɣe}{}{ⓔsɯɣe} 
\classe{vt} \paradigme{dir}{\_}
\begin{définition}\pfra{faire venir, inviter à venir}\end{définition}
\begin{définition}\pcmn{请人来}\end{définition}
\begin{exemple}\pjya{ɯʑo kha jɤ-sɯɣe-t-a}\hspace{5pt}\pcmn{我把他请到家里来}\end{exemple}\end{entrée}

\begin{entrée}{sɯɣjɤɣ}{}{ⓔsɯɣjɤɣ} 
\classe{vt} \paradigme{dir}{\_}
\begin{définition}\pfra{finir}\end{définition}
\begin{définition}\pcmn{结束}\end{définition}
\begin{exemple}\pjya{kɤ-rɤtʂɯβ kɤ-sɯɣjaɣ-a}\hspace{5pt}\pcmn{我缝完了}\end{exemple}\relationsémantique{同义词}{\lien{ⓔsthɯt}{sthɯt}}\relationsémantique{参考}{\lien{ⓔjɤɣ}{jɤɣ}}\end{entrée}

\begin{entrée}{sɯɣjɯm}{}{ⓔsɯɣjɯm}\relationsémantique{参考}{\lien{ⓔjɯmⓗ2}{jɯm}}\end{entrée}

\begin{entrée}{sɯɣli}{}{ⓔsɯɣli}\relationsémantique{参考}{\lien{ⓔliⓗ3}{li₃}}\end{entrée}

\begin{entrée}{sɯɣlɯɣ}{}{ⓔsɯɣlɯɣ}\relationsémantique{参考}{\lien{ⓔlɯɣ}{lɯɣ}}\end{entrée}

\begin{entrée}{sɯɣlɯz}{}{ⓔsɯɣlɯz} 
\classe{vt}  
\grammaire{caus} \paradigme{dir}{nɯ-}
\begin{définition}\pfra{laisser}\end{définition}
\begin{définition}\pcmn{留下一部分}\end{définition}
\begin{exemple}\pjya{lonba ma-pɯ-tɯ-ci tɕe, tsuku nɯ-sɯɣlɯz}\hspace{5pt}\pcmn{你不要全部打掉,留一些(茶)}\end{exemple}\relationsémantique{参考}{\lien{ⓔlɯz}{lɯz}}\end{entrée}

\begin{entrée}{sɯɣmbɤβ}{}{ⓔsɯɣmbɤβ}\relationsémantique{参考}{\lien{ⓔmbɤβⓗ1}{mbɤβ₁}}\end{entrée}

\begin{entrée}{sɯɣmbuz}{}{ⓔsɯɣmbuz}\relationsémantique{参考}{\lien{ⓔmbuz}{mbuz}}\end{entrée}

\begin{entrée}{sɯɣnɤz}{}{ⓔsɯɣnɤz}\relationsémantique{参考}{\lien{ⓔnɤz}{nɤz}}\end{entrée}

\begin{entrée}{sɯɣndɤɣ}{}{ⓔsɯɣndɤɣ}\relationsémantique{参考}{\lien{ⓔndɤɣ}{ndɤɣ}}\end{entrée}

\begin{entrée}{sɯɣndɯl}{}{ⓔsɯɣndɯl}\relationsémantique{参考}{\lien{ⓔndɯlⓗ1}{ndɯl₁}}\end{entrée}

\begin{entrée}{sɯɣndzu}{}{ⓔsɯɣndzu}\relationsémantique{参考}{\lien{ⓔndzu}{ndzu}}\end{entrée}

\begin{entrée}{sɯɣndzar}{}{ⓔsɯɣndzar}\relationsémantique{参考}{\lien{ⓔndzar}{ndzar}}\end{entrée}

\begin{entrée}{sɯɣndzi}{}{ⓔsɯɣndzi}\relationsémantique{参考}{\lien{ⓔndzi}{ndzi}}\end{entrée}

\begin{entrée}{sɯɣndzur}{}{ⓔsɯɣndzur} 
\classe{vt} \paradigme{dir}{tɤ-}
\begin{définition}\pfra{dresser, relever}\end{définition}
\begin{définition}\pcmn{立起来}\end{définition}
\begin{exemple}\pjya{tɤ-jtsi tɤ-sɯɣndzur}\hspace{5pt}\pcmn{把柱子立起来吧}\end{exemple}
\begin{exemple}\pjya{romɲa tɤ-sɯɣndzur}\hspace{5pt}\pcmn{把小梁立起来吧}\end{exemple}
\begin{exemple}\pjya{tɤ-pɤtso tɤ-sɯɣndzur}\hspace{5pt}\pcmn{把孩子扶起来吧}\end{exemple}\relationsémantique{参考}{\lien{ⓔndzur}{ndzur}}\end{entrée}

\begin{entrée}{sɯɣndzɯr}{}{ⓔsɯɣndzɯr}\relationsémantique{参考}{\lien{ⓔndzɯr}{ndzɯr}}\end{entrée}

\begin{entrée}{sɯɣndʐi}{}{ⓔsɯɣndʐi}\relationsémantique{参考}{\lien{ⓔndʐi}{ndʐi}}\relationsémantique{参考}{\lien{ⓔftʂi}{ftʂi}}\end{entrée}

\begin{entrée}{sɯɣndʐoʁ}{}{ⓔsɯɣndʐoʁ}\relationsémantique{参考}{\lien{ⓔndʐoʁ}{ndʐoʁ}}\end{entrée}

\begin{entrée}{sɯɣndʐɯm}{}{ⓔsɯɣndʐɯm}\relationsémantique{参考}{\lien{ⓔndʐɯm}{ndʐɯm}}\end{entrée}

\begin{entrée}{sɯɣɲaʁ}{}{ⓔsɯɣɲaʁ} 
\classe{vt}  
\grammaire{refl} \sens{1}\paradigme{dir}{nɯ-}
\begin{définition}\pfra{rendre noir}\end{définition}
\begin{définition}\pcmn{使其变黑}\end{définition}
\begin{exemple}\pjya{tɯthɯ ta-sɯɣɲaʁ}\hspace{5pt}\pcmn{火把锅子烧黑了}\end{exemple}
\begin{exemple}\pjya{ʁmɯrtsɯ ɯ-mat kɯ a-ɕɣa ɲɤ-sɯɣɲaʁ}\hspace{5pt}\pcmn{果子把我的牙齿弄黑}\end{exemple}
\begin{exemple}\pjya{tɯ-ŋga nɯ-sɯɣɲaʁ-a}\hspace{5pt}\pcmn{我把衣服染成黑色了}\end{exemple}\sens{2}\paradigme{dir}{pɯ-}\paradigme{dir}{nɯ-}
\begin{définition}\pfra{calomnier, médire de}\end{définition}
\begin{définition}\pcmn{诬蔑}\end{définition}
\begin{exemple}\pjya{aʑo tɤ-nɯndo-t-a me tɕe, kɤ-sɯɣɲaʁ mɤ-tɯ-cha}\hspace{5pt}\pcmn{不是我拿的,你不可以诬蔑我}\end{exemple}
\begin{exemple}\pjya{ma-pɯ-kɯ-sɯɣɲaʁ-a}\hspace{5pt}\pcmn{你不要诬蔑我}\end{exemple}
\begin{sous-entrée}{ʑɣɤsɯɣɲaʁ}{ⓔsɯɣɲaʁⓢ2ⓝʑɣɤsɯɣɲaʁ} 
\classe{vi} \end{sous-entrée}

\begin{définition}\pfra{se noircir}\end{définition}
\begin{définition}\pcmn{给自己抹黑}\end{définition}
\begin{exemple}\pjya{tɤ-pɤtso ni tɤɣro kɯ ɲɯ-ʑɣɤsɯɣɲaʁ-ndʑi ʑo}\hspace{5pt}\pcmn{小孩子玩着把自己弄黑了}\end{exemple}
\begin{sous-entrée}{asɯɣɲɯɣɲaʁ}{ⓔsɯɣɲaʁⓝasɯɣɲɯɣɲaʁ} 
\classe{vi}  
\grammaire{recip} 
\begin{définition}\pfra{se noircir les uns les autres}\end{définition}
\begin{définition}\pcmn{互相抹黑}\end{définition}
\begin{exemple}\pjya{ɲɤ-k-ɤsɯɴqhɯɴqhi-ndʑi tɕe, ɲɤ-k-ɤsɯɣɲɯɣɲaʁ-ndʑi-ci ʑo}\hspace{5pt}\pcmn{他们俩互相弄脏了,互相弄黑了}\end{exemple}\end{sous-entrée}

\begin{sous-entrée}{nɯɣɯsɯɣɲaʁ}{ⓔsɯɣɲaʁⓝnɯɣɯsɯɣɲaʁ} 
\classe{vs} 
\begin{définition}\pfra{être facile à noircir}\end{définition}
\begin{définition}\pcmn{容易抹黑}\end{définition}\relationsémantique{参考}{\lien{ⓔɲaʁ}{ɲaʁ}}\end{sous-entrée}

\end{entrée}

\begin{entrée}{sɯɣɲat}{}{ⓔsɯɣɲat}\relationsémantique{参考}{\lien{ⓔɲat}{ɲat}}\end{entrée}

\begin{entrée}{sɯɣɲɟo}{}{ⓔsɯɣɲɟo}\relationsémantique{参考}{\lien{ⓔɲɟo}{ɲɟo}}\end{entrée}

\begin{entrée}{sɯɣɲo}{}{ⓔsɯɣɲo} 
\classe{vt}  
\grammaire{caus} \paradigme{dir}{tɤ-}
\begin{définition}\pfra{préparer}\end{définition}
\begin{définition}\pcmn{准备}\end{définition}
\begin{exemple}\pjya{kɤ-ndza tɤ-sɯɣɲo-t-a}\hspace{5pt}\pcmn{我准备了吃的东西}\end{exemple}
\begin{exemple}\pjya{kɤ-ŋga tɤ-sɯɣɲno-t-a}\hspace{5pt}\pcmn{我准备了吃的东西}\end{exemple}
\begin{exemple}\pjya{kɤ-tshi tɤ-sɯɣɲo-t-a}\hspace{5pt}\pcmn{我准备了喝的东西}\end{exemple}\relationsémantique{参考}{\lien{ⓔɲo}{ɲo}}\relationsémantique{同义词}{\lien{}{mɲo}}\end{entrée}

\begin{entrée}{sɯɣru}{}{ⓔsɯɣru}\relationsémantique{参考}{\lien{ⓔruⓗ1}{ru₁}}\end{entrée}

\begin{entrée}{sɯɣraʁ}{}{ⓔsɯɣraʁ}\relationsémantique{参考}{\lien{ⓔraʁⓗ1}{raʁ₁}}\end{entrée}

\begin{entrée}{sɯɣri}{}{ⓔsɯɣri} 
\classe{vt} \paradigme{dir}{pɯ-}
\begin{définition}\pfra{annuler}\end{définition}
\begin{définition}\pcmn{取消}\end{définition}
\begin{exemple}\pjya{jɯfɕɯr ju-nɯɕe-a nɯ-sɯso-t-a ri, mɯ́j-ŋgrɯ tɕe pɯ-sɯɣri-t-a}\hspace{5pt}\pcmn{我昨天准备回家,因为某种原因取消了(这种想法)}\end{exemple}\end{entrée}

\begin{entrée}{sɯɣrum}{}{ⓔsɯɣrum}\relationsémantique{参考}{\lien{ⓔwɣrum}{wɣrum}}\end{entrée}

\begin{entrée}{sɯɣrom}{}{ⓔsɯɣrom}\relationsémantique{参考}{\lien{ⓔrom}{rom}}\end{entrée}

\begin{entrée}{sɯɣʑaʁ}{}{ⓔsɯɣʑaʁ} 
\classe{vt} \paradigme{dir}{nɯ-}
\begin{définition}\pfra{s'entraîner}\end{définition}
\begin{définition}\pcmn{练习}\end{définition}
\begin{exemple}\pjya{tɤ-scoz (jɯɣi) ɲɯ-sɯɣʑaʁ ɲɯ-ŋu}\hspace{5pt}\pcmn{他在念书}\end{exemple}
\begin{exemple}\pjya{jɯɣi na-sɯɣʑaʁ}\hspace{5pt}\pcmn{他练了字}\end{exemple}
\begin{exemple}\pjya{ɕɤmɯɣdɯ kɤ-lɤt ɲɯ-sɯɣʑaʁ}\hspace{5pt}\pcmn{他在练习射枪}\end{exemple}
\begin{exemple}\pjya{kɤ-ŋke na-sɯɣʑaʁ, tɯ-ŋke na-sɯɣʑaʁ}\hspace{5pt}\pcmn{他练习走路了}\end{exemple}\end{entrée}

\begin{entrée}{sɯɣʑi}{}{ⓔsɯɣʑi}\relationsémantique{参考}{\lien{ⓔʑi}{ʑi}}\end{entrée}

\begin{entrée}{sɯjaʁndzu}{}{ⓔsɯjaʁndzu} 
\classe{vt} \paradigme{dir}{kɤ-}\paradigme{dir}{kɤ-}
\begin{définition}\pfra{montrer du doigt}\end{définition}
\begin{définition}\pcmn{指}\end{définition}
\begin{définition}\pfra{se montrer du doigt les uns les autres}\end{définition}
\begin{définition}\pcmn{互相指}\end{définition}
\begin{exemple}\pjya{ma-kɤ-kɯ-sɯjaʁndzu-a}\hspace{5pt}\pcmn{你不要指我}\end{exemple}
\begin{exemple}\pjya{tɤtʂu ŋotɕu ɲɯ-ŋu, kɤ-sɯjaʁndze}\hspace{5pt}\pcmn{灯在哪里,你指(给我看)}\end{exemple}
\begin{exemple}\pjya{ɯ-mphrɯmɯ ɲɯ-mto tɕe, kɤ-kɤ-sɯjaʁndzu tu-ste ɲɯ-ŋu}\hspace{5pt}\pcmn{他的卦应验了,非常准}\end{exemple}
\begin{exemple}\pjya{sla tú-wɣ-sɯjaʁndzu tɕe, tɯ-rna pjɯ-ɴɢraʁ ŋu}\hspace{5pt}\pcmn{如果用手指指月亮,耳朵就会受伤}\end{exemple}
\begin{sous-entrée}{asɯjaʁndzɯʁndzu}{ⓔsɯjaʁndzuⓝasɯjaʁndzɯʁndzu} 
\classe{vi} \end{sous-entrée}

\end{entrée}

\begin{entrée}{sɯjɣɤt}{}{ⓔsɯjɣɤt}\relationsémantique{参考}{\lien{ⓔjɣɤt}{jɣɤt}}\end{entrée}

\begin{entrée}{sɯjno}{₁}{ⓔsɯjnoⓗ1} 
\classe{n} 
\begin{définition}\pfra{herbe}\end{définition}
\begin{définition}\pcmn{草}\end{définition}\relationsémantique{参考}{\lien{ⓔarɯsɯjno}{arɯsɯjno}}\end{entrée}

\begin{entrée}{sɯjno}{₂}{ⓔsɯjnoⓗ2} 
\classe{vi} \paradigme{dir}{pɯ-}\paradigme{dir}{lɤ-}
\begin{définition}\pfra{enlever les mauvaises herbes}\end{définition}
\begin{définition}\pcmn{拔杂草}\end{définition}
\begin{exemple}\pjya{tɤɕi qaj ɯ-ŋgɯ ɕ-pɯ-sɯjno-a}\hspace{5pt}\pcmn{我去拔了在青稞和麦子里面的杂草}\end{exemple}
\begin{exemple}\pjya{ʑ-lo-sɯjno}\hspace{5pt}\pcmn{他去拔了杂草}\end{exemple}
\begin{exemple}\pjya{pɯ-sɯjno-j}\hspace{5pt}\pcmn{我们拔了杂草}\end{exemple}\end{entrée}

\begin{entrée}{sɯjnombrombro}{}{ⓔsɯjnombrombro} 
\classe{n} 
\begin{définition}\pfra{phasme}\end{définition}
\begin{définition}\pcmn{树枝虫,竹节虫【秤杆虫】}\end{définition}
\begin{exemple}\pjya{sɯjnombrombro nɯ qajɯ ci ŋu, ɯ-phoŋbu nɯ qaʑmbri kɯ-ɤrtɯ-rtaʁ fse, ɯ-mdoʁ ldʑaŋkɯ ci tu, kɯ-pɣi ci tu, kɯ-pɣi nɯ kɯ-jpum. ftɕar tɕe kɤ-mto tu ma qartsɯ tɕe kɤ-mto me.}\hspace{5pt}\pcmn{秤杆虫是一种虫子,身子像藤子树的枝桠。有一种是绿色的,还有一种是灰色的,灰色的那种粗壮一些。夏天常见,冬天看不到。}\end{exemple}\end{entrée}

\begin{entrée}{sɯjnoqa}{}{ⓔsɯjnoqa} 
\classe{n} 
\begin{définition}\pfra{racines}\end{définition}
\begin{définition}\pcmn{根}\end{définition}\end{entrée}

\begin{entrée}{sɯjnormbjɤβ}{}{ⓔsɯjnormbjɤβ} 
\classe{n} 
\begin{définition}\pfra{herbes coupées en bottes}\end{définition}
\begin{définition}\pcmn{捆成一把的杂草}\end{définition}\end{entrée}

\begin{entrée}{sɯjpɣom}{}{ⓔsɯjpɣom}\relationsémantique{参考}{\lien{ⓔjpɣom}{jpɣom}}\end{entrée}

\begin{entrée}{sɯjɯ}{}{ⓔsɯjɯ} 
\classe{n} 
\begin{définition}\pfra{louche en bois}\end{définition}
\begin{définition}\pcmn{木瓢}\end{définition}\end{entrée}

\begin{entrée}{sɯku}{}{ⓔsɯku} 
\classe{n} 
\begin{définition}\pfra{arbre}\end{définition}
\begin{définition}\pcmn{树}\end{définition}\end{entrée}

\begin{entrée}{sɯkɤcɯku}{}{ⓔsɯkɤcɯku} 
\classe{n} 
\begin{définition}\pfra{petites branches}\end{définition}
\begin{définition}\pcmn{各种树的枝桠;枝枝桠桠}\end{définition}\end{entrée}

\begin{entrée}{sɯkɤku}{}{ⓔsɯkɤku} 
\classe{n} 
\begin{définition}\pfra{sommet de l'arbre}\end{définition}
\begin{définition}\pcmn{树梢}\end{définition}\relationsémantique{参考}{\lien{ⓔsiⓗ2}{si₂}}\relationsémantique{参考}{\lien{ⓔtɯ-ku}{tɯ-ku}}\end{entrée}

\begin{entrée}{sɯkhɤrma}{}{ⓔsɯkhɤrma} 
\classe{vt} 
\begin{définition}\pfra{maudire}\end{définition}
\begin{définition}\pcmn{诅咒}\end{définition}
\begin{exemple}\pjya{pɯ́-wɣ-sɯkhɤrma-a}\hspace{5pt}\pcmn{他咒了我}\end{exemple}\relationsémantique{参考}{\lien{ⓔkhɤrma}{khɤrma}}\end{entrée}

\begin{entrée}{sɯkho}{}{ⓔsɯkho} 
\classe{vt}  
\grammaire{caus} \paradigme{dir}{nɯ-}\paradigme{dir}{kɤ-}\paradigme{dir}{\_}
\begin{définition}\pfra{inciter qqn à donner un objet, faire envoyer un objet}\end{définition}
\begin{définition}\pcmn{使交出来;使递过来}\end{définition}
\begin{exemple}\pjya{laχtɕha ɲɤ-sɯkho}\hspace{5pt}\pcmn{他让他把东西递过去了}\end{exemple}
\begin{exemple}\pjya{ɕ-kɤ-sɯkho-t-a, ʑ-nɯ-sɯkho-t-a}\hspace{5pt}\pcmn{我让他递过来了(递过去了)}\end{exemple}\relationsémantique{参考}{\lien{ⓔkhoⓗ1}{kho₁}}\end{entrée}

\begin{entrée}{sɯkhrɤt}{}{ⓔsɯkhrɤt}\relationsémantique{参考}{\lien{ⓔkhrɤtⓗ1}{khrɤt₁}}\end{entrée}

\begin{entrée}{sɯko}{}{ⓔsɯko}\relationsémantique{参考}{\lien{ⓔkoⓗ2}{ko}}\end{entrée}

\begin{entrée}{sɯkɯm}{}{ⓔsɯkɯm} 
\classe{n} 
\begin{définition}\pfra{orifice pour insérer le bois (dans les socs de charrue ou les pioches)}\end{définition}
\begin{définition}\pcmn{装木头用的洞(铧、锄头里面)}\end{définition}\end{entrée}

\begin{entrée}{sɯkɯnthoʁ}{}{ⓔsɯkɯnthoʁ} 
\classe{n} 
\begin{définition}\pfra{pic-vert}\end{définition}
\begin{définition}\pcmn{啄木鸟}\end{définition}
\begin{exemple}\pjya{sɯkɯnthoʁ nɯ pɣa ci ŋu, qapɣɤmtɯmtɯ sɤznɤ wxti, ɯ-mtsioʁ kɯ-rɲɟɯ-rɲɟi kɯ-ɤmtɕɯ-mtɕoʁ ŋu tɕe, si ɯ-ru ku-sɯspoʁ tɕe, ɯ-ŋgɯ qajɯ ɲɯ-nɯ-tɕɤt cha, tɕe si ɯ-ŋgɯ qajɯ nɯ ʁɟa ʑo tu-ndze ŋu, sɯkɯnthoʁ ɯ-βri kɯ-ɣɯrni tu, kɯ-ɲaʁ tu, kɯ-qarŋe tu, kɯ-wɣrum tu nɯ ra akhra tɕe wuma ʑo mpɕɤr. ɯ-jme rɲɟi tsa. aʁɤndɯndɤt ʑo ju-ɕe cha.}\hspace{5pt}\pcmn{啄木鸟是一种鸟,比戴胜大,嘴长而尖,在树干上打洞,自己能够啄出虫子,专门吃树干里的虫子。啄木鸟身子上有红、黑、黄、白等颜色的花纹,非常美丽。尾巴有点长。到处都能去。}\end{exemple}\end{entrée}

\begin{entrée}{sɯlaʁrdɤβ}{}{ⓔsɯlaʁrdɤβ} 
\classe{vl}  
\grammaire{denom} \paradigme{dir}{tɤ-}
\begin{définition}\pfra{donner un coup de pied (animal)}\end{définition}
\begin{définition}\pcmn{踢(用前肢)}\end{définition}
\begin{exemple}\pjya{mbro ɲɯ-sɯlaʁrdɤβ}\hspace{5pt}\pcmn{马在踢}\end{exemple}
\begin{exemple}\pjya{jla ɲɯ-sɯlaʁrdɤβ}\hspace{5pt}\pcmn{犏牛在踢}\end{exemple}
\begin{exemple}\pjya{tɤ́-wɣ-sɯlaʁrdaβ-a}\hspace{5pt}\pcmn{它踢了我一脚}\end{exemple}\relationsémantique{参考}{\lien{ⓔlaʁrdɤβ}{laʁrdɤβ}}\end{entrée}

\begin{entrée}{sɯlɤt}{}{ⓔsɯlɤt}\relationsémantique{参考}{\lien{}{lɤt}}\end{entrée}

\begin{entrée}{sɯldʑoʁ}{}{ⓔsɯldʑoʁ}\relationsémantique{参考}{\lien{ⓔldʑoʁ}{ldʑoʁ}}\end{entrée}

\begin{entrée}{sɯli}{}{ⓔsɯli} 
\classe{n} 
\begin{définition}\pfra{bovidé de couleur noire dont la tête, le haut du dos et la queue sont blancs}\end{définition}
\begin{définition}\pcmn{全身黑色,头、背梁和尾巴白色的牛}\end{définition}\end{entrée}

\begin{entrée}{sɯluj}{}{ⓔsɯluj} 
\classe{vt} \paradigme{dir}{\_}
\begin{définition}\pfra{recouvrir complètement la surface pour cacher}\end{définition}
\begin{définition}\pcmn{包起来;遮蔽}\end{définition}
\begin{exemple}\pjya{laχtɕha pɯ-sɯluj}\hspace{5pt}\pcmn{你把东西盖起来吧}\end{exemple}
\begin{exemple}\pjya{tɯjpu pɯ-sɯluj}\hspace{5pt}\pcmn{你把粮食盖起来吧}\end{exemple}
\begin{exemple}\pjya{tɤrcoʁ kɯ @yangyu to-sɯluj ʑo}\hspace{5pt}\pcmn{洋芋被泥巴覆盖了}\end{exemple}
\begin{exemple}\pjya{raz kɯ khɯtsa tɤ-sɯluj-a}\hspace{5pt}\pcmn{我用布把碗盖起来了}\end{exemple}\relationsémantique{参考}{\lien{ⓔɯ-luj}{ɯ-luj}}\relationsémantique{同义词}{\lien{ⓔɕɯfkaβ}{ɕɯfkaβ}}\end{entrée}

\begin{entrée}{sɯmat}{}{ⓔsɯmat} 
\classe{n} 
\begin{définition}\pfra{fruit}\end{définition}
\begin{définition}\pcmn{水果}\end{définition}\end{entrée}

\begin{entrée}{sɯmciphɯt}{}{ⓔsɯmciphɯt} 
\classe{vt} \paradigme{dir}{tɤ-}
\begin{définition}\pfra{cracher}\end{définition}
\begin{définition}\pcmn{吐唾液}\end{définition}
\begin{sous-entrée}{nɯmciphɯt}{ⓔsɯmciphɯtⓝnɯmciphɯt} 
\classe{vt} 
\begin{définition}\pfra{cracher}\end{définition}
\begin{définition}\pcmn{吐唾液}\end{définition}
\begin{exemple}\pjya{tɤ-nɯmciphɯt-a}\hspace{5pt}\pcmn{我向他吐了唾液}\end{exemple}\relationsémantique{参考}{\lien{ⓔmciphɯt}{mciphɯt}}\end{sous-entrée}

\end{entrée}

\begin{entrée}{sɯmdʑɯtɕoʁ}{}{ⓔsɯmdʑɯtɕoʁ} 
\classe{vi} \paradigme{dir}{tɤ-}
\begin{définition}\pfra{s'agenouiller sur une jambe (la manière dont les femmes et les serviteurs doivent s'asseoir)}\end{définition}
\begin{définition}\pcmn{跪(只跪一只膝盖)}\end{définition}
\begin{exemple}\pjya{tɤ-sɯmdʑɯtɕoʁ-a}\hspace{5pt}\pcmn{我跪下了}\end{exemple}
\begin{exemple}\pjya{to-sɯmdʑɯtɕoʁ}\hspace{5pt}\pcmn{她跪下了}\end{exemple}\end{entrée}

\begin{entrée}{sɯmɟa}{}{ⓔsɯmɟa} 
\classe{n} 
\begin{définition}\pfra{allumage d'un feu}\end{définition}
\begin{définition}\pcmn{引燃;点火}\end{définition}
\begin{exemple}\pjya{sɯmɟa tɤ-βzu-t-a}\hspace{5pt}\pcmn{我点了火}\end{exemple}\relationsémantique{参考}{\lien{ⓔmɟa}{mɟa}}\end{entrée}

\begin{entrée}{sɯmɟa}{}{ⓔsɯmɟa}\relationsémantique{参考}{\lien{ⓔmɟa}{mɟa}}\end{entrée}

\begin{entrée}{sɯmnɤr}{}{ⓔsɯmnɤr} 
\classe{n} 
\begin{définition}\pfra{pensée}\end{définition}
\begin{définition}\pcmn{想法}\end{définition}\end{entrée}

\begin{entrée}{sɯmɲo}{}{ⓔsɯmɲo}\relationsémantique{参考}{\lien{ⓔmɲoⓗ1}{mɲo₁}}\end{entrée}

\begin{entrée}{sɯmɲo}{}{ⓔsɯmɲo} 
\classe{vt} \sens{1}\paradigme{dir}{nɯ-}
\begin{définition}\pfra{comprendre}\end{définition}
\begin{définition}\pcmn{理解}\end{définition}
\begin{exemple}\pjya{nɯ ɲɯ-sɯmɲam-a ɕti}\hspace{5pt}\pcmn{我可以理解他}\end{exemple}
\begin{exemple}\pjya{ɲɯ́-wɣ-sɯmɲo-a}\hspace{5pt}\pcmn{他可以理解我}\end{exemple}
\begin{exemple}\pjya{jiɕqha nɯ ɲɯ-nɯzdɯɣ tɕe, aj ɲɯ-sɯmɲam-a ɕti}\hspace{5pt}\pcmn{他很担心,我可以理解他}\end{exemple}
\begin{exemple}\pjya{ta-sɯmɲo}\hspace{5pt}\pcmn{我可以理解你}\end{exemple}\sens{2}
\begin{définition}\pfra{faire préparer}\end{définition}
\begin{définition}\pcmn{让……准备好}\end{définition}\relationsémantique{参考}{\lien{ⓔmɲoⓗ1}{mɲo₁}}\end{entrée}

\begin{entrée}{sɯmŋɤn}{}{ⓔsɯmŋɤn} 
\classe{n} 
\begin{définition}\pfra{doute}\end{définition}
\begin{définition}\pcmn{疑心}\end{définition}
\begin{exemple}\pjya{sɯmŋɤn ma-tɤ-tɯ-βze}\hspace{5pt}\pcmn{你不要起疑心}\end{exemple}\relationsémantique{参考}{\lien{ⓔnɯsɯmŋɤn}{nɯsɯmŋɤn}}\end{entrée}

\begin{entrée}{sɯmoʁ}{}{ⓔsɯmoʁ}\relationsémantique{参考}{\lien{ⓔmoʁ}{moʁ}}\end{entrée}

\begin{entrée}{sɯmphru}{}{ⓔsɯmphru} 
\classe{n} 
\begin{définition}\pfra{le bois que l'on n'a pas encore fini de couper}\end{définition}
\begin{définition}\pcmn{尚未砍完的柴}\end{définition}
\begin{exemple}\pjya{a-sɯmphru}\hspace{5pt}\pcmn{我没有砍完的那一部分}\end{exemple}\relationsémantique{参考}{\lien{ⓔsiⓗ1}{si₁}}\relationsémantique{参考}{\lien{ⓔɯ-mphru}{ɯ-mphru}}\end{entrée}

\begin{entrée}{sɯmphrɤt}{}{ⓔsɯmphrɤt}\relationsémantique{参考}{\lien{ⓔmphrɤt}{mphrɤt}}\end{entrée}

\begin{entrée}{sɯmphɯ}{}{ⓔsɯmphɯ} 
\classe{n} 
\begin{définition}\pfra{outil pour casser les mottes de terre}\end{définition}
\begin{définition}\pcmn{土巴捶}\end{définition}
\begin{exemple}\pjya{sɯmphɯ ɯ-ru}\hspace{5pt}\pcmn{土巴捶的把子}\end{exemple}
\begin{exemple}\pjya{sɯmphɯ ɯ-pɤl}\hspace{5pt}\pcmn{土巴捶的刀}\end{exemple}\end{entrée}

\begin{entrée}{sɯmsɯm}{}{ⓔsɯmsɯm} 
\classe{idph.2} 
\begin{définition}\pfra{formant une couche fine}\end{définition}
\begin{définition}\pcmn{构成了薄薄的一层}\end{définition}
\begin{exemple}\pjya{tɤjpa sɯmsɯm ko-lɤt}\hspace{5pt}\pcmn{下了薄薄的一层雪}\end{exemple}
\begin{exemple}\pjya{nɤ-βri ɯ-rme sɯmsɯm ci ɣɤʑu}\hspace{5pt}\pcmn{你身上有点毛茸茸的}\end{exemple}
\begin{exemple}\pjya{tɤ-tsrɯ sɯmsɯm ʑo to-ɬoʁ}\hspace{5pt}\pcmn{新发的芽,(很短)仿佛看得见}\end{exemple}\relationsémantique{参考}{\lien{ⓔɕɯmɕɯm}{ɕɯmɕɯm}}\end{entrée}

\begin{entrée}{sɯmtɕɤn}{}{ⓔsɯmtɕɤn} 
\classe{n} 
\begin{définition}\pfra{animaux}\end{définition}
\begin{définition}\pcmn{动物}\end{définition}\étymologie{sems.tɕan}\end{entrée}

\begin{entrée}{sɯmtɕɤnrtazoʁ}{}{ⓔsɯmtɕɤnrtazoʁ} 
\classe{n} 
\begin{définition}\pfra{animaux domestiques}\end{définition}
\begin{définition}\pcmn{牲畜}\end{définition}\étymologie{sems.tɕan rta.zog}\end{entrée}

\begin{entrée}{sɯmtɕɯr}{}{ⓔsɯmtɕɯr} 
\classe{vt}  
\grammaire{caus} \paradigme{dir}{\_}
\begin{définition}\pfra{faire tourner}\end{définition}
\begin{définition}\pcmn{使转动}\end{définition}
\begin{exemple}\pjya{laʁŋkhɤr thɯ-sɯmtɕɯr-a}\hspace{5pt}\pcmn{我把转经筒转动了}\end{exemple}
\begin{exemple}\pjya{mkhɯrlu kɤ-sɯmtɕɯr}\hspace{5pt}\pcmn{把轮子转动}\end{exemple}
\begin{exemple}\pjya{nɯ-sɯmtɕɯr-a}\hspace{5pt}\pcmn{我把它转动了}\end{exemple}
\begin{exemple}\pjya{pa-sɯmtɕɯr}\hspace{5pt}\pcmn{他把它转动了}\end{exemple}
\begin{exemple}\pjya{tɯ-ku tú-wɣ-sɯmtɕɯr tsa ra}\hspace{5pt}\pcmn{要动脑筋}\end{exemple}\relationsémantique{参考}{\lien{ⓔmtɕɯr}{mtɕɯr}}\end{entrée}

\begin{entrée}{sɯmto}{}{ⓔsɯmto}\relationsémantique{参考}{\lien{ⓔmtoⓝmto}{mto}}\end{entrée}

\begin{entrée}{sɯmtshɤm}{}{ⓔsɯmtshɤm} 
\classe{vt}  
\grammaire{caus} \paradigme{dir}{pɯ-}
\begin{définition}\pfra{informer}\end{définition}
\begin{définition}\pcmn{通知}\end{définition}
\begin{exemple}\pjya{ɯʑo kha mɯ́j-rɤʑi tɕe pɯ-sɯmtsham-a}\hspace{5pt}\pcmn{因为他没在家里,我就把事情转告他了}\end{exemple}\relationsémantique{参考}{\lien{ⓔmtshɤm}{mtshɤm}}\end{entrée}

\begin{entrée}{sɯmtshɤt}{}{ⓔsɯmtshɤt} 
\classe{vt} \paradigme{dir}{tɤ-}\paradigme{dir}{\_}
\begin{définition}\pfra{remplir}\end{définition}
\begin{définition}\pcmn{填满}\end{définition}
\begin{exemple}\pjya{tɤ-fkɯm ta-sɯmtshɤt}\hspace{5pt}\pcmn{他把袋子填满了}\end{exemple}
\begin{exemple}\pjya{kha nɯ-sɯmtshɤt-i}\hspace{5pt}\pcmn{我们坐满了房间}\end{exemple}\relationsémantique{参考}{\lien{ⓔmtshɤt}{mtshɤt}}\end{entrée}

\begin{entrée}{sɯmtshoŋ}{}{ⓔsɯmtshoŋ}\relationsémantique{参考}{\lien{ⓔmtshoŋ}{mtshoŋ}}\end{entrée}

\begin{entrée}{sɯmtshɯβ}{}{ⓔsɯmtshɯβ}\relationsémantique{参考}{\lien{ⓔmtshɯβ}{mtshɯβ}}\end{entrée}

\begin{entrée}{sɯmtso}{}{ⓔsɯmtso} 
\classe{n} 
\begin{définition}\pfra{récit de ce qui s'est passé depuis que l'on s'est séparé}\end{définition}
\begin{définition}\pcmn{分开以来的经过}\end{définition}
\begin{exemple}\pjya{ɯ-sɯmtso na-βzu}\hspace{5pt}\pcmn{他给他讲了(分开以后发生的事情)}\end{exemple}
\begin{exemple}\pjya{ɯ-sɯmtso nɯ-βzu-t-a}\hspace{5pt}\pcmn{我给他讲了}\end{exemple}
\begin{exemple}\pjya{a-sɯmtso na-βzu}\hspace{5pt}\pcmn{他给我讲了}\end{exemple}\end{entrée}

\begin{entrée}{sɯmtsɯr}{}{ⓔsɯmtsɯr}\relationsémantique{参考}{\lien{ⓔmtsɯr}{mtsɯr}}\end{entrée}

\begin{entrée}{sɯmɯzdɯɣ}{}{ⓔsɯmɯzdɯɣ} 
\classe{n} 
\begin{définition}\pfra{inquiétude}\end{définition}
\begin{définition}\pcmn{操心}\end{définition}\relationsémantique{参考}{\lien{ⓔnɯsɯmɯzdɯɣ}{nɯsɯmɯzdɯɣ}}\étymologie{sems.sdug}\end{entrée}

\begin{entrée}{sɯndɤrmbjom}{}{ⓔsɯndɤrmbjom}\relationsémantique{参考}{\lien{ⓔndɤrmbjom}{ndɤrmbjom}}\end{entrée}

\begin{entrée}{sɯndo}{}{ⓔsɯndo}\relationsémantique{参考}{\lien{ⓔndo}{ndo}}\end{entrée}

\begin{entrée}{sɯndza}{}{ⓔsɯndza}\relationsémantique{参考}{\lien{ⓔndza}{ndza}}\end{entrée}

\begin{entrée}{sɯndzɯ}{}{ⓔsɯndzɯ}\relationsémantique{参考}{\lien{ⓔndzɯ}{ndzɯ}}\end{entrée}

\begin{entrée}{sɯndzɯpe}{}{ⓔsɯndzɯpe} 
\classe{vi} \paradigme{dir}{nɯ-}
\begin{définition}\pfra{s'asseoir par terre avec les deux jambes l'une sur l'autre en travers (la manière dont les femmes doivent s'asseoir lorsqu'elle n'ont pas de travail à faire)}\end{définition}
\begin{définition}\pcmn{双腿斜着坐(藏族妇女坐的姿势)}\end{définition}
\begin{exemple}\pjya{nɯtɕu nɯ-sɯndzɯpe}\hspace{5pt}\pcmn{你坐在那里}\end{exemple}
\begin{exemple}\pjya{nɯ-sɯndzɯpe-a}\hspace{5pt}\pcmn{我坐下了}\end{exemple}\end{entrée}

\begin{entrée}{sɯndʑaʁskɯsko/\variante{sɯndʑaʁfskɯfsko}}{}{ⓔsɯndʑaʁskɯsko} 
\classe{vi} \paradigme{dir}{nɯ-}
\begin{définition}\pfra{s'étirer}\end{définition}
\begin{définition}\pcmn{舒展筋骨;伸懒腰}\end{définition}
\begin{exemple}\pjya{tɤ-pɤtso ɲɯ-sɯndʑaʁfskɯfsko ɲɯ-cha}\hspace{5pt}\pcmn{小孩子会舒展筋骨}\end{exemple}
\begin{exemple}\pjya{nɯ-sɯndʑaʁfskɯfsko-a}\hspace{5pt}\pcmn{我舒展了筋骨}\end{exemple}\end{entrée}

\begin{entrée}{sɯndʑutɤndʑu}{}{ⓔsɯndʑutɤndʑu} 
\classe{n} 
\begin{définition}\pfra{petites pousses d'arbre autour des champs cultivés}\end{définition}
\begin{définition}\pcmn{田周围长的树芽}\end{définition}\end{entrée}

\begin{entrée}{sɯngrɯβ}{}{ⓔsɯngrɯβ}\relationsémantique{参考}{\lien{ⓔngrɯβ}{ngrɯβ}}\end{entrée}

\begin{entrée}{sɯntɕhɣaʁ}{}{ⓔsɯntɕhɣaʁ}\relationsémantique{参考}{\lien{ⓔntɕhɣaʁ}{ntɕhɣaʁ}}\end{entrée}

\begin{entrée}{sɯnthɯ}{}{ⓔsɯnthɯ} 
\classe{n} 
\begin{définition}\pfra{emplanture}\end{définition}
\begin{définition}\pcmn{榫头}\end{définition}\étymologie{fn:榫头}\end{entrée}

\begin{entrée}{sɯntshɤβ}{}{ⓔsɯntshɤβ}\relationsémantique{参考}{\lien{ⓔntshɤβ}{ntshɤβ}}\end{entrée}

\begin{entrée}{sɯŋgi}{}{ⓔsɯŋgi} 
\classe{n} 
\begin{définition}\pfra{lion}\end{définition}
\begin{définition}\pcmn{狮子}\end{définition}\étymologie{seŋ.ge}\end{entrée}

\begin{entrée}{sɯŋgo}{}{ⓔsɯŋgo}\relationsémantique{参考}{\lien{ⓔŋgo}{ŋgo}}\end{entrée}

\begin{entrée}{sɯŋgɯ}{}{ⓔsɯŋgɯ} 
\classe{n} 
\begin{définition}\pfra{forêt}\end{définition}
\begin{définition}\pcmn{森林}\end{définition}\relationsémantique{参考}{\lien{ⓔsiⓗ1}{si₁}}\relationsémantique{参考}{\lien{ⓔnɯsɯŋgɯ}{nɯsɯŋgɯ}}\relationsémantique{参考}{\lien{ⓔsɯŋgɯɟu}{sɯŋgɯɟu}}\relationsémantique{参考}{\lien{}{sɯŋgɯnaχtɕin}}\end{entrée}

\begin{entrée}{sɯŋgɯɟu}{}{ⓔsɯŋgɯɟu} 
\classe{n} 
\begin{définition}\pfra{une espèce d'arbrisseau}\end{définition}
\begin{définition}\pcmn{灌木的一种}\end{définition}
\begin{exemple}\pjya{sɯŋgɯ ɟu nɯ ɯ-jwaʁ kɯ-ɤrtɯm tɕe kɯ-ɤmtɕoʁ ŋu. ɯ-ru cho ɯ-jwaʁ qartsɯmɤftɕar ʑo anɯrŋi ɕti, ɯ-mat kɯnɤ kɯ-ɤrŋi ŋu thɯ-tɯt tɕe chɯ-ɲaʁ, ɯ-ru wuma ʑo mpɕu, ɲɯ́-wɣ-phɯt tɕe ndoʁ, tu-wxti wuma mɤ-cha tɕe, ndɯβ tɕe, zɣɤmbu ɲɯ́-wɣ-βzu tɕe pe.}\hspace{5pt}\pcmn{\lien{ⓔsɯŋgɯɟu}{sɯŋgɯɟu}有圆而尖的叶子。树干和叶子一年四季都是绿的,果子也是绿色的,成熟后变黑。树干很光滑,折断的时候是脆的,长不大,细小,可以用来做扫把}\end{exemple}\end{entrée}

\begin{entrée}{sɯŋgɯnaχtɕɯn}{}{ⓔsɯŋgɯnaχtɕɯn} 
\classe{n} 
\begin{définition}\pfra{forêt primaire de la montagne}\end{définition}
\begin{définition}\pcmn{深山老林}\end{définition}\étymologie{nags.tɕʰen}\end{entrée}

\begin{entrée}{sɯŋgɯpɤjka}{}{ⓔsɯŋgɯpɤjka} 
\classe{n} 
\begin{définition}\pfra{courge sauvage}\end{définition}
\begin{définition}\pcmn{野生瓜子的一种}\end{définition}
\begin{exemple}\pjya{sɯŋgɯ pɤjka nɯ sɯjno ci ɯ-ru kɯ-zɯ-zri tɕe kɯ-ɤrqhi tsa jɯ-ɕe tɕe, kɯmaʁ si ɯ-taʁ cho sɯjno ɯ-ru ɯ-taʁ tu-ortɯrtɤβ tɕe, ɲɯ-rɯmɯntoʁ tɕe ɯ-mat ku-tshoʁ ɲɯ-ŋu. ɯ-mɯntoʁ nɯ @laba kɯ-fse ci ŋu, tɕe kɯ-qarŋe ɲɯ-ŋu. ɯ-mat nɯ @huanggua kɯ-fsɯ-fse ci ŋu, kɯ-xtɕi tsa ɲɯ-ŋu. ɯ-jwaʁ nɯ pɤjka ɯ-jwaʁ cho ɲɯ-naχtɕɯɣ, ɯ-jwaʁ ɯ-taʁ ɯ-mdzu ɣɤʑu, ftɕar tɕe tu-ɬoʁ, qartsɯ tɕe pjɯ-rom ɲɯ-ŋu.}\hspace{5pt}\pcmn{\lien{ⓔsɯŋgɯpɤjka}{sɯŋgɯpɤjka}是一种草,茎很长,可以爬到很远,可以缠在其他树干和草的茎上开花结果。花像喇叭一样,是黄色的。果实很像黄瓜,但小一点。叶子和南瓜的叶子一样,叶子上还长有刺,春夏生长,冬天枯萎。}\end{exemple}\end{entrée}

\begin{entrée}{sɯŋgɯrmɤβja}{}{ⓔsɯŋgɯrmɤβja} 
\classe{n} 
\begin{définition}\pfra{faisan (lophophorus lhuysii)}\end{définition}
\begin{définition}\pcmn{绿尾虹雉}\end{définition}\end{entrée}

\begin{entrée}{sɯŋsɯŋ}{}{ⓔsɯŋsɯŋ} 
\classe{idph.2} 
\begin{définition}\pfra{blanc, pur, propre}\end{définition}
\begin{définition}\pcmn{形容又白又清洁的样子}\end{définition}
\begin{exemple}\pjya{ɯ-khɯtsa sɯŋsɯŋ to-nɯntsɯɣ}\hspace{5pt}\pcmn{他把碗舔得很干净了}\end{exemple}
\begin{exemple}\pjya{sɯŋsɯŋ ʑo to-ɕkɯt}\hspace{5pt}\pcmn{他吃得很干净了}\end{exemple}
\begin{exemple}\pjya{smɤɣ ɲɯ-wɣrum sɯŋsɯŋ ʑo}\hspace{5pt}\pcmn{羊毛又白又清洁}\end{exemple}\end{entrée}

\begin{entrée}{sɯɴɢoʁ}{}{ⓔsɯɴɢoʁ} 
\classe{n} 
\begin{définition}\pfra{bois mort}\end{définition}
\begin{définition}\pcmn{干柴}\end{définition}\end{entrée}

\begin{entrée}{sɯɴqhi}{}{ⓔsɯɴqhi}\relationsémantique{参考}{\lien{ⓔɴqhi}{ɴqhi}}\end{entrée}

\begin{entrée}{sɯpa}{₂}{ⓔsɯpaⓗ2} 
\classe{n} 
\begin{définition}\pfra{bois de chauffage découpé}\end{définition}
\begin{définition}\pcmn{劈开了的木柴【柴划子】}\end{définition}\relationsémantique{反义词}{\lien{ⓔɕɯrdɯm}{ɕɯrdɯm}}\end{entrée}

\begin{entrée}{sɯpa}{₁}{ⓔsɯpaⓗ1} 
\classe{vt}
\classe{num}
\classe{vt} \paradigme{dir}{tɤ-}
\begin{définition}\pfra{considérer}\end{définition}
\begin{définition}\pcmn{认为,当作}\end{définition}
\begin{exemple}\pjya{ɯʑo kɯ-ɕqraʁ tu-sɯpe-a pɯ-ŋu ri, mɯ́j-ɕqraʁ}\hspace{5pt}\pcmn{我以为他很聪明,其实他不聪明}\end{exemple}
\begin{sous-entrée}{ci,sɯpa}{ⓔsɯpaⓗ1ⓝci,sɯpa}\sens{1}
\begin{définition}\pfra{mettre ensemble}\end{définition}
\begin{définition}\pcmn{加起来}\end{définition}
\begin{exemple}\pjya{jɯfɕɯr aʑo ʁnɯ-ɣjɤn kɤ-rŋgɯ-a tɕe, ci tú-wɣ-sɯpa tɕe tɯtshot kɯngɯt jamar pɯ-nɯʑɯβ-a}\hspace{5pt}\pcmn{我昨天睡了两次,加起来一共睡了大概九个小时。}\end{exemple}\end{sous-entrée}

\sens{2}
\begin{définition}\pfra{marier}\end{définition}
\begin{définition}\pcmn{让……结婚}\end{définition}\relationsémantique{Component 1}{\lien{ⓔciⓗ2}{ci}}\relationsémantique{Component 2}{\lien{ⓔsɯpaⓗ2}{sɯpa}}\end{entrée}

\begin{entrée}{sɯpɣo}{}{ⓔsɯpɣo} 
\classe{n} 
\begin{définition}\pfra{une pile de bois}\end{définition}
\begin{définition}\pcmn{柴垛子【柴码子】}\end{définition}\end{entrée}

\begin{entrée}{sɯphɯ}{}{ⓔsɯphɯ} 
\classe{n} 
\begin{définition}\pfra{arbre}\end{définition}
\begin{définition}\pcmn{树}\end{définition}\étymologie{ɕiŋ}\end{entrée}

\begin{entrée}{sɯphɯt}{}{ⓔsɯphɯt} 
\classe{n} 
\begin{définition}\pfra{fait de couper du bois}\end{définition}
\begin{définition}\pcmn{砍柴}\end{définition}
\begin{exemple}\pjya{qartsɯ tɕe, sɯphɯt kɤ-βzu ra}\hspace{5pt}\pcmn{到了冬天要砍柴}\end{exemple}
\begin{exemple}\pjya{a-βɣo sɯphɯt wuma ʑo χɕu}\hspace{5pt}\pcmn{我的叔叔砍柴很厉害}\end{exemple}\relationsémantique{参考}{\lien{ⓔsiⓗ1}{si₁}}\relationsémantique{参考}{\lien{ⓔphɯt}{phɯt}}\relationsémantique{参考}{\lien{ⓔɣɯsɯphɯt}{ɣɯsɯphɯt}}\end{entrée}

\begin{entrée}{sɯprɤt}{}{ⓔsɯprɤt}\relationsémantique{参考}{\lien{ⓔprɤt}{prɤt}}\end{entrée}

\begin{entrée}{sɯqartsɯ}{}{ⓔsɯqartsɯ} 
\classe{vl} \paradigme{dir}{tɤ-}
\begin{définition}\pfra{donner un coup de pied}\end{définition}
\begin{définition}\pcmn{踢(用后肢)}\end{définition}
\begin{exemple}\pjya{jla ɲɯ-sɯqartsɯ (=tɯ-qartsɯ ta-lɤt)}\hspace{5pt}\pcmn{犏牛在踢}\end{exemple}
\begin{exemple}\pjya{mdzadi tɤ-sɯqartsɯ tɕe tɯ-mɯ mɤ-ɕaβ}\hspace{5pt}\pcmn{跳蚤再踢也登不上天(你有再大的能力也影响不了人家)}\end{exemple}\relationsémantique{同义词}{\lien{ⓔsɯlaʁrdɤβ}{sɯlaʁrdɤβ}}\relationsémantique{参考}{\lien{}{tɯ-qartsɯ}}\end{entrée}

\begin{entrée}{sɯrɤt}{}{ⓔsɯrɤt}\relationsémantique{参考}{\lien{ⓔrɤt}{rɤt}}\end{entrée}

\begin{entrée}{sɯrdɤl}{}{ⓔsɯrdɤl}\relationsémantique{参考}{\lien{ⓔrdɤl}{rdɤl}}\end{entrée}

\begin{entrée}{sɯrɟɯɣ}{}{ⓔsɯrɟɯɣ}\relationsémantique{参考}{\lien{ⓔrɟɯɣⓗ2}{rɟɯɣ}}\end{entrée}

\begin{entrée}{sɯrku}{}{ⓔsɯrku}\relationsémantique{参考}{\lien{ⓔrku}{rku}}\end{entrée}

\begin{entrée}{sɯrkɤz}{}{ⓔsɯrkɤz} 
\classe{n} 
\begin{définition}\pfra{gravures sur bois}\end{définition}
\begin{définition}\pcmn{刻的木板}\end{définition}
\begin{exemple}\pjya{sɯrkɤz ta-βzu}\hspace{5pt}\pcmn{他刻了木板}\end{exemple}\end{entrée}

\begin{entrée}{sɯrma}{}{ⓔsɯrma} 
\classe{vt} \paradigme{dir}{kɤ-}\paradigme{dir}{pɯ-}\sens{1}
\begin{définition}\pfra{laisser habiter}\end{définition}
\begin{définition}\pcmn{留宿}\end{définition}
\begin{exemple}\pjya{jɯɣmɯr kutɕu kɤ-sɯrma-t-a}\hspace{5pt}\pcmn{今天晚上让他在这里睡了}\end{exemple}\sens{2}
\begin{définition}\pfra{enterrer}\end{définition}
\begin{définition}\pcmn{掩埋}\end{définition}
\begin{exemple}\pjya{smi ka-sɯrma}\hspace{5pt}\pcmn{(用草木灰)把火盖住了}\end{exemple}\relationsémantique{同义词}{\lien{ⓔsɯrʑaʁ}{sɯrʑaʁ}}\end{entrée}

\begin{entrée}{sɯrmbɣotɯm}{}{ⓔsɯrmbɣotɯm} 
\classe{vi} 
\begin{définition}\pfra{s'asseoir en tailleur}\end{définition}
\begin{définition}\pcmn{缠腿坐}\end{définition}\relationsémantique{同义词}{\lien{ⓔsɯχcoŋkroŋ}{sɯχcoŋkroŋ}}\end{entrée}

\begin{entrée}{sɯrna}{₁}{ⓔsɯrnaⓗ1} 
\classe{n} 
\begin{définition}\pfra{figurine en argile}\end{définition}
\begin{définition}\pcmn{泥偶}\end{définition}\end{entrée}

\begin{entrée}{sɯrna}{₂}{ⓔsɯrnaⓗ2} 
\classe{n} 
\begin{définition}\pfra{champignon noir}\end{définition}
\begin{définition}\pcmn{木耳}\end{définition}\relationsémantique{参考}{\lien{ⓔsiⓗ1}{si₁}}\relationsémantique{参考}{\lien{ⓔtɯ-rna}{tɯ-rna}}\end{entrée}

\begin{entrée}{sɯrna}{₃}{ⓔsɯrnaⓗ3} 
\classe{n} 
\begin{définition}\pfra{bélier}\end{définition}
\begin{définition}\pcmn{公绵羊}\end{définition}\end{entrée}

\begin{entrée}{sɯrpjɯ}{}{ⓔsɯrpjɯ}\relationsémantique{参考}{\lien{ⓔrpjɯ}{rpjɯ}}\end{entrée}

\begin{entrée}{sɯrsɯr}{}{ⓔsɯrsɯr} 
\classe{idph.2} 
\begin{définition}\pfra{rond}\end{définition}
\begin{définition}\pcmn{形容圆形}\end{définition}
\begin{exemple}\pjya{ɲɯ-ɤrtɯm sɯrsɯr ʑo}\hspace{5pt}\pcmn{圆溜溜的}\end{exemple}
\begin{exemple}\pjya{lɯlu cho khɯna kɤ-nɯ-rŋgɯ-nɯ tɕe ku-ortɯm-nɯ sɯrsɯr ʑo ŋu}\hspace{5pt}\pcmn{猫和狗睡觉的时候蜷成一团}\end{exemple}
\begin{sous-entrée}{sɯrinɤsɯri}{ⓔsɯrsɯrⓝsɯrinɤsɯri} 
\classe{idph.8} 
\begin{définition}\pfra{qui tourne vite}\end{définition}
\begin{définition}\pcmn{形容旋转得很快的样子}\end{définition}
\begin{exemple}\pjya{sɯrinɤsɯri ʑo ɲɯ-mtɕɯr}\hspace{5pt}\pcmn{转得飞快}\end{exemple}\relationsémantique{同义词}{\lien{ⓔxɯrxɯr}{xɯrxɯr}}\end{sous-entrée}

\end{entrée}

\begin{entrée}{sɯrtaʁ}{}{ⓔsɯrtaʁ} 
\classe{n} 
\begin{définition}\pfra{branche}\end{définition}
\begin{définition}\pcmn{树枝}\end{définition}\relationsémantique{参考}{\lien{ⓔtɤ-rtaʁ}{tɤ-rtaʁ}}\end{entrée}

\begin{entrée}{sɯrtoʁ}{}{ⓔsɯrtoʁ}\relationsémantique{参考}{\lien{ⓔrtoʁ}{rtoʁ}}\end{entrée}

\begin{entrée}{sɯrtsho}{}{ⓔsɯrtsho} 
\classe{n} 
\begin{définition}\pfra{partie ouverte de la souche}\end{définition}
\begin{définition}\pcmn{树墩(被锯掉的)表面}\end{définition}\end{entrée}

\begin{entrée}{sɯrtshɯm}{}{ⓔsɯrtshɯm} 
\classe{n} 
\begin{définition}\pfra{souche}\end{définition}
\begin{définition}\pcmn{树墩;树桩}\end{définition}\end{entrée}

\begin{entrée}{sɯrtsi}{}{ⓔsɯrtsi} 
\classe{vt} \paradigme{dir}{tɤ-}\paradigme{dir}{thɯ-}
\begin{définition}\pfra{appliquer de la laque}\end{définition}
\begin{définition}\pcmn{上漆}\end{définition}
\begin{exemple}\pjya{tɕoχtsi thɯ-sɯrtsi-t-a}\hspace{5pt}\pcmn{我给桌子上了漆}\end{exemple}
\begin{exemple}\pjya{to-sɯrtsi (ɯ-rtsi to-lɤt)}\hspace{5pt}\pcmn{他上了漆}\end{exemple}\relationsémantique{参考}{\lien{ⓔɯ-rtsi}{ɯ-rtsi}}\end{entrée}

\begin{entrée}{sɯrʑaʁ}{}{ⓔsɯrʑaʁ} 
\classe{vt} \paradigme{dir}{kɤ-}\paradigme{dir}{pɯ-}
\begin{définition}\pfra{enterrer}\end{définition}
\begin{définition}\pcmn{掩埋}\end{définition}
\begin{exemple}\pjya{smi thɤlwa kɯ ka-sɯrʑaʁ}\hspace{5pt}\pcmn{他用草木灰把火盖住了}\end{exemple}
\begin{exemple}\pjya{lɯlu kɯ ɯ-qe ka-sɯrʑaʁ}\hspace{5pt}\pcmn{猫把它自己的屎(用土)盖住了}\end{exemple}\relationsémantique{同义词}{\lien{ⓔsɯrma}{sɯrma}}\end{entrée}

\begin{entrée}{sɯʁaʁ}{}{ⓔsɯʁaʁ}\relationsémantique{参考}{\lien{ⓔʁaʁ}{ʁaʁ}}\end{entrée}

\begin{entrée}{sɯʁejlu}{}{ⓔsɯʁejlu} 
\classe{vi} \paradigme{dir}{tɤ-}
\begin{définition}\pfra{être gaucher}\end{définition}
\begin{définition}\pcmn{左撇子}\end{définition}
\begin{exemple}\pjya{jiɕqha nɯ ɲɯ-sɯʁejlu}\hspace{5pt}\pcmn{那个人是左撇子}\end{exemple}\relationsémantique{参考}{\lien{ⓔʁejlu}{ʁejlu}}\end{entrée}

\begin{entrée}{sɯʁjit}{}{ⓔsɯʁjit} 
\classe{vt} \paradigme{dir}{tɤ-}\sens{1}
\begin{définition}\pfra{se souvenir}\end{définition}
\begin{définition}\pcmn{记得;想起}\end{définition}
\begin{exemple}\pjya{mɤ-tɯ-sɯʁjit ʑo maŋe}\hspace{5pt}\pcmn{没有你想不出来的事情}\end{exemple}\sens{2}
\begin{définition}\pfra{manquer à}\end{définition}
\begin{définition}\pcmn{想念}\end{définition}
\begin{exemple}\pjya{ɯ-kha ra to-sɯʁjit}\hspace{5pt}\pcmn{他想念他的家属}\end{exemple}
\begin{exemple}\pjya{ɯ-mu to-sɯʁjit}\hspace{5pt}\pcmn{他想念他母亲}\end{exemple}
\begin{exemple}\pjya{tɤ́-wɣ-sɯʁjit-a}\hspace{5pt}\pcmn{他想念我了}\end{exemple}\relationsémantique{参考}{\lien{ⓔʁjit}{ʁjit}}\end{entrée}

\begin{entrée}{sɯʁjoʁ}{}{ⓔsɯʁjoʁ} 
\classe{vt}  
\grammaire{denom} \paradigme{dir}{pɯ-}
\begin{définition}\pfra{se servir (d'un tissu) pour confectionner la couche extérieure d'un vêtement}\end{définition}
\begin{définition}\pcmn{做成衣服的外层}\end{définition}
\begin{exemple}\pjya{raz kɯ-ɣɯrni kɯ χpɯn ɯ-ŋga ɯ-smɤʁjoʁ nɯ pjɯ́-wɣ-sɯʁjoʁ ra}\hspace{5pt}\pcmn{要用红色的布料制作和尚衣服的外层}\end{exemple}\relationsémantique{参考}{\lien{ⓔɯ-ʁjoʁ}{ɯ-ʁjoʁ}}\end{entrée}

\begin{entrée}{sɯʁndzɤr}{}{ⓔsɯʁndzɤr}\relationsémantique{参考}{\lien{ⓔʁndzɤr}{ʁndzɤr}}\end{entrée}

\begin{entrée}{sɯʁnɯ}{}{ⓔsɯʁnɯ}\relationsémantique{参考}{\lien{ⓔʁnɯ}{ʁnɯ}}\end{entrée}

\begin{entrée}{sɯsu}{}{ⓔsɯsu} 
\classe{vi}  
\grammaire{caus} \paradigme{dir}{tɤ-}\paradigme{dir}{tɤ-}
\begin{définition}\pfra{vivant}\end{définition}
\begin{définition}\pcmn{活}\end{définition}
\begin{exemple}\pjya{ɯ-kɯ-mɲɤm wuma ʑo pjɤ-thɯ ri, to-sɯsu}\hspace{5pt}\pcmn{他差一点死了,又活过来了}\end{exemple}
\begin{sous-entrée}{sɯsɯsu}{ⓔsɯsuⓝsɯsɯsu} 
\classe{vt} \end{sous-entrée}

\begin{définition}\pfra{faire revivre}\end{définition}
\begin{définition}\pcmn{让……复活}\end{définition}\end{entrée}

\begin{entrée}{sɯsat}{}{ⓔsɯsat}\relationsémantique{参考}{\lien{ⓔsat}{sat}}\end{entrée}

\begin{entrée}{sɯsaχsɤl}{}{ⓔsɯsaχsɤl}\relationsémantique{参考}{\lien{}{saχsɤl₁}}\end{entrée}

\begin{entrée}{sɯsɤɕqali}{}{ⓔsɯsɤɕqali}\relationsémantique{参考}{\lien{ⓔɣɤɕqali}{ɣɤɕqali}}\end{entrée}

\begin{entrée}{sɯschɤt}{}{ⓔsɯschɤt}\relationsémantique{参考}{\lien{ⓔschɤt}{schɤt}}\end{entrée}

\begin{entrée}{sɯsci}{}{ⓔsɯsci}\relationsémantique{参考}{\lien{ⓔsciⓗ2}{sci}}\end{entrée}

\begin{entrée}{sɯskɯrma}{}{ⓔsɯskɯrma} 
\classe{vt}  
\grammaire{denom} \paradigme{dir}{\_}
\begin{définition}\pfra{demander à quelqu'un d'emporter un cadeau à quelqu'un}\end{définition}
\begin{définition}\pcmn{请人带礼物}\end{définition}
\begin{exemple}\pjya{jɤ-ta-sɯskɯrma nɯ nɤ-jaʁ ɯ-jɤ-ázɣɯt?}\hspace{5pt}\pcmn{我送给你的礼物,你收到了没有}\end{exemple}\relationsémantique{参考}{\lien{ⓔskɯrma}{skɯrma}}\end{entrée}

\begin{entrée}{sɯsloʁ}{}{ⓔsɯsloʁ}\relationsémantique{参考}{\lien{ⓔsloʁ}{sloʁ}}\end{entrée}

\begin{entrée}{sɯsɲu}{}{ⓔsɯsɲu}\relationsémantique{参考}{\lien{ⓔsɲu}{sɲu}}\end{entrée}

\begin{entrée}{sɯsŋa}{}{ⓔsɯsŋa}\relationsémantique{参考}{\lien{ⓔsŋa}{sŋa}}\end{entrée}

\begin{entrée}{sɯsŋaʁ}{}{ⓔsɯsŋaʁ}\relationsémantique{参考}{\lien{ⓔsŋaʁⓗ1}{sŋaʁ₁}}\end{entrée}

\begin{entrée}{sɯso}{}{ⓔsɯso} 
\classe{vt} \paradigme{dir}{nɯ-}\sens{1}
\begin{définition}\pfra{penser}\end{définition}
\begin{définition}\pcmn{想}\end{définition}
\begin{exemple}\pjya{@dianhua ɯ-kɯ-lɤt mataŋe tɕe tɕhindʐa kɯ-ŋu kɯ nɯ-sɯso-t-a}\hspace{5pt}\pcmn{你没有打电话,我想了怎么回事}\end{exemple}
\begin{exemple}\pjya{aʑo a-kɤ-sɯso nɯ ʑo ŋu}\hspace{5pt}\pcmn{这就是我的意思}\end{exemple}
\begin{exemple}\pjya{ɯ-kɤ-sɯso ɲɯ-dɤn}\hspace{5pt}\pcmn{他想的事情很多}\end{exemple}
\begin{exemple}\pjya{``aʑo kɯ-fse kɯ-mpɕɤr me" ɲɯ-nɯ-sɯsɤm pjɤ-ŋu}\hspace{5pt}\pcmn{他想着,没有比我漂亮的人}\end{exemple}\sens{2}
\begin{définition}\pfra{vouloir}\end{définition}
\begin{définition}\pcmn{想要}\end{définition}
\begin{exemple}\pjya{laχtɕha kɤ-ntsɣe ɲɯ-sɯsɤm}\hspace{5pt}\pcmn{他想卖东西}\end{exemple}\sens{3}
\begin{définition}\pfra{manquer}\end{définition}
\begin{définition}\pcmn{想念}\end{définition}
\begin{sous-entrée}{kɤsɯso}{ⓔsɯsoⓢ3ⓝkɤsɯso}
\begin{définition}\pfra{pensant, se disant que}\end{définition}
\begin{définition}\pcmn{想着;等到……的时候}\end{définition}
\begin{exemple}\pjya{nɯ jamar rtaʁ kɤsɯso tɕe, a-tɤ-tɯ-z-nɯne ra}\hspace{5pt}\pcmn{(你估计)够的时候就要停}\end{exemple}\end{sous-entrée}

\begin{sous-entrée}{ɲɯkɯsɯso/\variante{pjɯkɯsɯso}}{ⓔsɯsoⓢ3ⓝɲɯkɯsɯso} 
\classe{adv} 
\begin{définition}\pfra{en comparaison avec...}\end{définition}
\begin{définition}\pcmn{比起……}\end{définition}
\begin{exemple}\pjya{nɤʑo ɲɯ-kɯ-sɯso tɕe, ɯʑo kɯ-wxtɯ-wxti ɕti}\hspace{5pt}\pcmn{跟你比起来,他长得很大}\end{exemple}
\begin{exemple}\pjya{aʑo pjɯ-kɯ-sɯso tɕe nɤʑo ʁo ɲɯ-tɯ-mbro}\hspace{5pt}\pcmn{跟我比起来,你倒是长得很高}\end{exemple}\end{sous-entrée}

\end{entrée}

\begin{entrée}{sɯspa}{}{ⓔsɯspa}\relationsémantique{参考}{\lien{ⓔspa}{spa}}\end{entrée}

\begin{entrée}{sɯsphjaʁ}{}{ⓔsɯsphjaʁ}\relationsémantique{参考}{\lien{ⓔsphjaʁ}{sphjaʁ}}\end{entrée}

\begin{entrée}{sɯspoʁ}{}{ⓔsɯspoʁ} 
\classe{vt} \paradigme{dir}{\_}\paradigme{dir}{\_}
\begin{définition}\pfra{faire un trou}\end{définition}
\begin{définition}\pcmn{穿孔}\end{définition}
\begin{définition}\pfra{faire un trou avec....}\end{définition}
\begin{définition}\pcmn{用……来打洞}\end{définition}
\begin{exemple}\pjya{tɯ-ŋga ko-sɯspoʁ}\hspace{5pt}\pcmn{他把衣服戳了个洞}\end{exemple}
\begin{exemple}\pjya{tɤ-fkɯm ko-sɯspoʁ}\hspace{5pt}\pcmn{他把口袋戳了个洞}\end{exemple}
\begin{exemple}\pjya{khoxtu pjɤ-sɯspoʁ}\hspace{5pt}\pcmn{他把屋顶戳了个洞}\end{exemple}
\begin{exemple}\pjya{rdɤstaʁ jɤ-lat-a tɕe, ɯ-ku kɤ-sɯspoʁ-a}\hspace{5pt}\pcmn{我扔了一块石头,把他的头打伤了}\end{exemple}
\begin{exemple}\pjya{ndzrɯ kɯ pɯ-sɯ-sɯspoʁ-a}\hspace{5pt}\pcmn{我用凿子打了洞}\end{exemple}\relationsémantique{参考}{\lien{ⓔspoʁ}{spoʁ}}
\begin{sous-entrée}{sɯsɯspoʁ}{ⓔsɯspoʁⓝsɯsɯspoʁ} 
\classe{vt} \end{sous-entrée}

\end{entrée}

\begin{entrée}{sɯstu}{}{ⓔsɯstu}\relationsémantique{参考}{\lien{ⓔstuⓗ2}{stu₂}}\end{entrée}

\begin{entrée}{sɯsta}{}{ⓔsɯsta}\relationsémantique{参考}{\lien{ⓔsta}{sta}}\end{entrée}

\begin{entrée}{sɯstɤm}{}{ⓔsɯstɤm}\relationsémantique{参考}{\lien{ⓔstɤm}{stɤm}}\end{entrée}

\begin{entrée}{sɯsɯsu}{}{ⓔsɯsɯsu}\relationsémantique{参考}{\lien{ⓔsɯsu}{sɯsu}}\end{entrée}

\begin{entrée}{sɯsɯspoʁ}{}{ⓔsɯsɯspoʁ}\relationsémantique{参考}{\lien{ⓔsɯspoʁ}{sɯspoʁ}}\end{entrée}

\begin{entrée}{sɯta}{}{ⓔsɯta} 
\classe{vt} \paradigme{dir}{nɯ-}
\begin{définition}\pfra{détacher}\end{définition}
\begin{définition}\pcmn{解开}\end{définition}
\begin{exemple}\pjya{nɯki ɲɤ-raʁ tɕe nɯ-sɯte}\hspace{5pt}\pcmn{线卡住了,你解开吧}\end{exemple}
\begin{exemple}\pjya{tɤ-ri nɯ-kɯ-raʁ nɯ-sɯta-t-a tɕe, kɤ-rɯkɤtɯm jɤɣ}\hspace{5pt}\pcmn{我把缠了的线解开了,可以牵线了}\end{exemple}\relationsémantique{参考}{\lien{ⓔsɯɕlɯɣ}{sɯɕlɯɣ}}\end{entrée}

\begin{entrée}{sɯtɤpɯz}{}{ⓔsɯtɤpɯz} 
\classe{n} 
\begin{définition}\pfra{bois pourri}\end{définition}
\begin{définition}\pcmn{朽木}\end{définition}\end{entrée}

\begin{entrée}{sɯtɕɤt}{}{ⓔsɯtɕɤt}\relationsémantique{参考}{\lien{ⓔtɕɤt}{tɕɤt}}\end{entrée}

\begin{entrée}{sɯtɕhaʁ}{}{ⓔsɯtɕhaʁ} 
\classe{n} 
\begin{définition}\pfra{se rétrécir (bois)}\end{définition}
\begin{définition}\pcmn{收缩(木头)}\end{définition}
\begin{exemple}\pjya{sɯtɕhaʁ ko-ɕe}\hspace{5pt}\pcmn{木头收缩了}\end{exemple}\relationsémantique{参考}{\lien{ⓔsiⓗ1}{si₁}}\relationsémantique{参考}{\lien{ⓔtɕhaʁ}{tɕhaʁ}}\end{entrée}

\begin{entrée}{sɯtɕɯn}{}{ⓔsɯtɕɯn} 
\classe{n} 
\begin{définition}\pfra{grande forêt}\end{définition}
\begin{définition}\pcmn{大森林}\end{définition}\relationsémantique{同义词}{\lien{}{sɯŋgɯ kɯ-rnaʁ}}\end{entrée}

\begin{entrée}{sɯtɕɯnjmɤɣ}{}{ⓔsɯtɕɯnjmɤɣ} 
\classe{n} 
\begin{définition}\pfra{russule rouge}\end{définition}
\begin{définition}\pcmn{红菇【杉木菌】}\end{définition}\relationsémantique{同义词}{\lien{ⓔjmɤɣni}{jmɤɣni}}\end{entrée}

\begin{entrée}{sɯti}{}{ⓔsɯti}\relationsémantique{参考}{\lien{ⓔti}{ti}}\end{entrée}

\begin{entrée}{sɯtsu}{}{ⓔsɯtsu}\relationsémantique{参考}{\lien{ⓔtsu}{tsu}}\end{entrée}

\begin{entrée}{sɯtsɯm}{}{ⓔsɯtsɯm}\relationsémantique{参考}{\lien{ⓔtsɯm}{tsɯm}}\end{entrée}

\begin{entrée}{sɯxcha}{}{ⓔsɯxcha} 
\classe{vt}  
\grammaire{habil} \paradigme{dir}{tɤ-}
\begin{définition}\pfra{être capable}\end{définition}
\begin{définition}\pcmn{有能力}\end{définition}
\begin{exemple}\pjya{ɲɯ-rʑi tɕe, mɤ-tɯ́-wɣ-sɯxcha}\hspace{5pt}\pcmn{很重,你不行(你抬不动)}\end{exemple}
\begin{exemple}\pjya{kɤ-fkur mɯ́j-wɣ-sɯxcha}\hspace{5pt}\pcmn{他背不起}\end{exemple}
\begin{exemple}\pjya{nɤki nɯ ʁo mɯ́j-ɴqa tɕe tɯ́-wɣ-sɯxcha loβ}\hspace{5pt}\pcmn{这个倒是没有很难,你能行吧}\end{exemple}\relationsémantique{参考}{\lien{ⓔchaⓗ1}{cha₁}}\end{entrée}

\begin{entrée}{sɯxchi}{}{ⓔsɯxchi}\relationsémantique{参考}{\lien{ⓔchi}{chi}}\end{entrée}

\begin{entrée}{sɯxcɯ}{}{ⓔsɯxcɯ}\relationsémantique{参考}{\lien{ⓔɯ-lu,cɯ}{ɯ-lu,cɯ}}\end{entrée}

\begin{entrée}{sɯxɕɤt}{}{ⓔsɯxɕɤt} 
\classe{vt}  
\grammaire{recip}
\grammaire{refl} \sens{1}\paradigme{dir}{pɯ-}\paradigme{dir}{kɤ-}
\begin{définition}\pfra{enseigner}\end{définition}
\begin{définition}\pcmn{教}\end{définition}
\begin{exemple}\pjya{kɤ-taʁ pɯ-sɯxɕat-a}\hspace{5pt}\pcmn{我教他织布了}\end{exemple}
\begin{exemple}\pjya{pɯ-sɯxɕat-a tɕe kɤ-sɯspa-t-a}\hspace{5pt}\pcmn{我教会了他}\end{exemple}\sens{2}\paradigme{dir}{nɯ-}
\begin{définition}\pfra{habituer à}\end{définition}
\begin{définition}\pcmn{训练,令……养成习惯}\end{définition}
\begin{exemple}\pjya{kɤ-nɤma nɯ-sɯxɕat-a}\hspace{5pt}\pcmn{我训练了他}\end{exemple}
\begin{exemple}\pjya{kɤ-nɯndzɤmdɯm nɯ-ta-sɯxɕɤt}\hspace{5pt}\pcmn{我让你养成吃零食的习惯}\end{exemple}\sens{3}\paradigme{dir}{tɤ-}\paradigme{dir}{pɯ-}\paradigme{dir}{pɯ-}
\begin{définition}\pfra{indiquer}\end{définition}
\begin{définition}\pcmn{指点}\end{définition}
\begin{définition}\pfra{enseigner}\end{définition}
\begin{définition}\pcmn{教书}\end{définition}
\begin{exemple}\pjya{rgɯnba sɤxɕe ɣɯ ɯ-tʂu nɯ tɤ-sɯxɕat-a}\hspace{5pt}\pcmn{我给他指点了去寺庙的路}\end{exemple}
\begin{exemple}\pjya{tɕhi pjɯ-tɯ-sɤsɯxɕɤt ŋu?}\hspace{5pt}\pcmn{你教什么?}\end{exemple}
\begin{exemple}\pjya{tɤrtsɯz pjɯ-sɤsɯxɕat-a ŋu}\hspace{5pt}\pcmn{你教数学}\end{exemple}
\begin{sous-entrée}{sɤsɯxɕɤt}{ⓔsɯxɕɤtⓢ3ⓝsɤsɯxɕɤt} 
\classe{vi}  
\grammaire{apass} \end{sous-entrée}

\begin{sous-entrée}{asɯxɕɯxɕɤt}{ⓔsɯxɕɤtⓢ3ⓝasɯxɕɯxɕɤt} 
\classe{vi} \end{sous-entrée}

\paradigme{dir}{nɯ-}
\begin{définition}\pfra{s'enseigner les uns aux autres}\end{définition}
\begin{définition}\pcmn{互相教}\end{définition}
\begin{sous-entrée}{ʑɣɤsɯxɕɤt}{ⓔsɯxɕɤtⓝʑɣɤsɯxɕɤt} 
\classe{vi} \end{sous-entrée}

\begin{définition}\pfra{s'entraîner}\end{définition}
\begin{définition}\pcmn{训练自己}\end{définition}
\begin{exemple}\pjya{ɲɯ-ʑɣɤsɯxɕat-a}\hspace{5pt}\pcmn{我自己训练}\end{exemple}\relationsémantique{同义词}{\lien{ⓔsɯɣʑaʁ}{sɯɣʑaʁ}}\relationsémantique{参考}{\lien{ⓔɕɤt}{ɕɤt}}\end{entrée}

\begin{entrée}{sɯxɕe}{}{ⓔsɯxɕe} 
\classe{vt} \paradigme{dir}{\_}\paradigme{past stem}{sɤɣri}
\begin{définition}\pfra{envoyer qqn}\end{définition}
\begin{définition}\pcmn{派人}\end{définition}
\begin{exemple}\pjya{kɯ-nɤphɯphu jɤ-sɤɣri-t-a (jɤ-no-t-a)}\hspace{5pt}\pcmn{我把乞丐赶走了}\end{exemple}
\begin{exemple}\pjya{ki kɯ-fse tɤ-rʑaʁ ɲɯ-sɯxɕe-a ɬoʁ}\hspace{5pt}\pcmn{我只有这样打发时间}\end{exemple}\relationsémantique{参考}{\lien{ⓔɕe}{ɕe}}\relationsémantique{参考}{\lien{ⓔɯ-pa,ɕeⓝɯ-pa,sɯxɕe}{ɯ-pa,sɯxɕe}}\end{entrée}

\begin{entrée}{sɯxsa}{}{ⓔsɯxsa}\relationsémantique{参考}{\lien{ⓔsa}{sa}}\end{entrée}

\begin{entrée}{sɯxso}{}{ⓔsɯxso}\relationsémantique{参考}{\lien{ⓔso}{so}}\end{entrée}

\begin{entrée}{sɯxtar}{}{ⓔsɯxtar}\relationsémantique{参考}{\lien{ⓔtar}{tar}}\end{entrée}

\begin{entrée}{sɯxtɕhaʁ}{}{ⓔsɯxtɕhaʁ} 
\classe{vt} \paradigme{dir}{pɯ-}\paradigme{dir}{nɯ-}
\begin{définition}\pfra{faire diminuer}\end{définition}
\begin{définition}\pcmn{减}\end{définition}
\begin{exemple}\pjya{pɯ-sɯxtɕhaʁ-a}\hspace{5pt}\pcmn{我减了}\end{exemple}
\begin{exemple}\pjya{@laoban kɯ a-ŋgra pjɤ-sɯxtɕhaʁ}\hspace{5pt}\pcmn{老板把我的工资给减少了}\end{exemple}
\begin{exemple}\pjya{sqi ɯ-ngɯ kɯtʂɤɣ chɯ́-wɣ-sɯxtɕhaʁ tɕe, ɯ-ro kɯβde ɲɯ-ri ŋu}\hspace{5pt}\pcmn{十减六等于四}\end{exemple}\relationsémantique{同义词}{\lien{ⓔɣɤtɕhaʁ}{ɣɤtɕhaʁ}}\relationsémantique{参考}{\lien{ⓔtɕhaʁ}{tɕhaʁ}}
\begin{sous-entrée}{ʑɣɤsɯxtɕhaʁ}{ⓔsɯxtɕhaʁⓝʑɣɤsɯxtɕhaʁ} 
\classe{vi}  
\grammaire{refl}
\grammaire{caus} 
\begin{définition}\pfra{diminuer de lui-même}\end{définition}
\begin{définition}\pcmn{令自己变少}\end{définition}
\begin{exemple}\pjya{pɕawtsɯ ɯʑo ndɤre mɤ-nɯ-ʑɣɤsɯxtɕhaʁ}\hspace{5pt}\pcmn{钱不会自己变少}\end{exemple}\end{sous-entrée}

\étymologie{tɕʰag}\end{entrée}

\begin{entrée}{sɯxtɕhɤt}{}{ⓔsɯxtɕhɤt}\relationsémantique{参考}{\lien{ⓔtɕhɤtⓗ1}{tɕhɤt₁}}\end{entrée}

\begin{entrée}{sɯxtɕhɯt}{}{ⓔsɯxtɕhɯt}\relationsémantique{参考}{\lien{ⓔtɕhɯt}{tɕhɯt}}\end{entrée}

\begin{entrée}{sɯxtsu/\variante{sɯtsu}}{₁}{ⓔsɯxtsuⓗ1} 
\classe{vt}  
\grammaire{caus} 
\begin{définition}\pfra{laisser du temps}\end{définition}
\begin{définition}\pcmn{给别人时间}\end{définition}
\begin{exemple}\pjya{ɲɯ́-wɣ-sɯtsu-a-nɯ}\hspace{5pt}\pcmn{他们给我时间}\end{exemple}
\begin{exemple}\pjya{jisŋi aʑo ju-ɣi-a jɤɣ ma ɲɤ́-wɣ-sɯtsu-a-nɯ}\hspace{5pt}\pcmn{我今天可以来,因为他们给我时间}\end{exemple}\relationsémantique{参考}{\lien{ⓔtsu}{tsu}}\end{entrée}

\begin{entrée}{sɯxtsu}{₂}{ⓔsɯxtsuⓗ2} 
\classe{vt} \paradigme{dir}{nɯ-}
\begin{définition}\pfra{faire fermenter}\end{définition}
\begin{définition}\pcmn{使发酵}\end{définition}\relationsémantique{参考}{\lien{ⓔxtsu}{xtsu}}\end{entrée}

\begin{entrée}{sɯxtshu}{}{ⓔsɯxtshu}\relationsémantique{参考}{\lien{ⓔtshu}{tshu}}\end{entrée}

\begin{entrée}{sɯxtshaʁ}{}{ⓔsɯxtshaʁ} 
\classe{vt}  
\grammaire{denom} \paradigme{dir}{pɯ-}
\begin{définition}\pfra{tamiser}\end{définition}
\begin{définition}\pcmn{筛}\end{définition}
\begin{exemple}\pjya{tɯjpu pɯ-sɯxtshaʁ-a}\hspace{5pt}\pcmn{我筛了粮食}\end{exemple}
\begin{exemple}\pjya{thɤlwa pɯ-sɯxtshaʁ-a}\hspace{5pt}\pcmn{我筛了灰}\end{exemple}
\begin{exemple}\pjya{tɤɕi nɯ pɯ-sɯxtshaʁ-a}\hspace{5pt}\pcmn{我筛了青稞}\end{exemple}\relationsémantique{同义词}{\lien{ⓔsɯɕɯɣra}{sɯɕɯɣra}}\relationsémantique{参考}{\lien{ⓔtshaʁ}{tshaʁ}}\end{entrée}

\begin{entrée}{sɯxtshoz}{}{ⓔsɯxtshoz} 
\classe{vt} \paradigme{dir}{tɤ-}
\begin{définition}\pfra{avoir au complet}\end{définition}
\begin{définition}\pcmn{具备齐全;一个也没有漏掉}\end{définition}
\begin{exemple}\pjya{laʁdɯn tɤ-sɯxtshoz-a tɕe, kɤ-rɤma khɯ}\hspace{5pt}\pcmn{农具具备齐全,可以工作了}\end{exemple}\relationsémantique{同义词}{\lien{ⓔɣɤtshoz}{ɣɤtshoz}}\relationsémantique{参考}{\lien{ⓔtshoz}{tshoz}}\étymologie{tsʰaŋs}\end{entrée}

\begin{entrée}{sɯxtshwi}{}{ⓔsɯxtshwi} 
\classe{vt}  
\grammaire{denom} \paradigme{dir}{pɯ-}
\begin{définition}\pfra{teindre}\end{définition}
\begin{définition}\pcmn{染}\end{définition}
\begin{exemple}\pjya{tɤ-ri pjɤ-sɯxtshwi}\hspace{5pt}\pcmn{他染了线}\end{exemple}
\begin{exemple}\pjya{tɯ-ŋga pjɤ-sɯxtshwi}\hspace{5pt}\pcmn{他染了衣服}\end{exemple}
\begin{exemple}\pjya{raz pɯ-sɯxtshwi-t-a / tshwi pɯ-lat-a}\hspace{5pt}\pcmn{我染了布料}\end{exemple}\relationsémantique{参考}{\lien{ⓔtshwi}{tshwi}}\étymologie{tsʰos}\end{entrée}

\begin{entrée}{sɯxtso}{}{ⓔsɯxtso}\relationsémantique{参考}{\lien{ⓔtso}{tso}}\end{entrée}

\begin{entrée}{sɯxtsɯɣ}{}{ⓔsɯxtsɯɣ}\relationsémantique{参考}{\lien{ⓔxtsɯɣ}{xtsɯɣ}}\end{entrée}

\begin{entrée}{sɯxtʂaŋ}{}{ⓔsɯxtʂaŋ}\relationsémantique{参考}{\lien{ⓔtʂaŋ}{tʂaŋ}}\end{entrée}

\begin{entrée}{sɯxtʂɯn}{}{ⓔsɯxtʂɯn} 
\classe{vt} \paradigme{dir}{pɯ-}
\begin{définition}\pfra{faire bénéficier de ses bienfaits}\end{définition}
\begin{définition}\pcmn{对别人好}\end{définition}
\begin{exemple}\pjya{ɯʑɤɣ pɯ-sɯxtʂɯn-a}\hspace{5pt}\pcmn{我对他很好}\end{exemple}
\begin{exemple}\pjya{a-mgɯr nɯ-tɯ-rɤβraʁ, tɤ-tɯ-sɯxtsɯɣ, ɲɯ-tɯ-sɯxtʂɯn}\hspace{5pt}\pcmn{你帮我抠背部,你抠到正确的地方,做的很好}\end{exemple}\relationsémantique{参考}{\lien{ⓔtɯ-tʂɯn}{tɯ-tʂɯn}}\end{entrée}

\begin{entrée}{sɯxtɯɣ}{}{ⓔsɯxtɯɣ}\relationsémantique{参考}{\lien{ⓔtɯɣⓗ1}{tɯɣ₁}}\end{entrée}

\begin{entrée}{sɯχcoŋkroŋ}{}{ⓔsɯχcoŋkroŋ} 
\classe{vi}  
\grammaire{denom} \paradigme{dir}{nɯ-}
\begin{définition}\pfra{s'asseoir en tailleur}\end{définition}
\begin{définition}\pcmn{盘腿坐(男人坐的姿势)}\end{définition}
\begin{exemple}\pjya{nɯ-sɯχcoŋkroŋ-a (=χcoŋkroŋ nɯ-βzu-t-a)}\hspace{5pt}\pcmn{我盘着腿坐下了}\end{exemple}\relationsémantique{参考}{\lien{ⓔχcoŋkroŋ}{χcoŋkroŋ}}\relationsémantique{参考}{\lien{ⓔtɯ-rpɣo}{tɯ-rpɣo}}\relationsémantique{同义词}{\lien{ⓔsɯrmbɣotɯm}{sɯrmbɣotɯm}}\end{entrée}

\begin{entrée}{sɯχpjɤt}{}{ⓔsɯχpjɤt}\relationsémantique{参考}{\lien{ⓔχpjɤt}{χpjɤt}}\end{entrée}

\begin{entrée}{sɯχsu}{}{ⓔsɯχsu}\relationsémantique{参考}{\lien{ⓔχsu}{χsu}}\end{entrée}

\begin{entrée}{sɯχsɤl}{}{ⓔsɯχsɤl} 
\classe{vt} \paradigme{dir}{pɯ-}
\begin{définition}\pfra{reconnaître, s'apercevoir}\end{définition}
\begin{définition}\pcmn{认得,发现}\end{définition}
\begin{exemple}\pjya{ɲɯ-sɯχsɤl}\hspace{5pt}\pcmn{他认得他}\end{exemple}
\begin{exemple}\pjya{pɯ-sɯχsal-a}\hspace{5pt}\pcmn{我发现了他}\end{exemple}
\begin{exemple}\pjya{pɯ́-wɣ-sɯχsal-a}\hspace{5pt}\pcmn{他发现了我}\end{exemple}
\begin{exemple}\pjya{pɯ-sɯχsɤl-tɕi}\hspace{5pt}\pcmn{我们俩发现了他}\end{exemple}\étymologie{gsal}\end{entrée}

\begin{entrée}{sɯχta}{}{ⓔsɯχta} 
\classe{postp} 
\begin{définition}\pfra{par rapport à}\end{définition}
\begin{définition}\pcmn{比}\end{définition}\relationsémantique{参考}{\lien{ⓔstaʁ}{staʁ}}\relationsémantique{参考}{\lien{ⓔsɤz}{sɤz}}\end{entrée}

\begin{entrée}{sɯχtɯ}{}{ⓔsɯχtɯ}\relationsémantique{参考}{\lien{ⓔχtɯ}{χtɯ}}\end{entrée}

\begin{entrée}{sɯz}{}{ⓔsɯz} 
\classe{vt} \paradigme{dir}{pɯ-}\paradigme{generic negative}{mɤ-xsi}
\begin{définition}\pfra{savoir}\end{définition}
\begin{définition}\pcmn{知道}\end{définition}
\begin{exemple}\pjya{pɯ-mto-t-a kɯ-fse ri, ŋu maʁ mɤxsi}\hspace{5pt}\pcmn{我好像看见了,但是不确定}\end{exemple}\relationsémantique{参考}{\lien{ⓔamɯsɯz}{amɯsɯz}}\end{entrée}

\begin{entrée}{sɯzbaʁ}{}{ⓔsɯzbaʁ}\relationsémantique{参考}{\lien{ⓔzbaʁ}{zbaʁ}}\end{entrée}

\begin{entrée}{sɯzbɤβ}{}{ⓔsɯzbɤβ} 
\classe{n} 
\begin{définition}\pfra{nœud (sur un arbre)}\end{définition}
\begin{définition}\pcmn{树瘤}\end{définition}\end{entrée}

\begin{entrée}{sɯzdɯɣ}{}{ⓔsɯzdɯɣ} 
\classe{vt}  
\grammaire{caus} \paradigme{dir}{pɯ-}
\begin{définition}\pfra{causer du souci}\end{définition}
\begin{définition}\pcmn{让……受苦;让……费心}\end{définition}
\begin{exemple}\pjya{jiɕqha nɯ pɯ-sɯzdɯɣ-a}\hspace{5pt}\pcmn{我令他受苦了}\end{exemple}
\begin{exemple}\pjya{pɯ́-wɣ-sɯzdɯɣ-a}\hspace{5pt}\pcmn{他令我受苦了}\end{exemple}
\begin{exemple}\pjya{pɯ-ta-sɯzdɯɣ}\hspace{5pt}\pcmn{我令你担心了}\end{exemple}
\begin{exemple}\pjya{ɲɯ-ta-sɯzdɯɣ nɯ!}\hspace{5pt}\pcmn{我令你受苦了!}\end{exemple}
\begin{sous-entrée}{ʑɣɤsɯzdɯɣ}{ⓔsɯzdɯɣⓝʑɣɤsɯzdɯɣ} 
\classe{vi}  
\grammaire{refl} 
\begin{définition}\pfra{se donner du mal}\end{définition}
\begin{définition}\pcmn{大费周章;不辞辛苦}\end{définition}\end{sous-entrée}

\begin{sous-entrée}{sɤsɯzdɯɣ}{ⓔsɯzdɯɣⓝsɤsɯzdɯɣ} 
\classe{vs} 
\begin{définition}\pfra{gêner les gens}\end{définition}
\begin{définition}\pcmn{麻烦别人}\end{définition}
\begin{exemple}\pjya{nɤki tɤ-pɤtso nɯ kɤ-ndo ɲɯ-ɴqa ma ɲɯ-sɤsɯzdɯɣ}\hspace{5pt}\pcmn{带这个小孩子很难,因为他很麻烦}\end{exemple}\relationsémantique{参考}{\lien{ⓔzdɯɣ}{zdɯɣ}}\relationsémantique{参考}{\lien{ⓔnɯzdɯɣ}{nɯzdɯɣ}}\end{sous-entrée}

\end{entrée}

\begin{entrée}{sɯzgrɯtɕhɯ}{}{ⓔsɯzgrɯtɕhɯ} 
\classe{vt}  
\grammaire{incorp} \paradigme{dir}{tɤ-}
\begin{définition}\pfra{donner un coup de coude}\end{définition}
\begin{définition}\pcmn{打一肘}\end{définition}
\begin{exemple}\pjya{ɯʑo ɲɯ-nɯɕmɯrga tɕe tɤ-sɯzgrɯtɕhɯ-t-a}\hspace{5pt}\pcmn{他多嘴了,我给他打了一肘}\end{exemple}\relationsémantique{参考}{\lien{ⓔzgrɯtɕhɯ}{zgrɯtɕhɯ}}\end{entrée}

\begin{entrée}{sɯzʁe}{}{ⓔsɯzʁe} 
\classe{n} 
\begin{définition}\pfra{action de transporter du bois}\end{définition}
\begin{définition}\pcmn{背柴}\end{définition}\relationsémantique{参考}{\lien{ⓔnɯsɯzʁe}{nɯsɯzʁe}}\end{entrée}

\begin{entrée}{sɯʑŋgrɯt}{}{ⓔsɯʑŋgrɯt} 
\classe{vi} \paradigme{dir}{tɤ-}
\begin{définition}\pfra{ravaler ses larmes, gémir}\end{définition}
\begin{définition}\pcmn{啜泣,呻吟}\end{définition}
\begin{exemple}\pjya{kɤ-rŋgɯ tɕe, ɲɯ-sɯʑŋgrɯt}\hspace{5pt}\pcmn{他睡了就呻吟}\end{exemple}\end{entrée}

\begin{entrée}{suwa}{}{ⓔsuwa} 
\classe{n} 
\begin{définition}\pfra{viseur}\end{définition}
\begin{définition}\pcmn{准星}\end{définition}
\begin{exemple}\pjya{suwa nɯ ɕɤmɯɣdɯ ɣɯ-lɤt tɤ-mda tɕe ɯ-sɤ-z-nɯpɯmɲɯɣ ŋu}\hspace{5pt}\pcmn{准星是打枪时用来瞄准的东西。}\end{exemple}\relationsémantique{参考}{\lien{ⓔnɯsuwa}{nɯsuwa}}\end{entrée}

\newpage\caractère{ʂ}

\begin{entrée}{ʂa}{}{ⓔʂa} 
\classe{vs} \paradigme{dir}{thɯ-}
\begin{définition}\pfra{capable}\end{définition}
\begin{définition}\pcmn{能干}\end{définition}
\begin{exemple}\pjya{kɤ-ŋke kɯ-ʂa ci ɲɯ-ŋu}\hspace{5pt}\pcmn{他是一个很能走路的人}\end{exemple}\étymologie{sra}\end{entrée}

\begin{entrée}{ʂaʁ,ta}{}{ⓔʂaʁ,ta} 
\classe{adv}
\classe{vt} \paradigme{dir}{kɤ-}
\begin{définition}\pfra{marquer au fer rouge}\end{définition}
\begin{définition}\pcmn{烙印}\end{définition}
\begin{exemple}\pjya{a-jaʁ ʂaʁ kɤ-nɯ-sɯta-t-a}\hspace{5pt}\pcmn{我烙(烫)到了手}\end{exemple}
\begin{exemple}\pjya{ʂaʁ ɯ-rŋa ko-ta}\hspace{5pt}\pcmn{在他脸上烙了印}\end{exemple}
\begin{exemple}\pjya{ɕom chɤ-z-ɣɯrni tɕe, ɯ-ɕa ɯ-taʁ ʂaʁ ko-ta}\hspace{5pt}\pcmn{把铁烧红了,就在它的肉上烙了印子}\end{exemple}\relationsémantique{Component 1}{\lien{}{ʂaʁ}}\relationsémantique{Component 2}{\lien{ⓔta}{ta}}\end{entrée}

\begin{entrée}{ʂɣɤlʂɣɤl}{}{ⓔʂɣɤlʂɣɤl} 
\classe{idph.2} 
\begin{définition}\pfra{transparent et brillant comme la rosée du matin}\end{définition}
\begin{définition}\pcmn{形容像草叶上滚动的露珠一样透明发亮的样子}\end{définition}
\begin{exemple}\pjya{tɤ-rɣe ʂɣɤlʂɣɤl ci ɲɯ-ŋu}\hspace{5pt}\pcmn{珍珠透明发亮}\end{exemple}
\begin{exemple}\pjya{tɤ-tsrɯ ɯ-taʁ tɤʑri ʂɣɤlʂɣɤl ɲɯ-pa}\hspace{5pt}\pcmn{萌芽上有透明发亮的露珠}\end{exemple}
\begin{exemple}\pjya{kɯ-wxti ra kɯ tu-ti-nɯ, ftɕar tɤ-tsrɯ tɤ-ɬoʁ tɕe, tɤ-tsrɯ ɯ-ku ɯ-taʁ zɯ tɤʑri ʂɣɤlʂɣɤl rɯri ɣɯ tɯka tu tɕe, tɯrme tɯ-fsonam nɯ, nɯ kɯ-fse, rɯri tɯkaka tɯ-fsonam nɯ nɯ fse tu-ti-nɯ tu-raχpi-nɯ ɲɯ-ŋgrɤl}\hspace{5pt}\pcmn{老年人说,夏天萌芽长出来的时候,每个萌芽上面有透明的露珠,代表着每个人的运气}\end{exemple}
\begin{exemple}\pjya{ɯ-jaʁ ɯ-taʁ cimbɤrom ʂɣɤlʂɣɤl ʑo to-rku}\hspace{5pt}\pcmn{他手上起了透明的水疱}\end{exemple}\end{entrée}

\begin{entrée}{ʂmɤβʂmɤβ}{}{ⓔʂmɤβʂmɤβ} 
\classe{idph.2} 
\begin{définition}\pfra{qui forme une fine couche}\end{définition}
\begin{définition}\pcmn{形容物体薄薄一层,不完全透明的样子}\end{définition}\relationsémantique{参考}{\lien{ⓔrmɤβrmɤβ}{rmɤβrmɤβ}}\end{entrée}

\begin{entrée}{ʂɲoʁʂɲoʁ}{}{ⓔʂɲoʁʂɲoʁ} 
\classe{idph.2} 
\begin{définition}\pfra{svelte}\end{définition}
\begin{définition}\pcmn{形容身材苗条,瘦高瘦高的模样}\end{définition}
\begin{exemple}\pjya{tɤ-pɤtso ɯ-phoŋbu ʂɲoʁʂɲoʁ ɲɯ-pa}\hspace{5pt}\pcmn{小孩子(身材)瘦高瘦高的。}\end{exemple}\end{entrée}

\begin{entrée}{ʂɲɯɣnɤʂɲɯɣ}{}{ⓔʂɲɯɣnɤʂɲɯɣ} 
\classe{idph.3} 
\begin{définition}\pfra{intermittent}\end{définition}
\begin{définition}\pcmn{形容运动、感觉等间发的样子}\end{définition}
\begin{exemple}\pjya{a-mthɤɣ ʂɲɯɣnɤʂɲɯɣ ɲɯ-mŋɤm}\hspace{5pt}\pcmn{我的腰一阵一阵地痛}\end{exemple}\relationsémantique{参考}{\lien{ⓔɣɤʂɲɯɣlɯɣ}{ɣɤʂɲɯɣlɯɣ}}\end{entrée}

\begin{entrée}{ʂɯɣʂɯɣ}{}{ⓔʂɯɣʂɯɣ} 
\classe{idph.2} 
\begin{définition}\pfra{clair}\end{définition}
\begin{définition}\pcmn{亮}\end{définition}
\begin{exemple}\pjya{ndʑa ʂɯɣʂɯɣ pjɤ-ndɯ}\hspace{5pt}\pcmn{彩虹出来了}\end{exemple}\relationsémantique{参考}{\lien{ⓔʂɯŋʂɯŋ}{ʂɯŋʂɯŋ}}\end{entrée}

\begin{entrée}{ʂɯŋʂɯŋ}{}{ⓔʂɯŋʂɯŋ} 
\classe{idph.2} 
\begin{définition}\pfra{clair}\end{définition}
\begin{définition}\pcmn{天晴}\end{définition}
\begin{exemple}\pjya{tɤŋe ʂɯŋʂɯŋ ɲɤ-ɬoʁ}\hspace{5pt}\pcmn{太阳出来了}\end{exemple}
\begin{exemple}\pjya{ʂɯŋʂɯŋ ɲɤ-jɯm}\hspace{5pt}\pcmn{天晴了}\end{exemple}
\begin{exemple}\pjya{ndʑa ʂɯŋʂɯŋ pjɤ-ndɯ}\hspace{5pt}\pcmn{彩虹出来了}\end{exemple}
\begin{exemple}\pjya{ɯ-mɲaʁ ɲɯ-mto ʂɯŋʂɯŋ ʑo}\hspace{5pt}\pcmn{他眼睛看得很清楚}\end{exemple}\relationsémantique{参考}{\lien{ⓔxɯŋxɯŋ}{xɯŋxɯŋ}}\end{entrée}

\begin{entrée}{ʂɯt}{}{ⓔʂɯt} 
\classe{idph.1} 
\begin{définition}\pfra{bruit que fait un oiseau en s'envolant brusquement}\end{définition}
\begin{définition}\pcmn{小鸟突然飞出来的声音}\end{définition}\end{entrée}

\begin{entrée}{ʂχinɤʂχi}{}{ⓔʂχinɤʂχi} 
\classe{idph.2} 
\begin{définition}\pfra{haletant}\end{définition}
\begin{définition}\pcmn{形容气喘吁吁的样子}\end{définition}
\begin{exemple}\pjya{jiɕqha nɯ ɲɯ-nɯtɕhomba tɕe, ɯ-rqo ʂχinɤʂχi ɲɤ-stu}\hspace{5pt}\pcmn{他感冒了,气喘吁吁的}\end{exemple}\end{entrée}

\begin{entrée}{ʂχɯʂχaj}{}{ⓔʂχɯʂχaj} 
\classe{idph.2} 
\begin{définition}\pfra{ayant plein de trous}\end{définition}
\begin{définition}\pcmn{形容满是洞洞眼眼的样子}\end{définition}
\begin{exemple}\pjya{a-ŋga ʂχɯʂχaj ɲɯ-xtsu ɕti ri, nɯ kɯnɤ ɲɯ-mpja}\hspace{5pt}\pcmn{虽然我衣服上到处都是洞洞眼眼,但还是很暖}\end{exemple}\relationsémantique{参考}{\lien{ⓔʂχɯʂχi}{ʂχɯʂχi}}\end{entrée}

\begin{entrée}{ʂχɯʂχi}{}{ⓔʂχɯʂχi} 
\classe{idph.2} 
\begin{définition}\pfra{ayant de grosses narines}\end{définition}
\begin{définition}\pcmn{形容鼻孔很大}\end{définition}
\begin{exemple}\pjya{mbro ɯ-ɕna ʂχɯʂχi to-stu, tɕe ndʐoʁ ɯβrɤ-ŋu ma}\hspace{5pt}\pcmn{马把鼻孔张大了,是不是要调皮了}\end{exemple}
\begin{sous-entrée}{ʂχɯwɯʂχawi}{ⓔʂχɯʂχiⓝʂχɯwɯʂχawi} 
\classe{idph.8} 
\begin{définition}\pfra{qui a des trous partout}\end{définition}
\begin{définition}\pcmn{到处都是洞}\end{définition}
\begin{exemple}\pjya{βʑɯ kɯ lʁa to-ndza tɕe ʂχɯwɯʂχawi ʑo ɲɤ-sɤβzu}\hspace{5pt}\pcmn{老鼠把口袋啃了,啃得到处都是洞}\end{exemple}\relationsémantique{参考}{\lien{ⓔʂχɯʂχaj}{ʂχɯʂχaj}}\end{sous-entrée}

\end{entrée}

\newpage\caractère{t}

\begin{entrée}{tu}{}{ⓔtu} 
\classe{vs} \paradigme{dir}{tɤ-}
\begin{définition}\pfra{exister, se trouver}\end{définition}
\begin{définition}\pcmn{有}\end{définition}
\begin{exemple}\pjya{aj nɯ kóʁmɯz @jieshang ɕ-pɯ-tu-a tɕe jɤ-azɣɯt-a}\hspace{5pt}\pcmn{我刚才在街上,刚刚到(家)}\end{exemple}
\begin{exemple}\pjya{mɤ-kɯ-pe tɤ-tu tɕe tu-ti-a}\hspace{5pt}\pcmn{有错误的时候我就说}\end{exemple}
\begin{exemple}\pjya{aʑɯɣ rŋɯl tu}\hspace{5pt}\pcmn{我有钱}\end{exemple}
\begin{exemple}\pjya{khɯtsa-ŋgɯ tɯ-ci tu}\hspace{5pt}\pcmn{碗里有水}\end{exemple}\end{entrée}

\begin{entrée}{ta}{}{ⓔta} 
\classe{vt}
\classe{np}
\classe{vt} \sens{1}\paradigme{dir}{\_}
\begin{définition}\pfra{poser, mettre}\end{définition}
\begin{définition}\pcmn{放置}\end{définition}
\begin{exemple}\pjya{ɯ-ŋga pɯ-ta-t-a}\hspace{5pt}\pcmn{我给他盖了被子}\end{exemple}\sens{2}\paradigme{dir}{tɤ-}\paradigme{dir}{pɯ-}
\begin{définition}\pfra{porter (lunettes, chapeau)}\end{définition}
\begin{définition}\pcmn{戴(眼镜。帽子)}\end{définition}
\begin{exemple}\pjya{χɕɤlmɯɣ tu-nɯ-te-a}\hspace{5pt}\pcmn{我戴眼镜}\end{exemple}
\begin{exemple}\pjya{χɕɤlmɯɣ tɤ-nɯ-te}\hspace{5pt}\pcmn{你戴上眼镜吧}\end{exemple}
\begin{exemple}\pjya{nɤ-ku ɯ-taʁ tɤ-rte ma-pɯ-tɯ-nɯ-te ma nɤ-tɯ-sɤjloʁ}\hspace{5pt}\pcmn{你不要戴上帽子,不好看}\end{exemple}\sens{3}\paradigme{dir}{nɯ-}
\begin{définition}\pfra{relâcher, laisser, abandonner}\end{définition}
\begin{définition}\pcmn{放(走);放下}\end{définition}
\begin{exemple}\pjya{mɤ-ta-ta}\hspace{5pt}\pcmn{我不会放过你的}\end{exemple}
\begin{exemple}\pjya{tɯrɣi nɯ-ta-t-a}\hspace{5pt}\pcmn{我留了种子}\end{exemple}
\begin{exemple}\pjya{kɯki kɤ-nɤma ki kɤ-ta a-ʁjiz mɯ́j-ɣi}\hspace{5pt}\pcmn{我不想放弃这个工作}\end{exemple}
\begin{exemple}\pjya{nɤ-ndza-ro ma-nɯ-tɯ-te ma ɯ-kɯ-ndza me (nɤ-khɯɲcɤr ma-nɯ-tɯ-te)}\hspace{5pt}\pcmn{你不要吃剩一口,不会有人吃你的剩饭}\end{exemple}\sens{4}\paradigme{dir}{kɤ-}
\begin{définition}\pfra{cuire}\end{définition}
\begin{définition}\pcmn{蒸;熬}\end{définition}
\begin{exemple}\pjya{qajɣi khon kɤ-ta-t-a}\hspace{5pt}\pcmn{我蒸了馍馍}\end{exemple}
\begin{exemple}\pjya{tʂha kɤ-ta-t-a}\hspace{5pt}\pcmn{我熬了茶}\end{exemple}\sens{5}\paradigme{dir}{pɯ-}\paradigme{dir}{nɯ-}
\begin{définition}\pfra{se produire (gel)}\end{définition}
\begin{définition}\pcmn{下(霜)}\end{définition}
\begin{définition}\pfra{reporter la faute sur}\end{définition}
\begin{définition}\pcmn{归罪于}\end{définition}
\begin{exemple}\pjya{ɕŋɤr pjɤ-t-a}\hspace{5pt}\pcmn{下了霜}\end{exemple}\relationsémantique{Component 1}{\lien{ⓔtaʁⓗ3ⓝɯ-taʁ}{ɯ-taʁ}}\relationsémantique{Component 2}{\lien{ⓔta}{ta}}
\begin{sous-entrée}{nɤtɯta}{ⓔtaⓢ5ⓝnɤtɯta} 
\classe{vt} 
\begin{définition}\pfra{mettre n'importe comment}\end{définition}
\begin{définition}\pcmn{随便放,放了放去}\end{définition}
\begin{exemple}\pjya{nɤ-laχtɕha ra aʁɤndɯndɤt ma-tɯ-nɤtɯte ma ɲɯ-me ŋu}\hspace{5pt}\pcmn{你不要随便吧东西乱放,不然丢失的}\end{exemple}\end{sous-entrée}

\begin{sous-entrée}{ɯ-taʁ,ta}{ⓔtaⓢ5ⓝɯ-taʁ,ta}\end{sous-entrée}

\paradigme{dir}{kɤ-}
\begin{définition}\pfra{épouser}\end{définition}
\begin{définition}\pcmn{娶妻}\end{définition}
\begin{exemple}\pjya{mɤ-kɯ-pe ci tu-tɯ-tu tɕe ɯʑo ɯ-taʁ ɲɯ-ta-nɯ pjɤ-ŋu}\hspace{5pt}\pcmn{每一次发生不好的事情的时候都归罪于他}\end{exemple}
\begin{exemple}\pjya{aʑo ki tɯ-ŋgo ki kɤ-ɣɤmna a-mɤ-pɯ-cha-a tɕe smɤnba kɤ-βzu mɤ-kɯ-spa ɯ-taʁ ɲɯ-ta-nɯ ɲɯ-ŋu ma tɯ-ŋgo kɤ-ɣɤmna mɤ-kɯ-khɯ nɯmɯ́j-tso-nɯ}\hspace{5pt}\pcmn{如果我治不好那个病,人们会认为医术不行,他们不懂这种病根本治不了}\end{exemple}
\begin{exemple}\pjya{rɟɤlpu kɯ ɯ-rʑaβ ko-nɯ-ta}\hspace{5pt}\pcmn{土司把她娶为妻子}\end{exemple}
\begin{exemple}\pjya{a-me nɤ-rʑaβ ku-tɯ-nɯte jɤɣ}\hspace{5pt}\pcmn{你可以娶我的女儿为妻}\end{exemple}
\begin{sous-entrée}{nɯta}{ⓔtaⓝnɯta} 
\classe{vt} \end{sous-entrée}

\begin{sous-entrée}{znɯta}{ⓔtaⓝznɯta} 
\classe{vt} \sens{1}\paradigme{dir}{kɤ-}
\begin{définition}\pfra{donner en marriage}\end{définition}
\begin{définition}\pcmn{许配}\end{définition}\end{sous-entrée}

\sens{2}\paradigme{dir}{nɯ-}
\begin{définition}\pfra{passer à}\end{définition}
\begin{définition}\pcmn{让给别人}\end{définition}
\begin{exemple}\pjya{nɤʑo tɤ-tɯ-χtɯ-t nɯ, nɤʑɯɣ mɯ-mɤ-ɲɯ-ra nɤ, aʑo ɲɯ-kɯ-znɯta-a ɯ́-jɤɣ?}\hspace{5pt}\pcmn{你买的东西如果对你没有用的话就让给我吗?}\end{exemple}\relationsémantique{参考}{\lien{ⓔsɯta}{sɯta}}\relationsémantique{参考}{\lien{ⓔatɯta}{atɯta}}\end{entrée}

\begin{entrée}{tacɯn}{}{ⓔtacɯn} 
\classe{n} 
\begin{définition}\pfra{habit que l'on rabat au niveau de l'épaule}\end{définition}
\begin{définition}\pcmn{大襟}\end{définition}\étymologie{fn:大襟}\end{entrée}

\begin{entrée}{tal}{}{ⓔtal} 
\classe{n} 
\begin{définition}\pfra{jianzi}\end{définition}
\begin{définition}\pcmn{毽子}\end{définition}\end{entrée}

\begin{entrée}{talɤn}{}{ⓔtalɤn} 
\classe{n} 
\begin{définition}\pfra{besace}\end{définition}
\begin{définition}\pcmn{褡裢}\end{définition}\étymologie{fn:褡裢}\end{entrée}

\begin{entrée}{ta-ma}{}{ⓔta-ma} 
\classe{np} 
\begin{définition}\pfra{occupation}\end{définition}
\begin{définition}\pcmn{事情}\end{définition}
\begin{exemple}\pjya{aʑo a-ma ku-nɤma-nɯ ɕti (=a-tshɤt ku-rɤma-nɯ ɕti)}\hspace{5pt}\pcmn{他们在替我工作}\end{exemple}\end{entrée}

\begin{entrée}{ta-mar}{}{ⓔta-mar} 
\classe{np} 
\begin{définition}\pfra{beurre}\end{définition}
\begin{définition}\pcmn{酥油}\end{définition}\relationsémantique{参考}{\lien{ⓔʑɯnmar}{ʑɯnmar}}\end{entrée}

\begin{entrée}{tamɢom}{}{ⓔtamɢom} 
\classe{n} 
\begin{définition}\pfra{tenaille, pince, pincette}\end{définition}
\begin{définition}\pcmn{夹子}\end{définition}\relationsémantique{参考}{\lien{ⓔmɢom}{mɢom}}\relationsémantique{参考}{\lien{ⓔɟɯmɢom}{ɟɯmɢom}}\end{entrée}

\begin{entrée}{taŋ}{}{ⓔtaŋ} 
\classe{vs} \paradigme{dir}{tɤ-}
\begin{définition}\pfra{intelligent (personne), authentique}\end{définition}
\begin{définition}\pcmn{聪明;真的}\end{définition}
\begin{exemple}\pjya{pjɯrɯ ɲɯ-taŋ}\hspace{5pt}\pcmn{珊瑚是真货}\end{exemple}
\begin{exemple}\pjya{rŋɯl ɲɯ-taŋ}\hspace{5pt}\pcmn{银是纯的}\end{exemple}
\begin{exemple}\pjya{smɤɣ kɯ-taŋ}\hspace{5pt}\pcmn{纯毛}\end{exemple}\end{entrée}

\begin{entrée}{taŋbu}{}{ⓔtaŋbu} 
\classe{n} 
\begin{définition}\pfra{premier mois}\end{définition}
\begin{définition}\pcmn{一月}\end{définition}\étymologie{daŋ.po}\end{entrée}

\begin{entrée}{taŋi}{}{ⓔtaŋi} 
\classe{n} 
\begin{définition}\pfra{sécheresse}\end{définition}
\begin{définition}\pcmn{旱灾}\end{définition}
\begin{exemple}\pjya{taŋi to-ɣi}\hspace{5pt}\pcmn{有了旱灾}\end{exemple}
\begin{exemple}\pjya{taŋi ɲɯ-thɯ ma tɯ-mɯ mɯ́j-lɤt}\hspace{5pt}\pcmn{旱灾很严重,不下雨}\end{exemple}\end{entrée}

\begin{entrée}{taɴɢoʁ}{}{ⓔtaɴɢoʁ} 
\classe{n} 
\begin{définition}\pfra{vanneries, cage}\end{définition}
\begin{définition}\pcmn{篮子;笼子}\end{définition}
\begin{exemple}\pjya{taɴɢoʁ tɤ-βzu-t-a}\hspace{5pt}\pcmn{我编了篮子}\end{exemple}\end{entrée}

\begin{entrée}{taqaβ}{}{ⓔtaqaβ} 
\classe{n} 
\begin{définition}\pfra{aiguille}\end{définition}
\begin{définition}\pcmn{针}\end{définition}\relationsémantique{参考}{\lien{ⓔrasqaβ}{rasqaβ}}\relationsémantique{参考}{\lien{ⓔarɯtaqaβ}{arɯtaqaβ}}\end{entrée}

\begin{entrée}{taqaβrna}{}{ⓔtaqaβrna} 
\classe{n} 
\begin{définition}\pfra{chas de l'aiguille}\end{définition}
\begin{définition}\pcmn{针眼}\end{définition}\end{entrée}

\begin{entrée}{tar}{}{ⓔtar} 
\classe{vi} \paradigme{dir}{thɯ-}\paradigme{dir}{tɤ-}
\begin{définition}\pfra{se développer}\end{définition}
\begin{définition}\pcmn{发展}\end{définition}
\begin{exemple}\pjya{rgali chɤ-tar}\hspace{5pt}\pcmn{小奶牛身体壮起来了}\end{exemple}
\begin{exemple}\pjya{jiɕqha nɯra to-tar-nɯ}\hspace{5pt}\pcmn{他变得比较富裕了}\end{exemple}
\begin{sous-entrée}{sɯxtar}{ⓔtarⓝsɯxtar} 
\classe{vt}  
\grammaire{caus} \end{sous-entrée}

\étymologie{dar}\end{entrée}

\begin{entrée}{tartɕɯn}{}{ⓔtartɕɯn} 
\classe{n} 
\begin{définition}\pfra{grand drapeau flottant au vent}\end{définition}
\begin{définition}\pcmn{大经幡}\end{définition}\étymologie{dar.tɕʰen}\end{entrée}

\begin{entrée}{taʁ}{₃}{ⓔtaʁⓗ3} 
\classe{adv} 
\begin{définition}\pfra{en haut}\end{définition}
\begin{définition}\pcmn{上面}\end{définition}
\begin{sous-entrée}{ɯ-taʁ}{ⓔtaʁⓗ3ⓝɯ-taʁ} 
\classe{np} 
\begin{définition}\pfra{le haut}\end{définition}
\begin{définition}\pcmn{上面}\end{définition}\end{sous-entrée}

\end{entrée}

\begin{entrée}{taʁ}{₁}{ⓔtaʁⓗ1} 
\classe{vl} \paradigme{dir}{thɯ-}
\begin{définition}\pfra{tisser}\end{définition}
\begin{définition}\pcmn{织}\end{définition}
\begin{exemple}\pjya{tɯ-ŋga ɲɯ-ɤsɯ-taʁ}\hspace{5pt}\pcmn{他在织衣服}\end{exemple}
\begin{exemple}\pjya{tɯŋgar ɲɯ-ɤsɯ-taʁ}\hspace{5pt}\pcmn{他在织羊毛}\end{exemple}
\begin{exemple}\pjya{a-mu pɯ-taʁ}\hspace{5pt}\pcmn{我的母亲织布}\end{exemple}\end{entrée}

\begin{entrée}{taʁ}{₂}{ⓔtaʁⓗ2} 
\classe{vs} \paradigme{dir}{pɯ-}\paradigme{dir}{tɤ-}\paradigme{dir}{tɤ-}\paradigme{dir}{tɤ-}
\begin{définition}\pfra{clair}\end{définition}
\begin{définition}\pcmn{清楚;准确}\end{définition}
\begin{définition}\pfra{dire clairement}\end{définition}
\begin{définition}\pcmn{说清楚}\end{définition}
\begin{définition}\pfra{dire clairement}\end{définition}
\begin{définition}\pcmn{说得清楚}\end{définition}
\begin{exemple}\pjya{kɯ-tɯ-taʁ ɲɯ-ɕti wo}\hspace{5pt}\pcmn{很清楚嘛}\end{exemple}
\begin{exemple}\pjya{ɯ-tɤ-rʑaʁ wuma ɲɯ-taʁ}\hspace{5pt}\pcmn{他的时间很准确}\end{exemple}
\begin{exemple}\pjya{ɯ-ʁndo ɲɯ-taʁ}\hspace{5pt}\pcmn{他的话很清楚}\end{exemple}
\begin{exemple}\pjya{kɤ-fɕɤt stɯsti kɯ mɯ-ɲɯ-ɤtɕɯxtaʁ, mɯ-ɲɯ-ɤmɯtso tɕe tɯpɤr pjɯ́-wɣ-lɤt ɲɯ-ra}\hspace{5pt}\pcmn{光是讲述不够清楚,所以要拍照片}\end{exemple}
\begin{exemple}\pjya{tɯ-rju ta-sɤtɕɯxtaʁ}\end{exemple}
\begin{exemple}\pjya{ki tɯ-rju ki kɤ-ɣɤtaʁ ɲɯ-ɴqa}\hspace{5pt}\pcmn{这个词的发音很难发准}\end{exemple}
\begin{exemple}\pjya{tú-wɣ-ɣɤtaʁ tɕe ``zdɯxpa" tu-kɯ-ti ŋu ri, tú-wɣ-sɤpɤmbat tɕe ``dɯxpa" tu-kɯ-ti ŋu}\hspace{5pt}\pcmn{要讲得清楚一点就应该说\lien{}{zdɯxpa},讲得简单一点就说\lien{ⓔ\_dɯxpa}{dɯxpa}}\end{exemple}
\begin{sous-entrée}{atɕɯxtaʁ}{ⓔtaʁⓗ2ⓝatɕɯxtaʁ} 
\classe{vs} \end{sous-entrée}

\begin{sous-entrée}{sɤtɕɯxtaʁ}{ⓔtaʁⓗ2ⓝsɤtɕɯxtaʁ} 
\classe{vt} \end{sous-entrée}

\begin{sous-entrée}{ɣɤtaʁ}{ⓔtaʁⓗ2ⓝɣɤtaʁ} 
\classe{vt} \end{sous-entrée}

\end{entrée}

\begin{entrée}{ta-ʁa}{}{ⓔta-ʁa} 
\classe{np} 
\begin{définition}\pfra{temps libre}\end{définition}
\begin{définition}\pcmn{空的时间}\end{définition}
\begin{exemple}\pjya{jisŋi a-ʁa tu}\hspace{5pt}\pcmn{我今天有空}\end{exemple}
\begin{exemple}\pjya{@xingqier a-ʁa me ma ɯ-ro ra tu}\hspace{5pt}\pcmn{星期二没有空,剩下的时间有空}\end{exemple}
\begin{exemple}\pjya{a-ʁa qajdo-ʁa}\hspace{5pt}\pcmn{我像乌鸦一样闲}\end{exemple}
\begin{exemple}\pjya{a-ʁa me}\hspace{5pt}\pcmn{我没有空}\end{exemple}\end{entrée}

\begin{entrée}{taʁɤndu}{}{ⓔtaʁɤndu} 
\classe{n} 
\begin{définition}\pfra{échange de travail}\end{définition}
\begin{définition}\pcmn{还工}\end{définition}
\begin{exemple}\pjya{a-taʁɤndu ɣɯ-tɤ-lɤt, tɕe aʑo nɤ-ʁɤndɤsi ju-ɣi-a}\hspace{5pt}\pcmn{你先来给我做工,然后我就来给你还工}\end{exemple}
\begin{exemple}\pjya{ɯʑo kɯ a-taʁɤndu ɣɯ-ta-lɤt tɕe, aʑo ɯ-ʁɤndɤsi ɕe-a ra}\hspace{5pt}\pcmn{他先来给我做工,我要给他还工}\end{exemple}\relationsémantique{参考}{\lien{ⓔta-ʁɤndɤsi}{ta-ʁɤndɤsi}}\relationsémantique{参考}{\lien{ⓔandu}{andu}}\end{entrée}

\begin{entrée}{ta-ʁɤndɤsci}{}{ⓔta-ʁɤndɤsci}\relationsémantique{参考}{\lien{ⓔta-ʁɤndɤsi}{ta-ʁɤndɤsi}}\end{entrée}

\begin{entrée}{ta-ʁɤndɤsi}{}{ⓔta-ʁɤndɤsi} 
\classe{np} 
\begin{définition}\pfra{échange de travail}\end{définition}
\begin{définition}\pcmn{还工}\end{définition}
\begin{exemple}\pjya{a-taʁɤndu ɣɯ-tɤ-lɤt, tɕe aʑo nɤ-ʁɤndɤsi ju-ɣi-a}\hspace{5pt}\pcmn{你先来给我做工,然后我就来给你还工}\end{exemple}
\begin{exemple}\pjya{ɯʑo kɯ a-taʁɤndu ɣɯ-ta-lɤt tɕe, aʑo ɯ-ʁɤndɤsi ɕe-a ra}\hspace{5pt}\pcmn{他先来给我做工,我要给他还工}\end{exemple}\relationsémantique{参考}{\lien{ⓔandu}{andu}}\relationsémantique{参考}{\lien{ⓔtaʁɤndu}{taʁɤndu}}\end{entrée}

\begin{entrée}{ta-ʁi}{}{ⓔta-ʁi} 
\classe{np} 
\begin{définition}\pfra{petit frère, petite sœur}\end{définition}
\begin{définition}\pcmn{弟弟;妹妹}\end{définition}\end{entrée}

\begin{entrée}{ta-ʁjɯβ}{}{ⓔta-ʁjɯβ} 
\classe{np} 
\begin{définition}\pfra{ombre}\end{définition}
\begin{définition}\pcmn{影子}\end{définition}\relationsémantique{参考}{\lien{ⓔnaʁjɯβ}{naʁjɯβ}}\relationsémantique{参考}{\lien{ⓔsqaβjɯβ}{sqaβjɯβ}}\end{entrée}

\begin{entrée}{ta-ʁɟaz}{}{ⓔta-ʁɟaz} 
\classe{np}
\classe{np}
\classe{vt} \sens{1}
\begin{définition}\pfra{suie}\end{définition}
\begin{définition}\pcmn{碳黑【锅烟墨】}\end{définition}\sens{2}\paradigme{dir}{nɯ-}
\begin{définition}\pfra{défaut}\end{définition}
\begin{définition}\pcmn{缺点}\end{définition}
\begin{définition}\pfra{décharger sa faute sur}\end{définition}
\begin{définition}\pcmn{令……背黑锅}\end{définition}
\begin{exemple}\pjya{nɤʑo nɤ-ʁɟaz nɯ aʑo ma-nɯ-kɯ-mar-a}\hspace{5pt}\pcmn{别让我替你背黑锅}\end{exemple}\relationsémantique{Component 1}{\lien{ⓔta-ʁɟaz}{ta-ʁɟaz}}\relationsémantique{Component 2}{\lien{ⓔmar}{mar}}
\begin{sous-entrée}{ta-ʁɟaz,mar}{ⓔta-ʁɟazⓢ2ⓝta-ʁɟaz,mar}\end{sous-entrée}

\end{entrée}

\begin{entrée}{taʁki}{}{ⓔtaʁki} 
\classe{n} 
\begin{définition}\pfra{de haut en bas}\end{définition}
\begin{définition}\pcmn{上下}\end{définition}\relationsémantique{参考}{\lien{ⓔtaʁⓗ3}{taʁ₃}}\end{entrée}

\begin{entrée}{taʁlɤkɯm}{}{ⓔtaʁlɤkɯm} 
\classe{n} 
\begin{définition}\pfra{balcon}\end{définition}
\begin{définition}\pcmn{藏式房屋的阳台}\end{définition}\end{entrée}

\begin{entrée}{taʁmbra,lɤt}{}{ⓔtaʁmbra,lɤt} 
\classe{n}
\classe{vt} \paradigme{dir}{nɯ-}\sens{1}
\begin{définition}\pfra{pleurer en s'agitant de toutes ses forces (enfant)}\end{définition}
\begin{définition}\pcmn{哭得大吵大闹(小孩子)}\end{définition}
\begin{exemple}\pjya{tɤ-pɤtso tɤ-wu ɯ-tɯ-χɕu kɯ taʁmbra ʑo ɲɯ-lɤt ɲɯ-ɕti}\hspace{5pt}\pcmn{小孩哭乱叫乱吼}\end{exemple}\sens{2}
\begin{définition}\pfra{s'élancer en sautant (cheval)}\end{définition}
\begin{définition}\pcmn{跳过去(马)}\end{définition}\relationsémantique{Component 1}{\lien{}{taʁmbra}}\relationsémantique{Component 2}{\lien{}{lɤt}}\relationsémantique{参考}{\lien{ⓔlɤtⓗ1}{lɤt₁}}\end{entrée}

\begin{entrée}{taʁndo}{}{ⓔtaʁndo} 
\classe{n} 
\begin{définition}\pfra{parole}\end{définition}
\begin{définition}\pcmn{话(长辈对下辈、老师对学生、领导对属下等)}\end{définition}\relationsémantique{参考}{\lien{ⓔnɯstɤraʁndo}{nɯstɤraʁndo}}\relationsémantique{参考}{\lien{ⓔnɯʁndomnɤt}{nɯʁndomnɤt}}\end{entrée}

\begin{entrée}{taʁndzɤr}{}{ⓔtaʁndzɤr} 
\classe{n} 
\begin{définition}\pfra{seau servant à mettre la pâtée des cochons}\end{définition}
\begin{définition}\pcmn{喂猪用的木桶}\end{définition}\end{entrée}

\begin{entrée}{ta-ʁrɤt}{}{ⓔta-ʁrɤt} 
\classe{np} 
\begin{définition}\pfra{charbon de bois}\end{définition}
\begin{définition}\pcmn{碳}\end{définition}
\begin{exemple}\pjya{si pɯ-nɯt ɯ-qhu tɕe ɯ-ʁrɤt ɲɯ-βze ŋu}\hspace{5pt}\pcmn{把树木烧了之后就会变成碳}\end{exemple}\relationsémantique{参考}{\lien{ⓔraʁrɤt}{raʁrɤt}}\end{entrée}

\begin{entrée}{taʁrdo}{}{ⓔtaʁrdo} 
\classe{n}  
\grammaire{n.lieu} 
\begin{définition}\pfra{l'un des hameaux de Kamnyu}\end{définition}
\begin{définition}\pcmn{干木鸟的大队之一}\end{définition}\end{entrée}

\begin{entrée}{ta-ʁri}{}{ⓔta-ʁri} 
\classe{np} 
\begin{définition}\pfra{saleté}\end{définition}
\begin{définition}\pcmn{污垢}\end{définition}
\begin{exemple}\pjya{ɯʑo ɯ-ʁri}\hspace{5pt}\pcmn{他身上的污垢}\end{exemple}\end{entrée}

\begin{entrée}{ta-ʁrɯ}{}{ⓔta-ʁrɯ} 
\classe{np} \sens{1}
\begin{définition}\pfra{corne}\end{définition}
\begin{définition}\pcmn{角}\end{définition}\sens{2}
\begin{définition}\pfra{cor}\end{définition}
\begin{définition}\pcmn{趼子}\end{définition}\relationsémantique{参考}{\lien{ⓔɣɤʁrɯ}{ɣɤʁrɯ}}\relationsémantique{参考}{\lien{ⓔʁrɯlu}{ʁrɯlu}}\end{entrée}

\begin{entrée}{ta-ʁrɯm}{}{ⓔta-ʁrɯm} 
\classe{np} \sens{1}
\begin{définition}\pfra{lumière}\end{définition}
\begin{définition}\pcmn{光}\end{définition}
\begin{exemple}\pjya{tɤtʂu ɯ-ʁrɯm}\hspace{5pt}\pcmn{灯的光}\end{exemple}\sens{2}
\begin{définition}\pfra{reflet}\end{définition}
\begin{définition}\pcmn{倒影}\end{définition}
\begin{exemple}\pjya{tɯ-ci ɯ-ŋgɯ a-ʁrɯm pjɤ-ntɕhɤr}\hspace{5pt}\pcmn{水中有我的倒影}\end{exemple}\sens{3}
\begin{définition}\pfra{endroit frais à l'ombre, ombre}\end{définition}
\begin{définition}\pcmn{阴凉的地方,影子}\end{définition}
\begin{exemple}\pjya{kha ɯ-ʁrɯm}\hspace{5pt}\pcmn{房子的影子}\end{exemple}\relationsémantique{参考}{\lien{ⓔtɯmɯʁrɯm}{tɯmɯʁrɯm}}\relationsémantique{参考}{\lien{ⓔsaʁrɯm}{saʁrɯm}}\end{entrée}

\begin{entrée}{tasa}{}{ⓔtasa} 
\classe{n} 
\begin{définition}\pfra{chanvre}\end{définition}
\begin{définition}\pcmn{大麻}\end{définition}\relationsémantique{参考}{\lien{ⓔtɤsɤmu}{tɤsɤmu}}\relationsémantique{参考}{\lien{ⓔtɤsɤɣʑa}{tɤsɤɣʑa}}\relationsémantique{参考}{\lien{ⓔtɤsɤrŋu}{tɤsɤrŋu}}\relationsémantique{参考}{\lien{ⓔtɤsɤsqɤri}{tɤsɤsqɤri}}\end{entrée}

\begin{entrée}{tatɕhoŋtɕhoŋ}{}{ⓔtatɕhoŋtɕhoŋ} 
\classe{n} 
\begin{définition}\pfra{chute d'eau}\end{définition}
\begin{définition}\pcmn{瀑布}\end{définition}\end{entrée}

\begin{entrée}{tatpa}{}{ⓔtatpa} 
\classe{n} 
\begin{définition}\pfra{foi}\end{définition}
\begin{définition}\pcmn{信仰}\end{définition}
\begin{exemple}\pjya{tatpa tɤ-ta-t-a}\hspace{5pt}\pcmn{我崇拜(了)他}\end{exemple}\étymologie{dad.pa}\end{entrée}

\begin{entrée}{tatshi}{}{ⓔtatshi} 
\classe{n}  
\grammaire{n.lieu} 
\begin{définition}\pfra{Datshang}\end{définition}
\begin{définition}\pcmn{大藏乡}\end{définition}\end{entrée}

\begin{entrée}{taχpa}{}{ⓔtaχpa} 
\classe{n} 
\begin{définition}\pfra{récolte d'un an}\end{définition}
\begin{définition}\pcmn{一年的庄稼}\end{définition}
\begin{exemple}\pjya{ɣɯjpa taχpa ɯ-ɲɯ-pe ?}\hspace{5pt}\pcmn{今年收成好不好?}\end{exemple}
\begin{exemple}\pjya{taχpa ɲɯ-nɤkɤro, ɲɯ-pe ɕti}\hspace{5pt}\pcmn{庄稼还可以}\end{exemple}\end{entrée}

\begin{entrée}{taχphe}{}{ⓔtaχphe} 
\classe{n} 
\begin{définition}\pfra{paume de la main (utilisée pour baffer)}\end{définition}
\begin{définition}\pcmn{手掌(掴耳光)}\end{définition}
\begin{exemple}\pjya{tɕoχtsi ɯ-taʁ taχphe pɯ-lat-a}\hspace{5pt}\pcmn{我拍了桌子}\end{exemple}\relationsémantique{参考}{\lien{ⓔnɤχphe}{nɤχphe}}\end{entrée}

\begin{entrée}{ta-χpi}{}{ⓔta-χpi} 
\classe{np} 
\begin{définition}\pfra{forme, modèle}\end{définition}
\begin{définition}\pcmn{形状;榜样}\end{définition}
\begin{exemple}\pjya{ɯʑo a-ta-χpi sna}\hspace{5pt}\pcmn{他值得做我的榜样}\end{exemple}\relationsémantique{参考}{\lien{ⓔznɯχpi}{znɯχpi}}\étymologie{dpe}\end{entrée}

\begin{entrée}{taχti}{}{ⓔtaχti} 
\classe{n} 
\begin{définition}\pfra{modèle}\end{définition}
\begin{définition}\pcmn{榜样}\end{définition}
\begin{exemple}\pjya{ɯ-taχti ɲɯ-sna}\hspace{5pt}\pcmn{可以当作榜样}\end{exemple}\end{entrée}

\begin{entrée}{tɤβ}{}{ⓔtɤβ} 
\classe{vl} \paradigme{dir}{pɯ-}\paradigme{dir}{kɤ-}
\begin{définition}\pfra{battre le grain}\end{définition}
\begin{définition}\pcmn{脱粒}\end{définition}
\begin{exemple}\pjya{ʑara pɯ-tɤβ-nɯ}\hspace{5pt}\pcmn{他们脱粒了}\end{exemple}
\begin{exemple}\pjya{tɤɕi pɯ-taβ-a}\hspace{5pt}\pcmn{我把青稞脱了粒}\end{exemple}
\begin{exemple}\pjya{qaj pɯ-taβ-a}\hspace{5pt}\pcmn{我把小麦脱了粒}\end{exemple}\end{entrée}

\begin{entrée}{tɤβɣemu}{}{ⓔtɤβɣemu} 
\classe{n} 
\begin{définition}\pfra{veuve}\end{définition}
\begin{définition}\pcmn{寡妇}\end{définition}\end{entrée}

\begin{entrée}{tɤβɣepɯ}{}{ⓔtɤβɣepɯ} 
\classe{n} 
\begin{définition}\pfra{orphelin}\end{définition}
\begin{définition}\pcmn{孤儿}\end{définition}\end{entrée}

\begin{entrée}{tɤβɣerɟit}{}{ⓔtɤβɣerɟit} 
\classe{n} 
\begin{définition}\pfra{orphelin}\end{définition}
\begin{définition}\pcmn{孤儿}\end{définition}\end{entrée}

\begin{entrée}{tɤβɣewa}{}{ⓔtɤβɣewa} 
\classe{n} 
\begin{définition}\pfra{veuf}\end{définition}
\begin{définition}\pcmn{鳏夫}\end{définition}\end{entrée}

\begin{entrée}{tɤ-βɣo}{}{ⓔtɤ-βɣo} 
\classe{np} \sens{1}
\begin{définition}\pfra{oncle (frère du père, mari de la sœur du père ou mari de la sœur de la mère)}\end{définition}
\begin{définition}\pcmn{伯父;叔叔}\end{définition}\sens{2}
\begin{définition}\pfra{lama}\end{définition}
\begin{définition}\pcmn{对喇嘛的尊称}\end{définition}\end{entrée}

\begin{entrée}{tɤ-βɟu}{}{ⓔtɤ-βɟu} 
\classe{np} 
\begin{définition}\pfra{matelas}\end{définition}
\begin{définition}\pcmn{褥子}\end{définition}
\begin{exemple}\pjya{a-βɟu}\hspace{5pt}\pcmn{我的褥子}\end{exemple}\end{entrée}

\begin{entrée}{tɤβri}{}{ⓔtɤβri} 
\classe{n} 
\begin{définition}\pfra{écheveau}\end{définition}
\begin{définition}\pcmn{一绞}\end{définition}
\begin{exemple}\pjya{tasa lɤ-pɣo-t-a, lɤ-rɯm-a, tɤβri tɤ-βzu-t-a}\hspace{5pt}\pcmn{我捻了大麻,搓成绞}\end{exemple}\end{entrée}

\begin{entrée}{tɤβʁa}{}{ⓔtɤβʁa} 
\classe{n} 
\begin{définition}\pfra{(avec) agressivité}\end{définition}
\begin{définition}\pcmn{气势汹汹}\end{définition}
\begin{exemple}\pjya{tɤβʁa kɯ kha jo-zɣɯt}\hspace{5pt}\pcmn{他气势汹汹地到了他家}\end{exemple}\relationsémantique{参考}{\lien{ⓔβʁa}{βʁa}}\end{entrée}

\begin{entrée}{tɤ-βzdɤr}{}{ⓔtɤ-βzdɤr} 
\classe{np} 
\begin{définition}\pfra{beurre (que l'on met dans le thé ou la tsampa)}\end{définition}
\begin{définition}\pcmn{加(在糌粑里、在茶里)的酥油}\end{définition}
\begin{exemple}\pjya{a-mu kɯ a-βzdɤr pa-lɤt}\hspace{5pt}\pcmn{母亲给我加了酥油}\end{exemple}\relationsémantique{参考}{\lien{ⓔβzdɤr}{βzdɤr}}\étymologie{sdor}\end{entrée}

\begin{entrée}{tɤ-cɤβ}{}{ⓔtɤ-cɤβ} 
\classe{np} 
\begin{définition}\pfra{espèce}\end{définition}
\begin{définition}\pcmn{种类}\end{définition}
\begin{exemple}\pjya{qarmɯrwa nɯ βʑɯ tɤ-cɤβ ŋu ɕi, pɣa tɤ-cɤβ ŋu mɤ-χsɤl}\hspace{5pt}\pcmn{不知道蝙蝠是老鼠的一种、还是鸟的一种}\end{exemple}\end{entrée}

\begin{entrée}{tɤ-chɯ}{}{ⓔtɤ-chɯ} 
\classe{np} 
\begin{définition}\pfra{coin}\end{définition}
\begin{définition}\pcmn{楔子}\end{définition}\étymologie{kʰʲewu}\end{entrée}

\begin{entrée}{tɤcoʁcoʁ}{}{ⓔtɤcoʁcoʁ} 
\classe{n} 
\begin{définition}\pfra{espèce d'oiseau}\end{définition}
\begin{définition}\pcmn{一种鸟}\end{définition}\end{entrée}

\begin{entrée}{tɤɕu}{}{ⓔtɤɕu} 
\classe{n} 
\begin{définition}\pfra{fraicheur}\end{définition}
\begin{définition}\pcmn{阴凉}\end{définition}
\begin{exemple}\pjya{tɤɕu ɣɤʑu (=ɲɯ-ɣɤɕu)}\hspace{5pt}\pcmn{(这里)很凉快}\end{exemple}\relationsémantique{参考}{\lien{ⓔnɤɕu}{nɤɕu}}\relationsémantique{参考}{\lien{ⓔɣɤɕu}{ɣɤɕu}}\end{entrée}

\begin{entrée}{tɤɕɤɣrum}{}{ⓔtɤɕɤɣrum} 
\classe{n} 
\begin{définition}\pfra{orge blanc}\end{définition}
\begin{définition}\pcmn{白青稞}\end{définition}\end{entrée}

\begin{entrée}{tɤɕɤɲaʁ}{}{ⓔtɤɕɤɲaʁ} 
\classe{n} 
\begin{définition}\pfra{orge noir}\end{définition}
\begin{définition}\pcmn{黑青稞}\end{définition}\end{entrée}

\begin{entrée}{tɤɕɤrloʁ}{}{ⓔtɤɕɤrloʁ} 
\classe{n} 
\begin{définition}\pfra{orge à barbe courte}\end{définition}
\begin{définition}\pcmn{短芒的青稞}\end{définition}\end{entrée}

\begin{entrée}{tɤɕɤrmbjɤβ}{}{ⓔtɤɕɤrmbjɤβ} 
\classe{n} 
\begin{définition}\pfra{orge en bottes}\end{définition}
\begin{définition}\pcmn{捆成一把的青稞杆}\end{définition}\end{entrée}

\begin{entrée}{tɤɕɤt}{}{ⓔtɤɕɤt} 
\classe{n} 
\begin{définition}\pfra{peigne}\end{définition}
\begin{définition}\pcmn{梳子}\end{définition}\étymologie{ɕad}\end{entrée}

\begin{entrée}{tɤɕi}{}{ⓔtɤɕi} 
\classe{n} 
\begin{définition}\pfra{orge}\end{définition}
\begin{définition}\pcmn{青稞}\end{définition}
\begin{exemple}\pjya{tɤɕi nɯ jiʑo ra ji-kɤ-ndza pjɯ-me mɤ-kɯ-khɯ ʑo ŋu, tɤɕi nɯ tɤ-ɬoʁ ɕɯmɯma tɕe, ɯ-jwaʁ rɟum tsa tɕe tɤ-rɲɟi tɕe ɯ-ku amtɕoʁ, ɯ-zrɤm dɤn, ɯ-ru lo-rɤrtsɯrtsɤɣ ŋu, ɯ-rtsɤɣ ɯ-pɤrthɤβ nɯ ɯ-ru tɯ-tɣa ro ro ntsɯ tu, ɯ-rtsɤɣ kɯβde jamar ta-lɤt tɕe, ɯ-kɯɕnom tu-lɤt ŋu. ɯ-kɯɕnom ɣɯ ɯ-ru nɯ lɤ-ɬoʁ tɕe ɯ-rtsɤɣ me, tɕe ɯ-kɯɕnom kɯ-wxti nɯ ra tɯ-tɣa kɯ-tu tu, ɯ-kɯɕnom ɯ-rdoʁ raŋri ɣɯ ɯ-ʑmbraʁ kɯ-zɯ-zri ʑo tu, ɯ-ʑmbraʁ wuma rʁom. tɤɕi nɯ wuma ʑo arɤphɤjqa tɕe tɯ-phɯ nɯ tɕu ɕnɤcɤ-ldʑa ŋgɯsqɯ-ldʑa jamar ɲɯ-βze kɯ-cha tu. tɤɕi nɯ tɯ-sqar ɯ-spa ŋu.}\hspace{5pt}\pcmn{青稞是我们必不可少的食物。青稞刚长出来的时候,叶子比较宽,顶端是尖的,根须多,茎是一节一节的,每一节之间的茎有一拃多长,一般长出四节就抽穗。穗干没有节,穗子有的有一拃长,穗子上每一颗粒上都有一根芒,十分粗糙。一颗青稞可以长出很多根苗,七八根,九十根的都有。青稞是糌粑的原料。}\end{exemple}\end{entrée}

\begin{entrée}{tɤɕime}{}{ⓔtɤɕime} 
\classe{n} 
\begin{définition}\pfra{jeune fille}\end{définition}
\begin{définition}\pcmn{小姐}\end{définition}\end{entrée}

\begin{entrée}{tɤɕiʑmbraʁ}{}{ⓔtɤɕiʑmbraʁ} 
\classe{n} 
\begin{définition}\pfra{barbe d'orge}\end{définition}
\begin{définition}\pcmn{青稞芒}\end{définition}\end{entrée}

\begin{entrée}{tɤ-ɕnɤz}{}{ⓔtɤ-ɕnɤz} 
\classe{np} 
\begin{définition}\pfra{bout d'un fil}\end{définition}
\begin{définition}\pcmn{线的一端}\end{définition}
\begin{exemple}\pjya{tɤ-ri ɲɤ-k-ɤɬɯt-ci tɕe ɯ-ɕnɤz mɯ-ɲɤ-sɤmto tɕe kɤ-ɕar ɲɯ-ra}\hspace{5pt}\pcmn{线乱了看不到头绪,要把它的一端找出来}\end{exemple}
\begin{exemple}\pjya{tɤ-ri ɯ-ɕnɤz ɲɯ́-wɣ-ɕar tɕe tú-wɣ-ɣɤrtɯm}\hspace{5pt}\pcmn{把线的一端找出来,(然后)把线缠起来}\end{exemple}\relationsémantique{参考}{\lien{ⓔtɯ-pɤɕnɤz}{tɯ-pɤɕnɤz}}\end{entrée}

\begin{entrée}{tɤɕpaʁ}{}{ⓔtɤɕpaʁ} 
\classe{n} 
\begin{définition}\pfra{soif}\end{définition}
\begin{définition}\pcmn{口干}\end{définition}
\begin{exemple}\pjya{tɤɕpaʁ ɲɤ-nɤɕqa tɕe mɯ-ko-tshi}\hspace{5pt}\pcmn{他忍住口干没有喝}\end{exemple}\relationsémantique{参考}{\lien{ⓔɕpaʁ}{ɕpaʁ}}\end{entrée}

\begin{entrée}{tɤ-ɕphɤt}{}{ⓔtɤ-ɕphɤt} 
\classe{np} 
\begin{définition}\pfra{pièce de tissu pour raccommoder les habits}\end{définition}
\begin{définition}\pcmn{补丁}\end{définition}
\begin{exemple}\pjya{tɯ-ŋga ɯ-ɕphɤt kɤ-ta-t-a}\hspace{5pt}\pcmn{衣服上打了补丁}\end{exemple}\relationsémantique{参考}{\lien{ⓔɕphɤt}{ɕphɤt}}\end{entrée}

\begin{entrée}{tɤɕphɤtta}{}{ⓔtɤɕphɤtta} 
\classe{n} 
\begin{définition}\pfra{type de pas d'aiguille}\end{définition}
\begin{définition}\pcmn{小针脚的缝法(沿着补丁的边缘)}\end{définition}\end{entrée}

\begin{entrée}{tɤɕqali}{}{ⓔtɤɕqali} 
\classe{n} 
\begin{définition}\pfra{cri}\end{définition}
\begin{définition}\pcmn{叫声}\end{définition}
\begin{exemple}\pjya{tɤɕqalɯli ta-βzu (=tɤ-ɣɤɕqali)}\hspace{5pt}\pcmn{大声地叫喊了一下}\end{exemple}\relationsémantique{参考}{\lien{ⓔɣɤɕqali}{ɣɤɕqali}}\end{entrée}

\begin{entrée}{tɤ-ɕqhe}{}{ⓔtɤ-ɕqhe} 
\classe{np} 
\begin{définition}\pfra{toux}\end{définition}
\begin{définition}\pcmn{咳嗽}\end{définition}
\begin{exemple}\pjya{nɤʑo tɤ-ɕqhe ɯ-ɲɯ-mna ?}\hspace{5pt}\pcmn{你的咳嗽好了没有?}\end{exemple}
\begin{exemple}\pjya{nɤʑo nɤ-ɕqhe ɯβrɤ-ɣɤʑu?}\hspace{5pt}\pcmn{你有没有咳嗽?}\end{exemple}
\begin{exemple}\pjya{nɤʑo nɤ-ɕqhe ɯ-ɲɯ-mna ?}\hspace{5pt}\pcmn{你的咳嗽好了没有?}\end{exemple}
\begin{exemple}\pjya{tɤ-ɕqhe ɣɤʑu ɕi ?}\hspace{5pt}\pcmn{你咳嗽吗?}\end{exemple}\relationsémantique{参考}{\lien{ⓔaɕqhe}{aɕqhe}}\end{entrée}

\begin{entrée}{tɤɕqraʁ}{}{ⓔtɤɕqraʁ} 
\classe{n} 
\begin{définition}\pfra{astuce}\end{définition}
\begin{définition}\pcmn{(耍)聪明,(耍)诡计}\end{définition}
\begin{exemple}\pjya{tɤɕqraʁ kɯ pjɯ-ʑɣɤsɯβʁa ɕti}\hspace{5pt}\pcmn{他耍了诡计就赢了}\end{exemple}
\begin{exemple}\pjya{ɯʑo kɯ tɤɕqraʁ ntsɯ tu-βze ŋgrɤl}\hspace{5pt}\pcmn{他总是耍聪明}\end{exemple}
\begin{exemple}\pjya{tɤɕqraʁ ma-tɤ-tɯ-βze}\hspace{5pt}\pcmn{不要耍聪明}\end{exemple}\relationsémantique{反义词}{\lien{ⓔtɤkhe}{tɤkhe}}\relationsémantique{参考}{\lien{ⓔɕqraʁ}{ɕqraʁ}}\end{entrée}

\begin{entrée}{tɤ-di}{}{ⓔtɤ-di} 
\classe{np} \sens{1}
\begin{définition}\pfra{odeur}\end{définition}
\begin{définition}\pcmn{气味}\end{définition}\sens{2}
\begin{définition}\pfra{puanteur}\end{définition}
\begin{définition}\pcmn{臭味}\end{définition}\relationsémantique{参考}{\lien{ⓔɯ-dɯɕŋaʁ}{ɯ-dɯɕŋaʁ}}\relationsémantique{参考}{\lien{ⓔɯ-dɯχɯn}{ɯ-dɯχɯn}}\relationsémantique{参考}{\lien{ⓔkɯɲidi}{kɯɲidi}}\relationsémantique{参考}{\lien{ⓔsɤŋɤβdi}{sɤŋɤβdi}}\end{entrée}

\begin{entrée}{tɤ-fkaβ}{}{ⓔtɤ-fkaβ} 
\classe{np} 
\begin{définition}\pfra{couvercle}\end{définition}
\begin{définition}\pcmn{盖子}\end{définition}\relationsémantique{参考}{\lien{ⓔfkaβ}{fkaβ}}\étymologie{ⁿgebs}\end{entrée}

\begin{entrée}{tɤ-fkɯm}{}{ⓔtɤ-fkɯm} 
\classe{np} 
\begin{définition}\pfra{récipient}\end{définition}
\begin{définition}\pcmn{可以装东西的物体(如口袋、盆子等)}\end{définition}\end{entrée}

\begin{entrée}{tɤ-fsa}{}{ⓔtɤ-fsa} 
\classe{np} 
\begin{définition}\pfra{piège}\end{définition}
\begin{définition}\pcmn{圈套;陷阱}\end{définition}
\begin{exemple}\pjya{kɯki qala ɯ-fsa tɤ-lat-a}\hspace{5pt}\pcmn{这是我给兔子布下的陷阱}\end{exemple}
\begin{exemple}\pjya{kɯki aʑo a-tɤ-fsa ŋu}\hspace{5pt}\pcmn{这是我的圈套}\end{exemple}\end{entrée}

\begin{entrée}{tɤfsaŋ}{}{ⓔtɤfsaŋ} 
\classe{n} 
\begin{définition}\pfra{feuilles de genévrier}\end{définition}
\begin{définition}\pcmn{柏树叶}\end{définition}\relationsémantique{参考}{\lien{ⓔfsaŋ}{fsaŋ}}\end{entrée}

\begin{entrée}{tɤ-fsɤri}{}{ⓔtɤ-fsɤri} 
\classe{np} 
\begin{définition}\pfra{ficelle en lin}\end{définition}
\begin{définition}\pcmn{麻绳}\end{définition}\relationsémantique{参考}{\lien{ⓔtasa}{tasa}}\relationsémantique{参考}{\lien{ⓔtɤ-ri}{tɤ-ri}}\relationsémantique{参考}{\lien{ⓔrɯfsɤri}{rɯfsɤri}}\end{entrée}

\begin{entrée}{tɤfsjit}{}{ⓔtɤfsjit} 
\classe{n} 
\begin{définition}\pfra{siffler}\end{définition}
\begin{définition}\pcmn{口哨}\end{définition}
\begin{exemple}\pjya{tɤfsjit ci thɯ-lat-a}\hspace{5pt}\pcmn{我吹了口哨}\end{exemple}\relationsémantique{参考}{\lien{ⓔrɤfsjit}{rɤfsjit}}\end{entrée}

\begin{entrée}{tɤfsɯr}{}{ⓔtɤfsɯr} 
\classe{n} 
\begin{définition}\pfra{cible}\end{définition}
\begin{définition}\pcmn{靶子}\end{définition}\end{entrée}

\begin{entrée}{tɤ-ftaʁ}{}{ⓔtɤ-ftaʁ} 
\classe{np} 
\begin{définition}\pfra{signe}\end{définition}
\begin{définition}\pcmn{记号}\end{définition}
\begin{exemple}\pjya{ɯ-ftaʁ to-ta}\hspace{5pt}\pcmn{他做了记号}\end{exemple}\étymologie{rtags}\end{entrée}

\begin{entrée}{tɤ-ftsa}{}{ⓔtɤ-ftsa} 
\classe{np} 
\begin{définition}\pfra{neveux (enfants de la sœur)}\end{définition}
\begin{définition}\pcmn{外甥}\end{définition}
\begin{exemple}\pjya{a-ftsa}\hspace{5pt}\pcmn{我的外甥}\end{exemple}\relationsémantique{参考}{\lien{ⓔkɤndʑɯrpɯftsa}{kɤndʑɯrpɯftsa}}\end{entrée}

\begin{entrée}{tɤɣ}{}{ⓔtɤɣ} 
\classe{num} 
\begin{définition}\pfra{un}\end{définition}
\begin{définition}\pcmn{一}\end{définition}\relationsémantique{同义词}{\lien{ⓔciⓗ2}{ci}}\end{entrée}

\begin{entrée}{tɤɣa/\variante{tɤɣal}}{}{ⓔtɤɣa} 
\classe{adv} 
\begin{définition}\pfra{visible}\end{définition}
\begin{définition}\pcmn{看得见的}\end{définition}
\begin{exemple}\pjya{tɤɣal ɣɤʑu tɕe, kɤfsɯfse ɲɯ-sɤmto}\hspace{5pt}\pcmn{完全看得见}\end{exemple}
\begin{exemple}\pjya{tɤɣal mɯ́j-rɤʑi tɕe, mɯ́j-sɤmto}\hspace{5pt}\pcmn{他不在看得见的地方,看不见他}\end{exemple}\end{entrée}

\begin{entrée}{tɤɣɤco}{}{ⓔtɤɣɤco} 
\classe{n}  
\grammaire{n.lieu} 
\begin{définition}\pfra{l'un des hameaux de Kamnyu}\end{définition}
\begin{définition}\pcmn{干木鸟的大队之一}\end{définition}\end{entrée}

\begin{entrée}{tɤ-ɣe}{}{ⓔtɤ-ɣe} 
\classe{np} 
\begin{définition}\pfra{petits enfants}\end{définition}
\begin{définition}\pcmn{孙子}\end{définition}\end{entrée}

\begin{entrée}{tɤ-ɣi}{₁}{ⓔtɤ-ɣiⓗ1} 
\classe{np} 
\begin{définition}\pfra{gens de la famille}\end{définition}
\begin{définition}\pcmn{家人}\end{définition}
\begin{exemple}\pjya{a-ɣi}\hspace{5pt}\pcmn{我的家人}\end{exemple}\end{entrée}

\begin{entrée}{tɤ-ɣi}{₂}{ⓔtɤ-ɣiⓗ2} 
\classe{np} 
\begin{définition}\pfra{glaise que l'on applique sur le toit}\end{définition}
\begin{définition}\pcmn{涂在房顶上的黄泥巴}\end{définition}\end{entrée}

\begin{entrée}{tɤɣɟaj}{₁}{ⓔtɤɣɟajⓗ1} 
\classe{n} 
\begin{définition}\pfra{fait de forcer}\end{définition}
\begin{définition}\pcmn{撬开}\end{définition}
\begin{exemple}\pjya{kɯm kɤ-cɯ mɯ́j-khɯ tɕe tɤɣɟaj tɤ-lɤt-i}\hspace{5pt}\pcmn{门打不开我们就把它撬开了}\end{exemple}\relationsémantique{参考}{\lien{ⓔnɤɣɟaj}{nɤɣɟaj}}\end{entrée}

\begin{entrée}{tɤ-ɣɟaj}{₂}{ⓔtɤ-ɣɟajⓗ2} 
\classe{np} 
\begin{définition}\pfra{rame}\end{définition}
\begin{définition}\pcmn{桨}\end{définition}
\begin{exemple}\pjya{ɯʑo kɯ tɤ-βɟaj ta-lɤt}\hspace{5pt}\pcmn{他划船了}\end{exemple}\relationsémantique{参考}{\lien{ⓔʑmbrɯβɟaj}{ʑmbrɯβɟaj}}\relationsémantique{参考}{\lien{ⓔtɕhaŋβɟaj}{tɕhaŋβɟaj}}\relationsémantique{参考}{\lien{}{nɤβɟaj}}\end{entrée}

\begin{entrée}{tɤɣle}{}{ⓔtɤɣle} 
\classe{n} 
\begin{définition}\pfra{bâton servant à maintenir à trame}\end{définition}
\begin{définition}\pcmn{拉住经线的木棒}\end{définition}\end{entrée}

\begin{entrée}{tɤ-ɣmbaj}{}{ⓔtɤ-ɣmbaj} 
\classe{np} \sens{1}
\begin{définition}\pfra{une face, surface}\end{définition}
\begin{définition}\pcmn{一面}\end{définition}\sens{2}
\begin{définition}\pfra{une page}\end{définition}
\begin{définition}\pcmn{一页}\end{définition}
\begin{exemple}\pjya{jɯɣi tɯ-tɤ-ɣmbaj pɯ-sthɯt-a}\hspace{5pt}\pcmn{我写完了一页}\end{exemple}\end{entrée}

\begin{entrée}{tɤ-ɣur}{}{ⓔtɤ-ɣur} 
\classe{np} 
\begin{définition}\pfra{haie}\end{définition}
\begin{définition}\pcmn{篱笆}\end{définition}
\begin{exemple}\pjya{ɯ-ɣur tu-βzu-nɯ tɕe ku-omdzɯ-nɯ ɲɯ-ŋu}\hspace{5pt}\pcmn{他们围着坐}\end{exemple}\end{entrée}

\begin{entrée}{tɤɣro}{}{ⓔtɤɣro} 
\classe{n} 
\begin{définition}\pfra{jeu}\end{définition}
\begin{définition}\pcmn{游戏}\end{définition}
\begin{exemple}\pjya{tɤ-pɤtso nɯ tɤɣro kɤ-βzu rga}\hspace{5pt}\pcmn{小孩子喜欢游戏}\end{exemple}
\begin{exemple}\pjya{ɯ-tɤɣro ra to-βzu}\hspace{5pt}\pcmn{他做了一下逗他玩的动作}\end{exemple}
\begin{sous-entrée}{tɤɣro tɤle}{ⓔtɤɣroⓝtɤɣro tɤle} 
\grammaire{emph} 
\begin{exemple}\pjya{tɤɣro tɤle kɯ ku-rɤʑi}\hspace{5pt}\pcmn{他一直都在玩}\end{exemple}\end{sous-entrée}

\end{entrée}

\begin{entrée}{tɤɣursi}{}{ⓔtɤɣursi} 
\classe{n} 
\begin{définition}\pfra{branches flexibles sur le balcon pour parer le vent}\end{définition}
\begin{définition}\pcmn{走缘边用来档风的树苗}\end{définition}
\begin{exemple}\pjya{jɤɣɤt laχtsɯ ɯ-pɤrthɤβ qale sɤ-tshi tɤ-kɤ-βzu si ɯ-mnɯ thɯ-kɤ-mphɯr nɯ, nɯ maʁ nɤ thɯ-kɤ-ndzri nɯ; rorʁe ɯ-taʁ pɯ-kɤ-sɤqatʂha nɯ tɤ-ɣursi rmi}\hspace{5pt}\pcmn{走缘柱头之间,用来挡风的树苗裹成或者拧成的,穿插在穿杆上的叫\lien{ⓔtɤɣursi}{tɤɣursi}}\end{exemple}\end{entrée}

\begin{entrée}{tɤjkɤspa}{}{ⓔtɤjkɤspa} 
\classe{n} 
\begin{définition}\pfra{navet (Brassica sp.)}\end{définition}
\begin{définition}\pcmn{芜菁【圆根】}\end{définition}\relationsémantique{同义词}{\lien{ⓔrasti}{rasti}}\relationsémantique{参考}{\lien{ⓔtɤjko}{tɤjko}}\end{entrée}

\begin{entrée}{tɤjko}{}{ⓔtɤjko} 
\classe{n} 
\begin{définition}\pfra{feuilles de navet}\end{définition}
\begin{définition}\pcmn{芜菁叶子【酸菜】}\end{définition}\end{entrée}

\begin{entrée}{tɤjkopu}{}{ⓔtɤjkopu} 
\classe{n} 
\begin{définition}\pfra{saucisse aux légumes}\end{définition}
\begin{définition}\pcmn{大肠}\end{définition}\relationsémantique{参考}{\lien{ⓔtɯ-pu}{tɯ-pu}}\end{entrée}

\begin{entrée}{tɤjkɯz}{}{ⓔtɤjkɯz} 
\classe{n} 
\begin{définition}\pfra{en secret}\end{définition}
\begin{définition}\pcmn{偷偷地}\end{définition}
\begin{exemple}\pjya{tɤjkɯz tu-βze ɲɯ-ŋu}\hspace{5pt}\pcmn{他瞒着别人做}\end{exemple}\relationsémantique{参考}{\lien{ⓔnɤjkɯz}{nɤjkɯz}}\end{entrée}

\begin{entrée}{tɤjlu}{}{ⓔtɤjlu} 
\classe{n} 
\begin{définition}\pfra{farine}\end{définition}
\begin{définition}\pcmn{面粉}\end{définition}\end{entrée}

\begin{entrée}{tɤjlɤβ}{}{ⓔtɤjlɤβ} 
\classe{n} 
\begin{définition}\pfra{vapeur}\end{définition}
\begin{définition}\pcmn{蒸汽}\end{définition}\relationsémantique{参考}{\lien{ⓔsɤjlɤβ}{sɤjlɤβ}}\end{entrée}

\begin{entrée}{tɤjlɤpi}{}{ⓔtɤjlɤpi} 
\classe{n} 
\begin{définition}\pfra{pâte}\end{définition}
\begin{définition}\pcmn{面团}\end{définition}\relationsémantique{参考}{\lien{ⓔtɤjlu}{tɤjlu}}\end{entrée}

\begin{entrée}{tɤ-jli}{}{ⓔtɤ-jli} 
\classe{np} 
\begin{définition}\pfra{valeur (d'une personne)}\end{définition}
\begin{définition}\pcmn{身价}\end{définition}
\begin{exemple}\pjya{tɤ-pɤtso ɯ-jli ɲɯ-wxti tɕe kɤ-sɤβlo ɲɯ-ɴqa}\hspace{5pt}\pcmn{小孩子很宝贝,不好伺候}\end{exemple}\end{entrée}

\begin{entrée}{tɤjmɤɣ}{}{ⓔtɤjmɤɣ} 
\classe{n} 
\begin{définition}\pfra{champignon}\end{définition}
\begin{définition}\pcmn{蘑菇}\end{définition}\end{entrée}

\begin{entrée}{tɤjmɤɣrʑɯɣ}{}{ⓔtɤjmɤɣrʑɯɣ} 
\classe{n} 
\begin{définition}\pfra{lamelles des champignons}\end{définition}
\begin{définition}\pcmn{菌褶}\end{définition}\end{entrée}

\begin{entrée}{tɤ-jme}{}{ⓔtɤ-jme} 
\classe{np} 
\begin{définition}\pfra{queue}\end{définition}
\begin{définition}\pcmn{尾巴}\end{définition}
\begin{exemple}\pjya{ɯ-jme}\hspace{5pt}\pcmn{它的尾巴}\end{exemple}\relationsémantique{参考}{\lien{ⓔjmɤrtaʁ}{jmɤrtaʁ}}\relationsémantique{参考}{\lien{ⓔjmɤlu}{jmɤlu}}\relationsémantique{参考}{\lien{ⓔɯ-kɤlɤjme}{ɯ-kɤlɤjme}}\end{entrée}

\begin{entrée}{tɤjmŋozdɯɣ}{}{ⓔtɤjmŋozdɯɣ} 
\classe{n} 
\begin{définition}\pfra{cauchemar}\end{définition}
\begin{définition}\pcmn{噩梦}\end{définition}
\begin{exemple}\pjya{tɤjmŋozdɯɣ pɯ-tu}\hspace{5pt}\pcmn{我做了噩梦}\end{exemple}\relationsémantique{参考}{\lien{ⓔnɤjmŋozdɯɣ}{nɤjmŋozdɯɣ}}\relationsémantique{参考}{\lien{ⓔtɯ-jmŋo}{tɯ-jmŋo}}\end{entrée}

\begin{entrée}{tɤ-jŋoʁ}{}{ⓔtɤ-jŋoʁ} 
\classe{np} 
\begin{définition}\pfra{crochet}\end{définition}
\begin{définition}\pcmn{钩子}\end{définition}\end{entrée}

\begin{entrée}{tɤjpa}{}{ⓔtɤjpa} 
\classe{n} 
\begin{définition}\pfra{neige}\end{définition}
\begin{définition}\pcmn{雪}\end{définition}
\begin{exemple}\pjya{tɤjpa ko-lɤt}\hspace{5pt}\pcmn{下雪了}\end{exemple}
\begin{exemple}\pjya{qartsɯ ja-zɣɯt tɤjpa ku-lɤt ŋgrɤl ɕti wo}\hspace{5pt}\pcmn{到了冬天就会下雪}\end{exemple}
\begin{exemple}\pjya{kutɕu ko tɤjpa mɯ-ka-lɤt}\hspace{5pt}\pcmn{这里倒没有下雪}\end{exemple}
\begin{exemple}\pjya{tɤjpa pɤjkhu mbarkhom @jieshang mɯ-pa-sɤzɣɯt}\hspace{5pt}\pcmn{雪还没有到马尔康街上}\end{exemple}\relationsémantique{参考}{\lien{ⓔarɯtɤjpa}{arɯtɤjpa}}\end{entrée}

\begin{entrée}{tɤjpɤqe}{}{ⓔtɤjpɤqe} 
\classe{n} 
\begin{définition}\pfra{espèce de corbeau}\end{définition}
\begin{définition}\pcmn{寒鸦}\end{définition}\relationsémantique{参考}{\lien{ⓔqajdo}{qajdo}}\end{entrée}

\begin{entrée}{tɤjpɣom}{}{ⓔtɤjpɣom} 
\classe{n} 
\begin{définition}\pfra{glace}\end{définition}
\begin{définition}\pcmn{冰}\end{définition}
\begin{exemple}\pjya{tɤjpɣom lɤ-k-ɤβzu-ci}\hspace{5pt}\pcmn{结了冰}\end{exemple}
\begin{exemple}\pjya{tɤjpɣom ko-ta}\hspace{5pt}\pcmn{结了冰}\end{exemple}
\begin{exemple}\pjya{tɯ-ci ɯ-taʁ tɤjpɣom kɤ-kɯ-ta}\hspace{5pt}\pcmn{结了冰的地方}\end{exemple}\relationsémantique{参考}{\lien{ⓔjpɣom}{jpɣom}}\end{entrée}

\begin{entrée}{tɤ-jroʁ}{}{ⓔtɤ-jroʁ} 
\classe{np} 
\begin{définition}\pfra{trace}\end{définition}
\begin{définition}\pcmn{痕迹}\end{définition}
\begin{exemple}\pjya{a-jroʁ}\hspace{5pt}\pcmn{我的痕迹}\end{exemple}
\begin{exemple}\pjya{ɯ-jroʁ jo-thɯ (jo-tɕɤt)}\hspace{5pt}\pcmn{他留下了痕迹}\end{exemple}\end{entrée}

\begin{entrée}{tɤjʁa}{}{ⓔtɤjʁa} 
\classe{n} 
\begin{définition}\pfra{col}\end{définition}
\begin{définition}\pcmn{垭口}\end{définition}\end{entrée}

\begin{entrée}{tɤjsaʁ}{}{ⓔtɤjsaʁ} 
\classe{n} 
\begin{définition}\pfra{débris}\end{définition}
\begin{définition}\pcmn{赃物;落叶}\end{définition}
\begin{exemple}\pjya{sɤtɕha pjɤ-mbɯt tɕe, ndzom ɯ-pa tɤjsaʁ cho-ɣi}\hspace{5pt}\pcmn{地塌下来了,漂浮物流到桥下}\end{exemple}\relationsémantique{同义词}{\lien{}{tɤɲɟoʁɲɟi}}\end{entrée}

\begin{entrée}{tɤ-jtsi}{}{ⓔtɤ-jtsi} 
\classe{np} 
\begin{définition}\pfra{pilier}\end{définition}
\begin{définition}\pcmn{柱子}\end{définition}\end{entrée}

\begin{entrée}{tɤ-jwaʁ}{}{ⓔtɤ-jwaʁ} 
\classe{np} \paradigme{comit}{kɤ́jwɯjwaʁ}
\begin{définition}\pfra{feuille}\end{définition}
\begin{définition}\pcmn{叶子}\end{définition}
\begin{exemple}\pjya{kɤ́jwɯjwaʁ ʑo nɯ-phɯt-a}\hspace{5pt}\pcmn{我连着叶子一起砍掉了}\end{exemple}
\begin{exemple}\pjya{sɯjwaʁ}\hspace{5pt}\pcmn{树叶}\end{exemple}\end{entrée}

\begin{entrée}{tɤkɤɣrum}{}{ⓔtɤkɤɣrum} 
\classe{n} 
\begin{définition}\pfra{à la chevelure blanche}\end{définition}
\begin{définition}\pcmn{白发}\end{définition}
\begin{exemple}\pjya{rgɤtpu tɤkɤɣrum}\hspace{5pt}\pcmn{白发老头}\end{exemple}\relationsémantique{参考}{\lien{ⓔtɯ-ku}{tɯ-ku}}\relationsémantique{参考}{\lien{ⓔwɣrum}{wɣrum}}\end{entrée}

\begin{entrée}{tɤ-kɤrme}{}{ⓔtɤ-kɤrme} 
\classe{np} 
\begin{définition}\pfra{cheveux}\end{définition}
\begin{définition}\pcmn{头发}\end{définition}\relationsémantique{参考}{\lien{ⓔtɯ-ku}{tɯ-ku}}\relationsémantique{参考}{\lien{ⓔtɤ-rme}{tɤ-rme}}\end{entrée}

\begin{entrée}{tɤ-kɤrtshi}{}{ⓔtɤ-kɤrtshi} 
\classe{np} 
\begin{définition}\pfra{épaisseur des cheveux}\end{définition}
\begin{définition}\pcmn{头发的密度}\end{définition}
\begin{exemple}\pjya{a-kɤrtshi ɲɯ-mba}\hspace{5pt}\pcmn{我的头发很稀疏}\end{exemple}
\begin{exemple}\pjya{a-kɤrtshi ɲɯ-jaʁ}\hspace{5pt}\pcmn{我的头发很密}\end{exemple}\relationsémantique{参考}{\lien{ⓔtɯ-ku}{tɯ-ku}}\end{entrée}

\begin{entrée}{tɤkɤzbɣaʁ}{}{ⓔtɤkɤzbɣaʁ} 
\classe{n} 
\begin{définition}\pfra{migraine}\end{définition}
\begin{définition}\pcmn{头风病}\end{définition}
\begin{exemple}\pjya{tɤkɤzbɣaʁ nɯ jɯsŋi soz tɕe a-tɤ-ʑe tɕe, fsosoz tɕe ɯ-tɯ-mŋɤm ɲɯ-ʑi ŋu}\hspace{5pt}\pcmn{头风病,如果今天早上得了这个病,要到明天早上才能好起来。}\end{exemple}\end{entrée}

\begin{entrée}{tɤ-kɤʑmbri}{}{ⓔtɤ-kɤʑmbri} 
\classe{np} 
\begin{définition}\pfra{insolation}\end{définition}
\begin{définition}\pcmn{中暑}\end{définition}
\begin{exemple}\pjya{tɤ-kɤʑmbri nɯ, nɯ-ɣɯtshɤdɯɣ nɯ-tɕhom tɕe tɯ-kɯr ɯ-ŋgɯ tɯ-mdʑu ɯ-taʁ tɤ-ndɤr ɲɯ-ɬoʁ ŋu tɕe wuma ʑo sɤɣdɯɣ. tɤ-pɤtso nɯ-kɤʑmbri kɯ-ɣi ŋgrɤl}\hspace{5pt}\pcmn{气候太热容易导致中暑。嘴里,舌头上出很多痘痘,很难受。小孩子中暑得比较多。}\end{exemple}
\begin{exemple}\pjya{tɤŋe wuma ʑo ɲɯ-sɤɕke tɕe a-kɤʑmbri pjɯ-sɯɣe ɲɯ-ŋu}\hspace{5pt}\pcmn{太阳很晒,令我中暑了}\end{exemple}\end{entrée}

\begin{entrée}{tɤkha}{}{ⓔtɤkha} 
\classe{cnj} 
\begin{définition}\pfra{au moment de}\end{définition}
\begin{définition}\pcmn{临...之前}\end{définition}\end{entrée}

\begin{entrée}{tɤkhe}{}{ⓔtɤkhe} 
\classe{n} 
\begin{définition}\pfra{imbécile}\end{définition}
\begin{définition}\pcmn{傻瓜}\end{définition}\relationsémantique{参考}{\lien{ⓔkhe}{khe}}\end{entrée}

\begin{entrée}{tɤkhepɣɤtɕɯ}{}{ⓔtɤkhepɣɤtɕɯ} 
\classe{n} 
\begin{définition}\pfra{Emberiza sp.}\end{définition}
\begin{définition}\pcmn{鹀}\end{définition}
\begin{exemple}\pjya{tɤkhepɣɤtɕɯ nɯ pɣa kɯ-xtɕi ci ŋu, phu nɯ pɣi ri ɯ-mke cho ɯ-rqo pa nɯ ra kɯ-ɣɯrni ŋu, ɯ-mi kɯ-qarŋe ŋu, ɯ-jme ɯ-ku nɯ ra kɯ-wɣrumtɕe mpɕɤr, mu nɯ mɤ-mpɕɤr, pha ɯ-phoŋbu ʑo pɣi, ɯ-taʁ kɯ-ɲaʁ kɯ-khra tu, tɤ-khe pɣɤtɕɯ nɯ kɯ qajɯ cho tɤ-rɤku tu-ndze ŋu, tɕe ɯ-kɯ-qha dɤn, tɕe pɣɤtɕɯ nɯ ɯ-ŋgɯz khe tu-kɯ-ti ɲɯ-ŋu. kha ɯ-rkarkɯ ra kɤ-rɤʑi χɕu. ɯ-loʁ nɯ tʂu ɯ-rkɯ kɯ-ɤrmbat ʑo sɯphɯ kɯ-xtɕi ɯ-qa, xɕaj ɯ-qa ra ku-βze ŋu tɕe, nɯɣɯmto. tɤkhepɣɤtɕɯ ɲɯ-khe tɕe βʑar kɯ ɯʑo ɯ-loʁ ɯ-ŋgɯ ɯ-ŋgɯm nɯ ra tu-ndze tɕe, ɯ-sta nɯ tɕu βʑar kɯ ɯʑo ɯ-ŋgɯm ko-lɤt tɕe, ɯ-ŋgɯm nɯ nɯ-ʁaʁ tɕe chɯ-wxti ɲɯ-ŋu tɕe, tɕe tɤkhepɣɤtɕɯ kɯ sɤ-mɯ-mu ʑo ku-χse ɲɯ-ra.}\hspace{5pt}\pcmn{\lien{}{tɤkhe pɣɤtɕɯ}是一种小鸟,公的虽然是灰的,颈和脖子下面带有红色,脚是黄色的,尾巴顶端有白色,很漂亮。母的不漂亮,全身是灰色的,上面还带有黑色的斑点。\lien{}{tɤkhe pɣɤtɕɯ}吃虫子和粮食,很多人讨厌它。据说它在鸟类当中是比较笨的一只。它喜欢在房子周边活动,把窝打在离路边不远的小树和草丛底下,容易发现。因为\lien{}{tɤkhe pɣɤtɕɯ}笨,所以鹞子就会在它的窝里吃掉它的蛋,然后把自己的蛋下在里面。蛋孵出了以后,小鹞子会长大,它也只好带着害怕的心理喂养它们。}\end{exemple}\end{entrée}

\begin{entrée}{tɤkhespɤlbu}{}{ⓔtɤkhespɤlbu} 
\classe{n} 
\begin{définition}\pfra{bêta}\end{définition}
\begin{définition}\pcmn{傻乎乎}\end{définition}\end{entrée}

\begin{entrée}{tɤ-khrɤl}{}{ⓔtɤ-khrɤl} 
\classe{np} 
\begin{définition}\pfra{prix à payer}\end{définition}
\begin{définition}\pcmn{自己应该承担的责任;应该付出的代价}\end{définition}
\begin{exemple}\pjya{ki a-khrɤl ŋu}\hspace{5pt}\pcmn{这是我的责任}\end{exemple}
\begin{exemple}\pjya{a-khrɤl nɯ-n-nɤma-t-a}\hspace{5pt}\pcmn{我做了我的那一份(工作)}\end{exemple}\étymologie{kʰral}\end{entrée}

\begin{entrée}{tɤ-khɯ}{}{ⓔtɤ-khɯ} 
\classe{np} 
\begin{définition}\pfra{fumée}\end{définition}
\begin{définition}\pcmn{烟}\end{définition}
\begin{exemple}\pjya{tɤ-khɯ ta-tɕɤt}\hspace{5pt}\pcmn{他求了烟}\end{exemple}\relationsémantique{参考}{\lien{ⓔnɤkhɯ}{nɤkhɯ}}\relationsémantique{参考}{\lien{ⓔɣɤkhɯⓗ1}{ɣɤkhɯ₁}}\relationsémantique{参考}{\lien{ⓔsɤkhɯ}{sɤkhɯ}}\relationsémantique{参考}{\lien{ⓔkhɯɣɲɟɯ}{khɯɣɲɟɯ}}\end{entrée}

\begin{entrée}{tɤkhɯɣɲɟɯ}{}{ⓔtɤkhɯɣɲɟɯ} 
\classe{n} 
\begin{définition}\pfra{cheminée}\end{définition}
\begin{définition}\pcmn{烟囱}\end{définition}\relationsémantique{参考}{\lien{ⓔɯ-ɣɲɟɯ}{ɯ-ɣɲɟɯ}}\end{entrée}

\begin{entrée}{tɤkhɯrɲɯl}{}{ⓔtɤkhɯrɲɯl} 
\classe{n} 
\begin{définition}\pfra{fumée qui ne se dissipe pas}\end{définition}
\begin{définition}\pcmn{驱散不了;在山腰停留的烟子(人为的烟子、炊烟)}\end{définition}
\begin{exemple}\pjya{tɤkhɯrɲɯl ɣɤʑu}\hspace{5pt}\pcmn{有驱散不了的烟子}\end{exemple}\end{entrée}

\begin{entrée}{tɤkusci}{}{ⓔtɤkusci} 
\classe{n} 
\begin{définition}\pfra{il était une fois}\end{définition}
\begin{définition}\pcmn{故事的开头语}\end{définition}\end{entrée}

\begin{entrée}{tɤ-lu}{}{ⓔtɤ-lu} 
\classe{np} \paradigme{comit}{kɤ́lɯlu}
\begin{définition}\pfra{lait}\end{définition}
\begin{définition}\pcmn{奶汁}\end{définition}
\begin{exemple}\pjya{tɤ-lu pɯ-tɕɤt}\hspace{5pt}\pcmn{你挤奶吧}\end{exemple}\relationsémantique{参考}{\lien{ⓔnɤlu}{nɤlu}}\end{entrée}

\begin{entrée}{tɤlɤβʑɤzu}{}{ⓔtɤlɤβʑɤzu} 
\classe{n} 
\begin{définition}\pfra{trompettes de la mort}\end{définition}
\begin{définition}\pcmn{灰喇叭菌}\end{définition}
\begin{exemple}\pjya{tɤlɤβʑɤzu nɯ tɯrgi ɯ-ŋgɯ tu-ɬoʁ ŋu, ɯ-mdoʁ nɯ qromke mdoʁ ŋu, ɯ-tshɯɣa nɯ @laba ɯ-taʁ tɤ-kɤ-ɕthɯz kɯ-fse ŋu, kɤ-ndza sna, ɯ-rʑɯɣ me, ndoʁ}\hspace{5pt}\pcmn{灰喇叭菌长在杉木林里,紫色,形状像朝天的喇叭,可以吃,没有菌褶,脆。}\end{exemple}\end{entrée}

\begin{entrée}{tɤlɤɕom}{}{ⓔtɤlɤɕom} 
\classe{n} 
\begin{définition}\pfra{peau du lait}\end{définition}
\begin{définition}\pcmn{奶皮}\end{définition}\relationsémantique{参考}{\lien{ⓔɕomⓗ2}{ɕom₂}}\end{entrée}

\begin{entrée}{tɤlɤndʑu}{}{ⓔtɤlɤndʑu} 
\classe{n} 
\begin{définition}\pfra{bâton à baratter}\end{définition}
\begin{définition}\pcmn{搅奶的棍子}\end{définition}\relationsémantique{参考}{\lien{ⓔtɤ-lu}{tɤ-lu}}\relationsémantique{参考}{\lien{ⓔndʑu}{ndʑu}}\end{entrée}

\begin{entrée}{tɤlɤɴqhi}{}{ⓔtɤlɤɴqhi} 
\classe{n} 
\begin{définition}\pfra{lait séché (sur les casseroles)}\end{définition}
\begin{définition}\pcmn{干了的奶渍(锅子上)}\end{définition}\relationsémantique{参考}{\lien{ⓔɴqhi}{ɴqhi}}\relationsémantique{参考}{\lien{ⓔtɤ-lu}{tɤ-lu}}\end{entrée}

\begin{entrée}{tɤlɤrpjɯ}{}{ⓔtɤlɤrpjɯ} 
\classe{n} 
\begin{définition}\pfra{lait caillé}\end{définition}
\begin{définition}\pcmn{变质了的牛奶}\end{définition}\relationsémantique{参考}{\lien{ⓔtɤ-lu}{tɤ-lu}}\relationsémantique{参考}{\lien{ⓔrpjɯ}{rpjɯ}}\end{entrée}

\begin{entrée}{tɤlɤtshaʁ}{}{ⓔtɤlɤtshaʁ} 
\classe{n} 
\begin{définition}\pfra{filtre à lait}\end{définition}
\begin{définition}\pcmn{用来过滤牛奶的瓢}\end{définition}\relationsémantique{参考}{\lien{ⓔtɤ-lu}{tɤ-lu}}\relationsémantique{参考}{\lien{ⓔtshaʁ}{tshaʁ}}\end{entrée}

\begin{entrée}{tɤlɤxchi}{}{ⓔtɤlɤxchi} 
\classe{n} 
\begin{définition}\pfra{lait frais}\end{définition}
\begin{définition}\pcmn{新鲜牛奶}\end{définition}\relationsémantique{参考}{\lien{ⓔtɤ-lu}{tɤ-lu}}\relationsémantique{参考}{\lien{ⓔchi}{chi}}\end{entrée}

\begin{entrée}{tɤlɤxtɕur}{}{ⓔtɤlɤxtɕur} 
\classe{n} 
\begin{définition}\pfra{lait aigre}\end{définition}
\begin{définition}\pcmn{酸奶,把酥油打出来以后剩下的奶水}\end{définition}\relationsémantique{参考}{\lien{ⓔtɤ-lu}{tɤ-lu}}\relationsémantique{参考}{\lien{ⓔtɕurⓗ1}{tɕur₁}}\end{entrée}

\begin{entrée}{tɤlɟɣo,lɤt}{}{ⓔtɤlɟɣo,lɤt} 
\classe{n}
\classe{vt} \paradigme{dir}{kɤ-}
\begin{définition}\pfra{attraper au collet (cheval, bovidé)}\end{définition}
\begin{définition}\pcmn{套(动物)}\end{définition}\relationsémantique{Component 1}{\lien{}{tɤlɟɣo}}\relationsémantique{Component 2}{\lien{}{lɤt}}\relationsémantique{参考}{\lien{ⓔlɤtⓗ1}{lɤt₁}}\end{entrée}

\begin{entrée}{tɤlmɯz}{}{ⓔtɤlmɯz} 
\classe{n} 
\begin{définition}\pfra{branches ou paille dont on recouvre les balcons}\end{définition}
\begin{définition}\pcmn{铺在走缘地上的泥土下面的麦草;豌豆;枝桠,用来防止泥土漏掉}\end{définition}\end{entrée}

\begin{entrée}{tɤ-loʁ}{₁}{ⓔtɤ-loʁⓗ1} 
\classe{np} \sens{1}
\begin{définition}\pfra{terrier, nid}\end{définition}
\begin{définition}\pcmn{鸟窝,野兽的洞}\end{définition}\sens{2}
\begin{définition}\pfra{berceau}\end{définition}
\begin{définition}\pcmn{摇篮}\end{définition}\end{entrée}

\begin{entrée}{tɤ-loʁ}{₂}{ⓔtɤ-loʁⓗ2} 
\classe{np} 
\begin{définition}\pfra{anneau}\end{définition}
\begin{définition}\pcmn{圆圈}\end{définition}\end{entrée}

\begin{entrée}{tɤlɯlɤt}{}{ⓔtɤlɯlɤt} 
\classe{n} 
\begin{définition}\pfra{guerre}\end{définition}
\begin{définition}\pcmn{战争}\end{définition}
\begin{exemple}\pjya{tɤlɯlɤt to-rɤru tɕe tɯŋgo jo-ɣi}\hspace{5pt}\pcmn{发生了战乱和瘟疫}\end{exemple}
\begin{exemple}\pjya{tɤlɯlɤt to-βzu-nɯ}\hspace{5pt}\pcmn{他们打仗了}\end{exemple}\relationsémantique{参考}{\lien{ⓔalɯlɤt}{alɯlɤt}}\end{entrée}

\begin{entrée}{tɤ-ɬaʁ}{}{ⓔtɤ-ɬaʁ} 
\classe{np} 
\begin{définition}\pfra{tante (sœur de la mère, femme du frère du père, femme du frère de la mère, femme du frère)}\end{définition}
\begin{définition}\pcmn{姨母;伯母;婶母;舅母;嫂子}\end{définition}
\begin{exemple}\pjya{a-ɬaʁ}\hspace{5pt}\pcmn{我的姨母}\end{exemple}\end{entrée}

\begin{entrée}{tɤ-mu}{}{ⓔtɤ-mu} 
\classe{np} 
\begin{définition}\pfra{mère}\end{définition}
\begin{définition}\pcmn{母亲}\end{définition}
\begin{exemple}\pjya{a-mu a-wa}\hspace{5pt}\pcmn{我的父母}\end{exemple}\end{entrée}

\begin{entrée}{tɤ-ma}{}{ⓔtɤ-ma} 
\classe{np} 
\begin{définition}\pfra{mère (noble)}\end{définition}
\begin{définition}\pcmn{母亲(贵族用语)}\end{définition}
\begin{exemple}\pjya{a-pa a-ma}\hspace{5pt}\pcmn{我父母}\end{exemple}\end{entrée}

\begin{entrée}{tɤ-mɤtɕɤz}{}{ⓔtɤ-mɤtɕɤz} 
\classe{np} 
\begin{définition}\pfra{trace de pied}\end{définition}
\begin{définition}\pcmn{脚印}\end{définition}
\begin{exemple}\pjya{a-mɤtɕɤz}\hspace{5pt}\pcmn{我的脚印}\end{exemple}\relationsémantique{参考}{\lien{ⓔtɤ-tɕɤz}{tɤ-tɕɤz}}\end{entrée}

\begin{entrée}{tɤ-mɤtsa}{}{ⓔtɤ-mɤtsa} 
\classe{np} 
\begin{définition}\pfra{cousin}\end{définition}
\begin{définition}\pcmn{堂兄弟姐妹}\end{définition}\end{entrée}

\begin{entrée}{tɤmbɤt}{}{ⓔtɤmbɤt} 
\classe{n} 
\begin{définition}\pfra{montagne}\end{définition}
\begin{définition}\pcmn{山}\end{définition}\relationsémantique{同义词}{\lien{ⓔzgo}{zgo}}\relationsémantique{参考}{\lien{ⓔlɤqhɤtɤmbɤt}{lɤqhɤtɤmbɤt}}\end{entrée}

\begin{entrée}{tɤ-mbe}{}{ⓔtɤ-mbe} 
\classe{np} \sens{1}
\begin{définition}\pfra{vieux et abîmé}\end{définition}
\begin{définition}\pcmn{破旧的}\end{définition}\sens{2}
\begin{définition}\pfra{habit rapiécé}\end{définition}
\begin{définition}\pcmn{烂衣服(背缝缝补补很多次的破旧衣服)}\end{définition}\relationsémantique{参考}{\lien{ⓔmbe}{mbe}}\relationsémantique{参考}{\lien{}{ɯ-mbambe}}\end{entrée}

\begin{entrée}{tɤmbextsa}{}{ⓔtɤmbextsa} 
\classe{n} 
\begin{définition}\pfra{botte faite de lin, de laine et de cuir}\end{définition}
\begin{définition}\pcmn{用麻布、羊毛和皮子作成的靴子}\end{définition}\end{entrée}

\begin{entrée}{tɤmbɣo}{}{ⓔtɤmbɣo} 
\classe{n} 
\begin{définition}\pfra{sourd}\end{définition}
\begin{définition}\pcmn{聋子}\end{définition}\relationsémantique{参考}{\lien{ⓔɣɤmbɣo}{ɣɤmbɣo}}\end{entrée}

\begin{entrée}{tɤ-mbrɯ}{}{ⓔtɤ-mbrɯ} 
\classe{np} \paradigme{emphatic}{tɤmbrɯ tɤʁɟa}
\begin{définition}\pfra{colère}\end{définition}
\begin{définition}\pcmn{生气(状态)}\end{définition}
\begin{exemple}\pjya{tɤmbrɯ tɤʁɟa kɯ ku-rɤʑi}\hspace{5pt}\pcmn{他非常生气}\end{exemple}\relationsémantique{参考}{\lien{ⓔtɤ-mbrɯ,ŋgɯ}{tɤ-mbrɯ,ŋgɯ}}\end{entrée}

\begin{entrée}{tɤmbrɯm}{}{ⓔtɤmbrɯm} 
\classe{n} 
\begin{définition}\pfra{rougeole}\end{définition}
\begin{définition}\pcmn{疹子}\end{définition}\étymologie{ⁿbrum}\end{entrée}

\begin{entrée}{tɤ-mbrɯ,ŋgɯ}{}{ⓔtɤ-mbrɯ,ŋgɯ} 
\classe{np}
\classe{vi} \paradigme{dir}{tɤ-}
\begin{définition}\pfra{s'énerver}\end{définition}
\begin{définition}\pcmn{生气}\end{définition}
\begin{exemple}\pjya{a-taʁ ɯ-mbrɯ ɲɯ-ŋgɯ}\hspace{5pt}\pcmn{他生我的气}\end{exemple}\relationsémantique{Component 1}{\lien{ⓔtɤ-mbrɯ}{tɤ-mbrɯ}}\relationsémantique{Component 2}{\lien{ⓔŋgɯ}{\_ŋgɯ}}
\begin{sous-entrée}{tɤ-mbrɯ,ɕɯŋgɯ}{ⓔtɤ-mbrɯ,ŋgɯⓝtɤ-mbrɯ,ɕɯŋgɯ} 
\classe{vt}
\classe{np}
\classe{vt}  
\grammaire{caus} \paradigme{dir}{tɤ-}
\begin{définition}\pfra{énerver}\end{définition}
\begin{définition}\pcmn{惹人生气}\end{définition}
\begin{exemple}\pjya{a-mbrɯ ta-ɕɯŋgɯ}\hspace{5pt}\pcmn{他惹我生气了}\end{exemple}\relationsémantique{Component 1}{\lien{ⓔtɤ-mbrɯ}{tɤ-mbrɯ}}\relationsémantique{Component 2}{\lien{ⓔɕɯŋgɯⓗ2}{\_ɕɯŋgɯ}}\relationsémantique{参考}{\lien{ⓔsɤmbrɯ}{sɤmbrɯ}}\relationsémantique{参考}{\lien{ⓔsɤmbrɯŋgɯ}{sɤmbrɯŋgɯ}}\end{sous-entrée}

\end{entrée}

\begin{entrée}{tɤmcar}{}{ⓔtɤmcar} 
\classe{n} 
\begin{définition}\pfra{pinces}\end{définition}
\begin{définition}\pcmn{火钳}\end{définition}\relationsémantique{参考}{\lien{ⓔɯ-tɤmcar}{ɯ-tɤmcar}}\end{entrée}

\begin{entrée}{tɤ-mdɯ}{}{ⓔtɤ-mdɯ} 
\classe{np} 
\begin{définition}\pfra{neveux (enfants du frère)}\end{définition}
\begin{définition}\pcmn{侄子}\end{définition}
\begin{exemple}\pjya{a-mdɯ}\hspace{5pt}\pcmn{我的侄子}\end{exemple}\relationsémantique{参考}{\lien{ⓔkɤndʑɯpɤmdɯ}{kɤndʑɯpɤmdɯ}}\end{entrée}

\begin{entrée}{tɤ-mdzu}{}{ⓔtɤ-mdzu} 
\classe{np} 
\begin{définition}\pfra{épine}\end{définition}
\begin{définition}\pcmn{刺}\end{définition}\end{entrée}

\begin{entrée}{tɤmdzɤqaqa}{}{ⓔtɤmdzɤqaqa} 
\classe{n} 
\begin{définition}\pfra{pousses du \lien{ⓔɴɢolo}{ɴɢolo}}\end{définition}
\begin{définition}\pcmn{\lien{ⓔɴɢolo}{ɴɢolo}的新苗}\end{définition}\end{entrée}

\begin{entrée}{tɤmdzɤrgi}{}{ⓔtɤmdzɤrgi} 
\classe{n} 
\begin{définition}\pfra{chardon}\end{définition}
\begin{définition}\pcmn{大蓟}\end{définition}
\begin{exemple}\pjya{tɤmdzɤrgi nɯ sɯjno ci ŋu, ɯ-zrɤm wɣrum khro mɤ-wxti, nɯɕɯmɯma ʑo kɤ-phɯt khɯ, ɯ-jwaʁ nɯ ɯ-thoʁ pjɯ-tɯɣ ʑo ŋu, ɯ-jwaʁ ɯ-mdzu wuma ʑo dɤn, ɯ-jwaʁ ɯ-βzɯr ɣɯ ɯ-mdzu nɯ mɤʑɯ ʑo mtɕoʁ cho rɲɟi. ɯ-χcɤl ɯ-ru tu-ɬoʁ tɕe, ɯ-βzɯr tu. ɯ-ru ɯ-taʁ nɯ ra kɯnɤ ɯ-mdzu tu. ɯ-ru tɤ-zri tsa tɕe, li ɯ-jwaʁ ɲɯ-ɬoʁ tɕe, ɯ-mɯntoʁ ɲɯ-lɤt ŋu. ɯ-mɯntoʁ ʁmɤrsmɯɣ tsa ŋu, ɯ-mɯntoʁ pɯ-ŋgra tɕe, ɯ-rɣi ɲɯ-βze tɕe, ɯ-rɣi ɯ-ku zɯ li ɯ-rme kɯ-fse tu, qale kɯ ju-nɯtsɯm cha. tɤmdzɤrgi nɯ ɲɯ́-wɣ-phɯt tɕe, ɯ-di ci tu.}\hspace{5pt}\pcmn{大蓟是一种植物,根是白色的,长得不大,一下子就可以拔掉,叶子贴在地面上,叶子上长满刺,叶子边的刺(比叶面的刺)长和尖。中间长茎,有棱角。茎上也长有刺。茎长高后,又长叶子,开花。花是紫色的,花凋谢后,就结种子,种子上也长有细毛状的东西,可以被风吹走。扯大蓟时会发出臭味。}\end{exemple}\end{entrée}

\begin{entrée}{tɤ-muj}{}{ⓔtɤ-muj} 
\classe{np} 
\begin{définition}\pfra{plumes}\end{définition}
\begin{définition}\pcmn{羽毛}\end{définition}\relationsémantique{参考}{\lien{ⓔpɣɤmuj}{pɣɤmuj}}\end{entrée}

\begin{entrée}{tɤ-mkɯm}{}{ⓔtɤ-mkɯm} 
\classe{np} 
\begin{définition}\pfra{oreiller}\end{définition}
\begin{définition}\pcmn{枕头}\end{définition}
\begin{exemple}\pjya{aʑo a-mkɯm}\hspace{5pt}\pcmn{我的枕头}\end{exemple}\end{entrée}

\begin{entrée}{tɤmphoʁ}{}{ⓔtɤmphoʁ} 
\classe{n} 
\begin{définition}\pfra{tsampa}\end{définition}
\begin{définition}\pcmn{糌粑的一种吃法}\end{définition}
\begin{exemple}\pjya{khɯtsa ɯ-ŋgɯ tʂha ɯ-qiɯ kɯ-xtɕi tú-wɣ-rku tɕe ɯ-taʁ tɯ-sqar pjɯ́-wɣ-lɤt tɕe tɕhɯβroʁ staʁnɤ kɯ-spɯ tsa ɲɯ́-wɣ-ɕmi tɕe tú-wɣ-ndza tɕe nɯnɯ tɤmphoʁ rmi.}\hspace{5pt}\pcmn{在碗里倒小半碗的水,再放上少量糌粑,搅均匀后就可以吃。比\lien{ⓔtɕhɯβroʁ}{tɕhɯβroʁ}干一点 。这种吃法叫作\lien{ⓔtɤmphoʁ}{tɤmphoʁ}。}\end{exemple}\end{entrée}

\begin{entrée}{tɤ-mtɕar}{}{ⓔtɤ-mtɕar} 
\classe{np} 
\begin{définition}\pfra{pièce de tissu triangulaire utilisée dans les habits tibétains}\end{définition}
\begin{définition}\pcmn{藏式服装中的长三角形的布料}\end{définition}
\begin{exemple}\pjya{kɯrɯŋga chɯ́-wɣ-tʂɯβ tɕe, tɯ-ŋga ɣɯ χchoʁe ʑo tɤ-mtɕar tú-wɣ-lɤt ra ma nɯ mɤɕtʂa mɤ-nɯɣɯŋke}\hspace{5pt}\pcmn{缝藏装时,衣服的左右两边必须要缝上三角形布料,不然不便于走路。}\end{exemple}\end{entrée}

\begin{entrée}{tɤ-mtɕho}{}{ⓔtɤ-mtɕho} 
\classe{np} 
\begin{définition}\pfra{cale, coin}\end{définition}
\begin{définition}\pcmn{楔子【尖】}\end{définition}
\begin{exemple}\pjya{qaʁ ɯ-mtɕho}\hspace{5pt}\pcmn{锄头的楔子}\end{exemple}\relationsémantique{同义词}{\lien{ⓔtɤ-chɯ}{tɤ-chɯ}}\end{entrée}

\begin{entrée}{tɤmtɕhoʁ}{}{ⓔtɤmtɕhoʁ} 
\classe{n} 
\begin{définition}\pfra{écharde}\end{définition}
\begin{définition}\pcmn{木刺(插入皮肉)}\end{définition}
\begin{exemple}\pjya{a-jaʁ tɤmtɕhoʁ to-ɕe}\hspace{5pt}\pcmn{我的手被木刺刺到}\end{exemple}
\begin{exemple}\pjya{a-jaʁ tɤmtɕhoʁ thɯ-ari tɕe ɲɯ-mŋɤm}\hspace{5pt}\pcmn{我的手被木刺刺到,很痛}\end{exemple}\end{entrée}

\begin{entrée}{tɤ-mthɯm}{}{ⓔtɤ-mthɯm} 
\classe{np} 
\begin{définition}\pfra{viande cuite}\end{définition}
\begin{définition}\pcmn{熟肉}\end{définition}\end{entrée}

\begin{entrée}{tɤmtshɤr}{}{ⓔtɤmtshɤr} 
\classe{n} 
\begin{définition}\pfra{chose étrange}\end{définition}
\begin{définition}\pcmn{怪事}\end{définition}
\begin{exemple}\pjya{ki ʑo tɤ-fse tɕe, tɤmtshɤr ci ɬoʁ}\hspace{5pt}\pcmn{这种情况下,会出现怪事}\end{exemple}\relationsémantique{参考}{\lien{ⓔsɤmtshɤr}{sɤmtshɤr}}\relationsémantique{参考}{\lien{ⓔnɤmtshɤr}{nɤmtshɤr}}\end{entrée}

\begin{entrée}{tɤmtshɤz}{}{ⓔtɤmtshɤz}\paradigme{dir}{kɤ-}
\begin{définition}\pfra{hyperostose}\end{définition}
\begin{définition}\pcmn{骨质增生}\end{définition}
\begin{exemple}\pjya{tɤmtshɤz ko-ndo}\hspace{5pt}\pcmn{他得了骨质增生}\end{exemple}
\begin{exemple}\pjya{a-tɤmtshɤz ɲɤ-ta}\hspace{5pt}\pcmn{我的骨质增生又发作了}\end{exemple}
\begin{exemple}\pjya{ɕɤrɯ pɯ-mɲɤt tɕe ɯ-rɯruz kɤ-ɣɤmna mɯ-pɯ-kɯ-khɯ tɕe, nɯ ɯ-stu nɯ ɯ-zbɤβ ku-ndzoʁ ɲɯ-ŋgrɤl tɕe, ɯ-nɯnɯ tɤ-rʑaʁ tɤ-rɲɟi tɕe, kɤ-nɤma nɤɴqa kɯ-fse, tɯ-mɯ tɤ-ɲɟɯr kɯ-fse tɕe wuma ʑo ɲɯ-mŋɤm. nɯnɯ tɤmtshɤz kɤ-kɯ-ndo rmi.}\hspace{5pt}\pcmn{骨折当时没能治好,在骨折的地方会长出胞来,时间一长,劳累了,天气变化了都会很痛。这种病叫骨质增生病。}\end{exemple}\end{entrée}

\begin{entrée}{tɤ-mtsɯ}{}{ⓔtɤ-mtsɯ} 
\classe{np} 
\begin{définition}\pfra{bouton}\end{définition}
\begin{définition}\pcmn{扣子}\end{définition}
\begin{exemple}\pjya{tɤmtsɯ kɤ-lɤt / nɯ-rle}\hspace{5pt}\pcmn{你把扣子扣上/解开}\end{exemple}\end{entrée}

\begin{entrée}{tɤmtsɯr}{}{ⓔtɤmtsɯr} 
\classe{n} 
\begin{définition}\pfra{faim}\end{définition}
\begin{définition}\pcmn{饥饿}\end{définition}
\begin{exemple}\pjya{tɤmtsɯr ɲɤ-nɤɕqa tɕe mɯ-to-ndza}\hspace{5pt}\pcmn{他忍住饥饿没有吃}\end{exemple}\relationsémantique{参考}{\lien{ⓔmtsɯr}{mtsɯr}}\end{entrée}

\begin{entrée}{tɤ-mtɯ}{}{ⓔtɤ-mtɯ} 
\classe{np} 
\begin{définition}\pfra{nœud}\end{définition}
\begin{définition}\pcmn{结}\end{définition}
\begin{exemple}\pjya{tɤ-mtɯ tɤ-lat-a}\hspace{5pt}\pcmn{我打了(个)结}\end{exemple}
\begin{exemple}\pjya{ɯ-ku ɯ-mtɯ (= ɯ-kɤχcɤl)}\hspace{5pt}\pcmn{他头顶上}\end{exemple}\relationsémantique{参考}{\lien{ⓔrɯtɤmtɯ}{rɯtɤmtɯ}}\end{entrée}

\begin{entrée}{tɤmtɯkɯnɤ}{}{ⓔtɤmtɯkɯnɤ} 
\classe{adv} 
\begin{définition}\pfra{exprès}\end{définition}
\begin{définition}\pcmn{故意}\end{définition}
\begin{exemple}\pjya{tɤmtɯkɯnɤ mɯ-ɲɤ-sɤŋo}\hspace{5pt}\pcmn{他是故意没有听的}\end{exemple}\relationsémantique{同义词}{\lien{ⓔtɤrkoz}{tɤrkoz}}\end{entrée}

\begin{entrée}{tɤmtɯɲaʁ}{}{ⓔtɤmtɯɲaʁ} 
\classe{n} 
\begin{définition}\pfra{nœud de vache}\end{définition}
\begin{définition}\pcmn{死结}\end{définition}\end{entrée}

\begin{entrée}{tɤmɯm}{}{ⓔtɤmɯm} 
\classe{n} 
\begin{définition}\pfra{chose que l'on aime manger}\end{définition}
\begin{définition}\pcmn{自己喜欢吃的东西}\end{définition}
\begin{exemple}\pjya{nɤ-tɤmɯm pɯ-ndzoʁ}\hspace{5pt}\pcmn{你可以享用好吃的东西了}\end{exemple}\relationsémantique{参考}{\lien{ⓔmɯm}{mɯm}}\end{entrée}

\begin{entrée}{tɤmɯmɯm}{}{ⓔtɤmɯmɯm} 
\classe{n} 
\begin{définition}\pfra{clochette}\end{définition}
\begin{définition}\pcmn{铃铛}\end{définition}\end{entrée}

\begin{entrée}{tɤmɯt}{}{ⓔtɤmɯt} 
\classe{n} 
\begin{définition}\pfra{souffle}\end{définition}
\begin{définition}\pcmn{吹出来的气}\end{définition}\relationsémantique{参考}{\lien{ⓔɣɤmɯt}{ɣɤmɯt}}\end{entrée}

\begin{entrée}{tɤndɤɣ}{}{ⓔtɤndɤɣ} 
\classe{n} 
\begin{définition}\pfra{poison}\end{définition}
\begin{définition}\pcmn{毒}\end{définition}\étymologie{dug}\end{entrée}

\begin{entrée}{tɤndɤɣri}{}{ⓔtɤndɤɣri} 
\classe{n} 
\begin{définition}\pfra{enfant illégitime}\end{définition}
\begin{définition}\pcmn{私生子}\end{définition}
\begin{exemple}\pjya{tɤndɤɣri pjɤ-tɕɤt-ndʑi}\hspace{5pt}\pcmn{他们俩有了私生子}\end{exemple}\relationsémantique{参考}{\lien{ⓔnɤndɤɣri}{nɤndɤɣri}}\end{entrée}

\begin{entrée}{tɤndɤku}{}{ⓔtɤndɤku} 
\classe{n} 
\begin{définition}\pfra{espèce de plante}\end{définition}
\begin{définition}\pcmn{【万年青】}\end{définition}
\begin{exemple}\pjya{tɤndɤku nɯ si kɯ-mbɯ-mbɤr ci ŋu, ɯ-jwaʁ cho ɯ-mɯntoʁ nɯ ra khɯjŋga fse ri xtɕi, ɯ-ru ldʑɯz, sɤndɤɣ}\hspace{5pt}\pcmn{野生万年青是矮小的树种,叶子和花和洋角花相似,但小一些,树干柔软,有毒性。}\end{exemple}\end{entrée}

\begin{entrée}{tɤndɤr}{}{ⓔtɤndɤr} 
\classe{n} 
\begin{définition}\pfra{bouton}\end{définition}
\begin{définition}\pcmn{粉刺}\end{définition}\end{entrée}

\begin{entrée}{tɤndoʁ}{}{ⓔtɤndoʁ} 
\classe{n} 
\begin{définition}\pfra{copeaux (à la hache)}\end{définition}
\begin{définition}\pcmn{切屑(斧头)}\end{définition}\end{entrée}

\begin{entrée}{tɤ-ndɯr}{}{ⓔtɤ-ndɯr} 
\classe{np} 
\begin{définition}\pfra{débris, lie}\end{définition}
\begin{définition}\pcmn{渣滓}\end{définition}
\begin{exemple}\pjya{tɤ-ndɯr ɲo-ri}\hspace{5pt}\pcmn{剩下了渣滓}\end{exemple}\end{entrée}

\begin{entrée}{tɤ-ndzraʁ}{}{ⓔtɤ-ndzraʁ} 
\classe{np} 
\begin{définition}\pfra{morceau de tsampa, de glaise roulé en boule}\end{définition}
\begin{définition}\pcmn{糌粑坨}\end{définition}
\begin{exemple}\pjya{chɤ-rgɤz tɕe, tɤ-ndzraʁ ma ɲo-me}\hspace{5pt}\pcmn{他老得只剩下一坨}\end{exemple}
\begin{exemple}\pjya{rɟɤɣi-ndzraʁ ɯ-tɯ́-ndze}\hspace{5pt}\pcmn{你吃不吃糌粑坨坨}\end{exemple}\end{entrée}

\begin{entrée}{tɤ-ndʑɯɣ}{}{ⓔtɤ-ndʑɯɣ} 
\classe{np} 
\begin{définition}\pfra{résine}\end{définition}
\begin{définition}\pcmn{松香;树脂}\end{définition}
\begin{exemple}\pjya{tɯrgi ɯ-ndʑɯɣ}\hspace{5pt}\pcmn{杉树树脂}\end{exemple}\relationsémantique{参考}{\lien{ⓔaɣɯndʑɯɣ}{aɣɯndʑɯɣ}}\end{entrée}

\begin{entrée}{tɤndʐo}{}{ⓔtɤndʐo} 
\classe{n} 
\begin{définition}\pfra{froid}\end{définition}
\begin{définition}\pcmn{寒冷的(天气)}\end{définition}\relationsémantique{参考}{\lien{ⓔɣɤndʐo}{ɣɤndʐo}}\relationsémantique{参考}{\lien{ⓔnɤndʐo}{nɤndʐo}}
\begin{sous-entrée}{tɤndʐo,tɤrqɯ}{ⓔtɤndʐoⓝtɤndʐo,tɤrqɯ} 
\classe{n} 
\begin{exemple}\pjya{ɣɯjpa rcanɯ tɤndʐo tɤrqɯ pɯ-thɯɣ}\hspace{5pt}\pcmn{今年非常寒冷}\end{exemple}\relationsémantique{参考}{\lien{ⓔnɤndʐɤrqɯ}{nɤndʐɤrqɯ}}\end{sous-entrée}

\end{entrée}

\begin{entrée}{tɤngɯt}{}{ⓔtɤngɯt} 
\classe{n} 
\begin{définition}\pfra{possessions en commun}\end{définition}
\begin{définition}\pcmn{共同拥有的东西}\end{définition}
\begin{exemple}\pjya{kɯki @luyinji tɕi-tɤngɯt ŋu}\hspace{5pt}\pcmn{这个录音机是我们俩共同拥有的}\end{exemple}\relationsémantique{参考}{\lien{ⓔnɤngɯt}{nɤngɯt}}\end{entrée}

\begin{entrée}{tɤ-nmaʁ}{}{ⓔtɤ-nmaʁ} 
\classe{np} 
\begin{définition}\pfra{mari}\end{définition}
\begin{définition}\pcmn{丈夫}\end{définition}\end{entrée}

\begin{entrée}{tɤntɤβ}{}{ⓔtɤntɤβ} 
\classe{n} \sens{1}
\begin{définition}\pfra{bulle}\end{définition}
\begin{définition}\pcmn{水泡}\end{définition}\sens{2}
\begin{définition}\pfra{écume}\end{définition}
\begin{définition}\pcmn{泡沫}\end{définition}\relationsémantique{参考}{\lien{ⓔaɣɯntɤβ}{aɣɯntɤβ}}\end{entrée}

\begin{entrée}{tɤɲi}{}{ⓔtɤɲi} 
\classe{n} 
\begin{définition}\pfra{bâton}\end{définition}
\begin{définition}\pcmn{拐棍}\end{définition}
\begin{exemple}\pjya{tɤɲi kɤ-ndo}\hspace{5pt}\pcmn{拄着拐棍}\end{exemple}\relationsémantique{参考}{\lien{ⓔnɤɲi}{nɤɲi}}\end{entrée}

\begin{entrée}{tɤ-ɲi}{}{ⓔtɤ-ɲi} 
\classe{np} 
\begin{définition}\pfra{tante (sœur du père)}\end{définition}
\begin{définition}\pcmn{姑母}\end{définition}
\begin{exemple}\pjya{a-ɲi}\hspace{5pt}\pcmn{我的姑妈}\end{exemple}\end{entrée}

\begin{entrée}{tɤ-ɲɟoʁɲɟi}{}{ⓔtɤ-ɲɟoʁɲɟi} 
\classe{np} 
\begin{définition}\pfra{ordure}\end{définition}
\begin{définition}\pcmn{垃圾}\end{définition}
\begin{exemple}\pjya{tɕiʑo saχsɯ nɯ-anɯri-tɕi ɯ-qhu tɕi-tɤ-ɲɟoʁɲɟi ra ɣɯ-jo-ru-nɯ}\hspace{5pt}\pcmn{在我们俩出去吃中午饭之后,他们捡了垃圾}\end{exemple}\end{entrée}

\begin{entrée}{tɤ-ŋɤm}{}{ⓔtɤ-ŋɤm} 
\classe{np} 
\begin{définition}\pfra{douleur}\end{définition}
\begin{définition}\pcmn{痛}\end{définition}
\begin{exemple}\pjya{ndʑi-ŋgo ndʑi-ŋɤm a-pɯ-me}\hspace{5pt}\pcmn{但愿你们俩没有什么病痛}\end{exemple}\relationsémantique{参考}{\lien{ⓔmŋɤm}{mŋɤm}}\end{entrée}

\begin{entrée}{tɤŋɤmɕɣɤphɯt}{}{ⓔtɤŋɤmɕɣɤphɯt} 
\classe{n} 
\begin{définition}\pfra{sensation de soulagement lorsqu'on arrache une dent qui fait souffrir}\end{définition}
\begin{définition}\pcmn{好得又快又彻底(牙齿痛得厉害的时候,把发痛的牙齿拔掉了就一下子不痛了)}\end{définition}
\begin{exemple}\pjya{tɤŋɤmɕɣɤphɯt to-βzu tɕe nɯɕɯmɯma ʑo to-mna}\hspace{5pt}\pcmn{牙一拔就不痛了}\end{exemple}\relationsémantique{参考}{\lien{ⓔtɯ-ɕɣa}{tɯ-ɕɣa}}\relationsémantique{参考}{\lien{ⓔphɯt}{phɯt}}\end{entrée}

\begin{entrée}{tɤŋe}{}{ⓔtɤŋe} 
\classe{n} 
\begin{définition}\pfra{soleil}\end{définition}
\begin{définition}\pcmn{太阳}\end{définition}
\begin{exemple}\pjya{tɤŋe ci ci ɣɤʑu ci ci maŋe}\hspace{5pt}\pcmn{一会有太阳,一会没有}\end{exemple}
\begin{exemple}\pjya{tɤŋe tɤ-ɬoʁ}\hspace{5pt}\pcmn{太阳升起了}\end{exemple}
\begin{exemple}\pjya{tɤŋe ɲɤ-k-ɤβzu-ci}\hspace{5pt}\pcmn{(云散了,)太阳就露面了}\end{exemple}\relationsémantique{参考}{\lien{ⓔslɤŋe}{slɤŋe}}\end{entrée}

\begin{entrée}{tɤŋgɤr}{}{ⓔtɤŋgɤr} 
\classe{n} 
\begin{définition}\pfra{lard}\end{définition}
\begin{définition}\pcmn{膘}\end{définition}\end{entrée}

\begin{entrée}{tɤŋgɯ}{}{ⓔtɤŋgɯ} 
\classe{n} 
\begin{définition}\pfra{prêt}\end{définition}
\begin{définition}\pcmn{借的东西}\end{définition}
\begin{exemple}\pjya{tɯjpu tɤŋgɯ na-mɟa (=tɯjpu na-nɤŋgɯ)}\hspace{5pt}\pcmn{他借了粮食}\end{exemple}\relationsémantique{参考}{\lien{ⓔnɤŋgɯ}{nɤŋgɯ}}\end{entrée}

\begin{entrée}{tɤ-ŋgɯm}{}{ⓔtɤ-ŋgɯm} 
\classe{np} 
\begin{définition}\pfra{œuf}\end{définition}
\begin{définition}\pcmn{蛋}\end{définition}\end{entrée}

\begin{entrée}{tɤŋkhɯt}{}{ⓔtɤŋkhɯt} 
\classe{n} 
\begin{définition}\pfra{poing}\end{définition}
\begin{définition}\pcmn{拳}\end{définition}
\begin{exemple}\pjya{a-tɤŋkhɯt}\hspace{5pt}\pcmn{我的拳头}\end{exemple}\relationsémantique{参考}{\lien{ⓔnɤŋkhɯt}{nɤŋkhɯt}}\end{entrée}

\begin{entrée}{tɤ-ŋkɯ}{}{ⓔtɤ-ŋkɯ} 
\classe{np} 
\begin{définition}\pfra{couenne}\end{définition}
\begin{définition}\pcmn{猪皮}\end{définition}
\begin{exemple}\pjya{paʁ ɯ-ŋkɯ ɲɯ-jaʁ}\hspace{5pt}\pcmn{猪的皮很厚}\end{exemple}\end{entrée}

\begin{entrée}{tɤɴqa}{}{ⓔtɤɴqa} 
\classe{n} 
\begin{définition}\pfra{difficulté}\end{définition}
\begin{définition}\pcmn{辛苦}\end{définition}
\begin{exemple}\pjya{kɯmɤlɤxso ji-tɤɴqa pjɤ-ɕti}\hspace{5pt}\pcmn{我们白辛苦了}\end{exemple}\relationsémantique{参考}{\lien{ⓔɴqa}{ɴqa}}\end{entrée}

\begin{entrée}{tɤɴqhi}{}{ⓔtɤɴqhi} 
\classe{n} 
\begin{définition}\pfra{saletés}\end{définition}
\begin{définition}\pcmn{污垢}\end{définition}\relationsémantique{参考}{\lien{ⓔɴqhi}{ɴqhi}}\relationsémantique{参考}{\lien{ⓔtɤlɤɴqhi}{tɤlɤɴqhi}}\end{entrée}

\begin{entrée}{tɤ-pa}{}{ⓔtɤ-pa} 
\classe{np} 
\begin{définition}\pfra{père (noble)}\end{définition}
\begin{définition}\pcmn{父亲(贵族用语)}\end{définition}
\begin{exemple}\pjya{a-pa a-ma}\hspace{5pt}\pcmn{我的父母}\end{exemple}\end{entrée}

\begin{entrée}{tɤ-pɤloʁ}{}{ⓔtɤ-pɤloʁ} 
\classe{np} 
\begin{définition}\pfra{manche}\end{définition}
\begin{définition}\pcmn{袖子}\end{définition}
\begin{exemple}\pjya{a-pɤloʁ}\hspace{5pt}\pcmn{我的袖子}\end{exemple}\end{entrée}

\begin{entrée}{tɤ-pɤndɯr}{}{ⓔtɤ-pɤndɯr} 
\classe{np} 
\begin{définition}\pfra{mauvais caractère}\end{définition}
\begin{définition}\pcmn{做事遇到挫折回家就发脾气}\end{définition}
\begin{exemple}\pjya{nɤ-pɤndɯr a-mɤ-jɤ-tɯ-ɣɯt je}\hspace{5pt}\pcmn{别回来发脾气}\end{exemple}\end{entrée}

\begin{entrée}{tɤpɤr}{}{ⓔtɤpɤr} 
\classe{n} 
\begin{définition}\pfra{épi de maïs}\end{définition}
\begin{définition}\pcmn{玉米包}\end{définition}\end{entrée}

\begin{entrée}{tɤ-pɤri}{}{ⓔtɤ-pɤri} 
\classe{np} 
\begin{définition}\pfra{repas du soir}\end{définition}
\begin{définition}\pcmn{晚饭}\end{définition}
\begin{exemple}\pjya{a-pɤri to-mda}\hspace{5pt}\pcmn{我要吃晚餐}\end{exemple}\end{entrée}

\begin{entrée}{tɤ-pɤro}{}{ⓔtɤ-pɤro} 
\classe{np} 
\begin{définition}\pfra{cadeau}\end{définition}
\begin{définition}\pcmn{礼物(自己亲手拿给别人)}\end{définition}
\begin{exemple}\pjya{a-pɤro}\hspace{5pt}\pcmn{我给别人的礼物}\end{exemple}
\begin{exemple}\pjya{a-pɤro ɲɯ-ta-mbi ŋu}\hspace{5pt}\pcmn{我把礼物送给你}\end{exemple}
\begin{exemple}\pjya{aʑɯɣ nɤ-pɤro jo-tɯ-ɣɯt}\hspace{5pt}\pcmn{你给我带了礼物}\end{exemple}\relationsémantique{同义词}{\lien{ⓔskɯrma}{skɯrma}}\relationsémantique{同义词}{\lien{ⓔtɤ-rkuz}{tɤ-rkuz}}\relationsémantique{同义词}{\lien{ⓔmpɕɯmɤr}{mpɕɯmɤr}}\end{entrée}

\begin{entrée}{tɤ-pɤtso}{}{ⓔtɤ-pɤtso} 
\classe{np} 
\begin{définition}\pfra{enfant}\end{définition}
\begin{définition}\pcmn{孩子}\end{définition}
\begin{exemple}\pjya{tɤ-pɤtso ɯ-skɤt}\hspace{5pt}\pcmn{小孩子的语气}\end{exemple}
\begin{exemple}\pjya{ɯ-pɤtso ɣɤʑu}\hspace{5pt}\pcmn{她怀上了小孩}\end{exemple}\relationsémantique{参考}{\lien{ⓔnɯtɤpɤtso}{nɯtɤpɤtso}}\relationsémantique{参考}{\lien{ⓔarɯtɤpɤtso}{arɯtɤpɤtso}}\end{entrée}

\begin{entrée}{tɤpɤtsoβraʁ}{}{ⓔtɤpɤtsoβraʁ} 
\classe{n} 
\begin{définition}\pfra{petit phasme}\end{définition}
\begin{définition}\pcmn{小的树枝虫}\end{définition}
\begin{exemple}\pjya{tɤpɤtsoβraʁ nɯ sɯjnombrombro cho kɯ-naχtɕɯ-χtɕɯɣ ŋu, li ʁnɯ-tɯphu tu, ldʑaŋkɯ ci kɯ-pɣi ci tu, ndʑi-tɯ-xtshɯm naχtɕɯɣ tɕe nɯ a-pɯ́-wɣ-mto tɕe nɯ maʁ nɤ tɯ-taʁ a-tɤ-ɣi tɕe, nɯ maʁ nɤ tɯʑo tɤ-rɟit tu, nɯ maʁ nɤ tɯ-kɯmdza ra nɯ-rɟit tu tu-kɯ-ti ɲɯ-ŋu tɕe núndʐa tɤ-pɤtso βraʁ ɲɯ-rmi}\hspace{5pt}\pcmn{\lien{ⓔtɤpɤtsoβraʁ}{tɤpɤtsoβraʁ}和秤杆虫一模一样,也有两种,绿色的和灰色的,两种一样细。据说如果人看见了它,或者如果它爬到人的身上来了,要么自己会有身孕,要么自己亲戚会有身孕,所以叫\lien{ⓔtɤpɤtsoβraʁ}{tɤpɤtsoβraʁ}(小孩子的象征)}\end{exemple}\end{entrée}

\begin{entrée}{tɤpɣi}{}{ⓔtɤpɣi} 
\classe{n} 
\begin{définition}\pfra{maladie de l'œil}\end{définition}
\begin{définition}\pcmn{眼病}\end{définition}
\begin{exemple}\pjya{a-mɲaʁ tɤpɣi to-ɣi}\hspace{5pt}\pcmn{我眼睛上长了白点}\end{exemple}
\begin{exemple}\pjya{tɤ-mɲaʁ-rdu kɯ-ɲaʁ ɯ-taʁ kɯ-wɣrum kɯ-xtɕɯ-xtɕi nɯ-kɯ-ɬoʁ nɯ, wuma ʑo mŋɤm tɕe tɤpɣi rmi}\hspace{5pt}\pcmn{黑眼珠上长了白点,很疼。}\end{exemple}\end{entrée}

\begin{entrée}{tɤphɯ}{}{ⓔtɤphɯ} 
\classe{n} 
\begin{définition}\pfra{motte de terre}\end{définition}
\begin{définition}\pcmn{土块}\end{définition}\end{entrée}

\begin{entrée}{tɤphɯxtsɯ}{}{ⓔtɤphɯxtsɯ} 
\classe{n} 
\begin{définition}\pfra{fait d'écraser les mottes de terre}\end{définition}
\begin{définition}\pcmn{打土巴}\end{définition}
\begin{exemple}\pjya{tɤphɯxtsɯ tɤ-βzu-t-a}\hspace{5pt}\pcmn{我打了土巴}\end{exemple}\relationsémantique{参考}{\lien{ⓔtɤphɯ}{tɤphɯ}}\relationsémantique{参考}{\lien{ⓔxtsɯ}{xtsɯ}}\relationsémantique{参考}{\lien{ⓔnɤphɯxtsɯ}{nɤphɯxtsɯ}}\end{entrée}

\begin{entrée}{tɤ-pi}{}{ⓔtɤ-pi} 
\classe{np} \sens{1}
\begin{définition}\pfra{grand frère, grande sœur}\end{définition}
\begin{définition}\pcmn{哥哥;姐姐}\end{définition}
\begin{exemple}\pjya{a-pi}\hspace{5pt}\pcmn{我的哥哥(我的姐姐)}\end{exemple}\sens{2}
\begin{définition}\pfra{hôte}\end{définition}
\begin{définition}\pcmn{客人}\end{définition}
\begin{exemple}\pjya{tɤ-ndza-nɯ je ma tɤ-pi ɯ-zɤz mɤ-sɤfka kɤ-ti tɕe tha mɤ-tɯ-fka-nɯ}\hspace{5pt}\pcmn{你们吃吧,俗话说:“做客人的食物吃不饱”,你们会吃不饱的}\end{exemple}\end{entrée}

\begin{entrée}{tɤpjaʁ}{}{ⓔtɤpjaʁ} 
\classe{n} 
\begin{définition}\pfra{morceau de bois coupé en parallélépipède}\end{définition}
\begin{définition}\pcmn{木方条}\end{définition}\end{entrée}

\begin{entrée}{tɤpjɤz}{}{ⓔtɤpjɤz} 
\classe{n} 
\begin{définition}\pfra{tresse}\end{définition}
\begin{définition}\pcmn{辫子}\end{définition}
\begin{exemple}\pjya{a-tɤpjɤz}\hspace{5pt}\pcmn{我的辫子}\end{exemple}
\begin{exemple}\pjya{tɤpjɤz tha-βzu}\hspace{5pt}\pcmn{他编了辫子}\end{exemple}\relationsémantique{参考}{\lien{ⓔrɤpjɤz}{rɤpjɤz}}\end{entrée}

\begin{entrée}{tɤpra}{}{ⓔtɤpra} 
\classe{n} 
\begin{définition}\pfra{messager, envoyé}\end{définition}
\begin{définition}\pcmn{使者;派出去的人}\end{définition}
\begin{exemple}\pjya{aʑo nɤ-tɤpra tu-βze-a jɤɣ}\hspace{5pt}\pcmn{我可以当你的使者}\end{exemple}\relationsémantique{参考}{\lien{ⓔɣɤxpra}{ɣɤxpra}}\relationsémantique{参考}{\lien{ⓔnɤpra}{nɤpra}}\end{entrée}

\begin{entrée}{tɤ-prɤm}{}{ⓔtɤ-prɤm} 
\classe{np} 
\begin{définition}\pfra{nourriture en poudre}\end{définition}
\begin{définition}\pcmn{粉状粮食}\end{définition}
\begin{exemple}\pjya{tɤ-prɤm pɯ-lɤt}\hspace{5pt}\pcmn{加一点粉吧}\end{exemple}\end{entrée}

\begin{entrée}{tɤprɯ}{}{ⓔtɤprɯ} 
\classe{n} 
\begin{définition}\pfra{abri de pluie}\end{définition}
\begin{définition}\pcmn{避雨的地方}\end{définition}\relationsémantique{参考}{\lien{ⓔprɯ}{prɯ}}\relationsémantique{参考}{\lien{ⓔnɤprɯ}{nɤprɯ}}\end{entrée}

\begin{entrée}{tɤ-pɯ}{}{ⓔtɤ-pɯ} 
\classe{np} \sens{1}
\begin{définition}\pfra{petit (animal)}\end{définition}
\begin{définition}\pcmn{崽子}\end{définition}\sens{2}
\begin{définition}\pfra{intérêt}\end{définition}
\begin{définition}\pcmn{利息}\end{définition}
\begin{exemple}\pjya{tɤ-pɯ nɯ ɯʑoz kú-wɣ-ja}\hspace{5pt}\pcmn{要把小的关要另外的圈里}\end{exemple}
\begin{exemple}\pjya{ɯ-pɯ tɤ-nɯ-pe}\hspace{5pt}\pcmn{你把它收藏起来}\end{exemple}
\begin{exemple}\pjya{ɯ-pɯ to-nɯ-pa}\hspace{5pt}\pcmn{他收藏起来了}\end{exemple}\relationsémantique{参考}{\lien{ⓔpaⓗ1ⓝɯ-pɯ,pa}{ɯ-pɯ,pa}}\end{entrée}

\begin{entrée}{tɤ-qaʁrɯ/\variante{ɯ-qataʁrɯ}}{}{ⓔtɤ-qaʁrɯ} 
\classe{np} 
\begin{définition}\pfra{sabot}\end{définition}
\begin{définition}\pcmn{蹄子}\end{définition}\relationsémantique{参考}{\lien{ⓔtɯ-qa}{tɯ-qa}}\end{entrée}

\begin{entrée}{tɤ-qɤtɕɤz}{}{ⓔtɤ-qɤtɕɤz} 
\classe{np} 
\begin{définition}\pfra{trace de patte}\end{définition}
\begin{définition}\pcmn{脚印(动物)}\end{définition}\relationsémantique{参考}{\lien{ⓔtɤ-tɕɤz}{tɤ-tɕɤz}}\relationsémantique{参考}{\lien{ⓔtɯ-qa}{tɯ-qa}}\end{entrée}

\begin{entrée}{tɤqɤt,lɤt}{}{ⓔtɤqɤt,lɤt} 
\classe{n}
\classe{vt} \paradigme{dir}{lɤ-}
\begin{définition}\pfra{séparer une chambre en deux}\end{définition}
\begin{définition}\pcmn{把大房间隔成两个小房间}\end{définition}
\begin{exemple}\pjya{kha tɤqɤt lɤ-lat-a}\hspace{5pt}\pcmn{我把房子隔开了}\end{exemple}\relationsémantique{Component 1}{\lien{}{tɤqɤt}}\relationsémantique{Component 2}{\lien{}{lɤt}}\relationsémantique{参考}{\lien{ⓔqɤt}{qɤt}}\relationsémantique{参考}{\lien{ⓔlɤtⓗ1}{lɤt₁}}\end{entrée}

\begin{entrée}{tɤqiaβjmɤɣ}{}{ⓔtɤqiaβjmɤɣ} 
\classe{n} 
\begin{définition}\pfra{lactaire}\end{définition}
\begin{définition}\pcmn{乳菇【苦苦菌】}\end{définition}
\begin{exemple}\pjya{tɤqiaβjmɤɣ nɯ tɯrgi ɕkrɤz ɯ-ŋgɯ ra tu-ɬoʁ ŋu. ɯ-tɯ-wxti nɯ jmɤɣni jamar fse, ɯ-mdoʁ nɯ pɣi, pjɯ́-wɣ-qru tɕe ɯ-ŋgɯ tɤ-lu kɯ-fse ɲɯ-nɯɬoʁ ŋu, kɤ-ndza mɤ-mɯm, qiaβ ri mɤ-sɤndɤɣ}\hspace{5pt}\pcmn{苦苦菌长在杉木林和青冈树林里,长得和杉木菌一样大小,颜色是灰色的,把它打烂时里面会流出像牛奶一样的汁,不好吃,因为太苦,但是没有毒。}\end{exemple}\end{entrée}

\begin{entrée}{tɤru}{₂}{ⓔtɤruⓗ2} 
\classe{n} 
\begin{définition}\pfra{espèce d'arbre}\end{définition}
\begin{définition}\pcmn{【火棘】}\end{définition}
\begin{exemple}\pjya{tɤ-ru nɯ si wuma mɤ-kɯ-mbro ci ŋu, ɯ-jwaʁ ɯ-qhu nɯ kɯ-pɣi tsa ŋu, ɯ-ru ɯ-rqhu nɯ li kɯ-pɣi tsa ŋu, kɯ-nɤrko ci ŋu, kɯ-rɤma ra ɣɯ nɯ-laʁdɯn ɯ-jɯ kɤ-nɯ-βzu sna. ɯ-mat nɯ thɯ-tɯt tɕe ɣɯrni, paʁ kɤ-sɯχsu sna. zgoku ɯ-taʁ pa ʑo tu-ɬoʁ cha.}\hspace{5pt}\pcmn{火棘是一种比较矮的树,叶子背面是灰色的,树皮也是灰色的,比较坚实,农民可以用来制造各种农具的把儿。果实成熟时是红色的,可以喂猪。山上山下都可以生长。}\end{exemple}\end{entrée}

\begin{entrée}{tɤ-ru}{₁}{ⓔtɤ-ruⓗ1} 
\classe{n} 
\begin{définition}\pfra{chef de village}\end{définition}
\begin{définition}\pcmn{寨首}\end{définition}
\begin{exemple}\pjya{a-ru}\hspace{5pt}\pcmn{先生(对别人的尊称)}\end{exemple}\relationsémantique{参考}{\lien{ⓔtɤrɤze}{tɤrɤze}}\end{entrée}

\begin{entrée}{tɤrɤɕom}{}{ⓔtɤrɤɕom} 
\classe{n} 
\begin{définition}\pfra{lame de binette}\end{définition}
\begin{définition}\pcmn{锄刃}\end{définition}
\begin{exemple}\pjya{tɤrɤɕom nɯ tɤrɤt ɯ-pa tu-kɤ-tshoʁ ɕom ci ŋu. sɤ-ntʂu ɣɯ ɯ-laʁdɯn nɯ tɤrɤt ŋu tɕe, tɤrɤt ɯ-spa nɯ si ɯ-rtaʁ pjɯ́-wɣ-phɯt tɕe, ɯ-rtaʁ nɯ li ɯ-rtaʁ kɯ-tu pjɯ-ŋu ra tɕe, ɯ-rtaʁ tɯ-rdoʁ nɯ pjɯ́-wɣ-ɣɤ-zri ɲɯ́-wɣ-βzu tɕe tɯ-rdoʁ nɯ pjɯ́-wɣ-ɣɤ-xtɯt, ɯ-rtaʁ kɯ-xtɯt pɯ-kɤ-βzu nɯ chɯ́-wɣ-sɯ-ɤmtɕoʁ tɕe nɯ tɕu tɤ-rɤɕom tú-wɣ-tshoʁ. ɯ-rtaʁ kɯ-zri nɯ chɯ́-wɣ-βʑoʁ chɯ́-wɣ-ɣɤ-mpɕu tɕe, nɯnɯ tɤrɤt ɯ-jɯ ŋu, kɤ-ntʂu tɕe nɯ tú-wɣ-ntɕhoz tɕe, kɤ-ntʂu aɲaj tɕe tɤ-rɤku mɤ-sɯ-mɲɤt.}\hspace{5pt}\pcmn{锄刃是安装在锄头下面的铁。锄草的专用工具叫锄头。锄头是用砍下的树枝作成的,树枝要有叉,其中的一支砍长,另一支砍短一点,砍得较短的那个叉要削尖一些,在那里安装锄刃。长的那一支要削光滑,成了锄头的把子。锄草的时候用它就速度快,不损坏庄稼。}\end{exemple}\end{entrée}

\begin{entrée}{tɤ-rɤku}{}{ⓔtɤ-rɤku} 
\classe{np} 
\begin{définition}\pfra{récolte}\end{définition}
\begin{définition}\pcmn{庄稼}\end{définition}
\begin{exemple}\pjya{ji-rɤku}\hspace{5pt}\pcmn{我们的庄稼}\end{exemple}\end{entrée}

\begin{entrée}{tɤrɤm}{}{ⓔtɤrɤm} 
\classe{n} 
\begin{définition}\pfra{planche de bois}\end{définition}
\begin{définition}\pcmn{木板}\end{définition}\end{entrée}

\begin{entrée}{tɤrɤmɕkho}{}{ⓔtɤrɤmɕkho} 
\classe{n} 
\begin{définition}\pfra{parquet}\end{définition}
\begin{définition}\pcmn{地板}\end{définition}\end{entrée}

\begin{entrée}{tɤrɤt}{}{ⓔtɤrɤt} 
\classe{n} 
\begin{définition}\pfra{binette}\end{définition}
\begin{définition}\pcmn{锄草用的锄头}\end{définition}\relationsémantique{参考}{\lien{ⓔtɤrɤɕom}{tɤrɤɕom}}\end{entrée}

\begin{entrée}{tɤrɤze}{}{ⓔtɤrɤze} 
\classe{n} 
\begin{définition}\pfra{prince, jeune maître de maison}\end{définition}
\begin{définition}\pcmn{少爷}\end{définition}\relationsémantique{参考}{\lien{ⓔtɤ-ruⓗ1}{tɤ-ru₁}}\end{entrée}

\begin{entrée}{tɤ-rca}{}{ⓔtɤ-rca} 
\classe{np}
\classe{vs} 
\begin{définition}\pfra{avec, en suivant}\end{définition}
\begin{définition}\pcmn{跟……一起}\end{définition}
\begin{exemple}\pjya{a-rca jɤ-ɣi}\hspace{5pt}\pcmn{跟我来!}\end{exemple}
\begin{exemple}\pjya{ɯʑo kɯnɤ a-rca lu-nɯɣi ŋu}\hspace{5pt}\pcmn{他也跟我回去}\end{exemple}\relationsémantique{Component 2}{\lien{ⓔmeⓗ2}{me}}
\begin{sous-entrée}{tɤ-rca,me}{ⓔtɤ-rcaⓝtɤ-rca,me} 
\classe{np} 
\begin{définition}\pfra{irrémédiable}\end{définition}
\begin{définition}\pcmn{无法挽救;无从下手}\end{définition}
\begin{exemple}\pjya{a-laʁtɕha ra thɯ-arɕo tɕe a-rca nɯ-me}\hspace{5pt}\pcmn{我东西没有了,再也无法挽救}\end{exemple}
\begin{exemple}\pjya{a-rca ci na-ɣɤme}\hspace{5pt}\pcmn{他把我的事情弄得很乱}\end{exemple}\relationsémantique{Component 1}{\lien{ⓔtɤ-rca}{tɤ-rca}}\end{sous-entrée}

\end{entrée}

\begin{entrée}{tɤrcoʁ}{}{ⓔtɤrcoʁ} 
\classe{n} 
\begin{définition}\pfra{boue}\end{définition}
\begin{définition}\pcmn{泥巴}\end{définition}
\begin{exemple}\pjya{tɤrcoʁ ɕ-pɯ-βzu-t-a}\hspace{5pt}\pcmn{我和了泥}\end{exemple}\relationsémantique{参考}{\lien{ⓔrɤrcoʁ}{rɤrcoʁ}}\relationsémantique{参考}{\lien{ⓔɣɤrcoʁ}{ɣɤrcoʁ}}\end{entrée}

\begin{entrée}{tɤrɕɤz}{}{ⓔtɤrɕɤz} 
\classe{n} 
\begin{définition}\pfra{mur en latte de saule}\end{définition}
\begin{définition}\pcmn{用杨柳树的细条编成的墙壁【巴巴】}\end{définition}
\begin{exemple}\pjya{tɤrɕɤz nɯ kɯɕɯŋgɯ tɤrɤm kɤ-tɕɤt tʂɤm kɤ-rku mɤ-kɯ-cha ra kɯ tɤqɤt ɯ-sɤ-lɤt nɯ-kɤ-βzu pjɤ-ŋu tɕe ʑmbri ɣɯ ɯ-rtaʁ kɯ-xtshɯm tsa kɤ-mɲɤm ɲɯ-ɕar-nɯ tɕe tɤ-jtsi pɤrthɤβ rorʁe ɲɯ-lɤt-nɯ tɕe nɯ ɯ-taʁ nɯ tɕu ɲɯ-taʁ-nɯ kɯ-fse tɕe ɲɯ-βzu-nɯ pjɤ-ŋu, tɕe nɯnɯ kɤ-βzu tɕe si pjɯ-ɣɯrŋi ra ma nɯ-rom tɕe tu-rko ɕti tɕe kɤ-taʁ mɤ-khɯ. tɕe nɯ tɤrɕɤz nɯ́-wɣ-rku nɯ-rom tɕe, wuma ʑo nɤrko, ɯ-ŋgɯ ku-kɯ-rɤʑi tɕe mpja. tsuku kɯ ɯ-pɕi tɤrcoʁ ʑala tu-lɤt-nɯ pjɤ-ŋu tɕe, nɯ kɯ-fse nɯ mɤʑɯ ʑo mpja.}\hspace{5pt}\pcmn{柳条墙是过去那些没钱改木板装板壁的人家用来隔房间的。他们找来比较细的、均匀的柳枝条,在柱子之间装上横干,(把枝条)编在上面,就成了柳条墙。要趁柳条没干的时候(编),因为干了就变硬,不能编。柳条墙装了以后,干了,就比较坚固,住在里面暖和。有的人在外面糊上细泥巴,这样更暖和。}\end{exemple}\end{entrée}

\begin{entrée}{tɤ-re}{}{ⓔtɤ-re} 
\classe{np} 
\begin{définition}\pfra{rire}\end{définition}
\begin{définition}\pcmn{笑}\end{définition}
\begin{exemple}\pjya{a-re ma-tɯ-tɕɤt}\hspace{5pt}\pcmn{你不要让我耻笑你(你小看我了,我不是那种人)}\end{exemple}
\begin{exemple}\pjya{tɤre sɤtɕɯtɕɤt}\hspace{5pt}\pcmn{当笑话}\end{exemple}
\begin{exemple}\pjya{tɤre sɤtɕɯtɕɤt ma-tɤ-kɯ-sɯβzu-a}\hspace{5pt}\pcmn{你不要取笑我}\end{exemple}\relationsémantique{参考}{\lien{ⓔtɤre tɤɟaʁ}{tɤre tɤɟaʁ}}\relationsémantique{参考}{\lien{ⓔnɤreⓗ1ⓢ2ⓝnɤre}{nɤre}}\relationsémantique{参考}{\lien{ⓔsɤre}{sɤre}}\end{entrée}

\begin{entrée}{tɤresɤpɯpa}{}{ⓔtɤresɤpɯpa} 
\classe{n} 
\begin{définition}\pfra{moquerie}\end{définition}
\begin{définition}\pcmn{取笑人}\end{définition}
\begin{exemple}\pjya{tɤresɤpɯpa ma-tɤ-kɯ-sɯβzu-a}\hspace{5pt}\pcmn{你不要嘲笑我}\end{exemple}
\begin{exemple}\pjya{tɤresɤpɯpa ta-βzu}\hspace{5pt}\pcmn{他取笑了他}\end{exemple}\relationsémantique{参考}{\lien{ⓔtɤ-re}{tɤ-re}}\end{entrée}

\begin{entrée}{tɤre tɤɟaʁ}{}{ⓔtɤre tɤɟaʁ} 
\classe{n} 
\begin{définition}\pfra{plaisanteries}\end{définition}
\begin{définition}\pcmn{说说笑笑}\end{définition}\relationsémantique{参考}{\lien{ⓔnɤrɤɟaʁ}{nɤrɤɟaʁ}}\end{entrée}

\begin{entrée}{tɤrga}{}{ⓔtɤrga} 
\classe{n} \paradigme{emphatic}{tɤrga tɤχi}\paradigme{emphatic}{tɤrga tɤle}
\begin{définition}\pfra{bonheur}\end{définition}
\begin{définition}\pcmn{幸福}\end{définition}
\begin{exemple}\pjya{tɤrga tɤχi kɯ ku-rɤʑi-a}\hspace{5pt}\pcmn{我非常幸福}\end{exemple}
\begin{exemple}\pjya{tɤrga tɤle kɯ jɤ-nɯɣe-a}\hspace{5pt}\pcmn{我兴高采烈地回家了}\end{exemple}\end{entrée}

\begin{entrée}{tɤ-rɣa}{}{ⓔtɤ-rɣa} 
\classe{np} 
\begin{définition}\pfra{voisin}\end{définition}
\begin{définition}\pcmn{邻居}\end{définition}
\begin{exemple}\pjya{a-rɣa}\hspace{5pt}\pcmn{我的邻居}\end{exemple}\relationsémantique{同义词}{\lien{ⓔjɯlco}{jɯlco}}\relationsémantique{参考}{\lien{ⓔandʑɯrɣa}{andʑɯrɣa}}\end{entrée}

\begin{entrée}{tɤ-rɣe}{}{ⓔtɤ-rɣe} 
\classe{np} 
\begin{définition}\pfra{perle}\end{définition}
\begin{définition}\pcmn{珍珠}\end{définition}
\begin{exemple}\pjya{a-rɣe}\hspace{5pt}\pcmn{我的珍珠}\end{exemple}\end{entrée}

\begin{entrée}{tɤ-ri}{}{ⓔtɤ-ri} 
\classe{np} 
\begin{définition}\pfra{fil}\end{définition}
\begin{définition}\pcmn{线}\end{définition}
\begin{exemple}\pjya{tɤ-ri nɯ-sɤβzu-t-a}\hspace{5pt}\pcmn{我把毛搓成了线}\end{exemple}
\begin{exemple}\pjya{nɤ-xtsa ɯ-ri nɯ-βzu-t-a}\hspace{5pt}\pcmn{我给你做了鞋带}\end{exemple}\relationsémantique{参考}{\lien{ⓔɣɯri}{ɣɯri}}\relationsémantique{参考}{\lien{ⓔsmɤɣri}{smɤɣri}}\relationsémantique{参考}{\lien{ⓔrazri}{razri}}\end{entrée}

\begin{entrée}{tɤ-rɟit}{}{ⓔtɤ-rɟit} 
\classe{np} 
\begin{définition}\pfra{enfant}\end{définition}
\begin{définition}\pcmn{孩子}\end{définition}\relationsémantique{参考}{\lien{ⓔrɤrɟit}{rɤrɟit}}\étymologie{rgʲud}\end{entrée}

\begin{entrée}{tɤrɟɯsti}{}{ⓔtɤrɟɯsti} 
\classe{n} 
\begin{définition}\pfra{enfant unique}\end{définition}
\begin{définition}\pcmn{独生子}\end{définition}\relationsémantique{参考}{\lien{ⓔtɤ-rɟit}{tɤ-rɟit}}\end{entrée}

\begin{entrée}{tɤrka}{₁}{ⓔtɤrkaⓗ1} 
\classe{n} 
\begin{définition}\pfra{mule}\end{définition}
\begin{définition}\pcmn{骡子}\end{définition}\end{entrée}

\begin{entrée}{tɤrka}{₂}{ⓔtɤrkaⓗ2} 
\classe{n} 
\begin{définition}\pfra{jumeaux}\end{définition}
\begin{définition}\pcmn{双胞胎}\end{définition}\end{entrée}

\begin{entrée}{tɤrkakɕi}{}{ⓔtɤrkakɕi} 
\classe{n} 
\begin{définition}\pfra{chien de berger}\end{définition}
\begin{définition}\pcmn{牧羊犬}\end{définition}\étymologie{kʰʲi}\end{entrée}

\begin{entrée}{tɤ-rkhɤrkhɤt}{}{ⓔtɤ-rkhɤrkhɤt} 
\classe{np} 
\begin{définition}\pfra{chemin de montagne en pierre avec des marches}\end{définition}
\begin{définition}\pcmn{用石板铺成的山路}\end{définition}
\begin{exemple}\pjya{cupa-rkhɤrkhɤt}\hspace{5pt}\pcmn{石板山路}\end{exemple}\end{entrée}

\begin{entrée}{tɤrkhɤz}{}{ⓔtɤrkhɤz} 
\classe{n} 
\begin{définition}\pfra{crasse qui s'accumule lorsqu'on ne se lave pas pendant longtemps}\end{définition}
\begin{définition}\pcmn{长期不洗而积累下来的污垢}\end{définition}\end{entrée}

\begin{entrée}{tɤ-rkhom}{}{ⓔtɤ-rkhom} 
\classe{np} 
\begin{définition}\pfra{rachis (plume)}\end{définition}
\begin{définition}\pcmn{羽干}\end{définition}\end{entrée}

\begin{entrée}{tɤrkopa}{}{ⓔtɤrkopa} 
\classe{n} 
\begin{définition}\pfra{forcer}\end{définition}
\begin{définition}\pcmn{迫使}\end{définition}
\begin{exemple}\pjya{ɯ-tɕɯ tɤrkopa ʑo jo-sɯxɕe ɕti ma ɯʑo kɯ pjɤ-nɤla pjɤ-maʁ}\hspace{5pt}\pcmn{他是强迫儿子去的,他儿子不是自愿的}\end{exemple}
\begin{exemple}\pjya{ɯ-tɕɯ tɤrkopa ʑo chɤ-sɯɕkɯt}\hspace{5pt}\pcmn{他强迫儿子把饭吃完了}\end{exemple}\relationsémantique{同义词}{\lien{ⓔmɤkɯftshi}{mɤkɯftshi}}\end{entrée}

\begin{entrée}{tɤrkoz}{}{ⓔtɤrkoz} 
\classe{n} 
\begin{définition}\pfra{exprès, de force}\end{définition}
\begin{définition}\pcmn{故意,强迫}\end{définition}
\begin{exemple}\pjya{kɤ-ndza a-ʁjiz mɯ́j-ɣi ri ɯʑo kɯ tɤrkoz thɯ́-wɣ-sɯ-ndza-a}\hspace{5pt}\pcmn{虽然我没饿,但他强迫我吃}\end{exemple}
\begin{exemple}\pjya{tɤrkoz tɤ-ndza-t-a pɯ-ra}\hspace{5pt}\pcmn{我被迫吃了}\end{exemple}
\begin{exemple}\pjya{ɯʑo kɯ tɤrkoz ta-lɤt ɕti}\hspace{5pt}\pcmn{是他故意打的}\end{exemple}
\begin{exemple}\pjya{ɯ-jaʁ tɤrcoʁ kɯ-tu nɯ, a-ŋga ɯ-taʁ tɤrkoz na-mar/na-sɤtɕaʁ}\hspace{5pt}\pcmn{他把手上的泥巴故意擦在了我衣服上}\end{exemple}\relationsémantique{同义词}{\lien{ⓔmɤkɯftshi}{mɤkɯftshi}}\end{entrée}

\begin{entrée}{tɤrkɯ}{}{ⓔtɤrkɯ} 
\classe{n} 
\begin{définition}\pfra{support pour les seaux d'eau que l'on porte sur le dos}\end{définition}
\begin{définition}\pcmn{背水的时候,用来垫水桶底子的圆圈}\end{définition}\relationsémantique{参考}{\lien{ⓔaɣɯrkɯrkɯ}{aɣɯrkɯrkɯ}}\end{entrée}

\begin{entrée}{tɤ-rkuz}{}{ⓔtɤ-rkuz} 
\classe{np} 
\begin{définition}\pfra{cadeau}\end{définition}
\begin{définition}\pcmn{礼物(临走之前给的)}\end{définition}
\begin{exemple}\pjya{aʑo tɤ-rɤŋga-t-a, tɕe a-me kɯ a-rkuz rŋɯl ta-rku (ta-βzu)}\hspace{5pt}\pcmn{我临走之前,我女儿给了我一点钱}\end{exemple}\end{entrée}

\begin{entrée}{tɤrmbɣo}{}{ⓔtɤrmbɣo} 
\classe{n} 
\begin{définition}\pfra{tambour}\end{définition}
\begin{définition}\pcmn{鼓}\end{définition}\end{entrée}

\begin{entrée}{tɤrmbja}{}{ⓔtɤrmbja} 
\classe{n} 
\begin{définition}\pfra{éclair}\end{définition}
\begin{définition}\pcmn{闪电}\end{définition}
\begin{exemple}\pjya{ftɕar tɕe tɯ-mɯ lɤt tɯ-kha tɕe tɤrmbja tu-βze ŋgrɤl, ɯ-mphru tɕe mbɣɯrloʁ tu-βze ŋu}\hspace{5pt}\pcmn{夏天下雨的时候经常会出现闪电,然后紧接着就会打雷}\end{exemple}
\begin{exemple}\pjya{tɤrmbja ɲɯ-ɤsɯ-βzu}\hspace{5pt}\pcmn{在闪电}\end{exemple}\end{entrée}

\begin{entrée}{tɤrmbjajmɤɣ}{}{ⓔtɤrmbjajmɤɣ} 
\classe{n} 
\begin{définition}\pfra{une espèce de champignon}\end{définition}
\begin{définition}\pcmn{一种蘑菇}\end{définition}
\begin{exemple}\pjya{tɤrmbjajmɤɣ to-ɬoʁ}\hspace{5pt}\pcmn{蓝菌长出来了}\end{exemple}\relationsémantique{同义词}{\lien{ⓔkɤrŋijmɤɣ}{kɤrŋijmɤɣ}}\end{entrée}

\begin{entrée}{tɤrmbjɤβ}{}{ⓔtɤrmbjɤβ} 
\classe{n} 
\begin{définition}\pfra{blé en botte}\end{définition}
\begin{définition}\pcmn{捆成一把的麦杆}\end{définition}\end{entrée}

\begin{entrée}{tɤ-rme}{}{ⓔtɤ-rme} 
\classe{np} \paradigme{comit}{kɤ́rmɯrme}
\begin{définition}\pfra{poils}\end{définition}
\begin{définition}\pcmn{毛}\end{définition}\relationsémantique{参考}{\lien{ⓔaɣɯrme}{aɣɯrme}}\end{entrée}

\begin{entrée}{tɤ-rmi}{}{ⓔtɤ-rmi} 
\classe{np} 
\begin{définition}\pfra{nom}\end{définition}
\begin{définition}\pcmn{名字}\end{définition}
\begin{exemple}\pjya{ɯ-rmi tɤ-tɕɤt-i (=tɤ-sɤrmi-j)}\hspace{5pt}\pcmn{我们给他取了名字}\end{exemple}
\begin{exemple}\pjya{a-tɤ-rmi pɯ-rɤt}\hspace{5pt}\pcmn{你给我写名单}\end{exemple}
\begin{exemple}\pjya{ɯ-rmi ɲɯ-ɬoʁ}\hspace{5pt}\pcmn{很出名}\end{exemple}\relationsémantique{参考}{\lien{ⓔrmi}{rmi}}\relationsémantique{参考}{\lien{ⓔsɤrmi}{sɤrmi}}\end{entrée}

\begin{entrée}{tɤrmɯɣlu}{}{ⓔtɤrmɯɣlu} 
\classe{n} 
\begin{définition}\pfra{année du dragon}\end{définition}
\begin{définition}\pcmn{龙年}\end{définition}\end{entrée}

\begin{entrée}{tɤ-rmɯχtɕɤz}{}{ⓔtɤ-rmɯχtɕɤz} 
\classe{np} 
\begin{définition}\pfra{surnom}\end{définition}
\begin{définition}\pcmn{小名}\end{définition}\relationsémantique{参考}{\lien{ⓔtɤ-rmi}{tɤ-rmi}}\end{entrée}

\begin{entrée}{tɤrɲɟo}{}{ⓔtɤrɲɟo} 
\classe{n} 
\begin{définition}\pfra{étagère où l'on pose les outils de cuisine}\end{définition}
\begin{définition}\pcmn{厨架;放厨具的木板(钉在墙上)}\end{définition}\end{entrée}

\begin{entrée}{tɤ-rɴɢioʁ}{}{ⓔtɤ-rɴɢioʁ} 
\classe{np} 
\begin{définition}\pfra{invagination}\end{définition}
\begin{définition}\pcmn{槽}\end{définition}
\begin{exemple}\pjya{tɤ-rɴɢioʁ thɯ-tɕat-a / thɯ-βzu-t-a}\hspace{5pt}\pcmn{我挖了一条槽}\end{exemple}\relationsémantique{参考}{\lien{ⓔtɤ-rqhioʁ}{tɤ-rqhioʁ}}\relationsémantique{参考}{\lien{ⓔarɤrɴɢioʁ}{arɤrɴɢioʁ}}\end{entrée}

\begin{entrée}{tɤ-ro}{}{ⓔtɤ-ro} 
\classe{np} \sens{1}
\begin{définition}\pfra{en trop}\end{définition}
\begin{définition}\pcmn{多余的}\end{définition}\sens{2}
\begin{définition}\pfra{reste}\end{définition}
\begin{définition}\pcmn{剩下的部分}\end{définition}
\begin{exemple}\pjya{kɯki tɤ-ro tɕe, nɯ ma mɯ́j-ra}\hspace{5pt}\pcmn{太多了,不需要了}\end{exemple}
\begin{exemple}\pjya{tɯ-tɣa ro ro kɯ-rɲɟi}\hspace{5pt}\pcmn{一拃多一点}\end{exemple}
\begin{exemple}\pjya{ki aʑo a-ro ŋu tɕe, ɯ-tɯ-ndze}\hspace{5pt}\pcmn{这是我吃剩的,你吃不吃?}\end{exemple}\relationsémantique{参考}{\lien{ⓔɯ-rozre}{ɯ-rozre}}\end{entrée}

\begin{entrée}{tɤrpat}{}{ⓔtɤrpat} 
\classe{n} 
\begin{définition}\pfra{suie sur le plafond}\end{définition}
\begin{définition}\pcmn{沾在天花板上的烟黑【烟层】}\end{définition}
\begin{exemple}\pjya{tɤrpat ɯ-mdoʁ}\hspace{5pt}\pcmn{咖啡色}\end{exemple}\end{entrée}

\begin{entrée}{tɤ-rpi}{}{ⓔtɤ-rpi} 
\classe{np} 
\begin{définition}\pfra{soutra}\end{définition}
\begin{définition}\pcmn{(诵)经}\end{définition}
\begin{exemple}\pjya{tɤ-rpi wuma ʑo kɯ-wxti ɲɤ-sɯ-βzu-nɯ pjɤ-ra}\hspace{5pt}\pcmn{只好请(喇嘛)诵经}\end{exemple}\end{entrée}

\begin{entrée}{tɤ-rpɯ}{}{ⓔtɤ-rpɯ} 
\classe{np} 
\begin{définition}\pfra{oncle (frère de la mère et ses fils)}\end{définition}
\begin{définition}\pcmn{舅舅;舅舅的儿子}\end{définition}
\begin{exemple}\pjya{a-rpɯ}\hspace{5pt}\pcmn{我的舅舅}\end{exemple}\relationsémantique{参考}{\lien{ⓔkɤndʑɯrpɯftsa}{kɤndʑɯrpɯftsa}}\end{entrée}

\begin{entrée}{tɤ-rqhu}{}{ⓔtɤ-rqhu} 
\classe{np} \sens{1}
\begin{définition}\pfra{enveloppe, coquille, carapace}\end{définition}
\begin{définition}\pcmn{壳}\end{définition}\sens{2}
\begin{définition}\pfra{écorce}\end{définition}
\begin{définition}\pcmn{树皮}\end{définition}\end{entrée}

\begin{entrée}{tɤ-rqhioʁ}{}{ⓔtɤ-rqhioʁ} 
\classe{np} 
\begin{définition}\pfra{invagination, entaille}\end{définition}
\begin{définition}\pcmn{槽}\end{définition}\relationsémantique{参考}{\lien{ⓔtɤ-rɴɢioʁ}{tɤ-rɴɢioʁ}}\end{entrée}

\begin{entrée}{tɤrʁaʁ}{}{ⓔtɤrʁaʁ} 
\classe{n} 
\begin{définition}\pfra{proie}\end{définition}
\begin{définition}\pcmn{猎物}\end{définition}\relationsémantique{参考}{\lien{ⓔɣɤrʁaʁ}{ɣɤrʁaʁ}}\relationsémantique{参考}{\lien{ⓔnɤrʁaʁ}{nɤrʁaʁ}}\end{entrée}

\begin{entrée}{tɤrʁaʁɕa}{}{ⓔtɤrʁaʁɕa} 
\classe{n} 
\begin{définition}\pfra{viande issue de la chasse}\end{définition}
\begin{définition}\pcmn{猎物的肉}\end{définition}\end{entrée}

\begin{entrée}{tɤrʁaʁkɕi}{}{ⓔtɤrʁaʁkɕi} 
\classe{n} 
\begin{définition}\pfra{chien de chasse}\end{définition}
\begin{définition}\pcmn{猎狗}\end{définition}\end{entrée}

\begin{entrée}{tɤ-rtaʁ}{}{ⓔtɤ-rtaʁ} 
\classe{np} 
\begin{définition}\pfra{branche}\end{définition}
\begin{définition}\pcmn{树杈}\end{définition}\relationsémantique{参考}{\lien{ⓔartaʁ}{artaʁ}}\end{entrée}

\begin{entrée}{tɤ-rtɕhɣaʁ,tɕɤt}{}{ⓔtɤ-rtɕhɣaʁ,tɕɤt} 
\classe{np}
\classe{vt} 
\begin{définition}\pfra{mettre des bâtons dans les roues, entraver}\end{définition}
\begin{définition}\pcmn{作梗}\end{définition}
\begin{exemple}\pjya{ɯ-rtɕhɣaʁ ɲɤ-tɕɤt}\hspace{5pt}\pcmn{(工作本来很顺利),是他从中作梗}\end{exemple}\relationsémantique{Component 1}{\lien{}{tɤ-rtɕhɣaʁ}}\relationsémantique{Component 2}{\lien{ⓔtɕɤt}{tɕɤt}}\end{entrée}

\begin{entrée}{tɤ-rtɕi}{}{ⓔtɤ-rtɕi} 
\classe{np} 
\begin{définition}\pfra{complément alimentaire}\end{définition}
\begin{définition}\pcmn{补品}\end{définition}
\begin{exemple}\pjya{a-rtɕi kɤ-βzu ɲɯ-ra ma a-qhoχpa mɯ́j-sna}\hspace{5pt}\pcmn{要补养身体,因为内脏不好}\end{exemple}\end{entrée}

\begin{entrée}{tɤ-rte}{}{ⓔtɤ-rte} 
\classe{np} \paradigme{comit}{kɤ́rtɯrte}
\begin{définition}\pfra{coiffe, chapeau}\end{définition}
\begin{définition}\pcmn{头帕;帽子}\end{définition}
\begin{exemple}\pjya{a-rte}\hspace{5pt}\pcmn{我的帽子}\end{exemple}
\begin{exemple}\pjya{nɤ-rte ma-nɯ-tɯ-nɯ-βde}\hspace{5pt}\pcmn{你不要把帽子弄丢了}\end{exemple}\relationsémantique{参考}{\lien{ⓔnɤrte}{nɤrte}}\end{entrée}

\begin{entrée}{tɤrtoʁlu}{}{ⓔtɤrtoʁlu} 
\classe{n} 
\begin{définition}\pfra{colostrum}\end{définition}
\begin{définition}\pcmn{初乳,母牛下了牛犊之后第一次挤的奶}\end{définition}
\begin{exemple}\pjya{nɯŋa ɲo-ɬoʁ tɕe ɯ-tɤrtoʁlu pɯ-arɕo pɯ-tsu}\hspace{5pt}\pcmn{奶牛生了仔,但是初乳的阶段已经过了}\end{exemple}\end{entrée}

\begin{entrée}{tɤrtsa}{}{ⓔtɤrtsa} 
\classe{n} 
\begin{définition}\pfra{vague}\end{définition}
\begin{définition}\pcmn{波浪;波纹}\end{définition}
\begin{exemple}\pjya{tɤrtsa to-βzu}\hspace{5pt}\pcmn{起了波浪}\end{exemple}\end{entrée}

\begin{entrée}{tɤ-rtsɤɣ}{}{ⓔtɤ-rtsɤɣ} 
\classe{clf} 
\begin{définition}\pfra{un étage}\end{définition}
\begin{définition}\pcmn{一层楼}\end{définition}
\begin{exemple}\pjya{χsɤ-rtsɤɣ}\hspace{5pt}\pcmn{三层楼}\end{exemple}
\begin{exemple}\pjya{ki kha ki ɯ-tɤ-rtsɤɣ ɲɯ-mbro}\hspace{5pt}\pcmn{这个房子(每一)层楼都很高}\end{exemple}\étymologie{rtseg}\end{entrée}

\begin{entrée}{tɤ-rtsho}{}{ⓔtɤ-rtsho} 
\classe{np} \sens{1}
\begin{définition}\pfra{surface de la partie coupée}\end{définition}
\begin{définition}\pcmn{锯过;砍过;剪过的口子}\end{définition}\sens{2}
\begin{définition}\pfra{éteule (de blé)}\end{définition}
\begin{définition}\pcmn{(麦)桩、(麦)茬}\end{définition}\end{entrée}

\begin{entrée}{tɤ-rtshom}{}{ⓔtɤ-rtshom} 
\classe{np} 
\begin{définition}\pfra{bruit}\end{définition}
\begin{définition}\pcmn{噪音;声音}\end{définition}
\begin{exemple}\pjya{tɯrme ɲɯ-nɤŋkɯŋke ma ɯ-rtshom ɣɤʑu}\hspace{5pt}\pcmn{有人在走来走去,(听得到)声音}\end{exemple}\end{entrée}

\begin{entrée}{tɤ-rtsi}{}{ⓔtɤ-rtsi} 
\classe{np} 
\begin{définition}\pfra{huile de porc}\end{définition}
\begin{définition}\pcmn{猪油}\end{définition}\relationsémantique{参考}{\lien{ⓔaɣɯrtsi}{aɣɯrtsi}}\étymologie{rtsi}\end{entrée}

\begin{entrée}{tɤ-rtsɯz}{}{ⓔtɤ-rtsɯz} 
\classe{np} 
\begin{définition}\pfra{nombre, chiffre}\end{définition}
\begin{définition}\pcmn{数目(计算的结果)}\end{définition}
\begin{exemple}\pjya{ɯ-rtsɯz ko-ndo (=ɯ-χsɤr ko-ndo)}\hspace{5pt}\pcmn{他记下了数字}\end{exemple}\relationsémantique{参考}{\lien{ⓔrtsi}{rtsi}}\étymologie{rtsis}\end{entrée}

\begin{entrée}{tɤ-rʑaβ}{}{ⓔtɤ-rʑaβ} 
\classe{np} 
\begin{définition}\pfra{épouse}\end{définition}
\begin{définition}\pcmn{妻子}\end{définition}
\begin{exemple}\pjya{a-rʑaβ}\hspace{5pt}\pcmn{我的妻子}\end{exemple}\end{entrée}

\begin{entrée}{tɤ-rʑaʁ}{₁₂}{ⓔtɤ-rʑaʁⓗ1ⓗ2} 
\classe{clf}
\classe{np} 
\begin{définition}\pfra{une nuit}\end{définition}
\begin{définition}\pcmn{一夜}\end{définition}
\begin{sous-entrée}{tɤ-rʑaʁ}{ⓔtɤ-rʑaʁⓗ1ⓝtɤ-rʑaʁ}\end{sous-entrée}

\begin{définition}\pfra{temps}\end{définition}
\begin{définition}\pcmn{时间}\end{définition}
\begin{exemple}\pjya{tɤ-rʑaʁ tɤ-rɲɟi tɕe, mɤ-saχsɤl}\hspace{5pt}\pcmn{时间长了就会不清楚}\end{exemple}
\begin{exemple}\pjya{ɯ-rʑaʁ ɲɯ-zri / ɯ-rʑaʁ mɯ́j-zri}\hspace{5pt}\pcmn{他很无聊/他不无聊}\end{exemple}
\begin{exemple}\pjya{a-rʑaʁ mɯ́j-ɕe}\hspace{5pt}\pcmn{我很无聊}\end{exemple}
\begin{exemple}\pjya{tɤ-rʑaʁ kɯmɤlɤxso a-mɤ-nɯ-ɕe ma nɤja}\hspace{5pt}\pcmn{不要浪费时间,因为可惜}\end{exemple}
\begin{exemple}\pjya{tɤ-rʑaʁ nɯ pɤrmɤloŋ ɲɯ-ɕe mɤ-pe}\hspace{5pt}\pcmn{浪费时间是不好的}\end{exemple}
\begin{exemple}\pjya{nɤ-rʑaʁ nɯfse ʑo a-mɤ-nɯ-tɯ-nɯɕe}\hspace{5pt}\pcmn{你不要白白浪费时间}\end{exemple}
\begin{exemple}\pjya{tɤ-rʑaʁ ɯ-taʁ nɯ tɕu ju-kɯ-zɣɯt ra}\hspace{5pt}\pcmn{要在规定的时间到达}\end{exemple}\end{entrée}

\begin{entrée}{tɤ-rʑɯɣ}{}{ⓔtɤ-rʑɯɣ} 
\classe{np} 
\begin{définition}\pfra{ride}\end{définition}
\begin{définition}\pcmn{皱纹}\end{définition}\end{entrée}

\begin{entrée}{tɤ-ʁamɟa}{}{ⓔtɤ-ʁamɟa} 
\classe{np} 
\begin{définition}\pfra{retard}\end{définition}
\begin{définition}\pcmn{耽误}\end{définition}
\begin{exemple}\pjya{a-ʁamɟa pɯ-tu}\hspace{5pt}\pcmn{耽误了我的时间}\end{exemple}
\begin{exemple}\pjya{nɤ-ʁamɟa pɯ-sɤβzu-t-a}\hspace{5pt}\pcmn{我耽误了你的时间}\end{exemple}
\begin{exemple}\pjya{nɤ-ʁamɟa ɣɤʑu}\hspace{5pt}\pcmn{耽误了你的时间}\end{exemple}\relationsémantique{参考}{\lien{ⓔznɤʁamɟa}{znɤʁamɟa}}\end{entrée}

\begin{entrée}{tɤ-ʁar}{}{ⓔtɤ-ʁar} 
\classe{np} 
\begin{définition}\pfra{ailes}\end{définition}
\begin{définition}\pcmn{翅膀}\end{définition}\relationsémantique{参考}{\lien{ⓔtɯ-ʁar}{tɯ-ʁar}}\end{entrée}

\begin{entrée}{tɤ-ʁarndzom}{}{ⓔtɤ-ʁarndzom} 
\classe{np} 
\begin{définition}\pfra{os des ailes}\end{définition}
\begin{définition}\pcmn{翅膀的骨头}\end{définition}\end{entrée}

\begin{entrée}{tɤʁaʁ}{}{ⓔtɤʁaʁ} 
\classe{n} 
\begin{définition}\pfra{fête, réunion}\end{définition}
\begin{définition}\pcmn{聚会}\end{définition}
\begin{exemple}\pjya{tɤʁaʁ ɲɯ-sɤscit}\hspace{5pt}\pcmn{聚会很开心}\end{exemple}\relationsémantique{参考}{\lien{ⓔnɤʁaʁ}{nɤʁaʁ}}\relationsémantique{参考}{\lien{}{sɤʁaʁ}}\end{entrée}

\begin{entrée}{tɤ-ʁdɤn}{}{ⓔtɤ-ʁdɤn} 
\classe{np} 
\begin{définition}\pfra{coussin}\end{définition}
\begin{définition}\pcmn{垫子}\end{définition}\relationsémantique{参考}{\lien{ⓔnɤʁdɤn}{nɤʁdɤn}}\étymologie{gdan}\end{entrée}

\begin{entrée}{tɤ-ʁjar}{}{ⓔtɤ-ʁjar} 
\classe{np} 
\begin{définition}\pfra{fils de chaîne}\end{définition}
\begin{définition}\pcmn{经线}\end{définition}
\begin{exemple}\pjya{kɤ-taʁ chɯ́-wɣ-βzu tɕe tɤ-ri lo-thi lu-kɯ-ɕe nɯ ɯ-ʁjar ŋu tɤ-ri ku-ndi ku-kɤ-lɤt nɯ ɯ-jlɤβ ŋu}\hspace{5pt}\pcmn{织布时,上下竖着的线叫经线,左右穿过去的线叫纬线。}\end{exemple}\relationsémantique{反义词}{\lien{ⓔtɯ-jlɤβ}{tɯ-jlɤβ}}\end{entrée}

\begin{entrée}{tɤ-ʁjoʁ}{}{ⓔtɤ-ʁjoʁ}\relationsémantique{参考}{\lien{ⓔʁjoʁ}{ʁjoʁ}}\end{entrée}

\begin{entrée}{tɤ-ʁlapaʁtsa}{}{ⓔtɤ-ʁlapaʁtsa} 
\classe{np} 
\begin{définition}\pfra{arrière-bras}\end{définition}
\begin{définition}\pcmn{胳膊}\end{définition}\end{entrée}

\begin{entrée}{tɤsapɣɤtɕɯ}{}{ⓔtɤsapɣɤtɕɯ} 
\classe{n} 
\begin{définition}\pfra{parus sp.}\end{définition}
\begin{définition}\pcmn{山雀}\end{définition}\end{entrée}

\begin{entrée}{tɤsɤɣ}{}{ⓔtɤsɤɣ} 
\classe{n} 
\begin{définition}\pfra{amant}\end{définition}
\begin{définition}\pcmn{情夫}\end{définition}
\begin{exemple}\pjya{aʑo a-tɤsɤɣ me}\hspace{5pt}\pcmn{我没有情夫}\end{exemple}\relationsémantique{参考}{\lien{ⓔnɤsɤɣ}{nɤsɤɣ}}\end{entrée}

\begin{entrée}{tɤsɤɣʑa}{}{ⓔtɤsɤɣʑa} 
\classe{n} 
\begin{définition}\pfra{type de chanvre}\end{définition}
\begin{définition}\pcmn{大麻的一种}\end{définition}\relationsémantique{参考}{\lien{ⓔtasa}{tasa}}\end{entrée}

\begin{entrée}{tɤsɤmu}{}{ⓔtɤsɤmu} 
\classe{n} 
\begin{définition}\pfra{type de chanvre}\end{définition}
\begin{définition}\pcmn{大麻的一种(能结种子的)}\end{définition}\relationsémantique{参考}{\lien{ⓔtasa}{tasa}}\end{entrée}

\begin{entrée}{tɤsɤrŋu}{}{ⓔtɤsɤrŋu} 
\classe{n} 
\begin{définition}\pfra{grains de chanvre frits}\end{définition}
\begin{définition}\pcmn{炒的麻子}\end{définition}\relationsémantique{参考}{\lien{ⓔrŋu}{rŋu}}\relationsémantique{参考}{\lien{ⓔtasa}{tasa}}\end{entrée}

\begin{entrée}{tɤsɤsqɤri}{}{ⓔtɤsɤsqɤri} 
\classe{n} 
\begin{définition}\pfra{fil de lin}\end{définition}
\begin{définition}\pcmn{麻线}\end{définition}\relationsémantique{参考}{\lien{ⓔtasa}{tasa}}\end{entrée}

\begin{entrée}{tɤscɤr}{}{ⓔtɤscɤr} 
\classe{n} 
\begin{définition}\pfra{frayeur}\end{définition}
\begin{définition}\pcmn{恐惧}\end{définition}
\begin{exemple}\pjya{tɤ-scɤr kɯ nɯ-kɤpa ʑo ɲɤ-me}\hspace{5pt}\pcmn{他们吓得不知所措}\end{exemple}\relationsémantique{参考}{\lien{ⓔnɤscɤr}{nɤscɤr}}\end{entrée}

\begin{entrée}{tɤ-scoz}{}{ⓔtɤ-scoz} 
\classe{np} 
\begin{définition}\pfra{lettre}\end{définition}
\begin{définition}\pcmn{字;信}\end{définition}
\begin{exemple}\pjya{@xiangbolin kɯ tɤ-scoz ɲɯ-ɤsɯ-rɤt}\hspace{5pt}\pcmn{向柏霖在写字}\end{exemple}
\begin{exemple}\pjya{a-jaʁ tɤ-scoz jɤ-ɣe}\hspace{5pt}\pcmn{我收到了一封信}\end{exemple}
\begin{exemple}\pjya{iʑo ji-rju nɯ ɯ-scoz maŋe tɕe mɯ́j-pe}\hspace{5pt}\pcmn{我们的语言没有文字是不好的}\end{exemple}\relationsémantique{参考}{\lien{ⓔrɤscoz}{rɤscoz}}\end{entrée}

\begin{entrée}{tɤ-se}{}{ⓔtɤ-se} 
\classe{np} 
\begin{définition}\pfra{sang}\end{définition}
\begin{définition}\pcmn{血}\end{définition}\end{entrée}

\begin{entrée}{tɤsepu}{}{ⓔtɤsepu} 
\classe{n} 
\begin{définition}\pfra{boudin}\end{définition}
\begin{définition}\pcmn{血肠}\end{définition}\end{entrée}

\begin{entrée}{tɤ-skrɤβ}{}{ⓔtɤ-skrɤβ} 
\classe{np} 
\begin{définition}\pfra{fil très fin, cheveu}\end{définition}
\begin{définition}\pcmn{细线;头发}\end{définition}
\begin{exemple}\pjya{@cai ɯ-ŋgɯ tɤ-skrɤβ ɣɤʑu}\hspace{5pt}\pcmn{菜里面有头发}\end{exemple}\end{entrée}

\begin{entrée}{tɤ-sno}{}{ⓔtɤ-sno} 
\classe{np} 
\begin{définition}\pfra{selle}\end{définition}
\begin{définition}\pcmn{马鞍}\end{définition}
\begin{exemple}\pjya{mbro ɯ-sno thɯ-ta-t-a}\hspace{5pt}\pcmn{我给马装上了鞍子}\end{exemple}
\begin{sous-entrée}{tɤ-sno ɯ-jaʁ}{ⓔtɤ-snoⓝtɤ-sno ɯ-jaʁ} 
\classe{n} 
\begin{définition}\pfra{partie inférieure de la selle en contact avec le dos du cheval, faite en laine}\end{définition}
\begin{définition}\pcmn{马鞍垫;鞍鞯}\end{définition}\end{sous-entrée}

\end{entrée}

\begin{entrée}{tɤ-snom}{}{ⓔtɤ-snom} 
\classe{np} 
\begin{définition}\pfra{sœur (employé par les garçons)}\end{définition}
\begin{définition}\pcmn{姐姐;妹妹(男性专用)}\end{définition}
\begin{exemple}\pjya{a-snom}\hspace{5pt}\pcmn{我的姐姐}\end{exemple}\end{entrée}

\begin{entrée}{tɤ-sɲa}{}{ⓔtɤ-sɲa} 
\classe{np} 
\begin{définition}\pfra{tresse}\end{définition}
\begin{définition}\pcmn{辫子}\end{définition}
\begin{exemple}\pjya{a-sɲa}\hspace{5pt}\pcmn{我的辫子}\end{exemple}\relationsémantique{同义词}{\lien{ⓔtɤpjɤz}{tɤpjɤz}}\end{entrée}

\begin{entrée}{tɤsɲɤmtsɯ}{}{ⓔtɤsɲɤmtsɯ} 
\classe{n} 
\begin{définition}\pfra{broche}\end{définition}
\begin{définition}\pcmn{夹头发的装饰品}\end{définition}\end{entrée}

\begin{entrée}{tɤ-sŋɯt}{}{ⓔtɤ-sŋɯt} 
\classe{np} 
\begin{définition}\pfra{morsure}\end{définition}
\begin{définition}\pcmn{(咬)一口}\end{définition}
\begin{exemple}\pjya{a-jaʁ mɯ́j-so tɕe, a-sŋɯt kɯ kɤ-sɯ-ndo-t-a}\hspace{5pt}\pcmn{因为我手上拿着东西,所以用牙齿咬住了}\end{exemple}
\begin{exemple}\pjya{aʑo paχɕi ɯ-taʁ tɤ-sŋɯt tu-nɯ-lat-a ɲɯ-jɤɣ ma mbrɯtɕɯ mɯ́j-ra}\hspace{5pt}\pcmn{我可以就这么咬苹果,不需要刀子}\end{exemple}\end{entrée}

\begin{entrée}{tɤ-spa}{}{ⓔtɤ-spa} 
\classe{np} \sens{1}
\begin{définition}\pfra{matériau}\end{définition}
\begin{définition}\pcmn{材料}\end{définition}\sens{2}
\begin{définition}\pfra{utilité, but}\end{définition}
\begin{définition}\pcmn{用途;目标}\end{définition}\sens{3}
\begin{définition}\pfra{pour}\end{définition}
\begin{définition}\pcmn{用来……}\end{définition}
\begin{exemple}\pjya{khɯna nɯ kha kɯ-rɯru ɯ-spa ŋu}\hspace{5pt}\pcmn{看家是狗的义务}\end{exemple}
\begin{exemple}\pjya{nɯ tɕhi ɯ-spa ɲɯ-ŋu ?}\hspace{5pt}\pcmn{那个有什么用呢?}\end{exemple}
\begin{exemple}\pjya{tɤ-pɤtso kɤ-mbi ɯ-spa kɯ-chi to-χtɯ.}\hspace{5pt}\pcmn{他买糖果给小孩子吃了}\end{exemple}\end{entrée}

\begin{entrée}{tɤ-spɯ}{}{ⓔtɤ-spɯ} 
\classe{np} 
\begin{définition}\pfra{pus}\end{définition}
\begin{définition}\pcmn{脓}\end{définition}\relationsémantique{参考}{\lien{ⓔrɤspɯ}{rɤspɯ}}\end{entrée}

\begin{entrée}{tɤspɯɕku}{}{ⓔtɤspɯɕku} 
\classe{n} 
\begin{définition}\pfra{poireau sauvage}\end{définition}
\begin{définition}\pcmn{野韭菜}\end{définition}
\begin{exemple}\pjya{tɤ-spɯ ɕku nɯ kha ɯ-mbe ɣɯ znde ku kɯ-fse nɯ ra tu-ɬoʁ rga, ɯ-mdoʁ pɣi, ɯ-jwaʁ kɤ-lɯ-lju ŋu, ɯ-ru me, tɯ-phɯ ɯ-ŋgɯ kɯ-dɯ-dɤn tu-ɬoʁ ɕti, ɯ-mɯntoʁ ndɯβ ri dɤn. ɯ-di nɯ kɯ-ɣɤjlu kɯ-fse tu. ɕku di mnɤm ri, kɤ-ndza mɤ-mɯm. ɯ-zrɤm dɤn.}\hspace{5pt}\pcmn{\lien{}{tɤ-spɯ ɕku} 一般生长在旧房子墙壁顶上,是灰色的,叶子是圆柱形的,没有茎,一秆里有很多根。花小而多。发出腥的味道。有点葱的味道,但不好吃。根很多。}\end{exemple}\end{entrée}

\begin{entrée}{tɤ-sqhaj}{}{ⓔtɤ-sqhaj} 
\classe{np} 
\begin{définition}\pfra{sœur (employé par les filles)}\end{définition}
\begin{définition}\pcmn{姐姐;妹妹(女性专用)}\end{définition}
\begin{exemple}\pjya{a-sqhaj}\hspace{5pt}\pcmn{我的姐姐}\end{exemple}\end{entrée}

\begin{entrée}{tɤstu}{}{ⓔtɤstu} 
\classe{vi} 
\begin{définition}\pfra{au revoir}\end{définition}
\begin{définition}\pcmn{再见}\end{définition}
\begin{exemple}\pjya{kɯ-sɤfstɯn tɤ-stu je}\hspace{5pt}\pcmn{再见,照顾人家}\end{exemple}\end{entrée}

\begin{entrée}{tɤ-sta}{}{ⓔtɤ-sta} 
\classe{np} 
\begin{définition}\pfra{endroit où on va enterrer un mort}\end{définition}
\begin{définition}\pcmn{坟地}\end{définition}
\begin{exemple}\pjya{tɤ-sta pjɯ́-wɣ-lɣa tɕe, tɯ-ɕpɤβ nɯ pjɯ́-wɣ-rku ŋu}\hspace{5pt}\pcmn{挖了坟地,就把尸体装下去了}\end{exemple}\relationsémantique{参考}{\lien{ⓔtɯ-sta}{tɯ-sta}}\relationsémantique{参考}{\lien{ⓔɯ-sta}{ɯ-sta}}\end{entrée}

\begin{entrée}{tɤ-ste}{}{ⓔtɤ-ste} 
\classe{np} 
\begin{définition}\pfra{vessie}\end{définition}
\begin{définition}\pcmn{膀胱}\end{définition}\relationsémantique{参考}{\lien{ⓔtɯcɯste}{tɯcɯste}}\end{entrée}

\begin{entrée}{tɤsthoʁsi}{}{ⓔtɤsthoʁsi} 
\classe{n} \sens{1}
\begin{définition}\pfra{poutre}\end{définition}
\begin{définition}\pcmn{梁}\end{définition}\sens{2}
\begin{définition}\pfra{poutre du balcon}\end{définition}
\begin{définition}\pcmn{走缘上的小梁【撑杆】}\end{définition}
\begin{exemple}\pjya{jɤɣɤt ɯ-taʁ khɤxtu ɯ-pa stukɤr ɯ-tshɤt ɕoŋtɕa kɯ-xtshɯm chɯ-kɤ-lɤt nɯ tɤsthoʁsi rmi}\hspace{5pt}\pcmn{在走缘和房背之间代替大梁的细梁叫\lien{ⓔtɤsthoʁsi}{tɤsthoʁsi}}\end{exemple}\relationsémantique{同义词}{\lien{ⓔɕɯjaʁ}{ɕɯjaʁ}}\end{entrée}

\begin{entrée}{tɤsto}{}{ⓔtɤsto} 
\classe{n} 
\begin{définition}\pfra{grande jarre}\end{définition}
\begin{définition}\pcmn{大坛子}\end{définition}\end{entrée}

\begin{entrée}{tɤtar}{}{ⓔtɤtar} 
\classe{n} 
\begin{définition}\pfra{bâton fin}\end{définition}
\begin{définition}\pcmn{细木棍}\end{définition}\relationsémantique{参考}{\lien{ⓔnɤtar}{nɤtar}}\relationsémantique{参考}{\lien{ⓔjaχpɤtar}{jaχpɤtar}}\end{entrée}

\begin{entrée}{tɤtɤɣ}{}{ⓔtɤtɤɣ} 
\classe{n} 
\begin{définition}\pfra{armoire}\end{définition}
\begin{définition}\pcmn{柜子(装粮食)}\end{définition}\end{entrée}

\begin{entrée}{tɤtɕɤfkɯm}{}{ⓔtɤtɕɤfkɯm} 
\classe{n} 
\begin{définition}\pfra{pommette}\end{définition}
\begin{définition}\pcmn{酒窝}\end{définition}\relationsémantique{同义词}{\lien{ⓔkhrambakɯm}{khrambakɯm}}\end{entrée}

\begin{entrée}{tɤtɕɤri}{}{ⓔtɤtɕɤri} 
\classe{n} 
\begin{définition}\pfra{type de pas d'aiguille}\end{définition}
\begin{définition}\pcmn{大针脚的线}\end{définition}
\begin{exemple}\pjya{kɯ-mɤku tɤtɕɤri pɯ-lat-a tɕe nɯ kóʁmɯz tɤ-ɕphɤt ɯ-ta thɯ-βzu-t-a}\hspace{5pt}\pcmn{我先用针脚大一点的线把补丁固定了,然后就(用小针脚)把补丁做好了}\end{exemple}\end{entrée}

\begin{entrée}{tɤ-tɕɤz}{}{ⓔtɤ-tɕɤz} 
\classe{np} 
\begin{définition}\pfra{trace de pied}\end{définition}
\begin{définition}\pcmn{痕迹;足印}\end{définition}
\begin{exemple}\pjya{a-tɕɤz}\hspace{5pt}\pcmn{我的脚印}\end{exemple}\end{entrée}

\begin{entrée}{tɤ-tɕhɤz}{}{ⓔtɤ-tɕhɤz} 
\classe{np} 
\begin{définition}\pfra{franges colorées}\end{définition}
\begin{définition}\pcmn{吊边布(衣服边缘的彩色布料)}\end{définition}
\begin{exemple}\pjya{ɯ-ŋga ɯ-tɕhɤz ɯ-tɯ-dɤn kɯ ɲɯ-ɣɤβlɯβlɯɣ ʑo}\hspace{5pt}\pcmn{他衣服的彩色布料很多,显得很耀眼}\end{exemple}\end{entrée}

\begin{entrée}{tɤ-tɕɯ}{}{ⓔtɤ-tɕɯ} 
\classe{np} \sens{1}
\begin{définition}\pfra{fils}\end{définition}
\begin{définition}\pcmn{儿子}\end{définition}
\begin{exemple}\pjya{a-tɕɯ}\hspace{5pt}\pcmn{我的儿子}\end{exemple}\sens{2}
\begin{définition}\pfra{garçon}\end{définition}
\begin{définition}\pcmn{男孩;男子}\end{définition}\relationsémantique{参考}{\lien{ⓔarɯtɤtɕɯ}{arɯtɤtɕɯ}}\end{entrée}

\begin{entrée}{tɤtɕɯβraʁ}{}{ⓔtɤtɕɯβraʁ} 
\classe{n} 
\begin{définition}\pfra{bardane}\end{définition}
\begin{définition}\pcmn{牛蒡}\end{définition}
\begin{exemple}\pjya{tɤtɕɯβraʁ nɯ arɤndɯndɤt tu-ɬoʁ ɕti, tɕe ɯ-qa rɲɟi, ngɯt, pakuku tu-ɬoʁ cha, ɯ-ru nɯ aɣɯrnɯɕɯr tsa tɕe tu-rɲɟi tsa cha. tɯrme tɯ-fsu jamar tu-mbro cha, ɯ-ru ɯ-χcɤl tɤ-kɯ-ɣe nɯ tɕu ɯ-jwaʁ ku-ndzoʁ tɕe nɯ ɯ-rca nɯ tɕu ɯ-rtaʁ ɲɯ-ɬoʁ, ɯ-jwaʁ wuma ʑo wxti, ɯ-jwaʁ ɯ-qhu chu nɯ kɯ-wɣrum tsa ŋu, ɯ-ʁɤri nɯ kɯ-ɤrŋi tsa ŋu. ɯ-rtaʁ ɯ-kɤχcɤl raŋri tɕu ɯ-mat kɯ-dɤn ʑo ku-ndzoʁ ŋu. ɯ-mat ɯ-rqhu nɯ ɯ-mdzu tu, kɯ-tɕɯ-tɕɤr kɯ-rɲɟi tsa ŋu, kɯ-dɤn ʑo aɣɯŋgɯŋgɯ tɕe nɯ ɯ-ŋgɯ kóʁmɯz ɯ-rɣi ŋu. ɯ-rɣi ɯ-kɤχcɤl zɯ ɯ-rme tu, ɯ-rɣi wuma ʑo dɤn, ɯ-rɣi ɣɯ ɯ-rme nɯ tɯ-ɕa ɯ-taʁ nɤ tɕaʁ tɕe rɤʑa, ɯ-mat kɤ́rqhɯrqhu nɯ ku-ondzoʁjoʁ cha tɕe tɯrme tɯ-ŋga ɯ-taʁ ku-ndɤm cha tɕe βʑɯ kɯ wuma ʑo nɯɣme ma ɯ-rme ɯ-taʁ ka-ndzoʁ tɕe kɤ-sɯ-ta mɯ́j-khɯ tɕe pjɯ́-wɣ-sat ɲɯ-ŋgrɤl, tɕe núndʐa ʑɯɣsɯr rmi}\hspace{5pt}\pcmn{牛蒡到处可以生长,根又长又结实,年年生长,茎是淡红色的,长得比较高,和人一样高,茎长出来再长出叶子,叶子长出来的地方就分叉,叶子很大,背面白色,正面绿色。每一根杈子顶上结很多果实,果实的外壳有刺,又细又长,有很多层,最里面才是种子。种子头上有很多毛,那些毛碰到皮肤上就会发痒。果实连同壳会粘在一起,也会粘在人的衣服上,老鼠最怕它,因为一旦粘在皮毛上,他们无法弄掉,它甚至会使他们丧命,所以它叫“\lien{ⓔʑɯxsɯr}{ʑɯxsɯr}”}\end{exemple}\relationsémantique{参考}{\lien{ⓔʑɯxsɯr}{ʑɯxsɯr}}\end{entrée}

\begin{entrée}{tɤ-tɕɯɣ}{}{ⓔtɤ-tɕɯɣ} 
\classe{np} 
\begin{définition}\pfra{germe d'arbre}\end{définition}
\begin{définition}\pcmn{树的萌芽;新发出来的叶子}\end{définition}
\begin{exemple}\pjya{tɤ-tɕɯɣ lo-lɤt}\hspace{5pt}\pcmn{(树)发芽了}\end{exemple}\relationsémantique{参考}{\lien{ⓔɣɤtɕɯɣ}{ɣɤtɕɯɣ}}\end{entrée}

\begin{entrée}{tɤ-tɕɯpɯ}{}{ⓔtɤ-tɕɯpɯ} 
\classe{np} 
\begin{définition}\pfra{garçon}\end{définition}
\begin{définition}\pcmn{小男孩}\end{définition}\relationsémantique{参考}{\lien{ⓔtɤ-tɕɯ}{tɤ-tɕɯ}}\end{entrée}

\begin{entrée}{tɤ-tɕɯrʑaβ}{}{ⓔtɤ-tɕɯrʑaβ} 
\classe{np} 
\begin{définition}\pfra{bru}\end{définition}
\begin{définition}\pcmn{媳妇}\end{définition}\relationsémantique{参考}{\lien{ⓔtɤ-rʑaβ}{tɤ-rʑaβ}}\relationsémantique{参考}{\lien{ⓔtɤ-tɕɯ}{tɤ-tɕɯ}}\end{entrée}

\begin{entrée}{tɤte}{}{ⓔtɤte} 
\classe{adv} 
\begin{définition}\pfra{c'est à dire, en gros}\end{définition}
\begin{définition}\pcmn{总的来说}\end{définition}\end{entrée}

\begin{entrée}{tɤthu}{}{ⓔtɤthu} 
\classe{n} 
\begin{définition}\pfra{laine}\end{définition}
\begin{définition}\pcmn{毛布}\end{définition}\relationsémantique{同义词}{\lien{ⓔtɯŋgar}{tɯŋgar}}\end{entrée}

\begin{entrée}{tɤ-thɤβ}{}{ⓔtɤ-thɤβ} 
\classe{np} 
\begin{définition}\pfra{mésentente}\end{définition}
\begin{définition}\pcmn{纠纷}\end{définition}
\begin{exemple}\pjya{ndʑi-thɤβ tɤ-βzu-t-a}\hspace{5pt}\pcmn{我给他们俩调解了纠纷}\end{exemple}
\begin{exemple}\pjya{tɤ-thɤβ tɤ-fɕɤt-i}\hspace{5pt}\pcmn{我们调解了纠纷}\end{exemple}\relationsémantique{同义词}{\lien{ⓔɣɤɕphɤr}{ɣɤɕphɤr}}\end{entrée}

\begin{entrée}{tɤtho}{}{ⓔtɤtho} 
\classe{n} 
\begin{définition}\pfra{pin}\end{définition}
\begin{définition}\pcmn{松树}\end{définition}
\begin{exemple}\pjya{tɤtho nɯ zgoku kɯ-mbro tsa tu-ɬoʁ ŋu, ɯ-jwaʁ nɯ taqaβ kɯ-fse kɯ-zri ŋu, ɯ-jwaʁ nɯ ɯ-rtaʁ ɣɯ ɯ-βri aʁɤndɯndɤt ku-ndzoʁ ŋu, aɣɯjwaʁ, ɯ-mdoʁ nɯ tɯrgi ɯ-jwaʁ mdoʁ cho naχtɕɯɣ, tɯ-xpa lɤ-skɤr ɯ-mdoʁ ɲɯ-nɤsci mɤ-cha. ɯ-ru jpum tsa aɣɯrtɯrtaʁ, ɯ-tɯ-ɤɣɯrtɯrtaʁ nɯ ɕɤɣ ʑo fse, ɯ-mat tu, tɯrgi laŋlaŋ cho naχtɕɯɣ ri, artɯm. tɤtho ɯ-ru nɯ li ɕɤɣ jamar ma kɤ-rɤɣdɤt mɤ-rtaʁ. ŋgɤjpɤn nɯ li kɯ-ʑru kɤ-sɯpa ŋu, ma qajɯ kɯ mɤ-ndze. tɤtho si nɯ wuma ʑo ɯ-tɤ-ndʑɯɣ dɤn tɕe wuma ʑo kɤ-βlɯ pe.}\hspace{5pt}\pcmn{松树生长在高山上,叶子长得像针一样,比较长,叶子在枝桠上到处生长,很茂盛,颜色和杉树的颜色一样,一年四季不变色。树干比较粗,长出很多枝桠,枝桠生长的方式很像柏树,有果实,像杉树的果实一样但是是球形的。松树的树干也只能锯成和柏树那么多的几段。用松树的木料制造的木板算是比较优质的,因为蛀虫不爱咬。因为松树的树脂多,所以很好烧。}\end{exemple}\étymologie{tʰaŋ}\end{entrée}

\begin{entrée}{tɤthotʂu}{}{ⓔtɤthotʂu} 
\classe{n} 
\begin{définition}\pfra{torche en pin}\end{définition}
\begin{définition}\pcmn{松明}\end{définition}\end{entrée}

\begin{entrée}{tɤton}{}{ⓔtɤton} 
\classe{n} 
\begin{définition}\pfra{vers le haut, vers l'amont}\end{définition}
\begin{définition}\pcmn{往上;往上游}\end{définition}\end{entrée}

\begin{entrée}{tɤtshoʁ}{}{ⓔtɤtshoʁ} 
\classe{n} 
\begin{définition}\pfra{clou}\end{définition}
\begin{définition}\pcmn{钉子}\end{définition}\relationsémantique{参考}{\lien{ⓔɕɤmtshoʁ}{ɕɤmtshoʁ}}\end{entrée}

\begin{entrée}{tɤ-tshɯɣ}{}{ⓔtɤ-tshɯɣ} 
\classe{np} 
\begin{définition}\pfra{œillère}\end{définition}
\begin{définition}\pcmn{马笼头}\end{définition}
\begin{exemple}\pjya{mbro ɯ-tshɯɣ tɤ-ta-t-a}\hspace{5pt}\pcmn{我给马戴上了马笼头}\end{exemple}\end{entrée}

\begin{entrée}{tɤtshɯtsha}{}{ⓔtɤtshɯtsha} 
\classe{n} 
\begin{définition}\pfra{salpêtre}\end{définition}
\begin{définition}\pcmn{硝}\end{définition}\end{entrée}

\begin{entrée}{tɤtsoʁ}{}{ⓔtɤtsoʁ} 
\classe{n} 
\begin{définition}\pfra{Potentilla anserina (gro-ma)}\end{définition}
\begin{définition}\pcmn{人参果}\end{définition}\end{entrée}

\begin{entrée}{tɤ-tsrɯ}{}{ⓔtɤ-tsrɯ} 
\classe{np} 
\begin{définition}\pfra{pousses}\end{définition}
\begin{définition}\pcmn{萌芽}\end{définition}\end{entrée}

\begin{entrée}{tɤ-tsɯr}{}{ⓔtɤ-tsɯr} 
\classe{np} 
\begin{définition}\pfra{fissure}\end{définition}
\begin{définition}\pcmn{裂缝}\end{définition}
\begin{exemple}\pjya{ɯ-rnom ɕɤrɯ ɣɯ ɯ-tsɯr lo-ɕe (=lo-ɣɤtsɯr)}\hspace{5pt}\pcmn{他肋骨骨折了}\end{exemple}\relationsémantique{参考}{\lien{ⓔɣɤtsɯr}{ɣɤtsɯr}}\end{entrée}

\begin{entrée}{tɤtʂu}{}{ⓔtɤtʂu} 
\classe{n} 
\begin{définition}\pfra{lampe}\end{définition}
\begin{définition}\pcmn{灯}\end{définition}
\begin{exemple}\pjya{aʑo tɤtʂu ci tu-ci-a ɲɯ-ntshi}\hspace{5pt}\pcmn{我要开灯}\end{exemple}\relationsémantique{参考}{\lien{ⓔsɤtʂu}{sɤtʂu}}\end{entrée}

\begin{entrée}{tɤ-tʂɤm}{}{ⓔtɤ-tʂɤm} 
\classe{np} 
\begin{définition}\pfra{huile animale}\end{définition}
\begin{définition}\pcmn{动物的油脂}\end{définition}\end{entrée}

\begin{entrée}{tɤtʂo}{}{ⓔtɤtʂo} 
\classe{n} 
\begin{définition}\pfra{lœss}\end{définition}
\begin{définition}\pcmn{粘土;黄土}\end{définition}\end{entrée}

\begin{entrée}{tɤtɯr}{}{ⓔtɤtɯr} 
\classe{n} 
\begin{définition}\pfra{outil pour graver les motifs sur l'argent}\end{définition}
\begin{définition}\pcmn{刻花纹的工具}\end{définition}\end{entrée}

\begin{entrée}{tɤwu}{}{ⓔtɤwu} 
\classe{n} 
\begin{définition}\pfra{pleurs}\end{définition}
\begin{définition}\pcmn{哭}\end{définition}
\begin{exemple}\pjya{tɤwu kɯ a-ku ʑo tɤ-mŋɤm}\hspace{5pt}\pcmn{我哭得令我头疼}\end{exemple}\relationsémantique{参考}{\lien{ⓔɣɤwu}{ɣɤwu}}\relationsémantique{参考}{\lien{ⓔnɤwu}{nɤwu}}\end{entrée}

\begin{entrée}{tɤ-wa}{}{ⓔtɤ-wa} 
\classe{np} 
\begin{définition}\pfra{père}\end{définition}
\begin{définition}\pcmn{父亲}\end{définition}
\begin{exemple}\pjya{a-mu a-wa}\hspace{5pt}\pcmn{我父母}\end{exemple}\end{entrée}

\begin{entrée}{tɤ-wɤmɯ}{}{ⓔtɤ-wɤmɯ} 
\classe{np} 
\begin{définition}\pfra{frère (employé par les filles)}\end{définition}
\begin{définition}\pcmn{兄弟(女性专用)}\end{définition}
\begin{exemple}\pjya{a-wɤmɯ}\hspace{5pt}\pcmn{我的哥哥(弟弟)}\end{exemple}\end{entrée}

\begin{entrée}{tɤ-wi}{}{ⓔtɤ-wi} 
\classe{np} 
\begin{définition}\pfra{grand-mère}\end{définition}
\begin{définition}\pcmn{奶奶;婆婆}\end{définition}\end{entrée}

\begin{entrée}{tɤwɯ}{₂}{ⓔtɤwɯⓗ2} 
\classe{n} 
\begin{définition}\pfra{couverture de feutre}\end{définition}
\begin{définition}\pcmn{遮雨的毡子}\end{définition}\end{entrée}

\begin{entrée}{tɤ-wɯ}{₁}{ⓔtɤ-wɯⓗ1} 
\classe{np} 
\begin{définition}\pfra{grand-père (c'est le terme par lequel les animaux s'adressent aux être humains dans les histoires)}\end{définition}
\begin{définition}\pcmn{爷爷;公公 (在故事里面,也是动物对人的称呼)}\end{définition}\end{entrée}

\begin{entrée}{tɤwɯrte}{}{ⓔtɤwɯrte} 
\classe{n} 
\begin{définition}\pfra{chapeau en feutre}\end{définition}
\begin{définition}\pcmn{毡子制成的帽子}\end{définition}\relationsémantique{参考}{\lien{ⓔtɤ-rte}{tɤ-rte}}\end{entrée}

\begin{entrée}{tɤ-xtɤɣ}{}{ⓔtɤ-xtɤɣ} 
\classe{np} 
\begin{définition}\pfra{frère (employé par les garçons)}\end{définition}
\begin{définition}\pcmn{兄弟(男性专用)}\end{définition}
\begin{exemple}\pjya{a-xtɤɣ}\hspace{5pt}\pcmn{我的兄弟}\end{exemple}\end{entrée}

\begin{entrée}{tɤxtɕɤβ}{}{ⓔtɤxtɕɤβ} 
\classe{n} 
\begin{définition}\pfra{herbe pour les animaux}\end{définition}
\begin{définition}\pcmn{喂动物的草}\end{définition}\relationsémantique{参考}{\lien{ⓔɣɤxtɕɤβ}{ɣɤxtɕɤβ}}\end{entrée}

\begin{entrée}{tɤ-xtɕɤr}{}{ⓔtɤ-xtɕɤr} 
\classe{np} 
\begin{définition}\pfra{corde}\end{définition}
\begin{définition}\pcmn{用来捆东西的绳索}\end{définition}\relationsémantique{参考}{\lien{ⓔxtɕɤr}{xtɕɤr}}\end{entrée}

\begin{entrée}{tɤ-xtsɤr}{}{ⓔtɤ-xtsɤr} 
\classe{np} 
\begin{définition}\pfra{enceinte (vache)}\end{définition}
\begin{définition}\pcmn{怀孕(母牛)}\end{définition}
\begin{exemple}\pjya{nɯŋa ɯ-xtsɤr tu tɕe, kɯ-ɤrqhi ma-jɤ́-wɣ-no ra}\hspace{5pt}\pcmn{这个母牛怀孕了,不可以赶到远处去}\end{exemple}
\begin{exemple}\pjya{ftsoʁ ɯ-xtsɤr kɯ-tu}\hspace{5pt}\pcmn{怀孕的母犏牛}\end{exemple}
\begin{exemple}\pjya{qra ɯ-xtsɤr kɯ-mbro}\hspace{5pt}\pcmn{快要生的母牦牛}\end{exemple}\end{entrée}

\begin{entrée}{tɤχsɤr}{}{ⓔtɤχsɤr} 
\classe{n} 
\begin{définition}\pfra{nombre}\end{définition}
\begin{définition}\pcmn{数目}\end{définition}
\begin{exemple}\pjya{tɤχsɤr ɯ-kɯ-sti ɕti-a ma koŋla a-kɤ-spa me}\hspace{5pt}\pcmn{我只是充数的,我什么也不会}\end{exemple}\end{entrée}

\begin{entrée}{tɤχtɯχtɤt}{}{ⓔtɤχtɯχtɤt} 
\classe{n} 
\begin{définition}\pfra{avis, information}\end{définition}
\begin{définition}\pcmn{通知}\end{définition}
\begin{exemple}\pjya{tɤχtɯχtɤt ɯ-kɯ-lɤt ɯʑo pɯ-ŋu}\hspace{5pt}\pcmn{通知大家的是他}\end{exemple}
\begin{exemple}\pjya{a-tɤχtɯχtɤt na-lɤt}\hspace{5pt}\pcmn{他通知了我}\end{exemple}\end{entrée}

\begin{entrée}{tɤzdɯɣ}{}{ⓔtɤzdɯɣ} 
\classe{n} \paradigme{emphatic}{tɤzdɯɣ tɤsŋɤl}
\begin{définition}\pfra{peine}\end{définition}
\begin{définition}\pcmn{辛苦;艰苦}\end{définition}
\begin{exemple}\pjya{a-tɤzdɯɣ a-tɤsŋɤl pɯ-rtaʁ}\hspace{5pt}\pcmn{我受够了苦难}\end{exemple}\relationsémantique{参考}{\lien{ⓔnɯzdɯsŋɤl}{nɯzdɯsŋɤl}}\étymologie{sdug.bsŋal}\end{entrée}

\begin{entrée}{tɤ-zgra}{}{ⓔtɤ-zgra} 
\classe{np} 
\begin{définition}\pfra{son}\end{définition}
\begin{définition}\pcmn{声音;噪音}\end{définition}
\begin{exemple}\pjya{ɯ-pɕi tɤ-zgra ɣɤʑu}\hspace{5pt}\pcmn{外面很吵}\end{exemple}
\begin{exemple}\pjya{tɤ-zgra ɲɯ-thɯ tɕe, koŋla mɯ́j-sɤmtshɤm}\hspace{5pt}\pcmn{噪音很严重,根本就听不见}\end{exemple}
\begin{exemple}\pjya{tɤ-zgra ɲɤ-ftshi}\hspace{5pt}\pcmn{没有那么吵了}\end{exemple}\end{entrée}

\begin{entrée}{tɤzmbɯr}{}{ⓔtɤzmbɯr} 
\classe{n} 
\begin{définition}\pfra{boue}\end{définition}
\begin{définition}\pcmn{泥沙}\end{définition}
\begin{exemple}\pjya{zgo pjɤ-mbɯt tɕe, tɤzmbɯr chɤ-ɣi tɕe, ndzom ɯ-pa chɤ-sti}\hspace{5pt}\pcmn{山塌下来了,泥沙把桥下堵住了,}\end{exemple}\end{entrée}

\begin{entrée}{tɤzɲɟoʁ}{}{ⓔtɤzɲɟoʁ} 
\classe{n} 
\begin{définition}\pfra{branche flexible utilisée pour fouetter les animaux}\end{définition}
\begin{définition}\pcmn{用来打牲畜的木条}\end{définition}\relationsémantique{参考}{\lien{ⓔnɤzɲɟoʁ}{nɤzɲɟoʁ}}\end{entrée}

\begin{entrée}{tɤzraj}{}{ⓔtɤzraj} 
\classe{n} 
\begin{définition}\pfra{espèce d'arbre}\end{définition}
\begin{définition}\pcmn{银木}\end{définition}\end{entrée}

\begin{entrée}{tɤzraʁ}{}{ⓔtɤzraʁ} 
\classe{n} 
\begin{définition}\pfra{honte}\end{définition}
\begin{définition}\pcmn{廉耻心}\end{définition}
\begin{exemple}\pjya{aʑo tɤzraʁ kɯ pɯ-thɯɣ ʑo}\hspace{5pt}\pcmn{我羞愧极了}\end{exemple}
\begin{exemple}\pjya{kɯki tɯrme ki tɤzraʁ mɯ́j-tso, tɤzraʁ mɯ́j-mtshɤm}\hspace{5pt}\pcmn{这个人不要脸,不懂羞耻}\end{exemple}\relationsémantique{参考}{\lien{ⓔnɤzraʁ}{nɤzraʁ}}\end{entrée}

\begin{entrée}{tɤ-zrɤm}{}{ⓔtɤ-zrɤm} 
\classe{np} 
\begin{définition}\pfra{racine}\end{définition}
\begin{définition}\pcmn{根}\end{définition}\end{entrée}

\begin{entrée}{tɤ-ʑi}{}{ⓔtɤ-ʑi} 
\classe{np} 
\begin{définition}\pfra{jeune femme (ayant des enfants)}\end{définition}
\begin{définition}\pcmn{少奶奶}\end{définition}\end{entrée}

\begin{entrée}{tɤʑŋgrɯt}{}{ⓔtɤʑŋgrɯt} 
\classe{n} 
\begin{définition}\pfra{cicatrice}\end{définition}
\begin{définition}\pcmn{疤痕}\end{définition}
\begin{exemple}\pjya{a-tɤʑŋgrɯt}\hspace{5pt}\pcmn{我的疤痕}\end{exemple}\end{entrée}

\begin{entrée}{tɤʑri}{}{ⓔtɤʑri} 
\classe{n} 
\begin{définition}\pfra{rosée}\end{définition}
\begin{définition}\pcmn{露水}\end{définition}\relationsémantique{参考}{\lien{ⓔnɤʑri}{nɤʑri}}\end{entrée}

\begin{entrée}{tɤʑɯn}{}{ⓔtɤʑɯn} 
\classe{n} 
\begin{définition}\pfra{vers le bas, vers l'aval}\end{définition}
\begin{définition}\pcmn{往下;往下游}\end{définition}\end{entrée}

\begin{entrée}{tɕa}{}{ⓔtɕa} 
\classe{vs} \paradigme{dir}{nɯ-}
\begin{définition}\pfra{maigre}\end{définition}
\begin{définition}\pcmn{瘦弱}\end{définition}
\begin{exemple}\pjya{ɯ-sɯm ɲɯ-tɕa}\hspace{5pt}\pcmn{他没有把握,没有信心}\end{exemple}\relationsémantique{参考}{\lien{ⓔtɕale}{tɕale}}\end{entrée}

\begin{entrée}{tɕaɣi}{}{ⓔtɕaɣi} 
\classe{n} 
\begin{définition}\pfra{perroquet}\end{définition}
\begin{définition}\pcmn{鹦鹉}\end{définition}
\begin{exemple}\pjya{tɕaɣi ɲɯ-tɯ-fse}\hspace{5pt}\pcmn{你话多,唠叨得不停(你像鹦鹉一样)}\end{exemple}\end{entrée}

\begin{entrée}{tɕakɯɣ}{}{ⓔtɕakɯɣ} 
\classe{n} 
\begin{définition}\pfra{sac pour les feuilles de thé}\end{définition}
\begin{définition}\pcmn{茶叶袋}\end{définition}\étymologie{dʑa.kʰug}\end{entrée}

\begin{entrée}{tɕale}{}{ⓔtɕale} 
\classe{vs} 
\begin{définition}\pfra{maigre}\end{définition}
\begin{définition}\pcmn{瘦}\end{définition}\relationsémantique{参考}{\lien{ⓔtɕa}{tɕa}}\relationsémantique{反义词}{\lien{ⓔjpumqa}{jpumqa}}\end{entrée}

\begin{entrée}{tɕamu}{}{ⓔtɕamu} 
\classe{n} 
\begin{définition}\pfra{espèce de plante}\end{définition}
\begin{définition}\pcmn{植物的一种}\end{définition}
\begin{exemple}\pjya{tɕamu nɯ χsɯ-tɯphu tu tɕe, sɯjno ci tu tɕe ɯ-jwaʁ kɯ-rʁom tsa ci ŋu, mɤ-mbro, ɯ-mɯntoʁ ɯ-ru kɯ-xtshɯm kɯ-zri tsa tu-ɬoʁ tɕe ɯ-mɯntoʁ kɯ-wɣrum ɲɯ-lɤt ŋu, ɯ-mɯntoʁ ɯ-qhu pɕoʁ nɯ kɯ-ɣɯrni kɯ-qandʐi tsa ŋu, ɯ-jwaʁ nɯ ɲɯ́-wɣ-sɯɣ-rom tɕe, tʂha kɤ-nɯ-ta sna, tɕe núndʐa tɕamu rmi. li ci tɯ-tɯphu nɯ, ɕkrɤz kɯ-do tsa ɣɯ ɯ-ci kɯ-ɣɯrni tɤ-se ʑo kɯ-fse pjɯ-nɯɬoʁ tɕe, cischiz tu-ojtɯ tɕe tɤ-rʑaʁ tɤ-rɲɟi tɕe ku-jkrɯt tɕe ɯ-rŋgɤm ɲɯ-βze tɕe ɯ-nɯnɯ wuma ʑo tʂha kɯ-ʑru tɯ-mtshi kɯ-mŋɤm kɯ-phɤn ɲɯ-ŋu khi, tɕeri wuma ʑo kɤ-mto rkɯn. ɕkrɤz ɯ-kɤχcɤl ri kɯ-rko ku-kɯ-ndzoʁ ci tu tɕe nɯ li tʂha ku-nɯ-ta-nɯ pɯ-ŋgrɤl. mɤʑɯ tɯ-tɯphu nɯ sɤjku ɯ-taʁ ku-kɯ-ndzoʁ tɕe, kɯ-rko ci ɲɯ-ŋu tɕe ɯ-nɯnɯ pjɯ́-wɣ-qrɯ tɕe ɲɯ-ɣɯrni tsa tɕe nɯnɯ li tʂha ɯ-rca kú-wɣ-nɯ-ta tɕe tʂha ɯ-mdoʁ ɲɯ-ɣɤmpɕɤr cha.}\hspace{5pt}\pcmn{\lien{ⓔtɕamu}{tɕamu}有三种,其中一个是一种草,叶子有点粗糙,长得不高,花茎比较细长,开白花,花背面是暗红色的,把叶子晒干了以后,可以熬茶喝,所以叫作\lien{ⓔtɕamu}{tɕamu}。还有一种是比较老的青冈树上流出红色像血一样的液体滴在某个地方时,时间一长就会凝结成的坚硬的固体,那是一种优质的茶,据说可以治胃病,但是很罕见。在青冈树上也会长出一种硬东西,以前曾经有人把它当茶喝。还有一种是长在白桦树上的硬东西,打碎了里面是红色的,把它放在马茶里熬使茶水颜色好看。}\end{exemple}\étymologie{dʑa.mo}\end{entrée}

\begin{entrée}{tɕamba}{}{ⓔtɕamba} 
\classe{n} 
\begin{définition}\pfra{une plante}\end{définition}
\begin{définition}\pcmn{【冬寒菜】}\end{définition}
\begin{exemple}\pjya{tɕamba nɯ sɯjno ci ɯ-jwaʁ kɯ-ɤrtɯm tɕe βzɯr kɯ-tu ci ŋu. ɯ-ru kɯ-zri tsa tu-βze cha, ɯ-ru ɯ-pɕi nɯ mɤ-mpɕu, qaɕpa ɯ-mgɯr tsa fse. ɯ-jwaʁ ɯ-sɤɣ-ndzoʁ nɯ tɕu, ɯ-mɯntoʁ ɲɯ-lɤt tɕe, ɯ-mat ɲɯ-βze ŋu. ɯ-mɯntoʁ kɯ-wɣrum ɯ-ŋgɯz kɯ-ɣɯrni ci ŋu, kɤ-ndza sna.}\hspace{5pt}\pcmn{冬寒菜是叶子圆形,有角的植物,茎长得比较高,茎表面不光滑,像青蛙的背一样,在长叶子的部位,开花结果,花是白里透红的。可以吃。}\end{exemple}\end{entrée}

\begin{entrée}{tɕaŋ}{}{ⓔtɕaŋ} 
\classe{n} 
\begin{définition}\pfra{fer}\end{définition}
\begin{définition}\pcmn{马蹄铁}\end{définition}\end{entrée}

\begin{entrée}{tɕaŋnɤ}{}{ⓔtɕaŋnɤ} 
\classe{cnj} 
\begin{définition}\pfra{alors, sinon}\end{définition}
\begin{définition}\pcmn{那就,不然就(提出自己的条件才答应别人的请求)}\end{définition}\end{entrée}

\begin{entrée}{tɕaŋtɣa}{}{ⓔtɕaŋtɣa} 
\classe{n} 
\begin{définition}\pfra{couteau}\end{définition}
\begin{définition}\pcmn{菜刀}\end{définition}\end{entrée}

\begin{entrée}{tɕaʁ}{}{ⓔtɕaʁ} 
\classe{vi} \paradigme{dir}{pɯ-}
\begin{définition}\pfra{être mûr (pêche)}\end{définition}
\begin{définition}\pcmn{成熟(桃子)}\end{définition}
\begin{exemple}\pjya{qaɕti pjɤ-tɕaʁ}\hspace{5pt}\pcmn{桃子成熟了}\end{exemple}\end{entrée}

\begin{entrée}{tɕaʁkɤr khɯtsa}{}{ⓔtɕaʁkɤr khɯtsa} 
\classe{n} 
\begin{définition}\pfra{bol en fer}\end{définition}
\begin{définition}\pcmn{铁碗}\end{définition}\end{entrée}

\begin{entrée}{tɕaʁmɤr}{}{ⓔtɕaʁmɤr} 
\classe{n} 
\begin{définition}\pfra{briquet}\end{définition}
\begin{définition}\pcmn{火镰}\end{définition}\étymologie{ltɕags.dmar}\end{entrée}

\begin{entrée}{tɕaʁtshɯɣ}{}{ⓔtɕaʁtshɯɣ} 
\classe{n} 
\begin{définition}\pfra{brûlure au fer rouge pour soigner les migraines}\end{définition}
\begin{définition}\pcmn{烙(治头风的方式)}\end{définition}
\begin{exemple}\pjya{tɕaʁtshɯɣ ko-ta}\hspace{5pt}\pcmn{他给他烙了印子。}\end{exemple}
\begin{exemple}\pjya{smɤnba kɯ tu-kɯ-nɯsmɤn, tɯ-ku ɯ-taʁ ku-te nɯ tɕaʁtshɯɣ}\hspace{5pt}\pcmn{医生治病的时候,放在头上的那个叫\lien{ⓔtɕaʁtshɯɣ}{tɕaʁtshɯɣ}}\end{exemple}\étymologie{ltɕags.tsʰigs}\end{entrée}

\begin{entrée}{tɕaʁzgroʁ}{}{ⓔtɕaʁzgroʁ} 
\classe{n} 
\begin{définition}\pfra{fers (pieds)}\end{définition}
\begin{définition}\pcmn{脚镣}\end{définition}\étymologie{ltɕags.sgrog}\end{entrée}

\begin{entrée}{tɕaχkɤr}{}{ⓔtɕaχkɤr} 
\classe{n} 
\begin{définition}\pfra{fer blanc}\end{définition}
\begin{définition}\pcmn{白铁皮}\end{définition}\étymologie{ltɕags.dkar}\end{entrée}

\begin{entrée}{tɕaχkhɤβ}{}{ⓔtɕaχkhɤβ} 
\classe{n} 
\begin{définition}\pfra{cheminée}\end{définition}
\begin{définition}\pcmn{火炉}\end{définition}\étymologie{ltɕags.khab}\end{entrée}

\begin{entrée}{tɕaχpa}{}{ⓔtɕaχpa} 
\classe{n} 
\begin{définition}\pfra{bandit}\end{définition}
\begin{définition}\pcmn{强盗}\end{définition}\relationsémantique{参考}{\lien{ⓔnɯtɕaχpa}{nɯtɕaχpa}}\étymologie{dʑag.pa}\end{entrée}

\begin{entrée}{tɕazga}{}{ⓔtɕazga} 
\classe{n} 
\begin{définition}\pfra{gingembre}\end{définition}
\begin{définition}\pcmn{姜}\end{définition}\étymologie{sga.skʲa}\end{entrée}

\begin{entrée}{tɕɤβ}{}{ⓔtɕɤβ} 
\classe{vt} \paradigme{dir}{pɯ-}\paradigme{dir}{lɤ-}\paradigme{dir}{pɯ-}
\begin{définition}\pfra{brûler}\end{définition}
\begin{définition}\pcmn{烧}\end{définition}
\begin{définition}\pfra{défricher par le feu}\end{définition}
\begin{définition}\pcmn{烧荒}\end{définition}
\begin{exemple}\pjya{pa-tɕɤβ}\hspace{5pt}\pcmn{他烧了}\end{exemple}
\begin{exemple}\pjya{jisŋi tɯji ɯ-ŋgɯ ɕ-pɯ-rɤtɕaβ-a}\hspace{5pt}\pcmn{我今天到田地里去烧荒了}\end{exemple}\relationsémantique{参考}{\lien{ⓔɲɟɤβ}{ɲɟɤβ}}
\begin{sous-entrée}{rɤtɕɤβ}{ⓔtɕɤβⓝrɤtɕɤβ} 
\classe{vi}  
\grammaire{apass} \end{sous-entrée}

\end{entrée}

\begin{entrée}{tɕɤkɯ}{}{ⓔtɕɤkɯ} 
\classe{adv} 
\begin{définition}\pfra{à l'est}\end{définition}
\begin{définition}\pcmn{在东边}\end{définition}
\begin{exemple}\pjya{tɕɤkɯ ku-ɕe-a}\hspace{5pt}\pcmn{我往东边去}\end{exemple}
\begin{exemple}\pjya{tɕɤkɯkɯ ʑo ri ku-rɤʑi-a}\hspace{5pt}\pcmn{我在东边}\end{exemple}\relationsémantique{反义词}{\lien{ⓔtɕɤndi}{tɕɤndi}}\relationsémantique{参考}{\lien{ⓔakɯ}{akɯ}}\end{entrée}

\begin{entrée}{tɕɤlo}{}{ⓔtɕɤlo} 
\classe{adv} 
\begin{définition}\pfra{en amont}\end{définition}
\begin{définition}\pcmn{上游}\end{définition}\relationsémantique{参考}{\lien{ⓔalo}{alo}}\end{entrée}

\begin{entrée}{tɕɤmɯ}{}{ⓔtɕɤmɯ} 
\classe{n} 
\begin{définition}\pfra{none}\end{définition}
\begin{définition}\pcmn{尼姑}\end{définition}\étymologie{dʑo.mo}\end{entrée}

\begin{entrée}{tɕɤndi}{}{ⓔtɕɤndi} 
\classe{adv} 
\begin{définition}\pfra{à l'ouest}\end{définition}
\begin{définition}\pcmn{在西边}\end{définition}
\begin{exemple}\pjya{tɕɤndɯndi ʑo ku-rɤʑi-a}\hspace{5pt}\pcmn{我在西边}\end{exemple}\relationsémantique{反义词}{\lien{ⓔtɕɤkɯ}{tɕɤkɯ}}\relationsémantique{参考}{\lien{ⓔandi}{andi}}\end{entrée}

\begin{entrée}{tɕɤphɯ/\variante{tɕhɤphɯ}}{}{ⓔtɕɤphɯ} 
\classe{n}  
\grammaire{n.lieu} 
\begin{définition}\pfra{Japhug}\end{définition}
\begin{définition}\pcmn{茶堡区}\end{définition}
\begin{exemple}\pjya{tɕɤphɯpɯ}\hspace{5pt}\pcmn{茶堡人}\end{exemple}\end{entrée}

\begin{entrée}{tɕɤr}{}{ⓔtɕɤr} 
\classe{vs} \paradigme{dir}{kɤ-}
\begin{définition}\pfra{étroit}\end{définition}
\begin{définition}\pcmn{窄}\end{définition}
\begin{exemple}\pjya{nɯ ɯ-spa ɲɯ-tɕɤr}\hspace{5pt}\pcmn{(布、板子的)材料很窄}\end{exemple}\relationsémantique{反义词}{\lien{ⓔrɟum}{rɟum}}\relationsémantique{参考}{\lien{ⓔarɟumtɕɤr}{arɟumtɕɤr}}\end{entrée}

\begin{entrée}{tɕɤt}{}{ⓔtɕɤt} 
\classe{vt}
\classe{vt} \sens{1}\paradigme{dir}{\_}
\begin{définition}\pfra{retirer de, extraire de}\end{définition}
\begin{définition}\pcmn{取出}\end{définition}
\begin{exemple}\pjya{sɤcɯ nɯ-tɕat-a}\hspace{5pt}\pcmn{我从锁里把钥匙取出来了}\end{exemple}
\begin{exemple}\pjya{qaɟy ɯ-naŋtɕɯ thɯ-tɕat-a}\hspace{5pt}\pcmn{我把鱼的内脏取出来了}\end{exemple}
\begin{exemple}\pjya{ɯ-kri nɯ-tɕat-a}\hspace{5pt}\pcmn{我把猪油熬出来了}\end{exemple}
\begin{exemple}\pjya{tɯ-ɣli thɯ-tɕɤt-i}\hspace{5pt}\pcmn{我们出圈肥了}\end{exemple}
\begin{exemple}\pjya{tɤ-lu pɯ-tɕɤt}\hspace{5pt}\pcmn{你挤奶吧}\end{exemple}
\begin{exemple}\pjya{tɤ-ro tɤ-tɕat-a}\hspace{5pt}\pcmn{我挺起胸膛了}\end{exemple}
\begin{exemple}\pjya{ɯ-qom ra pjɤ-tɕɤt}\hspace{5pt}\pcmn{她掉下了眼泪}\end{exemple}\sens{2}\paradigme{dir}{nɯ-}
\begin{définition}\pfra{enlever (habits)}\end{définition}
\begin{définition}\pcmn{脱(衣服)}\end{définition}
\begin{exemple}\pjya{a-ŋga nɯ-tɕat-a}\hspace{5pt}\pcmn{我脱了衣服}\end{exemple}\sens{3}\paradigme{dir}{tɤ-}
\begin{définition}\pfra{attraper, pêcher}\end{définition}
\begin{définition}\pcmn{捕到}\end{définition}
\begin{exemple}\pjya{qaɟy tɤ-tɕat-a}\hspace{5pt}\pcmn{我钓到鱼了}\end{exemple}\sens{4}\paradigme{dir}{thɯ-}
\begin{définition}\pfra{élever}\end{définition}
\begin{définition}\pcmn{抚养(孩子)}\end{définition}
\begin{exemple}\pjya{a-mu a-wa cho a-pi ra kɯ thɯ́-wɣ-tɕat-a-nɯ ŋu}\hspace{5pt}\pcmn{我父母和姐姐把我抚养成人}\end{exemple}\sens{5}\paradigme{dir}{thɯ-}\paradigme{dir}{pɯ-}
\begin{définition}\pfra{bannir de la maison, sortir de la maison}\end{définition}
\begin{définition}\pcmn{赶出家门}\end{définition}
\begin{exemple}\pjya{ɯʑo taʁndo mɯ-pjɤ-tso tɕe ɯ-ɣi ra kɯ pjɤ́-wɣ-tɕɤt}\hspace{5pt}\pcmn{他不听指挥,家人就把他赶出家门了}\end{exemple}
\begin{exemple}\pjya{tʂapa ɯ-ŋgɯ fsapaʁ ra chɯ́-wɣ-tɕɤt ɲɯ-ra}\hspace{5pt}\pcmn{要把圈里的牲畜赶出来}\end{exemple}
\begin{exemple}\pjya{kha ɯ-pɕi chɤ-tɕɤt}\hspace{5pt}\pcmn{把他赶出家门了}\end{exemple}\sens{6}\paradigme{dir}{nɯ-}
\begin{définition}\pfra{causatif}\end{définition}
\begin{définition}\pcmn{使}\end{définition}
\begin{exemple}\pjya{nɤʑo pjɯ-kɯ-nɯβlu-a mɤ-kɯ-ŋgrɯ ɲɯ-tɕat-a ra}\hspace{5pt}\pcmn{我不会让你骗我的}\end{exemple}
\begin{exemple}\pjya{kɤ-ɤlɯlɤt mɤ-kɯ-ra ɲɤ-tɯ-tɕɤt}\hspace{5pt}\pcmn{你们令他们不要打架}\end{exemple}\relationsémantique{同义词}{\lien{ⓔsɤβzu}{sɤβzu}}\sens{7}\paradigme{dir}{tɤ-}\paradigme{dir}{thɯ-}
\begin{définition}\pfra{publier}\end{définition}
\begin{définition}\pcmn{出版}\end{définition}
\begin{exemple}\pjya{jɯɣi to-tɕɤt (chɤ-tɕɤt)}\hspace{5pt}\pcmn{他出版了一本书}\end{exemple}\sens{8}\paradigme{dir}{tɤ-}
\begin{définition}\pfra{choisir (nom)}\end{définition}
\begin{définition}\pcmn{取(名)}\end{définition}
\begin{exemple}\pjya{ɯ-rmi tɤ-tɕat-a}\hspace{5pt}\pcmn{我给他取了名字}\end{exemple}\sens{9}\paradigme{dir}{thɯ-}
\begin{définition}\pfra{finir}\end{définition}
\begin{définition}\pcmn{结束}\end{définition}
\begin{exemple}\pjya{kɤ-rɤndɯn ɯ-qa chɤ-tɕɤt pjɤ-ra}\hspace{5pt}\pcmn{他把经念到最后}\end{exemple}
\begin{exemple}\pjya{a-ɣe kɤ-ndo nɯ, ɯ-ndo ʑo thɯ-tɕat-a pɯ-ra}\hspace{5pt}\pcmn{我把孙子养到他长大成人}\end{exemple}
\begin{exemple}\pjya{kɤ-nɤma ɯ-jme chɯ-tɕat-a ra}\hspace{5pt}\pcmn{我要把事情做到底}\end{exemple}\relationsémantique{参考}{\lien{ⓔrŋamaⓝrŋama,tɕɤt}{rŋama,tɕɤt}}\relationsémantique{同义词}{\lien{ⓔsɯɣjɤɣ}{sɯɣjɤɣ}}\relationsémantique{Component 2}{\lien{ⓔtɕɤt}{tɕɤt}}
\begin{sous-entrée}{sɯtɕɤt}{ⓔtɕɤtⓢ9ⓝsɯtɕɤt} 
\classe{vt}  
\grammaire{caus} \end{sous-entrée}

\begin{sous-entrée}{ʑɣɤsɯtɕɤt}{ⓔtɕɤtⓢ9ⓝʑɣɤsɯtɕɤt} 
\classe{vs}  
\grammaire{caus}
\grammaire{refl} \end{sous-entrée}

\begin{sous-entrée}{ɯ-βlu,tɕɤt}{ⓔtɕɤtⓢ9ⓝɯ-βlu,tɕɤt} 
\classe{np} 
\begin{définition}\pfra{trouver une solution}\end{définition}
\begin{définition}\pcmn{想办法,出主意}\end{définition}
\begin{exemple}\pjya{nɤ-βlu ci tɤ-tɕɤt}\hspace{5pt}\pcmn{你想一下办法}\end{exemple}\relationsémantique{Component 1}{\lien{ⓔɯ-βlu}{ɯ-βlu}}\end{sous-entrée}

\begin{sous-entrée}{ɯ-βlu,sɯtɕɤt}{ⓔtɕɤtⓝɯ-βlu,sɯtɕɤt} 
\classe{vt}
\classe{np}
\classe{vt}  
\grammaire{habil} \paradigme{dir}{tɤ-}
\begin{définition}\pfra{parvenir à trouver une solution}\end{définition}
\begin{définition}\pcmn{想得出办法}\end{définition}
\begin{exemple}\pjya{a-βlu mɯ́j-sɯtɕat-a}\hspace{5pt}\pcmn{我想不出办法来,我犹豫不决}\end{exemple}
\begin{exemple}\pjya{ki kɤ-nɤma tɕhi tu-ste-a ra ma a-βlu mɯ́j-sɯtɕat-a}\hspace{5pt}\pcmn{我想不出怎么做这个工作}\end{exemple}
\begin{exemple}\pjya{a-βlu tɤ-sɯtɕa-t-a}\hspace{5pt}\pcmn{我想出了办法}\end{exemple}
\begin{exemple}\pjya{a-βlu ɲɯ-sɯtɕa-t-a}\hspace{5pt}\pcmn{我想得出办法}\end{exemple}
\begin{exemple}\pjya{nɤ-βlu ɯ-ɲɯ-tɯ-sɯtɕɤt}\hspace{5pt}\pcmn{你想得出办法吗?}\end{exemple}\relationsémantique{Component 1}{\lien{ⓔɯ-βlu}{ɯ-βlu}}\relationsémantique{Component 2}{\lien{ⓔsɯtɕɤt}{sɯtɕɤt}}\end{sous-entrée}

\end{entrée}

\begin{entrée}{tɕɤthi}{}{ⓔtɕɤthi} 
\classe{adv} 
\begin{définition}\pfra{en aval}\end{définition}
\begin{définition}\pcmn{在下游}\end{définition}\relationsémantique{参考}{\lien{ⓔathi}{athi}}\end{entrée}

\begin{entrée}{tɕɤtsaʁ}{}{ⓔtɕɤtsaʁ} 
\classe{n} 
\begin{définition}\pfra{petites tresses (coiffure de femme)}\end{définition}
\begin{définition}\pcmn{女人在额头上辫的小辫子}\end{définition}\end{entrée}

\begin{entrée}{tɕe}{}{ⓔtɕe} 
\classe{cnj} 
\begin{définition}\pfra{ensuite}\end{définition}
\begin{définition}\pcmn{然后}\end{définition}\end{entrée}

\begin{entrée}{tɕendɤre}{}{ⓔtɕendɤre} 
\classe{cnj} 
\begin{définition}\pfra{ensuite}\end{définition}
\begin{définition}\pcmn{然后}\end{définition}\end{entrée}

\begin{entrée}{tɕeri}{}{ⓔtɕeri} 
\classe{cnj} 
\begin{définition}\pfra{mais}\end{définition}
\begin{définition}\pcmn{但是}\end{définition}\end{entrée}

\begin{entrée}{tɕetha}{}{ⓔtɕetha} 
\classe{n} 
\begin{définition}\pfra{(action de) sonder les gens}\end{définition}
\begin{définition}\pcmn{试探人}\end{définition}
\begin{exemple}\pjya{tɕetha kɤ-βzu mɤ-mbat}\hspace{5pt}\pcmn{试探人是不容易的}\end{exemple}\relationsémantique{参考}{\lien{ⓔnɯtɕetha}{nɯtɕetha}}\end{entrée}

\begin{entrée}{tɕɣaʁ}{}{ⓔtɕɣaʁ} 
\classe{vt} \paradigme{dir}{tɤ-}\paradigme{dir}{nɯ-}
\begin{définition}\pfra{extraire en pressant sur}\end{définition}
\begin{définition}\pcmn{挤出来}\end{définition}
\begin{exemple}\pjya{nɤ-jaʁ to-rɤspɯ, tɤ-tɕɣaʁ}\hspace{5pt}\pcmn{你手上化脓,你挤一下}\end{exemple}
\begin{exemple}\pjya{tɯ-ɣmbɤβ tɤ-tɕɣaʁ-a}\hspace{5pt}\pcmn{我挤了脓肿}\end{exemple}
\begin{exemple}\pjya{sɯmat nɯ-tɕɣaʁ-a}\hspace{5pt}\pcmn{我把果子挤出来了}\end{exemple}\relationsémantique{参考}{\lien{ⓔndʑɣaʁ}{ndʑɣaʁ}}\end{entrée}

\begin{entrée}{tɕɣɤrtɕɣɤr}{}{ⓔtɕɣɤrtɕɣɤr} 
\classe{idph.2} 
\begin{définition}\pfra{rouge vif}\end{définition}
\begin{définition}\pcmn{形容(颜色)红艳艳}\end{définition}
\begin{exemple}\pjya{ɯ-ŋga ɲɯ-ɣɯrni tɕɣɤrtɕɣɤr ʑo}\hspace{5pt}\pcmn{衣服颜色非常红}\end{exemple}
\begin{sous-entrée}{tɕɣɤrnɤtɕɣɤr}{ⓔtɕɣɤrtɕɣɤrⓝtɕɣɤrnɤtɕɣɤr} 
\classe{idph.3} 
\begin{exemple}\pjya{tɤrmbja tɕɣɤrnɤtɕɣɤr ɲɯ-ɤsɯ-βzu}\hspace{5pt}\pcmn{形容闪电一道一道地闪过发光}\end{exemple}\end{sous-entrée}

\begin{sous-entrée}{ɣɤtɕɣɤrtɕɣɤr}{ⓔtɕɣɤrtɕɣɤrⓝɣɤtɕɣɤrtɕɣɤr} 
\classe{vi} \sens{1}
\begin{définition}\pfra{hurler}\end{définition}
\begin{définition}\pcmn{叫唤}\end{définition}\end{sous-entrée}

\sens{2}
\begin{définition}\pfra{scintiller}\end{définition}
\begin{définition}\pcmn{一闪一闪地发光}\end{définition}
\begin{exemple}\pjya{paʁ ɲɯ-ɣɤwu ɲɯ-ɣɤtɕɣɤrtɕɣɤr}\hspace{5pt}\pcmn{猪在叫唤(要宰猪的时候)}\end{exemple}
\begin{exemple}\pjya{tɤrmbja ɲɯ-ɣɤtɕɣɤrtɕɣɤr (ʑo ɲɯ-ɤsɯ-βzu)}\hspace{5pt}\pcmn{一道道电光闪过。}\end{exemple}
\begin{sous-entrée}{sɤtɕɣɤrtɕɣɤr}{ⓔtɕɣɤrtɕɣɤrⓢ2ⓝsɤtɕɣɤrtɕɣɤr} 
\classe{vt} \end{sous-entrée}

\end{entrée}

\begin{entrée}{tɕɣɤtɕɣɤt}{}{ⓔtɕɣɤtɕɣɤt} 
\classe{idph.2} 
\begin{définition}\pfra{qui retient ses larmes}\end{définition}
\begin{définition}\pcmn{形容(眼泪)含在眼眶里,快要流出来的样子}\end{définition}
\begin{exemple}\pjya{ɯ-qom tɕɣɤtɕɣɤt ʑo to-stu}\hspace{5pt}\pcmn{他的眼泪含在眼眶里}\end{exemple}\end{entrée}

\begin{entrée}{tɕɣom}{}{ⓔtɕɣom} 
\classe{n} 
\begin{définition}\pfra{xanthoxyle}\end{définition}
\begin{définition}\pcmn{花椒}\end{définition}
\begin{exemple}\pjya{tɕɣom nɯ si mɤ-kɯ-mbro ci ŋu, ɯ-mdzu tu, ɯ-ru ɯ-taʁ ɯ-mdzu rkɯn, ɯ-rtaʁ ɯ-taʁ ɯ-mdzu dɤn, tɕɣom nɯ sɯŋgɯ tɕɣom ci tu, kha ɯ-rkɯ pɯ-kɤ-nɯ-ji tɕɣom ci tu, pɯ-kɤ-nɯji tɕɣom nɯ tɕɣomte rmi, sɯŋgɯ tɕɣom nɯ tɕɣomzaʁ rmi, tɕɣomzaʁ ɣɯ ɯ-mat ndɯβ, tɕɣomte ɣɯ ɯ-mat jndʐɤz, thɯ-tɯt tɕe, chɯ-ɣɯrni ŋu, ɯ-mat ɯ-pɕi zɯ ɯ-kri tu, tɕe kú-wɣ-rtoʁ tɕe nɤmbju ʑo. ɲɯ́-wɣ-phɯt tɕe, ɯ-kri nɯ tɯ-mɲaʁ ɯ-ŋgɯ ku-ɕe tɕe toʁde wuma ʑo ɕɯmŋɤm. tú-wɣ-ndza tɕe mɤrtsaβ tɕe ɣɤzɯβzɯβ ʑo, ɯ-dɯχɯn mɯm. ndzɤtshi ɯ-ŋgɯ kú-wɣ-nɯ-lɤt tɕe mɯm.}\hspace{5pt}\pcmn{花椒是一种矮小的树木,有刺,树干上刺不多,枝桠上刺多,有野生花椒,也有家边自己种的。自己种的叫\lien{ⓔtɕɣomte}{tɕɣomte}(真花椒),野生的叫\lien{ⓔtɕɣomzaʁ}{tɕɣomzaʁ}。野花椒的果实小一些,家花椒的果实大一些。成熟了就变红,果实外面有一层油,看起来有光泽。摘花椒的时候,油进了眼睛,会痛一阵子。吃起来是麻的,很香。放在食物里很香。}\end{exemple}\relationsémantique{参考}{\lien{ⓔtɕɣomte}{tɕɣomte}}\relationsémantique{参考}{\lien{ⓔtɕɣomzaʁ}{tɕɣomzaʁ}}\relationsémantique{参考}{\lien{ⓔnɯtɕɣom}{nɯtɕɣom}}\end{entrée}

\begin{entrée}{tɕɣomte}{}{ⓔtɕɣomte} 
\classe{n} 
\begin{définition}\pfra{xanthoxyle cultivé}\end{définition}
\begin{définition}\pcmn{家花椒}\end{définition}\end{entrée}

\begin{entrée}{tɕɣomzaʁ}{}{ⓔtɕɣomzaʁ} 
\classe{n} 
\begin{définition}\pfra{xanthoxyle sauvage}\end{définition}
\begin{définition}\pcmn{野花椒}\end{définition}\end{entrée}

\begin{entrée}{tɕhaŋβɟaj}{}{ⓔtɕhaŋβɟaj} 
\classe{n} 
\begin{définition}\pfra{outil pour brasser la bière d'orge}\end{définition}
\begin{définition}\pcmn{搅拌青稞酒的工具}\end{définition}\end{entrée}

\begin{entrée}{tɕhaŋkha}{}{ⓔtɕhaŋkha} 
\classe{n} 
\begin{définition}\pfra{entonnoir}\end{définition}
\begin{définition}\pcmn{漏斗}\end{définition}
\begin{exemple}\pjya{cha nɯ tɕhaŋkha kɯ tɤ-sɯ-rku-t-a}\hspace{5pt}\pcmn{我用漏斗倒了酒进去}\end{exemple}
\begin{exemple}\pjya{tɕhaŋkha a-pɯ-tu tɕe, cha kɤ-rku tɤ-mda tɕe mɤ-jit}\hspace{5pt}\pcmn{有漏斗的话,倒酒就不会洒}\end{exemple}\end{entrée}

\begin{entrée}{tɕhaʁ}{}{ⓔtɕhaʁ} 
\classe{vi} \paradigme{dir}{tɤ-}\paradigme{dir}{nɯ-}\sens{1}
\begin{définition}\pfra{diminuer}\end{définition}
\begin{définition}\pcmn{减少}\end{définition}
\begin{exemple}\pjya{tɯ-rdoʁ ɲɤ-tɕhaʁ}\hspace{5pt}\pcmn{少了一个}\end{exemple}\sens{2}
\begin{définition}\pfra{supporter, pouvoir rester en place}\end{définition}
\begin{définition}\pcmn{稳得住;坐得住‘忍得住}\end{définition}
\begin{exemple}\pjya{mɯ́j-tɕhaʁ-a}\hspace{5pt}\pcmn{我坐不住}\end{exemple}
\begin{exemple}\pjya{tɤ-pɤtso kɤ-rɤʑi mɯ́j-tɕhaʁ}\hspace{5pt}\pcmn{小孩子坐不住}\end{exemple}
\begin{exemple}\pjya{ɯ-tɯ-rga kɯ mɯ-pjɤ-tɕhaʁ}\hspace{5pt}\pcmn{她高兴得忍不住(说漏了)}\end{exemple}\relationsémantique{参考}{\lien{ⓔɯ-tɕhaʁⓗ2}{ɯ-tɕhaʁ₂}}\relationsémantique{参考}{\lien{ⓔsɯxtɕhaʁ}{sɯxtɕhaʁ}}\relationsémantique{参考}{\lien{ⓔɣɤtɕhaʁ}{ɣɤtɕhaʁ}}\relationsémantique{参考}{\lien{ⓔzndɤtɕhaʁ}{zndɤtɕhaʁ}}\étymologie{tɕʰag}\end{entrée}

\begin{entrée}{tɕhaʁla}{}{ⓔtɕhaʁla} 
\classe{n} 
\begin{définition}\pfra{cour}\end{définition}
\begin{définition}\pcmn{院子}\end{définition}\relationsémantique{同义词}{\lien{ⓔrɟara}{rɟara}}\end{entrée}

\begin{entrée}{tɕhaʁra}{}{ⓔtɕhaʁra} 
\classe{n} 
\begin{définition}\pfra{toilettes}\end{définition}
\begin{définition}\pcmn{厕所}\end{définition}\end{entrée}

\begin{entrée}{tɕhaχɕaŋ}{}{ⓔtɕhaχɕaŋ} 
\classe{n} 
\begin{définition}\pfra{attelle}\end{définition}
\begin{définition}\pcmn{(骨折)夹板}\end{définition}\end{entrée}

\begin{entrée}{tɕhaχpu}{}{ⓔtɕhaχpu} 
\classe{n} 
\begin{définition}\pfra{handicapé}\end{définition}
\begin{définition}\pcmn{残疾人}\end{définition}\relationsémantique{参考}{\lien{ⓔɯ-tɕhaʁⓗ2}{ɯ-tɕhaʁ₂}}\end{entrée}

\begin{entrée}{tɕhaχɯ}{}{ⓔtɕhaχɯ} 
\classe{n} 
\begin{définition}\pfra{théière}\end{définition}
\begin{définition}\pcmn{茶壶}\end{définition}\étymologie{fn:茶壶}\end{entrée}

\begin{entrée}{tɕhɤfɕɤt}{}{ⓔtɕhɤfɕɤt} 
\classe{n} 
\begin{définition}\pfra{débats philolophiques sur le bouddhisme}\end{définition}
\begin{définition}\pcmn{辩经}\end{définition}
\begin{exemple}\pjya{χpɯn ra tɕhɤfɕɤt ɲɯ-ɤsɯ-βzu-nɯ (=ɲɯ-rɯtɕhɤfɕɤt-nɯ)}\hspace{5pt}\pcmn{和尚们在辩经}\end{exemple}\relationsémantique{参考}{\lien{ⓔrɯtɕhɤfɕɤt}{rɯtɕhɤfɕɤt}}\relationsémantique{参考}{\lien{ⓔfɕɤtⓗ1}{fɕɤt₁}}\end{entrée}

\begin{entrée}{tɕhɤɣdɯ}{}{ⓔtɕhɤɣdɯ} 
\classe{n} 
\begin{définition}\pfra{jarre de vin}\end{définition}
\begin{définition}\pcmn{酒坛子}\end{définition}\relationsémantique{同义词}{\lien{ⓔtɕhorzi}{tɕhorzi}}
\begin{sous-entrée}{tɯ-tɕhɤɣdɯ}{ⓔtɕhɤɣdɯⓝtɯ-tɕhɤɣdɯ} 
\classe{clf} \end{sous-entrée}

\end{entrée}

\begin{entrée}{tɕhɤjlɯz}{}{ⓔtɕhɤjlɯz} 
\classe{n} 
\begin{définition}\pfra{coutume}\end{définition}
\begin{définition}\pcmn{风俗}\end{définition}
\begin{exemple}\pjya{aʑo kɯɕɯŋgɯ ɣɯ ɯ-tɕhɤjlɯz nɯ kɤ-βzu rga-a}\hspace{5pt}\pcmn{我喜欢根据古代的风俗穿衣服打扮}\end{exemple}\relationsémantique{参考}{\lien{ⓔnɯtɕhɤjlɯz}{nɯtɕhɤjlɯz}}\étymologie{tɕʰos.lugs}\end{entrée}

\begin{entrée}{tɕhɤjʁo}{}{ⓔtɕhɤjʁo} 
\classe{n} 
\begin{définition}\pfra{état d'ébriété}\end{définition}
\begin{définition}\pcmn{发酒疯}\end{définition}\relationsémantique{参考}{\lien{ⓔnɯtɕhɤjʁo}{nɯtɕhɤjʁo}}\end{entrée}

\begin{entrée}{tɕhɤmkɯm}{}{ⓔtɕhɤmkɯm} 
\classe{n} 
\begin{définition}\pfra{jarre}\end{définition}
\begin{définition}\pcmn{坛子}\end{définition}\end{entrée}

\begin{entrée}{tɕhɤmlaŋ}{}{ⓔtɕhɤmlaŋ} 
\classe{n} 
\begin{définition}\pfra{gobelet}\end{définition}
\begin{définition}\pcmn{缸子}\end{définition}\end{entrée}

\begin{entrée}{tɕhɤphɯ}{}{ⓔtɕhɤphɯ} 
\classe{n} 
\begin{définition}\pfra{prix à payer, avantage}\end{définition}
\begin{définition}\pcmn{代价;收获}\end{définition}
\begin{exemple}\pjya{ɯ-tɕhɤphɯ me}\hspace{5pt}\pcmn{不值得}\end{exemple}\relationsémantique{参考}{\lien{ⓔɯ-phɯ}{ɯ-phɯ}}\end{entrée}

\begin{entrée}{tɕhɤrɕɤt}{}{ⓔtɕhɤrɕɤt} 
\classe{n} 
\begin{définition}\pfra{pluie torrentielle avec des grêlons}\end{définition}
\begin{définition}\pcmn{夹着小冰雹的猛雨}\end{définition}\end{entrée}

\begin{entrée}{tɕhɤrnaʁ}{}{ⓔtɕhɤrnaʁ} 
\classe{n} 
\begin{définition}\pfra{pluie}\end{définition}
\begin{définition}\pcmn{雨}\end{définition}
\begin{exemple}\pjya{tɕhɤrnaʁ ɲɯ-ɤsɯ-lɤt}\hspace{5pt}\pcmn{下雨}\end{exemple}\étymologie{tɕʰar.nag}\end{entrée}

\begin{entrée}{tɕhɤrprɯ}{}{ⓔtɕhɤrprɯ} 
\classe{n} 
\begin{définition}\pfra{abri de pluie}\end{définition}
\begin{définition}\pcmn{遮雨的地方}\end{définition}\end{entrée}

\begin{entrée}{tɕhɤrʁu}{}{ⓔtɕhɤrʁu} 
\classe{n} 
\begin{définition}\pfra{habit de femme en laine}\end{définition}
\begin{définition}\pcmn{女式藏装,毛织品}\end{définition}\end{entrée}

\begin{entrée}{tɕhɤrʁɯ}{}{ⓔtɕhɤrʁɯ} 
\classe{n} 
\begin{définition}\pfra{habit de femme en tissu}\end{définition}
\begin{définition}\pcmn{氆氇、羊毛、牛毛制成的女装}\end{définition}\end{entrée}

\begin{entrée}{tɕhɤrzɤthɯm}{}{ⓔtɕhɤrzɤthɯm} 
\classe{n} 
\begin{définition}\pfra{bouchon au fond des jarres d'alcool}\end{définition}
\begin{définition}\pcmn{酒缸底部的塞子}\end{définition}\relationsémantique{参考}{\lien{ⓔtɕhorzi}{tɕhorzi}}\relationsémantique{参考}{\lien{ⓔɯ-thɯm}{ɯ-thɯm}}\end{entrée}

\begin{entrée}{tɕhɤska}{}{ⓔtɕhɤska} 
\classe{n} 
\begin{définition}\pfra{clarinette}\end{définition}
\begin{définition}\pcmn{唢呐}\end{définition}\end{entrée}

\begin{entrée}{tɕhɤt}{₂}{ⓔtɕhɤtⓗ2} 
\classe{n} 
\begin{définition}\pfra{rendez-vous}\end{définition}
\begin{définition}\pcmn{约会}\end{définition}
\begin{exemple}\pjya{tɕhɤt nɯ-βzu-tɕi}\hspace{5pt}\pcmn{我们俩约了时间}\end{exemple}\end{entrée}

\begin{entrée}{tɕhɤt}{₁}{ⓔtɕhɤtⓗ1} 
\classe{vi} \sens{1}\paradigme{dir}{nɯ-}\paradigme{dir}{lɤ-}
\begin{définition}\pfra{s'effondrer de fatigue}\end{définition}
\begin{définition}\pcmn{累倒}\end{définition}
\begin{exemple}\pjya{ɲɤ-tɕhat-a}\hspace{5pt}\pcmn{我累倒了}\end{exemple}
\begin{exemple}\pjya{mbro lo-tɕhɤt}\hspace{5pt}\pcmn{马走不动了}\end{exemple}\sens{2}\paradigme{dir}{nɯ-}
\begin{définition}\pfra{disparaître (culture)}\end{définition}
\begin{définition}\pcmn{消失(庄稼)}\end{définition}
\begin{définition}\pfra{supprimer}\end{définition}
\begin{définition}\pcmn{消除}\end{définition}
\begin{exemple}\pjya{tɯpɕi jinde mɯ-la-ji-nɯ tɕe, ɲɤ-tɕhɤt}\hspace{5pt}\pcmn{现在不种亚麻了,连种子都没有}\end{exemple}
\begin{exemple}\pjya{ji-@baicai pɯ-dɤn ri, ɲɤ-sɯxtɕhɤt-nɯ}\hspace{5pt}\pcmn{以前我们的白菜很多,但他们把它们拔除了}\end{exemple}\relationsémantique{同义词}{\lien{ⓔɲat}{ɲat}}
\begin{sous-entrée}{sɯxtɕhɤt}{ⓔtɕhɤtⓗ1ⓢ2ⓝsɯxtɕhɤt} 
\classe{vt} \end{sous-entrée}

\étymologie{tɕʰad}\end{entrée}

\begin{entrée}{tɕhɤtpa}{}{ⓔtɕhɤtpa} 
\classe{n} 
\begin{définition}\pfra{punition}\end{définition}
\begin{définition}\pcmn{惩罚}\end{définition}
\begin{exemple}\pjya{nɤ-tɕhɤtpa βze (=tɯ́-wɣ-znɯtɕhɤl)}\hspace{5pt}\pcmn{他会惩罚你的}\end{exemple}\relationsémantique{参考}{\lien{ⓔznɯtɕhɤtpa}{znɯtɕhɤtpa}}\relationsémantique{同义词}{\lien{ⓔɯ-tɕhɤl}{ɯ-tɕhɤl}}\étymologie{tɕʰad.pa}\end{entrée}

\begin{entrée}{tɕhɤz}{}{ⓔtɕhɤz} 
\classe{vs} 
\begin{définition}\pfra{contraire au bouddhisme}\end{définition}
\begin{définition}\pcmn{违背佛教}\end{définition}
\begin{exemple}\pjya{mɤ-kɯ-tɕhɤz me}\hspace{5pt}\pcmn{不违背佛教}\end{exemple}
\begin{exemple}\pjya{sroχtɕɤn kɤ-lɤt mɤ-tɕhɤz}\hspace{5pt}\pcmn{杀生是违背佛教的行为}\end{exemple}\end{entrée}

\begin{entrée}{tɕhemɤli}{}{ⓔtɕhemɤli} 
\classe{n} 
\begin{définition}\pfra{jeune fille}\end{définition}
\begin{définition}\pcmn{姑娘}\end{définition}\end{entrée}

\begin{entrée}{tɕhemɤpɯ}{}{ⓔtɕhemɤpɯ} 
\classe{n} 
\begin{définition}\pfra{jeune fille}\end{définition}
\begin{définition}\pcmn{姑娘}\end{définition}\relationsémantique{参考}{\lien{ⓔtɕheme}{tɕheme}}\end{entrée}

\begin{entrée}{tɕheme}{}{ⓔtɕheme} 
\classe{n} 
\begin{définition}\pfra{fille}\end{définition}
\begin{définition}\pcmn{妇女}\end{définition}\end{entrée}

\begin{entrée}{tɕhemekɤtsa}{}{ⓔtɕhemekɤtsa} 
\classe{n} 
\begin{définition}\pfra{une plante}\end{définition}
\begin{définition}\pcmn{植物的一种}\end{définition}
\begin{exemple}\pjya{tɕhemekɤtsa nɯ sɯjno kɯ-mbɯ-mbɤr ci ŋu. ɯ-ru kɯ-ɤβʑɯrdu fse, ɯ-mdoʁ ɣɯrni. ɯ-βzɯr kɯ-fse kɯ-tu ŋu. ɯ-ru tu-ɬoʁ tɕe, ɯ-jwaʁ nɯ ɯ-ru ɯ-taʁ tɯ-rtsɤɣ tɯ-rtsɤɣ ku-fskɤr tɕe ku-ndzoʁ ŋu. ɯ-jwaʁ ɯ-rchɤβ nɯ tɕu ɯ-mɯntoʁ ku-oʑɯrja ŋu, tɕe ɯ-mɯntoʁ ku-kɤ-fskɤr nɯ ndʐa tɕhemekɤtsa ɲɯ-rmi. ɯ-ru tɯ-ldʑa ɯ-taʁ nɯ ɯ-jwaʁ χsɯ-tɤxɯr kɯβde-tɤxɯr jamar ma me, ɯ-jwaʁ nɯ kɯ-ɤɲaʁndzɯm ŋu.}\hspace{5pt}\pcmn{\lien{}{tɕheme kɤtsa}是矮小的植物。茎四方形、红色、有棱角。茎长出来时,叶子就在茎上一节绕一圈地生长。花排列在叶子之间。因为花是绕着茎而长的,所以叫它\lien{}{tɕheme kɤ-tsa}(母女难分的意思)。一根茎上只有三四圈叶子,叶子是深绿色的。}\end{exemple}\end{entrée}

\begin{entrée}{tɕhɣaʁtɕhɣaʁ}{}{ⓔtɕhɣaʁtɕhɣaʁ} 
\classe{idph.2} 
\begin{définition}\pfra{propre, sans dommage}\end{définition}
\begin{définition}\pcmn{形容完整、干净的样子,没有受到损伤}\end{définition}
\begin{exemple}\pjya{tɕhɣaʁtɕhɣaʁ ɲɤ-χtɕi}\hspace{5pt}\pcmn{他洗得很干净了}\end{exemple}
\begin{exemple}\pjya{si tɕhɣaʁtɕhɣaʁ lo-ftɕhur}\hspace{5pt}\pcmn{(原来倒下来的)柴全部立起来了}\end{exemple}\end{entrée}

\begin{entrée}{tɕhi}{₁}{ⓔtɕhiⓗ1} 
\classe{pro} 
\begin{définition}\pfra{quoi}\end{définition}
\begin{définition}\pcmn{什么}\end{définition}
\begin{exemple}\pjya{kutɕu tɕhi ɯ-kɯ-pa jɤ-tɯ-ɣe?}\hspace{5pt}\pcmn{你来这里做什么?}\end{exemple}
\begin{exemple}\pjya{tɕhi kɯ-ra ʑo kɤ-nɤma cha}\hspace{5pt}\pcmn{她什么都能做}\end{exemple}
\begin{exemple}\pjya{tɕhi ɯ-skɤt ŋu, ɯ-ɲɯ́-tɯ-tso?}\hspace{5pt}\pcmn{你知道什么意思吗?}\end{exemple}\étymologie{tɕʰi}\end{entrée}

\begin{entrée}{tɕhi}{₂}{ⓔtɕhiⓗ2} 
\classe{n} 
\begin{définition}\pfra{escalier taillé dans un tronc}\end{définition}
\begin{définition}\pcmn{用树干刻成的楼梯}\end{définition}\end{entrée}

\begin{entrée}{tɕhindʐa}{}{ⓔtɕhindʐa} 
\classe{pro} 
\begin{définition}\pfra{pourquoi}\end{définition}
\begin{définition}\pcmn{为什么}\end{définition}\étymologie{tɕʰi.ⁿdra}\end{entrée}

\begin{entrée}{tɕhinthaʁ}{}{ⓔtɕhinthaʁ} 
\classe{n} 
\begin{définition}\pfra{corde de tente}\end{définition}
\begin{définition}\pcmn{用来搭帐篷的拉线}\end{définition}\end{entrée}

\begin{entrée}{tɕhitsuku}{}{ⓔtɕhitsuku} 
\classe{pro} 
\begin{définition}\pfra{quoi que ce soit}\end{définition}
\begin{définition}\pcmn{无论什么,一切}\end{définition}\end{entrée}

\begin{entrée}{tɕhom}{}{ⓔtɕhom} 
\classe{vs} \paradigme{dir}{tɤ-}\paradigme{dir}{nɯ-}
\begin{définition}\pfra{trop, excessif}\end{définition}
\begin{définition}\pcmn{过分;太}\end{définition}
\begin{exemple}\pjya{ɯ-tɯ-xtɕi ɲo-tɕhom}\hspace{5pt}\pcmn{变得太小}\end{exemple}
\begin{exemple}\pjya{a-tɯ-ɤɕqhe ɲɯ-tɕhom}\hspace{5pt}\pcmn{我咳得太多}\end{exemple}\end{entrée}

\begin{entrée}{tɕhoma}{}{ⓔtɕhoma} 
\classe{n} 
\begin{définition}\pfra{ceinture}\end{définition}
\begin{définition}\pcmn{皮带}\end{définition}
\begin{exemple}\pjya{tɕhoma ɯ-mdʑu}\hspace{5pt}\pcmn{腰带口子里面的钉子}\end{exemple}\end{entrée}

\begin{entrée}{tɕhomba}{}{ⓔtɕhomba} 
\classe{n} 
\begin{définition}\pfra{rhume}\end{définition}
\begin{définition}\pcmn{感冒}\end{définition}\étymologie{tɕʰam.pa}\end{entrée}

\begin{entrée}{tɕhorzi}{}{ⓔtɕhorzi} 
\classe{n} 
\begin{définition}\pfra{jarre}\end{définition}
\begin{définition}\pcmn{坛子}\end{définition}\end{entrée}

\begin{entrée}{tɕhoʁtɕhoʁ/\variante{ftɕhoʁftɕhoʁ}}{}{ⓔtɕhoʁtɕhoʁ} 
\classe{idph.2} 
\begin{définition}\pfra{les deux oreilles dressées}\end{définition}
\begin{définition}\pcmn{形容两只耳朵立起来,看起来很灵活的样子}\end{définition}
\begin{exemple}\pjya{qala ɣɯ ɯ-rna nɯ tu-z-nɯndzi tɕhoʁtɕhoʁ ʑo}\hspace{5pt}\pcmn{兔子把两只耳朵立起来了}\end{exemple}
\begin{exemple}\pjya{sɯjno to-ɬoʁ tɕe tɕhoʁtɕhoʁ ʑo to-pa}\hspace{5pt}\pcmn{草刚出土,直直地竖起来}\end{exemple}\end{entrée}

\begin{entrée}{tɕhoz}{}{ⓔtɕhoz} 
\classe{n} 
\begin{définition}\pfra{religion}\end{définition}
\begin{définition}\pcmn{佛学}\end{définition}\étymologie{tɕʰos}\end{entrée}

\begin{entrée}{tɕhʁɯβnɤtɕhʁɯβ}{}{ⓔtɕhʁɯβnɤtɕhʁɯβ} 
\classe{idph.3} \paradigme{dir}{pɯ-}
\begin{définition}\pfra{croquant}\end{définition}
\begin{définition}\pcmn{形容食物脆}\end{définition}
\begin{définition}\pfra{croquer}\end{définition}
\begin{définition}\pcmn{使劲地咀嚼(脆的食物)}\end{définition}\relationsémantique{反义词}{\lien{ⓔtɕʁɯβnɤtɕʁɯβ}{tɕʁɯβnɤtɕʁɯβ}}
\begin{sous-entrée}{nɯtɕhʁɯβ}{ⓔtɕhʁɯβnɤtɕhʁɯβⓝnɯtɕhʁɯβ} 
\classe{vt} \end{sous-entrée}

\end{entrée}

\begin{entrée}{tɕhɯ}{}{ⓔtɕhɯ} 
\classe{vt} \paradigme{dir}{tɤ-}\paradigme{dir}{tɤ-}
\begin{définition}\pfra{encorner}\end{définition}
\begin{définition}\pcmn{用角顶(牛)}\end{définition}
\begin{définition}\pfra{s'encorner les uns les autres}\end{définition}
\begin{définition}\pcmn{互相顶}\end{définition}
\begin{exemple}\pjya{mbala kɯ tɤ́-wɣ-tɕhɯ-a}\hspace{5pt}\pcmn{牛用角顶了我}\end{exemple}
\begin{exemple}\pjya{zdoŋbu kɯ tɯ-mɲaʁ mɤ-tɕhi, xɕɤndʑu kɯ tɯ-mɲaʁ tɕhi}\hspace{5pt}\pcmn{大事影响不了人,小事会伤害到人}\end{exemple}
\begin{exemple}\pjya{mbala nɯ ɲɯ-ɤtɕhɯtɕhɯ-ndʑi}\hspace{5pt}\pcmn{公牛互相顶着}\end{exemple}\relationsémantique{参考}{\lien{ⓔsɤtɕhɯ}{sɤtɕhɯ}}\relationsémantique{参考}{\lien{ⓔnɤkɤtɕhɯ}{nɤkɤtɕhɯ}}
\begin{sous-entrée}{atɕhɯtɕhɯ}{ⓔtɕhɯⓝatɕhɯtɕhɯ} 
\classe{vi} \end{sous-entrée}

\begin{sous-entrée}{ɯ-tɕhɯ,lɤt}{ⓔtɕhɯⓝɯ-tɕhɯ,lɤt}
\begin{définition}\pfra{donner un coup de couteau, de baïonnette}\end{définition}
\begin{définition}\pcmn{用……戳}\end{définition}
\begin{exemple}\pjya{ɕɤmɯɣdɯ ɯ-tɕhɯ to-lɤt}\hspace{5pt}\pcmn{用刺刀刺伤了他}\end{exemple}\relationsémantique{参考}{\lien{ⓔnɯtɯtɕhɯ}{nɯtɯtɕhɯ}}\end{sous-entrée}

\end{entrée}

\begin{entrée}{tɕhɯβja}{}{ⓔtɕhɯβja} 
\classe{n} 
\begin{définition}\pfra{espèce d'oiseau}\end{définition}
\begin{définition}\pcmn{一种鸟}\end{définition}\étymologie{tɕʰɯ.bʲa}\end{entrée}

\begin{entrée}{tɕhɯβroʁ}{}{ⓔtɕhɯβroʁ} 
\classe{n} 
\begin{définition}\pfra{tsampa}\end{définition}
\begin{définition}\pcmn{糌粑的一种吃法}\end{définition}
\begin{exemple}\pjya{tɤ-prɤm tɯ-sqar tɯ-mtɕoʁ khɯtsa ɯ-ŋgɯ pjɯ́-wɣ-lɤt tɕe ɯ-taʁ tʂha tú-wɣ-rku tɕe ɲɯ́-wɣ-ɕmi tɕe tɯ-tshi kɯ-fse nɯ kú-wɣ-tshi tɕe nɯ tɕhɯβroʁ rmi}\hspace{5pt}\pcmn{在碗里放一撮糌粑,倒上马茶,搅成粥就可以吃,这叫作\lien{ⓔtɕhɯβroʁ}{tɕhɯβroʁ}。}\end{exemple}\end{entrée}

\begin{entrée}{tɕhɯβtɕhɯβ/\variante{tɕhɯptɕhɯp}}{}{ⓔtɕhɯβtɕhɯβ} 
\classe{idph.2} 
\begin{définition}\pfra{apparition de gouttes d'eau}\end{définition}
\begin{définition}\pcmn{(看得到)水珠}\end{définition}
\begin{définition}\pcmn{在下雨(有珍珠般的雨点)}\end{définition}
\begin{exemple}\pjya{tɯ-ci ɯ-pɕi tɕhɯβtɕhɯβ ʑo ɲɤ-nɯ-ɬoʁ}\hspace{5pt}\pcmn{(口袋、布)冒了水珠}\end{exemple}
\begin{exemple}\pjya{tɯ-mɯ tɕhɯptɕhɯp ɲɯ-ɤsɯ-lɤt}\end{exemple}
\begin{sous-entrée}{ɣɤtɕhɯβtɕhɯβ}{ⓔtɕhɯβtɕhɯβⓝɣɤtɕhɯβtɕhɯβ} 
\classe{vs} 
\begin{définition}\pfra{qui coule}\end{définition}
\begin{définition}\pcmn{漏水的}\end{définition}\relationsémantique{参考}{\lien{ⓔtshɯptshɯp}{tshɯptshɯp}}\end{sous-entrée}

\end{entrée}

\begin{entrée}{tɕhɯɕɤl}{}{ⓔtɕhɯɕɤl} 
\classe{n} 
\begin{définition}\pfra{cristal}\end{définition}
\begin{définition}\pcmn{水晶}\end{définition}
\begin{exemple}\pjya{tɕhɯɕɤl-mprɯwa}\hspace{5pt}\pcmn{水晶的念珠}\end{exemple}\étymologie{tɕʰu.ɕel}\end{entrée}

\begin{entrée}{tɕhɯɕrɤm}{}{ⓔtɕhɯɕrɤm} 
\classe{n} 
\begin{définition}\pfra{loutre}\end{définition}
\begin{définition}\pcmn{水獭}\end{définition}\étymologie{tɕʰu.sram}\end{entrée}

\begin{entrée}{tɕhɯɣur}{}{ⓔtɕhɯɣur} 
\classe{n} 
\begin{définition}\pfra{digue}\end{définition}
\begin{définition}\pcmn{堤坝}\end{définition}\end{entrée}

\begin{entrée}{tɕhɯjɤr}{}{ⓔtɕhɯjɤr} 
\classe{n} 
\begin{définition}\pfra{digue}\end{définition}
\begin{définition}\pcmn{堤坝}\end{définition}\étymologie{tɕʰu.jur}\end{entrée}

\begin{entrée}{tɕhɯkɤɣar}{}{ⓔtɕhɯkɤɣar} 
\classe{n} 
\begin{définition}\pfra{bord de l'eau}\end{définition}
\begin{définition}\pcmn{水边}\end{définition}
\begin{exemple}\pjya{ftɕar tɕe tɕhɯkɤɣar ɲ-ɯkɯ-nɯrɤχtɕi tɕe sɤscit}\hspace{5pt}\pcmn{夏天的时候在水边洗东西很舒服}\end{exemple}
\begin{exemple}\pjya{tɤ-pɤtso ra tɕhɯkɤɣar k-ɤɕe rga-nɯ}\hspace{5pt}\pcmn{小孩子喜欢到水边去}\end{exemple}
\begin{exemple}\pjya{tɤ-pɤtso ra tɕhɯkɤɣar kɤ-ɤnɯɣro rga-nɯ}\hspace{5pt}\pcmn{小孩子喜欢在水边玩}\end{exemple}\end{entrée}

\begin{entrée}{tɕhɯla}{}{ⓔtɕhɯla} 
\classe{n} 
\begin{définition}\pfra{habit d'homme en laine}\end{définition}
\begin{définition}\pcmn{男式藏装,毛织品}\end{définition}\end{entrée}

\begin{entrée}{tɕhɯlɤɣrum}{}{ⓔtɕhɯlɤɣrum} 
\classe{n} 
\begin{définition}\pfra{habit tibétain blanc}\end{définition}
\begin{définition}\pcmn{白色的藏式服装}\end{définition}\relationsémantique{参考}{\lien{ⓔwɣrum}{wɣrum}}\relationsémantique{参考}{\lien{ⓔtɕhɯla}{tɕhɯla}}\end{entrée}

\begin{entrée}{tɕhɯma}{}{ⓔtɕhɯma} 
\classe{n} 
\begin{définition}\pfra{navet (Brassica sp.)}\end{définition}
\begin{définition}\pcmn{芜菁的一种【冬圆根】}\end{définition}
\begin{exemple}\pjya{tɕhɯma nɯ ɯ-tshɯɣa ra rasti cho naχtɕɯɣ, tɕeri ɯ-mdoʁ mɤ-nɤχtɕɯɣ, tɕhɯma nɯ ɯ-jwaʁ nɯ mpɕu, kɯ-xtɕɯ-xtɕi nɤmbju, ɯ-jwaʁ cho ɯ-jndoʁ nɯ ra rasti sɤznɤ mpɯ, rasti nɯ χɕitka tɕe pjɯ́-wɣ-ji, tɕe zgoku nɯ ra stonka mɤɕtʂa mɤ-kɤ-phɯt khɯ. a-pɯ-nɯɣur tɕe, mɤʑɯ arŋi cho mpɯ. tɕhɯma nɯ χɕitka sthɯci mɤ-lɤ́-wɣ-ji kɯ stonka tɕe lú-wɣ-ji. kɯmaʁ tɤ-rɤku kɤ́-wɣ-phɯt ɯ-mphru ɕɯmɯma tɕe lú-wɣ-ji tɕe tɯ-sla jamar tɕe kɤ-ndza tu-βze cha. li tɤjko ɯ-spa kɯ-pe ŋu. ɯ-jndoʁ nɯ paʁndza pe, kɤ́-wɣ-sqa tɕe tɯrme kɤ-ndza sna.}\hspace{5pt}\pcmn{冬圆根样子和圆根一样,但是颜色不同,冬圆根的叶子是光滑的,有点光泽,叶子和根比圆根嫩。圆根在春天种下,在高山上的那些可以等到秋天才拔,打霜后才收会更绿、更嫩。冬圆根不要在春天种,等其他庄稼收完以后,马上把它种下,不到一个月就可以吃了。也是煮酸菜的好材料,根有是喂猪的好饲料,煮熟后人也可以吃。}\end{exemple}\end{entrée}

\begin{entrée}{tɕhɯmɲɯɣ}{}{ⓔtɕhɯmɲɯɣ} 
\classe{n} 
\begin{définition}\pfra{source}\end{définition}
\begin{définition}\pcmn{泉水}\end{définition}\end{entrée}

\begin{entrée}{tɕhɯndza}{}{ⓔtɕhɯndza} 
\classe{n} 
\begin{définition}\pfra{flot}\end{définition}
\begin{définition}\pcmn{水流}\end{définition}
\begin{exemple}\pjya{tɕhɯndza mɤ-kɯ-βdi}\hspace{5pt}\pcmn{水流很急(水面不平稳)}\end{exemple}
\begin{exemple}\pjya{tɕhɯndza a-mɤ-pɯ-βdi tɕe, tɯ-ci ɯ-zgra wxti}\hspace{5pt}\pcmn{水流很急的时候很吵}\end{exemple}\end{entrée}

\begin{entrée}{tɕhɯŋ}{}{ⓔtɕhɯŋ} 
\classe{idph.1} 
\begin{définition}\pfra{cliquetis de métaux}\end{définition}
\begin{définition}\pcmn{形容(金属)撞击声,叮当声}\end{définition}
\begin{sous-entrée}{sɤtɕhɯŋtɕhɯŋ}{ⓔtɕhɯŋⓝsɤtɕhɯŋtɕhɯŋ} 
\classe{vt} 
\begin{exemple}\pjya{ɲɯ-sɤtɕhɯŋtɕhɯŋ ɲɯ-ɤsɯ-ʑmbri}\hspace{5pt}\pcmn{他(敲着铁的东西)发出叮当声}\end{exemple}\end{sous-entrée}

\end{entrée}

\begin{entrée}{tɕhɯŋkhɤr}{}{ⓔtɕhɯŋkhɤr} 
\classe{n} 
\begin{définition}\pfra{moulin à eau}\end{définition}
\begin{définition}\pcmn{水车}\end{définition}\étymologie{tɕʰu.ⁿkʰor}\end{entrée}

\begin{entrée}{tɕhɯŋkhɤru}{}{ⓔtɕhɯŋkhɤru} 
\classe{n} 
\begin{définition}\pfra{axe du moulin}\end{définition}
\begin{définition}\pcmn{水磨的主杆}\end{définition}
\begin{exemple}\pjya{tɕhɯŋkhɤru nɯ tɕhɯŋkhɤr ɯ-χcɤl ʑo ɕoŋtɕa tu-kɯ-ɕe tɕe βɣɤsni kɯ-ndo nɯ ŋu, ɯ-taʁ βɣɤrnɤjwaʁ kú-wɣ-tshoʁ ra}\hspace{5pt}\pcmn{\lien{}{tɕhɯŋkhɤr-ru}是水磨中间立着的部件,用来支撑磨心,前后左右装有木板。}\end{exemple}\end{entrée}

\begin{entrée}{tɕhɯŋtɕhɯŋ}{}{ⓔtɕhɯŋtɕhɯŋ} 
\classe{idph.2} 
\begin{définition}\pfra{pure, propre (eau)}\end{définition}
\begin{définition}\pcmn{形容(水)纯净、干净的样子}\end{définition}
\begin{exemple}\pjya{tɯ-ci tɕhɯŋtɕhɯŋ ɲɯ-ɤmgri}\hspace{5pt}\pcmn{水澄清得一点渣滓也没有的样子}\end{exemple}\end{entrée}

\begin{entrée}{tɕhɯpɣa}{}{ⓔtɕhɯpɣa} 
\classe{n} 
\begin{définition}\pfra{canard}\end{définition}
\begin{définition}\pcmn{鸭子}\end{définition}\relationsémantique{参考}{\lien{ⓔpɣa}{pɣa}}\end{entrée}

\begin{entrée}{tɕhɯphɯɣ}{}{ⓔtɕhɯphɯɣ} 
\classe{n} 
\begin{définition}\pfra{source d'un fleuve}\end{définition}
\begin{définition}\pcmn{水源}\end{définition}\relationsémantique{参考}{\lien{ⓔɯ-phɯɣ}{ɯ-phɯɣ}}\étymologie{tɕʰu.pʰugs}\end{entrée}

\begin{entrée}{tɕhɯpi}{}{ⓔtɕhɯpi} 
\classe{n} 
\begin{définition}\pfra{être trempé par la pluie}\end{définition}
\begin{définition}\pcmn{被雨淋得湿透了}\end{définition}
\begin{exemple}\pjya{tɕhɯpi ʑo ɲɤ-tɯ́-wɣ-ta}\hspace{5pt}\pcmn{你被雨淋到了}\end{exemple}\end{entrée}

\begin{entrée}{tɕhɯqhu}{}{ⓔtɕhɯqhu} 
\classe{adv} 
\begin{définition}\pfra{dessous de l'échelle}\end{définition}
\begin{définition}\pcmn{梯子后面}\end{définition}\relationsémantique{参考}{\lien{ⓔtɕhiⓗ2}{tɕhi₂}}\end{entrée}

\begin{entrée}{tɕhɯra}{}{ⓔtɕhɯra} 
\classe{n} 
\begin{définition}\pfra{cuve à eau}\end{définition}
\begin{définition}\pcmn{水缸(积累水的缸子)}\end{définition}\end{entrée}

\begin{entrée}{tɕhɯrdu}{}{ⓔtɕhɯrdu} 
\classe{n} 
\begin{définition}\pfra{galet}\end{définition}
\begin{définition}\pcmn{卵石}\end{définition}\étymologie{tɕʰu.rdo}\end{entrée}

\begin{entrée}{tɕhɯrkɯ}{}{ⓔtɕhɯrkɯ} 
\classe{n} 
\begin{définition}\pfra{gamelle de chien}\end{définition}
\begin{définition}\pcmn{狗的饭盆,给狗吃饭的盆}\end{définition}\end{entrée}

\begin{entrée}{tɕhɯrqhioʁ}{}{ⓔtɕhɯrqhioʁ} 
\classe{n} 
\begin{définition}\pfra{canal}\end{définition}
\begin{définition}\pcmn{渠道}\end{définition}\end{entrée}

\begin{entrée}{tɕhɯrtsɤm}{}{ⓔtɕhɯrtsɤm} 
\classe{n} 
\begin{définition}\pfra{tsampa}\end{définition}
\begin{définition}\pcmn{糌粑的一种吃法}\end{définition}\étymologie{tɕʰu.rtsam.pa}\end{entrée}

\begin{entrée}{tɕhɯrwa}{}{ⓔtɕhɯrwa} 
\classe{n} 
\begin{définition}\pfra{quark, tvorog}\end{définition}
\begin{définition}\pcmn{奶渣}\end{définition}\étymologie{pʰʲur.ba}\end{entrée}

\begin{entrée}{tɕhɯʁja}{}{ⓔtɕhɯʁja} 
\classe{n} 
\begin{définition}\pfra{lentille d'eau}\end{définition}
\begin{définition}\pcmn{浮萍}\end{définition}\étymologie{*tɕʰu.gja}\end{entrée}

\begin{entrée}{tɕhɯʁjɯ}{}{ⓔtɕhɯʁjɯ} 
\classe{n} 
\begin{définition}\pfra{animaux aquatiques}\end{définition}
\begin{définition}\pcmn{水虫}\end{définition}\end{entrée}

\begin{entrée}{tɕhɯʁnɤz/\variante{tɕhɯʁɲɤz}}{}{ⓔtɕhɯʁnɤz} 
\classe{n} 
\begin{définition}\pfra{monstre aquatique}\end{définition}
\begin{définition}\pcmn{水怪}\end{définition}\end{entrée}

\begin{entrée}{tɕhɯskrɯt}{}{ⓔtɕhɯskrɯt} 
\classe{n} 
\begin{définition}\pfra{fil épais que l'on coud sur le bord des habits tibétains}\end{définition}
\begin{définition}\pcmn{(藏装)缝在衣服边缘的粗线(头星子)}\end{définition}\end{entrée}

\begin{entrée}{tɕhɯsloŋ}{}{ⓔtɕhɯsloŋ} 
\classe{n} 
\begin{définition}\pfra{inondation}\end{définition}
\begin{définition}\pcmn{洪水}\end{définition}\end{entrée}

\begin{entrée}{tɕhɯsmɤn}{}{ⓔtɕhɯsmɤn} 
\classe{n} 
\begin{définition}\pfra{source chaude}\end{définition}
\begin{définition}\pcmn{温泉}\end{définition}\end{entrée}

\begin{entrée}{tɕhɯt/\variante{xtɕhɯt}}{}{ⓔtɕhɯt} 
\classe{vs} \paradigme{dir}{tɤ-}\paradigme{dir}{tɤ-}\paradigme{dir}{tɤ-}
\begin{définition}\pfra{pouvoir contenir}\end{définition}
\begin{définition}\pcmn{容得下;装得下}\end{définition}
\begin{définition}\pfra{faire de la place}\end{définition}
\begin{définition}\pcmn{空出……的位子}\end{définition}
\begin{définition}\pfra{faire de la place}\end{définition}
\begin{définition}\pcmn{空出……的位子}\end{définition}
\begin{exemple}\pjya{tɤ-fkɯm ɲɯ-tɕhɯt}\hspace{5pt}\pcmn{口袋容得下(那么多)}\end{exemple}
\begin{exemple}\pjya{tɤ-fkɯm ɯ-ŋgɯ ɲɯ-xtɕhɯt (mɯ́j-xtɕhɯt)}\hspace{5pt}\pcmn{袋子里装得下(装不下)}\end{exemple}
\begin{exemple}\pjya{lʁa ɯ-ŋgɯ thɯ-ɣnde ma a-pɯ-xtɕhɯt ɬoʁ}\hspace{5pt}\pcmn{你把口袋塞紧一点因为必须装下}\end{exemple}
\begin{exemple}\pjya{kɯmaʁ laχtɕha ra nɯ-nɤscɯscat-a tɕe, rgɤm kɤ-ta tɤ-ɣɤtɕhɯt-a}\hspace{5pt}\pcmn{我把其他东西搬走了,空出了放箱子的位子}\end{exemple}
\begin{exemple}\pjya{tɤ-sɯxtɕhɯt-a}\hspace{5pt}\pcmn{我都装进去了}\end{exemple}
\begin{exemple}\pjya{ɯʑo kɤ-ɤmdzɯ tɤ-sɯxtɕhɯt-a}\hspace{5pt}\pcmn{我给他找了个位子坐(本来没有位子)}\end{exemple}
\begin{sous-entrée}{sɯxtɕhɯt}{ⓔtɕhɯtⓝsɯxtɕhɯt} 
\classe{vt} \end{sous-entrée}

\begin{sous-entrée}{ɣɤtɕhɯt}{ⓔtɕhɯtⓝɣɤtɕhɯt} 
\classe{vt} \end{sous-entrée}

\end{entrée}

\begin{entrée}{tɕhɯtɕhɯtɕhɯt}{}{ⓔtɕhɯtɕhɯtɕhɯt} 
\classe{adv} 
\begin{définition}\pfra{en mettre autant que possible}\end{définition}
\begin{définition}\pcmn{能装多少就装多少}\end{définition}
\begin{exemple}\pjya{nɤ-khɯtsa tɕhɯtɕhɯtɕhɯt nɯ pɯ-nɯrke tɕe, ɯ-ro nɯ nɯ-βde}\hspace{5pt}\pcmn{你碗里能装多少就装多少,把剩余的放在那里}\end{exemple}\end{entrée}

\begin{entrée}{tɕhɯtɕɯn}{}{ⓔtɕhɯtɕɯn} 
\classe{n}  
\grammaire{n.lieu} 
\begin{définition}\pfra{Chuchen}\end{définition}
\begin{définition}\pcmn{金川}\end{définition}\end{entrée}

\begin{entrée}{tɕhɯtɕɯnpaχɕi}{}{ⓔtɕhɯtɕɯnpaχɕi} 
\classe{n} 
\begin{définition}\pfra{poire}\end{définition}
\begin{définition}\pcmn{梨子}\end{définition}\relationsémantique{参考}{\lien{ⓔnɯtɕhɯtɕɯnpaχɕi}{nɯtɕhɯtɕɯnpaχɕi}}\end{entrée}

\begin{entrée}{tɕhɯte}{}{ⓔtɕhɯte} 
\classe{n} 
\begin{définition}\pfra{grand fleuve}\end{définition}
\begin{définition}\pcmn{大河}\end{définition}\end{entrée}

\begin{entrée}{tɕhɯthɤn}{}{ⓔtɕhɯthɤn} 
\classe{n} 
\begin{définition}\pfra{coulée de boue}\end{définition}
\begin{définition}\pcmn{洪水;泥石流}\end{définition}
\begin{exemple}\pjya{tɕhɯthɤn chɤ-ɣi}\hspace{5pt}\pcmn{来了泥石流}\end{exemple}\étymologie{tɕʰu.tʰan}\end{entrée}

\begin{entrée}{tɕhɯtoʁ}{}{ⓔtɕhɯtoʁ} 
\classe{n} 
\begin{définition}\pfra{marais, endroit humide}\end{définition}
\begin{définition}\pcmn{湿地}\end{définition}
\begin{exemple}\pjya{tɕhɯtoʁ ɯ-ku}\hspace{5pt}\pcmn{沼泽}\end{exemple}\end{entrée}

\begin{entrée}{tɕhɯtsa}{}{ⓔtɕhɯtsa} 
\classe{n} 
\begin{définition}\pfra{petit ruisseau}\end{définition}
\begin{définition}\pcmn{小河}\end{définition}\end{entrée}

\begin{entrée}{tɕhɯtɯɣ}{}{ⓔtɕhɯtɯɣ} 
\classe{n} 
\begin{définition}\pfra{source empoisonnée}\end{définition}
\begin{définition}\pcmn{有毒的山泉}\end{définition}\étymologie{tɕʰu.dug}\end{entrée}

\begin{entrée}{tɕhɯwɯr}{}{ⓔtɕhɯwɯr} 
\classe{n} 
\begin{définition}\pfra{ampoule}\end{définition}
\begin{définition}\pcmn{水疱}\end{définition}
\begin{exemple}\pjya{a-jaʁ tɕhɯwɯr rɣurɣu ʑo to-rku}\hspace{5pt}\pcmn{我手上长了水泡}\end{exemple}\relationsémantique{参考}{\lien{ⓔcɯmbɤrom}{cɯmbɤrom}}\étymologie{tɕʰu.bur}\end{entrée}

\begin{entrée}{tɕhɯχpri}{}{ⓔtɕhɯχpri} 
\classe{n} 
\begin{définition}\pfra{salamandre}\end{définition}
\begin{définition}\pcmn{四脚蛇}\end{définition}
\begin{exemple}\pjya{tɕhɯχpri nɯ tɯ-ci ɯ-ŋgɯ ku-rɤʑi ŋu, qajɯ ci ŋu, kɯ-ɲɯ-ɲaʁ kɯ-nɤmbɯ-mbju ci ŋu, ɯ-mi kɯβdɤ-ldʑa tu, ɯ-mɤndzu kɯβdɤ-ldʑa ɣɤʑu, tɤ-ŋke tɕe ɯ-mɤlɤjaʁ ju-scɤt ɲɯ-ŋu, ɯ-jme ju-nɤkhɯkhrɯt ŋu, ɯ-ku nɯ qapri cho naχtɕɯɣ, ɯ-phoŋbu acilaj ʑo, tɯ-mtshi kɯ-mŋɤm smɤn ɲɯ-ŋu. kɯ-mdoʁmdi mɤ-kɯ-si chɯ́-wɣ-mqlaʁ tɕe stu kɯ-phɤn ɲɯ-ŋu khi.}\hspace{5pt}\pcmn{四脚蛇栖息在水里,是一种虫,黑色,有光泽。有四只脚,每只脚上有四只脚趾,用四肢爬行,拖着尾巴。头部像蛇的一样,全身湿漉漉的,是治胃病的好药材,据说如果能整个活吞效果最好。}\end{exemple}\end{entrée}

\begin{entrée}{tɕhɯχtɤrci}{}{ⓔtɕhɯχtɤrci} 
\classe{n} 
\begin{définition}\pfra{source d'où coule une eau de couleur blanche}\end{définition}
\begin{définition}\pcmn{流出白色的水的山泉}\end{définition}\end{entrée}

\begin{entrée}{tɕhɯχtso}{}{ⓔtɕhɯχtso} 
\classe{n} 
\begin{définition}\pfra{eau propre, potable}\end{définition}
\begin{définition}\pcmn{净水}\end{définition}\étymologie{tɕʰu.gtsaŋ}\end{entrée}

\begin{entrée}{tɕhɯzɯ}{}{ⓔtɕhɯzɯ} 
\classe{n} 
\begin{définition}\pfra{élément du métier à tisser}\end{définition}
\begin{définition}\pcmn{筘}\end{définition}
\begin{exemple}\pjya{tɕhɯzɯ nɯ thɯ-kɯ-taʁ tɕe kɤ-taʁ ɯ-sɤ-ndo ɯ-kɯ-z-rɤsta spa ɣɯ si nɯ-kɤ-βzu ŋu, tsuku tɕe phu mu tu, tsuku tɕe kɯ-ɤntɤm ɕti}\hspace{5pt}\pcmn{\lien{ⓔtɕhɯzɯ}{tɕhɯzɯ}是用来夹住布料控制它的松紧程度的木条,有些是由凸出来和凹进去的两个部分组成的,有的是平的。}\end{exemple}\end{entrée}

\begin{entrée}{tɕhɯʑaŋ}{}{ⓔtɕhɯʑaŋ} 
\classe{n} 
\begin{définition}\pfra{irrigation}\end{définition}
\begin{définition}\pcmn{灌溉}\end{définition}
\begin{exemple}\pjya{tɕhɯʑaŋ thɯ-lat-a}\hspace{5pt}\pcmn{我灌溉了(农田)}\end{exemple}\étymologie{tɕʰu.ʑiŋ}\end{entrée}

\begin{entrée}{tɕi}{₁}{ⓔtɕiⓗ1} 
\classe{adv} 
\begin{définition}\pfra{aussi}\end{définition}
\begin{définition}\pcmn{也}\end{définition}\end{entrée}

\begin{entrée}{tɕi}{₂}{ⓔtɕiⓗ2} 
\classe{part} 
\begin{définition}\pfra{marque de topique}\end{définition}
\begin{définition}\pcmn{嘛}\end{définition}
\begin{exemple}\pjya{kɤ-ntɕhoz ɯ-spa tɕi, ɕ-tɤ-χti}\hspace{5pt}\pcmn{要用的东西嘛,得去买}\end{exemple}\end{entrée}

\begin{entrée}{tɕiʑo}{}{ⓔtɕiʑo} 
\classe{pro} 
\begin{définition}\pfra{nous deux}\end{définition}
\begin{définition}\pcmn{我们俩}\end{définition}\end{entrée}

\begin{entrée}{tɕoχtsi}{}{ⓔtɕoχtsi} 
\classe{n} 
\begin{définition}\pfra{table}\end{définition}
\begin{définition}\pcmn{桌子}\end{définition}\étymologie{ltɕog.rtsi}\end{entrée}

\begin{entrée}{tɕur}{₁}{ⓔtɕurⓗ1} 
\classe{vs} \paradigme{dir}{nɯ-}\paradigme{dir}{pɯ-}\paradigme{dir}{pɯ-}\paradigme{dir}{thɯ-}\paradigme{dir}{pɯ-}
\begin{définition}\pfra{acide}\end{définition}
\begin{définition}\pcmn{酸}\end{définition}
\begin{définition}\pfra{rendre acide}\end{définition}
\begin{définition}\pcmn{令……变得更酸}\end{définition}
\begin{définition}\pcmn{因为菜不够酸,我放了一点醋,把菜弄得更酸}\end{définition}
\begin{définition}\pfra{rendre acide}\end{définition}
\begin{définition}\pcmn{令……变酸}\end{définition}
\begin{définition}\pfra{trouver acide}\end{définition}
\begin{définition}\pcmn{觉得酸}\end{définition}
\begin{exemple}\pjya{ɯ-tɯ-tɕur kɯ tɯ-ku ɲɯ-kɯ-sɯ-sɤphɤr ɲɯ-ŋu}\hspace{5pt}\pcmn{酸到昏了头}\end{exemple}
\begin{exemple}\pjya{@cai ɯ-tɯ-tɕur mɯ́j-rtaʁ tɕe, @cu pɯ-lat-a tɕe pɯ-ɣɤtɕur-a}\end{exemple}
\begin{exemple}\pjya{tɤjko thɯ-sɯxtɕur-a}\hspace{5pt}\pcmn{我把圆根(煮熟了以后)弄酸了}\end{exemple}
\begin{exemple}\pjya{pɯ-nɤxtɕur-a}\hspace{5pt}\pcmn{我觉得很酸}\end{exemple}
\begin{sous-entrée}{ɣɤtɕur}{ⓔtɕurⓗ1ⓝɣɤtɕur} 
\classe{vt}  
\grammaire{caus} \end{sous-entrée}

\begin{sous-entrée}{sɯxtɕur}{ⓔtɕurⓗ1ⓝsɯxtɕur}\end{sous-entrée}

\begin{sous-entrée}{nɤxtɕur}{ⓔtɕurⓗ1ⓝnɤxtɕur} 
\classe{vt}  
\grammaire{trop} \end{sous-entrée}

\end{entrée}

\begin{entrée}{tɕur}{₂}{ⓔtɕurⓗ2} 
\classe{vt} 
\begin{définition}\pfra{insérer dans}\end{définition}
\begin{définition}\pcmn{插(筷子、吸管、笔)}\end{définition}
\begin{exemple}\pjya{@cai ɯ-ŋgɯ ndʑu pɯ-tɕur-a}\hspace{5pt}\pcmn{你用筷子夹菜}\end{exemple}
\begin{exemple}\pjya{chɤmda ɯ-ŋgɯ tɕe chɤmdɤru pɯ-tɕur-a}\hspace{5pt}\pcmn{你把吸管插进坛坛酒了}\end{exemple}\relationsémantique{同义词}{\lien{}{sɤsta}}\end{entrée}

\begin{entrée}{tɕrɯɣnɤtɕrɯɣ}{}{ⓔtɕrɯɣnɤtɕrɯɣ} 
\classe{idph.3} 
\begin{définition}\pfra{grincement de dent}\end{définition}
\begin{définition}\pcmn{形容牙齿摩擦的响声}\end{définition}\end{entrée}

\begin{entrée}{tɕʁɯβnɤtɕʁɯβ}{}{ⓔtɕʁɯβnɤtɕʁɯβ} 
\classe{idph.3} \paradigme{dir}{pɯ-}
\begin{définition}\pfra{croquant}\end{définition}
\begin{définition}\pcmn{形容食物脆}\end{définition}
\begin{définition}\pfra{croquer}\end{définition}
\begin{définition}\pcmn{使劲地咀嚼(脆的食物)}\end{définition}
\begin{exemple}\pjya{paχɕi tɤ-nɯtɕʁɯβ-a ʑo tɤ-ndza-t-a}\hspace{5pt}\pcmn{我吃了苹果,发出咀嚼的声音}\end{exemple}\relationsémantique{反义词}{\lien{ⓔtɕʁɯznɤtɕʁɯz}{tɕʁɯznɤtɕʁɯz}}\relationsémantique{反义词}{\lien{ⓔtɕhʁɯβnɤtɕhʁɯβ}{tɕhʁɯβnɤtɕhʁɯβ}}
\begin{sous-entrée}{nɯtɕʁɯβ}{ⓔtɕʁɯβnɤtɕʁɯβⓝnɯtɕʁɯβ} 
\classe{vt} \end{sous-entrée}

\end{entrée}

\begin{entrée}{tɕʁɯznɤtɕʁɯz}{}{ⓔtɕʁɯznɤtɕʁɯz} 
\classe{idph.3} 
\begin{définition}\pfra{croquant}\end{définition}
\begin{définition}\pcmn{形容食物脆}\end{définition}
\begin{exemple}\pjya{tú-wɣ-ndza tɕe tɕʁɯznɤtɕʁɯz ɲɯ-ti}\hspace{5pt}\pcmn{吃起来很脆}\end{exemple}\relationsémantique{反义词}{\lien{ⓔzwaʁnɤzwaʁ}{zwaʁnɤzwaʁ}}\relationsémantique{反义词}{\lien{ⓔtɕʁɯβnɤtɕʁɯβ}{tɕʁɯβnɤtɕʁɯβ}}\end{entrée}

\begin{entrée}{tɕɯlɤβ}{}{ⓔtɕɯlɤβ} 
\classe{n} 
\begin{définition}\pfra{pipe}\end{définition}
\begin{définition}\pcmn{烟斗}\end{définition}\end{entrée}

\begin{entrée}{tɕɯxpa}{}{ⓔtɕɯxpa} 
\classe{n} 
\begin{définition}\pfra{conclue, réglée (affaire)}\end{définition}
\begin{définition}\pcmn{定好(交易)}\end{définition}
\begin{exemple}\pjya{tɕɯxpa pɯ-nɯ-rku-tɕi}\hspace{5pt}\pcmn{我们把事情定好了}\end{exemple}\end{entrée}

\begin{entrée}{tɕɯχtsi}{}{ⓔtɕɯχtsi} 
\classe{n} 
\begin{définition}\pfra{Tchogtse}\end{définition}
\begin{définition}\pcmn{卓克基}\end{définition}\end{entrée}

\begin{entrée}{tɣa}{}{ⓔtɣa} 
\classe{vi} \paradigme{dir}{kɤ-}
\begin{définition}\pfra{récolter}\end{définition}
\begin{définition}\pcmn{收割}\end{définition}
\begin{exemple}\pjya{kɤ-tɣa tɤ-sɤsqɤr-i}\hspace{5pt}\pcmn{我们请了别人帮忙收割}\end{exemple}\relationsémantique{参考}{\lien{ⓔtɯtɣaⓗ2}{tɯtɣa₂}}\end{entrée}

\begin{entrée}{thu}{₂}{ⓔthuⓗ2} 
\classe{n} 
\begin{définition}\pfra{borne, marque (pile de pierre)}\end{définition}
\begin{définition}\pcmn{用来当标记的石堆}\end{définition}\end{entrée}

\begin{entrée}{thu}{₁}{ⓔthuⓗ1} 
\classe{vt} \paradigme{dir}{tɤ-}
\begin{définition}\pfra{demander}\end{définition}
\begin{définition}\pcmn{问}\end{définition}
\begin{exemple}\pjya{tɤ-thu-t-a}\hspace{5pt}\pcmn{我问了他}\end{exemple}
\begin{exemple}\pjya{nɤ-kɤ-thu tɤ-the jɤɣ}\hspace{5pt}\pcmn{你可以问你的问题}\end{exemple}
\begin{exemple}\pjya{a-kɤ-thu nɯ ma thɯ-arɕo}\hspace{5pt}\pcmn{我没有其他问题了}\end{exemple}
\begin{exemple}\pjya{tɤ́-wɣ-thu-a}\hspace{5pt}\pcmn{他向别人问了我的情况}\end{exemple}
\begin{exemple}\pjya{a-ɕki ta-thu}\hspace{5pt}\pcmn{他问了我}\end{exemple}
\begin{exemple}\pjya{kɯmaʁ ci tu-the-a ŋu}\hspace{5pt}\pcmn{我再问一个问题}\end{exemple}
\begin{exemple}\pjya{tʂu tɤ-thu-t-a}\hspace{5pt}\pcmn{我问了路}\end{exemple}
\begin{sous-entrée}{rɤthu}{ⓔthuⓗ1ⓝrɤthu} 
\classe{vi}  
\grammaire{apass} 
\begin{définition}\pfra{demander à quelqu'un}\end{définition}
\begin{définition}\pcmn{问某人;向某人请教}\end{définition}
\begin{exemple}\pjya{ɯʑo ɯ-ɕki ɕɯ-rɤthu-a}\hspace{5pt}\pcmn{我去向他请教}\end{exemple}
\begin{exemple}\pjya{a-ɕki tɤ-tɯ-rɤthu}\hspace{5pt}\pcmn{你问了我}\end{exemple}\relationsémantique{参考}{\lien{ⓔrɤthuthe}{rɤthuthe}}\end{sous-entrée}

\end{entrée}

\begin{entrée}{tha}{}{ⓔtha} 
\classe{adv} 
\begin{définition}\pfra{dans un moment}\end{définition}
\begin{définition}\pcmn{(不然)过一会儿就}\end{définition}
\begin{exemple}\pjya{tɕe tha tɕe a-pɯ-ŋu}\hspace{5pt}\pcmn{一会再说吧}\end{exemple}
\begin{exemple}\pjya{nɤ-ŋga tɤ-ŋge ma tha tɯ-nɯtɕhomba}\hspace{5pt}\pcmn{你把衣服穿上,不然会感冒}\end{exemple}\end{entrée}

\begin{entrée}{thamaka}{}{ⓔthamaka} 
\classe{n} 
\begin{définition}\pfra{tabac}\end{définition}
\begin{définition}\pcmn{烟}\end{définition}
\begin{exemple}\pjya{thamaka a-nɯ-tɯ-ftɕɤt}\hspace{5pt}\pcmn{你戒烟吧}\end{exemple}\étymologie{tʰa.mag}\end{entrée}

\begin{entrée}{thamatham}{}{ⓔthamatham}\relationsémantique{参考}{\lien{ⓔthamtham}{thamtham}}\end{entrée}

\begin{entrée}{thamtɕɤt}{}{ⓔthamtɕɤt} 
\classe{adv} 
\begin{définition}\pfra{complètement, tout}\end{définition}
\begin{définition}\pcmn{全部}\end{définition}\étymologie{tʰams.tɕad}\end{entrée}

\begin{entrée}{thamtham}{}{ⓔthamtham} 
\classe{adv} 
\begin{définition}\pfra{maintenant}\end{définition}
\begin{définition}\pcmn{现在}\end{définition}\étymologie{tʰam}\end{entrée}

\begin{entrée}{thaŋ}{₁}{ⓔthaŋⓗ1} 
\classe{n} 
\begin{définition}\pfra{plaine}\end{définition}
\begin{définition}\pcmn{平坝}\end{définition}\étymologie{tʰaŋ}\end{entrée}

\begin{entrée}{thaŋ}{₂}{ⓔthaŋⓗ2} 
\classe{part} 
\begin{définition}\pfra{marqueur de supposition}\end{définition}
\begin{définition}\pcmn{表示推测}\end{définition}
\begin{exemple}\pjya{wo nɯnɯ wuma ʑo mɯm thaŋ nɤ!}\hspace{5pt}\pcmn{这个应该很好吃吧!}\end{exemple}\end{entrée}

\begin{entrée}{thaʁɕa}{}{ⓔthaʁɕa} 
\classe{n} 
\begin{définition}\pfra{élément du métier à tisser}\end{définition}
\begin{définition}\pcmn{筘【板板】}\end{définition}
\begin{exemple}\pjya{thaʁɕa nɯ tɤrɤm thɯ-kɤ-βʑoʁ tɕe nɯ-kɤ-ɣɤmpɕu tɕe stɤsmɤt thɯ-kɤ-sɤmtɕoʁ ŋu, thɯ-kɯ-taʁ tɕe kɤ-taʁ ɯ-sqar ɯ-sɤ-lɤt ɯ-spa ŋu}\hspace{5pt}\pcmn{\lien{ⓔthaʁɕa}{thaʁɕa}是把木板削平了以后,两头削尖了,用来让经线和纬线的交叉部分上下移动的工具}\end{exemple}\étymologie{ⁿtʰag.ɕa}\end{entrée}

\begin{entrée}{thaʁmu}{}{ⓔthaʁmu} 
\classe{n} 
\begin{définition}\pfra{élément du métier à tisser}\end{définition}
\begin{définition}\pcmn{榨刀}\end{définition}
\begin{exemple}\pjya{thaʁmu nɯ thɯ-kɯ-taʁ tɕe kɤ-taʁ ɯ-sqar ɯ-chɯ-sɤ-ɤsɯɣ ɯ-jlɤβ ɯ-ɲɯ-sɤ-ɤrʁe ɯ-spa ŋu. tɤrɤm ɯ-rkɯ zɯ ɕom kɯ-mba tsa pɯ-kɤ-tshoʁ ci ŋu}\hspace{5pt}\pcmn{织布时,榨刀是用来把经线和纬线的交叉部分弄紧,然后穿纬线的工具。是边上装着一块薄铁片的木板。}\end{exemple}\étymologie{ⁿtʰag.ma}\end{entrée}

\begin{entrée}{thaʁŋkhor}{}{ⓔthaʁŋkhor} 
\classe{n} 
\begin{définition}\pfra{moulin à prière que l'on fait tourner avec les doigts}\end{définition}
\begin{définition}\pcmn{指捻转经筒}\end{définition}\étymologie{tʰag.ⁿkʰor}\end{entrée}

\begin{entrée}{thaʁtɕɤz}{}{ⓔthaʁtɕɤz} 
\classe{n} 
\begin{définition}\pfra{appareil à tisser}\end{définition}
\begin{définition}\pcmn{织布机}\end{définition}\étymologie{tʰag.btɕas}\end{entrée}

\begin{entrée}{thaʁ,tɕhot}{}{ⓔthaʁ,tɕhot} 
\classe{n}
\classe{vt} \paradigme{dir}{nɯ-}\paradigme{dir}{pɯ-}
\begin{définition}\pfra{prendre une décision}\end{définition}
\begin{définition}\pcmn{决定}\end{définition}
\begin{exemple}\pjya{thaʁ pɯ-tɕhot-a}\hspace{5pt}\pcmn{我决定了}\end{exemple}
\begin{exemple}\pjya{tɕi-tɯkrɤz tɤ-ɣe tɕe thaʁ nɯ-tɕhot-tɕi}\hspace{5pt}\pcmn{我们商议好了就做了决定}\end{exemple}\relationsémantique{Component 1}{\lien{}{thaʁ}}\relationsémantique{Component 2}{\lien{}{tɕhot}}\étymologie{tʰag.tɕʰod}\end{entrée}

\begin{entrée}{thaʁtsa}{}{ⓔthaʁtsa} 
\classe{n} 
\begin{définition}\pfra{ceinture colorée}\end{définition}
\begin{définition}\pcmn{花带}\end{définition}\end{entrée}

\begin{entrée}{thathor}{}{ⓔthathor} 
\classe{n} 
\begin{définition}\pfra{partie en métal de la ceinture}\end{définition}
\begin{définition}\pcmn{腰带的扣子}\end{définition}\relationsémantique{参考}{\lien{ⓔtɕhoma}{tɕhoma}}\end{entrée}

\begin{entrée}{thawaʁ}{}{ⓔthawaʁ} 
\classe{n} 
\begin{définition}\pfra{assiette}\end{définition}
\begin{définition}\pcmn{木盘子}\end{définition}\end{entrée}

\begin{entrée}{thaχthi}{}{ⓔthaχthi} 
\classe{n} 
\begin{définition}\pfra{lanière}\end{définition}
\begin{définition}\pcmn{背带(用线织成的)}\end{définition}\end{entrée}

\begin{entrée}{thɤβ/\variante{lthɤβ}}{}{ⓔthɤβ} 
\classe{n} 
\begin{définition}\pfra{clin d'œil}\end{définition}
\begin{définition}\pcmn{眨眼}\end{définition}
\begin{exemple}\pjya{a-mɲaʁ thɤβ ʑo tɤ-stu-t-a}\hspace{5pt}\pcmn{他眨了眼}\end{exemple}\end{entrée}

\begin{entrée}{thɤfka}{}{ⓔthɤfka} 
\classe{n} 
\begin{définition}\pfra{foyer}\end{définition}
\begin{définition}\pcmn{灶}\end{définition}\étymologie{tʰab.ka}\end{entrée}

\begin{entrée}{thɤfkɤlɤɣi}{}{ⓔthɤfkɤlɤɣi} 
\classe{n} 
\begin{définition}\pfra{cendre végétale}\end{définition}
\begin{définition}\pcmn{草木灰}\end{définition}\relationsémantique{同义词}{\lien{ⓔsqhɤthɤlɤɣi}{sqhɤthɤlɤɣi}}\end{entrée}

\begin{entrée}{thɤjbra}{}{ⓔthɤjbra} 
\classe{n} 
\begin{définition}\pfra{type de herse}\end{définition}
\begin{définition}\pcmn{簧耙}\end{définition}\end{entrée}

\begin{entrée}{thɤjco}{}{ⓔthɤjco} 
\classe{n} 
\begin{définition}\pfra{palanquin}\end{définition}
\begin{définition}\pcmn{轿子}\end{définition}\étymologie{fn:抬轿}\end{entrée}

\begin{entrée}{thɤjtɕu}{}{ⓔthɤjtɕu} 
\classe{pro} 
\begin{définition}\pfra{quand}\end{définition}
\begin{définition}\pcmn{什么时候}\end{définition}
\begin{exemple}\pjya{thɤjtɕu jamar tɯ-lɤt?}\hspace{5pt}\pcmn{你大概什么时候打电话?}\end{exemple}
\begin{exemple}\pjya{thɤjtɕu chiz tɯ-nɯɣi kɯ?}\hspace{5pt}\pcmn{不知道你什么时候回来?}\end{exemple}\end{entrée}

\begin{entrée}{thɤlwa}{}{ⓔthɤlwa} 
\classe{n} \paradigme{comit}{kɤ́thɤlwɯlwa}
\begin{définition}\pfra{terre}\end{définition}
\begin{définition}\pcmn{土}\end{définition}\étymologie{tʰal.ba}\end{entrée}

\begin{entrée}{thɤlwaɲaʁ}{}{ⓔthɤlwaɲaʁ} 
\classe{n} 
\begin{définition}\pfra{tchernozyom}\end{définition}
\begin{définition}\pcmn{黑土地}\end{définition}\relationsémantique{参考}{\lien{ⓔɲaʁ}{ɲaʁ}}\end{entrée}

\begin{entrée}{thɤr}{₁}{ⓔthɤrⓗ1} 
\classe{vi} \paradigme{dir}{tɤ-}
\begin{définition}\pfra{se sauver}\end{définition}
\begin{définition}\pcmn{保命}\end{définition}
\begin{exemple}\pjya{to-thɤr}\hspace{5pt}\pcmn{他得救了}\end{exemple}\étymologie{tʰar}\end{entrée}

\begin{entrée}{thɤr}{₂}{ⓔthɤrⓗ2} 
\classe{vs} \paradigme{dir}{tɤ-}
\begin{définition}\pfra{complet}\end{définition}
\begin{définition}\pcmn{满满}\end{définition}
\begin{exemple}\pjya{tɯ-xpa kɯ-thɤr ʑo}\hspace{5pt}\pcmn{整整一年}\end{exemple}\relationsémantique{同义词}{\lien{ⓔmtshɤt}{mtshɤt}}\end{entrée}

\begin{entrée}{thɤstɯɣ}{}{ⓔthɤstɯɣ} 
\classe{pro} 
\begin{définition}\pfra{combien}\end{définition}
\begin{définition}\pcmn{多少}\end{définition}
\begin{exemple}\pjya{thɤstɯɣ tɯ-zɣɯt ?}\hspace{5pt}\pcmn{你有多大?}\end{exemple}
\begin{exemple}\pjya{thɤstɯɣ thɯ-tɯ-ɤzɣɯt ?}\hspace{5pt}\pcmn{你多大了?}\end{exemple}
\begin{exemple}\pjya{nɤʑo kɯrɯskɤt pɯ-tɯ-βzjoz nɯ thɤstɯɣ to-tsu ?}\hspace{5pt}\pcmn{你藏语学了多久?}\end{exemple}
\begin{exemple}\pjya{thɤstɯ-tɯrpa}\hspace{5pt}\pcmn{几斤}\end{exemple}\end{entrée}

\begin{entrée}{thɤtɕɯ}{}{ⓔthɤtɕɯ} 
\classe{n} 
\begin{définition}\pfra{marteau}\end{définition}
\begin{définition}\pcmn{二锤}\end{définition}
\begin{exemple}\pjya{thɤtɕɯ nɯ kɯ-rɤznde ra ɣɯ nɯ-laʁdɤn ŋu rdɤstaʁ ra ɯ-sɤz-ɣɤβdoʁβdi ŋu, ɯ-ku nɯ ɕom tɕe ɯ-jɯ laʁjɯɣ thɯ-kɤ-tshoʁ ci ŋu}\hspace{5pt}\pcmn{榔头是石匠的工具,是用来修理石头的,头是铁作成,把子是一节木棒。}\end{exemple}\end{entrée}

\begin{entrée}{thɤwum}{}{ⓔthɤwum} 
\classe{n} 
\begin{définition}\pfra{une espèce d'arbre}\end{définition}
\begin{définition}\pcmn{乔木的一种【马鹿柴】}\end{définition}
\begin{exemple}\pjya{thɤwum nɯ si kɯ-mbro kɯ-jpum tsa ci ŋu, aʁɤndɯndɤt ʑo tu-ɬoʁ cha. wuma ʑo ɲɯ-ɤɣɯrtɯrtaʁ cha. ɯ-jwaʁ artɯm tɕe ɯ-ku lu-omtɕoʁ ŋu. si ɯ-ru cho ɯ-rtaʁ nɯ kɯ-ɣɯrni ŋu. ɯ-si mɤ-ngɯt, ndoʁ, sna me, kɤ-nɯβlɯ kɯnɤ khro mɤ-pe, ɯ-mɯntoʁ kɯ-wɣrum ɲɯ-lɤt ŋu. ɯ-mɯntoʁ ɯ-jɯ nɯ zri tsa ɯ-mat thɯ-tɯt tɕe kɯ-ɣɯrni ʂɣɤlʂɣɤl ʑo kɯ-pa ŋu, laŋlaŋ ɯ-mat cho naχtɕɯɣ, tɕeri kɤ-ndza mɤ-sna ma wuma ʑo qiaβ.}\hspace{5pt}\pcmn{马鹿柴是长得又高又粗的树种,到处都可以生长,长很多枝条。叶子是圆形的,一头尖。树干和树枝都是紫红色的。木质不结实,很脆,不能做什么材料,连烧火都不好。开白色的花。花梗有点长。果实成熟时是红而透明的,像\lien{ⓔlaŋlaŋ}{laŋlaŋ}的果实,但是不能吃,因为很苦。}\end{exemple}\end{entrée}

\begin{entrée}{thɣe}{}{ⓔthɣe} 
\classe{n} 
\begin{définition}\pfra{gland}\end{définition}
\begin{définition}\pcmn{橡子}\end{définition}\relationsémantique{参考}{\lien{ⓔnɯthɣe}{nɯthɣe}}\end{entrée}

\begin{entrée}{thoɲa}{}{ⓔthoɲa} 
\classe{n} 
\begin{définition}\pfra{ovidé}\end{définition}
\begin{définition}\pcmn{羊}\end{définition}\end{entrée}

\begin{entrée}{thoŋkɤn}{}{ⓔthoŋkɤn} 
\classe{n} 
\begin{définition}\pfra{récipient en cuivre}\end{définition}
\begin{définition}\pcmn{红铜铸成的罐子,没有盖子}\end{définition}\end{entrée}

\begin{entrée}{thoŋlaʁ}{}{ⓔthoŋlaʁ} 
\classe{n} 
\begin{définition}\pfra{une période, un endroit}\end{définition}
\begin{définition}\pcmn{一段(时间、地方)}\end{définition}
\begin{exemple}\pjya{nɤʑo jɤxtshi jɤ-tɯ-ɣe ɯ-thoŋlaʁ nɯ mɯ́j-ɣɤndʐo, mɯ́j-ɣɯtshɤdɯɣ tɕe ɲɯ-sɤscit}\hspace{5pt}\pcmn{你这一次来的那一段时间既不冷也不热,很舒服}\end{exemple}\end{entrée}

\begin{entrée}{thoŋthɤr}{}{ⓔthoŋthɤr} 
\classe{n} 
\begin{définition}\pfra{baguette de tassage}\end{définition}
\begin{définition}\pcmn{推弹杆}\end{définition}
\begin{exemple}\pjya{thoŋthɤr nɯ thɯ́-wɣ-rzoŋ tɕe qandʑi sɤ-ɣnda ɯ-spa ŋu}\hspace{5pt}\pcmn{推弹杆是用来把子弹塞进枪筒里的工具。}\end{exemple}\end{entrée}

\begin{entrée}{thoʁ}{₂}{ⓔthoʁⓗ2} 
\classe{n} 
\begin{définition}\pfra{foudre}\end{définition}
\begin{définition}\pcmn{霹雷}\end{définition}
\begin{exemple}\pjya{thoʁ pjɤ-ɣi}\hspace{5pt}\pcmn{打雷了}\end{exemple}
\begin{exemple}\pjya{ɣɯjpa kɯ-nɯtʂoŋtshaβ lo-ɕe-nɯ tɕe, tɤ-tɕɯ ci thoʁ kɯ tó-wɣ-tsɯɣ tɕe pjɤ-si}\hspace{5pt}\pcmn{今年,他们去采虫草时,有一个男人给雷劈死了。}\end{exemple}\étymologie{tʰog}\end{entrée}

\begin{entrée}{thoʁ}{₁}{ⓔthoʁⓗ1} 
\classe{vt} \paradigme{dir}{pɯ-}
\begin{définition}\pfra{marcher sur}\end{définition}
\begin{définition}\pcmn{踏}\end{définition}
\begin{exemple}\pjya{a-mi pɯ-thoʁ-a (a-mi pɯ-ta-t-a, a-mi kɯ pɯ-zrɤtɕaʁ-a)}\hspace{5pt}\pcmn{我踩上了}\end{exemple}\relationsémantique{同义词}{\lien{ⓔrɤtɕaʁ}{rɤtɕaʁ}}\end{entrée}

\begin{entrée}{thoʁltɕi}{}{ⓔthoʁltɕi} 
\classe{n} 
\begin{définition}\pfra{fer météoritique?}\end{définition}
\begin{définition}\pcmn{铁块(陨石)}\end{définition}\end{entrée}

\begin{entrée}{tho,thɯɣ/\variante{thoʁ,thɯɣ}}{}{ⓔtho,thɯɣ} 
\classe{n}
\classe{vs} \paradigme{dir}{tɤ-}
\begin{définition}\pfra{concordant}\end{définition}
\begin{définition}\pcmn{相符;一致(消息);很巧}\end{définition}
\begin{exemple}\pjya{tho ɲɯ-thɯɣ}\hspace{5pt}\pcmn{是相符的}\end{exemple}
\begin{exemple}\pjya{ɯʑo kɯ ta-tɯt cho nɤj tu-tɯ-ti nɯ tho ɲɯ-thɯɣ}\hspace{5pt}\pcmn{他说的和你说的完全相符}\end{exemple}\relationsémantique{Component 1}{\lien{}{tho}}\relationsémantique{Component 2}{\lien{ⓔthɯɣⓗ1}{thɯɣ}}\end{entrée}

\begin{entrée}{thotsi}{}{ⓔthotsi} 
\classe{n} 
\begin{définition}\pfra{sceau}\end{définition}
\begin{définition}\pcmn{印章(在馍馍上)}\end{définition}
\begin{exemple}\pjya{qajɣi ɯ-taʁ thotsi kɤ-ta}\hspace{5pt}\pcmn{要在馍馍上盖印章}\end{exemple}\relationsémantique{同义词}{\lien{ⓔmthɯmɤr}{mthɯmɤr}}\end{entrée}

\begin{entrée}{thoχtɤm}{}{ⓔthoχtɤm} 
\classe{n} 
\begin{définition}\pfra{impôt}\end{définition}
\begin{définition}\pcmn{税}\end{définition}
\begin{exemple}\pjya{ɯ-thoχtɤm lɤ-kho-j}\hspace{5pt}\pcmn{我们(给土司)交了粮食}\end{exemple}\end{entrée}

\begin{entrée}{thrɤβthrɤβ}{}{ⓔthrɤβthrɤβ} 
\classe{idph.2} 
\begin{définition}\pfra{dont l'épaisseur n'est pas uniforme (soupe de riz)}\end{définition}
\begin{définition}\pcmn{形容稀稠不均匀的样子(稀饭)}\end{définition}
\begin{exemple}\pjya{tɯtshi thrɤβthrɤβ ʑo ɲɯ-pa}\hspace{5pt}\pcmn{稀饭稀稠不均匀}\end{exemple}\end{entrée}

\begin{entrée}{thɯ}{₁}{ⓔthɯⓗ1} 
\classe{vt} \sens{1}\paradigme{dir}{kɤ-}\paradigme{dir}{pɯ-}
\begin{définition}\pfra{monter une tente, faire un pont}\end{définition}
\begin{définition}\pcmn{搭(桥、帐篷)}\end{définition}
\begin{exemple}\pjya{ndzom kɤ-thɯ-t-a (kú-wɣ-ta, kú-wɣ-thɯ)}\hspace{5pt}\pcmn{我搭了桥}\end{exemple}
\begin{exemple}\pjya{zgɤr pɯ-thɯ-t-a}\hspace{5pt}\pcmn{我搭了帐篷}\end{exemple}\sens{2}
\begin{définition}\pfra{construire une route}\end{définition}
\begin{définition}\pcmn{修(路)}\end{définition}
\begin{exemple}\pjya{tʂu lɤ-thɯ-t-a (=lɤ-tɕat-a)}\hspace{5pt}\pcmn{我修了路}\end{exemple}\sens{3}
\begin{définition}\pfra{séparer les fils}\end{définition}
\begin{définition}\pcmn{牵(线)}\end{définition}
\begin{exemple}\pjya{kɤ-pɣo lɤ-jɤɣ tɕe, kɤtaʁ kɤ-thɯ-t-a}\hspace{5pt}\pcmn{搓完线,我就把它牵了}\end{exemple}\sens{4}
\begin{définition}\pfra{laisser (une trace)}\end{définition}
\begin{définition}\pcmn{留下痕迹}\end{définition}
\begin{exemple}\pjya{ɯ-jroʁ jo-thɯ (jo-tɕɤt)}\hspace{5pt}\pcmn{它留了痕迹}\end{exemple}\relationsémantique{参考}{\lien{ⓔndɯⓗ1}{ndɯ₁}}\end{entrée}

\begin{entrée}{thɯ}{₂}{ⓔthɯⓗ2} 
\classe{vi}  
\grammaire{caus} \paradigme{dir}{nɯ-}\paradigme{dir}{nɯ-}
\begin{définition}\pfra{grave}\end{définition}
\begin{définition}\pcmn{严重}\end{définition}
\begin{exemple}\pjya{wuma sthɯci mɯ́j-thɯ ɲɯ-ti, ɯ́-ŋu}\hspace{5pt}\pcmn{他说没有那么严重,是吗?}\end{exemple}
\begin{exemple}\pjya{a-rna ɲɯ-thɯ ɲɯ-ŋu tɕe, koŋla mɯ́j-mtsham-a}\hspace{5pt}\pcmn{我耳背,听不清楚}\end{exemple}
\begin{exemple}\pjya{ɯ-ku kɯ-mɲɤm ɲɯ-thɯ}\hspace{5pt}\pcmn{他头疼得很厉害}\end{exemple}
\begin{sous-entrée}{ɣɤthɯ}{ⓔthɯⓗ2ⓝɣɤthɯ} 
\classe{vt} \end{sous-entrée}

\begin{définition}\pfra{aggraver}\end{définition}
\begin{définition}\pcmn{令……变得更严重}\end{définition}\relationsémantique{参考}{\lien{ⓔnɤxthɯ}{nɤxthɯ}}\end{entrée}

\begin{entrée}{thɯchu}{}{ⓔthɯchu} 
\classe{adv} 
\begin{définition}\pfra{en aval}\end{définition}
\begin{définition}\pcmn{在下游}\end{définition}\relationsémantique{参考}{\lien{ⓔɯ-thɤcu}{ɯ-thɤcu}}\end{entrée}

\begin{entrée}{thɯci}{}{ⓔthɯci} 
\classe{pro} 
\begin{définition}\pfra{quelque chose, n'importe quoi, n'importe lequel}\end{définition}
\begin{définition}\pcmn{某个;随便什么;任何一个}\end{définition}\relationsémantique{参考}{\lien{ⓔthɯthɤci}{thɯthɤci}}
\begin{sous-entrée}{thɯci fse ci ndʐa ɕti kɯ}{ⓔthɯciⓝthɯci fse ci ndʐa ɕti kɯ}
\begin{définition}\pfra{il y a sans doute une raison}\end{définition}
\begin{définition}\pcmn{也许会有办法;也许会过得去;也许是应该的}\end{définition}
\begin{exemple}\pjya{tɯmgo kɤ-sɯ-ndo mɯ́j-khɯ ri, thɯci fse ci ndʐa ɕti kɯ (mɤ-mtsɯr ndʐa ɕti kɯ)}\hspace{5pt}\pcmn{他不肯带食物走,他可能自己有办法(他可能不饿)}\end{exemple}\end{sous-entrée}

\end{entrée}

\begin{entrée}{thɯɣ}{₁}{ⓔthɯɣⓗ1} 
\classe{n} 
\begin{définition}\pfra{taureau, bouc non castré}\end{définition}
\begin{définition}\pcmn{种羊;种牛}\end{définition}\étymologie{tʰug}\end{entrée}

\begin{entrée}{thɯɣ}{₃}{ⓔthɯɣⓗ3} 
\classe{n} 
\begin{définition}\pfra{signe}\end{définition}
\begin{définition}\pcmn{记号}\end{définition}
\begin{exemple}\pjya{thɯɣ tɤ-ta-t-a}\hspace{5pt}\pcmn{我打了记号}\end{exemple}\end{entrée}

\begin{entrée}{thɯɣ}{₂}{ⓔthɯɣⓗ2} 
\classe{vi} \paradigme{dir}{kɤ-}
\begin{définition}\pfra{être détestable}\end{définition}
\begin{définition}\pcmn{令人讨厌}\end{définition}
\begin{exemple}\pjya{kɤ-thɯɣ!}\hspace{5pt}\pcmn{糟糕!}\end{exemple}
\begin{exemple}\pjya{nɤʑo ki ndɤre, ko-tɯ-thɯɣ!}\hspace{5pt}\pcmn{你这个人糟糕到无可救药的地步}\end{exemple}
\begin{exemple}\pjya{nɤʑɯɣ thɯɣ ma!}\hspace{5pt}\pcmn{到最后受苦就是你自己}\end{exemple}
\begin{exemple}\pjya{ji-zdɯɣ ɲɯ-thɯɣ}\hspace{5pt}\pcmn{我们在受苦受难}\end{exemple}\end{entrée}

\begin{entrée}{thɯɣɕɤr}{}{ⓔthɯɣɕɤr} 
\classe{n} 
\begin{définition}\pfra{marque des charpentiers sur le bois}\end{définition}
\begin{définition}\pcmn{木工打记号的线;墨线}\end{définition}\end{entrée}

\begin{entrée}{thɯɣskrɯt}{}{ⓔthɯɣskrɯt} 
\classe{n} 
\begin{définition}\pfra{marque des tailleurs sur les tissus}\end{définition}
\begin{définition}\pcmn{裁缝在布料上用来打记号的线}\end{définition}\end{entrée}

\begin{entrée}{thɯm}{}{ⓔthɯm} 
\classe{n} 
\begin{définition}\pfra{récipient}\end{définition}
\begin{définition}\pcmn{瓢子}\end{définition}
\begin{exemple}\pjya{ɲchɣaʁthɯm}\hspace{5pt}\pcmn{桦树皮的瓢子}\end{exemple}\end{entrée}

\begin{entrée}{thɯraŋ}{}{ⓔthɯraŋ} 
\classe{n} 
\begin{définition}\pfra{nain, une sorte de démon}\end{définition}
\begin{définition}\pcmn{一种鬼,矮人}\end{définition}\étymologie{tʰeɦu.raŋ}\end{entrée}

\begin{entrée}{thɯrdu}{}{ⓔthɯrdu} 
\classe{n} 
\begin{définition}\pfra{poids (d'une balance)}\end{définition}
\begin{définition}\pcmn{砝码,秤砣}\end{définition}\étymologie{tʰur.rdo}\end{entrée}

\begin{entrée}{thɯrnaʁ}{}{ⓔthɯrnaʁ} 
\classe{n} 
\begin{définition}\pfra{balance de précision}\end{définition}
\begin{définition}\pcmn{秤}\end{définition}\étymologie{tʰur.sraŋ}\end{entrée}

\begin{entrée}{thɯrʑi}{}{ⓔthɯrʑi} 
\classe{n} 
\begin{définition}\pfra{compassion}\end{définition}
\begin{définition}\pcmn{同情,怜悯}\end{définition}\étymologie{tʰugs.rje}\end{entrée}

\begin{entrée}{thɯrʑi,ʑɯ}{}{ⓔthɯrʑi,ʑɯ} 
\classe{n}
\classe{vt} \paradigme{dir}{tɤ-}
\begin{définition}\pfra{implorer la miséricorde}\end{définition}
\begin{définition}\pcmn{求饶}\end{définition}
\begin{exemple}\pjya{aʑo ju-ɕe-a tɕe thɯrʑi ɕ-tu-ʑi-a ɲɯ-ntshi}\hspace{5pt}\pcmn{我要去求饶}\end{exemple}\relationsémantique{Component 1}{\lien{ⓔthɯrʑi}{thɯrʑi}}\relationsémantique{Component 2}{\lien{}{ʑɯ}}\étymologie{thugs.rdʑe.ʑu}\end{entrée}

\begin{entrée}{thɯthɤci}{}{ⓔthɯthɤci} 
\classe{pro} 
\begin{définition}\pfra{quel, lesquels}\end{définition}
\begin{définition}\pcmn{一些什么}\end{définition}
\begin{exemple}\pjya{phɯrkhɯɣ ɯ-ŋgɯ nɤ-ŋga thɯthɤci arku}\hspace{5pt}\pcmn{在你挎包装了一些什么衣服}\end{exemple}\relationsémantique{参考}{\lien{ⓔthɯci}{thɯci}}\end{entrée}

\begin{entrée}{ti}{}{ⓔti} 
\classe{vt}
\classe{vt}  
\grammaire{caus} \sens{1}\paradigme{dir}{tɤ-}\paradigme{past stem}{tɯt}\paradigme{generic}{kɯ-ti}\paradigme{inf.1sg}{to-ti-a}
\begin{définition}\pfra{dire}\end{définition}
\begin{définition}\pcmn{说}\end{définition}
\begin{exemple}\pjya{nɯ tu-kɯ-ti mɤ-ŋgrɤl}\hspace{5pt}\pcmn{不能这样说}\end{exemple}
\begin{exemple}\pjya{li ci tɤ-ti}\hspace{5pt}\pcmn{你再说一遍}\end{exemple}
\begin{exemple}\pjya{pɤjkhu a-ʁa tu me mɤ-xsi, a-kɤ-ti me}\hspace{5pt}\pcmn{我现在说不准}\end{exemple}
\begin{exemple}\pjya{tɕhi tu-ti-a a-pɯ-ŋu ɲɯ-ra?}\hspace{5pt}\pcmn{我本应该说什么呢?}\end{exemple}
\begin{exemple}\pjya{ki kɤ-nɤma ki nɤj nɤ-taʁ ɲɯ-ti-a ŋu nɤ!}\hspace{5pt}\pcmn{这件事情全靠你了}\end{exemple}
\begin{exemple}\pjya{ki kɤ-nɤma ki ɯʑo ɯ-taʁ ɲɯ-ti-a ntshi}\hspace{5pt}\pcmn{这件事情只好靠他了}\end{exemple}
\begin{exemple}\pjya{tɕhi nɯ mɤ-tɯ-nɯ-ti}\hspace{5pt}\pcmn{你什么话都说出来}\end{exemple}
\begin{exemple}\pjya{nɤ-kɤ-nɯ-ti ci ɣɤʑu}\hspace{5pt}\pcmn{你还好意思说}\end{exemple}\sens{2}\paradigme{dir}{thɯ-}\paradigme{dir}{pɯ-}\paradigme{past stem}{sɯtɯt}
\begin{définition}\pfra{annoncer}\end{définition}
\begin{définition}\pcmn{宣布}\end{définition}
\begin{exemple}\pjya{tɤ-kɤ-nɯkrɤz nɯ tɯrme ra nɯ-ɕki chɯ-tɯ-ti ra}\hspace{5pt}\pcmn{你要向人们宣布决议}\end{exemple}
\begin{sous-entrée}{sɯti}{ⓔtiⓢ2ⓝsɯti}\end{sous-entrée}

\begin{définition}\pfra{faire parler}\end{définition}
\begin{définition}\pcmn{使讲话;播放}\end{définition}
\begin{exemple}\pjya{kɯ-dɤn tsa kɤ-sɯti}\hspace{5pt}\pcmn{播放多一点}\end{exemple}
\begin{sous-entrée}{kɤti}{ⓔtiⓝkɤti}
\begin{définition}\pfra{on dirait que}\end{définition}
\begin{définition}\pcmn{看起来,表面上}\end{définition}
\begin{exemple}\pjya{pjɯ-rɤβzjoz kɤti ŋu}\hspace{5pt}\pcmn{他看起来是在读书}\end{exemple}\end{sous-entrée}

\begin{sous-entrée}{nɯɣɯti}{ⓔtiⓝnɯɣɯti} 
\classe{vs} 
\begin{définition}\pfra{facile à dire}\end{définition}
\begin{définition}\pcmn{容易说}\end{définition}\end{sous-entrée}

\end{entrée}

\begin{entrée}{toŋku}{}{ⓔtoŋku} 
\classe{n} 
\begin{définition}\pfra{casserole en cuivre}\end{définition}
\begin{définition}\pcmn{铜、生铁铸造的锅子【鼎锅】}\end{définition}\end{entrée}

\begin{entrée}{toŋkɤr}{}{ⓔtoŋkɤr} 
\classe{n} 
\begin{définition}\pfra{conque}\end{définition}
\begin{définition}\pcmn{螺}\end{définition}\étymologie{duŋ.dkar}\end{entrée}

\begin{entrée}{toŋtsi}{}{ⓔtoŋtsi} 
\classe{n} 
\begin{définition}\pfra{centime}\end{définition}
\begin{définition}\pcmn{一角}\end{définition}\étymologie{doŋ.rtse}\end{entrée}

\begin{entrée}{toʁde}{}{ⓔtoʁde} 
\classe{adv} 
\begin{définition}\pfra{moment}\end{définition}
\begin{définition}\pcmn{一会儿}\end{définition}\end{entrée}

\begin{entrée}{tsu}{}{ⓔtsu} 
\classe{vi} \sens{1}
\begin{définition}\pfra{avoir le temps}\end{définition}
\begin{définition}\pcmn{来得及}\end{définition}
\begin{exemple}\pjya{mɯ́j-tsu-a}\hspace{5pt}\pcmn{我来不及}\end{exemple}
\begin{exemple}\pjya{pɤjkhu tɤ-rʑaʁ ɣɤʑu, ɲɯ-tsu}\hspace{5pt}\pcmn{还有时间,还来得及}\end{exemple}\sens{2}\paradigme{dir}{tɤ-}
\begin{définition}\pfra{se passer ... (temps)}\end{définition}
\begin{définition}\pcmn{到(时间)}\end{définition}
\begin{exemple}\pjya{kɯmŋu-xpa pɯ-tsu}\hspace{5pt}\pcmn{已经过了五年}\end{exemple}
\begin{exemple}\pjya{kɯmŋu-xpa to-tsu}\hspace{5pt}\pcmn{到了五年}\end{exemple}
\begin{exemple}\pjya{kɯtʂɤ-sŋi ma mɯ-jɤ-tsu-a}\hspace{5pt}\pcmn{我只来了六天了}\end{exemple}\sens{3}\paradigme{dir}{pɯ-}
\begin{définition}\pfra{en arriver au point de ...}\end{définition}
\begin{définition}\pcmn{到了……的地步}\end{définition}
\begin{exemple}\pjya{ɲɤ-nɯzdɯɣ ʑo pjɤ-tsu}\hspace{5pt}\pcmn{已经到了非常担心的地步了}\end{exemple}
\begin{exemple}\pjya{nɤʑo ʑaʑa ʑo mɯ-jɤ-tɯ-ɤzɣɯt tɕe, nɯ-ta-nɯzdɯɣ ʑo pɯ-tsu}\hspace{5pt}\pcmn{你很久都不到,我已经到了很担心你的地步了}\end{exemple}
\begin{exemple}\pjya{jɤ-ɣe-a pɯ-tsu, mɤʑɯ ɲɯ-nɤrɯra-a ra}\hspace{5pt}\pcmn{我既然来了,我就再看一下}\end{exemple}\relationsémantique{参考}{\lien{ⓔatsɯtsu}{atsɯtsu}}\relationsémantique{参考}{\lien{ⓔsɤtsu}{sɤtsu}}\relationsémantique{参考}{\lien{ⓔsɯxtsuⓗ1}{sɯxtsu₁}}\end{entrée}

\begin{entrée}{tsa}{}{ⓔtsa} 
\classe{adv} 
\begin{définition}\pfra{plutôt, un peu}\end{définition}
\begin{définition}\pcmn{比较;稍微}\end{définition}
\begin{exemple}\pjya{nɤʑo ndi tsa nɯ-cit}\hspace{5pt}\pcmn{你稍微转过去一点}\end{exemple}
\begin{exemple}\pjya{khɯtsa ɯ-ŋgɯ tɯ-ci kɯ-dɤn tsa tɤ-rke}\hspace{5pt}\pcmn{你在碗里倒多一点水}\end{exemple}\étymologie{tsa}\end{entrée}

\begin{entrée}{tsaβ}{}{ⓔtsaβ} 
\classe{vs} 
\begin{définition}\pfra{fort (alcool)}\end{définition}
\begin{définition}\pcmn{浓度高(酒)}\end{définition}
\begin{exemple}\pjya{cha ɲɯ-tsaβ}\hspace{5pt}\pcmn{酒的浓度高}\end{exemple}
\begin{exemple}\pjya{ki tɕheme ki ɯ-sɯm sna ri, ɯ-mtɕhi tsaβ wo!}\hspace{5pt}\pcmn{这个女子心地好,但是嘴上很泼辣}\end{exemple}\relationsémantique{反义词}{\lien{ⓔmnu}{mnu}}\end{entrée}

\begin{entrée}{tsanla}{}{ⓔtsanla} 
\classe{n}  
\grammaire{n.lieu} 
\begin{définition}\pfra{Btsanlha}\end{définition}
\begin{définition}\pcmn{小金}\end{définition}\end{entrée}

\begin{entrée}{tsantʂa}{}{ⓔtsantʂa} 
\classe{n} 
\begin{définition}\pfra{if}\end{définition}
\begin{définition}\pcmn{红豆杉}\end{définition}\end{entrée}

\begin{entrée}{tsaŋga}{}{ⓔtsaŋga} 
\classe{n} 
\begin{définition}\pfra{corbeau (corvus dauuricus)}\end{définition}
\begin{définition}\pcmn{达乌里寒鸦}\end{définition}\end{entrée}

\begin{entrée}{tsaʁ}{}{ⓔtsaʁ} 
\classe{adv} 
\begin{définition}\pfra{au moins}\end{définition}
\begin{définition}\pcmn{至少,起码}\end{définition}
\begin{exemple}\pjya{tɤ-rɤru tɕe, tɕhi maʁ nɤ nɤ-rŋa tsaʁ pɯ-χtɕi ma}\hspace{5pt}\pcmn{你起床,起码把脸洗一下}\end{exemple}
\begin{exemple}\pjya{tɕhi maʁ nɤ tɯ-sŋi smɤn tɯ-ɣjɤn kɤ-ndza ra}\hspace{5pt}\pcmn{一天至少要吃一次药}\end{exemple}\end{entrée}

\begin{entrée}{tsɤndɤn}{}{ⓔtsɤndɤn} 
\classe{n} 
\begin{définition}\pfra{santal}\end{définition}
\begin{définition}\pcmn{檀木}\end{définition}\étymologie{tsan.dan}\end{entrée}

\begin{entrée}{tsɣaʁtsɣaʁ}{}{ⓔtsɣaʁtsɣaʁ} 
\classe{idph.2} 
\begin{définition}\pfra{rouge vif}\end{définition}
\begin{définition}\pcmn{红艳艳}\end{définition}
\begin{exemple}\pjya{mɯntoʁ ɲɯ-ɣɯrni tsɣaʁtsɣaʁ ʑo}\hspace{5pt}\pcmn{花红艳艳}\end{exemple}\relationsémantique{参考}{\lien{ⓔtɕɣɤrtɕɣɤr}{tɕɣɤrtɕɣɤr}}\end{entrée}

\begin{entrée}{tsɣi}{}{ⓔtsɣi} 
\classe{vi} \paradigme{dir}{pɯ-}\paradigme{dir}{nɯ-}\paradigme{dir}{pɯ-}
\begin{définition}\pfra{pourrir}\end{définition}
\begin{définition}\pcmn{腐烂}\end{définition}
\begin{définition}\pfra{laisser pourrir}\end{définition}
\begin{définition}\pcmn{让腐烂}\end{définition}
\begin{exemple}\pjya{@yangyu pjɤ-tsɣi}\hspace{5pt}\pcmn{洋芋腐烂了}\end{exemple}
\begin{exemple}\pjya{tɯ-ŋga pjɤ-tsɣi}\hspace{5pt}\pcmn{衣服烂了}\end{exemple}
\begin{exemple}\pjya{ɕoŋtɕa pjɤ-tsɣi}\hspace{5pt}\pcmn{木料腐烂了}\end{exemple}
\begin{exemple}\pjya{kɯki kɤ-ndza ki ʑa mɯ-tɤ-nɯβdaʁ-a tɕe, pjɤ-sɯtsɣi-t-a}\hspace{5pt}\pcmn{我很久没有管这个食物,它就腐烂了}\end{exemple}
\begin{sous-entrée}{sɯtsɣi}{ⓔtsɣiⓝsɯtsɣi} 
\classe{vt}  
\grammaire{caus} \end{sous-entrée}

\end{entrée}

\begin{entrée}{tshu}{}{ⓔtshu} 
\classe{vs} \paradigme{dir}{kɤ-}\paradigme{dir}{thɯ-}\paradigme{dir}{thɯ-}
\begin{définition}\pfra{gros}\end{définition}
\begin{définition}\pcmn{胖}\end{définition}
\begin{définition}\pfra{faire grossir}\end{définition}
\begin{définition}\pcmn{催肥;令……变胖}\end{définition}
\begin{exemple}\pjya{paʁ ko-tshu}\hspace{5pt}\pcmn{猪变胖了}\end{exemple}
\begin{exemple}\pjya{skɤm pjɤ-tshu}\hspace{5pt}\pcmn{菜牛是胖的}\end{exemple}
\begin{exemple}\pjya{nɤ-tɕɯ ɯ-ɲɯ́-tshu?}\hspace{5pt}\pcmn{你儿子胖不胖?}\end{exemple}
\begin{exemple}\pjya{paʁ kɤ-skaɣ-a tɕe thɯ-sɯxtshu-t-a}\hspace{5pt}\pcmn{我把猪喂得很肥了}\end{exemple}
\begin{sous-entrée}{sɯxtshu}{ⓔtshuⓝsɯxtshu} 
\classe{vt} \end{sous-entrée}

\begin{sous-entrée}{zɣɤsɯxtshu}{ⓔtshuⓝzɣɤsɯxtshu} 
\classe{vi} 
\begin{définition}\pfra{se faire grossir}\end{définition}
\begin{définition}\pcmn{令自己变胖}\end{définition}
\begin{exemple}\pjya{thamtham kɯ-xtɕi ra koŋla mɯ-chɯ-rɯndzɤtshi-nɯ tɕe, mɯ-chɯ-ʑɣɤsɯxtshu-nɯ ɲɯ-ŋu}\hspace{5pt}\pcmn{现在年轻女子吃得少,是为了不要令自己变胖}\end{exemple}\end{sous-entrée}

\end{entrée}

\begin{entrée}{tsha}{}{ⓔtsha} 
\classe{n} 
\begin{définition}\pfra{sel}\end{définition}
\begin{définition}\pcmn{盐}\end{définition}\étymologie{tsʰwa}\end{entrée}

\begin{entrée}{tshajqajɯ}{}{ⓔtshajqajɯ} 
\classe{n} 
\begin{définition}\pfra{puceron}\end{définition}
\begin{définition}\pcmn{蚜虫}\end{définition}\end{entrée}

\begin{entrée}{tshala}{}{ⓔtshala} 
\classe{n} 
\begin{définition}\pfra{soudage}\end{définition}
\begin{définition}\pcmn{(焊接用的)铁水}\end{définition}
\begin{exemple}\pjya{tshala ka-lɤt}\hspace{5pt}\pcmn{他(把东西)焊了。}\end{exemple}\étymologie{tsʰa.la}\end{entrée}

\begin{entrée}{tshaŋ}{}{ⓔtshaŋ} 
\classe{n} 
\begin{définition}\pfra{armoire}\end{définition}
\begin{définition}\pcmn{柜子}\end{définition}\relationsémantique{同义词}{\lien{ⓔwaŋtshaŋ}{waŋtshaŋ}}\end{entrée}

\begin{entrée}{tshaŋlaŋ}{}{ⓔtshaŋlaŋ} 
\classe{n} 
\begin{définition}\pfra{clochette}\end{définition}
\begin{définition}\pcmn{铃}\end{définition}\end{entrée}

\begin{entrée}{tshapa}{}{ⓔtshapa} 
\classe{n}  
\grammaire{n.lieu} 
\begin{définition}\pfra{l'un des hameaux de Gyutshapa}\end{définition}
\begin{définition}\pcmn{二茶村的大队之一}\end{définition}\end{entrée}

\begin{entrée}{tshaʁ}{}{ⓔtshaʁ} 
\classe{n} 
\begin{définition}\pfra{crible}\end{définition}
\begin{définition}\pcmn{筛子}\end{définition}
\begin{exemple}\pjya{ki tɤɕi ki tshaʁ pɯ-lat-a}\hspace{5pt}\pcmn{我筛了青稞}\end{exemple}
\begin{exemple}\pjya{tshaʁ ɯ-mɲaʁ}\hspace{5pt}\pcmn{筛子的漏孔,眼子}\end{exemple}\relationsémantique{参考}{\lien{ⓔsɯxtshaʁ}{sɯxtshaʁ}}\étymologie{tsʰags}\end{entrée}

\begin{entrée}{tshatsha}{}{ⓔtshatsha} 
\classe{n} 
\begin{définition}\pfra{impressions sur argile}\end{définition}
\begin{définition}\pcmn{(用模型印出来的)小泥像}\end{définition}\relationsémantique{参考}{\lien{ⓔtshɤcɯm}{tshɤcɯm}}\relationsémantique{参考}{\lien{ⓔtshɤrkɯ}{tshɤrkɯ}}\étymologie{tsʰa.tsʰa}\end{entrée}

\begin{entrée}{tshawa}{}{ⓔtshawa} 
\classe{n} 
\begin{définition}\pfra{tuberculose}\end{définition}
\begin{définition}\pcmn{肺结核}\end{définition}\étymologie{tsʰa.ba}\end{entrée}

\begin{entrée}{tshɤcɯm}{}{ⓔtshɤcɯm} 
\classe{n} 
\begin{définition}\pfra{petite maison où l'on met les tshatsha}\end{définition}
\begin{définition}\pcmn{装泥像用的小房子}\end{définition}\relationsémantique{参考}{\lien{ⓔtshatsha}{tshatsha}}\étymologie{tsʰa.kʰʲim?}\end{entrée}

\begin{entrée}{tshɤdɯɣ}{}{ⓔtshɤdɯɣ} 
\classe{n} 
\begin{définition}\pfra{chaleur}\end{définition}
\begin{définition}\pcmn{天气热}\end{définition}\relationsémantique{参考}{\lien{ⓔɣɯtshɤdɯɣ}{ɣɯtshɤdɯɣ}}\relationsémantique{参考}{\lien{ⓔnɯtshɤdɯɣ}{nɯtshɤdɯɣ}}\relationsémantique{反义词}{\lien{ⓔtɤndʐo}{tɤndʐo}}\end{entrée}

\begin{entrée}{tshɤko}{}{ⓔtshɤko} 
\classe{n} 
\begin{définition}\pfra{tas de pierre, symbole bouddhique}\end{définition}
\begin{définition}\pcmn{玛尼堆}\end{définition}\end{entrée}

\begin{entrée}{tshɤmbɤr}{}{ⓔtshɤmbɤr} 
\classe{n} 
\begin{définition}\pfra{grande lampe à beurre}\end{définition}
\begin{définition}\pcmn{最大的酥油灯}\end{définition}\end{entrée}

\begin{entrée}{tshɤmdzu}{}{ⓔtshɤmdzu} 
\classe{n} 
\begin{définition}\pfra{espèce d'arbrisseau}\end{définition}
\begin{définition}\pcmn{灌木的一种}\end{définition}
\begin{exemple}\pjya{tshɤmdzu nɯ si ci ŋu, wuma ʑo mbɤr, ɯ-mdzu rɲɟi cho mtɕoʁ, ɯ-jwaʁ ɯ-βzɯr ra kɯnɤ ɯ-mdzu tu, ɯ-zrɤm nɯ kɯ-qarŋe kɯ-qiaβ ŋu tɕe tɯ-xtu kɯ-mŋɤm ɯ-smɤn ɯ-spa ŋu}\hspace{5pt}\pcmn{\lien{ⓔtshɤmdzu}{tshɤmdzu}是一种树,比较矮,刺长而尖,叶子边缘也有刺,根是黄色的,带有苦味,是治拉肚子的药材原料之一。}\end{exemple}\end{entrée}

\begin{entrée}{tshɤndʐi}{}{ⓔtshɤndʐi} 
\classe{n} 
\begin{définition}\pfra{peau de chèvre}\end{définition}
\begin{définition}\pcmn{山羊皮}\end{définition}\relationsémantique{参考}{\lien{ⓔtshɤtⓗ2}{tshɤt₂}}\relationsémantique{参考}{\lien{ⓔtɯ-ndʐi}{tɯ-ndʐi}}\end{entrée}

\begin{entrée}{tshɤnmu}{}{ⓔtshɤnmu} 
\classe{n} 
\begin{définition}\pfra{chèvre}\end{définition}
\begin{définition}\pcmn{母山羊}\end{définition}\relationsémantique{参考}{\lien{ⓔtshɤtⓗ2}{tshɤt₂}}\end{entrée}

\begin{entrée}{tshɤɲcɤnɯ}{}{ⓔtshɤɲcɤnɯ} 
\classe{n} 
\begin{définition}\pfra{une espèce de champignon}\end{définition}
\begin{définition}\pcmn{【刷把菌】}\end{définition}
\begin{exemple}\pjya{tshɤɲcɤnɯ nɯ ɕkrɤz ɯ-ŋgɯ tɯrgi ɯ-ŋgɯ tu-ɬoʁ ŋu, ɯ-mdoʁ nɯ kɯ-qarŋe tu, ɕa ɯ-mdoʁ tu, ɯ-tshɯɣa nɯ zɣɤmbu fse, ɯ-ku nɯ tɯ-jaʁndzu kɯ-fse kɯ-xtshɯm-xtshɯm ʑo ŋu, kɤ-ndza sna}\hspace{5pt}\pcmn{刷把菌长在青冈树林里,有的是黄色的,有的颜色像人的皮肤一样,形状像扫把,尖端像细小的手指,能吃。}\end{exemple}\end{entrée}

\begin{entrée}{tshɤrkɯ}{}{ⓔtshɤrkɯ} 
\classe{n} 
\begin{définition}\pfra{moule pour tshatsha}\end{définition}
\begin{définition}\pcmn{泥像的模型}\end{définition}\relationsémantique{参考}{\lien{ⓔtshatsha}{tshatsha}}\end{entrée}

\begin{entrée}{tshɤrme}{}{ⓔtshɤrme} 
\classe{n} 
\begin{définition}\pfra{poil de chèvre}\end{définition}
\begin{définition}\pcmn{山羊毛}\end{définition}\relationsémantique{参考}{\lien{ⓔtshɤtⓗ2}{tshɤt₂}}\relationsémantique{参考}{\lien{ⓔtɤ-rme}{tɤ-rme}}\end{entrée}

\begin{entrée}{tshɤrqhu}{}{ⓔtshɤrqhu} 
\classe{n} 
\begin{définition}\pfra{natte}\end{définition}
\begin{définition}\pcmn{席子}\end{définition}\end{entrée}

\begin{entrée}{tshɤrtɯl}{}{ⓔtshɤrtɯl} 
\classe{n} 
\begin{définition}\pfra{habit en peau d'agneau}\end{définition}
\begin{définition}\pcmn{羔羊皮袄}\end{définition}\end{entrée}

\begin{entrée}{tshɤrɯ}{}{ⓔtshɤrɯ} 
\classe{n} 
\begin{définition}\pfra{veste de peau d'agneau}\end{définition}
\begin{définition}\pcmn{羔羊皮袄}\end{définition}\end{entrée}

\begin{entrée}{tshɤʁrɯ}{}{ⓔtshɤʁrɯ} 
\classe{n} 
\begin{définition}\pfra{corne de chèvre}\end{définition}
\begin{définition}\pcmn{山羊角}\end{définition}\relationsémantique{参考}{\lien{ⓔtshɤtⓗ2}{tshɤt₂}}\relationsémantique{参考}{\lien{ⓔta-ʁrɯ}{ta-ʁrɯ}}\end{entrée}

\begin{entrée}{tshɤt}{₂}{ⓔtshɤtⓗ2} 
\classe{n} 
\begin{définition}\pfra{chèvre}\end{définition}
\begin{définition}\pcmn{山羊}\end{définition}\relationsémantique{参考}{\lien{ⓔtshɤnmu}{tshɤnmu}}\end{entrée}

\begin{entrée}{tshɤt}{₁}{ⓔtshɤtⓗ1} 
\classe{vt} \paradigme{dir}{tɤ-}\paradigme{dir}{kɤ-}\sens{1}
\begin{définition}\pfra{essayer}\end{définition}
\begin{définition}\pcmn{试}\end{définition}
\begin{exemple}\pjya{nɤ-tsa ɯ-ɲɯ-βze tɤ-tshɤt}\hspace{5pt}\pcmn{你试一下适不适合你}\end{exemple}
\begin{exemple}\pjya{@xingqiwu a-kɤ-tɯ-tshɤt}\hspace{5pt}\pcmn{你星期五试一下吧}\end{exemple}\sens{2}
\begin{définition}\pfra{tester}\end{définition}
\begin{définition}\pcmn{考验}\end{définition}\relationsémantique{参考}{\lien{ⓔrɤtshɤt}{rɤtshɤt}}\end{entrée}

\begin{entrée}{tshɤthɤr}{}{ⓔtshɤthɤr} 
\classe{n} 
\begin{définition}\pfra{libérer des animaux vivant}\end{définition}
\begin{définition}\pcmn{放生}\end{définition}
\begin{exemple}\pjya{tshɤthɤr lo-lɤt}\hspace{5pt}\pcmn{他放生了(某种生物)}\end{exemple}\étymologie{tsʰe.tʰar}\end{entrée}

\begin{entrée}{tshɤtʂot}{}{ⓔtshɤtʂot} 
\classe{n} 
\begin{définition}\pfra{fièvre}\end{définition}
\begin{définition}\pcmn{发烧}\end{définition}\relationsémantique{参考}{\lien{ⓔnɯtshɤtʂot}{nɯtshɤtʂot}}\étymologie{tsʰa.drod}\end{entrée}

\begin{entrée}{tshɤwɤre}{}{ⓔtshɤwɤre} 
\classe{n} 
\begin{définition}\pfra{lézard}\end{définition}
\begin{définition}\pcmn{壁虎}\end{définition}
\begin{exemple}\pjya{tshɤwɤre nɯ praʁ ɯ-rchɤβ, zndɤrchɤβ ra ku-rɤʑi ɲɯ-ŋu, qajɯ kɯ-wxti tsa ci ɲɯ-ŋu, ɯ-phoŋbu alɯlju, ɯ-jme nɯ ɲɯ-ɤɕpɯɕpa tɕe, ɯ-phaʁ βzɯr nɯ ɯ-thoʁ pjɯ-tɯɣ ŋu, ɯ-mɤlɤjaʁ ɣɤʑu, ɯ-mdoʁ nɯ qapri mdoʁ cho naχtɕɯɣ, ɯʑo stu ʑo kɤ-cha nɯ ju-mtsaʁ tɕe qapri ɲɯ-prɤt, tɕe li tɯ-jwaʁ ci z-ɲɯ-ɕar tɕe, qapri na-prɤt ɯ-stu nɯ tɕu ku-tshoʁ tɕe qapri li ku-sɤndɯ-ndɤm ɲɯ-cha, tɕe qapri tu-nɯsmɤn ɲɯ-cha.}\hspace{5pt}\pcmn{壁虎生活在岩石缝里,墙壁缝里,是一种大虫。身子是圆柱形的,尾巴扁,尾巴的棱边着地。有四只脚,颜色和蛇一样。(传说)壁虎从蛇背跳过时会使蛇断裂,它最大的本事就是去找一片草叶,夹在蛇断裂的部位,可以把蛇的两段接起来,把蛇治好。}\end{exemple}\end{entrée}

\begin{entrée}{tshɤχɕaŋ}{}{ⓔtshɤχɕaŋ} 
\classe{n} 
\begin{définition}\pfra{branche d'arbre que l'on met sur le toit après les cérémonie religieuse}\end{définition}
\begin{définition}\pcmn{念完经后插在屋顶上的树枝}\end{définition}
\begin{exemple}\pjya{tshɤχɕaŋ nɯ pakuku ʑo nɯ-kɯ-ɣɤrpi tɕe pjɯ-tu ra, sɤjku ɣɯ ɯ-rtaʁ ɯ-rtsimu kɯ-βdi, ɯ-rtaʁ ra ɯ-tɯ-nɯgrɤl kɯ-ɤmɲɤm tsa pjɯ-ŋu ra, ɯ-rtaʁ ɣɯ ɯ-mujmaj raŋri nɯ tɕu qarma muj kɯ-wɣrum nɯ maʁ nɤ smɤɣ kɯ-wɣrum tɯ-sna ntsɯ kú-wɣ-sthoʁ. tɕe tɤ-rpi ɲɯ-jɤɣ tɕe, kha ɣɯ lɤftsɤz ɯ-taʁ pjɯ́-wɣ-sɤtsa ra, kɯ-ɤrqhi ju-kɯ-ru tɕe, mɯntoʁ kɯ-wɣrum nɯ-kɯ-lɤt fse.}\hspace{5pt}\pcmn{每年请和尚来家里念经时,必须要有\lien{ⓔtshɤχɕaŋ}{tshɤχɕaŋ},必须是长得美观的白桦树枝,枝桠排列得比较均匀的,在每一根枝桠顶端上都要扎上白马鸡的羽毛或者白羊毛。念完经后就要插在房背上的白石头堆中,从远方看,像盛开的白花。}\end{exemple}\end{entrée}

\begin{entrée}{tshɤz}{}{ⓔtshɤz} 
\classe{vs} 
\begin{définition}\pfra{frais et tendre}\end{définition}
\begin{définition}\pcmn{新鲜;清脆(吃起来很脆)}\end{définition}
\begin{exemple}\pjya{ki @yangyu ki ɲɯ-tshɤz tɕe ɲɯ-mɯm}\hspace{5pt}\pcmn{这个洋芋又新鲜又好吃}\end{exemple}\end{entrée}

\begin{entrée}{tshuβdɯn}{}{ⓔtshuβdɯn} 
\classe{n}  
\grammaire{n.lieu} 
\begin{définition}\pfra{Tshobdun}\end{définition}
\begin{définition}\pcmn{草登乡}\end{définition}\end{entrée}

\begin{entrée}{tshi}{₄}{ⓔtshiⓗ4} 
\classe{pro} 
\begin{définition}\pfra{quoi}\end{définition}
\begin{définition}\pcmn{什么}\end{définition}
\begin{exemple}\pjya{tshi tú-wɣ-pa ɲɯ-ra?}\hspace{5pt}\pcmn{应该怎么办?}\end{exemple}\relationsémantique{同义词}{\lien{ⓔtɕhiⓗ1}{tɕhi₁}}\relationsémantique{参考}{\lien{ⓔtshitsuku}{tshitsuku}}\end{entrée}

\begin{entrée}{tshi}{₁}{ⓔtshiⓗ1} 
\classe{vt} \paradigme{dir}{kɤ-}\paradigme{dir}{pɯ-}
\begin{définition}\pfra{boire}\end{définition}
\begin{définition}\pcmn{喝}\end{définition}
\begin{exemple}\pjya{tɯ-ci ka-tshi}\hspace{5pt}\pcmn{他喝了水}\end{exemple}
\begin{exemple}\pjya{cha pɯ-asɯ-tshi-j}\hspace{5pt}\pcmn{我们在喝酒}\end{exemple}
\begin{exemple}\pjya{chɤmda pɯ-tshi-t-a}\hspace{5pt}\pcmn{我喝了坛坛酒}\end{exemple}
\begin{exemple}\pjya{tʂha ku-tshi-tɕi pɯ-ŋu}\hspace{5pt}\pcmn{我们俩在喝茶(原来)}\end{exemple}\relationsémantique{参考}{\lien{ⓔjtshi}{jtshi}}
\begin{sous-entrée}{nɯɣɯtshi}{ⓔtshiⓗ1ⓝnɯɣɯtshi} 
\classe{vs} 
\begin{définition}\pfra{agréable à boire}\end{définition}
\begin{définition}\pcmn{喝着爽口}\end{définition}
\begin{exemple}\pjya{ki tʂha ki wuma ɲɯ-nɯɣɯtshi}\hspace{5pt}\pcmn{这个茶喝着很爽口}\end{exemple}\end{sous-entrée}

\end{entrée}

\begin{entrée}{tshi}{₂}{ⓔtshiⓗ2} 
\classe{vt} \paradigme{dir}{nɯ-}
\begin{définition}\pfra{étrangler}\end{définition}
\begin{définition}\pcmn{勒住}\end{définition}
\begin{exemple}\pjya{ɯ-mke ɲɤ-tshi}\hspace{5pt}\pcmn{他把他的脖子勒住了(横着)}\end{exemple}
\begin{exemple}\pjya{ɯ-mke to-tshi}\hspace{5pt}\pcmn{他把他吊死了}\end{exemple}\relationsémantique{参考}{\lien{ⓔʑɣɤtshi}{ʑɣɤtshi}}\end{entrée}

\begin{entrée}{tshi}{₃}{ⓔtshiⓗ3} 
\classe{vt} \paradigme{dir}{\_}
\begin{définition}\pfra{bloquer}\end{définition}
\begin{définition}\pcmn{挡住,拦住}\end{définition}
\begin{exemple}\pjya{ɕkom nɯ jɤ-tshi}\hspace{5pt}\pcmn{你把麂子拦住吧!}\end{exemple}
\begin{exemple}\pjya{fsapaʁ nɯ-tshi}\hspace{5pt}\pcmn{你把牲畜拦住吧!}\end{exemple}
\begin{exemple}\pjya{ʑmbri to-wxti tɕe qale ju-tshi, mɯ-ju-sɯɣe ɲɯ-cha}\hspace{5pt}\pcmn{杨树长出高了以后可以挡风}\end{exemple}\relationsémantique{参考}{\lien{ⓔnɤtʂɤtshi}{nɤtʂɤtshi}}\end{entrée}

\begin{entrée}{tshitsuku}{}{ⓔtshitsuku} 
\classe{pro} 
\begin{définition}\pfra{quoi que ce soit}\end{définition}
\begin{définition}\pcmn{无论什么,一切}\end{définition}\end{entrée}

\begin{entrée}{tshjencɯ}{}{ⓔtshjencɯ} 
\classe{n} 
\begin{définition}\pfra{couteau}\end{définition}
\begin{définition}\pcmn{猎刀}\end{définition}\end{entrée}

\begin{entrée}{tshoŋ}{}{ⓔtshoŋ} 
\classe{n} 
\begin{définition}\pfra{commerce}\end{définition}
\begin{définition}\pcmn{生意}\end{définition}\étymologie{tsʰoŋ}\end{entrée}

\begin{entrée}{tshoŋpawa}{}{ⓔtshoŋpawa} 
\classe{n} 
\begin{définition}\pfra{marchand}\end{définition}
\begin{définition}\pcmn{老板;商人}\end{définition}\étymologie{tsʰoŋ.pa.ba}\end{entrée}

\begin{entrée}{tshoŋwa}{}{ⓔtshoŋwa} 
\classe{n} 
\begin{définition}\pfra{marchand}\end{définition}
\begin{définition}\pcmn{商人}\end{définition}\étymologie{tsʰoŋ.ba}\end{entrée}

\begin{entrée}{tshoʁ}{}{ⓔtshoʁ} 
\classe{vt}
\classe{n}
\classe{vt} \sens{1}\paradigme{dir}{kɤ-}
\begin{définition}\pfra{attacher sur, mettre sur}\end{définition}
\begin{définition}\pcmn{带上,放在……上}\end{définition}
\begin{exemple}\pjya{khɯna ɯ-mke tɤmɯmɯm ko-tshoʁ}\hspace{5pt}\pcmn{他在狗的脖子上拴了铃铛}\end{exemple}
\begin{exemple}\pjya{@dianlu kɤ-nɯ-tshoʁ-i}\hspace{5pt}\pcmn{我们装了电炉}\end{exemple}\sens{2}\paradigme{dir}{pɯ-}\paradigme{dir}{pɯ-}\paradigme{dir}{thɯ-}
\begin{définition}\pfra{faire des fruits}\end{définition}
\begin{définition}\pcmn{结(果子)}\end{définition}
\begin{définition}\pfra{s'agenouiller}\end{définition}
\begin{définition}\pcmn{跪下}\end{définition}
\begin{définition}\pfra{faire s'agenouiller}\end{définition}
\begin{définition}\pcmn{使……跪下}\end{définition}
\begin{définition}\pfra{lâcher les chiens}\end{définition}
\begin{définition}\pcmn{放狗}\end{définition}
\begin{exemple}\pjya{paχɕi kɯ ɯ-mat ko-tshoʁ}\hspace{5pt}\pcmn{苹果树结了果}\end{exemple}
\begin{exemple}\pjya{ʑɴɢɯloʁ kɯ ɯ-mat ko-tshoʁ}\hspace{5pt}\pcmn{核桃结了果}\end{exemple}
\begin{exemple}\pjya{a-χpɯm pɯ-tshoʁ-a}\hspace{5pt}\pcmn{我跪下了}\end{exemple}
\begin{exemple}\pjya{khɯna thɯ-tshoʁ-i}\hspace{5pt}\pcmn{我们放了狗}\end{exemple}\relationsémantique{Component 1}{\lien{ⓔkhɯna}{khɯna}}\relationsémantique{Component 2}{\lien{ⓔtshoʁ}{tshoʁ}}\relationsémantique{同义词}{\lien{ⓔɣɯkhɯtshoʁ}{ɣɯkhɯtshoʁ}}
\begin{sous-entrée}{tɯ-χpɯm,tshoʁ}{ⓔtshoʁⓢ2ⓝtɯ-χpɯm,tshoʁ}\end{sous-entrée}

\begin{sous-entrée}{tɯ-χpɯm,sɯtshoʁ}{ⓔtshoʁⓢ2ⓝtɯ-χpɯm,sɯtshoʁ}\end{sous-entrée}

\begin{sous-entrée}{khɯna,tshoʁ}{ⓔtshoʁⓢ2ⓝkhɯna,tshoʁ}\end{sous-entrée}

\begin{sous-entrée}{atshoʁ}{ⓔtshoʁⓢ2ⓝatshoʁ} 
\classe{vi} 
\begin{définition}\pfra{être attaché}\end{définition}
\begin{définition}\pcmn{附着}\end{définition}\relationsémantique{参考}{\lien{ⓔndzoʁ}{ndzoʁ}}\end{sous-entrée}

\end{entrée}

\begin{entrée}{tshoʁɕaŋ}{}{ⓔtshoʁɕaŋ} 
\classe{n} 
\begin{définition}\pfra{décoration faite avec des branches et des plumes de crossoptilon}\end{définition}
\begin{définition}\pcmn{用树枝和白马鸡的羽毛做成的装饰}\end{définition}\end{entrée}

\begin{entrée}{tshoz}{}{ⓔtshoz} 
\classe{vi} \paradigme{dir}{tɤ-}
\begin{définition}\pfra{être au complet}\end{définition}
\begin{définition}\pcmn{齐全(零散的东西,几个人)}\end{définition}
\begin{exemple}\pjya{a-rɣe pɯ-nɯ-prat-a tɕe, pɯ-ʁndɤr tɕeri tɤ-wum-a tɕe nɯ ʁo tɤ-tshoz}\hspace{5pt}\pcmn{我把珠子弄断,撒在地上了捡了以后倒是齐全的}\end{exemple}
\begin{exemple}\pjya{kɯki jɯɣi ki ɲɯ-tshoz}\hspace{5pt}\pcmn{这本书是完整的}\end{exemple}\relationsémantique{同义词}{\lien{ⓔndzɯr}{ndzɯr}}
\begin{sous-entrée}{ɣɤtshoz}{ⓔtshozⓝɣɤtshoz} 
\classe{vt} 
\begin{définition}\pfra{rendre complet}\end{définition}
\begin{définition}\pcmn{使……齐全}\end{définition}
\begin{exemple}\pjya{nɤ-kɯ-ra ra kɤ-ɕar kɤ-ɣɤtshoz pɯ-cha-a}\hspace{5pt}\pcmn{我找到了你所有需要的东西}\end{exemple}\relationsémantique{参考}{\lien{ⓔsɯxtshoz}{sɯxtshoz}}\end{sous-entrée}

\étymologie{tsʰaŋs}\end{entrée}

\begin{entrée}{tshupa}{}{ⓔtshupa} 
\classe{n} 
\begin{définition}\pfra{village}\end{définition}
\begin{définition}\pcmn{村子}\end{définition}\étymologie{tsʰo.pa}\end{entrée}

\begin{entrée}{tshɯ}{}{ⓔtshɯ} 
\classe{vt} \paradigme{dir}{\_}
\begin{définition}\pfra{s'accommoder de ce qu'il y a}\end{définition}
\begin{définition}\pcmn{去一个地方使用本地的东西(不带自己的东西)、将就}\end{définition}
\begin{exemple}\pjya{kɤ-tshɯ mɯ́j-khɯ ma kutɕu maŋe tɕe, kɤ-nɯɣɯt pɯ-ra}\hspace{5pt}\pcmn{没有办法使用本地的东西,只好带过来了}\end{exemple}
\begin{exemple}\pjya{kutɕu ɣɯ kɤndza nɯra kɤ-tshɯ-t-a mɯ-kɤ-nɯɣɯt-a}\hspace{5pt}\pcmn{我没有把自己地方的食物带来,我吃了这里的食物}\end{exemple}\end{entrée}

\begin{entrée}{tshɯɣru}{}{ⓔtshɯɣru} 
\classe{n} 
\begin{définition}\pfra{soude; alcali}\end{définition}
\begin{définition}\pcmn{碱}\end{définition}\end{entrée}

\begin{entrée}{tshɯmɕtʂat}{}{ⓔtshɯmɕtʂat} 
\classe{n} 
\begin{définition}\pfra{capacité à économiser}\end{définition}
\begin{définition}\pcmn{节约开支}\end{définition}
\begin{exemple}\pjya{kɤndzɤtshi tshɯmɕtʂat ɯ-kɯ-βzu nɯ kha tɤ-mu nɯ ŋu}\hspace{5pt}\pcmn{一家人的开支是家庭主妇计划的}\end{exemple}\end{entrée}

\begin{entrée}{tshɯntshɯn}{}{ⓔtshɯntshɯn} 
\classe{idph.2} 
\begin{définition}\pfra{en bon état}\end{définition}
\begin{définition}\pcmn{形容塌实、齐全的状态}\end{définition}
\begin{exemple}\pjya{laχtɕha nɯtɕu nɯ-ta-t-a tɕe, tshɯntshɯn ɲɯ-ɤta}\hspace{5pt}\pcmn{我把东西放在那里,没人会碰}\end{exemple}
\begin{exemple}\pjya{nɤ-ŋga tɤ-tɯ-ɕɯɴqoʁ ɯ-sta tshɯntshɯn ɲɯ-ɤta}\hspace{5pt}\pcmn{你把衣服挂在那里,没人会碰}\end{exemple}
\begin{exemple}\pjya{kha ra tɤ-rɤwum-a tshɯntshɯn ʑo}\hspace{5pt}\pcmn{我把家的东西收拾得很好}\end{exemple}\end{entrée}

\begin{entrée}{tshɯptshɯp}{}{ⓔtshɯptshɯp} 
\classe{idph.2} 
\begin{définition}\pfra{sensation de condensation (dans la brume)}\end{définition}
\begin{définition}\pcmn{形容空气中弥漫着潮气(水蒸气凝结)的感觉}\end{définition}
\begin{sous-entrée}{ɣɤtshɯptshɯp}{ⓔtshɯptshɯpⓝɣɤtshɯptshɯp} 
\classe{vs} 
\begin{exemple}\pjya{ɲɯ-ɣɤtshɯptshɯp}\hspace{5pt}\pcmn{空气中弥漫着潮气}\end{exemple}\relationsémantique{同义词}{\lien{ⓔtɕhɯβtɕhɯβ}{tɕhɯβtɕhɯβ}}\end{sous-entrée}

\end{entrée}

\begin{entrée}{tshɯrɟɯn}{}{ⓔtshɯrɟɯn} 
\classe{adv} 
\begin{définition}\pfra{souvent}\end{définition}
\begin{définition}\pcmn{经常}\end{définition}
\begin{exemple}\pjya{tshɯrɟɯn nɯɣi ɕti}\hspace{5pt}\pcmn{他经常回来}\end{exemple}\étymologie{tsʰe.rgʲun}\end{entrée}

\begin{entrée}{tshɯtho}{}{ⓔtshɯtho} 
\classe{n} 
\begin{définition}\pfra{cabri}\end{définition}
\begin{définition}\pcmn{山羊羔}\end{définition}\end{entrée}

\begin{entrée}{tshwi}{}{ⓔtshwi} 
\classe{n} 
\begin{définition}\pfra{teinture}\end{définition}
\begin{définition}\pcmn{染料}\end{définition}\étymologie{tsʰos}\end{entrée}

\begin{entrée}{tshuxtoʁ}{}{ⓔtshuxtoʁ} 
\classe{n} 
\begin{définition}\pfra{confiance}\end{définition}
\begin{définition}\pcmn{信用}\end{définition}
\begin{exemple}\pjya{kɯki tɯrme ki ɯ-tshuxtoʁ kɯ-tu ci ŋu}\hspace{5pt}\pcmn{他是个讲信用的人}\end{exemple}
\begin{exemple}\pjya{nɤʑo nɤ-tshuxtoʁ ɣɤʑu}\hspace{5pt}\pcmn{你讲信用}\end{exemple}
\begin{exemple}\pjya{aʑo ki tshuxtoʁ ɯ-ku rɤʑi-a ɕti}\hspace{5pt}\pcmn{我会守信用的}\end{exemple}
\begin{exemple}\pjya{nɤʑo tshuxtoʁ a-kɤ-tɯ-rɤʑi ra nɤ!}\hspace{5pt}\pcmn{你要讲信用}\end{exemple}\end{entrée}

\begin{entrée}{tsjaŋtsjaŋ}{}{ⓔtsjaŋtsjaŋ} 
\classe{idph.2} 
\begin{définition}\pfra{haut}\end{définition}
\begin{définition}\pcmn{身子高(比其他人高)}\end{définition}
\begin{exemple}\pjya{ɲɯ-mbro ʑo tsjaŋtsjaŋ}\hspace{5pt}\pcmn{他很高}\end{exemple}\relationsémantique{参考}{\lien{ⓔzjaŋzjaŋ}{zjaŋzjaŋ}}\relationsémantique{参考}{\lien{ⓔzjɤɣzjɤɣ}{zjɤɣzjɤɣ}}
\begin{sous-entrée}{tsjaŋnɤtsjaŋ}{ⓔtsjaŋtsjaŋⓝtsjaŋnɤtsjaŋ} 
\classe{idph.3} \end{sous-entrée}

\end{entrée}

\begin{entrée}{tsuku}{}{ⓔtsuku} 
\classe{n} 
\begin{définition}\pfra{certains}\end{définition}
\begin{définition}\pcmn{一些;某些}\end{définition}\end{entrée}

\begin{entrée}{tslɯɣtslɯɣ}{}{ⓔtslɯɣtslɯɣ} 
\classe{idph.2} \sens{1}
\begin{définition}\pfra{envelopper complètement}\end{définition}
\begin{définition}\pcmn{形容包成一团(成了圆形)、包得又紧又大状}\end{définition}
\begin{exemple}\pjya{ɯ-ku pjɤ-ɴɢraʁ tɕe, tslɯɣtslɯɣ to-mphɯr}\hspace{5pt}\pcmn{他的头破了,被包成一团了}\end{exemple}
\begin{exemple}\pjya{ɯ-jaʁ to-mphɯr tslɯɣtslɯɣ ʑo}\hspace{5pt}\pcmn{他把手包成一团了}\end{exemple}\sens{2}
\begin{définition}\pfra{rond et dur}\end{définition}
\begin{définition}\pcmn{形容又圆又硬的样子}\end{définition}
\begin{exemple}\pjya{a-rqo ci thɯci chɤ-ɕe tɕe tslɯɣtslɯɣ ʑo ɲɯ-ɤrku tɕe kɤ-tɕɤt mɯ́j-khɯ}\hspace{5pt}\pcmn{我喉咙里卡了个东西,取不出来}\end{exemple}
\begin{sous-entrée}{tslɯɣnɤtslɯɣ}{ⓔtslɯɣtslɯɣⓢ2ⓝtslɯɣnɤtslɯɣ} 
\classe{idph.3} 
\begin{définition}\pfra{bouillir en faisant de grosses bulles}\end{définition}
\begin{définition}\pcmn{水沸腾,翻滚的样子}\end{définition}\end{sous-entrée}

\begin{sous-entrée}{ɣɤtslɯɣtslɯɣ}{ⓔtslɯɣtslɯɣⓢ2ⓝɣɤtslɯɣtslɯɣ} 
\classe{vi} 
\begin{exemple}\pjya{tɯ-ci ɲɯ-ɤla ɲɯ-ɣɤtslɯɣtslɯɣ}\hspace{5pt}\pcmn{水沸腾了}\end{exemple}\end{sous-entrée}

\end{entrée}

\begin{entrée}{tso}{}{ⓔtso} 
\classe{vi-t} \paradigme{dir}{kɤ-}
\begin{définition}\pfra{comprendre, se rendre compte, savoir}\end{définition}
\begin{définition}\pcmn{懂;发觉;知道}\end{définition}
\begin{exemple}\pjya{hanɯni ci, mɤʑɯ kɯ-taʁ tsa a-kɤ-tso-a ɲɯ-ra}\hspace{5pt}\pcmn{我还需要弄清楚一点(你再讲一次)}\end{exemple}
\begin{exemple}\pjya{lonba kɤ-tso-a}\hspace{5pt}\pcmn{我完全懂了}\end{exemple}
\begin{exemple}\pjya{tu-kɯ-ti a-pɯ-ŋu tɕe, tɯrme ra ʑatsa mɤ-tso-nɯ}\hspace{5pt}\pcmn{如果说这句话,人家听不懂}\end{exemple}
\begin{exemple}\pjya{nɤj tɤ-tɯ-tɯt nɯ aj mɯ́j-tso-a}\hspace{5pt}\pcmn{我不明白你刚说的话}\end{exemple}
\begin{sous-entrée}{tso me}{ⓔtsoⓝtso me}\sens{1}
\begin{définition}\pfra{perdre conscience}\end{définition}
\begin{définition}\pcmn{失去知觉}\end{définition}
\begin{exemple}\pjya{tso ɲɤ-me-a tɕe pjɤ-nɯʑɯβ-a}\hspace{5pt}\pcmn{我不只知不觉地睡着了}\end{exemple}\end{sous-entrée}

\sens{2}\paradigme{dir}{kɤ-}
\begin{définition}\pfra{ne pas être en âge de comprendre}\end{définition}
\begin{définition}\pcmn{不懂事,幼稚}\end{définition}
\begin{définition}\pfra{faire comprendre}\end{définition}
\begin{définition}\pcmn{令人明白}\end{définition}
\begin{exemple}\pjya{ki tɤ-pɤtso ki pɤjkhu tso maŋe}\hspace{5pt}\pcmn{这个小孩子还不懂事}\end{exemple}
\begin{exemple}\pjya{tɯ-rju kɤ-sɯxtso-t-a}\hspace{5pt}\pcmn{我令他明白了这句话的意思}\end{exemple}
\begin{sous-entrée}{sɯxtso}{ⓔtsoⓢ2ⓝsɯxtso} 
\classe{vt} \end{sous-entrée}

\begin{sous-entrée}{ʑɣɤsɯxtso}{ⓔtsoⓢ2ⓝʑɣɤsɯxtso} 
\classe{vi}  
\grammaire{refl}
\grammaire{caus} 
\begin{définition}\pfra{faire en sorte de comprendre}\end{définition}
\begin{définition}\pcmn{让自己明白}\end{définition}
\begin{exemple}\pjya{tɯ-rju kɤ-ʑɣɤsɯxtso ra}\hspace{5pt}\pcmn{要弄明白这句话的意思}\end{exemple}
\begin{exemple}\pjya{tɤ-kɤ-tɯt nɯra kɤ-ʑɣɤsɯxtso ma tha nɤ-ɕɯ-kɤ-rɤfɕɤt me}\hspace{5pt}\pcmn{你要弄清他所讲的话,不然的话你转述不出来}\end{exemple}\end{sous-entrée}

\begin{sous-entrée}{sɤtso}{ⓔtsoⓢ2ⓝsɤtso} 
\classe{vs}  
\grammaire{deexp} 
\begin{définition}\pfra{être compréhensible}\end{définition}
\begin{définition}\pcmn{听起来清楚}\end{définition}
\begin{exemple}\pjya{tɕhi tu-ti ŋu mɤ-sɤtso wo?}\hspace{5pt}\pcmn{听不懂他在讲什么吗?}\end{exemple}
\begin{exemple}\pjya{ɲɯ-sɤtso}\hspace{5pt}\pcmn{很清楚}\end{exemple}\end{sous-entrée}

\end{entrée}

\begin{entrée}{tsrɤt}{}{ⓔtsrɤt} 
\classe{vt} \paradigme{dir}{lɤ-}\paradigme{dir}{\_}
\begin{définition}\pfra{allonger}\end{définition}
\begin{définition}\pcmn{伸长}\end{définition}
\begin{exemple}\pjya{ɯ-mke lo-tsrɤt}\hspace{5pt}\pcmn{他伸了脖子}\end{exemple}\end{entrée}

\begin{entrée}{tsri}{}{ⓔtsri} 
\classe{vs}  
\grammaire{trop} \paradigme{dir}{nɯ-}\paradigme{dir}{pɯ-}
\begin{définition}\pfra{salé}\end{définition}
\begin{définition}\pcmn{咸}\end{définition}
\begin{exemple}\pjya{tsha nɯ kɯ-tsri ŋu}\hspace{5pt}\pcmn{盐是咸的}\end{exemple}
\begin{exemple}\pjya{ɯ-tɯ-tsri ko-tɕhom}\hspace{5pt}\pcmn{太咸了}\end{exemple}
\begin{exemple}\pjya{ɯ-tɯ-tsri mɯ́j-rtaʁ}\hspace{5pt}\pcmn{不够咸}\end{exemple}
\begin{sous-entrée}{nɤtsri}{ⓔtsriⓝnɤtsri} 
\classe{vt} \end{sous-entrée}

\begin{définition}\pfra{trouver salé}\end{définition}
\begin{définition}\pcmn{觉得咸}\end{définition}
\begin{exemple}\pjya{pɯ-nɤtsri-t-a}\hspace{5pt}\pcmn{我觉得很咸}\end{exemple}\relationsémantique{参考}{\lien{ⓔɣɤtsri}{ɣɤtsri}}\end{entrée}

\begin{entrée}{tsɯm}{}{ⓔtsɯm} 
\classe{vt}  
\grammaire{vert} \paradigme{dir}{\_}\paradigme{dir}{\_}\paradigme{dir}{\_}\paradigme{dir}{\_}
\begin{définition}\pfra{emporter}\end{définition}
\begin{définition}\pcmn{拿走;带去}\end{définition}
\begin{définition}\pfra{faire emporter}\end{définition}
\begin{définition}\pcmn{令人带走}\end{définition}
\begin{définition}\pfra{emporter partout}\end{définition}
\begin{définition}\pcmn{带来带去}\end{définition}
\begin{exemple}\pjya{tɯrme ɣɯ ɯ-laχtɕha nɯ tɯʑo kɯ jú-wɣ-tsɯm ra ma kɤ-sɯtsɯm mɤ-pe ma a-pɯ-tshoz ra}\hspace{5pt}\pcmn{别人的东西要亲自送过去,不要请人,因为要齐全}\end{exemple}
\begin{sous-entrée}{sɯtsɯm}{ⓔtsɯmⓝsɯtsɯm} 
\classe{vt} \end{sous-entrée}

\begin{sous-entrée}{nɤtsɯtsɯm/\variante{nɤtsɯmtsɯm}}{ⓔtsɯmⓝnɤtsɯtsɯm} 
\classe{vt} \end{sous-entrée}

\begin{sous-entrée}{nɯtsɯm}{ⓔtsɯmⓝnɯtsɯm} 
\classe{vt} \end{sous-entrée}

\begin{définition}\pfra{emporter (chez soi)}\end{définition}
\begin{définition}\pcmn{带回家}\end{définition}\end{entrée}

\begin{entrée}{tsɯmnɤtsɯm}{}{ⓔtsɯmnɤtsɯm} 
\classe{idph.3} 
\begin{définition}\pfra{scintillant}\end{définition}
\begin{définition}\pcmn{形容星星点点的闪光}\end{définition}
\begin{exemple}\pjya{tɕheme ɯ-rte ɯ-taʁ kɯ-nɤmbju tsɯmnɤtsɯm ʑo ɲɯ-pa}\hspace{5pt}\pcmn{妇女的帽子一闪一闪地发光}\end{exemple}
\begin{sous-entrée}{tsɯmɯtsami}{ⓔtsɯmnɤtsɯmⓝtsɯmɯtsami} 
\classe{idph.8} 
\begin{exemple}\pjya{ɯ-rŋa ra qame tsɯmɯtsami ɣɤʑu}\hspace{5pt}\pcmn{他满脸都是黑痣}\end{exemple}\end{sous-entrée}

\end{entrée}

\begin{entrée}{tsɯntu}{}{ⓔtsɯntu} 
\classe{n} 
\begin{définition}\pfra{ciseaux}\end{définition}
\begin{définition}\pcmn{剪刀}\end{définition}\end{entrée}

\begin{entrée}{tsɯʁot}{}{ⓔtsɯʁot} 
\classe{n} 
\begin{définition}\pfra{faisan (phasianus colchicus)}\end{définition}
\begin{définition}\pcmn{雉鸡【野鸡】}\end{définition}
\begin{exemple}\pjya{tsɯʁot nɯ pɣa ci ŋu, khro mɤ-wxti, kumpɣa jamar tu, phu nɯ wuma ʑo mpɕɤr, ɯ-jme rɲɟi, ɯ-ku kɯ-ɲaʁ ŋu, ɯ-taʁ ldʑaŋkɯ kɯ-nɤmbju tu, ɯ-phoŋbu kɯ-ɣɯrni, kɯ-qarŋe, ldʑaŋkɯ kɯ-ɲaʁ nɯ ra kɯ-ɤtʂoʁloʁ tu, tɕe wuma ʑo mpɕɤr, ɯ-mi aqarŋɯrŋe tsa ŋu, tɕe mpɕɤr, tsɯʁot mu nɯ kɯ-pɣi tsa ŋu tɕe nɯ mɤ-mpɕɤr. kɯ-mbro ʑo ɲɯ-nɯqambɯmbjom mɤ-cha. sɯku jamar ma tu-ɕe mɤ-cha. sɯmat, tɤ-rɤku, qajɯ nɯ ra tu-ndze ŋu. ɯʑo ɯ-kɯ-ndza dɤn. qandʑɣi, qaliaʁ, xɕiri, khɯna, lɯlu nɯnɯ ra kɯ tú-wɣ-ndza ɕti, ɯʑo tɯ-ji ɯ-rkɯ kɤ-nɤru wuma ʑo χɕu. tsɯʁot nɯ tɯ-ji ɯ-ŋgɯ sɯŋgɯ nɯ ra chɯ-rɤŋgɯm ŋu, ɯ-ŋgɯm nɯ kumpɣa ɯ-ŋgɯm jamar tu. kɤ-ndza sna.}\hspace{5pt}\pcmn{野鸡是一种鸟,有鸡那么大,公的很漂亮,尾巴长,头是黑的,上面有绿色的光泽,身子有红色、黄色、绿色、黑色交错着,非常漂亮,脚是浅黄色。母野鸡全身是灰色的,不漂亮。飞不高,只能飞到树上去。吃野果、粮食、虫子。很多动物吃野鸡,隼、雕、黄鼠狼、狗、猫都吃。野鸡吃粮食很厉害,在田地里和森林里下蛋。蛋和鸡蛋一样大,可以吃。}\end{exemple}\end{entrée}

\begin{entrée}{tʂu}{}{ⓔtʂu} 
\classe{n} 
\begin{définition}\pfra{chemin}\end{définition}
\begin{définition}\pcmn{路}\end{définition}
\begin{exemple}\pjya{ɯ-tɯ-tsɣe wuma ɯ-tʂu ɲɯ-ɕe}\hspace{5pt}\pcmn{他生意做得很顺利}\end{exemple}
\begin{exemple}\pjya{tʂu lo-βzu}\hspace{5pt}\pcmn{开路了}\end{exemple}
\begin{exemple}\pjya{nɤ-tʂu!}\hspace{5pt}\pcmn{请进!(主人对客人的客套话)}\end{exemple}
\begin{exemple}\pjya{ɯ-tʂu tɤ-cɯ-t-a}\hspace{5pt}\pcmn{我给他让了路}\end{exemple}
\begin{exemple}\pjya{a-tʂu nɯ-βze (=tɤ-ci)}\hspace{5pt}\pcmn{给我让路}\end{exemple}\relationsémantique{参考}{\lien{ⓔtʂɯtʂu}{tʂɯtʂu}}\relationsémantique{参考}{\lien{ⓔftɕɤru}{ftɕɤru}}\relationsémantique{参考}{\lien{ⓔnɯtʂuⓗ1}{nɯtʂu₁}}\relationsémantique{参考}{\lien{ⓔnɯtʂuⓗ2}{nɯtʂu₂}}\relationsémantique{参考}{\lien{ⓔntʂu}{ntʂu}}\relationsémantique{参考}{\lien{ⓔnɤtʂɤtshi}{nɤtʂɤtshi}}\end{entrée}

\begin{entrée}{tʂaβ}{}{ⓔtʂaβ} 
\classe{vt} \paradigme{dir}{pɯ-}\paradigme{dir}{thɯ-}\paradigme{dir}{\_}\sens{1}
\begin{définition}\pfra{faire dégringoler, rouler}\end{définition}
\begin{définition}\pcmn{令……滚下去}\end{définition}\sens{2}\paradigme{dir}{\_}
\begin{définition}\pfra{faire s'effrondrer}\end{définition}
\begin{définition}\pcmn{令……倒下}\end{définition}
\begin{définition}\pfra{se laisser tomber par terre}\end{définition}
\begin{définition}\pcmn{(故意)倒在地上}\end{définition}
\begin{exemple}\pjya{pɯ-kɯ-tʂaβ-a}\hspace{5pt}\pcmn{你把我绊倒了}\end{exemple}
\begin{exemple}\pjya{pɯ-tʂaβ-a}\hspace{5pt}\pcmn{我把他绊倒了}\end{exemple}
\begin{exemple}\pjya{a-ʑɯβ ɯ-tɯ-ɣi kɯ ɣɯ-tʂaβ-a ɲɯ-ɕti nɤma, ɲɯ-βʁa}\hspace{5pt}\pcmn{我困得快要倒下去了,因为睡意很厉害}\end{exemple}\relationsémantique{参考}{\lien{ⓔndʐaβ}{ndʐaβ}}\relationsémantique{参考}{\lien{ⓔnɤtʂaβlaβ}{nɤtʂaβlaβ}}
\begin{sous-entrée}{ʑɣɤtʂaβ}{ⓔtʂaβⓢ2ⓝʑɣɤtʂaβ} 
\classe{vi} \end{sous-entrée}

\end{entrée}

\begin{entrée}{tʂamɯɣ}{}{ⓔtʂamɯɣ} 
\classe{n} 
\begin{définition}\pfra{placard}\end{définition}
\begin{définition}\pcmn{藏式的碗柜}\end{définition}\end{entrée}

\begin{entrée}{tʂaŋ}{}{ⓔtʂaŋ} 
\classe{vs} \paradigme{dir}{tɤ-}
\begin{définition}\pfra{être juste}\end{définition}
\begin{définition}\pcmn{公平}\end{définition}
\begin{exemple}\pjya{tɯ-rju ɲɯ-tʂaŋ}\hspace{5pt}\pcmn{那句话是公平的}\end{exemple}
\begin{exemple}\pjya{rɟama ɲɯ-tʂaŋ}\hspace{5pt}\pcmn{称很准}\end{exemple}\relationsémantique{参考}{\lien{ⓔtɯtʂaŋ}{tɯtʂaŋ}}
\begin{sous-entrée}{nɤtʂaŋ}{ⓔtʂaŋⓝnɤtʂaŋ} 
\classe{vt} 
\begin{définition}\pfra{trouver juste}\end{définition}
\begin{définition}\pcmn{觉得公平}\end{définition}\end{sous-entrée}

\begin{sous-entrée}{nɯtɯtʂaŋ}{ⓔtʂaŋⓝnɯtɯtʂaŋ} 
\classe{vs} 
\begin{définition}\pfra{être juste}\end{définition}
\begin{définition}\pcmn{公平}\end{définition}\end{sous-entrée}

\begin{sous-entrée}{sɯxtʂaŋ}{ⓔtʂaŋⓝsɯxtʂaŋ} 
\classe{vs} 
\begin{définition}\pfra{régler de façon juste un contentieux}\end{définition}
\begin{définition}\pcmn{讨个公道;解决纠纷}\end{définition}
\begin{exemple}\pjya{kɯki iʑora ji-kɤ-nɤndɯt ki kɤ-sɯxtʂaŋ mɯ-mɤ-pɯ-khɯ nɤ, mɤ-nɤɕqe-a}\hspace{5pt}\pcmn{如果不能把我们之间的纠纷解决好的话,我不会放过你的}\end{exemple}\end{sous-entrée}

\étymologie{draŋ}\end{entrée}

\begin{entrée}{tʂaŋka}{}{ⓔtʂaŋka} 
\classe{n} 
\begin{définition}\pfra{pièce (d'or, d'argent)}\end{définition}
\begin{définition}\pcmn{(金、银)币}\end{définition}\end{entrée}

\begin{entrée}{tʂaŋχtɤm}{}{ⓔtʂaŋχtɤm} 
\classe{n} 
\begin{définition}\pfra{vérité}\end{définition}
\begin{définition}\pcmn{真话}\end{définition}\relationsémantique{参考}{\lien{ⓔɯ-stɤrju}{ɯ-stɤrju}}\étymologie{draŋ.gtam}\end{entrée}

\begin{entrée}{tʂapa}{}{ⓔtʂapa} 
\classe{n}
\classe{n} 
\begin{définition}\pfra{étable (bovins, ovins)}\end{définition}
\begin{définition}\pcmn{牛圈;羊圈}\end{définition}\end{entrée}

\begin{entrée}{tʂaphɤr}{}{ⓔtʂaphɤr} 
\classe{n} 
\begin{définition}\pfra{bol de moine}\end{définition}
\begin{définition}\pcmn{僧碗}\end{définition}\étymologie{grʷa.pʰor}\end{entrée}

\begin{entrée}{tʂaqhu}{}{ⓔtʂaqhu} 
\classe{n} 
\begin{définition}\pfra{bord du chemin du côté de la montagne, du côté le plus haut}\end{définition}
\begin{définition}\pcmn{靠山的路边}\end{définition}\relationsémantique{反义词}{\lien{ⓔtʂɤndo}{tʂɤndo}}\relationsémantique{参考}{\lien{ⓔtʂu}{tʂu}}\relationsémantique{参考}{\lien{ⓔɯ-qhu}{ɯ-qhu}}\end{entrée}

\begin{entrée}{tʂaʁ}{₁}{ⓔtʂaʁⓗ1} 
\classe{vs} 
\begin{définition}\pfra{avoir de l'effet, aller mieux}\end{définition}
\begin{définition}\pcmn{有效果,好}\end{définition}
\begin{exemple}\pjya{iɕqha a-ʑɯβ wuma ʑo pɯ-ɣe ri, tɤ-rɯɕmi-tɕi, tʂha kɤ-tshi-t-a tɕe ɲɤ-tʂaʁ, tɕe nɯtshɯci a-ʑɯβ mɯ-ɲɤ-ɣi}\hspace{5pt}\pcmn{刚才很瞌睡,我们聊一下,我又喝了茶,现在就好多了}\end{exemple}
\begin{exemple}\pjya{a-kɯ-mŋɤm ɲɤ-tʂaʁ}\hspace{5pt}\pcmn{我病好了}\end{exemple}\relationsémantique{同义词}{\lien{ⓔmna}{mna}}\relationsémantique{同义词}{\lien{ⓔftshi}{ftshi}}\étymologie{drag}\end{entrée}

\begin{entrée}{tʂaʁ}{₂}{ⓔtʂaʁⓗ2} 
\classe{vt} \sens{1}\paradigme{dir}{nɯ-}
\begin{définition}\pfra{mettre en morceaux avec ses doigts}\end{définition}
\begin{définition}\pcmn{捏烂;捏细}\end{définition}
\begin{exemple}\pjya{paʁtshi ɲɤ-tʂaʁ}\hspace{5pt}\pcmn{他把猪食捏碎了}\end{exemple}
\begin{exemple}\pjya{(@yangyu) ɲɤ-rɤndzraʁ nɤ ɲɤ-tʂaʁ}\hspace{5pt}\pcmn{他把洋芋捏了一把,捏烂了}\end{exemple}\sens{2}\paradigme{dir}{pɯ-}
\begin{définition}\pfra{réprimer}\end{définition}
\begin{définition}\pcmn{镇压}\end{définition}
\begin{exemple}\pjya{ŋgundʑɯɣ ɣɯ-ndo tɤ-ra tɕe, tɯrme kɯ-ŋɤn nɯra kɤ-tʂaʁ pjɯ-kɯ-cha ra, tɯrme kɯ-pe nɯra kɤ-ntɕhoz pjɯ-kɯ-cha ra.}\hspace{5pt}\pcmn{当了领导要会镇住坏人,重用好人}\end{exemple}\end{entrée}

\begin{entrée}{tʂɤɕphɤt}{}{ⓔtʂɤɕphɤt} 
\classe{n} 
\begin{définition}\pfra{plantain}\end{définition}
\begin{définition}\pcmn{车前草}\end{définition}
\begin{exemple}\pjya{tʂɤɕphɤt nɯ tʂɤrkɯ ra wuma ʑo kɤ-ɬoʁ rga, ɯ-jwaʁ sɤtɕha ɯ-taʁ pjɯ-ɤɲɟoʁ ʑo ŋu, ɯ-mɯntoʁ nɯ ɯ-ru kɯ-zri tsa tu-ɬoʁ tɕe nɯ-taʁ ku-ndzoʁ ŋu, ɯ-mdoʁ kɯ-mpɕɤr ra me, ɲɯ́-wɣ-phɯt tɕe ɯ-ŋgru tu. tɯ-ɕɣa kɯ-mŋɤm ɯ-smɤn ŋu. ɯ-mdoʁ kɯ-ɤpɣɯlu tsa ŋu. pakuku tu-ɬoʁ ŋu.}\hspace{5pt}\pcmn{车前草一般生长在路边上,叶子贴在地面上,茎长到一定程度后在上面开花,花色不美,撕扯叶子时有(较结实的)筋。可以治牙痛。颜色是淡灰色。年年都长。}\end{exemple}\relationsémantique{参考}{\lien{ⓔtʂu}{tʂu}}\relationsémantique{参考}{\lien{ⓔɕphɤt}{ɕphɤt}}\end{entrée}

\begin{entrée}{tʂɤkɤcu}{}{ⓔtʂɤkɤcu} 
\classe{n} 
\begin{définition}\pfra{bord du chemin du côté est (lorsqu'il n'y a pas de distinction de hauteur entre les deux bords du chemin)}\end{définition}
\begin{définition}\pcmn{路的两边没有高低之分时,靠近东方的边缘}\end{définition}\relationsémantique{参考}{\lien{ⓔtʂu}{tʂu}}\end{entrée}

\begin{entrée}{tʂɤm}{}{ⓔtʂɤm} 
\classe{n} 
\begin{définition}\pfra{cloison}\end{définition}
\begin{définition}\pcmn{板壁}\end{définition}\end{entrée}

\begin{entrée}{tʂɤmar}{}{ⓔtʂɤmar} 
\classe{n} 
\begin{définition}\pfra{beurre que l'on emporte pour le voyage}\end{définition}
\begin{définition}\pcmn{路上吃的酥油}\end{définition}\relationsémantique{参考}{\lien{ⓔtʂu}{tʂu}}\relationsémantique{参考}{\lien{ⓔta-mar}{ta-mar}}\end{entrée}

\begin{entrée}{tʂɤmthɯm}{}{ⓔtʂɤmthɯm} 
\classe{n} 
\begin{définition}\pfra{viande que l'on emporte pour le voyage}\end{définition}
\begin{définition}\pcmn{路上吃的肉}\end{définition}\relationsémantique{参考}{\lien{ⓔtʂu}{tʂu}}\relationsémantique{参考}{\lien{ⓔtɤ-mthɯm}{tɤ-mthɯm}}\end{entrée}

\begin{entrée}{tʂɤmtshi}{}{ⓔtʂɤmtshi} 
\classe{n} 
\begin{définition}\pfra{fait de conduire le chemin}\end{définition}
\begin{définition}\pcmn{引路}\end{définition}
\begin{exemple}\pjya{ɯʑo kɯ a-tʂɤmtshi ta-βzu}\hspace{5pt}\pcmn{他给我引了路;他指引了我}\end{exemple}\relationsémantique{参考}{\lien{ⓔmtshi}{mtshi}}\relationsémantique{参考}{\lien{ⓔtʂu}{tʂu}}\relationsémantique{参考}{\lien{ⓔɣɯtʂɤmtshi}{ɣɯtʂɤmtshi}}\end{entrée}

\begin{entrée}{tʂɤmtshi}{}{ⓔtʂɤmtshi} 
\classe{n} 
\begin{définition}\pfra{fait de guider le chemin}\end{définition}
\begin{définition}\pcmn{带路;引路}\end{définition}
\begin{exemple}\pjya{kɯki tʂu ki nɤ-tʂɤmtshi aj tu-βze-a}\hspace{5pt}\pcmn{我给你带路}\end{exemple}\relationsémantique{参考}{\lien{ⓔɣɯtʂɤmtshi}{ɣɯtʂɤmtshi}}\end{entrée}

\begin{entrée}{tʂɤndɤcu}{}{ⓔtʂɤndɤcu} 
\classe{n} 
\begin{définition}\pfra{bord du chemin du côté ouest (lorsqu'il n'y a pas de distinction de hauteur entre les deux bords du chemin)}\end{définition}
\begin{définition}\pcmn{路的两边没有高低之分时,靠近西方的边缘}\end{définition}\end{entrée}

\begin{entrée}{tʂɤndo}{}{ⓔtʂɤndo} 
\classe{n} 
\begin{définition}\pfra{bord du chemin du côté du fleuve, du côté le plus bas}\end{définition}
\begin{définition}\pcmn{靠水的路边}\end{définition}\relationsémantique{反义词}{\lien{ⓔtʂaqhu}{tʂaqhu}}\relationsémantique{参考}{\lien{ⓔtʂu}{tʂu}}\relationsémantique{参考}{\lien{ⓔɯ-ndo}{ɯ-ndo}}\end{entrée}

\begin{entrée}{tʂɤrkɯ}{}{ⓔtʂɤrkɯ} 
\classe{n} 
\begin{définition}\pfra{bord du chemin}\end{définition}
\begin{définition}\pcmn{路边}\end{définition}\relationsémantique{参考}{\lien{ⓔtʂu}{tʂu}}\relationsémantique{参考}{\lien{ⓔɯ-rkɯ}{ɯ-rkɯ}}\end{entrée}

\begin{entrée}{tʂɤrtaʁ}{}{ⓔtʂɤrtaʁ} 
\classe{n} 
\begin{définition}\pfra{carrefour}\end{définition}
\begin{définition}\pcmn{岔路}\end{définition}\relationsémantique{参考}{\lien{ⓔtʂu}{tʂu}}\relationsémantique{参考}{\lien{ⓔtɤ-rtaʁ}{tɤ-rtaʁ}}\end{entrée}

\begin{entrée}{tʂɤsɤɴɢɤt}{}{ⓔtʂɤsɤɴɢɤt} 
\classe{n} 
\begin{définition}\pfra{croisée de chemin}\end{définition}
\begin{définition}\pcmn{岔路}\end{définition}\relationsémantique{参考}{\lien{ⓔtʂu}{tʂu}}\relationsémantique{参考}{\lien{ⓔɴɢɤt}{ɴɢɤt}}\end{entrée}

\begin{entrée}{tʂɤχa}{}{ⓔtʂɤχa} 
\classe{n} 
\begin{définition}\pfra{fondrière, nid-de-poule}\end{définition}
\begin{définition}\pcmn{路上的坑洼(缺口)}\end{définition}
\begin{exemple}\pjya{tɯ-ɤtɤr pɯ-kɯ-rɲo kɯ tʂɤχa nɯɣme}\hspace{5pt}\pcmn{曾经摔倒过的人怕路上的缺口(惊弓之鸟)}\end{exemple}\relationsémantique{参考}{\lien{ⓔaχa}{aχa}}\relationsémantique{参考}{\lien{ⓔtʂu}{tʂu}}\end{entrée}

\begin{entrée}{tʂɤχcɤl}{}{ⓔtʂɤχcɤl} 
\classe{n} 
\begin{définition}\pfra{milieu du chemin}\end{définition}
\begin{définition}\pcmn{路中间}\end{définition}\relationsémantique{参考}{\lien{ⓔtʂu}{tʂu}}\relationsémantique{参考}{\lien{ⓔɯ-χcɤl}{ɯ-χcɤl}}\end{entrée}

\begin{entrée}{tʂɤzɤn}{}{ⓔtʂɤzɤn} 
\classe{n} 
\begin{définition}\pfra{ration pour la route}\end{définition}
\begin{définition}\pcmn{盘缠}\end{définition}\relationsémantique{参考}{\lien{ⓔtʂu}{tʂu}}\end{entrée}

\begin{entrée}{tʂha}{}{ⓔtʂha} 
\classe{n} 
\begin{définition}\pfra{thé, petit déjeuner}\end{définition}
\begin{définition}\pcmn{茶,早饭}\end{définition}
\begin{exemple}\pjya{tʂha kɤ́tɤlɯlu}\hspace{5pt}\pcmn{奶茶}\end{exemple}\end{entrée}

\begin{entrée}{tʂhazwa}{}{ⓔtʂhazwa} 
\classe{n} 
\begin{définition}\pfra{feuilles de thé qui restent après que le thé ait été bu}\end{définition}
\begin{définition}\pcmn{茶喝完了以后留下的茶叶}\end{définition}\end{entrée}

\begin{entrée}{tʂhɤlu}{}{ⓔtʂhɤlu} 
\classe{n} 
\begin{définition}\pfra{thé au lait}\end{définition}
\begin{définition}\pcmn{奶茶}\end{définition}\relationsémantique{参考}{\lien{ⓔnɯtʂhɤlu}{nɯtʂhɤlu}}\end{entrée}

\begin{entrée}{tʂhɤrqhu}{}{ⓔtʂhɤrqhu} 
\classe{n} 
\begin{définition}\pfra{natte utilisée pour conserver le thé}\end{définition}
\begin{définition}\pcmn{装茶的席子}\end{définition}\end{entrée}

\begin{entrée}{tʂhɤt}{₂}{ⓔtʂhɤtⓗ2} 
\classe{idph.1} 
\begin{définition}\pfra{bruit de gouttes qui tombent}\end{définition}
\begin{définition}\pcmn{滴水的声音}\end{définition}
\begin{sous-entrée}{tʂhɤtnɤtʂhɤt}{ⓔtʂhɤtⓗ2ⓝtʂhɤtnɤtʂhɤt} 
\classe{idph.3} 
\begin{exemple}\pjya{tɯ-ci tʂhɤtnɤtʂhɤt ɲɯ-nɯftsaʁ}\end{exemple}
\begin{exemple}\pjya{tɯftsaʁ tʂhɤtnɤtʂhɤt ʑo ɲɯ-nɯftsaʁ}\end{exemple}
\begin{exemple}\pjya{tɯftsaʁ tʂhɤtnɤtʂhɤt ʑo ɲɯ-ɣi}\end{exemple}
\begin{exemple}\pjya{tɯftsaʁ tʂhɤtnɤtʂhɤt ʑo ɲɯ-ti}\hspace{5pt}\pcmn{一滴一滴地漏水}\end{exemple}\end{sous-entrée}

\end{entrée}

\begin{entrée}{tʂhɤt}{₁}{ⓔtʂhɤtⓗ1} 
\classe{vs} 
\begin{définition}\pfra{arrogant}\end{définition}
\begin{définition}\pcmn{傲慢}\end{définition}
\begin{exemple}\pjya{jiɕqha tɯrme nɯ kɯ-tʂhɤt ci ŋu}\hspace{5pt}\pcmn{那个人}\end{exemple}
\begin{sous-entrée}{znɤtʂhɯtʂhɯt/\variante{znɤtʂhɤtʂhɤt}}{ⓔtʂhɤtⓗ1ⓝznɤtʂhɯtʂhɯt} 
\classe{vs} 
\begin{définition}\pfra{arrogant}\end{définition}
\begin{définition}\pcmn{自以为是;霸道}\end{définition}
\begin{exemple}\pjya{kɯ-znɤtʂhɯtʂhɯt ci ɲɯ-ŋu}\hspace{5pt}\pcmn{他是个霸道的人}\end{exemple}\end{sous-entrée}

\end{entrée}

\begin{entrée}{tʂhɤzwa}{}{ⓔtʂhɤzwa} 
\classe{n} 
\begin{définition}\pfra{lie du thé}\end{définition}
\begin{définition}\pcmn{茶里的渣滓}\end{définition}\end{entrée}

\begin{entrée}{tʂhɯβnɤtʂhɯβ}{}{ⓔtʂhɯβnɤtʂhɯβ} 
\classe{idph.3} 
\begin{définition}\pfra{bruit produit lorsque l'on se mouche le nez}\end{définition}
\begin{définition}\pcmn{擤鼻涕的声音}\end{définition}
\begin{exemple}\pjya{tɤ-pɤtso kɯ ɯ-ɕna mɯ-chɯ-pɕiz tɕe, tʂhɯβnɤtʂhɯβ tu-sɯ-ti ɲɯ-ŋu}\hspace{5pt}\pcmn{小孩子没有擤好鼻涕就发出呼哧呼哧的声音}\end{exemple}
\begin{sous-entrée}{phɯtʂhɯβ}{ⓔtʂhɯβnɤtʂhɯβⓝphɯtʂhɯβ} 
\classe{idph.7} \end{sous-entrée}

\end{entrée}

\begin{entrée}{tʂhɯɣ}{}{ⓔtʂhɯɣ} 
\classe{adv} 
\begin{définition}\pfra{peut-être}\end{définition}
\begin{définition}\pcmn{大概;可能}\end{définition}
\begin{exemple}\pjya{nɤʑo tʂhɯɣ thɯ-tɯ-nɯkɯmaʁ}\hspace{5pt}\pcmn{你可能错了}\end{exemple}
\begin{exemple}\pjya{nɯ tʂhɯɣ maʁ lo}\hspace{5pt}\pcmn{大概不是吧}\end{exemple}\end{entrée}

\begin{entrée}{tʂhɯznɤtʂhɯz}{}{ⓔtʂhɯznɤtʂhɯz} 
\classe{idph.3} 
\begin{définition}\pfra{petit bruit d'explosion}\end{définition}
\begin{définition}\pcmn{形容爆炸的细小声音}\end{définition}
\begin{exemple}\pjya{tɤɕi chɯ́-wɣ-rŋu tɕe, tɤ-smi tɕe, tʂhɯznɤtʂhɯz ʑo ɲɯ-ɤmboʁ ŋu}\hspace{5pt}\pcmn{炒青稞的时候,炒熟了就会爆炸发出声音}\end{exemple}
\begin{sous-entrée}{phɯtʂhɯz}{ⓔtʂhɯznɤtʂhɯzⓝphɯtʂhɯz} 
\classe{idph.7} 
\begin{définition}\pfra{bruit d'explosion soudaine}\end{définition}
\begin{définition}\pcmn{形容突然爆炸的声音}\end{définition}
\begin{exemple}\pjya{tɤntɤβ phɯtʂhɯz ʑo ɲɯ-ɤmboʁ}\hspace{5pt}\pcmn{水泡噗嗤一声就爆裂了}\end{exemple}\end{sous-entrée}

\end{entrée}

\begin{entrée}{tʂo}{}{ⓔtʂo} 
\classe{vt} \paradigme{dir}{nɯ-}\paradigme{dir}{pɯ-}
\begin{définition}\pfra{payer}\end{définition}
\begin{définition}\pcmn{付}\end{définition}
\begin{exemple}\pjya{ɯ-nŋa ɲɤ-tʂo}\hspace{5pt}\pcmn{他还了债}\end{exemple}\relationsémantique{参考}{\lien{ⓔnɯnŋɤtʂo}{nɯnŋɤtʂo}}\end{entrée}

\begin{entrée}{tʂoʁ}{}{ⓔtʂoʁ} 
\classe{vt} \paradigme{dir}{kɤ-}\paradigme{dir}{tɤ-}\paradigme{dir}{pɯ-}
\begin{définition}\pfra{ajouter de l’eau}\end{définition}
\begin{définition}\pcmn{掺和}\end{définition}
\begin{exemple}\pjya{tɯ-ci kɤ-tʂoʁ}\hspace{5pt}\pcmn{掺水吧}\end{exemple}
\begin{exemple}\pjya{cha to-tʂoʁ}\hspace{5pt}\pcmn{他掺了酒}\end{exemple}\end{entrée}

\begin{entrée}{tʂot}{}{ⓔtʂot} 
\classe{vs} \paradigme{dir}{tɤ-}
\begin{définition}\pfra{clair}\end{définition}
\begin{définition}\pcmn{清楚;清晰;明显}\end{définition}
\begin{exemple}\pjya{tʂu ɲɯ-tʂot}\hspace{5pt}\pcmn{路很清晰(因为走的人多,路没有消失)}\end{exemple}\end{entrée}

\begin{entrée}{tʂɯβ}{}{ⓔtʂɯβ} 
\classe{vt} \paradigme{dir}{kɤ-}\paradigme{dir}{thɯ-}
\begin{définition}\pfra{coudre}\end{définition}
\begin{définition}\pcmn{缝}\end{définition}\paradigme{dir}{thɯ-}\paradigme{dir}{pɯ-}
\begin{définition}\pfra{coudre}\end{définition}
\begin{définition}\pcmn{缝;补}\end{définition}
\begin{exemple}\pjya{tɯ-xtsa kɤ-tʂɯβ-a}\hspace{5pt}\pcmn{我缝了鞋子}\end{exemple}
\begin{exemple}\pjya{chɤ-rɤtʂɯβ}\hspace{5pt}\pcmn{他缝了}\end{exemple}
\begin{sous-entrée}{rɤtʂɯβ}{ⓔtʂɯβⓝrɤtʂɯβ} 
\classe{vi}  
\grammaire{apass} \end{sous-entrée}

\étymologie{ⁿdrub}\end{entrée}

\begin{entrée}{tʂɯɣlaʁ}{}{ⓔtʂɯɣlaʁ} 
\classe{n} 
\begin{définition}\pfra{ramure de cerf à 12 branches}\end{définition}
\begin{définition}\pcmn{十二叉的鹿角}\end{définition}\end{entrée}

\begin{entrée}{tʂɯɣpa}{}{ⓔtʂɯɣpa} 
\classe{n} 
\begin{définition}\pfra{sixième mois}\end{définition}
\begin{définition}\pcmn{六月}\end{définition}\étymologie{drug.pa}\end{entrée}

\begin{entrée}{tʂɯl}{}{ⓔtʂɯl} 
\classe{vt} \paradigme{dir}{tɤ-}\paradigme{dir}{thɯ-}
\begin{définition}\pfra{enrouler (vêtement)}\end{définition}
\begin{définition}\pcmn{包、裹(衣服)}\end{définition}
\begin{exemple}\pjya{thɯ-tʂɯl-a}\hspace{5pt}\pcmn{我裹了(衣服)}\end{exemple}
\begin{exemple}\pjya{tɯ-ŋga thɯ-tʂɯl}\hspace{5pt}\pcmn{你把衣服裹起来吧}\end{exemple}
\begin{exemple}\pjya{ɯ-ku to-tʂɯl tslɯɣtslɯɣ ʑo (=to-mphɯr)}\hspace{5pt}\pcmn{他把头包得又厚有紧}\end{exemple}\relationsémantique{同义词}{\lien{ⓔmphɯr}{mphɯr}}\étymologie{sgril}\end{entrée}

\begin{entrée}{tʂɯmpa}{}{ⓔtʂɯmpa} 
\classe{n} 
\begin{définition}\pfra{tablier}\end{définition}
\begin{définition}\pcmn{围腰帕}\end{définition}\end{entrée}

\begin{entrée}{tʂɯnlɤn}{}{ⓔtʂɯnlɤn} 
\classe{n} 
\begin{définition}\pfra{bienfait}\end{définition}
\begin{définition}\pcmn{报恩}\end{définition}
\begin{exemple}\pjya{nɤ-tʂɯnlɤn ɲɯ-ta-fsɯɣ ra}\hspace{5pt}\pcmn{我要回报你的恩情}\end{exemple}\étymologie{drin.len}\end{entrée}

\begin{entrée}{tʂɯtʂu}{}{ⓔtʂɯtʂu} 
\classe{adv} 
\begin{définition}\pfra{en chemin}\end{définition}
\begin{définition}\pcmn{路上}\end{définition}\relationsémantique{参考}{\lien{ⓔtʂu}{tʂu}}\end{entrée}

\begin{entrée}{tɯ-boʁ}{}{ⓔtɯ-boʁ} 
\classe{clf} 
\begin{définition}\pfra{troupeau, groupe}\end{définition}
\begin{définition}\pcmn{一群;一队}\end{définition}
\begin{exemple}\pjya{qaʑo tɯ-boʁ}\hspace{5pt}\pcmn{一群羊}\end{exemple}
\begin{exemple}\pjya{kɯ-nɤʁaʁ tɯrme tɯboʁ tu}\hspace{5pt}\pcmn{有一群人在休息}\end{exemple}\end{entrée}

\begin{entrée}{tɯ-βɣi}{}{ⓔtɯ-βɣi} 
\classe{np} \sens{1}
\begin{définition}\pfra{poudre}\end{définition}
\begin{définition}\pcmn{粉末}\end{définition}\sens{2}
\begin{définition}\pfra{cendres}\end{définition}
\begin{définition}\pcmn{灰(烧成……)}\end{définition}\sens{3}
\begin{définition}\pfra{balle}\end{définition}
\begin{définition}\pcmn{糠}\end{définition}\relationsémantique{参考}{\lien{ⓔsɯβɣi}{sɯβɣi}}\end{entrée}

\begin{entrée}{tɯ-βlɤz}{}{ⓔtɯ-βlɤz} 
\classe{np} 
\begin{définition}\pfra{devoir}\end{définition}
\begin{définition}\pcmn{义务}\end{définition}
\begin{exemple}\pjya{tɕhaʁra chɯ-raʁrɯz-a ma a-βlɤz ɕti}\hspace{5pt}\pcmn{我扫厕所,因为这是我的义务}\end{exemple}
\begin{exemple}\pjya{aʑo nɤme-a ra ma a-βlɤz ɕti}\hspace{5pt}\pcmn{我一定要做,因为是我的义务}\end{exemple}
\begin{exemple}\pjya{a-βlɤz kɯ-tu me}\hspace{5pt}\pcmn{我没有什么要承担的责任}\end{exemple}\étymologie{blas}\end{entrée}

\begin{entrée}{tɯ-βli}{}{ⓔtɯ-βli} 
\classe{np} 
\begin{définition}\pfra{pousses}\end{définition}
\begin{définition}\pcmn{秧}\end{définition}
\begin{exemple}\pjya{tɯ-βli pɯ-ta-t-a}\end{exemple}
\begin{exemple}\pjya{tɯ-βli pɯ-ji-t-a}\hspace{5pt}\pcmn{我插了秧了}\end{exemple}\end{entrée}

\begin{entrée}{tɯ-βlɯz}{}{ⓔtɯ-βlɯz} 
\classe{np} 
\begin{définition}\pfra{par cœur}\end{définition}
\begin{définition}\pcmn{背诵,不需要模型}\end{définition}
\begin{exemple}\pjya{a-βlɯz pɯ-ndɯn-a}\hspace{5pt}\pcmn{我背诵了}\end{exemple}
\begin{exemple}\pjya{a-βlɯz nɯ-ari}\hspace{5pt}\pcmn{我不需要模型就做了}\end{exemple}
\begin{exemple}\pjya{a-βlɯz chɯ-taʁ-a khɯ}\hspace{5pt}\pcmn{我不用模型就会织(花带)}\end{exemple}\relationsémantique{参考}{\lien{ⓔnɯβlɯz}{nɯβlɯz}}\end{entrée}

\begin{entrée}{tɯ-βra}{}{ⓔtɯ-βra} 
\classe{np} \sens{1}
\begin{définition}\pfra{part}\end{définition}
\begin{définition}\pcmn{自己的那一份}\end{définition}
\begin{exemple}\pjya{a-βra nɯ-nɯ-ta-t-a}\hspace{5pt}\pcmn{我给自己留了一份}\end{exemple}\sens{2}
\begin{définition}\pfra{au tour de...}\end{définition}
\begin{définition}\pcmn{轮到……}\end{définition}
\begin{exemple}\pjya{aʑo tɤ-ndza-t-a tɕe nɤʑo (nɤ-)βra tɤ-ndze}\hspace{5pt}\pcmn{我已经吃了,轮到你吃了}\end{exemple}\end{entrée}

\begin{entrée}{tɯ-βri}{}{ⓔtɯ-βri} 
\classe{np}
\classe{vi} 
\begin{définition}\pfra{corps}\end{définition}
\begin{définition}\pcmn{身体(上的皮肤)}\end{définition}\relationsémantique{Component 2}{\lien{ⓔɕe}{ɕe}}
\begin{sous-entrée}{tɯ-βri,ɕe}{ⓔtɯ-βriⓝtɯ-βri,ɕe} 
\classe{np} 
\begin{définition}\pfra{se faire avoir}\end{définition}
\begin{définition}\pcmn{吃亏}\end{définition}
\begin{exemple}\pjya{nɤ-βri ɕe}\hspace{5pt}\pcmn{你要吃亏}\end{exemple}\relationsémantique{Component 1}{\lien{ⓔtɯ-βri}{tɯ-βri}}\end{sous-entrée}

\end{entrée}

\begin{entrée}{tɯ-ci}{}{ⓔtɯ-ci} 
\classe{np} \sens{1}
\begin{définition}\pfra{eau}\end{définition}
\begin{définition}\pcmn{水}\end{définition}
\begin{exemple}\pjya{mɯntoʁ ɯ-tɯ-ci kɤ-lɤt ɲɯ-ra}\hspace{5pt}\pcmn{要给花浇水}\end{exemple}\sens{2}
\begin{définition}\pfra{jus, lait (d'une plante)}\end{définition}
\begin{définition}\pcmn{汁(植物)}\end{définition}\relationsémantique{参考}{\lien{ⓔaɣɯci}{aɣɯci}}\end{entrée}

\begin{entrée}{tɯciβɣuβɣu}{}{ⓔtɯciβɣuβɣu} 
\classe{n} 
\begin{définition}\pfra{têtard}\end{définition}
\begin{définition}\pcmn{蝌蚪}\end{définition}\end{entrée}

\begin{entrée}{tɯciɕku}{}{ⓔtɯciɕku} 
\classe{n} 
\begin{définition}\pfra{poireau sauvage}\end{définition}
\begin{définition}\pcmn{野韭菜}\end{définition}
\begin{exemple}\pjya{tɯciɕku nɯ tɯ-ci ɯ-rkɯ qambɯt ɯ-ŋgɯ rdɤstaʁ ɯ-rchɤβ tu-ɬoʁ ŋu. ɯ-tshɯɣa ra lonba tɤ-spɯ ɕku fse, tɕeri ɯ-mdoʁ arŋi. kɯ-ɣɤjlu di me. tu-tɯ-ɬoʁ tú-wɣ-ndza tɕe mɯm.}\hspace{5pt}\pcmn{\lien{ⓔtɯciɕku}{tɯciɕku}生长在大河边的沙滩里和石头缝里,形状完全和\lien{ⓔtɤspɯɕku}{tɤspɯɕku} 一样,但是颜色是绿色的,没有腥味。刚长出来时吃起来很香。}\end{exemple}\end{entrée}

\begin{entrée}{tɯcifkɯm}{}{ⓔtɯcifkɯm} 
\classe{n} 
\begin{définition}\pfra{gourde}\end{définition}
\begin{définition}\pcmn{水壶,装水的东西}\end{définition}\relationsémantique{参考}{\lien{ⓔtɤ-fkɯm}{tɤ-fkɯm}}\end{entrée}

\begin{entrée}{tɯci pɣɤtɕɯ}{}{ⓔtɯci pɣɤtɕɯ} 
\classe{n} 
\begin{définition}\pfra{espèce d'oiseau}\end{définition}
\begin{définition}\pcmn{一种鸟}\end{définition}\end{entrée}

\begin{entrée}{tɯciwɯrwɯr}{}{ⓔtɯciwɯrwɯr} 
\classe{n} 
\begin{définition}\pfra{tourbillon}\end{définition}
\begin{définition}\pcmn{漩涡}\end{définition}\end{entrée}

\begin{entrée}{tɯcizʁe}{}{ⓔtɯcizʁe} 
\classe{n} 
\begin{définition}\pfra{action de transporter de l'eau}\end{définition}
\begin{définition}\pcmn{背水}\end{définition}
\begin{exemple}\pjya{tɯcizʁe tɤ-βzu-t-a}\hspace{5pt}\pcmn{我背了很多水}\end{exemple}\relationsémantique{参考}{\lien{ⓔnɯtɯcizʁe}{nɯtɯcizʁe}}\end{entrée}

\begin{entrée}{tɯcɯla}{}{ⓔtɯcɯla} 
\classe{n} 
\begin{définition}\pfra{eau bouillante}\end{définition}
\begin{définition}\pcmn{开水}\end{définition}\relationsémantique{参考}{\lien{ⓔala}{ala}}\end{entrée}

\begin{entrée}{tɯcɯrqɯ}{}{ⓔtɯcɯrqɯ} 
\classe{n} 
\begin{définition}\pfra{eau froide}\end{définition}
\begin{définition}\pcmn{冷水}\end{définition}\end{entrée}

\begin{entrée}{tɯcɯste}{}{ⓔtɯcɯste} 
\classe{n} 
\begin{définition}\pfra{sac amniotique}\end{définition}
\begin{définition}\pcmn{羊膜囊}\end{définition}
\begin{exemple}\pjya{ɯ-tɯcɯste chɤ-ndʑɣaʁ}\hspace{5pt}\pcmn{她羊水破了}\end{exemple}\end{entrée}

\begin{entrée}{tɯ-ɕa}{}{ⓔtɯ-ɕa} 
\classe{np} 
\begin{définition}\pfra{muscle}\end{définition}
\begin{définition}\pcmn{肌肉}\end{définition}\étymologie{ɕa}\end{entrée}

\begin{entrée}{tɯ-ɕaqraʁ}{}{ⓔtɯ-ɕaqraʁ} 
\classe{np} 
\begin{définition}\pfra{omoplate}\end{définition}
\begin{définition}\pcmn{肩胛骨}\end{définition}\end{entrée}

\begin{entrée}{tɯɕɤt}{}{ⓔtɯɕɤt} 
\classe{n} 
\begin{définition}\pfra{habitude}\end{définition}
\begin{définition}\pcmn{习惯}\end{définition}
\begin{exemple}\pjya{ki nɤʑo nɤ-tɯɕɤt ɕti}\hspace{5pt}\pcmn{这是你的习惯}\end{exemple}\end{entrée}

\begin{entrée}{tɯ-ɕɣa}{}{ⓔtɯ-ɕɣa} 
\classe{np} 
\begin{définition}\pfra{dent}\end{définition}
\begin{définition}\pcmn{牙齿}\end{définition}
\begin{exemple}\pjya{a-ɕɣa qajɯ kɯ ɲɯ-ɤsɯ-ndza}\hspace{5pt}\pcmn{我长了虫牙了。}\end{exemple}
\begin{exemple}\pjya{tɯ-ɕɣa qajɯ kɯ tu-ndze tɕe wuma ʑo mŋɤm}\hspace{5pt}\pcmn{有蛀牙,很痛}\end{exemple}
\begin{exemple}\pjya{a-ɕɣa ɲɯ-mŋɤm}\hspace{5pt}\pcmn{牙疼!}\end{exemple}\relationsémantique{参考}{\lien{ⓔaɕɣa}{aɕɣa}}\end{entrée}

\begin{entrée}{tɯ-ɕɣɤdi}{}{ⓔtɯ-ɕɣɤdi} 
\classe{np} 
\begin{définition}\pfra{haleine fétide}\end{définition}
\begin{définition}\pcmn{口臭}\end{définition}
\begin{exemple}\pjya{nɤ-ɕɣa nɯ-χtɕi ma nɤ-ɕɣɤdi mnɤm ko}\hspace{5pt}\pcmn{你要刷牙,你有口臭}\end{exemple}\relationsémantique{参考}{\lien{ⓔtɯ-ɕɣa}{tɯ-ɕɣa}}\relationsémantique{参考}{\lien{ⓔtɤ-di}{tɤ-di}}\end{entrée}

\begin{entrée}{tɯɕɣɤŋɤm}{}{ⓔtɯɕɣɤŋɤm} 
\classe{n} 
\begin{définition}\pfra{mal aux dents}\end{définition}
\begin{définition}\pcmn{牙疼}\end{définition}\relationsémantique{参考}{\lien{ⓔtɯ-ɕɣa}{tɯ-ɕɣa}}\relationsémantique{参考}{\lien{ⓔmŋɤm}{mŋɤm}}\end{entrée}

\begin{entrée}{tɯ-ɕɣɤrgu}{}{ⓔtɯ-ɕɣɤrgu} 
\classe{np} 
\begin{définition}\pfra{qualité de la dentition}\end{définition}
\begin{définition}\pcmn{牙齿的质量}\end{définition}
\begin{exemple}\pjya{ɯ-ɕɣɤrgu ɲɯ-sna}\hspace{5pt}\pcmn{他牙口很好}\end{exemple}\relationsémantique{参考}{\lien{ⓔtɯ-ɕɣa}{tɯ-ɕɣa}}\end{entrée}

\begin{entrée}{tɯ-ɕɣɤse}{}{ⓔtɯ-ɕɣɤse} 
\classe{np} 
\begin{définition}\pfra{saignement des dents}\end{définition}
\begin{définition}\pcmn{牙龈流血}\end{définition}
\begin{exemple}\pjya{ɯ-ɕɣɤse pjɤ-ɬoʁ}\hspace{5pt}\pcmn{他牙龈流血了}\end{exemple}\end{entrée}

\begin{entrée}{tɯ-ɕɣɤte}{}{ⓔtɯ-ɕɣɤte} 
\classe{np} 
\begin{définition}\pfra{molaires et prémolaires}\end{définition}
\begin{définition}\pcmn{臼齿}\end{définition}\relationsémantique{参考}{\lien{ⓔtɯ-sa}{tɯ-sa}}\end{entrée}

\begin{entrée}{tɯ-ɕkat}{}{ⓔtɯ-ɕkat} 
\classe{clf} 
\begin{définition}\pfra{charge sur une bête de somme}\end{définition}
\begin{définition}\pcmn{驮子}\end{définition}\relationsémantique{参考}{\lien{ⓔɣɯɕkat}{ɣɯɕkat}}\end{entrée}

\begin{entrée}{tɯɕkho}{}{ⓔtɯɕkho} 
\classe{n} 
\begin{définition}\pfra{chose en train d'être séchée}\end{définition}
\begin{définition}\pcmn{正在晒干的东西}\end{définition}\relationsémantique{参考}{\lien{ⓔɕkho}{ɕkho}}\end{entrée}

\begin{entrée}{tɯ-ɕkrɯt}{}{ⓔtɯ-ɕkrɯt} 
\classe{np} 
\begin{définition}\pfra{bile}\end{définition}
\begin{définition}\pcmn{胆}\end{définition}\étymologie{mkʰris}\end{entrée}

\begin{entrée}{tɯɕlu}{}{ⓔtɯɕlu} 
\classe{n} 
\begin{définition}\pfra{labour}\end{définition}
\begin{définition}\pcmn{耕地}\end{définition}
\begin{exemple}\pjya{jla kɯ tɯɕlu ɲɯ-rɤɕi pɯ-ra tɕe, jla a-pɯ-me tɕe tɯ-mgo kɤ-ndza mɯ-pɯ-ŋgrɯ}\hspace{5pt}\pcmn{只有犏牛才能耕地,没有犏牛就没有饭吃}\end{exemple}\relationsémantique{参考}{\lien{ⓔɕlu}{ɕlu}}\end{entrée}

\begin{entrée}{tɯ-ɕmi}{}{ⓔtɯ-ɕmi} 
\classe{np} 
\begin{définition}\pfra{parole}\end{définition}
\begin{définition}\pcmn{话}\end{définition}
\begin{exemple}\pjya{ɯ-ɕmi ɲɯ-dɤn}\hspace{5pt}\pcmn{他话很多}\end{exemple}\relationsémantique{参考}{\lien{ⓔrɯɕmi}{rɯɕmi}}\end{entrée}

\begin{entrée}{tɯ-ɕna}{}{ⓔtɯ-ɕna} 
\classe{np} 
\begin{définition}\pfra{nez}\end{définition}
\begin{définition}\pcmn{鼻子}\end{définition}\relationsémantique{参考}{\lien{ⓔɕnɤxsɯr}{ɕnɤxsɯr}}\relationsémantique{参考}{\lien{ⓔtɯ-ɕnɤɣɲɟɯ}{tɯ-ɕnɤɣɲɟɯ}}\end{entrée}

\begin{entrée}{tɯ-ɕnaβ}{}{ⓔtɯ-ɕnaβ} 
\classe{np} 
\begin{définition}\pfra{morve sèche}\end{définition}
\begin{définition}\pcmn{干的鼻涕}\end{définition}\relationsémantique{参考}{\lien{ⓔaɣɯɕnɯɕnaβ}{aɣɯɕnɯɕnaβ}}\end{entrée}

\begin{entrée}{tɯ-ɕnɤɣɲɟɯ}{}{ⓔtɯ-ɕnɤɣɲɟɯ} 
\classe{np} 
\begin{définition}\pfra{narine}\end{définition}
\begin{définition}\pcmn{鼻孔}\end{définition}\relationsémantique{参考}{\lien{ⓔtɯ-ɕna}{tɯ-ɕna}}\relationsémantique{参考}{\lien{ⓔɯ-ɣɲɟɯ}{ɯ-ɣɲɟɯ}}\end{entrée}

\begin{entrée}{tɯ-ɕnɤjtsi}{}{ⓔtɯ-ɕnɤjtsi} 
\classe{np} 
\begin{définition}\pfra{arête du nez}\end{définition}
\begin{définition}\pcmn{鼻梁}\end{définition}\relationsémantique{参考}{\lien{ⓔtɤ-jtsi}{tɤ-jtsi}}\end{entrée}

\begin{entrée}{tɯ-ɕnɤku}{}{ⓔtɯ-ɕnɤku} 
\classe{np} 
\begin{définition}\pfra{bout du nez}\end{définition}
\begin{définition}\pcmn{鼻尖}\end{définition}\end{entrée}

\begin{entrée}{tɯ-ɕnɤkɯm}{}{ⓔtɯ-ɕnɤkɯm} 
\classe{np} 
\begin{définition}\pfra{partie entre le nez et la lèvre}\end{définition}
\begin{définition}\pcmn{人中}\end{définition}\end{entrée}

\begin{entrée}{tɯ-ɕnɤmtsrɯɣ}{}{ⓔtɯ-ɕnɤmtsrɯɣ} 
\classe{np} 
\begin{définition}\pfra{morve liquide}\end{définition}
\begin{définition}\pcmn{湿鼻涕}\end{définition}\relationsémantique{参考}{\lien{ⓔtɯ-ɕnaβ}{tɯ-ɕnaβ}}\end{entrée}

\begin{entrée}{tɯ-ɕnɤɴqhi}{}{ⓔtɯ-ɕnɤɴqhi} 
\classe{np} 
\begin{définition}\pfra{crottes de nez}\end{définition}
\begin{définition}\pcmn{鼻子上的干鼻涕}\end{définition}\end{entrée}

\begin{entrée}{tɯ-ɕnɤse}{}{ⓔtɯ-ɕnɤse} 
\classe{np} 
\begin{définition}\pfra{saigner du nez}\end{définition}
\begin{définition}\pcmn{鼻血}\end{définition}
\begin{exemple}\pjya{a-ɕnɤse ɲɯ-ɬoʁ}\hspace{5pt}\pcmn{我正流鼻血}\end{exemple}\end{entrée}

\begin{entrée}{tɯɕoʁ}{₁}{ⓔtɯɕoʁⓗ1} 
\classe{n} 
\begin{définition}\pfra{type de servage}\end{définition}
\begin{définition}\pcmn{负责交粮食以及干徭役的农民}\end{définition}\end{entrée}

\begin{entrée}{tɯ-ɕoʁ}{₂}{ⓔtɯ-ɕoʁⓗ2} 
\classe{clf} 
\begin{définition}\pfra{une douelle}\end{définition}
\begin{définition}\pcmn{一条桶板}\end{définition}
\begin{exemple}\pjya{zɯm ɯ-ɕoʁ}\hspace{5pt}\pcmn{木桶的板}\end{exemple}\end{entrée}

\begin{entrée}{tɯ-ɕpɤβ}{}{ⓔtɯ-ɕpɤβ} 
\classe{np} 
\begin{définition}\pfra{cadavre}\end{définition}
\begin{définition}\pcmn{尸体}\end{définition}\end{entrée}

\begin{entrée}{tɯ-ɕtʂi}{}{ⓔtɯ-ɕtʂi} 
\classe{np} 
\begin{définition}\pfra{sueur}\end{définition}
\begin{définition}\pcmn{汗}\end{définition}
\begin{exemple}\pjya{a-ɕtʂi pɯ-ɬoʁ}\hspace{5pt}\pcmn{我流了汗}\end{exemple}
\begin{exemple}\pjya{ki ta-ma ɲɯ-ɴqa tɕe a-ɕtʂi ʑo pa-tɕɤt}\hspace{5pt}\pcmn{这个工作很辛苦,搞得我一身都是汗}\end{exemple}\relationsémantique{参考}{\lien{ⓔsɯɕtʂi}{sɯɕtʂi}}\end{entrée}

\begin{entrée}{tɯ-ɕtɯ}{}{ⓔtɯ-ɕtɯ} 
\classe{np} 
\begin{définition}\pfra{organe sexuel féminin}\end{définition}
\begin{définition}\pcmn{女性生殖器}\end{définition}\étymologie{stu}\end{entrée}

\begin{entrée}{tɯ-ɕɯrɲo}{}{ⓔtɯ-ɕɯrɲo} 
\classe{np} 
\begin{définition}\pfra{expérience}\end{définition}
\begin{définition}\pcmn{经验}\end{définition}
\begin{exemple}\pjya{nɤ-ɕɯrɲo kɯ-tu ci tɯ-ɕti}\hspace{5pt}\pcmn{你是个见过世面的人}\end{exemple}
\begin{exemple}\pjya{ki kɯ-fse kɤ-nɤma aʑɯɣ a-ɕɯrɲo me}\hspace{5pt}\pcmn{这种事情我没尝试过}\end{exemple}\relationsémantique{参考}{\lien{ⓔrɲo}{rɲo}}\end{entrée}

\begin{entrée}{tɯdi}{}{ⓔtɯdi} 
\classe{n} 
\begin{définition}\pfra{arc}\end{définition}
\begin{définition}\pcmn{弓}\end{définition}
\begin{exemple}\pjya{kɯɣɤrʁaʁ kɯ tɯdi to-lɤt}\hspace{5pt}\pcmn{猎人射了箭}\end{exemple}\end{entrée}

\begin{entrée}{tɯfcaʁ}{}{ⓔtɯfcaʁ} 
\classe{n} 
\begin{définition}\pfra{tissu ou morceau de cuir utilisé pour protéger les vêtements lorsque l'on porte quelque chose sur le dos}\end{définition}
\begin{définition}\pcmn{背东西时垫在背上保护衣服的麻布或皮子}\end{définition}\relationsémantique{参考}{\lien{ⓔnɤfcaʁ}{nɤfcaʁ}}\end{entrée}

\begin{entrée}{tɯfcɤr}{}{ⓔtɯfcɤr} 
\classe{n} 
\begin{définition}\pfra{poterie}\end{définition}
\begin{définition}\pcmn{泥工}\end{définition}
\begin{exemple}\pjya{tɯfcɤr tɤ-βzu-t-a}\hspace{5pt}\pcmn{我做了泥工}\end{exemple}\relationsémantique{参考}{\lien{ⓔrɤfcɤr}{rɤfcɤr}}\end{entrée}

\begin{entrée}{tɯfɕɤl}{}{ⓔtɯfɕɤl} 
\classe{n} 
\begin{définition}\pfra{diarrhée}\end{définition}
\begin{définition}\pcmn{拉肚子}\end{définition}\étymologie{bɕal}\end{entrée}

\begin{entrée}{tɯfɕɤt}{₁}{ⓔtɯfɕɤtⓗ1} 
\classe{n} 
\begin{définition}\pfra{récit (de ce dont on a été témoin)}\end{définition}
\begin{définition}\pcmn{叙述(自己的所见所闻)}\end{définition}
\begin{exemple}\pjya{a-tɯfɕɤt pɯ-βze}\hspace{5pt}\pcmn{跟我说一下你所经历的事情}\end{exemple}\relationsémantique{参考}{\lien{ⓔfɕɤtⓗ1}{fɕɤt₁}}\end{entrée}

\begin{entrée}{tɯ-fɕɤt}{₂}{ⓔtɯ-fɕɤtⓗ2} 
\classe{clf} 
\begin{définition}\pfra{un brin}\end{définition}
\begin{définition}\pcmn{一股;一根}\end{définition}
\begin{exemple}\pjya{tɤrɤm χsɯ-fɕɤt ci tɤ-kɤ-sɯpa}\hspace{5pt}\pcmn{把三张木板拼在一起(平着放)}\end{exemple}\end{entrée}

\begin{entrée}{tɯ-fkur}{}{ⓔtɯ-fkur} 
\classe{clf} 
\begin{définition}\pfra{fardeau}\end{définition}
\begin{définition}\pcmn{一背}\end{définition}
\begin{exemple}\pjya{si tɯ-fkur}\hspace{5pt}\pcmn{一背柴}\end{exemple}
\begin{exemple}\pjya{si sqɯ-fkur z-jɤ-re ra}\hspace{5pt}\pcmn{你要背十背柴回来}\end{exemple}\end{entrée}

\begin{entrée}{tɯfsɤkha}{}{ⓔtɯfsɤkha} 
\classe{n} 
\begin{définition}\pfra{aube}\end{définition}
\begin{définition}\pcmn{黎明}\end{définition}
\begin{exemple}\pjya{tɯfsɤkha ɲɤ-k-ɤβzu-ci}\hspace{5pt}\pcmn{到了黎明时分}\end{exemple}
\begin{exemple}\pjya{tɯfsɤkha ʑŋgri}\hspace{5pt}\pcmn{金星}\end{exemple}\relationsémantique{参考}{\lien{ⓔfsoʁⓗ2}{fsoʁ₂}}\relationsémantique{参考}{\lien{ⓔtɤkha}{tɤkha}}\end{entrée}

\begin{entrée}{tɯ-fsonam}{}{ⓔtɯ-fsonam} 
\classe{np} 
\begin{définition}\pfra{chance}\end{définition}
\begin{définition}\pcmn{运气}\end{définition}\étymologie{bsod.nams}\end{entrée}

\begin{entrée}{tɯftsaʁ}{}{ⓔtɯftsaʁ} 
\classe{n} 
\begin{définition}\pfra{eau qui coule dans la maison lorsqu'il pleut}\end{définition}
\begin{définition}\pcmn{下雨的时候,房子里漏水}\end{définition}
\begin{exemple}\pjya{tɯftsaʁ tʂhɤtnɤtʂhɤt ʑo ɲɯ-nɯftsaʁ}\end{exemple}
\begin{exemple}\pjya{tɯftsaʁ tʂhɤtnɤtʂhɤt ʑo ɲɯ-ɣi}\end{exemple}
\begin{exemple}\pjya{tɯftsaʁ tʂhɤtnɤtʂhɤt ʑo ɲɯ-ti}\hspace{5pt}\pcmn{一滴一滴地漏水}\end{exemple}\relationsémantique{参考}{\lien{ⓔnɯftsaʁ}{nɯftsaʁ}}\end{entrée}

\begin{entrée}{tɯɣ}{₂}{ⓔtɯɣⓗ2} 
\classe{n} 
\begin{définition}\pfra{poison}\end{définition}
\begin{définition}\pcmn{毒}\end{définition}\étymologie{dug}\end{entrée}

\begin{entrée}{tɯɣ}{₁}{ⓔtɯɣⓗ1} 
\classe{vi} \paradigme{dir}{\_}
\begin{définition}\pfra{entrer en contact avec}\end{définition}
\begin{définition}\pcmn{接触到,碰到}\end{définition}
\begin{exemple}\pjya{ɯ-thoʁ pɯ-tɯɣ}\hspace{5pt}\pcmn{着地了}\end{exemple}
\begin{sous-entrée}{sɯxtɯɣ}{ⓔtɯɣⓗ1ⓝsɯxtɯɣ} 
\classe{vt} 
\begin{définition}\pfra{mettre en contact avec}\end{définition}
\begin{définition}\pcmn{使……接触到}\end{définition}
\begin{exemple}\pjya{laχtɕha thɯ-sthoʁ-a tɕe, znde ɯ-taʁ thɯ-sɯxtɯɣ-a}\hspace{5pt}\pcmn{我把东西推过去,靠到墙上了}\end{exemple}\end{sous-entrée}

\end{entrée}

\begin{entrée}{tɯ-ɣdɤt}{}{ⓔtɯ-ɣdɤt} 
\classe{clf} 
\begin{définition}\pfra{section}\end{définition}
\begin{définition}\pcmn{一段}\end{définition}
\begin{exemple}\pjya{tʂu tɯ-ɣdɤt}\hspace{5pt}\pcmn{一段路}\end{exemple}
\begin{exemple}\pjya{tɤ-ri χsɯ-ɣdɤt tɤ-sɯxɕe-t-a (tɤ-βzu-t-a; tɤ-lat-a; nɯ-sɤɣri-t-a)}\hspace{5pt}\pcmn{我把线切成了三段}\end{exemple}\end{entrée}

\begin{entrée}{tɯ-ɣjɤn}{}{ⓔtɯ-ɣjɤn} 
\classe{clf} 
\begin{définition}\pfra{fois}\end{définition}
\begin{définition}\pcmn{一次}\end{définition}
\begin{exemple}\pjya{χsjɤn; χsɯ-ɣjɤn}\hspace{5pt}\pcmn{三次}\end{exemple}\end{entrée}

\begin{entrée}{tɯɣɟaβ}{}{ⓔtɯɣɟaβ} 
\classe{n} 
\begin{définition}\pfra{barattage}\end{définition}
\begin{définition}\pcmn{搅酥油}\end{définition}
\begin{exemple}\pjya{ta-mar tɯ-tɯɣɟaβ}\hspace{5pt}\pcmn{一桶酥油}\end{exemple}\relationsémantique{参考}{\lien{ⓔɣɟaβ}{ɣɟaβ}}\end{entrée}

\begin{entrée}{tɯ-ɣli}{}{ⓔtɯ-ɣli} 
\classe{np} 
\begin{définition}\pfra{engrais, purin}\end{définition}
\begin{définition}\pcmn{肥料;粪}\end{définition}\relationsémantique{参考}{\lien{ⓔaɣɯɣli}{aɣɯɣli}}\end{entrée}

\begin{entrée}{tɯ-ɣmaz}{}{ⓔtɯ-ɣmaz} 
\classe{np} 
\begin{définition}\pfra{blessure}\end{définition}
\begin{définition}\pcmn{伤口}\end{définition}
\begin{exemple}\pjya{ɯ-ɣmaz pɯ-tɕat-a}\hspace{5pt}\pcmn{我给他留了伤口}\end{exemple}\relationsémantique{参考}{\lien{ⓔnɯɣmaz}{nɯɣmaz}}\end{entrée}

\begin{entrée}{tɯ-ɣmɤr}{}{ⓔtɯ-ɣmɤr} 
\classe{np} 
\begin{définition}\pfra{bouche}\end{définition}
\begin{définition}\pcmn{嘴}\end{définition}\end{entrée}

\begin{entrée}{tɯ-ɣmba}{}{ⓔtɯ-ɣmba} 
\classe{np} 
\begin{définition}\pfra{joue}\end{définition}
\begin{définition}\pcmn{腮}\end{définition}\end{entrée}

\begin{entrée}{tɯ-ɣmbaɕɤrɯ}{}{ⓔtɯ-ɣmbaɕɤrɯ} 
\classe{np} 
\begin{définition}\pfra{pommettes}\end{définition}
\begin{définition}\pcmn{颧骨}\end{définition}\end{entrée}

\begin{entrée}{tɯ-ɣmbɤβ}{}{ⓔtɯ-ɣmbɤβ} 
\classe{np} 
\begin{définition}\pfra{pustule}\end{définition}
\begin{définition}\pcmn{脓包}\end{définition}\end{entrée}

\begin{entrée}{tɯ-ɣmɯr}{}{ⓔtɯ-ɣmɯr} 
\classe{clf} 
\begin{définition}\pfra{un soir}\end{définition}
\begin{définition}\pcmn{一天晚上}\end{définition}
\begin{exemple}\pjya{nɯ ɯ-ɣmɯr}\hspace{5pt}\pcmn{那一天晚上}\end{exemple}\relationsémantique{参考}{\lien{ⓔjɯɣmɯr}{jɯɣmɯr}}\relationsémantique{参考}{\lien{ⓔɕɯŋgɯmɯr}{ɕɯŋgɯmɯr}}\end{entrée}

\begin{entrée}{tɯ-ɣna}{}{ⓔtɯ-ɣna} 
\classe{clf} 
\begin{définition}\pfra{trente boisseaux}\end{définition}
\begin{définition}\pcmn{三十升}\end{définition}\end{entrée}

\begin{entrée}{tɯ-ɣndʑɤr}{}{ⓔtɯ-ɣndʑɤr} 
\classe{np} 
\begin{définition}\pfra{farine, tsampa}\end{définition}
\begin{définition}\pcmn{面粉;糌粑}\end{définition}\relationsémantique{参考}{\lien{ⓔɣndʑɯr}{ɣndʑɯr}}\end{entrée}

\begin{entrée}{tɯ-ɣɲi}{}{ⓔtɯ-ɣɲi} 
\classe{np} 
\begin{définition}\pfra{ami, allié}\end{définition}
\begin{définition}\pcmn{朋友;友人}\end{définition}\relationsémantique{反义词}{\lien{ⓔʁgra}{ʁgra}}\end{entrée}

\begin{entrée}{tɯɣur}{}{ⓔtɯɣur} 
\classe{n} 
\begin{définition}\pfra{givre}\end{définition}
\begin{définition}\pcmn{霜}\end{définition}
\begin{exemple}\pjya{tɯɣur pjɤ-ta}\hspace{5pt}\pcmn{下了霜}\end{exemple}\relationsémantique{参考}{\lien{ⓔɕŋɤr}{ɕŋɤr}}\end{entrée}

\begin{entrée}{tɯ-ɣrɤz}{}{ⓔtɯ-ɣrɤz} 
\classe{np} 
\begin{définition}\pfra{ensemble}\end{définition}
\begin{définition}\pcmn{跟别人一起}\end{définition}\end{entrée}

\begin{entrée}{tɯɣro}{}{ⓔtɯɣro} 
\classe{n} 
\begin{définition}\pfra{paille}\end{définition}
\begin{définition}\pcmn{干草}\end{définition}
\begin{exemple}\pjya{tɯɣro sɤ-ɕkho}\hspace{5pt}\pcmn{晒草的地方}\end{exemple}\end{entrée}

\begin{entrée}{tɯ-ɣrɯmke}{}{ⓔtɯ-ɣrɯmke} 
\classe{np} 
\begin{définition}\pfra{poignet}\end{définition}
\begin{définition}\pcmn{手腕}\end{définition}\relationsémantique{参考}{\lien{ⓔtɯ-zgrɯ}{tɯ-zgrɯ}}\end{entrée}

\begin{entrée}{tɯɣurʑaʁ}{}{ⓔtɯɣurʑaʁ} 
\classe{n} 
\begin{définition}\pfra{blé d'hiver}\end{définition}
\begin{définition}\pcmn{冬种}\end{définition}\end{entrée}

\begin{entrée}{tɯ-ja}{}{ⓔtɯ-ja} 
\classe{np} 
\begin{définition}\pfra{grand frère, grande sœur (terme utilisé par les nobles)}\end{définition}
\begin{définition}\pcmn{哥哥;姐姐(贵族用语)}\end{définition}\end{entrée}

\begin{entrée}{tɯ-jaʁ}{}{ⓔtɯ-jaʁ} 
\classe{np} 
\begin{définition}\pfra{main; bras}\end{définition}
\begin{définition}\pcmn{手}\end{définition}
\begin{exemple}\pjya{a-jaʁqhu / a-jaʁ ɯ-qhu}\hspace{5pt}\pcmn{我的手背}\end{exemple}
\begin{exemple}\pjya{a-jaχpa}\hspace{5pt}\pcmn{我的手掌}\end{exemple}
\begin{exemple}\pjya{jiɕqha nɯ ɯ-jaʁ kɯ-rɲɟi ɕi ŋu}\hspace{5pt}\pcmn{那个人喜欢偷东西}\end{exemple}
\begin{exemple}\pjya{ɯ-jaʁ kɯ-βdi}\hspace{5pt}\pcmn{手艺很好的人}\end{exemple}\end{entrée}

\begin{entrée}{tɯ-jaʁfkɯm}{}{ⓔtɯ-jaʁfkɯm} 
\classe{np} 
\begin{définition}\pfra{gant}\end{définition}
\begin{définition}\pcmn{手套}\end{définition}\end{entrée}

\begin{entrée}{tɯ-jaʁmu}{}{ⓔtɯ-jaʁmu} 
\classe{np} 
\begin{définition}\pfra{pouce}\end{définition}
\begin{définition}\pcmn{大拇指}\end{définition}\end{entrée}

\begin{entrée}{tɯ-jaʁmɤχa/\variante{tɯ-jaʁmuχa}}{}{ⓔtɯ-jaʁmɤχa} 
\classe{np} 
\begin{définition}\pfra{espace entre le pouce et l'index}\end{définition}
\begin{définition}\pcmn{虎口}\end{définition}\end{entrée}

\begin{entrée}{tɯ-jaʁmundzoʁ}{}{ⓔtɯ-jaʁmundzoʁ} 
\classe{clf} 
\begin{définition}\pfra{un doigt}\end{définition}
\begin{définition}\pcmn{一根指头}\end{définition}\relationsémantique{参考}{\lien{ⓔtɯ-ndzoʁ}{tɯ-ndzoʁ}}\end{entrée}

\begin{entrée}{tɯ-jaʁndzu}{}{ⓔtɯ-jaʁndzu} 
\classe{np} 
\begin{définition}\pfra{doigt}\end{définition}
\begin{définition}\pcmn{手指}\end{définition}\end{entrée}

\begin{entrée}{tɯ-jaʁndzu aŋɤn}{}{ⓔtɯ-jaʁndzu aŋɤn} 
\classe{np} 
\begin{définition}\pfra{auriculaire}\end{définition}
\begin{définition}\pcmn{小指}\end{définition}\end{entrée}

\begin{entrée}{tɯ-jaʁndzumaŋlo}{}{ⓔtɯ-jaʁndzumaŋlo} 
\classe{np} 
\begin{définition}\pfra{index}\end{définition}
\begin{définition}\pcmn{食指}\end{définition}\end{entrée}

\begin{entrée}{tɯ-jaʁndzumɤpaχcɤl}{}{ⓔtɯ-jaʁndzumɤpaχcɤl} 
\classe{np} 
\begin{définition}\pfra{majeur}\end{définition}
\begin{définition}\pcmn{中指}\end{définition}\end{entrée}

\begin{entrée}{tɯ-jaʁqhu}{}{ⓔtɯ-jaʁqhu} 
\classe{np} 
\begin{définition}\pfra{dessus de la main}\end{définition}
\begin{définition}\pcmn{手背}\end{définition}\end{entrée}

\begin{entrée}{tɯ-jaʁsta}{}{ⓔtɯ-jaʁsta} 
\classe{n} 
\begin{définition}\pfra{assurance, entraînement}\end{définition}
\begin{définition}\pcmn{心中有数;熟练}\end{définition}
\begin{exemple}\pjya{a-jaʁsta nɯ-aβzu}\hspace{5pt}\pcmn{我心中有数要怎么做}\end{exemple}
\begin{exemple}\pjya{tɤ-scoz kɤ-nɤma nɯ aʑo a-jaʁsta ɕti}\hspace{5pt}\pcmn{我对写字(整理文件)心中有数}\end{exemple}
\begin{exemple}\pjya{tɕhi kɤ-nɤma pɯ-nnɯ-ŋɯ-ŋu nɯnɯ tɯ-jaʁsta ɲɯ-βze kɯ-ra ɲɯ-ɕti ɲɯ-ŋu, nɯ maʁ nɤ kɤ-ɤɣɯmphɯphru mɯ́j-khɯ}\hspace{5pt}\pcmn{无论什么工作都要练习才行}\end{exemple}\relationsémantique{参考}{\lien{ⓔtɯ-jaʁ}{tɯ-jaʁ}}\end{entrée}

\begin{entrée}{tɯ-jaχpa}{}{ⓔtɯ-jaχpa} 
\classe{np} 
\begin{définition}\pfra{paume}\end{définition}
\begin{définition}\pcmn{手掌}\end{définition}
\begin{exemple}\pjya{jaχpa rɯmu}\hspace{5pt}\pcmn{手纹}\end{exemple}\relationsémantique{参考}{\lien{ⓔtɯ-jaʁ}{tɯ-jaʁ}}\relationsémantique{参考}{\lien{ⓔpaⓗ3ⓝɯ-pa}{ɯ-pa}}\end{entrée}

\begin{entrée}{tɯji}{₂}{ⓔtɯjiⓗ2} 
\classe{n} 
\begin{définition}\pfra{semailles}\end{définition}
\begin{définition}\pcmn{播种}\end{définition}
\begin{exemple}\pjya{tɯji to-mda}\hspace{5pt}\pcmn{播种的时间到了}\end{exemple}\relationsémantique{参考}{\lien{ⓔji}{ji}}\end{entrée}

\begin{entrée}{tɯ-ji}{₁}{ⓔtɯ-jiⓗ1} 
\classe{np} 
\begin{définition}\pfra{champs}\end{définition}
\begin{définition}\pcmn{田地}\end{définition}\relationsémantique{参考}{\lien{ⓔji}{ji}}\end{entrée}

\begin{entrée}{tɯjimŋu}{}{ⓔtɯjimŋu} 
\classe{n} 
\begin{définition}\pfra{le bord du champs du côté de la montagne}\end{définition}
\begin{définition}\pcmn{靠山的田边}\end{définition}\end{entrée}

\begin{entrée}{tɯjindo}{}{ⓔtɯjindo} 
\classe{n} 
\begin{définition}\pfra{le bord du champs du côte opposé à la montagne}\end{définition}
\begin{définition}\pcmn{靠水的田边,靠山那一边的对面}\end{définition}\end{entrée}

\begin{entrée}{tɯ-jlɤβ}{}{ⓔtɯ-jlɤβ} 
\classe{np} 
\begin{définition}\pfra{fils de trame}\end{définition}
\begin{définition}\pcmn{纬线}\end{définition}\relationsémantique{反义词}{\lien{ⓔtɤ-ʁjar}{tɤ-ʁjar}}\relationsémantique{参考}{\lien{ⓔjlɤβndʑu}{jlɤβndʑu}}\end{entrée}

\begin{entrée}{tɯ-jmetɕɯrɯrɯ}{}{ⓔtɯ-jmetɕɯrɯrɯ} 
\classe{n} 
\begin{définition}\pfra{excroissance du bassin}\end{définition}
\begin{définition}\pcmn{尾椎骨}\end{définition}\end{entrée}

\begin{entrée}{tɯ-jmŋo}{}{ⓔtɯ-jmŋo} 
\classe{np} 
\begin{définition}\pfra{rêve}\end{définition}
\begin{définition}\pcmn{梦}\end{définition}
\begin{exemple}\pjya{ɯ-jmŋo ko-ntɕhɤr}\hspace{5pt}\pcmn{他做了梦}\end{exemple}\relationsémantique{参考}{\lien{ⓔɣɤjmŋo}{ɣɤjmŋo}}\end{entrée}

\begin{entrée}{tɯjno}{}{ⓔtɯjno} 
\classe{n} 
\begin{définition}\pfra{légume}\end{définition}
\begin{définition}\pcmn{蔬菜}\end{définition}
\begin{exemple}\pjya{tɯjno tɤ-kɤ-xsɯr}\hspace{5pt}\pcmn{炒的菜}\end{exemple}\end{entrée}

\begin{entrée}{tɯjnozwa}{}{ⓔtɯjnozwa} 
\classe{n} 
\begin{définition}\pfra{légumes dans la soupe}\end{définition}
\begin{définition}\pcmn{菜汤里的菜叶}\end{définition}\end{entrée}

\begin{entrée}{tɯjpu}{}{ⓔtɯjpu} 
\classe{n} 
\begin{définition}\pfra{nourriture}\end{définition}
\begin{définition}\pcmn{粮食}\end{définition}\end{entrée}

\begin{entrée}{tɯ-jroʁ}{}{ⓔtɯ-jroʁ} 
\classe{clf} 
\begin{définition}\pfra{une rangée}\end{définition}
\begin{définition}\pcmn{一行,一路线}\end{définition}
\begin{exemple}\pjya{qaj tɯ-jroʁ}\hspace{5pt}\pcmn{一行麦子}\end{exemple}\relationsémantique{同义词}{\lien{ⓔtɯ-rɣɯt}{tɯ-rɣɯt}}\relationsémantique{参考}{\lien{ⓔtɤ-jroʁ}{tɤ-jroʁ}}\relationsémantique{参考}{\lien{ⓔnɯjroʁ}{nɯjroʁ}}\relationsémantique{参考}{\lien{ⓔrɤjroʁ}{rɤjroʁ}}\end{entrée}

\begin{entrée}{tɯjʁo}{}{ⓔtɯjʁo} 
\classe{n} 
\begin{définition}\pfra{insulte}\end{définition}
\begin{définition}\pcmn{骂人的话}\end{définition}
\begin{exemple}\pjya{tɯjʁo ta-khɤt}\hspace{5pt}\pcmn{他骂了很久}\end{exemple}
\begin{exemple}\pjya{tɯjʁo χɕu}\hspace{5pt}\pcmn{他骂人很厉害}\end{exemple}\end{entrée}

\begin{entrée}{tɯ-jɯɣ}{}{ⓔtɯ-jɯɣ} 
\classe{clf} 
\begin{définition}\pfra{pièce de tissu}\end{définition}
\begin{définition}\pcmn{一匹布}\end{définition}
\begin{exemple}\pjya{raz tɯ-jɯɣ}\hspace{5pt}\pcmn{一匹布}\end{exemple}\end{entrée}

\begin{entrée}{tɯɟo}{}{ⓔtɯɟo} 
\classe{n} 
\begin{définition}\pfra{fantôme}\end{définition}
\begin{définition}\pcmn{鬼(草登话)}\end{définition}\end{entrée}

\begin{entrée}{tɯ-ɟom}{}{ⓔtɯ-ɟom} 
\classe{clf} 
\begin{définition}\pfra{longueur des deux bras étendus}\end{définition}
\begin{définition}\pcmn{一庹【一排】}\end{définition}\étymologie{ⁿdom.pa}\end{entrée}

\begin{entrée}{tɯ-ɟrɯɣ}{}{ⓔtɯ-ɟrɯɣ} 
\classe{clf} 
\begin{définition}\pfra{plein de choses en désordre}\end{définition}
\begin{définition}\pcmn{一大堆(很乱)}\end{définition}
\begin{exemple}\pjya{laχtɕha tɯ-ɟrɯɣ ʑo jo-ɣɯt}\hspace{5pt}\pcmn{他带来一大堆很乱的东西}\end{exemple}\relationsémantique{参考}{\lien{ⓔɟrɯɣɟrɯɣ}{ɟrɯɣɟrɯɣ}}\end{entrée}

\begin{entrée}{tɯ-ɟɯɣ}{}{ⓔtɯ-ɟɯɣ} 
\classe{clf} 
\begin{définition}\pfra{un troupeau}\end{définition}
\begin{définition}\pcmn{一群}\end{définition}
\begin{exemple}\pjya{mbro tɯ-ɟɯɣ}\hspace{5pt}\pcmn{一群马}\end{exemple}\end{entrée}

\begin{entrée}{tɯ-ku}{}{ⓔtɯ-ku} 
\classe{np} \sens{1}
\begin{définition}\pfra{tête}\end{définition}
\begin{définition}\pcmn{头}\end{définition}
\begin{exemple}\pjya{a-ku nɯ-βze}\hspace{5pt}\pcmn{给我编头发吧}\end{exemple}
\begin{exemple}\pjya{ndʑi-ku a-nɯ-tɯ-ɤnɯβzɯβzu-ndʑi je}\hspace{5pt}\pcmn{你们互相编头发吧}\end{exemple}\sens{2}
\begin{définition}\pfra{haut}\end{définition}
\begin{définition}\pcmn{上面}\end{définition}
\begin{exemple}\pjya{sɯjno ɣɯ ɯ-ku}\hspace{5pt}\pcmn{草的叶子和茎}\end{exemple}\relationsémantique{参考}{\lien{ⓔnɤkɤtɕhɯ}{nɤkɤtɕhɯ}}\relationsémantique{参考}{\lien{ⓔkɤlu}{kɤlu}}\relationsémantique{参考}{\lien{ⓔɯ-kɤlɤjme}{ɯ-kɤlɤjme}}
\begin{sous-entrée}{ɯ-kuɕɯku}{ⓔtɯ-kuⓢ2ⓝɯ-kuɕɯku} 
\classe{np} 
\begin{définition}\pfra{sommet}\end{définition}
\begin{définition}\pcmn{最顶端}\end{définition}
\begin{exemple}\pjya{pɣɤtɕɯ si ɯ-kuɕɯku ʑo zɯ ko-zo}\hspace{5pt}\pcmn{鸟落在树的顶端}\end{exemple}\end{sous-entrée}

\begin{sous-entrée}{tɯ-ku,ta}{ⓔtɯ-kuⓢ2ⓝtɯ-ku,ta}
\begin{définition}\pfra{s'allonger, poser la tête}\end{définition}
\begin{définition}\pcmn{躺下}\end{définition}
\begin{exemple}\pjya{a-ku pɯ-nɯ-ta-t-a tɕe pjɤ-nɯʑɯβ-a}\hspace{5pt}\pcmn{我躺下睡觉了}\end{exemple}\end{sous-entrée}

\end{entrée}

\begin{entrée}{tɯ-kɤcɯɣ}{}{ⓔtɯ-kɤcɯɣ} 
\classe{np} 
\begin{définition}\pfra{sommet de la tête, fontanelle}\end{définition}
\begin{définition}\pcmn{头顶;囟门}\end{définition}\end{entrée}

\begin{entrée}{tɯ-kɤftɕaka}{}{ⓔtɯ-kɤftɕaka} 
\classe{np} 
\begin{définition}\pfra{coiffure, bijoux portés sur la tête}\end{définition}
\begin{définition}\pcmn{头饰}\end{définition}\relationsémantique{参考}{\lien{ⓔtɯ-ku}{tɯ-ku}}\relationsémantique{参考}{\lien{ⓔftɕaka}{ftɕaka}}\end{entrée}

\begin{entrée}{tɯ-kɤlɤmɲaʁ}{}{ⓔtɯ-kɤlɤmɲaʁ} 
\classe{np} 
\begin{définition}\pfra{traits du visage}\end{définition}
\begin{définition}\pcmn{面容的五官}\end{définition}\relationsémantique{参考}{\lien{ⓔtɯ-mɤlɤjaʁ}{tɯ-mɤlɤjaʁ}}\relationsémantique{参考}{\lien{ⓔtɯ-ku}{tɯ-ku}}\relationsémantique{参考}{\lien{ⓔtɯ-mɲaʁ}{tɯ-mɲaʁ}}\end{entrée}

\begin{entrée}{tɯ-kɤpɤla}{}{ⓔtɯ-kɤpɤla} 
\classe{np} 
\begin{définition}\pfra{sommet du crâne}\end{définition}
\begin{définition}\pcmn{头盖骨}\end{définition}\étymologie{kapāla}\end{entrée}

\begin{entrée}{tɯ-kɤrnoʁ,mtɕɯr}{}{ⓔtɯ-kɤrnoʁ,mtɕɯr} 
\classe{vi} 
\begin{définition}\pfra{avoir un vertige}\end{définition}
\begin{définition}\pcmn{头晕}\end{définition}
\begin{exemple}\pjya{a-kɤrnoʁ ɲɯ-mtɕɯr ntsɯ pɯ-ŋu.}\hspace{5pt}\pcmn{我当时总是头晕}\end{exemple}\end{entrée}

\begin{entrée}{tɯ-kha}{}{ⓔtɯ-kha} 
\classe{clf} 
\begin{définition}\pfra{pied}\end{définition}
\begin{définition}\pcmn{尺}\end{définition}\étymologie{kʰa}\end{entrée}

\begin{entrée}{tɯ-khɤftsɯɣ}{}{ⓔtɯ-khɤftsɯɣ} 
\classe{clf} 
\begin{définition}\pfra{un tour d'aiguille}\end{définition}
\begin{définition}\pcmn{一针(缝衣服的时候)}\end{définition}\end{entrée}

\begin{entrée}{tɯ-khɤl}{}{ⓔtɯ-khɤl} 
\classe{clf} 
\begin{définition}\pfra{endroit}\end{définition}
\begin{définition}\pcmn{一个地方}\end{définition}
\begin{exemple}\pjya{jiʑo zgo tɤ-ari tɕe, χsɯ-khɤl tɤ-nɯna-j}\hspace{5pt}\pcmn{我们上山时,在三个不同的地方休息了一下}\end{exemple}\relationsémantique{参考}{\lien{ⓔɯ-khɯkhɤl}{ɯ-khɯkhɤl}}\relationsémantique{参考}{\lien{ⓔarɤkhɯmkhɤl}{arɤkhɯmkhɤl}}\étymologie{kʰol?}\end{entrée}

\begin{entrée}{tɯ-khi}{}{ⓔtɯ-khi} 
\classe{np} 
\begin{définition}\pfra{chance}\end{définition}
\begin{définition}\pcmn{运气}\end{définition}
\begin{exemple}\pjya{ji-khi ma jisŋi tɯ-mɯ ɲɯ-jɯm}\hspace{5pt}\pcmn{我们很幸运今天天气很好}\end{exemple}
\begin{exemple}\pjya{jɯfɕɯr, ji-rɣa ra kɯ @cai nɯ́-wɣ-mbi-j tɕe ji-khi pɯ-ŋu}\hspace{5pt}\pcmn{昨天邻居给了我们菜,我们很幸运}\end{exemple}
\begin{exemple}\pjya{nɤ-khi ɲɯ-ŋgɯ}\hspace{5pt}\pcmn{你运气真好}\end{exemple}\end{entrée}

\begin{entrée}{tɯ-khroŋkhroŋ}{}{ⓔtɯ-khroŋkhroŋ} 
\classe{np} 
\begin{définition}\pfra{trachée}\end{définition}
\begin{définition}\pcmn{喉管}\end{définition}\end{entrée}

\begin{entrée}{tɯ-khɯr}{}{ⓔtɯ-khɯr} 
\classe{np} 
\begin{définition}\pfra{place dans la hiérarchie}\end{définition}
\begin{définition}\pcmn{官职}\end{définition}\étymologie{kʰur}\end{entrée}

\begin{entrée}{tɯ-kuŋa}{}{ⓔtɯ-kuŋa} 
\classe{n} 
\begin{définition}\pfra{col}\end{définition}
\begin{définition}\pcmn{衣领}\end{définition}\relationsémantique{同义词}{\lien{ⓔtɯ-mkɤscur}{tɯ-mkɤscur}}\end{entrée}

\begin{entrée}{tɯkon}{}{ⓔtɯkon} 
\classe{n} 
\begin{définition}\pfra{pouvoir}\end{définition}
\begin{définition}\pcmn{权力}\end{définition}\étymologie{fn:权}\end{entrée}

\begin{entrée}{tɯkrɤz}{}{ⓔtɯkrɤz} 
\classe{n} 
\begin{définition}\pfra{discussion}\end{définition}
\begin{définition}\pcmn{商量}\end{définition}
\begin{exemple}\pjya{nɯ-tɯkrɤz to-ɣi}\hspace{5pt}\pcmn{他们商议好了}\end{exemple}\relationsémantique{参考}{\lien{ⓔrɤkrɤz}{rɤkrɤz}}\relationsémantique{参考}{\lien{ⓔnɯkrɤz}{nɯkrɤz}}\étymologie{gros}\end{entrée}

\begin{entrée}{tɯ-kri}{}{ⓔtɯ-kri} 
\classe{np} 
\begin{définition}\pfra{huile}\end{définition}
\begin{définition}\pcmn{油}\end{définition}
\begin{exemple}\pjya{nɤ-tɯ-kri ɲɯ-sɤle-a}\hspace{5pt}\pcmn{我给你熬一点油}\end{exemple}\end{entrée}

\begin{entrée}{tɯkrimgo}{}{ⓔtɯkrimgo} 
\classe{n} 
\begin{définition}\pfra{pain frit}\end{définition}
\begin{définition}\pcmn{油馍馍}\end{définition}\end{entrée}

\begin{entrée}{tɯ-kɯr}{}{ⓔtɯ-kɯr} 
\classe{np} 
\begin{définition}\pfra{bouche}\end{définition}
\begin{définition}\pcmn{嘴}\end{définition}
\begin{exemple}\pjya{nɤʑo nɤ-kɯr ɯ-tɯ-wxti!}\hspace{5pt}\pcmn{你真自吹自捧!}\end{exemple}\end{entrée}

\begin{entrée}{tɯlu}{}{ⓔtɯlu} 
\classe{n}  
\grammaire{n.lieu} 
\begin{définition}\pfra{l'un des hameaux de Kamnyu}\end{définition}
\begin{définition}\pcmn{干木鸟的大队之一}\end{définition}\end{entrée}

\begin{entrée}{tɯl}{}{ⓔtɯl} 
\classe{vi} \paradigme{dir}{nɯ-}
\begin{définition}\pfra{devenir mauvais à manger (tsampa)}\end{définition}
\begin{définition}\pcmn{面或者糌粑变味了}\end{définition}
\begin{exemple}\pjya{tɯsqar ɲɤ-tɯl}\hspace{5pt}\pcmn{糌粑变味了}\end{exemple}
\begin{exemple}\pjya{tɤjlu ɲɤ-tɯl}\hspace{5pt}\pcmn{面变味了}\end{exemple}\end{entrée}

\begin{entrée}{tɯ-lasqra}{}{ⓔtɯ-lasqra} 
\classe{np} 
\begin{définition}\pfra{raie des cheveux}\end{définition}
\begin{définition}\pcmn{头路}\end{définition}\end{entrée}

\begin{entrée}{tɯ-laz}{₁}{ⓔtɯ-lazⓗ1} 
\classe{np} 
\begin{définition}\pfra{front}\end{définition}
\begin{définition}\pcmn{额头}\end{définition}\end{entrée}

\begin{entrée}{tɯ-laz}{₂}{ⓔtɯ-lazⓗ2} 
\classe{np} 
\begin{définition}\pfra{karma}\end{définition}
\begin{définition}\pcmn{运气}\end{définition}
\begin{exemple}\pjya{a-laz ɯ-tɯ-khe}\hspace{5pt}\pcmn{我的命很苦}\end{exemple}
\begin{exemple}\pjya{a-laz tu}\hspace{5pt}\pcmn{我运气好}\end{exemple}
\begin{exemple}\pjya{nɤʑo nɤ-laz ɲɯ-sna}\hspace{5pt}\pcmn{你命好}\end{exemple}
\begin{exemple}\pjya{tɕi-laz mɤ-khɤm}\hspace{5pt}\pcmn{我们没有缘分}\end{exemple}\étymologie{las}\end{entrée}

\begin{entrée}{tɯ-lɤn}{}{ⓔtɯ-lɤn} 
\classe{np} 
\begin{définition}\pfra{réponse}\end{définition}
\begin{définition}\pcmn{答案}\end{définition}
\begin{exemple}\pjya{ɯʑo kɯ tɤrkoz ʑo a-lɤn mɯ-tu-βze ɲɯ-ŋu}\hspace{5pt}\pcmn{他故意不给我回音(不理我)}\end{exemple}\relationsémantique{参考}{\lien{ⓔɣɯlɤn}{ɣɯlɤn}}\end{entrée}

\begin{entrée}{tɯ-lɤt}{}{ⓔtɯ-lɤt} 
\classe{np} 
\begin{définition}\pfra{puîné}\end{définition}
\begin{définition}\pcmn{老二(兄弟姐妹)}\end{définition}\end{entrée}

\begin{entrée}{tɯ-lchɯɣ}{}{ⓔtɯ-lchɯɣ} 
\classe{clf} 
\begin{définition}\pfra{section (d'un sac, d'un récipient)}\end{définition}
\begin{définition}\pcmn{一节(口袋、容器)}\end{définition}\end{entrée}

\begin{entrée}{tɯ-ldʑa}{}{ⓔtɯ-ldʑa} 
\classe{clf} 
\begin{définition}\pfra{un brin}\end{définition}
\begin{définition}\pcmn{一根;一条}\end{définition}
\begin{exemple}\pjya{tɯ-mbri tɯ-ldʑa}\hspace{5pt}\pcmn{一根绳子}\end{exemple}
\begin{exemple}\pjya{tɤ-ri tɯ-ldʑa}\hspace{5pt}\pcmn{一条线}\end{exemple}
\begin{exemple}\pjya{sɯjno tɯ-ldʑa}\hspace{5pt}\pcmn{一根草}\end{exemple}
\begin{exemple}\pjya{tɯ-ci tɯ-ldʑa}\hspace{5pt}\pcmn{一条河}\end{exemple}
\begin{exemple}\pjya{ɯ-mi tɯ-ldʑa}\hspace{5pt}\pcmn{一只脚}\end{exemple}\end{entrée}

\begin{entrée}{tɯ-lpɤɣ}{}{ⓔtɯ-lpɤɣ} 
\classe{clf} 
\begin{définition}\pfra{gros morceau}\end{définition}
\begin{définition}\pcmn{一大块}\end{définition}
\begin{exemple}\pjya{tɤ-rcoʁ tɯ-lpɤɣ}\hspace{5pt}\pcmn{一滩泥}\end{exemple}
\begin{exemple}\pjya{smɤɣ tɯ-lpɤɣ}\hspace{5pt}\pcmn{一团一团的羊毛}\end{exemple}
\begin{exemple}\pjya{tɯ-mtɕhi ʁnɯ-lpɤɣ kɤ-sɤmɯrpu ʁo mbat ɕti}\hspace{5pt}\pcmn{动嘴唇倒很容易}\end{exemple}
\begin{exemple}\pjya{nɤki nɯ ɣɯ ʁnɯ-lpɤɣ nɯ χo, nɯ ma ɯ-kɤ-spa me}\hspace{5pt}\pcmn{那个人只会动嘴,没有什么本事}\end{exemple}\end{entrée}

\begin{entrée}{tɯ-ltɕhɯz}{}{ⓔtɯ-ltɕhɯz} 
\classe{clf} 
\begin{définition}\pfra{une touffe}\end{définition}
\begin{définition}\pcmn{一绺(头发)}\end{définition}
\begin{exemple}\pjya{tɯ-kɤrme tɯ-ltɕhɯz}\hspace{5pt}\pcmn{一绺头发}\end{exemple}\end{entrée}

\begin{entrée}{tɯ-lɯm}{}{ⓔtɯ-lɯm} 
\classe{np} 
\begin{définition}\pfra{dimension}\end{définition}
\begin{définition}\pcmn{体积}\end{définition}
\begin{exemple}\pjya{tɤ-fkɯm ɣɯ tɯjpu chɯ́-wɣ-rku tɕe, kɯdɤn chɯ́-wɣ-rku tɕe, ɯ-lɯm wxti, kɯrkɯn chɯ́-wɣ-rku tɕe ɯ-lɯm xtɕi}\hspace{5pt}\pcmn{口袋里的粮食装的多,(体积)就撑得大,装得少,(体积)就撑不大}\end{exemple}
\begin{exemple}\pjya{nɤki tɯrme nɯ ɯ-lɯm wxti mɤ-wxti maŋe ma ɯ-sni kɯ-xtɕɯ-xtɕi ɲɯ-ɕti}\hspace{5pt}\pcmn{这个人体积大不大都没有用,胆子很小}\end{exemple}
\begin{exemple}\pjya{nɤki kha ɯ-lɯm ndɤre ɲɯ-wxti ri, ɯ-ŋgɯ ra ku-nɯ-pe mɯ-ku-nɯ-pe mɤ-xsi}\hspace{5pt}\pcmn{这个房子看起来体积很大,不知里面好还是不好}\end{exemple}\relationsémantique{参考}{\lien{ⓔmɤlɯm}{mɤlɯm}}\end{entrée}

\begin{entrée}{tɯ-lɯz}{}{ⓔtɯ-lɯz} 
\classe{np} 
\begin{définition}\pfra{âge}\end{définition}
\begin{définition}\pcmn{年龄}\end{définition}
\begin{sous-entrée}{tɯ-lɯz,ɣi}{ⓔtɯ-lɯzⓝtɯ-lɯz,ɣi}
\begin{définition}\pfra{prendre de l'âge}\end{définition}
\begin{définition}\pcmn{上了年纪}\end{définition}
\begin{exemple}\pjya{a-wi ɯ-lɯz thɯ-ɣe}\hspace{5pt}\pcmn{我奶奶年龄大了}\end{exemple}
\begin{exemple}\pjya{a-mu ɯ-lɯz thɯ-ɣe}\hspace{5pt}\pcmn{我母亲年龄大了}\end{exemple}\end{sous-entrée}

\end{entrée}

\begin{entrée}{tɯ-ɬɯm}{}{ⓔtɯ-ɬɯm} 
\classe{clf} 
\begin{définition}\pfra{une période de sommeil}\end{définition}
\begin{définition}\pcmn{(睡)一觉}\end{définition}
\begin{exemple}\pjya{tɯ-ɬɯm pɯ-nɯʑɯβ-a}\hspace{5pt}\pcmn{我睡一觉}\end{exemple}
\begin{exemple}\pjya{a-ʑɯβ ci tɯ-ɬɯm pɯ-ɣe}\hspace{5pt}\pcmn{我睡一觉}\end{exemple}\end{entrée}

\begin{entrée}{tɯmu}{}{ⓔtɯmu} 
\classe{n} 
\begin{définition}\pfra{peur}\end{définition}
\begin{définition}\pcmn{害怕}\end{définition}
\begin{exemple}\pjya{tɯmu kɯ ɯ-lu ʑo pjɤ-cɯ}\hspace{5pt}\pcmn{他被吓得魂飞魄散}\end{exemple}\end{entrée}

\begin{entrée}{tɯ-mɤkhɤxtu}{}{ⓔtɯ-mɤkhɤxtu} 
\classe{np} 
\begin{définition}\pfra{partie supérieure du pied}\end{définition}
\begin{définition}\pcmn{脚背}\end{définition}\end{entrée}

\begin{entrée}{tɯ-mɤlɤjaʁ}{}{ⓔtɯ-mɤlɤjaʁ} 
\classe{np} 
\begin{définition}\pfra{quatre membres}\end{définition}
\begin{définition}\pcmn{四肢,手脚}\end{définition}\relationsémantique{参考}{\lien{ⓔtɯ-kɤlɤmɲaʁ}{tɯ-kɤlɤmɲaʁ}}\relationsémantique{参考}{\lien{ⓔtɯ-mi}{tɯ-mi}}\relationsémantique{参考}{\lien{ⓔtɯ-jaʁ}{tɯ-jaʁ}}\end{entrée}

\begin{entrée}{tɯ-mɤmu}{}{ⓔtɯ-mɤmu} 
\classe{np} 
\begin{définition}\pfra{gros orteil}\end{définition}
\begin{définition}\pcmn{大脚趾}\end{définition}\end{entrée}

\begin{entrée}{tɯ-mɤmke}{}{ⓔtɯ-mɤmke} 
\classe{np} 
\begin{définition}\pfra{partie de la jambe entre le mollet et la cheville}\end{définition}
\begin{définition}\pcmn{小腿和踝骨节相连的部分}\end{définition}\relationsémantique{参考}{\lien{ⓔtɯ-mi}{tɯ-mi}}\relationsémantique{参考}{\lien{ⓔtɯ-mke}{tɯ-mke}}\end{entrée}

\begin{entrée}{tɯ-mɤmɲaʁ}{}{ⓔtɯ-mɤmɲaʁ} 
\classe{np} 
\begin{définition}\pfra{astragale}\end{définition}
\begin{définition}\pcmn{距骨}\end{définition}\end{entrée}

\begin{entrée}{tɯ-mɤndzu}{}{ⓔtɯ-mɤndzu} 
\classe{np} 
\begin{définition}\pfra{doigts de pied}\end{définition}
\begin{définition}\pcmn{脚趾}\end{définition}\relationsémantique{参考}{\lien{ⓔtɯ-mi}{tɯ-mi}}\end{entrée}

\begin{entrée}{tɯ-mɤndzoʁ}{}{ⓔtɯ-mɤndzoʁ} 
\classe{np} 
\begin{définition}\pfra{orteil}\end{définition}
\begin{définition}\pcmn{脚趾}\end{définition}\end{entrée}

\begin{entrée}{tɯ-mɤndzrɯ}{}{ⓔtɯ-mɤndzrɯ} 
\classe{np} 
\begin{définition}\pfra{griffes}\end{définition}
\begin{définition}\pcmn{爪子}\end{définition}\end{entrée}

\begin{entrée}{tɯ-mɤŋɤn}{}{ⓔtɯ-mɤŋɤn} 
\classe{np} 
\begin{définition}\pfra{petit orteil}\end{définition}
\begin{définition}\pcmn{小脚趾}\end{définition}\end{entrée}

\begin{entrée}{tɯ-mɤpu}{}{ⓔtɯ-mɤpu} 
\classe{np} 
\begin{définition}\pfra{mollet}\end{définition}
\begin{définition}\pcmn{小腿}\end{définition}\end{entrée}

\begin{entrée}{tɯ-mɤpa}{}{ⓔtɯ-mɤpa} 
\classe{np} 
\begin{définition}\pfra{plante du pied}\end{définition}
\begin{définition}\pcmn{脚底}\end{définition}\relationsémantique{参考}{\lien{ⓔtɯ-mi}{tɯ-mi}}\end{entrée}

\begin{entrée}{tɯ-mɤpɤl}{}{ⓔtɯ-mɤpɤl} 
\classe{np} 
\begin{définition}\pfra{plante du pied}\end{définition}
\begin{définition}\pcmn{脚底}\end{définition}\étymologie{sbar.mo}\end{entrée}

\begin{entrée}{tɯ-mɤru}{}{ⓔtɯ-mɤru} 
\classe{np} 
\begin{définition}\pfra{trace de pieds}\end{définition}
\begin{définition}\pcmn{脚印}\end{définition}\relationsémantique{参考}{\lien{ⓔtɯ-mi}{tɯ-mi}}\end{entrée}

\begin{entrée}{tɯ-mɤsɯmsɯm}{}{ⓔtɯ-mɤsɯmsɯm} 
\classe{np} 
\begin{définition}\pfra{talon}\end{définition}
\begin{définition}\pcmn{脚跟}\end{définition}\end{entrée}

\begin{entrée}{tɯ-mɤtɕɤŋoʁ}{}{ⓔtɯ-mɤtɕɤŋoʁ} 
\classe{np} 
\begin{définition}\pfra{creux du genou}\end{définition}
\begin{définition}\pcmn{膝盖后面;腘}\end{définition}\end{entrée}

\begin{entrée}{tɯmbar}{}{ⓔtɯmbar} 
\classe{n} 
\begin{définition}\pfra{ventre de bovidé}\end{définition}
\begin{définition}\pcmn{牛肚子}\end{définition}\end{entrée}

\begin{entrée}{tɯmbaz}{}{ⓔtɯmbaz} 
\classe{n} 
\begin{définition}\pfra{coulée de boue}\end{définition}
\begin{définition}\pcmn{泥石流}\end{définition}
\begin{exemple}\pjya{tɯmbaz chɤ-ɣi}\hspace{5pt}\pcmn{发生了泥石流}\end{exemple}\end{entrée}

\begin{entrée}{tɯ-mbɤtɯm}{}{ⓔtɯ-mbɤtɯm} 
\classe{np} 
\begin{définition}\pfra{rein}\end{définition}
\begin{définition}\pcmn{肾}\end{définition}\end{entrée}

\begin{entrée}{tɯ-mbur}{}{ⓔtɯ-mbur} 
\classe{np} 
\begin{définition}\pfra{sur les cuisses (lorsqu'on est assis en tailleur)}\end{définition}
\begin{définition}\pcmn{盘着坐时,腿上的部位}\end{définition}\relationsémantique{同义词}{\lien{ⓔtɯ-rpɣo}{tɯ-rpɣo}}\end{entrée}

\begin{entrée}{tɯmbri}{}{ⓔtɯmbri} 
\classe{n} 
\begin{définition}\pfra{corde}\end{définition}
\begin{définition}\pcmn{绳子}\end{définition}\relationsémantique{参考}{\lien{ⓔnɯmbrɯmtsaʁ}{nɯmbrɯmtsaʁ}}\étymologie{ⁿbreŋ}\end{entrée}

\begin{entrée}{tɯ-mbɯ}{}{ⓔtɯ-mbɯ} 
\classe{np} 
\begin{définition}\pfra{organe sexuel masculin}\end{définition}
\begin{définition}\pcmn{男生殖器}\end{définition}\end{entrée}

\begin{entrée}{tɯ-mchi}{}{ⓔtɯ-mchi} 
\classe{np} 
\begin{définition}\pfra{bile}\end{définition}
\begin{définition}\pcmn{胆(动物)}\end{définition}\relationsémantique{参考}{\lien{ⓔprɤmchi}{prɤmchi}}\end{entrée}

\begin{entrée}{tɯ-mci}{}{ⓔtɯ-mci} 
\classe{np} 
\begin{définition}\pfra{salive}\end{définition}
\begin{définition}\pcmn{口水}\end{définition}
\begin{exemple}\pjya{tɯ-mci thɯ́-wɣ-βde ɯ-qhu kɤ-nɯɕɣɤz mɤ-khɯ}\hspace{5pt}\pcmn{吐出了口水不能收回去(你送出的礼物不能拿回来)}\end{exemple}\relationsémantique{参考}{\lien{ⓔmcɯrɯβrɯβ}{mcɯrɯβrɯβ}}\relationsémantique{参考}{\lien{ⓔmciphɯt}{mciphɯt}}\end{entrée}

\begin{entrée}{tɯmda}{}{ⓔtɯmda} 
\classe{n} 
\begin{définition}\pfra{fusils traditionnels}\end{définition}
\begin{définition}\pcmn{土枪}\end{définition}\étymologie{mda}\end{entrée}

\begin{entrée}{tɯ-mdzɤɣ}{}{ⓔtɯ-mdzɤɣ} 
\classe{np} 
\begin{définition}\pfra{clitoris}\end{définition}
\begin{définition}\pcmn{阴蒂}\end{définition}\end{entrée}

\begin{entrée}{tɯmdzoz}{}{ⓔtɯmdzoz} 
\classe{n} 
\begin{définition}\pfra{interdit, tabou}\end{définition}
\begin{définition}\pcmn{忌讳}\end{définition}\relationsémantique{参考}{\lien{ⓔmdzoz}{mdzoz}}\end{entrée}

\begin{entrée}{tɯ-mdzɯt}{}{ⓔtɯ-mdzɯt} 
\classe{np} 
\begin{définition}\pfra{ordre}\end{définition}
\begin{définition}\pcmn{命令}\end{définition}
\begin{exemple}\pjya{rɟɤlpu nɯ kɯ ɯ-pa ra ɣɯ nɯ-mdzɯt pjɤ-lɤt}\hspace{5pt}\pcmn{土司给他的属下下来命令}\end{exemple}\end{entrée}

\begin{entrée}{tɯ-mdʑu}{}{ⓔtɯ-mdʑu} 
\classe{np} 
\begin{définition}\pfra{langue}\end{définition}
\begin{définition}\pcmn{舌头}\end{définition}
\begin{exemple}\pjya{tɕhoma ɯ-mdʑu}\hspace{5pt}\pcmn{皮带的尖头}\end{exemple}\end{entrée}

\begin{entrée}{tɯ-mdʑuri}{}{ⓔtɯ-mdʑuri} 
\classe{np} 
\begin{définition}\pfra{tendon de la langue}\end{définition}
\begin{définition}\pcmn{舌头的筋}\end{définition}\end{entrée}

\begin{entrée}{tɯ-me}{}{ⓔtɯ-me} 
\classe{np} 
\begin{définition}\pfra{fille}\end{définition}
\begin{définition}\pcmn{女儿}\end{définition}\end{entrée}

\begin{entrée}{tɯ-menmaʁ}{}{ⓔtɯ-menmaʁ} 
\classe{np} 
\begin{définition}\pfra{beau-fils}\end{définition}
\begin{définition}\pcmn{女婿}\end{définition}\étymologie{mag.pa}\end{entrée}

\begin{entrée}{tɯ-mga}{}{ⓔtɯ-mga} 
\classe{np} 
\begin{définition}\pfra{ce que l'on obtient}\end{définition}
\begin{définition}\pcmn{收获}\end{définition}
\begin{exemple}\pjya{kɯmɤlɤxso pjɯ-tɯ-zdɯɣ ɲɯ-ɕti ma nɤ-mga me}\hspace{5pt}\pcmn{你白白辛苦了,没有得到任何好处}\end{exemple}\relationsémantique{同义词}{\lien{ⓔɯ-mbrɤzɯ}{ɯ-mbrɤzɯ}}\relationsémantique{参考}{\lien{ⓔnɯmga}{nɯmga}}\end{entrée}

\begin{entrée}{tɯ-mgo}{}{ⓔtɯ-mgo} 
\classe{np} 
\begin{définition}\pfra{nourriture}\end{définition}
\begin{définition}\pcmn{粮食}\end{définition}\end{entrée}

\begin{entrée}{tɯmgozmɤrɤβ}{}{ⓔtɯmgozmɤrɤβ} 
\classe{n} 
\begin{définition}\pfra{plat}\end{définition}
\begin{définition}\pcmn{菜}\end{définition}\end{entrée}

\begin{entrée}{tɯmgrɯnphoŋ}{}{ⓔtɯmgrɯnphoŋ} 
\classe{n} 
\begin{définition}\pfra{bouteille d'alcool pour offrir aux hôtes}\end{définition}
\begin{définition}\pcmn{招待客人的酒瓶}\end{définition}\end{entrée}

\begin{entrée}{tɯ-mgɯr}{}{ⓔtɯ-mgɯr} 
\classe{np} 
\begin{définition}\pfra{dos}\end{définition}
\begin{définition}\pcmn{背}\end{définition}\end{entrée}

\begin{entrée}{tɯ-mɢla}{}{ⓔtɯ-mɢla} 
\classe{clf} 
\begin{définition}\pfra{pas}\end{définition}
\begin{définition}\pcmn{一步}\end{définition}
\begin{exemple}\pjya{tɯ-mɢla kɤ-scɤt ɯ-tɤ́-cha}\hspace{5pt}\pcmn{(你儿子)会走路了吗?}\end{exemple}
\begin{exemple}\pjya{tɯ-mɢla cinɤ ma-jɤ-tɯ-te!}\hspace{5pt}\pcmn{一步都不要迈进!}\end{exemple}
\begin{exemple}\pjya{tɯ-mɢla ju-cit ŋu}\hspace{5pt}\pcmn{(小孩子会)走路}\end{exemple}\end{entrée}

\begin{entrée}{tɯmɢɯt}{}{ⓔtɯmɢɯt} 
\classe{n} 
\begin{définition}\pfra{poutre qui soutient le balcon}\end{définition}
\begin{définition}\pcmn{支撑走檐的梁}\end{définition}
\begin{exemple}\pjya{jɤɣɤt ɯ-pa stukɤr ɯ-tshɤt ɕoŋtɕa nɯ tɯmɢɯt rmi}\hspace{5pt}\pcmn{走缘下面起大梁作用的木料叫\lien{ⓔtɯmɢɯt}{tɯmɢɯt}}\end{exemple}\end{entrée}

\begin{entrée}{tɯ-mi}{}{ⓔtɯ-mi} 
\classe{np} 
\begin{définition}\pfra{jambe; pied}\end{définition}
\begin{définition}\pcmn{脚}\end{définition}
\begin{exemple}\pjya{tɯ-mi ɯ-pɤl}\hspace{5pt}\pcmn{脚掌}\end{exemple}\relationsémantique{参考}{\lien{ⓔtɯ-mɤndzu}{tɯ-mɤndzu}}\relationsémantique{参考}{\lien{ⓔamɤʁu}{amɤʁu}}\end{entrée}

\begin{entrée}{tɯ-midi}{}{ⓔtɯ-midi} 
\classe{np} 
\begin{définition}\pfra{odeur de pied}\end{définition}
\begin{définition}\pcmn{脚臭}\end{définition}
\begin{exemple}\pjya{nɤ-midi ɯ-tɯ-mnɤm nɯ}\hspace{5pt}\pcmn{你的脚很臭啊}\end{exemple}\end{entrée}

\begin{entrée}{tɯ-mɟa}{₁}{ⓔtɯ-mɟaⓗ1} 
\classe{np} 
\begin{définition}\pfra{mâchoire inférieure}\end{définition}
\begin{définition}\pcmn{下巴}\end{définition}\end{entrée}

\begin{entrée}{tɯ-mɟa}{₂}{ⓔtɯ-mɟaⓗ2} 
\classe{np} 
\begin{définition}\pfra{avantage, ce que l'on reçoit}\end{définition}
\begin{définition}\pcmn{收入}\end{définition}\relationsémantique{参考}{\lien{ⓔmɟa}{mɟa}}\end{entrée}

\begin{entrée}{tɯ-mɟɤrme}{}{ⓔtɯ-mɟɤrme} 
\classe{np} 
\begin{définition}\pfra{barbe}\end{définition}
\begin{définition}\pcmn{胡子}\end{définition}\relationsémantique{参考}{\lien{ⓔtɤ-rme}{tɤ-rme}}\end{entrée}

\begin{entrée}{tɯ-mkɤqhu}{}{ⓔtɯ-mkɤqhu} 
\classe{np} 
\begin{définition}\pfra{nuque}\end{définition}
\begin{définition}\pcmn{项,颈背【脑后勺】}\end{définition}\relationsémantique{参考}{\lien{ⓔɯ-qhu}{ɯ-qhu}}\end{entrée}

\begin{entrée}{tɯ-mkɤscur}{}{ⓔtɯ-mkɤscur} 
\classe{np} 
\begin{définition}\pfra{col}\end{définition}
\begin{définition}\pcmn{衣领}\end{définition}\relationsémantique{参考}{\lien{ⓔtɯ-mke}{tɯ-mke}}\relationsémantique{同义词}{\lien{ⓔtɯ-kuŋa}{tɯ-kuŋa}}\end{entrée}

\begin{entrée}{tɯ-mke}{}{ⓔtɯ-mke} 
\classe{np} 
\begin{définition}\pfra{cou}\end{définition}
\begin{définition}\pcmn{脖子}\end{définition}\étymologie{ske}\end{entrée}

\begin{entrée}{tɯmnɯ}{}{ⓔtɯmnɯ} 
\classe{n} 
\begin{définition}\pfra{alène}\end{définition}
\begin{définition}\pcmn{锥}\end{définition}\end{entrée}

\begin{entrée}{tɯmɲa}{}{ⓔtɯmɲa} 
\classe{n} 
\begin{définition}\pfra{flèche}\end{définition}
\begin{définition}\pcmn{箭}\end{définition}\end{entrée}

\begin{entrée}{tɯ-mɲaʁ}{}{ⓔtɯ-mɲaʁ} 
\classe{np} 
\begin{définition}\pfra{œil}\end{définition}
\begin{définition}\pcmn{眼睛}\end{définition}\relationsémantique{参考}{\lien{ⓔtɯ-kɤlɤmɲaʁ}{tɯ-kɤlɤmɲaʁ}}\end{entrée}

\begin{entrée}{tɯ-mɲaʁfkaβ}{}{ⓔtɯ-mɲaʁfkaβ} 
\classe{np} 
\begin{définition}\pfra{paupière supérieure}\end{définition}
\begin{définition}\pcmn{上眼皮}\end{définition}\relationsémantique{参考}{\lien{ⓔfkaβ}{fkaβ}}\end{entrée}

\begin{entrée}{tɯ-mɲaʁndo}{}{ⓔtɯ-mɲaʁndo} 
\classe{np} 
\begin{définition}\pfra{coin du l'œil}\end{définition}
\begin{définition}\pcmn{眼边}\end{définition}\relationsémantique{参考}{\lien{ⓔɯ-ndo}{ɯ-ndo}}\end{entrée}

\begin{entrée}{tɯ-mɲaʁrdu}{}{ⓔtɯ-mɲaʁrdu} 
\classe{np} 
\begin{définition}\pfra{prunelle}\end{définition}
\begin{définition}\pcmn{眼珠}\end{définition}\end{entrée}

\begin{entrée}{tɯ-mɲaʁrme}{}{ⓔtɯ-mɲaʁrme} 
\classe{np} 
\begin{définition}\pfra{sourcil}\end{définition}
\begin{définition}\pcmn{眉毛}\end{définition}\end{entrée}

\begin{entrée}{tɯ-mɲaʁspɯ}{}{ⓔtɯ-mɲaʁspɯ} 
\classe{np} 
\begin{définition}\pfra{chassie}\end{définition}
\begin{définition}\pcmn{眼屎}\end{définition}\relationsémantique{参考}{\lien{ⓔtɤ-spɯ}{tɤ-spɯ}}\end{entrée}

\begin{entrée}{tɯmɲɯɣ}{}{ⓔtɯmɲɯɣ} 
\classe{n} 
\begin{définition}\pfra{cancer de l'estomac}\end{définition}
\begin{définition}\pcmn{胃癌}\end{définition}
\begin{exemple}\pjya{lɯlu ɯ-rme chɯ́-wɣ-ndza tɕe, tɯmɲɯɣ βze}\hspace{5pt}\pcmn{吞了猫的毛就会得胃癌。}\end{exemple}\relationsémantique{参考}{\lien{ⓔnɯmɲɯɣ}{nɯmɲɯɣ}}\end{entrée}

\begin{entrée}{tɯ-mɲɯɣ}{}{ⓔtɯ-mɲɯɣ} 
\classe{np} 
\begin{définition}\pfra{œsophage}\end{définition}
\begin{définition}\pcmn{食道}\end{définition}\relationsémantique{参考}{\lien{ⓔnɯmɲɯɣ}{nɯmɲɯɣ}}\étymologie{mid}\end{entrée}

\begin{entrée}{tɯ-mɲɯtsi}{}{ⓔtɯ-mɲɯtsi} 
\classe{clf} 
\begin{définition}\pfra{vie}\end{définition}
\begin{définition}\pcmn{一辈子}\end{définition}
\begin{exemple}\pjya{kɯki kha ki χsɯ-tɯmɲɯtsi to-tsu (=chɤ-mda)}\hspace{5pt}\pcmn{这座房子已经住过三代人}\end{exemple}\étymologie{mi.tsʰe}\end{entrée}

\begin{entrée}{tɯmŋu}{}{ⓔtɯmŋu} 
\classe{n}  
\grammaire{n.lieu}
\grammaire{n.lieu} 
\begin{définition}\pfra{nom commun à plusieurs champs à Kamnyu}\end{définition}
\begin{définition}\pcmn{干木鸟村几块田地的统称}\end{définition}\end{entrée}

\begin{entrée}{tɯ-mpɕar}{}{ⓔtɯ-mpɕar} 
\classe{clf} 
\begin{définition}\pfra{feuille, un yuan}\end{définition}
\begin{définition}\pcmn{一张;一元}\end{définition}
\begin{exemple}\pjya{sɯjwaʁ tɯ-mpɕar}\hspace{5pt}\pcmn{一片叶子}\end{exemple}
\begin{exemple}\pjya{jɯɣi tɯ-mpɕar}\hspace{5pt}\pcmn{一张纸}\end{exemple}\end{entrée}

\begin{entrée}{tɯ-mphɯr}{}{ⓔtɯ-mphɯr} 
\classe{clf} 
\begin{définition}\pfra{rouleau, paquet}\end{définition}
\begin{définition}\pcmn{一包;一卷}\end{définition}
\begin{exemple}\pjya{raz tɯ-mphɯr}\hspace{5pt}\pcmn{一卷布}\end{exemple}\end{entrée}

\begin{entrée}{tɯ-mphɯz}{}{ⓔtɯ-mphɯz} 
\classe{np} 
\begin{définition}\pfra{fesse}\end{définition}
\begin{définition}\pcmn{屁股}\end{définition}\end{entrée}

\begin{entrée}{tɯ-mqaj}{}{ⓔtɯ-mqaj} 
\classe{np} 
\begin{définition}\pfra{critique}\end{définition}
\begin{définition}\pcmn{批评}\end{définition}
\begin{exemple}\pjya{a-mqaj pɯ-tu}\hspace{5pt}\pcmn{我被批评了}\end{exemple}\end{entrée}

\begin{entrée}{tɯ-mtɕhi}{}{ⓔtɯ-mtɕhi} 
\classe{np} \sens{1}
\begin{définition}\pfra{lèvres}\end{définition}
\begin{définition}\pcmn{嘴唇}\end{définition}\sens{2}
\begin{définition}\pfra{bouche}\end{définition}
\begin{définition}\pcmn{嘴}\end{définition}
\begin{exemple}\pjya{nɤ-mtɕhi kɤ-ndɤm}\hspace{5pt}\pcmn{你闭嘴}\end{exemple}
\begin{exemple}\pjya{cha ɯ-kɯ-tshi ɯ-mtɕhi kɯ-nɯʑɯβ ɯ-qe}\hspace{5pt}\pcmn{喝了酒的人的嘴,瞌睡的人的屁(酒话是心理话)}\end{exemple}
\begin{exemple}\pjya{nɤ-mtɕhi mɤ-mpɕɤr}\hspace{5pt}\pcmn{你讲话很不客气(没有甜言蜜语)}\end{exemple}\relationsémantique{参考}{\lien{ⓔtɯ-mtɕhi,χo}{tɯ-mtɕhi,χo}}\étymologie{mtɕʰu}\end{entrée}

\begin{entrée}{tɯ-mtɕhirme}{}{ⓔtɯ-mtɕhirme} 
\classe{np} 
\begin{définition}\pfra{moustaches}\end{définition}
\begin{définition}\pcmn{胡子,胡须}\end{définition}\end{entrée}

\begin{entrée}{tɯ-mtɕhi,χo}{}{ⓔtɯ-mtɕhi,χo} 
\classe{np}
\classe{vs} \paradigme{dir}{tɤ-}
\begin{définition}\pfra{fanfaronner}\end{définition}
\begin{définition}\pcmn{说大话}\end{définition}
\begin{exemple}\pjya{ɯ-mtɕhi ɲɯ-χo}\hspace{5pt}\pcmn{他爱说大话}\end{exemple}\relationsémantique{Component 1}{\lien{ⓔtɯ-mtɕhi}{tɯ-mtɕhi}}\relationsémantique{Component 2}{\lien{}{χo}}
\begin{sous-entrée}{tɯ-mtɕhi,ɣɤχo}{ⓔtɯ-mtɕhi,χoⓝtɯ-mtɕhi,ɣɤχo} 
\classe{np}
\classe{vt} 
\begin{définition}\pfra{fanfaronner}\end{définition}
\begin{définition}\pcmn{说大话}\end{définition}
\begin{exemple}\pjya{ɯ-mtɕhi ɲɯ-ɣɤχɤm}\hspace{5pt}\pcmn{他在说大话}\end{exemple}
\begin{exemple}\pjya{a-mtɕhi ku-ɣɤχam-a}\hspace{5pt}\pcmn{我正在说大话}\end{exemple}\relationsémantique{Component 1}{\lien{ⓔtɯ-mtɕhi}{tɯ-mtɕhi}}\relationsémantique{Component 2}{\lien{}{ɣɤχo}}\relationsémantique{参考}{\lien{ⓔtɯ-mtɕhi}{tɯ-mtɕhi}}\end{sous-entrée}

\end{entrée}

\begin{entrée}{tɯmtɕhɯ}{}{ⓔtɯmtɕhɯ} 
\classe{n} 
\begin{définition}\pfra{crachat rituel}\end{définition}
\begin{définition}\pcmn{念经时,一边吹一边吐口水,治病的方式}\end{définition}\étymologie{mtɕʰu}\end{entrée}

\begin{entrée}{tɯmtɕi}{}{ⓔtɯmtɕi} 
\classe{n} 
\begin{définition}\pfra{matin}\end{définition}
\begin{définition}\pcmn{早晨}\end{définition}\end{entrée}

\begin{entrée}{tɯmtɕiβzɤrnaʁ}{}{ⓔtɯmtɕiβzɤrnaʁ} 
\classe{n} 
\begin{définition}\pfra{très tôt le matin}\end{définition}
\begin{définition}\pcmn{大清早}\end{définition}
\begin{exemple}\pjya{tɯmtɕiβzɤrnaʁ ʑo tɤ-rɤru-a}\hspace{5pt}\pcmn{我起得很早}\end{exemple}\end{entrée}

\begin{entrée}{tɯ-mtɕoʁ}{}{ⓔtɯ-mtɕoʁ} 
\classe{clf} 
\begin{définition}\pfra{une pincée; une touffe}\end{définition}
\begin{définition}\pcmn{一撮}\end{définition}\end{entrée}

\begin{entrée}{tɯ-mthɤɣ}{}{ⓔtɯ-mthɤɣ} 
\classe{np} 
\begin{définition}\pfra{taille}\end{définition}
\begin{définition}\pcmn{腰}\end{définition}\relationsémantique{参考}{\lien{ⓔtɯ-mthɤrɴɢɤβ}{tɯ-mthɤrɴɢɤβ}}\relationsémantique{参考}{\lien{ⓔmthɯxtɕɤr}{mthɯxtɕɤr}}\end{entrée}

\begin{entrée}{tɯ-mthɤrɴɢɤβ}{}{ⓔtɯ-mthɤrɴɢɤβ} 
\classe{n} 
\begin{définition}\pfra{encolure du pantalon}\end{définition}
\begin{définition}\pcmn{裤兜}\end{définition}
\begin{exemple}\pjya{ɯ-mthɤrɴɢɤβ chɤmdɤru chɯ-nɯrʁe}\hspace{5pt}\pcmn{他把竹竿插在裤兜里}\end{exemple}
\begin{exemple}\pjya{tɯɲcɣa ɯ-mthɤrɴɢɤβ to-rʁe}\hspace{5pt}\pcmn{他把镰刀插在裤兜里}\end{exemple}\relationsémantique{参考}{\lien{ⓔtɯ-mthɤɣ}{tɯ-mthɤɣ}}\relationsémantique{参考}{\lien{ⓔrŋgɤβ}{rŋgɤβ}}\end{entrée}

\begin{entrée}{tɯ-mthɯ}{}{ⓔtɯ-mthɯ} 
\classe{np} 
\begin{définition}\pfra{bénéfice}\end{définition}
\begin{définition}\pcmn{赚到的钱(收入减去开支)}\end{définition}
\begin{exemple}\pjya{tɯ-sla nɤ-mthɯ tɕhi jamar ɣɤʑu}\hspace{5pt}\pcmn{你一个月能赚多少钱?}\end{exemple}
\begin{exemple}\pjya{ɯ-mthɯ to-ndza}\hspace{5pt}\pcmn{他赚了他的钱}\end{exemple}\relationsémantique{参考}{\lien{}{nɯmthɯ}}\end{entrée}

\begin{entrée}{tɯ-mtshi}{}{ⓔtɯ-mtshi} 
\classe{np} 
\begin{définition}\pfra{foie}\end{définition}
\begin{définition}\pcmn{肝}\end{définition}\étymologie{mtɕʰin.pa}\end{entrée}

\begin{entrée}{tɯ-mtso}{}{ⓔtɯ-mtso} 
\classe{np} 
\begin{définition}\pfra{repas du midi, goûter}\end{définition}
\begin{définition}\pcmn{带去山上吃的中午饭【路餐】、五点钟吃的那顿饭}\end{définition}
\begin{exemple}\pjya{tɯ-mtsomthɯm}\hspace{5pt}\pcmn{路餐里面的肉}\end{exemple}\end{entrée}

\begin{entrée}{tɯ-mtɯ}{}{ⓔtɯ-mtɯ} 
\classe{np} 
\begin{définition}\pfra{coiffure traditionnelle des hommes tibétains}\end{définition}
\begin{définition}\pcmn{藏族男人的传统发型}\end{définition}
\begin{exemple}\pjya{ɯ-mtɯ ɲɤ-nɯ-sɯ-ta}\hspace{5pt}\pcmn{他剃头的时候在头顶上留了一撮头发}\end{exemple}\end{entrée}

\begin{entrée}{tɯ-mɯ}{}{ⓔtɯ-mɯ} 
\classe{np} 
\begin{définition}\pfra{temps, pluie}\end{définition}
\begin{définition}\pcmn{天气;雨}\end{définition}
\begin{exemple}\pjya{tɯ-mɯ kɯ-ɤrŋi}\hspace{5pt}\pcmn{青天(天宫)}\end{exemple}
\begin{exemple}\pjya{tɯ-mɯ ɲɯ-ɤsɯ-lɤt}\hspace{5pt}\pcmn{在下雨}\end{exemple}
\begin{exemple}\pjya{tɯ-mɯ ko-lɤt (=jo-ɣɯt)}\hspace{5pt}\pcmn{下雨了}\end{exemple}
\begin{exemple}\pjya{tɯ-mɯ ci ci ku-lɤt, ci ci mɯ́j-lɤt}\hspace{5pt}\pcmn{有时候下雨,有时候不下雨}\end{exemple}
\begin{exemple}\pjya{tɯ-mɯ lɤt ɲɯ-ŋu rca}\hspace{5pt}\pcmn{快要下雨了}\end{exemple}
\begin{exemple}\pjya{tɯ-mɯ chɯ tu-ru}\hspace{5pt}\pcmn{他仰着(睡)}\end{exemple}
\begin{exemple}\pjya{jɯfɕɯr ji-mɯ pɯ-pe}\hspace{5pt}\pcmn{我们昨天(遇到了)很好的天气}\end{exemple}
\begin{exemple}\pjya{ɯ-ndzɯ mɤ-kɯ-sɤŋo ɯ-mɯ mbɯt}\hspace{5pt}\pcmn{不听劝告的人没有好下场(他的天要垮下来)}\end{exemple}
\begin{exemple}\pjya{tɯ-mɯ pɯ-pa-fkaβ ʑo ɕ-tɤ-khat-a}\hspace{5pt}\pcmn{我走遍了天下}\end{exemple}\end{entrée}

\begin{entrée}{tɯmɯɕoʁ}{}{ⓔtɯmɯɕoʁ} 
\classe{n} 
\begin{définition}\pfra{sarrasin}\end{définition}
\begin{définition}\pcmn{荞麦}\end{définition}\end{entrée}

\begin{entrée}{tɯmɯkɤmpɕi/\variante{tɯmɯ kɯmɤɕi}}{}{ⓔtɯmɯkɤmpɕi} 
\classe{n} 
\begin{définition}\pfra{paradis}\end{définition}
\begin{définition}\pcmn{天堂}\end{définition}\end{entrée}

\begin{entrée}{tɯ-mɯm}{}{ⓔtɯ-mɯm} 
\classe{clf} 
\begin{définition}\pfra{une gorgée}\end{définition}
\begin{définition}\pcmn{一口(水)}\end{définition}\end{entrée}

\begin{entrée}{tɯmɯpaʁ}{}{ⓔtɯmɯpaʁ} 
\classe{n} 
\begin{définition}\pfra{limace}\end{définition}
\begin{définition}\pcmn{蛞蝓}\end{définition}\end{entrée}

\begin{entrée}{tɯ-mɯrʁɯz}{}{ⓔtɯ-mɯrʁɯz} 
\classe{clf} 
\begin{définition}\pfra{coup de griffe}\end{définition}
\begin{définition}\pcmn{(抓)一下}\end{définition}
\begin{exemple}\pjya{lɯlu kɯ tɯ-mɯrʁɯz ci to-lɤt}\hspace{5pt}\pcmn{猫抓了一下}\end{exemple}\relationsémantique{参考}{\lien{ⓔmɯrʁɯz}{mɯrʁɯz}}\end{entrée}

\begin{entrée}{tɯmɯrtsɯɣ}{}{ⓔtɯmɯrtsɯɣ} 
\classe{n} 
\begin{définition}\pfra{pincer}\end{définition}
\begin{définition}\pcmn{捏,掐,拧}\end{définition}
\begin{exemple}\pjya{a-rʑaβ kɯ tɯmɯrtsɯɣ ta-lɤt}\hspace{5pt}\pcmn{我的妻子掐了我一下}\end{exemple}
\begin{exemple}\pjya{nɤ-qe thɯ-tɯ-rɤrɕɯβ tɕe, kutɕu tɯrme kɯmŋu ɣɤʑu-j tɕe, tɯmɯrtsɯɣ kɯmŋu tu-lat-a ra}\hspace{5pt}\pcmn{你放了屁,我们这里有五个人,所以我就要掐你五次}\end{exemple}\relationsémantique{参考}{\lien{ⓔmɯrtsɯɣ}{mɯrtsɯɣ}}\end{entrée}

\begin{entrée}{tɯmɯʁrɯm}{}{ⓔtɯmɯʁrɯm} 
\classe{n} 
\begin{définition}\pfra{si haut qu'il cache le ciel}\end{définition}
\begin{définition}\pcmn{高得遮住蓝天}\end{définition}
\begin{exemple}\pjya{kha nɯ tɯmɯʁrɯm ʑo to-zɣɯt}\hspace{5pt}\pcmn{这个房子高得遮住蓝天}\end{exemple}\relationsémantique{参考}{\lien{ⓔta-ʁrɯm}{ta-ʁrɯm}}\relationsémantique{参考}{\lien{ⓔtɯ-mɯ}{tɯ-mɯ}}\end{entrée}

\begin{entrée}{tɯ-ndzɤfkɯm/\variante{tɯ-ndzɤkɯm}}{}{ⓔtɯ-ndzɤfkɯm} 
\classe{np} 
\begin{définition}\pfra{estomac}\end{définition}
\begin{définition}\pcmn{胃}\end{définition}\end{entrée}

\begin{entrée}{tɯ-ndzɤŋgrɯm}{}{ⓔtɯ-ndzɤŋgrɯm} 
\classe{np} 
\begin{définition}\pfra{tempes}\end{définition}
\begin{définition}\pcmn{太阳穴}\end{définition}\end{entrée}

\begin{entrée}{tɯ-ndzɣi}{}{ⓔtɯ-ndzɣi} 
\classe{np} 
\begin{définition}\pfra{canines}\end{définition}
\begin{définition}\pcmn{獠牙}\end{définition}\end{entrée}

\begin{entrée}{tɯ-ndzoʁ}{}{ⓔtɯ-ndzoʁ} 
\classe{clf} 
\begin{définition}\pfra{gousse}\end{définition}
\begin{définition}\pcmn{蒜瓣}\end{définition}
\begin{exemple}\pjya{kɯmɕku tɯ-ndzoʁ}\hspace{5pt}\pcmn{一瓣蒜}\end{exemple}\end{entrée}

\begin{entrée}{tɯ-ndzrɯ}{}{ⓔtɯ-ndzrɯ} 
\classe{np} 
\begin{définition}\pfra{ongle}\end{définition}
\begin{définition}\pcmn{指甲}\end{définition}
\begin{exemple}\pjya{a-ndzrɯ nɯ-nɯ-phɯt-a}\hspace{5pt}\pcmn{我剪了指甲}\end{exemple}\end{entrée}

\begin{entrée}{tɯ-ndzʁi}{}{ⓔtɯ-ndzʁi} 
\classe{np} 
\begin{définition}\pfra{clavicule}\end{définition}
\begin{définition}\pcmn{锁骨}\end{définition}\end{entrée}

\begin{entrée}{tɯ-ndzɯ}{}{ⓔtɯ-ndzɯ} 
\classe{np}
\classe{np}
\classe{vt} \paradigme{dir}{pɯ-}
\begin{définition}\pfra{conseil}\end{définition}
\begin{définition}\pcmn{教育、劝告的话}\end{définition}
\begin{définition}\pfra{ordonner}\end{définition}
\begin{définition}\pcmn{命令}\end{définition}
\begin{exemple}\pjya{ɯ-ndzɯ mɤ-kɯ-sɤŋo ɯ-mɯ mbɯt}\hspace{5pt}\pcmn{不听劝告的人没有好下场(他的天要垮下来)}\end{exemple}
\begin{exemple}\pjya{nɤ-ndzɯ ɲɯ-βze-a}\hspace{5pt}\pcmn{我命令你}\end{exemple}
\begin{exemple}\pjya{@dangzhongyang kɯ ji-ndzɯ pa-βzu}\hspace{5pt}\pcmn{党中央命令我们}\end{exemple}\relationsémantique{Component 1}{\lien{ⓔtɯ-ndzɯ}{tɯ-ndzɯ}}\relationsémantique{Component 2}{\lien{}{βzu}}\relationsémantique{参考}{\lien{ⓔβzuⓗ1}{βzu₁}}
\begin{sous-entrée}{tɯ-ndzɯ,βzu}{ⓔtɯ-ndzɯⓝtɯ-ndzɯ,βzu}\end{sous-entrée}

\end{entrée}

\begin{entrée}{tɯ-ndʐi}{}{ⓔtɯ-ndʐi} 
\classe{np} 
\begin{définition}\pfra{peau}\end{définition}
\begin{définition}\pcmn{皮肤}\end{définition}\relationsémantique{参考}{\lien{ⓔcɤndʐi}{cɤndʐi}}\relationsémantique{参考}{\lien{ⓔtshɤndʐi}{tshɤndʐi}}\relationsémantique{参考}{\lien{ⓔqartshɤndʐi}{qartshɤndʐi}}\end{entrée}

\begin{entrée}{tɯndʐiŋga}{}{ⓔtɯndʐiŋga} 
\classe{n} 
\begin{définition}\pfra{habit en peau}\end{définition}
\begin{définition}\pcmn{皮衣}\end{définition}\end{entrée}

\begin{entrée}{tɯ-nŋa}{}{ⓔtɯ-nŋa} 
\classe{np} 
\begin{définition}\pfra{dette}\end{définition}
\begin{définition}\pcmn{债}\end{définition}
\begin{exemple}\pjya{tɯ-nŋa sqɯ-mpɕar nɯ-tɕat-a}\hspace{5pt}\pcmn{我欠了十块钱}\end{exemple}
\begin{exemple}\pjya{a-nŋa nɯ mɤʑɯ tʂam-a ra}\hspace{5pt}\pcmn{我还得还债}\end{exemple}\relationsémantique{参考}{\lien{ⓔŋa}{ŋa}}\end{entrée}

\begin{entrée}{tɯnoʁ}{}{ⓔtɯnoʁ} 
\classe{n} 
\begin{définition}\pfra{sauce}\end{définition}
\begin{définition}\pcmn{沾水;酱}\end{définition}\end{entrée}

\begin{entrée}{tɯ-ntɕhaʁ}{}{ⓔtɯ-ntɕhaʁ} 
\classe{clf} 
\begin{définition}\pfra{goutte}\end{définition}
\begin{définition}\pcmn{一滴}\end{définition}
\begin{exemple}\pjya{tɯ-mɯ ɯ-ntɕhɯ-ntɕhaʁ ɲɯ-ɤsɯ-lɤt}\hspace{5pt}\pcmn{点点滴滴地下雨(大雨点子)}\end{exemple}\end{entrée}

\begin{entrée}{tɯ-ntɕhɯr}{}{ⓔtɯ-ntɕhɯr} 
\classe{clf} 
\begin{définition}\pfra{morceau, débris}\end{définition}
\begin{définition}\pcmn{碎片}\end{définition}
\begin{exemple}\pjya{rdɤstaʁ tɯ-ntɕhɯr}\hspace{5pt}\pcmn{一个石头碎片}\end{exemple}
\begin{exemple}\pjya{tɯ-ji tɯ-ntɕhɯr}\hspace{5pt}\pcmn{一块地}\end{exemple}\relationsémantique{参考}{\lien{ⓔɯ-ntɕhantɕhɯr}{ɯ-ntɕhantɕhɯr}}\end{entrée}

\begin{entrée}{tɯ-nthoʁ}{}{ⓔtɯ-nthoʁ} 
\classe{clf} 
\begin{définition}\pfra{petit rond}\end{définition}
\begin{définition}\pcmn{小圆点}\end{définition}
\begin{exemple}\pjya{tɤ-se tɯ-nthoʁ pjɤ-ɕe}\hspace{5pt}\pcmn{地上滴了一滴血}\end{exemple}\end{entrée}

\begin{entrée}{tɯ-ntsi}{}{ⓔtɯ-ntsi} 
\classe{clf} 
\begin{définition}\pfra{un membre d'une paire}\end{définition}
\begin{définition}\pcmn{一只}\end{définition}
\begin{exemple}\pjya{tɯ-xtsa tɯ-ntsi}\hspace{5pt}\pcmn{一只鞋子}\end{exemple}
\begin{sous-entrée}{ɯ-ntsi,βzu}{ⓔtɯ-ntsiⓝɯ-ntsi,βzu}
\begin{définition}\pfra{répondre}\end{définition}
\begin{définition}\pcmn{答复}\end{définition}\relationsémantique{同义词}{\lien{}{ɯ-lɤn,βzu}}\relationsémantique{同义词}{\lien{ⓔɯ-sciⓝɯ-sci,βzu}{ɯ-sci,βzu}}\end{sous-entrée}

\end{entrée}

\begin{entrée}{tɯ-nɯ}{}{ⓔtɯ-nɯ} 
\classe{np} 
\begin{définition}\pfra{sein}\end{définition}
\begin{définition}\pcmn{乳房}\end{définition}\end{entrée}

\begin{entrée}{tɯ-ɲɤm}{}{ⓔtɯ-ɲɤm} 
\classe{np} 
\begin{définition}\pfra{chair, gras}\end{définition}
\begin{définition}\pcmn{身上的肉(人、动物)}\end{définition}
\begin{sous-entrée}{tɯ-ɲɤm,phɤn}{ⓔtɯ-ɲɤmⓝtɯ-ɲɤm,phɤn} 
\classe{np}
\classe{vs} 
\begin{définition}\pfra{très utile}\end{définition}
\begin{définition}\pcmn{有用}\end{définition}
\begin{exemple}\pjya{kɯki sɲɯɣjɯ ki a-ɲɤm wuma pɯ-phɤn ma khro tɤ-ntɕhoz-a}\hspace{5pt}\pcmn{这支笔对我很有用,我用了很久}\end{exemple}\relationsémantique{Component 1}{\lien{ⓔtɯ-ɲɤm}{tɯ-ɲɤm}}\relationsémantique{Component 2}{\lien{ⓔphɤn}{phɤn}}\end{sous-entrée}

\begin{sous-entrée}{tɯ-ɲɤm,khe}{ⓔtɯ-ɲɤmⓝtɯ-ɲɤm,khe} 
\classe{np}
\classe{vs} 
\begin{définition}\pfra{maigre}\end{définition}
\begin{définition}\pcmn{瘦}\end{définition}\relationsémantique{Component 1}{\lien{ⓔtɯ-ɲɤm}{tɯ-ɲɤm}}\relationsémantique{Component 2}{\lien{ⓔkhe}{khe}}\relationsémantique{同义词}{\lien{ⓔnɯɲɤmkhe}{nɯɲɤmkhe}}\end{sous-entrée}

\begin{sous-entrée}{tɯ-ɲɤm,sɯ}{ⓔtɯ-ɲɤmⓝtɯ-ɲɤm,sɯ} 
\classe{np}
\classe{vs} 
\begin{définition}\pfra{gros, gras}\end{définition}
\begin{définition}\pcmn{肥;胖}\end{définition}\relationsémantique{Component 1}{\lien{ⓔtɯ-ɲɤm}{tɯ-ɲɤm}}\relationsémantique{Component 2}{\lien{ⓔsɯ}{sɯ}}\relationsémantique{参考}{\lien{ⓔnɯɲɤmsɯ}{nɯɲɤmsɯ}}\end{sous-entrée}

\étymologie{ɲam}\end{entrée}

\begin{entrée}{tɯɲɤt}{}{ⓔtɯɲɤt} 
\classe{n} 
\begin{définition}\pfra{éboulement}\end{définition}
\begin{définition}\pcmn{山崩;滑坡}\end{définition}
\begin{exemple}\pjya{tɯɲɤt pjɤ-ɣi}\hspace{5pt}\pcmn{出现了山崩}\end{exemple}\end{entrée}

\begin{entrée}{tɯɲcɣa}{}{ⓔtɯɲcɣa} 
\classe{n} 
\begin{définition}\pfra{faucille}\end{définition}
\begin{définition}\pcmn{镰刀}\end{définition}\end{entrée}

\begin{entrée}{tɯɲɟoʁ}{}{ⓔtɯɲɟoʁ} 
\classe{n} 
\begin{définition}\pfra{homme de main}\end{définition}
\begin{définition}\pcmn{助手}\end{définition}\end{entrée}

\begin{entrée}{tɯɲoʁ}{}{ⓔtɯɲoʁ} 
\classe{n} 
\begin{définition}\pfra{grains et balle}\end{définition}
\begin{définition}\pcmn{颗粒和糠秕混合}\end{définition}\end{entrée}

\begin{entrée}{tɯŋgu}{}{ⓔtɯŋgu} 
\classe{n} 
\begin{définition}\pfra{casserole pour faire frire la tsampa}\end{définition}
\begin{définition}\pcmn{炒青稞的锅}\end{définition}
\begin{exemple}\pjya{tɯŋgu nɯ tɯsqar ɯ-sɤ-rŋu ɯ-rkoz ŋu, tɯthɯ nɯ rnaʁ, tɯŋgu nɯ mɤ-rnaʁ, antɤm}\hspace{5pt}\pcmn{炒锅是专门用来炒青稞的}\end{exemple}\end{entrée}

\begin{entrée}{tɯ-ŋga}{}{ⓔtɯ-ŋga} 
\classe{np} 
\begin{définition}\pfra{habit}\end{définition}
\begin{définition}\pcmn{衣服}\end{définition}\relationsémantique{参考}{\lien{ⓔŋga}{ŋga}}\relationsémantique{参考}{\lien{ⓔstɤnga}{stɤnga}}\relationsémantique{参考}{\lien{ⓔkɯrɯŋga}{kɯrɯŋga}}\relationsémantique{参考}{\lien{ⓔkupaŋga}{kupaŋga}}\end{entrée}

\begin{entrée}{tɯŋgar}{}{ⓔtɯŋgar} 
\classe{n} 
\begin{définition}\pfra{tissu de laine}\end{définition}
\begin{définition}\pcmn{羊毛布(尚未缝成衣服)}\end{définition}\end{entrée}

\begin{entrée}{tɯŋgɤmbe}{}{ⓔtɯŋgɤmbe} 
\classe{n} 
\begin{définition}\pfra{vêtements abîmés}\end{définition}
\begin{définition}\pcmn{破旧的衣服}\end{définition}\relationsémantique{参考}{\lien{ⓔtɯ-ŋga}{tɯ-ŋga}}\relationsémantique{参考}{\lien{ⓔtɤ-mbe}{tɤ-mbe}}\end{entrée}

\begin{entrée}{tɯ-ŋgɤndo}{}{ⓔtɯ-ŋgɤndo} 
\classe{np} 
\begin{définition}\pfra{bord des vêtements}\end{définition}
\begin{définition}\pcmn{衣角}\end{définition}\end{entrée}

\begin{entrée}{tɯ-ŋgo}{}{ⓔtɯ-ŋgo} 
\classe{np} 
\begin{définition}\pfra{maladie}\end{définition}
\begin{définition}\pcmn{病}\end{définition}\relationsémantique{参考}{\lien{ⓔngo}{ngo}}\end{entrée}

\begin{entrée}{tɯ-ŋgru}{}{ⓔtɯ-ŋgru} 
\classe{np} 
\begin{définition}\pfra{tendon}\end{définition}
\begin{définition}\pcmn{筋}\end{définition}\end{entrée}

\begin{entrée}{tɯ-ŋgra}{}{ⓔtɯ-ŋgra} 
\classe{np} 
\begin{définition}\pfra{salaire}\end{définition}
\begin{définition}\pcmn{工资}\end{définition}\relationsémantique{参考}{\lien{ⓔnɯŋgra}{nɯŋgra}}\end{entrée}

\begin{entrée}{tɯ-ŋgɯl}{}{ⓔtɯ-ŋgɯl} 
\classe{clf} 
\begin{définition}\pfra{une boucle, un tour (à propos d'intestins enroulés comme des cordes)}\end{définition}
\begin{définition}\pcmn{一圈(肠子)}\end{définition}
\begin{exemple}\pjya{tɯ-pu tɯ-ŋgɯl}\hspace{5pt}\pcmn{一圈肠子}\end{exemple}\relationsémantique{参考}{\lien{ⓔtɯ-tɤjŋgɤɣ}{tɯ-tɤjŋgɤɣ}}\end{entrée}

\begin{entrée}{tɯ-ŋka}{}{ⓔtɯ-ŋka} 
\classe{clf} \sens{1}
\begin{définition}\pfra{une parole, un bruit}\end{définition}
\begin{définition}\pcmn{一声;一句}\end{définition}
\begin{exemple}\pjya{tɯ-ŋka tɯ-ŋka tu-ti-a ŋu nɤ}\hspace{5pt}\pcmn{我一句一句地说}\end{exemple}\sens{2}
\begin{définition}\pfra{une bouchée}\end{définition}
\begin{définition}\pcmn{一口,嚼过的食物}\end{définition}
\begin{exemple}\pjya{tɤ-rɟit nɯ tɯ-ŋka kɤ-mbi kɯ chɯ́-wɣ-ɣɤwxti ɕti}\hspace{5pt}\pcmn{小孩子是用大人嚼过的食物喂大的}\end{exemple}\relationsémantique{参考}{\lien{ⓔnɤŋka}{nɤŋka}}\end{entrée}

\begin{entrée}{tɯ-ɴɢar}{}{ⓔtɯ-ɴɢar} 
\classe{np} 
\begin{définition}\pfra{crachat}\end{définition}
\begin{définition}\pcmn{痰}\end{définition}
\begin{exemple}\pjya{ɯ-thoʁ nɤ-ɴɢar ma-thɯ-βde ma ɲɯ-sɤʑɯloʁ}\hspace{5pt}\pcmn{你不要在地上吐痰,很恶心。}\end{exemple}\end{entrée}

\begin{entrée}{tɯpu}{}{ⓔtɯpu} 
\classe{n} 
\begin{définition}\pfra{moxibustion}\end{définition}
\begin{définition}\pcmn{艾灸}\end{définition}
\begin{exemple}\pjya{a-tɯpu ka-ta}\hspace{5pt}\pcmn{他给我烧艾灸}\end{exemple}\end{entrée}

\begin{entrée}{tɯ-pu}{}{ⓔtɯ-pu} 
\classe{np} 
\begin{définition}\pfra{intestin}\end{définition}
\begin{définition}\pcmn{肠}\end{définition}\end{entrée}

\begin{entrée}{tɯ-pɤchaʁ}{}{ⓔtɯ-pɤchaʁ} 
\classe{np} 
\begin{définition}\pfra{nombril}\end{définition}
\begin{définition}\pcmn{肚脐}\end{définition}\end{entrée}

\begin{entrée}{tɯ-pɤɕnɤz}{}{ⓔtɯ-pɤɕnɤz} 
\classe{np} 
\begin{définition}\pfra{anus}\end{définition}
\begin{définition}\pcmn{肛门}\end{définition}\relationsémantique{参考}{\lien{ⓔtɯ-pu}{tɯ-pu}}\relationsémantique{参考}{\lien{ⓔtɤ-ɕnɤz}{tɤ-ɕnɤz}}\end{entrée}

\begin{entrée}{tɯ-pɤɣrum}{}{ⓔtɯ-pɤɣrum} 
\classe{np} 
\begin{définition}\pfra{gros intestin}\end{définition}
\begin{définition}\pcmn{大肠}\end{définition}\relationsémantique{参考}{\lien{ⓔtɯ-pu}{tɯ-pu}}\end{entrée}

\begin{entrée}{tɯ-pɤɲɟi}{}{ⓔtɯ-pɤɲɟi} 
\classe{np} 
\begin{définition}\pfra{bas-ventre}\end{définition}
\begin{définition}\pcmn{小肚子}\end{définition}\end{entrée}

\begin{entrée}{tɯ-pɤŋi}{}{ⓔtɯ-pɤŋi} 
\classe{np} 
\begin{définition}\pfra{intestin grêle}\end{définition}
\begin{définition}\pcmn{小肠}\end{définition}\relationsémantique{参考}{\lien{ⓔtɯ-pu}{tɯ-pu}}\end{entrée}

\begin{entrée}{tɯpɤr}{}{ⓔtɯpɤr} 
\classe{n} 
\begin{définition}\pfra{dessin}\end{définition}
\begin{définition}\pcmn{画,照片}\end{définition}
\begin{exemple}\pjya{nɤ-tɯpɤr pjɯ-lat-a tɕe, tɯrme ɯ-ɕki ɲɯ-kham-a}\hspace{5pt}\pcmn{我给你拍照片,给别人看}\end{exemple}\étymologie{par}\end{entrée}

\begin{entrée}{tɯ-pɤrme}{}{ⓔtɯ-pɤrme} 
\classe{clf} 
\begin{définition}\pfra{année}\end{définition}
\begin{définition}\pcmn{一岁(年龄)}\end{définition}
\begin{exemple}\pjya{nɤʑo thɤstɯ-pɤrme thɯ-tɯ-azɣɯt?}\hspace{5pt}\pcmn{你多大了?}\end{exemple}
\begin{exemple}\pjya{aʑo fsusqafsum-pɤrme thɯ-azɣɯt-a}\hspace{5pt}\pcmn{我三十三岁}\end{exemple}\end{entrée}

\begin{entrée}{tɯpɕi}{}{ⓔtɯpɕi} 
\classe{n} 
\begin{définition}\pfra{lin}\end{définition}
\begin{définition}\pcmn{亚麻}\end{définition}\end{entrée}

\begin{entrée}{tɯ-pɕoʁ}{}{ⓔtɯ-pɕoʁ} 
\classe{n} 
\begin{définition}\pfra{côté, direction}\end{définition}
\begin{définition}\pcmn{方向}\end{définition}
\begin{exemple}\pjya{tɯ-pɕoʁ ci pjɯ-phɤn, tɯ-pɕoʁ ci pjɯ-ʁdɯɣ ɲɯ-ɕti}\hspace{5pt}\pcmn{(这种药)一方面有好处,一方面又有副作用}\end{exemple}
\begin{exemple}\pjya{ɯ-pɕoʁ a-mɤ-pɯ-naχtɕɯɣ qhe li ɯ-ti mɤ-naχtɕɯɣ}\hspace{5pt}\pcmn{只要方向不一样说法就不一样(解释动词的趋向前缀的时候)}\end{exemple}\étymologie{pʰʲogs}\end{entrée}

\begin{entrée}{tɯ-pɕɯrtɕhaʁ}{}{ⓔtɯ-pɕɯrtɕhaʁ} 
\classe{clf} 
\begin{définition}\pfra{une fois}\end{définition}
\begin{définition}\pcmn{一倍}\end{définition}\relationsémantique{同义词}{\lien{ⓔtɯ-tɤlɤβ}{tɯ-tɤlɤβ}}\relationsémantique{参考}{\lien{ⓔtɯ-tɯpɕɯrtɕhaʁ}{tɯ-tɯpɕɯrtɕhaʁ}}\end{entrée}

\begin{entrée}{tɯpɣaʁ}{}{ⓔtɯpɣaʁ} 
\classe{n} 
\begin{définition}\pfra{défrichage}\end{définition}
\begin{définition}\pcmn{开荒}\end{définition}
\begin{exemple}\pjya{tɯpɣaʁ lo-tɕɤt-ndʑi}\hspace{5pt}\pcmn{他们俩开荒了}\end{exemple}\relationsémantique{参考}{\lien{ⓔpɣaʁ}{pɣaʁ}}\end{entrée}

\begin{entrée}{tɯ-phaʁ}{}{ⓔtɯ-phaʁ} 
\classe{np} 
\begin{définition}\pfra{un côté}\end{définition}
\begin{définition}\pcmn{一边;半边}\end{définition}
\begin{exemple}\pjya{ɯ-phaʁ ntsi (kɯ) ko-sɯ-rtoʁ}\hspace{5pt}\pcmn{他斜着眼睛看了}\end{exemple}\end{entrée}

\begin{entrée}{tɯ-phaʁja}{}{ⓔtɯ-phaʁja} 
\classe{np} 
\begin{définition}\pfra{époux}\end{définition}
\begin{définition}\pcmn{丈夫;伴侣}\end{définition}\end{entrée}

\begin{entrée}{tɯ-phoŋbu}{}{ⓔtɯ-phoŋbu} 
\classe{np} 
\begin{définition}\pfra{corps}\end{définition}
\begin{définition}\pcmn{身体}\end{définition}
\begin{exemple}\pjya{nɤ-phoŋbu ɣɯ ɯ-βri ma-pɯ-tɯ-sɯxɕe ma!}\hspace{5pt}\pcmn{你不要伤着身体!}\end{exemple}\relationsémantique{参考}{\lien{ⓔndoⓢ7ⓝtɯ-phoŋbu,ndo}{tɯ-phoŋbu,ndo}}\étymologie{pʰuŋ.po}\end{entrée}

\begin{entrée}{tɯ-phoʁ}{}{ⓔtɯ-phoʁ} 
\classe{np} 
\begin{définition}\pfra{salaire}\end{définition}
\begin{définition}\pcmn{工资}\end{définition}\relationsémantique{同义词}{\lien{ⓔtɯ-ŋgra}{tɯ-ŋgra}}\end{entrée}

\begin{entrée}{tɯ-phɯ}{}{ⓔtɯ-phɯ} 
\classe{clf} 
\begin{définition}\pfra{tronc}\end{définition}
\begin{définition}\pcmn{一棵}\end{définition}
\begin{exemple}\pjya{si tɯ-phɯ}\hspace{5pt}\pcmn{一棵树}\end{exemple}\end{entrée}

\begin{entrée}{tɯ-phɯɣ}{}{ⓔtɯ-phɯɣ} 
\classe{np} 
\begin{définition}\pfra{fortune}\end{définition}
\begin{définition}\pcmn{财产,财力}\end{définition}
\begin{exemple}\pjya{jiɕqha tɯrme nɯ ɯ-phɯɣ tu}\hspace{5pt}\pcmn{那个人很有财力}\end{exemple}\end{entrée}

\begin{entrée}{tɯ-phɯm}{}{ⓔtɯ-phɯm} 
\classe{np} 
\begin{définition}\pfra{pan du vêtement}\end{définition}
\begin{définition}\pcmn{衣兜}\end{définition}
\begin{exemple}\pjya{ɯ-phɯm nɯ tɕu tasa-rŋu ci tɯ-lʁɤtɕɯ ɲɤ-rku}\hspace{5pt}\pcmn{他把一袋炒麻籽装在衣兜里了}\end{exemple}\end{entrée}

\begin{entrée}{tɯ-phɯxpa}{}{ⓔtɯ-phɯxpa} 
\classe{np} 
\begin{définition}\pfra{cuisse}\end{définition}
\begin{définition}\pcmn{大腿}\end{définition}\end{entrée}

\begin{entrée}{tɯ-pju}{}{ⓔtɯ-pju} 
\classe{np} 
\begin{définition}\pfra{moelle}\end{définition}
\begin{définition}\pcmn{骨髓}\end{définition}\end{entrée}

\begin{entrée}{tɯ-pjaχpa}{}{ⓔtɯ-pjaχpa} 
\classe{np} 
\begin{définition}\pfra{aisselle}\end{définition}
\begin{définition}\pcmn{膈肢窝}\end{définition}
\begin{exemple}\pjya{jɯɣi ɯ-pjaχpa to-rku}\hspace{5pt}\pcmn{他把书夹在腋下}\end{exemple}\relationsémantique{参考}{\lien{ⓔnɯpjaχpa}{nɯpjaχpa}}\end{entrée}

\begin{entrée}{tɯ-po}{}{ⓔtɯ-po} 
\classe{clf} 
\begin{définition}\pfra{unité de mesure}\end{définition}
\begin{définition}\pcmn{一斗}\end{définition}\étymologie{ⁿbo}\end{entrée}

\begin{entrée}{tɯpoli}{}{ⓔtɯpoli} 
\classe{n} 
\begin{définition}\pfra{une gerbe d'herbe}\end{définition}
\begin{définition}\pcmn{一捆青草}\end{définition}\end{entrée}

\begin{entrée}{tɯpri}{}{ⓔtɯpri} 
\classe{n} 
\begin{définition}\pfra{message}\end{définition}
\begin{définition}\pcmn{口信}\end{définition}
\begin{exemple}\pjya{a-tɯpri jo-lɤt}\hspace{5pt}\pcmn{他给我带了口信}\end{exemple}\relationsémantique{参考}{\lien{ⓔznɯxpri}{znɯxpri}}\end{entrée}

\begin{entrée}{tɯ-pɯsqhɯt}{}{ⓔtɯ-pɯsqhɯt} 
\classe{np} 
\begin{définition}\pfra{fin de l'œsophage}\end{définition}
\begin{définition}\pcmn{食管的末端}\end{définition}\end{entrée}

\begin{entrée}{tɯ-qa}{}{ⓔtɯ-qa} 
\classe{np} \sens{1}
\begin{définition}\pfra{racine}\end{définition}
\begin{définition}\pcmn{根}\end{définition}\sens{2}
\begin{définition}\pfra{patte}\end{définition}
\begin{définition}\pcmn{(动物)的脚}\end{définition}\sens{3}
\begin{définition}\pfra{fond}\end{définition}
\begin{définition}\pcmn{底部}\end{définition}
\begin{exemple}\pjya{tɤ-fkɯm ɣɯ ɯ-qa}\hspace{5pt}\pcmn{袋子的底部}\end{exemple}
\begin{exemple}\pjya{mtshu ɯ-qa zɯ}\hspace{5pt}\pcmn{在湖底}\end{exemple}
\begin{exemple}\pjya{ɯ-qa ʑo tu-nɯɬoʁ naʁzi-a}\hspace{5pt}\pcmn{我想寻根问底}\end{exemple}
\begin{exemple}\pjya{ɯ-kɤ-thu nɯ ɯ-qa ʑo tu-nɯɬoʁ naʁzi}\hspace{5pt}\pcmn{他想寻根问底}\end{exemple}
\begin{exemple}\pjya{nɯ-tɯ-khe kɯ ɯ-qa ʑo ɲɤ-me}\hspace{5pt}\pcmn{他们笨到极点}\end{exemple}
\begin{sous-entrée}{ɯ-qaɕɯqa}{ⓔtɯ-qaⓢ3ⓝɯ-qaɕɯqa}
\begin{définition}\pfra{le plus profond}\end{définition}
\begin{définition}\pcmn{最底层}\end{définition}
\begin{exemple}\pjya{rɟɤmtshu ɯ-qaɕɯqa}\hspace{5pt}\pcmn{海洋的最底部}\end{exemple}\relationsémantique{参考}{\lien{ⓔtɤ-qaʁrɯ}{tɤ-qaʁrɯ}}\end{sous-entrée}

\end{entrée}

\begin{entrée}{tɯqartsɯ}{}{ⓔtɯqartsɯ} 
\classe{n} 
\begin{définition}\pfra{coup de pied}\end{définition}
\begin{définition}\pcmn{踢一脚}\end{définition}
\begin{exemple}\pjya{tɯqartsɯ ta-lɤt}\hspace{5pt}\pcmn{它踢了一脚}\end{exemple}\relationsémantique{参考}{\lien{ⓔsɯqartsɯ}{sɯqartsɯ}}\end{entrée}

\begin{entrée}{tɯ-qazgra}{}{ⓔtɯ-qazgra} 
\classe{np} 
\begin{définition}\pfra{bruit de pas}\end{définition}
\begin{définition}\pcmn{脚步声}\end{définition}\relationsémantique{参考}{\lien{ⓔtɤ-zgra}{tɤ-zgra}}\end{entrée}

\begin{entrée}{tɯ-qɤsɤlɤt}{}{ⓔtɯ-qɤsɤlɤt} 
\classe{np} 
\begin{définition}\pfra{anus}\end{définition}
\begin{définition}\pcmn{肛门}\end{définition}\end{entrée}

\begin{entrée}{tɯ-qe}{}{ⓔtɯ-qe} 
\classe{np} 
\begin{définition}\pfra{excrément, pet}\end{définition}
\begin{définition}\pcmn{屎;屁}\end{définition}
\begin{exemple}\pjya{a-qe nɯ-lat-a}\hspace{5pt}\pcmn{我拉了屎}\end{exemple}
\begin{exemple}\pjya{a-qe thɯ-lat-a}\hspace{5pt}\pcmn{我放了屁}\end{exemple}\relationsémantique{参考}{\lien{ⓔkhrambaqe}{khrambaqe}}\relationsémantique{参考}{\lien{ⓔzdɯmqe}{zdɯmqe}}\end{entrée}

\begin{entrée}{tɯ-qejdi}{}{ⓔtɯ-qejdi} 
\classe{np} 
\begin{définition}\pfra{odeur de bouse}\end{définition}
\begin{définition}\pcmn{屎的臭味}\end{définition}
\begin{exemple}\pjya{nɤ-qejdi ɯ-tɯ-sɤjloʁ nɯ}\hspace{5pt}\pcmn{你的屎的臭味很难闻}\end{exemple}\relationsémantique{参考}{\lien{ⓔtɯ-qe}{tɯ-qe}}\relationsémantique{参考}{\lien{ⓔtɤ-di}{tɤ-di}}\end{entrée}

\begin{entrée}{tɯqejmɤɣ}{}{ⓔtɯqejmɤɣ} 
\classe{n} 
\begin{définition}\pfra{une espèce de champignon}\end{définition}
\begin{définition}\pcmn{【牛屎菌】}\end{définition}
\begin{exemple}\pjya{tɯ-qe jmɤɣ nɯ tɯ-qe ɯ-taʁ tu-ɬoʁ ŋu, ɯ-mdoʁ kɯ-wɣrum tu, kɯ-qandʐi tu, kɯ-ɣɯrni tu, kɯ-sɤndɤɣ me ri ɯ-kɯ-ndza me}\hspace{5pt}\pcmn{牛屎菌长在粪上,有的是白色的,有的是乌色的,有的是红的,没有毒性,但也没人吃。}\end{exemple}\end{entrée}

\begin{entrée}{tɯ-qe,rɤrɕɯβ}{}{ⓔtɯ-qe,rɤrɕɯβ} 
\classe{np}
\classe{vt} \paradigme{dir}{thɯ-}
\begin{définition}\pfra{péter sans faire de bruit}\end{définition}
\begin{définition}\pcmn{悄悄地放屁}\end{définition}
\begin{exemple}\pjya{ɯ-qe tha-rɤrɕɯβ}\hspace{5pt}\pcmn{他放了屁}\end{exemple}\relationsémantique{Component 1}{\lien{ⓔtɯ-qe}{tɯ-qe}}\relationsémantique{Component 2}{\lien{}{rɤrɕɯβ}}\end{entrée}

\begin{entrée}{tɯ-qhoχpa}{}{ⓔtɯ-qhoχpa} 
\classe{np} \sens{1}
\begin{définition}\pfra{organes}\end{définition}
\begin{définition}\pcmn{内脏}\end{définition}\sens{2}
\begin{définition}\pfra{état d'esprit}\end{définition}
\begin{définition}\pcmn{性情}\end{définition}\étymologie{kʰog.pa}\end{entrée}

\begin{entrée}{tɯ-qhrɯmbɤβ}{}{ⓔtɯ-qhrɯmbɤβ} 
\classe{np} 
\begin{définition}\pfra{rot}\end{définition}
\begin{définition}\pcmn{饱嗝}\end{définition}
\begin{exemple}\pjya{a-qhrɯmbɤβ ɲɯ-sɯɣe}\hspace{5pt}\pcmn{我打了个嗝儿}\end{exemple}
\begin{exemple}\pjya{tɕɣom tɤ-ndza-t-a, a-qhrɯmbɤβ la-sɯɣe}\hspace{5pt}\pcmn{我吃了花椒,就打嗝了}\end{exemple}
\begin{exemple}\pjya{@pijiu kɤ-tshi-t-a, a-qhrɯmbɤβ ja-sɯɣe}\hspace{5pt}\pcmn{我一喝啤酒就要打嗝}\end{exemple}\relationsémantique{参考}{\lien{ⓔnɤqhrɯmbɤβ}{nɤqhrɯmbɤβ}}\end{entrée}

\begin{entrée}{tɯqioʁ}{}{ⓔtɯqioʁ} 
\classe{n} 
\begin{définition}\pfra{vomi}\end{définition}
\begin{définition}\pcmn{呕吐物}\end{définition}
\begin{exemple}\pjya{tɯqioʁ tɤ-khat-a}\hspace{5pt}\pcmn{我吐了很久(很多)}\end{exemple}\relationsémantique{参考}{\lien{ⓔqioʁ}{qioʁ}}\end{entrée}

\begin{entrée}{tɯ-qom}{}{ⓔtɯ-qom} 
\classe{np} 
\begin{définition}\pfra{larme}\end{définition}
\begin{définition}\pcmn{眼泪}\end{définition}\end{entrée}

\begin{entrée}{tɯr}{}{ⓔtɯr} 
\classe{vt} \paradigme{dir}{\_}
\begin{définition}\pfra{dépasser}\end{définition}
\begin{définition}\pcmn{冲过去}\end{définition}
\begin{exemple}\pjya{fsapaʁ kɯ kɤ́-wɣ-tɯr-a}\hspace{5pt}\pcmn{牲畜在我面前冲过去了}\end{exemple}\end{entrée}

\begin{entrée}{tɯ-rɤʁaŋ}{}{ⓔtɯ-rɤʁaŋ} 
\classe{np} 
\begin{définition}\pfra{capacité de décision}\end{définition}
\begin{définition}\pcmn{自我主张的权利}\end{définition}
\begin{exemple}\pjya{a-rɤʁaŋ me}\hspace{5pt}\pcmn{我身不由己}\end{exemple}\relationsémantique{参考}{\lien{ⓔznɯrɤʁaŋ}{znɯrɤʁaŋ}}\étymologie{raŋ.dbaŋ}\end{entrée}

\begin{entrée}{tɯrɤt}{}{ⓔtɯrɤt} 
\classe{n} 
\begin{définition}\pfra{style d'écriture, façon d'écrire}\end{définition}
\begin{définition}\pcmn{字体}\end{définition}\relationsémantique{参考}{\lien{ⓔrɤt}{rɤt}}\end{entrée}

\begin{entrée}{tɯ-rcu}{}{ⓔtɯ-rcu} 
\classe{np} 
\begin{définition}\pfra{veste}\end{définition}
\begin{définition}\pcmn{皮袄}\end{définition}
\begin{exemple}\pjya{a-rcu}\hspace{5pt}\pcmn{我的皮袄}\end{exemple}\end{entrée}

\begin{entrée}{tɯ-rcɤmbe}{}{ⓔtɯ-rcɤmbe} 
\classe{np} 
\begin{définition}\pfra{vieille veste}\end{définition}
\begin{définition}\pcmn{旧皮袄}\end{définition}\relationsémantique{参考}{\lien{ⓔtɯ-rcu}{tɯ-rcu}}\relationsémantique{参考}{\lien{ⓔtɤ-mbe}{tɤ-mbe}}\end{entrée}

\begin{entrée}{tɯ-rdoʁ}{}{ⓔtɯ-rdoʁ} 
\classe{clf} \sens{1}
\begin{définition}\pfra{un morceau}\end{définition}
\begin{définition}\pcmn{一块;一个}\end{définition}
\begin{exemple}\pjya{khɯtsa tɯ-rdoʁ}\hspace{5pt}\pcmn{一个碗}\end{exemple}
\begin{exemple}\pjya{tɤ-ŋgɯm tɯ-rdoʁ}\hspace{5pt}\pcmn{一个鸡蛋}\end{exemple}
\begin{exemple}\pjya{mbrɤz tɯ-rdoʁ}\hspace{5pt}\pcmn{一粒米}\end{exemple}
\begin{exemple}\pjya{zɣɤmbu tɯ-rdoʁ}\hspace{5pt}\pcmn{一把扫把}\end{exemple}
\begin{exemple}\pjya{mbrɯtɕɯ tɯ-rdoʁ}\hspace{5pt}\pcmn{一把刀}\end{exemple}
\begin{exemple}\pjya{ndzom tɯ-rdoʁ}\hspace{5pt}\pcmn{一座桥}\end{exemple}
\begin{exemple}\pjya{rɤɣo tɯ-rdoʁ}\hspace{5pt}\pcmn{一首歌}\end{exemple}
\begin{exemple}\pjya{tɯ-ŋga tɯ-rdoʁ}\hspace{5pt}\pcmn{一件衣服}\end{exemple}
\begin{exemple}\pjya{ʁmaʁdɤr tɯ-rdoʁ}\hspace{5pt}\pcmn{一面旗}\end{exemple}
\begin{exemple}\pjya{mɯntoʁ tɯ-rdoʁ}\hspace{5pt}\pcmn{一朵花}\end{exemple}
\begin{exemple}\pjya{ɯ-rdɯ-rdoʁ ʑo ma me}\hspace{5pt}\pcmn{只剩下几个}\end{exemple}
\begin{exemple}\pjya{ɯ-zda ra ʁnɯz ɣɯ nɯ-kɤ-ndza nɯ ɯʑo tɯ-rdoʁ kɯ tu-ndze mɤɕtʂa mɯ́j-rtaʁ}\hspace{5pt}\pcmn{他的吃量是其他(小孩子)的两倍}\end{exemple}\sens{2}
\begin{définition}\pfra{grains}\end{définition}
\begin{définition}\pcmn{粮食}\end{définition}\relationsémantique{同义词}{\lien{ⓔtɯjpu}{tɯjpu}}\étymologie{rdog.po}\end{entrée}

\begin{entrée}{tɯrgi}{}{ⓔtɯrgi} 
\classe{n} 
\begin{définition}\pfra{sapin}\end{définition}
\begin{définition}\pcmn{杉树}\end{définition}
\begin{exemple}\pjya{tɯrgi nɯ zgo kɯ-mbro ɴqiaβ tsa tu-ɬoʁ ŋu, si wuma ʑo kɯ-mbro, ɯ-ru nɯ kɯ-jpɯ-jpum kɯ-mbɯ-mbro ŋu, ɯ-rtaʁ nɯ khro mɤ-jpum tɕe, ɯ-βri nɯ tɕu mɤ-kɯ-ɤmtɕhoʁ tu-oʑɯrja ŋu. ɯ-rtaʁ maŋpa nɯ ra zri, taʁ tɤ-ari ɯ-jija ɯ-rtaʁ nɯ tu-xtɯt ŋu, ɯ-rtaʁ nɯ ɯ-taʁ nɯ tɕu, li ɯ-rtaʁ ɲɯ-nɯ-ɴɢɤt ŋu, ɯ-jwaʁ nɯ taqaβ fse ri, xtɯt aɕpɯɕpa, tɕe ɯ-rtaʁ ɯ-taʁ ɯ-jwaʁ ɯ-tɯ-ndzoʁ nɯ tɤ-muj kɯ-fse ɯ-tshɯɣa ŋu. tɯrgi ɯ-jwaʁ nɯ arŋi tɕe pɣi tsa, ɯ-mat nɯ qaɟy ɯ-rqhu tsa fse, alɯlju tɕe rɲɟi tsa, tɕe nɯ ɯ-ŋgɯ nɯ tɕu, ɯ-rɣi arku, thɯ-tɯt tɕe, qaɟy rqhu kɯ-fse nɯ raŋri ʑo ɲɤ-ɴɢaʁ, tɕe ɯ-rɣi pjɯ-nɯɬoʁ ŋu ma ɯ-zrɤm taʁ tu-mphɯl mɤ-cha, tɕe ɯ-mat nɯ tɯrgi laŋlaŋ rmi. tɯrgi ɯ-ru nɯ wuma ʑo tɤrɤm ɯ-spa kɯ-ʑru, tɯrgi tɯ-phɯ nɯ kɯβdɤsqi-ɟom, kɯmŋɤsqi-ɟom kɯ-mbro ɲɯ-ɕti. tɕe nɯ nɯ́-wɣ-phɯt tɕe ɕoŋtɕa ɕnɤcɤ-rzɯɣ kɤ-βzu rtaʁ. tɕe ɯ-rtaʁ nɯ ra kɤ-nɯβlɯ ma mɯ́j-sna, ɯ-ru nɯ ŋgɤjpɤn chɯ́-wɣ-lɤt tɕe, laχtɕha tɕhi kɯ-ra kɤ-βzu sna.}\hspace{5pt}\pcmn{杉树生长在比较高的山阴(背阴的山坡)上,是一种高大的树。树干长得又粗又高,枝桠都不粗,在树上长得不整齐。树枝下面的长,越是长在上面就越短。在枝桠上有分杈,叶子长得像针,短而扁,叶子在枝桠上的长法像羽毛的形状。杉树的叶子青而灰,果实像鱼鳞,是圆柱形的,种子装在里面。成熟后,像鳞片的那些东西个个都展开了,种子就会出来,因为杉树不能用根繁殖。这种果实叫\lien{ⓔtɯrgilaŋlaŋ}{tɯrgilaŋlaŋ}。杉树的树干是做木板最好的原料,一棵杉树有14-15米高,砍下来足够锯成七八节木料。枝桠只能烧火用,树干锯成木板,可以作成各种家具。}\end{exemple}\end{entrée}

\begin{entrée}{tɯrgigrɯβgrɯβ}{}{ⓔtɯrgigrɯβgrɯβ} 
\classe{n} 
\begin{définition}\pfra{une espèce de champignon}\end{définition}
\begin{définition}\pcmn{【杉木蘑菇】}\end{définition}
\begin{exemple}\pjya{tɯrgi grɯβgrɯβ nɯ tɯrgi kɯ-wxti kɯ-ʁjɤr ɯ-ŋgɯ tu-ɬoʁ ŋu, stonka mɤɕtʂa mɤ-ɬoʁ, grɯβgrɯβ cho ɯ-mdoʁ naχtɕɯɣ ɯ-ru jpum cho wxti ɯ-di mɤ-naχtɕɯɣ}\hspace{5pt}\pcmn{杉木蘑菇是长在茂密高大的杉木林里,到秋天才能生长,颜色和松茸一样,但主干比较粗大,味道不一样。}\end{exemple}\end{entrée}

\begin{entrée}{tɯrgilaŋlaŋ}{}{ⓔtɯrgilaŋlaŋ} 
\classe{n} 
\begin{définition}\pfra{pomme du sapin}\end{définition}
\begin{définition}\pcmn{杉树果}\end{définition}
\begin{exemple}\pjya{nɯ-tɯrgilaŋlaŋ}\end{exemple}\end{entrée}

\begin{entrée}{tɯrgipaʁtsa}{}{ⓔtɯrgipaʁtsa} 
\classe{n} 
\begin{définition}\pfra{écureuil}\end{définition}
\begin{définition}\pcmn{松鼠}\end{définition}\end{entrée}

\begin{entrée}{tɯrgismɤɣ}{}{ⓔtɯrgismɤɣ} 
\classe{n} 
\begin{définition}\pfra{Usnea sp.}\end{définition}
\begin{définition}\pcmn{松萝}\end{définition}\end{entrée}

\begin{entrée}{tɯ-rɣi}{}{ⓔtɯ-rɣi} 
\classe{n} 
\begin{définition}\pfra{graine}\end{définition}
\begin{définition}\pcmn{种子}\end{définition}\end{entrée}

\begin{entrée}{tɯ-rɣɯt}{}{ⓔtɯ-rɣɯt} 
\classe{clf} 
\begin{définition}\pfra{brin de fil}\end{définition}
\begin{définition}\pcmn{一股线}\end{définition}\end{entrée}

\begin{entrée}{tɯ-ri}{}{ⓔtɯ-ri} 
\classe{clf} 
\begin{définition}\pfra{cent}\end{définition}
\begin{définition}\pcmn{一百}\end{définition}\relationsémantique{同义词}{\lien{ⓔɣurʑa}{ɣurʑa}}\end{entrée}

\begin{entrée}{tɯ-rju}{}{ⓔtɯ-rju} 
\classe{np} 
\begin{définition}\pfra{parole}\end{définition}
\begin{définition}\pcmn{话}\end{définition}
\begin{exemple}\pjya{tɯ-rju to-nɤtsɯmɣɯt}\hspace{5pt}\pcmn{他传播了谣言}\end{exemple}
\begin{exemple}\pjya{tɯ-rju kɯ-ɕɤɣ tɯ-ŋka kɤ-spa-t-a}\hspace{5pt}\pcmn{我学了一个新词}\end{exemple}\end{entrée}

\begin{entrée}{tɯrɟaʁ}{}{ⓔtɯrɟaʁ} 
\classe{n} 
\begin{définition}\pfra{danse}\end{définition}
\begin{définition}\pcmn{舞蹈}\end{définition}
\begin{exemple}\pjya{a-tɯrɟaʁ ci kɤ-fɕɤt}\hspace{5pt}\pcmn{你给我跳一支舞}\end{exemple}
\begin{exemple}\pjya{tɯrɟaʁ kɯ-rɲɟɯ-rɲɟi ʑo ko-rɤɕi-nɯ (ko-mtshi-nɯ)}\hspace{5pt}\pcmn{他们跳舞的队伍拉得很长}\end{exemple}\relationsémantique{参考}{\lien{ⓔrɟaʁ}{rɟaʁ}}\relationsémantique{参考}{\lien{ⓔfɕɤtⓗ1}{fɕɤt₁}}\end{entrée}

\begin{entrée}{tɯ-rɟɯ}{}{ⓔtɯ-rɟɯ} 
\classe{np} 
\begin{définition}\pfra{fortune}\end{définition}
\begin{définition}\pcmn{财富}\end{définition}
\begin{exemple}\pjya{tɯ-tsɣe kɤ-βzu ɣɯ ɯ-rɟɯ ɲɯ-rtaʁ}\hspace{5pt}\pcmn{他有做生意的本钱}\end{exemple}\étymologie{rgʲu}\end{entrée}

\begin{entrée}{tɯ-rkɤn}{}{ⓔtɯ-rkɤn} 
\classe{np} 
\begin{définition}\pfra{palais}\end{définition}
\begin{définition}\pcmn{上腭}\end{définition}\étymologie{rkan}\end{entrée}

\begin{entrée}{tɯrkɤz}{}{ⓔtɯrkɤz} 
\classe{n} 
\begin{définition}\pfra{sculpture}\end{définition}
\begin{définition}\pcmn{雕塑}\end{définition}\étymologie{rkos}\end{entrée}

\begin{entrée}{tɯ-rkoŋɕɤl}{}{ⓔtɯ-rkoŋɕɤl} 
\classe{clf} 
\begin{définition}\pfra{une are}\end{définition}
\begin{définition}\pcmn{一亩}\end{définition}\étymologie{rkaŋ}\end{entrée}

\begin{entrée}{tɯ-rla}{}{ⓔtɯ-rla} 
\classe{np} 
\begin{définition}\pfra{âme, principe vital}\end{définition}
\begin{définition}\pcmn{灵魂,生命的根源}\end{définition}
\begin{exemple}\pjya{tɯ-mu kɯ ɯ-rla ɲɤ-me}\hspace{5pt}\pcmn{他吓坏了}\end{exemple}\end{entrée}

\begin{entrée}{tɯrma}{}{ⓔtɯrma} 
\classe{n} 
\begin{définition}\pfra{vie quotidienne; tâches quotidiennes}\end{définition}
\begin{définition}\pcmn{过日子;日常生活}\end{définition}
\begin{exemple}\pjya{tɯrma ko-ndo (tɯrma ɯʑoz ko-ndo)}\hspace{5pt}\pcmn{他安家过日子(他离家在另一个地方过日子)}\end{exemple}
\begin{exemple}\pjya{ndʑi-tɯrma nɯ wuma pjɤ-khɯ}\hspace{5pt}\pcmn{他们俩日子过得很好}\end{exemple}\end{entrée}

\begin{entrée}{tɯrmbɣi}{}{ⓔtɯrmbɣi} 
\classe{n}  
\grammaire{n.lieu} 
\begin{définition}\pfra{l'un des hameaux de Gyutshapa}\end{définition}
\begin{définition}\pcmn{二茶村的大队之一}\end{définition}\end{entrée}

\begin{entrée}{tɯ-rmbi}{}{ⓔtɯ-rmbi} 
\classe{np} 
\begin{définition}\pfra{urine}\end{définition}
\begin{définition}\pcmn{尿}\end{définition}\end{entrée}

\begin{entrée}{tɯrme}{}{ⓔtɯrme} 
\classe{n} \sens{1}
\begin{définition}\pfra{homme}\end{définition}
\begin{définition}\pcmn{人}\end{définition}\sens{2}
\begin{définition}\pfra{quelqu'un d'autre}\end{définition}
\begin{définition}\pcmn{别人}\end{définition}
\begin{exemple}\pjya{kɯki tɯrme ɣɯ ɯ-@shouji ŋu}\hspace{5pt}\pcmn{这是别人的手机}\end{exemple}\end{entrée}

\begin{entrée}{tɯrmɯ}{}{ⓔtɯrmɯ} 
\classe{n} 
\begin{définition}\pfra{après midi}\end{définition}
\begin{définition}\pcmn{下午}\end{définition}
\begin{exemple}\pjya{nɤ-tɯrmɯ ko-ɣi}\hspace{5pt}\pcmn{已经很晚,你来不及了}\end{exemple}
\begin{exemple}\pjya{soz tɕi tu ŋgrɤl tɯrmɯ tɕi tu ŋgrɤl}\hspace{5pt}\pcmn{早上也有,下午也有}\end{exemple}\relationsémantique{参考}{\lien{ⓔnɯrmɯ}{nɯrmɯ}}\end{entrée}

\begin{entrée}{tɯrmɯkha}{}{ⓔtɯrmɯkha} 
\classe{n} 
\begin{définition}\pfra{crépuscule}\end{définition}
\begin{définition}\pcmn{黄昏}\end{définition}\end{entrée}

\begin{entrée}{tɯ-rna}{}{ⓔtɯ-rna} 
\classe{np} 
\begin{définition}\pfra{oreille}\end{définition}
\begin{définition}\pcmn{耳朵}\end{définition}\paradigme{comit}{kɤ́rnɯrna}\relationsémantique{参考}{\lien{ⓔmɤrnɤsɤŋo}{mɤrnɤsɤŋo}}\étymologie{rna}\end{entrée}

\begin{entrée}{tɯ-rnamɕɤz}{}{ⓔtɯ-rnamɕɤz} 
\classe{np} 
\begin{définition}\pfra{âme}\end{définition}
\begin{définition}\pcmn{灵魂}\end{définition}\étymologie{rnam.ɕes}\end{entrée}

\begin{entrée}{tɯ-rnɤfsɯr}{}{ⓔtɯ-rnɤfsɯr} 
\classe{np} 
\begin{définition}\pfra{partie poilue devant les oreilles}\end{définition}
\begin{définition}\pcmn{耳朵前面长小头发的部位}\end{définition}\end{entrée}

\begin{entrée}{tɯ-rnɤɣɲɟɯ}{}{ⓔtɯ-rnɤɣɲɟɯ} 
\classe{np} 
\begin{définition}\pfra{conduit auditif}\end{définition}
\begin{définition}\pcmn{耳孔}\end{définition}\relationsémantique{参考}{\lien{ⓔtɯ-rna}{tɯ-rna}}\relationsémantique{参考}{\lien{ⓔɯ-ɣɲɟɯ}{ɯ-ɣɲɟɯ}}\end{entrée}

\begin{entrée}{tɯ-rnɤpɤl}{}{ⓔtɯ-rnɤpɤl} 
\classe{np} 
\begin{définition}\pfra{lobe de l'oreille}\end{définition}
\begin{définition}\pcmn{耳垂}\end{définition}\end{entrée}

\begin{entrée}{tɯrnda}{}{ⓔtɯrnda} 
\classe{n} 
\begin{définition}\pfra{partie en bois des maisons tibétains}\end{définition}
\begin{définition}\pcmn{藏式房屋的用木料做成的部分}\end{définition}
\begin{exemple}\pjya{kɯrɯ kha ɯ-taʁ ɕoŋtɕa kɤ-ntɕhoz kɯ-kɯ-ra nɯ jɤɣɤt cho khɤxtu nɯ ra tɯrnda rmi}\hspace{5pt}\pcmn{藏式房屋上所有用木料做成的部分,如走缘、房背等都叫\lien{ⓔtɯrnda}{tɯrnda}}\end{exemple}\end{entrée}

\begin{entrée}{tɯrndaku}{}{ⓔtɯrndaku} 
\classe{n} 
\begin{définition}\pfra{étage au dessus}\end{définition}
\begin{définition}\pcmn{楼上}\end{définition}\end{entrée}

\begin{entrée}{tɯ-rni}{}{ⓔtɯ-rni} 
\classe{np} 
\begin{définition}\pfra{gencive}\end{définition}
\begin{définition}\pcmn{牙龈}\end{définition}\end{entrée}

\begin{entrée}{tɯ-rnom}{}{ⓔtɯ-rnom} 
\classe{np} 
\begin{définition}\pfra{côte}\end{définition}
\begin{définition}\pcmn{肋骨}\end{définition}\end{entrée}

\begin{entrée}{tɯ-rnoʁ}{}{ⓔtɯ-rnoʁ} 
\classe{np} 
\begin{définition}\pfra{cerveau}\end{définition}
\begin{définition}\pcmn{脑子}\end{définition}
\begin{exemple}\pjya{a-rnoʁ ma-tɯ-ci}\hspace{5pt}\pcmn{你不要(在我耳边)这么吵(令我受不了)}\end{exemple}\relationsémantique{参考}{\lien{ⓔnɤrnoʁ}{nɤrnoʁ}}\end{entrée}

\begin{entrée}{tɯrŋu}{}{ⓔtɯrŋu} 
\classe{n} 
\begin{définition}\pfra{orge frit}\end{définition}
\begin{définition}\pcmn{炒青稞}\end{définition}\relationsémantique{参考}{\lien{ⓔrŋu}{rŋu}}\end{entrée}

\begin{entrée}{tɯ-rŋa}{}{ⓔtɯ-rŋa} 
\classe{np} 
\begin{définition}\pfra{visage}\end{définition}
\begin{définition}\pcmn{脸}\end{définition}
\begin{exemple}\pjya{a-rŋa tɤ-chaβ-a ma ɲɯ-qiaβ}\hspace{5pt}\pcmn{我做鬼脸因为很苦}\end{exemple}
\begin{exemple}\pjya{tɯʑo tɯ-rŋa qambrɯ tɤ-ari kɯnɤ mɤ́-wɣ-nɯ-mto, tɯrme ra nɯ-rŋa zrɯɣ tɤ-ari kɯnɤ ɣɯ́-mto}\hspace{5pt}\pcmn{容易看到别人的缺点,不容易看到自己的缺点}\end{exemple}\relationsémantique{参考}{\lien{ⓔaɣɯrŋa}{aɣɯrŋa}}\relationsémantique{参考}{\lien{ⓔɣɤrŋa}{ɣɤrŋa}}\relationsémantique{参考}{\lien{ⓔanɯrŋɤrɯru}{anɯrŋɤrɯru}}\end{entrée}

\begin{entrée}{tɯrŋɤt}{}{ⓔtɯrŋɤt} 
\classe{n} 
\begin{définition}\pfra{piège}\end{définition}
\begin{définition}\pcmn{陷阱}\end{définition}
\begin{exemple}\pjya{tɯrŋɤt pɯ-βzu-t-a}\hspace{5pt}\pcmn{我设了陷阱}\end{exemple}\end{entrée}

\begin{entrée}{tɯ-ro}{}{ⓔtɯ-ro} 
\classe{np} \sens{1}
\begin{définition}\pfra{poitrine}\end{définition}
\begin{définition}\pcmn{胸膛}\end{définition}
\begin{exemple}\pjya{ɯ-ro ɲɯ-ŋgɤr}\hspace{5pt}\pcmn{他心胸狭窄(小气)}\end{exemple}
\begin{exemple}\pjya{ɯ-ro ɲɯ-jom}\hspace{5pt}\pcmn{他心胸广阔(不计较)}\end{exemple}\sens{2}
\begin{définition}\pfra{colère}\end{définition}
\begin{définition}\pcmn{(生的)气}\end{définition}
\begin{exemple}\pjya{ɯ-ro lo-ɣi}\hspace{5pt}\pcmn{他生气了}\end{exemple}
\begin{exemple}\pjya{ɯ-ro jo-zɣɯt}\hspace{5pt}\pcmn{他的坏脾气又发了}\end{exemple}
\begin{exemple}\pjya{ɯ-ro ɲɤ-ʑi}\hspace{5pt}\pcmn{他气消了}\end{exemple}
\begin{exemple}\pjya{nɯ kɯnɤ ʑo ɯ-ro mɯ-pjɤ-k-ɤfɕu-ci}\hspace{5pt}\pcmn{这样他都不解恨}\end{exemple}\étymologie{braŋ}\end{entrée}

\begin{entrée}{tɯrpa}{}{ⓔtɯrpa} 
\classe{n} 
\begin{définition}\pfra{hache}\end{définition}
\begin{définition}\pcmn{斧头}\end{définition}\relationsémantique{参考}{\lien{ⓔtɯ-tɯrpa}{tɯ-tɯrpa}}\end{entrée}

\begin{entrée}{tɯ-rpaʁ}{}{ⓔtɯ-rpaʁ} 
\classe{np} 
\begin{définition}\pfra{épaule}\end{définition}
\begin{définition}\pcmn{肩膀}\end{définition}\étymologie{pʰrag}\end{entrée}

\begin{entrée}{tɯ-rpɣo}{}{ⓔtɯ-rpɣo} 
\classe{np} 
\begin{définition}\pfra{sur les cuisses (lorsqu'on est assis en tailleur)}\end{définition}
\begin{définition}\pcmn{盘着坐时,腿上的部位}\end{définition}
\begin{exemple}\pjya{a-rpɣo}\hspace{5pt}\pcmn{我的腿上}\end{exemple}\relationsémantique{同义词}{\lien{ⓔtɯ-mbur}{tɯ-mbur}}\end{entrée}

\begin{entrée}{tɯ-rqɤpa}{}{ⓔtɯ-rqɤpa} 
\classe{np} 
\begin{définition}\pfra{poitrine}\end{définition}
\begin{définition}\pcmn{胸膛(的上半部分)}\end{définition}
\begin{exemple}\pjya{tɤ-pɤtso ɯ-rqɤpa ɲɯ-ɤci tɕe ɯ-mtʂɤkhoz ɲɯ-ra}\hspace{5pt}\pcmn{小孩子的胸膛很湿,要给他带口水巾}\end{exemple}\relationsémantique{参考}{\lien{ⓔtɯ-rqo}{tɯ-rqo}}\end{entrée}

\begin{entrée}{tɯ-rqo}{}{ⓔtɯ-rqo} 
\classe{np} 
\begin{définition}\pfra{gorge}\end{définition}
\begin{définition}\pcmn{喉咙}\end{définition}
\begin{exemple}\pjya{a-pɯ-sɤtso ra ma a-rqo ɲɯ-qhrɯt ʑo tɕe, mɤ-tɯ-tso thaŋ ɲɯ-sɯsam-a ŋu}\hspace{5pt}\pcmn{希望听得清楚,因为我的喉咙倒了(嗓子哑了),我怕你听不清楚}\end{exemple}\end{entrée}

\begin{entrée}{tɯ-rqoloʁloʁ}{}{ⓔtɯ-rqoloʁloʁ} 
\classe{np} 
\begin{définition}\pfra{pomme d'Adam}\end{définition}
\begin{définition}\pcmn{喉结}\end{définition}\end{entrée}

\begin{entrée}{tɯ-rqopa}{}{ⓔtɯ-rqopa} 
\classe{np} 
\begin{définition}\pfra{bas de la gorge}\end{définition}
\begin{définition}\pcmn{喉咙的下半部分}\end{définition}\relationsémantique{参考}{\lien{ⓔtɯ-rqo}{tɯ-rqo}}\end{entrée}

\begin{entrée}{tɯ-rqorɣe}{}{ⓔtɯ-rqorɣe} 
\classe{np} 
\begin{définition}\pfra{collier}\end{définition}
\begin{définition}\pcmn{项圈}\end{définition}\end{entrée}

\begin{entrée}{tɯ-rqoʁ}{}{ⓔtɯ-rqoʁ} 
\classe{clf} 
\begin{définition}\pfra{quantité qui peut tenir entre l'avant-bras et le bras fléchi}\end{définition}
\begin{définition}\pcmn{一抱}\end{définition}
\begin{exemple}\pjya{si tɯ-rqoʁ tɤ-ɣɯt-a}\hspace{5pt}\pcmn{我抱了一抱的柴}\end{exemple}\relationsémantique{参考}{\lien{ⓔrqoʁ}{rqoʁ}}\end{entrée}

\begin{entrée}{tɯ-rqozbɤβ}{}{ⓔtɯ-rqozbɤβ} 
\classe{np} 
\begin{définition}\pfra{goitre}\end{définition}
\begin{définition}\pcmn{甲状腺肿瘤}\end{définition}\end{entrée}

\begin{entrée}{tɯrsa}{}{ⓔtɯrsa} 
\classe{n} 
\begin{définition}\pfra{tombe}\end{définition}
\begin{définition}\pcmn{坟墓}\end{définition}\étymologie{dur.sa}\end{entrée}

\begin{entrée}{tɯrtɕhi}{}{ⓔtɯrtɕhi} 
\classe{n} 
\begin{définition}\pfra{espèce de plante}\end{définition}
\begin{définition}\pcmn{【酸酸草】}\end{définition}
\begin{exemple}\pjya{tɯrtɕhi nɯ sɯjno kɯ-mbɤr tsa ci ŋu, ɯ-zrɤm wuma ʑo rɲɟi dɤn, ɯ-jwaʁ nɯ kɯ-ɤrtɯm xɯrxɯr tɕe ɯ-rkɯ nɯ ra kɯ-rʁom tsa tu, ɯ-ʁɤri nɯ kɯ-ɤrŋi ŋu, ɯ-qhu nɯ kɯ-ɤrŋi tu kɯ-ɣɯrni tu, ɯ-ru tú-wɣ-ndza tɕe wuma zo tɕur, paʁ kɤ-mbi sna.}\hspace{5pt}\pcmn{酸酸草是一种矮小的草。根又多又长。叶子是圆形的,边缘有些粗糙的小齿,正面是绿色的,背面有的是绿色的,有的是红色的。茎吃起来很酸,可以喂猪。}\end{exemple}\end{entrée}

\begin{entrée}{tɯ-rti}{}{ⓔtɯ-rti} 
\classe{np} 
\begin{définition}\pfra{jupe}\end{définition}
\begin{définition}\pcmn{裙子}\end{définition}\relationsémantique{参考}{\lien{ⓔɯ-rti}{ɯ-rti}}\end{entrée}

\begin{entrée}{tɯ-rtsa}{}{ⓔtɯ-rtsa} 
\classe{np} 
\begin{définition}\pfra{pouls}\end{définition}
\begin{définition}\pcmn{脉搏}\end{définition}\étymologie{rtsa}\end{entrée}

\begin{entrée}{tɯ-rtsaku}{}{ⓔtɯ-rtsaku} 
\classe{np} 
\begin{définition}\pfra{point d'acupuncture}\end{définition}
\begin{définition}\pcmn{穴位}\end{définition}\étymologie{rtsa}\end{entrée}

\begin{entrée}{tɯ-rtsɤɣ}{}{ⓔtɯ-rtsɤɣ} 
\classe{clf} 
\begin{définition}\pfra{une section, un étage, une toise}\end{définition}
\begin{définition}\pcmn{一节;一层楼;一丈}\end{définition}\end{entrée}

\begin{entrée}{tɯ-rtshɤz}{}{ⓔtɯ-rtshɤz} 
\classe{np} 
\begin{définition}\pfra{poumon}\end{définition}
\begin{définition}\pcmn{肺}\end{définition}\relationsémantique{参考}{\lien{ⓔkhɯrtshɤz}{khɯrtshɤz}}\end{entrée}

\begin{entrée}{tɯ-rtsi}{}{ⓔtɯ-rtsi} 
\classe{np} 
\begin{définition}\pfra{nourriture pour bovidés}\end{définition}
\begin{définition}\pcmn{牛的食物}\end{définition}\end{entrée}

\begin{entrée}{tɯrtɯthɯ}{}{ⓔtɯrtɯthɯ} 
\classe{n} 
\begin{définition}\pfra{tissu de lin}\end{définition}
\begin{définition}\pcmn{麻布}\end{définition}\end{entrée}

\begin{entrée}{tɯrɯm}{}{ⓔtɯrɯm} 
\classe{n} 
\begin{définition}\pfra{sorte}\end{définition}
\begin{définition}\pcmn{种类}\end{définition}
\begin{exemple}\pjya{jɯfɕɯr tɤ-rɯndzɤtshi tɕe jɯ-mgo zmɤrɤβ kɤntɕhɯ tɯrɯm pɯ-tu}\hspace{5pt}\pcmn{昨天我们吃饭的时候,有好几种菜}\end{exemple}\end{entrée}

\begin{entrée}{tɯ-rɯrtsɤɣ}{}{ⓔtɯ-rɯrtsɤɣ} 
\classe{np} 
\begin{définition}\pfra{articulation}\end{définition}
\begin{définition}\pcmn{关节}\end{définition}\relationsémantique{参考}{\lien{ⓔtɯ-tshɯɣ}{tɯ-tshɯɣ}}\end{entrée}

\begin{entrée}{tɯ-rɯxpa}{}{ⓔtɯ-rɯxpa} 
\classe{np} 
\begin{définition}\pfra{mémoire}\end{définition}
\begin{définition}\pcmn{记性}\end{définition}
\begin{exemple}\pjya{a-rɯxpa mɯ-ɲo-sna}\hspace{5pt}\pcmn{我记性不好}\end{exemple}\étymologie{rig.pa}\end{entrée}

\begin{entrée}{tɯ-rzɤz}{}{ⓔtɯ-rzɤz} 
\classe{np} \sens{1}
\begin{définition}\pfra{bagages (cadeau que l'on offre avant le départ)}\end{définition}
\begin{définition}\pcmn{行李}\end{définition}\relationsémantique{同义词}{\lien{ⓔtɤ-rkuz}{tɤ-rkuz}}\sens{2}
\begin{définition}\pfra{dot}\end{définition}
\begin{définition}\pcmn{嫁妆}\end{définition}\étymologie{rdzas}\end{entrée}

\begin{entrée}{tɯ-rzɯɣ}{}{ⓔtɯ-rzɯɣ} 
\classe{clf} \sens{1}
\begin{définition}\pfra{section}\end{définition}
\begin{définition}\pcmn{一段}\end{définition}\sens{2}
\begin{définition}\pfra{un moment}\end{définition}
\begin{définition}\pcmn{一会儿}\end{définition}
\begin{exemple}\pjya{nɤʑo pɤjkhu tɯ-rzɯɣ tɤ-nɯna tɕe tɕetha rɤma-tɕi}\hspace{5pt}\pcmn{你暂时休息一下,等一会我们再工作}\end{exemple}
\begin{exemple}\pjya{tʂu tɯ-rzɯɣ}\hspace{5pt}\pcmn{一段路}\end{exemple}
\begin{exemple}\pjya{ɟu tɯ-rzɯɣ}\hspace{5pt}\pcmn{一节竹子}\end{exemple}\relationsémantique{参考}{\lien{}{rɤrzɯrzɯɣ}}\end{entrée}

\begin{entrée}{tɯ-rʑaβspa}{}{ⓔtɯ-rʑaβspa} 
\classe{np} 
\begin{définition}\pfra{fiancée}\end{définition}
\begin{définition}\pcmn{未婚妻}\end{définition}\end{entrée}

\begin{entrée}{tɯ-ʁar}{}{ⓔtɯ-ʁar} 
\classe{clf} 
\begin{définition}\pfra{longueur d'un bras}\end{définition}
\begin{définition}\pcmn{一只手臂的长度}\end{définition}\relationsémantique{参考}{\lien{ⓔtɯ-ʁar}{tɯ-ʁar}}\end{entrée}

\begin{entrée}{tɯ-ʁɤjtshɯz}{}{ⓔtɯ-ʁɤjtshɯz} 
\classe{np} 
\begin{définition}\pfra{avoir une utilité}\end{définition}
\begin{définition}\pcmn{发挥作用}\end{définition}
\begin{exemple}\pjya{tɤtʂu nɯ ɕɤr tɕe tɯ-ʁɤjtshɯz ɲɯ-ɕe}\hspace{5pt}\pcmn{晚上的时候,灯发挥很大的作用}\end{exemple}
\begin{exemple}\pjya{nɤ-rɟit nɯ ju-nɯ-ɕe ɲɯ-ɕti tɕe, nɤ-ʁɤjtshɯz mɯ́j-ɕe}\hspace{5pt}\pcmn{你的孩子离开家了,对你没有用了(不会再照顾你了)}\end{exemple}\end{entrée}

\begin{entrée}{tɯ-ʁɤriɕɣa}{}{ⓔtɯ-ʁɤriɕɣa} 
\classe{np} 
\begin{définition}\pfra{incisives}\end{définition}
\begin{définition}\pcmn{门牙}\end{définition}\end{entrée}

\begin{entrée}{tɯ-ʁɤt}{}{ⓔtɯ-ʁɤt} 
\classe{np} 
\begin{définition}\pfra{capacité de travail}\end{définition}
\begin{définition}\pcmn{(工作或办事的)能力}\end{définition}
\begin{exemple}\pjya{a-ʁɤt maŋe}\hspace{5pt}\pcmn{我没有能力做那么多事情}\end{exemple}\end{entrée}

\begin{entrée}{tɯ-ʁjiz,ɣi}{}{ⓔtɯ-ʁjiz,ɣi} 
\classe{np}
\classe{np}
\classe{vi} 
\begin{définition}\pfra{avoir envie}\end{définition}
\begin{définition}\pcmn{想吃,想要}\end{définition}
\begin{exemple}\pjya{a-ʁjiz mɯ́j-ɣi}\hspace{5pt}\pcmn{我不想}\end{exemple}\relationsémantique{Component 1}{\lien{}{tɯ-ʁjiz}}\relationsémantique{Component 2}{\lien{ⓔɣi}{ɣi}}\end{entrée}

\begin{entrée}{tɯ-ʁla}{}{ⓔtɯ-ʁla} 
\classe{np} 
\begin{définition}\pfra{avant-bras}\end{définition}
\begin{définition}\pcmn{胳膊}\end{définition}\end{entrée}

\begin{entrée}{tɯ-ʁnoŋ}{}{ⓔtɯ-ʁnoŋ} 
\classe{np} 
\begin{définition}\pfra{remords}\end{définition}
\begin{définition}\pcmn{内疚}\end{définition}
\begin{exemple}\pjya{a-ʁnoŋ me}\hspace{5pt}\pcmn{我问心无愧}\end{exemple}
\begin{exemple}\pjya{nɤ-ʁnoŋ maŋe}\hspace{5pt}\pcmn{你问心无愧}\end{exemple}\relationsémantique{参考}{\lien{ⓔnɯʁnoŋ}{nɯʁnoŋ}}\étymologie{gnoŋ}\end{entrée}

\begin{entrée}{tɯ-ʁɲɯrgɯm}{}{ⓔtɯ-ʁɲɯrgɯm} 
\classe{np} 
\begin{définition}\pfra{pourtour des yeux}\end{définition}
\begin{définition}\pcmn{眼圈骨}\end{définition}
\begin{exemple}\pjya{nɤ-ʁɲɯrgɯm chɯ-tɯ-phɤβ}\hspace{5pt}\pcmn{你的脸色变了,你的表情变了。}\end{exemple}\end{entrée}

\begin{entrée}{tɯ-ʁo,mbi}{}{ⓔtɯ-ʁo,mbi} 
\classe{np}
\classe{vi}  
\grammaire{acaus} \paradigme{dir}{nɯ-}
\begin{définition}\pfra{regretter, ne pas être satisfait, être déçu}\end{définition}
\begin{définition}\pcmn{介意;灰心;失望}\end{définition}
\begin{exemple}\pjya{a-ʁo mɤ-mbi}\hspace{5pt}\pcmn{我不介意}\end{exemple}
\begin{exemple}\pjya{a-ʁo ɲɯ-mbi ɕti}\hspace{5pt}\pcmn{我很失望}\end{exemple}\relationsémantique{参考}{\lien{ⓔnɤʁombi}{nɤʁombi}}\relationsémantique{参考}{\lien{ⓔtɯ-ʁo,phi}{tɯ-ʁo,phi}}\relationsémantique{Component 1}{\lien{}{tɯ-ʁo}}\relationsémantique{Component 2}{\lien{ⓔmbi}{mbi}}\end{entrée}

\begin{entrée}{tɯ-ʁo,phi}{}{ⓔtɯ-ʁo,phi} 
\classe{np}
\classe{vt} 
\begin{définition}\pfra{décevoir}\end{définition}
\begin{définition}\pcmn{令……失望}\end{définition}
\begin{exemple}\pjya{ɯʑo kɯ a-ʁo ɲɯ-phi}\hspace{5pt}\pcmn{我对他很失望}\end{exemple}
\begin{exemple}\pjya{a-ʁo ɲɯ-tɯ-phi}\hspace{5pt}\pcmn{我对你很失望}\end{exemple}
\begin{exemple}\pjya{a-ʁo ɲɯ-nɯ-phi-a}\hspace{5pt}\pcmn{我对我自己很失望}\end{exemple}\relationsémantique{参考}{\lien{ⓔtɯ-ʁo,mbi}{tɯ-ʁo,mbi}}\relationsémantique{Component 1}{\lien{}{tɯ-ʁo}}\relationsémantique{Component 2}{\lien{}{phi}}\end{entrée}

\begin{entrée}{tɯ-ʁwaŋ}{}{ⓔtɯ-ʁwaŋ} 
\classe{np} 
\begin{définition}\pfra{pouvoir}\end{définition}
\begin{définition}\pcmn{权力}\end{définition}\étymologie{dbaŋ}\end{entrée}

\begin{entrée}{tɯ-sa}{}{ⓔtɯ-sa} 
\classe{np} 
\begin{définition}\pfra{dents de sagesse}\end{définition}
\begin{définition}\pcmn{智齿}\end{définition}\end{entrée}

\begin{entrée}{tɯ-sɤsci}{}{ⓔtɯ-sɤsci} 
\classe{np} 
\begin{définition}\pfra{anniversaire, lieu de naissance}\end{définition}
\begin{définition}\pcmn{生日;出生的地方}\end{définition}
\begin{exemple}\pjya{nɤ-sɤsci ŋotɕu pɯ-ŋu?}\hspace{5pt}\pcmn{你生在哪里?}\end{exemple}\étymologie{skʲe}\end{entrée}

\begin{entrée}{tɯ-sɤxɕe}{}{ⓔtɯ-sɤxɕe} 
\classe{np} 
\begin{définition}\pfra{direction, but vers lequel on va}\end{définition}
\begin{définition}\pcmn{去向;去的目标}\end{définition}\relationsémantique{参考}{\lien{ⓔɕe}{ɕe}}\end{entrée}

\begin{entrée}{tɯ-scur}{}{ⓔtɯ-scur} 
\classe{np} 
\begin{définition}\pfra{creux de la main}\end{définition}
\begin{définition}\pcmn{手掌心}\end{définition}\end{entrée}

\begin{entrée}{tɯsi}{}{ⓔtɯsi} 
\classe{n} 
\begin{définition}\pfra{mort}\end{définition}
\begin{définition}\pcmn{死亡}\end{définition}
\begin{exemple}\pjya{tɯsi kɤ-nɤjo ɕti}\hspace{5pt}\pcmn{只有等死}\end{exemple}
\begin{exemple}\pjya{ɯ-tɯsi mda}\hspace{5pt}\pcmn{他的死期到了}\end{exemple}\relationsémantique{参考}{\lien{ⓔsiⓗ1}{si₁}}\end{entrée}

\begin{entrée}{tɯ-skɤlwa}{}{ⓔtɯ-skɤlwa} 
\classe{np} 
\begin{définition}\pfra{partie}\end{définition}
\begin{définition}\pcmn{份}\end{définition}\étymologie{skal.ba}\end{entrée}

\begin{entrée}{tɯ-skɤt}{}{ⓔtɯ-skɤt} 
\classe{np}
\classe{vt} 
\begin{définition}\pfra{voix}\end{définition}
\begin{définition}\pcmn{声音}\end{définition}
\begin{définition}\pfra{parole}\end{définition}
\begin{définition}\pcmn{话}\end{définition}
\begin{définition}\pfra{langue}\end{définition}
\begin{définition}\pcmn{语言}\end{définition}
\begin{exemple}\pjya{kɤ-nɯrɤɣo ɯ-skɤt ɲɯ-sna}\hspace{5pt}\pcmn{他唱歌的声音}\end{exemple}
\begin{exemple}\pjya{tɕhi ɯ-skɤt ɲɯ-ŋu, ɯ-ɲɯ-tɯ-tso}\hspace{5pt}\pcmn{你懂不懂什么意思?}\end{exemple}
\begin{exemple}\pjya{mbro tɯ-skɤt kɯ-tso ci, a-mbro tɤ-rku-nɯ ra}\hspace{5pt}\pcmn{给我准备一匹能听懂人话的马}\end{exemple}
\begin{exemple}\pjya{aʑo kɯ-ŋɤn ɯ-skɤt ntsɯ tu-βze ɲɯ-ŋu}\hspace{5pt}\pcmn{他总是说我的坏话}\end{exemple}\relationsémantique{Component 2}{\lien{}{βzu}}\relationsémantique{参考}{\lien{ⓔβzuⓗ1}{βzu₁}}
\begin{sous-entrée}{tɯ-skɤt,βzu}{ⓔtɯ-skɤtⓝtɯ-skɤt,βzu} 
\classe{np} 
\begin{définition}\pfra{imiter}\end{définition}
\begin{définition}\pcmn{模仿;装作}\end{définition}
\begin{exemple}\pjya{tɕaɣi nɯ pjɯ́-wɣ-sɯxɕɤt tɕe, tɯrme ɯ-skɤt tu-βze spe tu-ti-nɯ ŋgrɤl}\hspace{5pt}\pcmn{如果教它的话,鹦鹉会模仿人话}\end{exemple}\relationsémantique{Component 1}{\lien{ⓔtɯ-skɤt}{tɯ-skɤt}}\end{sous-entrée}

\étymologie{skad}\end{entrée}

\begin{entrée}{tɯ-skhrɯ}{}{ⓔtɯ-skhrɯ} 
\classe{np} 
\begin{définition}\pfra{corps}\end{définition}
\begin{définition}\pcmn{身体}\end{définition}
\begin{exemple}\pjya{ɯ-skhrɯ mɯ-pjɤ-βdi}\hspace{5pt}\pcmn{她怀孕了}\end{exemple}\end{entrée}

\begin{entrée}{tɯ-skhɯrdoʁ}{}{ⓔtɯ-skhɯrdoʁ} 
\classe{np} 
\begin{définition}\pfra{testicules}\end{définition}
\begin{définition}\pcmn{睾丸}\end{définition}\end{entrée}

\begin{entrée}{tɯ-sla}{}{ⓔtɯ-sla} 
\classe{clf} 
\begin{définition}\pfra{un mois}\end{définition}
\begin{définition}\pcmn{一个月}\end{définition}\relationsémantique{参考}{\lien{ⓔsla}{sla}}\relationsémantique{参考}{\lien{ⓔkɤrɤsla}{kɤrɤsla}}\relationsémantique{参考}{\lien{ⓔslɤŋe}{slɤŋe}}\end{entrée}

\begin{entrée}{tɯ-smɤt}{}{ⓔtɯ-smɤt} 
\classe{np} 
\begin{définition}\pfra{le bas du corps}\end{définition}
\begin{définition}\pcmn{下半身}\end{définition}\relationsémantique{参考}{\lien{ⓔsmɤʁjoʁ}{smɤʁjoʁ}}\étymologie{smad}\end{entrée}

\begin{entrée}{tɯ-snaʁ}{}{ⓔtɯ-snaʁ} 
\classe{clf} 
\begin{définition}\pfra{morceau}\end{définition}
\begin{définition}\pcmn{一小块}\end{définition}
\begin{exemple}\pjya{tɯ-ji tɯ-snaʁ}\hspace{5pt}\pcmn{一块地}\end{exemple}\end{entrée}

\begin{entrée}{tɯ-sni}{}{ⓔtɯ-sni} 
\classe{np} 
\begin{définition}\pfra{cœur}\end{définition}
\begin{définition}\pcmn{心脏}\end{définition}
\begin{exemple}\pjya{ɯ-sni ɲɯ-xtɕi}\hspace{5pt}\pcmn{他很胆小}\end{exemple}
\begin{exemple}\pjya{ɯ-sni ɲɤ-zdɯɣ}\hspace{5pt}\pcmn{他很伤心}\end{exemple}\relationsémantique{参考}{\lien{ⓔnɤsnɯndo}{nɤsnɯndo}}\end{entrée}

\begin{entrée}{tɯ-sŋi}{}{ⓔtɯ-sŋi} 
\classe{clf} 
\begin{définition}\pfra{un jour}\end{définition}
\begin{définition}\pcmn{一天}\end{définition}
\begin{exemple}\pjya{wo, nɯ ɯ-sŋi ʑo tɕe a-pɯ-ŋu}\hspace{5pt}\pcmn{就在那一天吧!}\end{exemple}\end{entrée}

\begin{entrée}{tɯsŋi ɲɯntaʁ}{}{ⓔtɯsŋi ɲɯntaʁ} 
\classe{adv} 
\begin{définition}\pfra{un jour entier}\end{définition}
\begin{définition}\pcmn{一整天}\end{définition}\end{entrée}

\begin{entrée}{tɯ-sŋɯro}{}{ⓔtɯ-sŋɯro} 
\classe{np} 
\begin{définition}\pfra{souffle}\end{définition}
\begin{définition}\pcmn{气(一口)}\end{définition}
\begin{exemple}\pjya{a-sŋɯro ci lɤ-tɕat-a}\hspace{5pt}\pcmn{我呼了气}\end{exemple}
\begin{exemple}\pjya{a-sŋɯro ci lɤ-rɤɕi-t-a}\hspace{5pt}\pcmn{我吸了一口气}\end{exemple}
\begin{exemple}\pjya{ɯ-sŋɯro kɤ-lɤt ɲɤ-nɤɕqa}\hspace{5pt}\pcmn{他屏住了呼吸}\end{exemple}
\begin{exemple}\pjya{ɯ-sŋɯro lo-k-ɤrɕo-ci (=ɯ-sroʁ lo-mbrɤt)}\hspace{5pt}\pcmn{他断气了}\end{exemple}\end{entrée}

\begin{entrée}{tɯ-spra}{}{ⓔtɯ-spra} 
\classe{clf} 
\begin{définition}\pfra{poignée}\end{définition}
\begin{définition}\pcmn{一捧}\end{définition}
\begin{exemple}\pjya{mbrɤz tɯ-spra}\hspace{5pt}\pcmn{一捧米}\end{exemple}
\begin{exemple}\pjya{tɯ-jaʁ ɯ-spra}\hspace{5pt}\pcmn{双手捧东西的姿势}\end{exemple}\end{entrée}

\begin{entrée}{tɯsqa}{}{ⓔtɯsqa} 
\classe{n} 
\begin{définition}\pfra{soupe d'avoine, de blé et de haricots que l'on donne aux enfants et à ceux qui récupèrent les excréments de chevaux pour faire de l'engrais}\end{définition}
\begin{définition}\pcmn{粥}\end{définition}
\begin{exemple}\pjya{tɯsqa kɤ-βzu-t-a}\hspace{5pt}\pcmn{我熬了粥}\end{exemple}
\begin{exemple}\pjya{qaj tɤɕi kú-wɣ-sqa tɕe, nɯnɯ tɯsqa tu-kɯ-ti ɲɯ-ŋu}\hspace{5pt}\pcmn{煮了小麦和青稞,那叫做粥}\end{exemple}\relationsémantique{参考}{\lien{ⓔsqa}{sqa}}\relationsémantique{参考}{\lien{ⓔrɯtɯsqa}{rɯtɯsqa}}\end{entrée}

\begin{entrée}{tɯsqar}{}{ⓔtɯsqar} 
\classe{n} 
\begin{définition}\pfra{tsampa}\end{définition}
\begin{définition}\pcmn{糌粑}\end{définition}\end{entrée}

\begin{entrée}{tɯ-sraŋ}{}{ⓔtɯ-sraŋ} 
\classe{clf} 
\begin{définition}\pfra{once}\end{définition}
\begin{définition}\pcmn{一两}\end{définition}\étymologie{sraŋ}\end{entrée}

\begin{entrée}{tɯ-srɤm}{}{ⓔtɯ-srɤm} 
\classe{np} 
\begin{définition}\pfra{lignée}\end{définition}
\begin{définition}\pcmn{人的根源(祖宗,财富等)}\end{définition}\relationsémantique{参考}{\lien{ⓔtɤ-zrɤm}{tɤ-zrɤm}}\end{entrée}

\begin{entrée}{tɯ-sroʁ}{}{ⓔtɯ-sroʁ} 
\classe{np}
\classe{np}
\classe{vi} 
\begin{définition}\pfra{vie}\end{définition}
\begin{définition}\pcmn{生命}\end{définition}\paradigme{dir}{pɯ-}
\begin{définition}\pfra{perdre la vie}\end{définition}
\begin{définition}\pcmn{丧命}\end{définition}
\begin{exemple}\pjya{a-sroʁ ngɯt}\hspace{5pt}\pcmn{我命大(不容易死)}\end{exemple}
\begin{exemple}\pjya{jla nɯ ŋkhorwapa ra nɯ-sroʁ ɯ-kɯ-ndo pɯ-ŋu}\hspace{5pt}\pcmn{犏牛是农民们的命根子}\end{exemple}
\begin{exemple}\pjya{ɯ-sroʁ to-tɕɤt}\hspace{5pt}\pcmn{让他丧命了}\end{exemple}
\begin{exemple}\pjya{nɯŋa dɯxpa ma ɯ-sroʁ pjɤ-ɕe, tɯ-ci ri mɯ-pjɤ-k-ɤʁe-ci}\hspace{5pt}\pcmn{奶牛很可怜,(去喝水的时候)丧了命,一口水也没有喝到}\end{exemple}\relationsémantique{Component 1}{\lien{ⓔtɯ-sroʁ}{tɯ-sroʁ}}\relationsémantique{Component 2}{\lien{ⓔɕe}{ɕe}}
\begin{sous-entrée}{tɯ-sroʁ,ɕe}{ⓔtɯ-sroʁⓝtɯ-sroʁ,ɕe}\end{sous-entrée}

\begin{sous-entrée}{tɯ-sroʁ,mbrɤt}{ⓔtɯ-sroʁⓝtɯ-sroʁ,mbrɤt}\paradigme{dir}{lɤ-}
\begin{définition}\pfra{expirer, rendre le dernier soupir}\end{définition}
\begin{définition}\pcmn{断气}\end{définition}\relationsémantique{参考}{\lien{ⓔnɯsroʁmbrɤt}{nɯsroʁmbrɤt}}\end{sous-entrée}

\étymologie{srog}\end{entrée}

\begin{entrée}{tɯ-sroʁ,nɯwɤtku}{}{ⓔtɯ-sroʁ,nɯwɤtku} 
\classe{np}
\classe{vt} \paradigme{dir}{tɤ-}
\begin{définition}\pfra{risquer sa vie}\end{définition}
\begin{définition}\pcmn{冒生命危险}\end{définition}
\begin{exemple}\pjya{a-sroʁ ku-nɯwɤtke-a ɕti}\hspace{5pt}\pcmn{我正冒着生命危险}\end{exemple}
\begin{exemple}\pjya{a-sroʁ kɤ-nɯwɤtku ʑo kutɕu jɤ-ɣe-a ɕti}\hspace{5pt}\pcmn{我冒着生命危险来到这里}\end{exemple}\relationsémantique{Component 1}{\lien{ⓔtɯ-sroʁ}{tɯ-sroʁ}}\relationsémantique{Component 2}{\lien{}{nɯwɤtku}}\relationsémantique{参考}{\lien{ⓔtɯ-wɤtku}{tɯ-wɤtku}}\end{entrée}

\begin{entrée}{tɯ-sroʁ,ri}{}{ⓔtɯ-sroʁ,ri} 
\classe{vt}
\classe{np}
\classe{vt} \paradigme{dir}{kɤ-}
\begin{définition}\pfra{sauver}\end{définition}
\begin{définition}\pcmn{救}\end{définition}
\begin{exemple}\pjya{nɤ-sroʁ kɤ-ri-t-a}\hspace{5pt}\pcmn{我救了你的命}\end{exemple}\relationsémantique{Component 1}{\lien{ⓔtɯ-sroʁ}{tɯ-sroʁ}}\relationsémantique{Component 2}{\lien{ⓔri}{ri}}\relationsémantique{同义词}{\lien{ⓔfsraŋ}{fsraŋ}}\relationsémantique{参考}{\lien{ⓔriⓗ3}{ri₃}}\end{entrée}

\begin{entrée}{tɯ-srɯt}{}{ⓔtɯ-srɯt} 
\classe{clf} 
\begin{définition}\pfra{un filet de lumière}\end{définition}
\begin{définition}\pcmn{细小缝隙里透过的光}\end{définition}
\begin{exemple}\pjya{tɤŋe tɯ-srɯt}\hspace{5pt}\pcmn{一丝阳光}\end{exemple}\end{entrée}

\begin{entrée}{tɯ-sta}{}{ⓔtɯ-sta} 
\classe{np} \sens{1}
\begin{définition}\pfra{lit}\end{définition}
\begin{définition}\pcmn{床位(睡过的地方)}\end{définition}
\begin{exemple}\pjya{a-sta ku-βze-a}\hspace{5pt}\pcmn{我在铺床}\end{exemple}
\begin{exemple}\pjya{a-sta na-βzu}\hspace{5pt}\pcmn{他给我留了床位(座位)}\end{exemple}\sens{2}
\begin{définition}\pfra{place}\end{définition}
\begin{définition}\pcmn{位子;原位(坐过的地方)}\end{définition}
\begin{exemple}\pjya{khɯtsa a-tʂha pɯ-kɤ-rku ɯ-sta nɯ tɕu a-tɯjno tɤ-rke}\hspace{5pt}\pcmn{在我原来喝过茶的那个碗里给我装菜}\end{exemple}
\begin{exemple}\pjya{jɤ-nɯ-ɣi jɤɣ ma nɤ-sta nɯ-βzu-t-a}\hspace{5pt}\pcmn{你来吧,我给你让位了}\end{exemple}
\begin{exemple}\pjya{nɤ-sta tɤ-nɯ-fse}\hspace{5pt}\pcmn{你规矩一下!;你恢复原来的面目吧!}\end{exemple}\relationsémantique{参考}{\lien{ⓔtɤ-sta}{tɤ-sta}}\relationsémantique{参考}{\lien{ⓔɯ-sta}{ɯ-sta}}\end{entrée}

\begin{entrée}{tɯ-staʁ}{}{ⓔtɯ-staʁ} 
\classe{clf} 
\begin{définition}\pfra{une poêlée}\end{définition}
\begin{définition}\pcmn{一锅,炒出一锅糌粑的时间}\end{définition}
\begin{exemple}\pjya{tɯ-staʁ thɯ-rŋu-t-a}\hspace{5pt}\pcmn{我炒了一锅糌粑}\end{exemple}
\begin{exemple}\pjya{tɯ-staʁ-rŋu jamar pɯ-atsɯtsu}\hspace{5pt}\pcmn{过了炒一锅糌粑的时间}\end{exemple}\end{entrée}

\begin{entrée}{tɯ-stɤt}{}{ⓔtɯ-stɤt} 
\classe{np} 
\begin{définition}\pfra{le haut du corps}\end{définition}
\begin{définition}\pcmn{上半身}\end{définition}
\begin{exemple}\pjya{tɯ-stɤt kɤ-βde}\hspace{5pt}\pcmn{脱下藏装右手的袖子(便于做事)}\end{exemple}\relationsémantique{参考}{\lien{ⓔstɤnga}{stɤnga}}\étymologie{stot}\end{entrée}

\begin{entrée}{tɯ-sɯm}{}{ⓔtɯ-sɯm} 
\classe{np}
\classe{vi} 
\begin{définition}\pfra{état d'esprit}\end{définition}
\begin{définition}\pcmn{性情}\end{définition}
\begin{exemple}\pjya{ɯ-sɯm wuma ʑo ŋɤn}\hspace{5pt}\pcmn{他疑心很重}\end{exemple}
\begin{exemple}\pjya{a-sɯm mba, kɤ-rɤntɕha mɤ-cha-a}\hspace{5pt}\pcmn{我心很软,不能宰猪}\end{exemple}
\begin{exemple}\pjya{ɯ-sɯm mɯ́j-βdi}\hspace{5pt}\pcmn{他不放心}\end{exemple}
\begin{exemple}\pjya{ɯ-sɯm ɲɯ-ɤstu}\hspace{5pt}\pcmn{他很正直,很忠诚}\end{exemple}
\begin{exemple}\pjya{ɯ-sɯm kɯ-sna tu-βze ɲɯ-ŋu}\hspace{5pt}\pcmn{他安的是好心}\end{exemple}
\begin{exemple}\pjya{aʑo a-sɯm tɕe ...}\hspace{5pt}\pcmn{我认为,……}\end{exemple}
\begin{exemple}\pjya{ɯ-sɯm kɤ-χtɤt ʑo pjɯ-rɤβzjoz ŋu}\hspace{5pt}\pcmn{他很专心,一心一意地读书}\end{exemple}
\begin{exemple}\pjya{ɯ-sɯm lu-mtshɤt kɯ-me ci pjɤ-ŋu}\hspace{5pt}\pcmn{他是个贪得无厌的人}\end{exemple}
\begin{exemple}\pjya{aʑo nɯtɕu tɕe nɤ-ɕki ɣi-a tɕe, nɤ-sɯm pjɯ-tu ra}\hspace{5pt}\pcmn{你要有心理准备,我那个时候去你家}\end{exemple}
\begin{exemple}\pjya{tɯrme ɯ-sɯm, mbro ɯ-xtu}\hspace{5pt}\pcmn{人的心,马的肚子(人心无法满足)}\end{exemple}\relationsémantique{Component 2}{\lien{ⓔɕe}{ɕe}}
\begin{sous-entrée}{tɯ-sɯm,ɕe}{ⓔtɯ-sɯmⓝtɯ-sɯm,ɕe} 
\classe{np} 
\begin{définition}\pfra{vouloir}\end{définition}
\begin{définition}\pcmn{想要}\end{définition}
\begin{exemple}\pjya{a-sɯm mɯ-pɯ-ari ri, tɤ-fse-a pɯ-ra}\hspace{5pt}\pcmn{虽然我不愿意,但是只好照着办}\end{exemple}
\begin{exemple}\pjya{ɕɯ-kɤ-rŋgɯ a-sɯm mɯ́j-ɕe}\hspace{5pt}\pcmn{我不想去睡觉}\end{exemple}\relationsémantique{Component 1}{\lien{ⓔtɯ-sɯm}{tɯ-sɯm}}\end{sous-entrée}

\étymologie{sems}\end{entrée}

\begin{entrée}{tɯ-sɯmpa}{}{ⓔtɯ-sɯmpa} 
\classe{n} 
\begin{définition}\pfra{pensée}\end{définition}
\begin{définition}\pcmn{想法}\end{définition}
\begin{exemple}\pjya{tɯ-sɯmpa kɯ-sɤɣdɯɣ}\hspace{5pt}\pcmn{心里很烦}\end{exemple}
\begin{exemple}\pjya{nɤ-sɯmpa kɤ-rku tɯrme}\hspace{5pt}\pcmn{你的意中人}\end{exemple}\étymologie{sems.pa}\end{entrée}

\begin{entrée}{tɯ-sɯso}{}{ⓔtɯ-sɯso} 
\classe{np}
\classe{np}
\classe{vt} \paradigme{dir}{thɯ-}
\begin{définition}\pfra{souvenir, pensée}\end{définition}
\begin{définition}\pcmn{记忆;思维}\end{définition}
\begin{définition}\pfra{avoir l'intention de}\end{définition}
\begin{définition}\pcmn{做……的打算;有……的想法}\end{définition}
\begin{exemple}\pjya{a-sɯso jɤ-ɣe}\hspace{5pt}\pcmn{我想起来了}\end{exemple}
\begin{exemple}\pjya{kɤ-rɤʑi ɣɯ ɯ-sɯso ma-thɯ-tɯ-lɤt}\hspace{5pt}\pcmn{你不要做留下来的打算}\end{exemple}\relationsémantique{参考}{\lien{}{tɯ-ʑosɯso}}\relationsémantique{参考}{\lien{ⓔrɯsɯso}{rɯsɯso}}\relationsémantique{参考}{\lien{ⓔsɯso}{sɯso}}\relationsémantique{Component 1}{\lien{ⓔtɯ-sɯso}{tɯ-sɯso}}\relationsémantique{Component 2}{\lien{}{lɤt}}
\begin{sous-entrée}{tɯ-sɯso,lɤt}{ⓔtɯ-sɯsoⓝtɯ-sɯso,lɤt}\end{sous-entrée}

\end{entrée}

\begin{entrée}{tɯ-ʂɯl}{}{ⓔtɯ-ʂɯl} 
\classe{clf} 
\begin{définition}\pfra{une bande de couleur}\end{définition}
\begin{définition}\pcmn{一条很细的条纹}\end{définition}\étymologie{srol}\end{entrée}

\begin{entrée}{tɯt}{}{ⓔtɯt} 
\classe{vi} \paradigme{dir}{thɯ-}\paradigme{dir}{pɯ-}
\begin{définition}\pfra{mûrir}\end{définition}
\begin{définition}\pcmn{成熟}\end{définition}
\begin{exemple}\pjya{qaj chɤ-tɯt}\hspace{5pt}\pcmn{小麦熟了}\end{exemple}
\begin{exemple}\pjya{sɯmat pjɤ-tɯt}\hspace{5pt}\pcmn{水果熟了}\end{exemple}
\begin{sous-entrée}{ɣɤtɯt}{ⓔtɯtⓝɣɤtɯt} 
\classe{vs} 
\begin{définition}\pfra{qui mûrit vite}\end{définition}
\begin{définition}\pcmn{早熟}\end{définition}\end{sous-entrée}

\end{entrée}

\begin{entrée}{tɯtaʁ}{}{ⓔtɯtaʁ} 
\classe{n} 
\begin{définition}\pfra{textile}\end{définition}
\begin{définition}\pcmn{纺织品}\end{définition}\relationsémantique{参考}{\lien{ⓔtaʁⓗ1}{taʁ₁}}\end{entrée}

\begin{entrée}{tɯtaχte}{}{ⓔtɯtaχte} 
\classe{n} 
\begin{définition}\pfra{méthode de tissage}\end{définition}
\begin{définition}\pcmn{织布的方法,四根线交错着【斜纹子】}\end{définition}
\begin{exemple}\pjya{raz nɯ tɯtaχte pɯ-nɯ-ŋu nɤ, tɯ-mpɕar, ŋgɤrom pɯ-nɯ-ŋu nɤ, tɯ-mpɕar}\hspace{5pt}\pcmn{斜纹子只穿一件,单巴子也只穿一件(因为衣服少)}\end{exemple}\end{entrée}

\begin{entrée}{tɯ-tɤɕrɤz}{}{ⓔtɯ-tɤɕrɤz} 
\classe{n} 
\begin{définition}\pfra{une bande colorée}\end{définition}
\begin{définition}\pcmn{一溜(花纹)}\end{définition}\relationsémantique{参考}{\lien{ⓔarɤɕɯɕrɤz}{arɤɕɯɕrɤz}}\end{entrée}

\begin{entrée}{tɯ-tɤfskɤr}{}{ⓔtɯ-tɤfskɤr} 
\classe{clf} 
\begin{définition}\pfra{un tour}\end{définition}
\begin{définition}\pcmn{一圈}\end{définition}
\begin{exemple}\pjya{tɯ-tɤfskɤr to-lɤt-nɯ}\hspace{5pt}\pcmn{他们绕了一圈}\end{exemple}\relationsémantique{参考}{\lien{ⓔfskɤr}{fskɤr}}\end{entrée}

\begin{entrée}{tɯ-tɤjŋgɤɣ}{}{ⓔtɯ-tɤjŋgɤɣ} 
\classe{clf} 
\begin{définition}\pfra{une boucle, un tour (à propos d'intestins enroulés comme des cordes)}\end{définition}
\begin{définition}\pcmn{一圈(肠子)}\end{définition}
\begin{exemple}\pjya{tɯ-pu tɯ-tɤjŋgɤɣ}\hspace{5pt}\pcmn{一圈肠子}\end{exemple}\relationsémantique{参考}{\lien{ⓔtɯ-ŋgɯl}{tɯ-ŋgɯl}}\end{entrée}

\begin{entrée}{tɯ-tɤkhar}{}{ⓔtɯ-tɤkhar} 
\classe{clf} \sens{1}
\begin{définition}\pfra{une muraille}\end{définition}
\begin{définition}\pcmn{一堵围墙}\end{définition}\sens{2}
\begin{définition}\pfra{une troupe de gens qui entourent un endroit}\end{définition}
\begin{définition}\pcmn{包围某个地方的一群人}\end{définition}\end{entrée}

\begin{entrée}{tɯ-tɤkhrɤz}{}{ⓔtɯ-tɤkhrɤz} 
\classe{clf} 
\begin{définition}\pfra{une rayure, une ligne}\end{définition}
\begin{définition}\pcmn{一排;一路;一层}\end{définition}
\begin{exemple}\pjya{ki tɯ-ŋga ki kɯ-wɣrum tɯ-tɤkhrɤz kɯ-ɲaʁ tɯ-tɤkhrɤz ku-ɕe ɲɯ-ŋu}\hspace{5pt}\pcmn{这件衣服的花纹是一路白色、一路黑色的}\end{exemple}\relationsémantique{同义词}{\lien{ⓔtɯ-tɤmphrɯm}{tɯ-tɤmphrɯm}}\relationsémantique{同义词}{\lien{ⓔtɯ-tɤɕrɤz}{tɯ-tɤɕrɤz}}\relationsémantique{参考}{\lien{ⓔkhrɤtⓗ1}{khrɤt₁}}\end{entrée}

\begin{entrée}{tɯ-tɤlɤβ}{}{ⓔtɯ-tɤlɤβ} 
\classe{clf} 
\begin{définition}\pfra{couche}\end{définition}
\begin{définition}\pcmn{层}\end{définition}
\begin{exemple}\pjya{tɤ-βɟu kɤntɕhɯ-tɤlɤβ lɤ-ta-j}\hspace{5pt}\pcmn{我们铺了几层褥子}\end{exemple}\end{entrée}

\begin{entrée}{tɯ-tɤlia}{}{ⓔtɯ-tɤlia} 
\classe{clf} 
\begin{définition}\pfra{un petit écheveau}\end{définition}
\begin{définition}\pcmn{一小绞}\end{définition}\relationsémantique{参考}{\lien{ⓔtɤβri}{tɤβri}}\end{entrée}

\begin{entrée}{tɯ-tɤlkɯɣ}{}{ⓔtɯ-tɤlkɯɣ} 
\classe{clf} 
\begin{définition}\pfra{un tour de corde (enroulée en cercle)}\end{définition}
\begin{définition}\pcmn{一圈绳子}\end{définition}\end{entrée}

\begin{entrée}{tɯ-tɤmphrɯm}{}{ⓔtɯ-tɤmphrɯm} 
\classe{clf} 
\begin{définition}\pfra{une rangée}\end{définition}
\begin{définition}\pcmn{一路;一排}\end{définition}
\begin{exemple}\pjya{jiʑora kutɕu ɲɯ-ɤʑɯrja-j tɕe, tɤ-tɕɯ tɯ-tɤmphrɯm, tɕheme tɯ-tɤmphrɯm nɯ kɯ-fse tu-kɯ-ndzur ɲɯ-ra}\hspace{5pt}\pcmn{我们在这里排队,男的一排、女的一排}\end{exemple}\relationsémantique{参考}{\lien{ⓔrɤmphrɯm}{rɤmphrɯm}}\end{entrée}

\begin{entrée}{tɯ-tɤndzri}{}{ⓔtɯ-tɤndzri} 
\classe{clf} 
\begin{définition}\pfra{une poignée}\end{définition}
\begin{définition}\pcmn{一绞}\end{définition}
\begin{exemple}\pjya{tɯɣro tɯ-tɤndzri}\hspace{5pt}\pcmn{一绞青草}\end{exemple}\end{entrée}

\begin{entrée}{tɯ-tɤri}{}{ⓔtɯ-tɤri} 
\classe{clf} 
\begin{définition}\pfra{ligature}\end{définition}
\begin{définition}\pcmn{一串}\end{définition}\relationsémantique{参考}{\lien{ⓔtɤ-ri}{tɤ-ri}}\end{entrée}

\begin{entrée}{tɯ-tɤrmbɯ}{}{ⓔtɯ-tɤrmbɯ} 
\classe{clf} 
\begin{définition}\pfra{tas}\end{définition}
\begin{définition}\pcmn{一堆}\end{définition}
\begin{exemple}\pjya{tɯ-ɣli tɯ-tɤrmbɯ}\hspace{5pt}\pcmn{一堆粪}\end{exemple}\relationsémantique{参考}{\lien{ⓔrmbɯ}{rmbɯ}}\end{entrée}

\begin{entrée}{tɯ-tɤrtsɯɣ}{}{ⓔtɯ-tɤrtsɯɣ} 
\classe{clf} 
\begin{définition}\pfra{pile}\end{définition}
\begin{définition}\pcmn{一堆}\end{définition}\end{entrée}

\begin{entrée}{tɯ-tɤrzɯɣ}{}{ⓔtɯ-tɤrzɯɣ} 
\classe{clf} 
\begin{définition}\pfra{une section}\end{définition}
\begin{définition}\pcmn{一段}\end{définition}\relationsémantique{参考}{\lien{ⓔtɯ-rzɯɣ}{tɯ-rzɯɣ}}\end{entrée}

\begin{entrée}{tɯ-tɤʁol}{}{ⓔtɯ-tɤʁol} 
\classe{np} 
\begin{définition}\pfra{se préoccuper de choses qui ne le regarde pas}\end{définition}
\begin{définition}\pcmn{多管闲事}\end{définition}\end{entrée}

\begin{entrée}{tɯ-tɤsɯm}{}{ⓔtɯ-tɤsɯm} 
\classe{clf} 
\begin{définition}\pfra{une partie}\end{définition}
\begin{définition}\pcmn{一份}\end{définition}
\begin{exemple}\pjya{χsɯ-tɤsɯm ɯ-ŋgɯ zɯ tɯ-tɤsɯm}\hspace{5pt}\pcmn{三分之一}\end{exemple}\relationsémantique{同义词}{\lien{ⓔtɯ-tɯcɯr}{tɯ-tɯcɯr}}\end{entrée}

\begin{entrée}{tɯ-tɤtɕhɯ}{}{ⓔtɯ-tɤtɕhɯ} 
\classe{clf} 
\begin{définition}\pfra{un coup de bêche}\end{définition}
\begin{définition}\pcmn{(挖)一锄头}\end{définition}
\begin{exemple}\pjya{qaʁ tɯ-tɤtɕhɯ ta-lɤt}\hspace{5pt}\pcmn{他挖了一锄头}\end{exemple}\end{entrée}

\begin{entrée}{tɯ-tɤxɯr}{}{ⓔtɯ-tɤxɯr} 
\classe{clf} 
\begin{définition}\pfra{un tour}\end{définition}
\begin{définition}\pcmn{一圈}\end{définition}
\begin{exemple}\pjya{jiʑo tɯ-tɤxɯr tɤ-lɤt-i}\hspace{5pt}\pcmn{我们绕了一圈}\end{exemple}\relationsémantique{参考}{\lien{ⓔxɯrxɯr}{xɯrxɯr}}\end{entrée}

\begin{entrée}{tɯ-tɕa}{}{ⓔtɯ-tɕa} 
\classe{np} 
\begin{définition}\pfra{faute}\end{définition}
\begin{définition}\pcmn{错}\end{définition}
\begin{exemple}\pjya{a-tɕa tu}\hspace{5pt}\pcmn{我有错}\end{exemple}
\begin{exemple}\pjya{ki aʑo a-tɕa ŋu}\hspace{5pt}\pcmn{这是我的错}\end{exemple}\end{entrée}

\begin{entrée}{tɯ-tɕa,nɯjɤt}{}{ⓔtɯ-tɕa,nɯjɤt} 
\classe{vt}
\classe{np}
\classe{vt} \paradigme{dir}{nɯ-}
\begin{définition}\pfra{présenter ses excuses}\end{définition}
\begin{définition}\pcmn{赔罪}\end{définition}
\begin{exemple}\pjya{nɤ-tɕa ɣɯ-nɯ-nɯjɤt}\hspace{5pt}\pcmn{你来赔罪}\end{exemple}\relationsémantique{Component 1}{\lien{ⓔtɯ-tɕa}{tɯ-tɕa}}\relationsémantique{Component 2}{\lien{}{nɯjɤt}}\end{entrée}

\begin{entrée}{tɯ-tɕha}{₁}{ⓔtɯ-tɕhaⓗ1} 
\classe{np} 
\begin{définition}\pfra{nouvelles}\end{définition}
\begin{définition}\pcmn{消息;回音}\end{définition}
\begin{exemple}\pjya{a-tɕha ku-me}\hspace{5pt}\pcmn{我没有得到(他的)消息}\end{exemple}
\begin{exemple}\pjya{tɯ-tɕha na-tsɯm}\hspace{5pt}\pcmn{他带了消息}\end{exemple}
\begin{exemple}\pjya{tɯ-tɕha ja-sɤzɣɯt}\hspace{5pt}\pcmn{他把消息带到了}\end{exemple}\relationsémantique{参考}{\lien{ⓔɣɯtɕha}{ɣɯtɕha}}
\begin{sous-entrée}{tɯ-tɕha,kho}{ⓔtɯ-tɕhaⓗ1ⓝtɯ-tɕha,kho}
\begin{définition}\pfra{répondre}\end{définition}
\begin{définition}\pcmn{答复;给……回音;转告;告诉}\end{définition}
\begin{exemple}\pjya{a-tɕɯ kɯ a-tɕha mɯ-na-kho}\hspace{5pt}\pcmn{我儿子没有给我回音}\end{exemple}
\begin{exemple}\pjya{nɤ-kɯ-mŋɤm ɯ-ɲɯ́-mna nɯra a-tɕha a-kɤ-tɯ-khɤm}\hspace{5pt}\pcmn{你病好的消息要告诉我}\end{exemple}
\begin{exemple}\pjya{tɕe tɤ-tɯ-ʑɣɤsɯrtoʁ tɕe, ɯ-ɲɯ́-tɯ-mna nɤ, a-tɕha a-kɤ-tɯ-khɤm}\hspace{5pt}\pcmn{你看了病以后,如果病好的话,你要告诉我}\end{exemple}\end{sous-entrée}

\end{entrée}

\begin{entrée}{tɯ-tɕha}{₂}{ⓔtɯ-tɕhaⓗ2} 
\classe{clf} 
\begin{définition}\pfra{paire}\end{définition}
\begin{définition}\pcmn{一对}\end{définition}
\begin{exemple}\pjya{tɯ-xtsa tɯ-tɕha}\hspace{5pt}\pcmn{一双鞋子}\end{exemple}
\begin{exemple}\pjya{qala tɯ-tɕha}\hspace{5pt}\pcmn{一对兔子}\end{exemple}
\begin{exemple}\pjya{ɯ-me tɯ-tɕha to-sci}\hspace{5pt}\pcmn{她生了一对女儿}\end{exemple}\end{entrée}

\begin{entrée}{tɯ-tɕhaʁ}{}{ⓔtɯ-tɕhaʁ} 
\classe{clf} 
\begin{définition}\pfra{bouquet, boisseau}\end{définition}
\begin{définition}\pcmn{一捆}\end{définition}\end{entrée}

\begin{entrée}{tɯ-tɕhaʁa}{}{ⓔtɯ-tɕhaʁa} 
\classe{np} 
\begin{définition}\pfra{à double paupière}\end{définition}
\begin{définition}\pcmn{双眼皮}\end{définition}
\begin{exemple}\pjya{aʑo a-tɕhaʁa me}\hspace{5pt}\pcmn{我没有双眼皮}\end{exemple}\end{entrée}

\begin{entrée}{tɯ-tɕɯlɤβ}{}{ⓔtɯ-tɕɯlɤβ} 
\classe{clf} 
\begin{définition}\pfra{temps de fumer une pipe}\end{définition}
\begin{définition}\pcmn{一个烟斗的功夫}\end{définition}
\begin{exemple}\pjya{thamakha tɯ-tɕɯlɤβ kɤ-sko ɯ-raŋ jamar}\hspace{5pt}\pcmn{抽一个烟斗的时间(一个烟斗的功夫)}\end{exemple}\relationsémantique{参考}{\lien{ⓔtɕɯlɤβ}{tɕɯlɤβ}}\end{entrée}

\begin{entrée}{tɯtɣa}{₂}{ⓔtɯtɣaⓗ2} 
\classe{n} 
\begin{définition}\pfra{récolte}\end{définition}
\begin{définition}\pcmn{庄稼}\end{définition}\relationsémantique{参考}{\lien{ⓔtɣa}{tɣa}}\end{entrée}

\begin{entrée}{tɯ-tɣa}{₁}{ⓔtɯ-tɣaⓗ1} 
\classe{clf} 
\begin{définition}\pfra{empan}\end{définition}
\begin{définition}\pcmn{一拃,张开大拇指和中指来测距的长度单位。}\end{définition}
\begin{exemple}\pjya{χsɯ-tɣa jamar ma tu-mbro mɤ-cha}\hspace{5pt}\pcmn{只能长到三拃高}\end{exemple}
\begin{sous-entrée}{tɯ-tɣa}{ⓔtɯ-tɣaⓗ1ⓝtɯ-tɣa} 
\classe{np} 
\begin{exemple}\pjya{χsɤ-tɣa, χsɯ-tɣa}\hspace{5pt}\pcmn{三拃}\end{exemple}
\begin{exemple}\pjya{aʑo a-tɣa xtɯt ɕti}\hspace{5pt}\pcmn{我的拃(手指的长度)很短}\end{exemple}\end{sous-entrée}

\end{entrée}

\begin{entrée}{tɯ-thɯ}{}{ⓔtɯ-thɯ} 
\classe{np} 
\begin{définition}\pfra{casserole}\end{définition}
\begin{définition}\pcmn{锅子}\end{définition}
\begin{exemple}\pjya{a-thɯ}\hspace{5pt}\pcmn{我的锅子}\end{exemple}\relationsémantique{参考}{\lien{ⓔnɯthɯ}{nɯthɯ}}\end{entrée}

\begin{entrée}{tɯ-tun}{}{ⓔtɯ-tun} 
\classe{np} 
\begin{définition}\pfra{but, sens}\end{définition}
\begin{définition}\pcmn{目的;意思}\end{définition}
\begin{exemple}\pjya{ɯ-sɤ-ntɕhoz mɯ́j-naχtɕɯɣ tɕe, tɕe ɯ-tun nɯ mɯ́j-naχtɕɯɣ ɲɯ-ŋu}\hspace{5pt}\pcmn{用在不同的语境里(这个词)的意思也不一样}\end{exemple}
\begin{exemple}\pjya{ʁnaʁna ɯ-tun naχtɕɯɣ ɕti}\hspace{5pt}\pcmn{两种(说法)的意思是一样的}\end{exemple}
\begin{exemple}\pjya{ɯ-ti mɤ-naχtɕɯɣ ma ɯ-tun naχtɕɯɣ ɕti}\hspace{5pt}\pcmn{说法不一样,意思一样}\end{exemple}\étymologie{don}\end{entrée}

\begin{entrée}{tɯtsa}{}{ⓔtɯtsa} 
\classe{n} 
\begin{définition}\pfra{enclume}\end{définition}
\begin{définition}\pcmn{铁镦}\end{définition}\end{entrée}

\begin{entrée}{tɯtsɣe}{}{ⓔtɯtsɣe} 
\classe{n} 
\begin{définition}\pfra{commerce}\end{définition}
\begin{définition}\pcmn{生意}\end{définition}
\begin{exemple}\pjya{tɯtsɣe chɤ-ta}\hspace{5pt}\pcmn{他摆了摊子}\end{exemple}\relationsémantique{参考}{\lien{ⓔntsɣe}{ntsɣe}}\end{entrée}

\begin{entrée}{tɯtshi}{}{ⓔtɯtshi} 
\classe{n} 
\begin{définition}\pfra{gruau de riz}\end{définition}
\begin{définition}\pcmn{粥;稀饭}\end{définition}
\begin{définition}\pcmn{我喝了粥}\end{définition}
\begin{définition}\pcmn{我喝了我的粥}\end{définition}
\begin{exemple}\pjya{tɯtshi kɤ-tshi-t-a}\end{exemple}
\begin{exemple}\pjya{a-tɯtshi kɤ-nɯ-tshi-t-a}\end{exemple}\relationsémantique{参考}{\lien{ⓔtshiⓗ1}{tshi₁}}\end{entrée}

\begin{entrée}{tɯtshot}{}{ⓔtɯtshot} 
\classe{n} 
\begin{définition}\pfra{heure}\end{définition}
\begin{définition}\pcmn{钟}\end{définition}
\begin{exemple}\pjya{mɤʑɯ tɯtshot ʁnɯz cho tɯ-phaʁ jamar tɕe tu}\hspace{5pt}\pcmn{还有两个半小时}\end{exemple}
\begin{exemple}\pjya{mɤʑɯ sqamŋu skɤrma tɕe tɯtshot kɯtʂɤɣ zɣɯt ɲɯ-ŋu}\hspace{5pt}\pcmn{还有十五分钟就到了六点种了}\end{exemple}
\begin{exemple}\pjya{kɯkutɕu tɯtshot kɯβde kɤ-azɣɯt}\hspace{5pt}\pcmn{这里已经是四点钟了}\end{exemple}
\begin{exemple}\pjya{tɯtshot χsɯm cho ɣnɤsqi-skɤrma ko-zɣɯt}\hspace{5pt}\pcmn{已经三点二十分了}\end{exemple}
\begin{exemple}\pjya{tɯtshot mɯ-tɤ-rtoʁ-a}\hspace{5pt}\pcmn{我没有看时间}\end{exemple}
\begin{exemple}\pjya{tɯtshot ʑatsa zɣɯt ɲɯ-ŋu}\hspace{5pt}\pcmn{时间差不多到了}\end{exemple}\étymologie{dus.tsʰod}\end{entrée}

\begin{entrée}{tɯ-tshoz}{}{ⓔtɯ-tshoz} 
\classe{clf} 
\begin{définition}\pfra{un ensemble}\end{définition}
\begin{définition}\pcmn{一套}\end{définition}\end{entrée}

\begin{entrée}{tɯ-tshɯɣ}{}{ⓔtɯ-tshɯɣ} 
\classe{np} \sens{1}
\begin{définition}\pfra{articulation}\end{définition}
\begin{définition}\pcmn{关节}\end{définition}\sens{2}
\begin{définition}\pfra{sens (parole)}\end{définition}
\begin{définition}\pcmn{本意}\end{définition}\relationsémantique{同义词}{\lien{ⓔtɯ-rɯrtsɤɣ}{tɯ-rɯrtsɤɣ}}\étymologie{tsʰigs}\end{entrée}

\begin{entrée}{tɯ-tsi}{}{ⓔtɯ-tsi} 
\classe{np} 
\begin{définition}\pfra{longévité}\end{définition}
\begin{définition}\pcmn{寿命}\end{définition}
\begin{définition}\pcmn{她很长寿}\end{définition}
\begin{exemple}\pjya{ɯ-tsi pjɤ-zri}\end{exemple}\end{entrée}

\begin{entrée}{tɯtʂaŋ}{}{ⓔtɯtʂaŋ} 
\classe{n} 
\begin{définition}\pfra{justice}\end{définition}
\begin{définition}\pcmn{公道}\end{définition}
\begin{exemple}\pjya{tɕi-tɯtʂaŋ ci nɯ-lɤt}\hspace{5pt}\pcmn{给我们讨个公道}\end{exemple}\relationsémantique{参考}{\lien{ⓔtʂaŋ}{tʂaŋ}}\end{entrée}

\begin{entrée}{tɯ-tʂɯn}{}{ⓔtɯ-tʂɯn} 
\classe{np} 
\begin{définition}\pfra{faveur, bonté}\end{définition}
\begin{définition}\pcmn{恩}\end{définition}
\begin{exemple}\pjya{a-taʁ nɤ-tʂɯn wxti}\hspace{5pt}\pcmn{你对我恩重如山}\end{exemple}\end{entrée}

\begin{entrée}{tɯ-tɯcɤβ}{}{ⓔtɯ-tɯcɤβ} 
\classe{clf} 
\begin{définition}\pfra{une espèce}\end{définition}
\begin{définition}\pcmn{一类;一种}\end{définition}
\begin{exemple}\pjya{nɯ tɯ-tɯcɤβ ɲɯ-maʁ-nɯ}\hspace{5pt}\pcmn{不是同一类的}\end{exemple}\relationsémantique{参考}{\lien{ⓔtɤ-cɤβ}{tɤ-cɤβ}}\end{entrée}

\begin{entrée}{tɯ-tɯcɯr}{}{ⓔtɯ-tɯcɯr} 
\classe{clf} 
\begin{définition}\pfra{une partie}\end{définition}
\begin{définition}\pcmn{一份}\end{définition}
\begin{exemple}\pjya{χsɯ-tɯcɯr tú-wɣ-lɤt tɕe tɯ-tɯcɯr}\hspace{5pt}\pcmn{三分之一}\end{exemple}
\begin{exemple}\pjya{kɯmŋu-tɯcɯr ɯ-ŋgɯ χsɯ-tɯcɯr}\hspace{5pt}\pcmn{五分之三}\end{exemple}\relationsémantique{同义词}{\lien{ⓔtɯ-tɤsɯm}{tɯ-tɤsɯm}}\end{entrée}

\begin{entrée}{tɯ-tɯkro}{}{ⓔtɯ-tɯkro} 
\classe{clf} 
\begin{définition}\pfra{une part}\end{définition}
\begin{définition}\pcmn{一份}\end{définition}\end{entrée}

\begin{entrée}{tɯ-tɯm}{}{ⓔtɯ-tɯm} 
\classe{clf} 
\begin{définition}\pfra{gousse d'ail}\end{définition}
\begin{définition}\pcmn{一头蒜}\end{définition}\end{entrée}

\begin{entrée}{tɯ-tɯmbrɯ}{}{ⓔtɯ-tɯmbrɯ} 
\classe{clf} 
\begin{définition}\pfra{section (du balcon, pour mettre la nourriture)}\end{définition}
\begin{définition}\pcmn{一格;一空(走檐上,两个柱头之间可以装的粮食)}\end{définition}
\begin{exemple}\pjya{tɤ-rɤku jɤɣɤt zɯ kɯtʂɤɣ-tɯmbrɯ tú-wɣ-ta tɕe mɤro ɯ-taʁ tɕe tɯ-tɯmbrɯ ma me}\hspace{5pt}\pcmn{走檐的六格等于粮架的一格}\end{exemple}\end{entrée}

\begin{entrée}{tɯ-tɯndzɯm}{}{ⓔtɯ-tɯndzɯm} 
\classe{clf} 
\begin{définition}\pfra{bœuf (pour tirer la charrue)}\end{définition}
\begin{définition}\pcmn{(套上犁的)一架牛}\end{définition}\end{entrée}

\begin{entrée}{tɯ-tɯnɯna}{}{ⓔtɯ-tɯnɯna} 
\classe{clf} 
\begin{définition}\pfra{borne (entre deux auberges de repos)}\end{définition}
\begin{définition}\pcmn{一里(两个休息的地点之间的距离)}\end{définition}\end{entrée}

\begin{entrée}{tɯ-tɯŋgɯ}{}{ⓔtɯ-tɯŋgɯ} 
\classe{clf} 
\begin{définition}\pfra{des épis d'orge sur tout le toit}\end{définition}
\begin{définition}\pcmn{满满一房背的青稞穗}\end{définition}
\begin{exemple}\pjya{saχsɯ ɕɯŋgɯ tɯ-tɯŋgɯ pɯ-tɤβ-i}\hspace{5pt}\pcmn{我们在午饭前打完了满满一房背的青稞穗}\end{exemple}\end{entrée}

\begin{entrée}{tɯ-tɯpɕɯrtɕhaʁ}{}{ⓔtɯ-tɯpɕɯrtɕhaʁ} 
\classe{clf} \sens{1}
\begin{définition}\pfra{une génération, une classe}\end{définition}
\begin{définition}\pcmn{一辈;一代、年级}\end{définition}
\begin{exemple}\pjya{jiʑo tɯ-tɯpɕɯrtɕhaʁ pɯ-ŋu-j}\hspace{5pt}\pcmn{我们是同一个年级的}\end{exemple}\sens{2}
\begin{définition}\pfra{un groupe (de gens)}\end{définition}
\begin{définition}\pcmn{一批(人)}\end{définition}\relationsémantique{参考}{\lien{ⓔtɯ-pɕɯrtɕhaʁ}{tɯ-pɕɯrtɕhaʁ}}\end{entrée}

\begin{entrée}{tɯ-tɯphu}{}{ⓔtɯ-tɯphu} 
\classe{clf} \sens{1}
\begin{définition}\pfra{espèce}\end{définition}
\begin{définition}\pcmn{种类,物种}\end{définition}
\begin{exemple}\pjya{tɯ-tɯphu ɕti, ɯ-rmi mɤ-naχtɕɯɣ}\hspace{5pt}\pcmn{是同一种,但是名字不一样}\end{exemple}
\begin{exemple}\pjya{pɣa ɯ-tɯphu dɤn}\hspace{5pt}\pcmn{鸟的种类很多}\end{exemple}\sens{2}
\begin{définition}\pfra{une ruche (abeilles)}\end{définition}
\begin{définition}\pcmn{一窝(蜜蜂)}\end{définition}\end{entrée}

\begin{entrée}{tɯ-tɯpɯ}{}{ⓔtɯ-tɯpɯ} 
\classe{clf} 
\begin{définition}\pfra{famille}\end{définition}
\begin{définition}\pcmn{一户人}\end{définition}\end{entrée}

\begin{entrée}{tɯ-tɯrpa}{}{ⓔtɯ-tɯrpa} 
\classe{clf} 
\begin{définition}\pfra{livre}\end{définition}
\begin{définition}\pcmn{一斤}\end{définition}\relationsémantique{参考}{\lien{ⓔtɯrpa}{tɯrpa}}\end{entrée}

\begin{entrée}{tɯ-wɤt}{}{ⓔtɯ-wɤt} 
\classe{np} 
\begin{définition}\pfra{manche}\end{définition}
\begin{définition}\pcmn{袖子}\end{définition}
\begin{exemple}\pjya{a-wɤt lɤ-pɣaʁ-a}\hspace{5pt}\pcmn{我挽起袖子了}\end{exemple}\relationsémantique{同义词}{\lien{ⓔtɤ-pɤloʁ}{tɤ-pɤloʁ}}\end{entrée}

\begin{entrée}{tɯ-wɤtku}{}{ⓔtɯ-wɤtku} 
\classe{np} 
\begin{définition}\pfra{bout des manches}\end{définition}
\begin{définition}\pcmn{袖口}\end{définition}\relationsémantique{参考}{\lien{ⓔtɯ-sroʁ,nɯwɤtku}{tɯ-sroʁ,nɯwɤtku}}\relationsémantique{参考}{\lien{ⓔtɯ-wɤt}{tɯ-wɤt}}\end{entrée}

\begin{entrée}{tɯwɯ}{}{ⓔtɯwɯ} 
\classe{n} 
\begin{définition}\pfra{bâton du fuseau}\end{définition}
\begin{définition}\pcmn{吊干(纺锤的木棒)}\end{définition}\end{entrée}

\begin{entrée}{tɯwɯr}{}{ⓔtɯwɯr} 
\classe{n} 
\begin{définition}\pfra{habit de pluie}\end{définition}
\begin{définition}\pcmn{蓑衣}\end{définition}\end{entrée}

\begin{entrée}{tɯ-xɕɤt}{}{ⓔtɯ-xɕɤt} 
\classe{np} 
\begin{définition}\pfra{force}\end{définition}
\begin{définition}\pcmn{力气}\end{définition}
\begin{exemple}\pjya{ɯ-xɕɤt tɯ-tu ʑo tha-ɣɤmɯt}\hspace{5pt}\pcmn{他用尽全力吹}\end{exemple}
\begin{sous-entrée}{ɯ-xɕɤt kɯ}{ⓔtɯ-xɕɤtⓝɯ-xɕɤt kɯ} 
\classe{lnk} 
\begin{exemple}\pjya{a-mu kɯ a-nɯzdɯɣ ɯ-xɕɤt kɯ to-ngo}\hspace{5pt}\pcmn{我母亲因为担心我生病了}\end{exemple}\end{sous-entrée}

\étymologie{ɕed}\end{entrée}

\begin{entrée}{tɯ-xɕiridi}{}{ⓔtɯ-xɕiridi} 
\classe{np} 
\begin{définition}\pfra{odeur de dessous de bras}\end{définition}
\begin{définition}\pcmn{狐臭}\end{définition}
\begin{exemple}\pjya{nɤ-xɕiridi ɯ-tɯ-sɤjloʁ nɯ}\hspace{5pt}\pcmn{你的狐臭很难闻}\end{exemple}\relationsémantique{参考}{\lien{ⓔtɤ-di}{tɤ-di}}\end{entrée}

\begin{entrée}{tɯ-xpa}{}{ⓔtɯ-xpa} 
\classe{clf} 
\begin{définition}\pfra{une année}\end{définition}
\begin{définition}\pcmn{一年}\end{définition}
\begin{exemple}\pjya{kɯmŋu-xpa to-tsu (=kɯmŋu-xpa ɲɤ-ɕe)}\hspace{5pt}\pcmn{过了五年}\end{exemple}\relationsémantique{参考}{\lien{ⓔkɤrɤxpa}{kɤrɤxpa}}\relationsémantique{参考}{\lien{ⓔjapa}{japa}}\relationsémantique{参考}{\lien{ⓔɣɯjpa}{ɣɯjpa}}\end{entrée}

\begin{entrée}{tɯxpalɤskɤr}{}{ⓔtɯxpalɤskɤr} 
\classe{adv} 
\begin{définition}\pfra{toute l'année}\end{définition}
\begin{définition}\pcmn{一年到头}\end{définition}\étymologie{lo.skor}\end{entrée}

\begin{entrée}{tɯ-xsoz}{}{ⓔtɯ-xsoz} 
\classe{clf} 
\begin{définition}\pfra{un matin}\end{définition}
\begin{définition}\pcmn{一个早晨}\end{définition}\end{entrée}

\begin{entrée}{tɯ-xtu}{}{ⓔtɯ-xtu} 
\classe{np} 
\begin{définition}\pfra{ventre}\end{définition}
\begin{définition}\pcmn{肚子}\end{définition}
\begin{exemple}\pjya{ɯ-xtu ɯ́-khɯ?}\hspace{5pt}\pcmn{他胃口好不好?}\end{exemple}\relationsémantique{参考}{\lien{ⓔɯ-xtɤfka}{ɯ-xtɤfka}}\relationsémantique{参考}{\lien{ⓔxtɤtshɤt}{xtɤtshɤt}}\end{entrée}

\begin{entrée}{tɯ-xtɤci}{}{ⓔtɯ-xtɤci} 
\classe{np} 
\begin{définition}\pfra{remontées gastriques}\end{définition}
\begin{définition}\pcmn{吐酸水}\end{définition}
\begin{exemple}\pjya{a-xtɤci lo-ɣi}\hspace{5pt}\pcmn{我吐酸水。}\end{exemple}\relationsémantique{参考}{\lien{ⓔtɯ-xtu}{tɯ-xtu}}\relationsémantique{参考}{\lien{ⓔtɯ-ci}{tɯ-ci}}\end{entrée}

\begin{entrée}{tɯxtɤŋɤm}{}{ⓔtɯxtɤŋɤm} 
\classe{n} 
\begin{définition}\pfra{dysenterie}\end{définition}
\begin{définition}\pcmn{痢疾}\end{définition}\relationsémantique{参考}{\lien{ⓔtɯ-xtu}{tɯ-xtu}}\relationsémantique{参考}{\lien{ⓔmŋɤm}{mŋɤm}}\end{entrée}

\begin{entrée}{tɯ-xtɤpa}{}{ⓔtɯ-xtɤpa} 
\classe{np} 
\begin{définition}\pfra{ventre (animal)}\end{définition}
\begin{définition}\pcmn{动物的身下 (因为动物通过四肢站立,因此指的是与肚子相连的部分)}\end{définition}\relationsémantique{参考}{\lien{ⓔtɯ-xtu}{tɯ-xtu}}\end{entrée}

\begin{entrée}{tɯ-xtsa}{}{ⓔtɯ-xtsa} 
\classe{np} 
\begin{définition}\pfra{chaussure}\end{définition}
\begin{définition}\pcmn{鞋子}\end{définition}
\begin{exemple}\pjya{nɤ-xtsa pɯ-tɕɤt}\hspace{5pt}\pcmn{你脱鞋子吧}\end{exemple}\end{entrée}

\begin{entrée}{tɯxtsakɤɣɯɕqri}{}{ⓔtɯxtsakɤɣɯɕqri} 
\classe{n} 
\begin{définition}\pfra{botte dont le haut est en peau de chevrotain, et le milieu en cuir teint en rouge}\end{définition}
\begin{définition}\pcmn{靴筒上部是獐皮子,下部是染成红色的牛皮的一种靴子}\end{définition}
\begin{exemple}\pjya{tɯ-xtsa kɤɣɯɕqri nɯ ɯ-ɕna ɯ-rkɯ cho ɯ-komɤr nɯ ra li mbanaʁtsa cho naχtɕɯɣ ri komɤr ɯ-sɤ-tʂɯβ tɤ-ri nɯ ɯ-mdoʁ kɯ-mpɕɤr ku-lɤt-nɯ ŋgrɤl. ɯ-rkɯ ɯ-tɯ-tʂɯβ nɯ ra kɤntɕhɯ ku-lɤt-nɯ ŋgrɤl. ɯ-rkɯ ɯ-taʁ tɕe komɤr pjɯ-tshoʁ-nɯ ɯ-taʁ kóʁmɯz cɤndʐi, nɯ maʁ nɤ mphrɯɣ nɯ maʁ nɤ raz kɯ-mpɕɤr kɯ-fse pjɯ-tshoʁ-nɯ. ɯ-xtsɤkɤŋgɯ nɯ li smɤɣ thɯ-kɤ-βzu pjɯ-tshoʁ-nɯ ŋu. tɕe nɯ tɯ-xtsa nɯ kɤ-tʂɯβ ɴqa, tɕe kɯ-mpɕɤr kɤ-nɯ-sɯpa ŋu}\hspace{5pt}\pcmn{\lien{}{tɯxtsa kɤɣɯɕqri}(鞋的一种)的鞋尖、鞋边和鞋背上用红皮子,(做法)和黑皮鞋一样。缝红皮子要用彩色的线,缝鞋边的时候要缝几道。鞋边上面先缝上红皮子(作鞋面),然后上面再缝上麝香鹿皮、氆氇或者美观的棉布(作鞋筒)。鞋筒的内层又是用羊毛制成的。缝这种鞋子很困难,但是可以当作是(藏族服装)的装饰品之一。}\end{exemple}\end{entrée}

\begin{entrée}{tɯ-xtsɤkɤŋgɯ}{}{ⓔtɯ-xtsɤkɤŋgɯ} 
\classe{np} 
\begin{définition}\pfra{intérieur de la botte}\end{définition}
\begin{définition}\pcmn{靴筒内衬}\end{définition}\end{entrée}

\begin{entrée}{tɯ-xtsɤmbe}{}{ⓔtɯ-xtsɤmbe} 
\classe{np} 
\begin{définition}\pfra{chaussures abîmées}\end{définition}
\begin{définition}\pcmn{破旧的鞋子}\end{définition}\relationsémantique{参考}{\lien{ⓔtɯ-xtsa}{tɯ-xtsa}}\relationsémantique{参考}{\lien{ⓔtɤ-mbe}{tɤ-mbe}}\end{entrée}

\begin{entrée}{tɯ-xtshi}{}{ⓔtɯ-xtshi} 
\classe{clf} 
\begin{définition}\pfra{une fois}\end{définition}
\begin{définition}\pcmn{一次}\end{définition}\end{entrée}

\begin{entrée}{tɯ-χpɣi}{}{ⓔtɯ-χpɣi} 
\classe{np} 
\begin{définition}\pfra{cuisse}\end{définition}
\begin{définition}\pcmn{大腿}\end{définition}\end{entrée}

\begin{entrée}{tɯ-χpɯm}{}{ⓔtɯ-χpɯm} 
\classe{np} 
\begin{définition}\pfra{genou}\end{définition}
\begin{définition}\pcmn{膝盖}\end{définition}
\begin{exemple}\pjya{a-χpɯm pɯ-tshoʁ-a (lɤ-tshoʁ-a)}\hspace{5pt}\pcmn{我跪下了}\end{exemple}
\begin{exemple}\pjya{tɯ-χpɯm ɯ-rna}\hspace{5pt}\pcmn{膝盖骨}\end{exemple}\relationsémantique{参考}{\lien{ⓔtshoʁⓢ2ⓝtɯ-χpɯm,tshoʁ}{tɯ-χpɯm,tshoʁ}}\end{entrée}

\begin{entrée}{tɯ-χsɯmχsoz/\variante{tɯ-χsoŋχsɤz}}{}{ⓔtɯ-χsɯmχsoz} 
\classe{np}
\classe{vi} 
\begin{définition}\pfra{vigueur}\end{définition}
\begin{définition}\pcmn{精神}\end{définition}
\begin{exemple}\pjya{nɤ-χsɯmχsoz ɯ-tɯ-me nɯ!}\hspace{5pt}\pcmn{你真没有精神}\end{exemple}\relationsémantique{Component 2}{\lien{ⓔɣi}{ɣi}}
\begin{sous-entrée}{ɯ-χsoŋχsɤz,ɣi}{ⓔtɯ-χsɯmχsozⓝɯ-χsoŋχsɤz,ɣi} 
\classe{np} 
\begin{définition}\pfra{retrouver son énergie}\end{définition}
\begin{définition}\pcmn{有精神}\end{définition}
\begin{exemple}\pjya{ɯ-χsoŋχsɤz to-ɣi}\hspace{5pt}\pcmn{精神提起来了}\end{exemple}
\begin{exemple}\pjya{a-χsoŋχsɤz mɯ́j-ɣi}\hspace{5pt}\pcmn{我没有精神}\end{exemple}\relationsémantique{Component 1}{\lien{}{ɯ-χsoŋχsɤz}}\end{sous-entrée}

\begin{sous-entrée}{ɯ-χsoŋχsɤz,sɯɣe}{ⓔtɯ-χsɯmχsozⓝɯ-χsoŋχsɤz,sɯɣe}
\begin{définition}\pfra{donner de l'énergie}\end{définition}
\begin{définition}\pcmn{提精神}\end{définition}
\begin{exemple}\pjya{tʂha kú-wɣ-tshi tɕe tɯ-χsoŋχsɤz tu-sɯɣe ɲɯ-ŋu}\hspace{5pt}\pcmn{喝茶就可以提精神}\end{exemple}\end{sous-entrée}

\end{entrée}

\begin{entrée}{tɯ-χti}{}{ⓔtɯ-χti} 
\classe{np} 
\begin{définition}\pfra{compagnon}\end{définition}
\begin{définition}\pcmn{伙伴}\end{définition}\end{entrée}

\begin{entrée}{tɯ-χtispa}{}{ⓔtɯ-χtispa} 
\classe{np} 
\begin{définition}\pfra{fiancé}\end{définition}
\begin{définition}\pcmn{未婚夫}\end{définition}\end{entrée}

\begin{entrée}{tɯz}{}{ⓔtɯz} 
\classe{n} 
\begin{définition}\pfra{époque}\end{définition}
\begin{définition}\pcmn{时代}\end{définition}
\begin{exemple}\pjya{ndʑi-tɯz a-pɯ-βdi}\hspace{5pt}\pcmn{祝你们俩一生平安}\end{exemple}\étymologie{dus}\end{entrée}

\begin{entrée}{tɯ-zboʁ}{}{ⓔtɯ-zboʁ} 
\classe{clf} 
\begin{définition}\pfra{une poignée (herbes, oignon)}\end{définition}
\begin{définition}\pcmn{一把(草,葱)}\end{définition}
\begin{exemple}\pjya{xɕaj tɯ-zboʁ}\hspace{5pt}\pcmn{一把草}\end{exemple}
\begin{exemple}\pjya{ɕku tɯ-zboʁ}\hspace{5pt}\pcmn{一把葱子}\end{exemple}
\begin{exemple}\pjya{tasa tɯ-zboʁ}\hspace{5pt}\pcmn{一把麻皮}\end{exemple}\end{entrée}

\begin{entrée}{tɯ-zda}{}{ⓔtɯ-zda} 
\classe{np} 
\begin{définition}\pfra{compagnon, autre}\end{définition}
\begin{définition}\pcmn{伙伴;别人}\end{définition}
\begin{exemple}\pjya{kha ɯ-zda kɤ-rɤʑit-a}\hspace{5pt}\pcmn{我待在家里陪他了}\end{exemple}\relationsémantique{参考}{\lien{ⓔɣɤzda}{ɣɤzda}}\relationsémantique{参考}{\lien{ⓔrɤzda}{rɤzda}}\relationsémantique{参考}{\lien{ⓔsɤzda}{sɤzda}}\end{entrée}

\begin{entrée}{tɯ-zgoɕɤrɯ}{}{ⓔtɯ-zgoɕɤrɯ} 
\classe{np} 
\begin{définition}\pfra{colonne vertébrale}\end{définition}
\begin{définition}\pcmn{脊椎骨}\end{définition}\end{entrée}

\begin{entrée}{tɯ-zgrɯ}{}{ⓔtɯ-zgrɯ} 
\classe{np} 
\begin{définition}\pfra{coude}\end{définition}
\begin{définition}\pcmn{肘}\end{définition}\relationsémantique{参考}{\lien{ⓔtɯ-ɣrɯmke}{tɯ-ɣrɯmke}}\relationsémantique{参考}{\lien{ⓔzgrɯtɕhɯ}{zgrɯtɕhɯ}}\end{entrée}

\begin{entrée}{tɯ-zloʁ}{}{ⓔtɯ-zloʁ} 
\classe{clf} 
\begin{définition}\pfra{fois}\end{définition}
\begin{définition}\pcmn{一倍}\end{définition}
\begin{exemple}\pjya{ʁɲɯ-zloʁ}\hspace{5pt}\pcmn{两倍}\end{exemple}
\begin{exemple}\pjya{kɯβde nɯ ʁnɯz ɣɯ ʁɲɯ-zloʁ ŋu}\hspace{5pt}\pcmn{四是二的两倍}\end{exemple}\end{entrée}

\begin{entrée}{tɯ-ʑi,loʁ}{}{ⓔtɯ-ʑi,loʁ} 
\classe{np}
\classe{vs} 
\begin{définition}\pfra{avoir la nausée}\end{définition}
\begin{définition}\pcmn{感到恶心}\end{définition}
\begin{exemple}\pjya{ɯʑo ɯ-ʑi ɲɯ-loʁ}\hspace{5pt}\pcmn{他感到恶心}\end{exemple}
\begin{exemple}\pjya{a-ʑi ɲɯ-loʁ}\hspace{5pt}\pcmn{我感到恶心}\end{exemple}\relationsémantique{参考}{\lien{ⓔnɤʑɯloʁ}{nɤʑɯloʁ}}\relationsémantique{Component 1}{\lien{}{tɯ-ʑi}}\relationsémantique{Component 2}{\lien{ⓔloʁ}{loʁ}}\end{entrée}

\begin{entrée}{tɯʑŋgrɯm}{}{ⓔtɯʑŋgrɯm} 
\classe{n} 
\begin{définition}\pfra{cartilage}\end{définition}
\begin{définition}\pcmn{软骨}\end{définition}
\begin{exemple}\pjya{a-ɕna ɣɯ ɯ-tɯʑŋgrɯm}\hspace{5pt}\pcmn{我鼻子的软骨}\end{exemple}\end{entrée}

\begin{entrée}{tɯʑo}{}{ⓔtɯʑo} 
\classe{pro} 
\begin{définition}\pfra{soi-même (générique)}\end{définition}
\begin{définition}\pcmn{自己(泛指的人称代词)}\end{définition}\end{entrée}

\begin{entrée}{tɯʑo-sɯso}{}{ⓔtɯʑo-sɯso} 
\classe{np} 
\begin{définition}\pfra{n'en faire qu'à sa tête}\end{définition}
\begin{définition}\pcmn{随心所欲}\end{définition}
\begin{exemple}\pjya{aʑo-sɯso kɤ-nɯpa mɯ́j-khɯ}\hspace{5pt}\pcmn{我不能随心所欲}\end{exemple}\relationsémantique{参考}{\lien{ⓔtɯ-sɯso}{tɯ-sɯso}}\relationsémantique{参考}{\lien{ⓔsɯso}{sɯso}}\end{entrée}

\begin{entrée}{tɯ-ʑrɤz}{}{ⓔtɯ-ʑrɤz} 
\classe{clf} 
\begin{définition}\pfra{une bande (couleur)}\end{définition}
\begin{définition}\pcmn{一路(颜色)}\end{définition}\end{entrée}

\begin{entrée}{tɯ-ʑɯβ}{}{ⓔtɯ-ʑɯβ} 
\classe{np} 
\begin{définition}\pfra{somnolence}\end{définition}
\begin{définition}\pcmn{瞌睡}\end{définition}
\begin{exemple}\pjya{a-ʑɯβ mɯ́j-ɣi}\hspace{5pt}\pcmn{我睡不着}\end{exemple}
\begin{exemple}\pjya{a-ʑɯβ mɯ́j-sɯɣe-nɯ}\hspace{5pt}\pcmn{他们不让我睡觉}\end{exemple}
\begin{exemple}\pjya{kɤntɕhaʁ ɲɯ-ɣɤɣɯrɣɯrnɯ tɕe, a-ʑɯβ mɯ́j-sɯɣe-nɯ}\hspace{5pt}\pcmn{街上很吵,弄得我睡不着}\end{exemple}
\begin{exemple}\pjya{a-rɣa ɲɯ-ɣɤɕqali-nɯ tɕe, a-ʑɯβ mɯ́j-sɯɣe-nɯ}\end{exemple}
\begin{exemple}\pjya{a-rɣa ɲɯ-ɣɤɲcɣɤlɤt-nɯ tɕe, a-ʑɯβ mɯ́j-sɯɣe-nɯ}\hspace{5pt}\pcmn{我的邻居很吵,弄得我睡不着}\end{exemple}\relationsémantique{参考}{\lien{ⓔnɯʑɯβ}{nɯʑɯβ}}\end{entrée}

\newpage\caractère{ɯ}

\begin{entrée}{ɯ-βdoʁ}{}{ⓔɯ-βdoʁ} 
\classe{np} 
\begin{définition}\pfra{pénis, zizi (enfant)}\end{définition}
\begin{définition}\pcmn{阴茎}\end{définition}\end{entrée}

\begin{entrée}{ɯ-βlu}{}{ⓔɯ-βlu} 
\classe{np} 
\begin{définition}\pfra{méthode, astuce, stratagème}\end{définition}
\begin{définition}\pcmn{办法,计谋}\end{définition}
\begin{exemple}\pjya{nɤ-βlu ci tu-tɕat-a}\hspace{5pt}\pcmn{我给你出个主意}\end{exemple}\relationsémantique{参考}{\lien{ⓔɯ-βlaβlu}{ɯ-βlaβlu}}\relationsémantique{参考}{\lien{ⓔnɯβlu}{nɯβlu}}\relationsémantique{参考}{\lien{ⓔaɣɯβlu}{aɣɯβlu}}\étymologie{blo}\end{entrée}

\begin{entrée}{ɯ-βlaβlu}{}{ⓔɯ-βlaβlu} 
\classe{n} 
\begin{définition}\pfra{astuce, méthode}\end{définition}
\begin{définition}\pcmn{办法}\end{définition}
\begin{exemple}\pjya{nɤ-βlaβlu ɲɯ-dɤn}\hspace{5pt}\pcmn{你有很多办法}\end{exemple}\relationsémantique{参考}{\lien{ⓔɯ-βlu}{ɯ-βlu}}\end{entrée}

\begin{entrée}{ɯ-βraʁ}{}{ⓔɯ-βraʁ} 
\classe{np} 
\begin{définition}\pfra{symbole, signe}\end{définition}
\begin{définition}\pcmn{象征;预兆}\end{définition}
\begin{exemple}\pjya{ɣɯjpa qartsɯ ʁmɯrcɯ ɲɯ-dɤn tɕe, nɯ tɤɕi kɯ-pe ɯ-βraʁ ŋu}\hspace{5pt}\pcmn{今年夏天画眉很多,表示今年青稞的产量很高}\end{exemple}\end{entrée}

\begin{entrée}{ɯ-βzɯr}{}{ⓔɯ-βzɯr} 
\classe{np} 
\begin{définition}\pfra{angle}\end{définition}
\begin{définition}\pcmn{角落}\end{définition}
\begin{exemple}\pjya{tɕoχtsi ɣɯ ɯ-βzɯr ri a-jaʁ kɤ-nɯ-rpu-t-a tɕe ɲɯ-mŋɤm}\hspace{5pt}\pcmn{我把手撞到桌子的一角,很痛}\end{exemple}\étymologie{bzur}\end{entrée}

\begin{entrée}{ɯ-cu}{}{ⓔɯ-cu} 
\classe{np} 
\begin{définition}\pfra{ingrédients}\end{définition}
\begin{définition}\pcmn{配料}\end{définition}
\begin{exemple}\pjya{hajtsu cho ɕku nɯ tɤjko ɯ-cu pjɯ́-wɣ-lɤt tɕe mɯm}\hspace{5pt}\pcmn{在酸菜里添加黑椒和蒜就好吃}\end{exemple}
\begin{exemple}\pjya{a-pi kɯ tatshi ɯ-skɤt ɣɯ ɯ-cu kɤmɲɯ tu-βze ɲɯ-ŋu tɕe ɲɯ-sɤtso, tatshi skɤt ʁɟa a-tɤ-ti tɕe mɯ́j-sɤtso}\hspace{5pt}\pcmn{他把干木鸟话掺到大藏话里面说就好懂一些}\end{exemple}\relationsémantique{参考}{\lien{ⓔacu}{acu}}\end{entrée}

\begin{entrée}{ɯ-cɤβ}{}{ⓔɯ-cɤβ} 
\classe{np} 
\begin{définition}\pfra{gousse, cosse}\end{définition}
\begin{définition}\pcmn{荚果}\end{définition}
\begin{exemple}\pjya{staχpɯ ɯ-cɤβ}\hspace{5pt}\pcmn{豌豆的荚果}\end{exemple}\relationsémantique{参考}{\lien{ⓔrɤcɤβ}{rɤcɤβ}}\end{entrée}

\begin{entrée}{ɯ-ciqa}{}{ⓔɯ-ciqa} 
\classe{adv} 
\begin{définition}\pfra{origine, depuis toujours, en fait}\end{définition}
\begin{définition}\pcmn{根源;从来;说到底}\end{définition}
\begin{exemple}\pjya{ɯʑo ɯ-ciqa ʑo pɯ-nɯ-pɯ-pe ɕti}\hspace{5pt}\pcmn{他从来都是个好人}\end{exemple}\end{entrée}

\begin{entrée}{ɯ-cɯma}{}{ⓔɯ-cɯma} 
\classe{np} 
\begin{définition}\pfra{malheur, calamité}\end{définition}
\begin{définition}\pcmn{闯的祸}\end{définition}
\begin{exemple}\pjya{aʑo a-cɯma ŋu}\hspace{5pt}\pcmn{是我闯的祸}\end{exemple}\end{entrée}

\begin{entrée}{ɯ-ɕi}{}{ⓔɯ-ɕi} 
\classe{np} 
\begin{définition}\pfra{plante fanée}\end{définition}
\begin{définition}\pcmn{枯萎了的草(一年生的植物)}\end{définition}
\begin{exemple}\pjya{dɤrʁɯ ɯ-ɕi}\hspace{5pt}\pcmn{枯萎了的蕨苔}\end{exemple}\end{entrée}

\begin{entrée}{ɯ-ɕki}{}{ⓔɯ-ɕki} 
\classe{postp} \sens{1}
\begin{définition}\pfra{à côté}\end{définition}
\begin{définition}\pcmn{旁边}\end{définition}\relationsémantique{同义词}{\lien{ⓔɯ-rkɯ}{ɯ-rkɯ}}\sens{2}
\begin{définition}\pfra{datif}\end{définition}
\begin{définition}\pcmn{与格}\end{définition}\relationsémantique{同义词}{\lien{ⓔɯ-phe}{ɯ-phe}}\end{entrée}

\begin{entrée}{ɯ-ɕpɯz}{}{ⓔɯ-ɕpɯz} 
\classe{np} 
\begin{définition}\pfra{imitation, dessin}\end{définition}
\begin{définition}\pcmn{模样;假冒}\end{définition}
\begin{exemple}\pjya{jɯɣi ɯ-taʁ tɯrme ɯ-ɕpɯz ci ɣɤʑu}\hspace{5pt}\pcmn{书上有人的模样}\end{exemple}
\begin{exemple}\pjya{tɯrme ɯ-ɕpɯz to-βzu}\hspace{5pt}\pcmn{他做了个人的模样}\end{exemple}
\begin{exemple}\pjya{nɤki nɤ-khɯtsa ɯ-ɕpɯz ɲɯ-ɕti ma koŋla ɲɯ-maʁ}\hspace{5pt}\pcmn{你那个碗是伪造品,不是真的}\end{exemple}\relationsémantique{参考}{\lien{ⓔnɯɕpɯz}{nɯɕpɯz}}\end{entrée}

\begin{entrée}{ɯ-ɕtɯrme}{}{ⓔɯ-ɕtɯrme} 
\classe{np} 
\begin{définition}\pfra{poils pubien (femme)}\end{définition}
\begin{définition}\pcmn{(女子的)阴毛}\end{définition}\end{entrée}

\begin{entrée}{ɯ-ɕɯɕaŋ}{}{ⓔɯ-ɕɯɕaŋ} 
\classe{np} 
\begin{définition}\pfra{limite}\end{définition}
\begin{définition}\pcmn{界限}\end{définition}
\begin{exemple}\pjya{ki ɯ-ɕɯɕaŋ ki kɤmɲɯ sɤtɕha ŋu}\hspace{5pt}\pcmn{以这个为界就是干木鸟村的地盘}\end{exemple}\end{entrée}

\begin{entrée}{ɯ-du}{}{ⓔɯ-du} 
\classe{np} 
\begin{définition}\pfra{(être tiré) au sort}\end{définition}
\begin{définition}\pcmn{抽到了(抽签的时候)}\end{définition}
\begin{exemple}\pjya{ɯ-du to-ɣi}\hspace{5pt}\pcmn{他抽到了签}\end{exemple}
\begin{exemple}\pjya{a-du tɤ-ɣe}\hspace{5pt}\pcmn{我抽到了签}\end{exemple}\end{entrée}

\begin{entrée}{ɯ-do}{}{ⓔɯ-do} 
\classe{np} 
\begin{définition}\pfra{vieux}\end{définition}
\begin{définition}\pcmn{老的}\end{définition}\end{entrée}

\begin{entrée}{ɯ-dɯɕŋaʁ}{}{ⓔɯ-dɯɕŋaʁ} 
\classe{np} 
\begin{définition}\pfra{puanteur très forte}\end{définition}
\begin{définition}\pcmn{特别浓的臭味}\end{définition}
\begin{exemple}\pjya{ndʑiŋgri ɯ-dɯɕŋaʁ ɲɯ-sɤjloʁ}\hspace{5pt}\pcmn{臭草非常臭}\end{exemple}\end{entrée}

\begin{entrée}{ɯ-dɯχɯn}{}{ⓔɯ-dɯχɯn} 
\classe{np} 
\begin{définition}\pfra{bonne odeur}\end{définition}
\begin{définition}\pcmn{香味}\end{définition}
\begin{exemple}\pjya{ɯ-dɯχɯn pɯ-mtsham-a}\hspace{5pt}\pcmn{我闻到了香味}\end{exemple}
\begin{exemple}\pjya{ɯ-dɯχɯn tɤ-nɤmnam-a}\hspace{5pt}\pcmn{我闻了一下香味}\end{exemple}\relationsémantique{同义词}{\lien{ⓔɯ-ɟɤm}{ɯ-ɟɤm}}\relationsémantique{参考}{\lien{ⓔaɣɯdɯχɯn}{aɣɯdɯχɯn}}\relationsémantique{参考}{\lien{ⓔtɤ-di}{tɤ-di}}\end{entrée}

\begin{entrée}{ɯ-fsu}{}{ⓔɯ-fsu} 
\classe{np} 
\begin{définition}\pfra{égal, autant}\end{définition}
\begin{définition}\pcmn{一样}\end{définition}
\begin{exemple}\pjya{tɤ-pɤtso cho-wxti tɕe, ɯ-mu ɯ-fsu ʑo cho-ɬoʁ}\hspace{5pt}\pcmn{孩子长大了,长得跟他母亲一样高}\end{exemple}\relationsémantique{参考}{\lien{ⓔsɤfsu}{sɤfsu}}\relationsémantique{参考}{\lien{ⓔɣɯfsu}{ɣɯfsu}}\relationsémantique{参考}{\lien{ⓔafsɯfsu}{afsɯfsu}}\end{entrée}

\begin{entrée}{ɯ-grɤl}{}{ⓔɯ-grɤl} 
\classe{np} 
\begin{définition}\pfra{raison}\end{définition}
\begin{définition}\pcmn{理由;规律;条理}\end{définition}\relationsémantique{参考}{\lien{ⓔnɯgrɤl}{nɯgrɤl}}
\begin{sous-entrée}{ɯ-grɤl,me}{ⓔɯ-grɤlⓝɯ-grɤl,me}\sens{1}
\begin{définition}\pfra{être en désordre, n'importe comment}\end{définition}
\begin{définition}\pcmn{没有条理;随便}\end{définition}
\begin{exemple}\pjya{laχtɕha ɯ-grɤl kɯ-me ʑo ɲɤ-ta}\hspace{5pt}\pcmn{东西放得没有条理}\end{exemple}
\begin{exemple}\pjya{si nɯ ɯ-grɤl kɯ-me ma-thɯ́-wɣ-phɯt ra ma tɕe chɯ-ɤrɕo ɕti}\hspace{5pt}\pcmn{不能随便砍树,不然(树林)就会被砍光}\end{exemple}\end{sous-entrée}

\sens{2}
\begin{définition}\pfra{être discourtois}\end{définition}
\begin{définition}\pcmn{没有分寸,没有礼貌}\end{définition}
\begin{exemple}\pjya{nɤ-grɤl ɯ-tɯ-me}\hspace{5pt}\pcmn{你无法无天,没有分寸,没有礼貌}\end{exemple}\sens{3}
\begin{définition}\pfra{innombrable}\end{définition}
\begin{définition}\pcmn{无数}\end{définition}\étymologie{gral}\end{entrée}

\begin{entrée}{ɯ-ɣɲaʁ}{}{ⓔɯ-ɣɲaʁ} 
\classe{np} 
\begin{définition}\pfra{résultat désastreux (que l'on se cause à soi-même)}\end{définition}
\begin{définition}\pcmn{恶果(自己给自己带来的)}\end{définition}
\begin{exemple}\pjya{a-ɣɲaʁ to-nɯ-βzu-t-a}\hspace{5pt}\pcmn{我自己给自己找了麻烦}\end{exemple}
\begin{exemple}\pjya{ɯ-ɣɲaʁ ɲɤ-ɕe}\hspace{5pt}\pcmn{对他不利了}\end{exemple}
\begin{exemple}\pjya{mtshalu ɲɤ-phɯt ri kó-wɣ-mtsɯɣ tɕe mɤ́ɣrɤz nɤ ɯ-ɣɲaʁ ɲɤ-ɕe}\hspace{5pt}\pcmn{他摘荨麻的时候被蛰到了,倒是自找麻烦}\end{exemple}\end{entrée}

\begin{entrée}{ɯ-ɣɲɟɯ}{}{ⓔɯ-ɣɲɟɯ} 
\classe{np} 
\begin{définition}\pfra{orifice}\end{définition}
\begin{définition}\pcmn{洞}\end{définition}\relationsémantique{参考}{\lien{ⓔɲɟɯ}{ɲɟɯ}}\relationsémantique{参考}{\lien{ⓔkhɯɣɲɟɯ}{khɯɣɲɟɯ}}\relationsémantique{参考}{\lien{ⓔqaɲɯɣɲɟɯ}{qaɲɯɣɲɟɯ}}\relationsémantique{参考}{\lien{ⓔtɯ-ɕnɤɣɲɟɯ}{tɯ-ɕnɤɣɲɟɯ}}\relationsémantique{参考}{\lien{ⓔtɤkhɯɣɲɟɯ}{tɤkhɯɣɲɟɯ}}\relationsémantique{参考}{\lien{ⓔtɯ-rnɤɣɲɟɯ}{tɯ-rnɤɣɲɟɯ}}\end{entrée}

\begin{entrée}{ɯ-ɣrom}{}{ⓔɯ-ɣrom} 
\classe{np} 
\begin{définition}\pfra{(chose) séchée}\end{définition}
\begin{définition}\pcmn{干的}\end{définition}
\begin{exemple}\pjya{tɤjmɤɣ ɯ-ɣrom}\hspace{5pt}\pcmn{干菌子}\end{exemple}
\begin{exemple}\pjya{tɤ-mthɯm ɯ-ɣrom}\hspace{5pt}\pcmn{干的肉}\end{exemple}\relationsémantique{参考}{\lien{ⓔrom}{rom}}\end{entrée}

\begin{entrée}{ɯ-jlu}{}{ⓔɯ-jlu} 
\classe{np} \sens{1}
\begin{définition}\pfra{cru, pas cuit}\end{définition}
\begin{définition}\pcmn{生的}\end{définition}
\begin{exemple}\pjya{stoʁ ɯ-jlu kɤ-ndza mɤ-mɯm}\hspace{5pt}\pcmn{生胡豆不好吃}\end{exemple}\sens{2}
\begin{définition}\pfra{uniquement}\end{définition}
\begin{définition}\pcmn{光是}\end{définition}
\begin{exemple}\pjya{@cai kɯnɤ tɤ-ndze, tɯmgo ɯ-jlu ʑo ma-tɤ-tɯ-ndze}\hspace{5pt}\pcmn{你也吃菜,不要光吃饭}\end{exemple}\end{entrée}

\begin{entrée}{ɯ-jndoʁ}{}{ⓔɯ-jndoʁ} 
\classe{np} 
\begin{définition}\pfra{racine de radis ou de navet}\end{définition}
\begin{définition}\pcmn{芜菁;萝卜的根}\end{définition}\end{entrée}

\begin{entrée}{ɯ-jndɯz}{}{ⓔɯ-jndɯz} 
\classe{np} 
\begin{définition}\pfra{fils qui débordent, verticilles}\end{définition}
\begin{définition}\pcmn{絮线}\end{définition}
\begin{exemple}\pjya{thaχtsa ɯ-sɤ-sɤʑa cho ɯ-sɤɣ-jɤɣ ɯ-tɤ-ri nɯ ɯ-jndɯz rmi}\hspace{5pt}\pcmn{花带的开头和结尾部分的线叫絮线}\end{exemple}\end{entrée}

\begin{entrée}{ɯ-jŋgɯ}{}{ⓔɯ-jŋgɯ} 
\classe{np} \sens{1}
\begin{définition}\pfra{mangeoir}\end{définition}
\begin{définition}\pcmn{槽}\end{définition}\sens{2}
\begin{définition}\pfra{gamelle (du chien)}\end{définition}
\begin{définition}\pcmn{狗槽;狗碗;槽}\end{définition}
\begin{exemple}\pjya{khɯna kɯ ɯ-jŋgɯ to-nɯntsɯɣ}\hspace{5pt}\pcmn{狗舔了它的碗}\end{exemple}\relationsémantique{参考}{\lien{ⓔpɤjŋgɯ}{pɤjŋgɯ}}\relationsémantique{参考}{\lien{ⓔqhajŋgɯ}{qhajŋgɯ}}\relationsémantique{参考}{\lien{ⓔkhɯjŋgɯ}{khɯjŋgɯ}}\end{entrée}

\begin{entrée}{ɯjo}{}{ⓔɯjo} 
\classe{n} 
\begin{définition}\pfra{cochon de deux ans}\end{définition}
\begin{définition}\pcmn{两岁的猪}\end{définition}
\begin{exemple}\pjya{paʁ ɯ-jo}\hspace{5pt}\pcmn{两岁的猪}\end{exemple}
\begin{exemple}\pjya{phaʁrgot ɯ-jo}\hspace{5pt}\pcmn{两岁的野猪}\end{exemple}\end{entrée}

\begin{entrée}{ɯ-jɯ}{}{ⓔɯ-jɯ} 
\classe{np} 
\begin{définition}\pfra{poignée}\end{définition}
\begin{définition}\pcmn{把子}\end{définition}\étymologie{ju.ba}\end{entrée}

\begin{entrée}{ɯ-jɯja}{}{ⓔɯ-jɯja} 
\classe{np} 
\begin{définition}\pfra{au fur et à mesure que (+verbe)}\end{définition}
\begin{définition}\pcmn{随着(加动词)}\end{définition}
\begin{exemple}\pjya{tɯ-mɯ kɤ-lɤt ɯ-jɯja tɕe, ɯ-thoʁ tu-ɣɤrcoʁ ɲɯ-ŋu}\hspace{5pt}\pcmn{雨水越大,地面的稀泥越多}\end{exemple}
\begin{exemple}\pjya{tɤ-pɤtso thɯ-wxti ɯ-jɯja nɤ, ɯ-ndzɯmbra kɤ-sɤsɯɣ ra}\hspace{5pt}\pcmn{随着孩子长大,要加强对他的教育}\end{exemple}\end{entrée}

\begin{entrée}{ɯ-ɟɤm}{}{ⓔɯ-ɟɤm} 
\classe{np} 
\begin{définition}\pfra{goût}\end{définition}
\begin{définition}\pcmn{香味}\end{définition}
\begin{exemple}\pjya{ki cha ki ɯ-ɟɤm kɯ-mɯm ci ɲɯ-ŋu}\hspace{5pt}\pcmn{这种茶味道很香}\end{exemple}\relationsémantique{同义词}{\lien{ⓔɯ-dɯχɯn}{ɯ-dɯχɯn}}\end{entrée}

\begin{entrée}{ɯ-kɤcu}{}{ⓔɯ-kɤcu} 
\classe{np} 
\begin{définition}\pfra{à l'est}\end{définition}
\begin{définition}\pcmn{在东方}\end{définition}\end{entrée}

\begin{entrée}{ɯ-kɤlɤjme}{}{ⓔɯ-kɤlɤjme} 
\classe{np} 
\begin{définition}\pfra{tête à queue}\end{définition}
\begin{définition}\pcmn{头尾颠倒}\end{définition}
\begin{exemple}\pjya{a-kɤlɤjme pɯ-ru tɕe, tɤ-rɤru-a}\hspace{5pt}\pcmn{我跌倒了再爬起来}\end{exemple}\end{entrée}

\begin{entrée}{ɯ-kɤndɯt}{}{ⓔɯ-kɤndɯt} 
\classe{np} 
\begin{définition}\pfra{amant, relation extra-conjugale}\end{définition}
\begin{définition}\pcmn{情人}\end{définition}\étymologie{ɴdod}\end{entrée}

\begin{entrée}{ɯ-kɤɲɟoʁ}{}{ⓔɯ-kɤɲɟoʁ} 
\classe{np} 
\begin{définition}\pfra{tissu cousu sur l'ouverture au bas des robes}\end{définition}
\begin{définition}\pcmn{缝在衣服叉开的部分上面的布块}\end{définition}\end{entrée}

\begin{entrée}{ɯ-kɤχcɤl}{}{ⓔɯ-kɤχcɤl} 
\classe{np} 
\begin{définition}\pfra{sur le dessus}\end{définition}
\begin{définition}\pcmn{在顶上,顶端}\end{définition}
\begin{exemple}\pjya{nɤ-kɤχcɤl}\hspace{5pt}\pcmn{在你的头顶部}\end{exemple}
\begin{exemple}\pjya{si ɯ-kɤχcɤl}\hspace{5pt}\pcmn{树的顶端}\end{exemple}\end{entrée}

\begin{entrée}{ɯ-khon}{}{ⓔɯ-khon} 
\classe{np} 
\begin{définition}\pfra{au cas où}\end{définition}
\begin{définition}\pcmn{以防万一}\end{définition}
\begin{exemple}\pjya{aʑo tɯ-ci khro tɤ-rku-t-a tɕe, fso ɣɯ ɯ-khon ŋu}\hspace{5pt}\pcmn{万一明天停水,我预备了一些水}\end{exemple}\relationsémantique{参考}{\lien{ⓔrɯkhon}{rɯkhon}}\end{entrée}

\begin{entrée}{ɯ-khrakhrɯ}{}{ⓔɯ-khrakhrɯ} 
\classe{np} 
\begin{définition}\pfra{(pains) rassis}\end{définition}
\begin{définition}\pcmn{干了的(馍馍)}\end{définition}
\begin{exemple}\pjya{qajɣi ɯ-khrakhrɯ nɯ ɲɤ-mbi}\hspace{5pt}\pcmn{(大女儿)把干了的馍馍给他吃}\end{exemple}\relationsémantique{参考}{\lien{ⓔkhrɯ}{khrɯ}}\end{entrée}

\begin{entrée}{ɯ-khrɤt}{}{ⓔɯ-khrɤt} 
\classe{np}
\classe{np}
\classe{vt} \paradigme{dir}{kɤ-}
\begin{définition}\pfra{déterminé}\end{définition}
\begin{définition}\pcmn{规定的}\end{définition}
\begin{définition}\pfra{contrôler}\end{définition}
\begin{définition}\pcmn{掌握}\end{définition}
\begin{exemple}\pjya{smɤn kɤ-ndza nɯ ɯ-khrɤt kɯ-tu ɕti}\hspace{5pt}\pcmn{吃是有规定的(时间和数量)}\end{exemple}
\begin{exemple}\pjya{mkhɯrlu ɯ-ŋgɯ laχtɕha ɯ-khrɤt ma tú-wɣ-rku mɤ-khɯ}\hspace{5pt}\pcmn{在车里装东西不能超过规定的重量}\end{exemple}
\begin{exemple}\pjya{@xingqier tɯtshot χsɯm nɯ tɕiʑo @dianhua kɤ-lɤt ɯ-khrɤt nɯ-βzu-tɕi ŋu.}\hspace{5pt}\pcmn{我们星期二三点钟约了时间打电话}\end{exemple}
\begin{exemple}\pjya{soz kɤ-rɤru tɕhi jamar mda ɯ-khrɤt kú-wɣ-nɯ-ndo ra}\hspace{5pt}\pcmn{要掌握好自己早上几点钟起床}\end{exemple}\relationsémantique{Component 1}{\lien{ⓔɯ-khrɤt}{ɯ-khrɤt}}\relationsémantique{Component 2}{\lien{ⓔndo}{ndo}}
\begin{sous-entrée}{ɯ-khrɤt,ndo}{ⓔɯ-khrɤtⓝɯ-khrɤt,ndo}\end{sous-entrée}

\end{entrée}

\begin{entrée}{ɯ-khɯkha}{}{ⓔɯ-khɯkha} 
\classe{np} \sens{1}
\begin{définition}\pfra{l'un après l'autre}\end{définition}
\begin{définition}\pcmn{一个接着一个}\end{définition}\sens{2}
\begin{définition}\pfra{au moment où}\end{définition}
\begin{définition}\pcmn{的时候}\end{définition}
\begin{exemple}\pjya{ɲɯ-nɤqambɯmbjom ɯ-khɯkha ʑo ju-βji}\hspace{5pt}\pcmn{小鸟飞的时候,隼来追它}\end{exemple}\end{entrée}

\begin{entrée}{ɯ-khɯkhɤl}{}{ⓔɯ-khɯkhɤl} 
\classe{np} 
\begin{définition}\pfra{à certains endroits}\end{définition}
\begin{définition}\pcmn{某些地方或部位,或指东一块儿西一块儿(不是完整的一块)}\end{définition}
\begin{exemple}\pjya{tɯ-ŋga ɯ-khɯkhɤl ko-spoʁ}\hspace{5pt}\pcmn{衣服有些地方破了}\end{exemple}
\begin{exemple}\pjya{tɯ-mɯ ɯ-khɯkhɤl ɲɤ-jɯm}\hspace{5pt}\pcmn{有些地方没有云}\end{exemple}\relationsémantique{参考}{\lien{ⓔtɯ-khɤl}{tɯ-khɤl}}\end{entrée}

\begin{entrée}{ɯkɯki}{}{ⓔɯkɯki} 
\classe{pro} 
\begin{définition}\pfra{celui-ci}\end{définition}
\begin{définition}\pcmn{这个}\end{définition}\relationsémantique{参考}{\lien{ⓔkɯki}{kɯki}}\end{entrée}

\begin{entrée}{ɯkɯkɯra}{}{ⓔɯkɯkɯra} 
\classe{pro} 
\begin{définition}\pfra{ceux-ci}\end{définition}
\begin{définition}\pcmn{这些}\end{définition}\relationsémantique{参考}{\lien{ⓔkɯra}{kɯra}}\end{entrée}

\begin{entrée}{ɯ-kɯmpɕɤr}{}{ⓔɯ-kɯmpɕɤr} 
\classe{np} 
\begin{définition}\pfra{décoration}\end{définition}
\begin{définition}\pcmn{装饰}\end{définition}
\begin{exemple}\pjya{ɯ-kɯmpɕɤr ci nɤme-a}\hspace{5pt}\pcmn{我要给他好看的!}\end{exemple}\relationsémantique{参考}{\lien{ⓔmpɕɤr}{mpɕɤr}}\end{entrée}

\begin{entrée}{ɯ-lu,cɯ}{}{ⓔɯ-lu,cɯ} 
\classe{np}
\classe{vt} \paradigme{dir}{pɯ-}
\begin{définition}\pfra{perdre conscience}\end{définition}
\begin{définition}\pcmn{昏倒;昏迷}\end{définition}
\begin{exemple}\pjya{aʑo pɯ-xtɕi-a, khɤxtɤndo pɯ-atar-a tɕe, a-lu pjɤ-cɯ khi}\hspace{5pt}\pcmn{我小的时候,从楼梯上摔下来,就昏倒了}\end{exemple}\relationsémantique{Component 1}{\lien{}{ɯ-lu}}\relationsémantique{Component 2}{\lien{ⓔcɯⓗ3}{cɯ}}
\begin{sous-entrée}{ɯ-lu,sɯxcɯ}{ⓔɯ-lu,cɯⓝɯ-lu,sɯxcɯ} 
\classe{vt}
\classe{np}
\classe{vt} \paradigme{dir}{pɯ-}
\begin{définition}\pfra{faire perdre conscience}\end{définition}
\begin{définition}\pcmn{令人失去知觉}\end{définition}
\begin{exemple}\pjya{to-ʁndɯ tɕe ɯ-lu ʑo pjɤ-sɯxcɯ}\hspace{5pt}\pcmn{他打了他,令他失去了知觉}\end{exemple}\relationsémantique{Component 1}{\lien{}{ɯ-lu}}\relationsémantique{Component 2}{\lien{ⓔsɯxcɯ}{sɯxcɯ}}\end{sous-entrée}

\end{entrée}

\begin{entrée}{ɯ-luj}{}{ⓔɯ-luj} 
\classe{np} 
\begin{définition}\pfra{enveloppe extérieure}\end{définition}
\begin{définition}\pcmn{外层}\end{définition}\end{entrée}

\begin{entrée}{ɯ-locu}{}{ⓔɯ-locu} 
\classe{np} 
\begin{définition}\pfra{en amont}\end{définition}
\begin{définition}\pcmn{在上游}\end{définition}\end{entrée}

\begin{entrée}{ɯ-ltɕi}{}{ⓔɯ-ltɕi} 
\classe{np} 
\begin{définition}\pfra{franges (sac)}\end{définition}
\begin{définition}\pcmn{【须须】(挎包的)}\end{définition}\end{entrée}

\begin{entrée}{ɯ-lɯtshɤt}{}{ⓔɯ-lɯtshɤt} 
\classe{np} 
\begin{définition}\pfra{du même âge}\end{définition}
\begin{définition}\pcmn{同岁,跟自己年龄差不多}\end{définition}
\begin{exemple}\pjya{ɯ-lɯtshɤt ra nɯ-rca ɲɯ-rga}\hspace{5pt}\pcmn{他喜欢跟自己同龄的小孩子在一起}\end{exemple}\relationsémantique{参考}{\lien{ⓔaɣɯlɯtshɤt}{aɣɯlɯtshɤt}}\end{entrée}

\begin{entrée}{ɯ-lwa}{}{ⓔɯ-lwa} 
\classe{np} 
\begin{définition}\pfra{crinière}\end{définition}
\begin{définition}\pcmn{马鬃}\end{définition}\end{entrée}

\begin{entrée}{ɯ-maŋ}{}{ⓔɯ-maŋ} 
\classe{np} 
\begin{définition}\pfra{en groupes}\end{définition}
\begin{définition}\pcmn{很多群}\end{définition}
\begin{exemple}\pjya{tɯrme ɯ-maŋ ʑo to-k-ɤwɯwum-nɯ-ci}\hspace{5pt}\pcmn{很多人集中在那里了}\end{exemple}\end{entrée}

\begin{entrée}{ɯ-mat}{}{ⓔɯ-mat} 
\classe{np} 
\begin{définition}\pfra{fruit}\end{définition}
\begin{définition}\pcmn{果子}\end{définition}\end{entrée}

\begin{entrée}{ɯ-mbrɤzɯ}{}{ⓔɯ-mbrɤzɯ} 
\classe{np} 
\begin{définition}\pfra{résultat}\end{définition}
\begin{définition}\pcmn{结果}\end{définition}
\begin{exemple}\pjya{a-mbrɤzɯ pɯ-smɯn}\hspace{5pt}\pcmn{我得到了(好)结果}\end{exemple}
\begin{exemple}\pjya{tɯʑo kɯ ɯ-mbrɤzɯ ɣɯ-nɯ-ndza kɯ-ra ŋu}\hspace{5pt}\pcmn{要自食其果的}\end{exemple}\étymologie{ⁿbras}\end{entrée}

\begin{entrée}{ɯ-mbɯrme}{}{ⓔɯ-mbɯrme} 
\classe{np} 
\begin{définition}\pfra{poils pubiens (homme)}\end{définition}
\begin{définition}\pcmn{阴毛}\end{définition}\relationsémantique{参考}{\lien{ⓔtɯ-mbɯ}{tɯ-mbɯ}}\end{entrée}

\begin{entrée}{ɯ-mbɯrqhu}{}{ⓔɯ-mbɯrqhu} 
\classe{np} 
\begin{définition}\pfra{prépuce}\end{définition}
\begin{définition}\pcmn{包皮}\end{définition}\relationsémantique{参考}{\lien{ⓔtɯ-mbɯ}{tɯ-mbɯ}}\end{entrée}

\begin{entrée}{ɯ-mdoʁ}{}{ⓔɯ-mdoʁ} 
\classe{np} \sens{1}
\begin{définition}\pfra{couleur}\end{définition}
\begin{définition}\pcmn{颜色}\end{définition}
\begin{exemple}\pjya{qromke ɯ-mdoʁ asɯ-ndo}\hspace{5pt}\pcmn{带有紫色}\end{exemple}\sens{2}
\begin{définition}\pfra{il semble que}\end{définition}
\begin{définition}\pcmn{看起来}\end{définition}
\begin{exemple}\pjya{jɯɣmɯr mɤʑɯ tɤjpa lɤt ɲɯ-ŋu ɯ-mdoʁ}\hspace{5pt}\pcmn{看起来今天晚上可能还会下下雪}\end{exemple}\relationsémantique{参考}{\lien{ⓔnɯmdoʁ}{nɯmdoʁ}}\étymologie{mdog}\end{entrée}

\begin{entrée}{ɯ-mdʑɯɣ}{}{ⓔɯ-mdʑɯɣ} 
\classe{np} 
\begin{définition}\pfra{à la fin}\end{définition}
\begin{définition}\pcmn{最后}\end{définition}
\begin{exemple}\pjya{ɯ-mdʑɯɣ tɕe saχsɤl ɕti nɤ}\hspace{5pt}\pcmn{最后就会清楚的}\end{exemple}\end{entrée}

\begin{entrée}{ɯ-mujmaj}{}{ⓔɯ-mujmaj} 
\classe{np} 
\begin{définition}\pfra{branche d'arbre}\end{définition}
\begin{définition}\pcmn{树的枝桠}\end{définition}\relationsémantique{参考}{\lien{ⓔnɤmujmaj}{nɤmujmaj}}\end{entrée}

\begin{entrée}{ɯ-mnɯ}{}{ⓔɯ-mnɯ} 
\classe{np} \sens{1}
\begin{définition}\pfra{pousse, petite branche}\end{définition}
\begin{définition}\pcmn{树木新发出来的枝条}\end{définition}\sens{2}
\begin{définition}\pfra{pousse de bambou}\end{définition}
\begin{définition}\pcmn{笋}\end{définition}\end{entrée}

\begin{entrée}{ɯ-mɲaʁsta}{}{ⓔɯ-mɲaʁsta} 
\classe{np} 
\begin{définition}\pfra{magnanimité}\end{définition}
\begin{définition}\pcmn{心胸}\end{définition}
\begin{exemple}\pjya{ɯ-mɲaʁsta kɯ-ŋgɤr ci ŋu, kɯ-xtɕɯ-xtɕi kɯnɤ ʑo naʁdɯɣ ɕti}\hspace{5pt}\pcmn{他是个心胸狭窄的人,小事情也计较}\end{exemple}
\begin{exemple}\pjya{ɯ-mɲaʁsta jom}\hspace{5pt}\pcmn{他心胸宽阔}\end{exemple}\end{entrée}

\begin{entrée}{ɯ-mɲoz}{}{ⓔɯ-mɲoz}\relationsémantique{参考}{\lien{ⓔmɲoⓗ1}{mɲo₁}}\end{entrée}

\begin{entrée}{ɯ-mŋu}{}{ⓔɯ-mŋu} 
\classe{np} \sens{1}
\begin{définition}\pfra{ouverture (sac)}\end{définition}
\begin{définition}\pcmn{(背篼、口袋)的口}\end{définition}
\begin{exemple}\pjya{tɤ-fkɯm ɯ-mŋu}\hspace{5pt}\pcmn{口袋的口}\end{exemple}\relationsémantique{参考}{\lien{ⓔnɯmŋu}{nɯmŋu}}\sens{2}
\begin{définition}\pfra{partie du toit sous laquelle il n'y a pas de balcon}\end{définition}
\begin{définition}\pcmn{房顶靠墙的一边(没有走檐)}\end{définition}
\begin{exemple}\pjya{khɤxtu ɯ-mŋu}\end{exemple}\sens{3}
\begin{définition}\pfra{côté du champs près de la montagne}\end{définition}
\begin{définition}\pcmn{(田地)靠山的那一边}\end{définition}
\begin{exemple}\pjya{tɯji ɯ-mŋu}\hspace{5pt}\pcmn{田地靠山的那一角}\end{exemple}\relationsémantique{参考}{\lien{ⓔkhɯmŋu}{khɯmŋu}}
\begin{sous-entrée}{ɯ-mŋuɕɯmŋu}{ⓔɯ-mŋuⓢ3ⓝɯ-mŋuɕɯmŋu} 
\classe{np} 
\begin{définition}\pfra{le plus au bord}\end{définition}
\begin{définition}\pcmn{最边缘}\end{définition}
\begin{exemple}\pjya{tʂha tɤ́-wɣ-rku tɕe khɯtsa ɯ-mŋuɕɯmŋu sthɯci tu-zɣɯt mɤ-ra ma kɤ-ndo tɕe sɤɕke}\hspace{5pt}\pcmn{倒茶的时候,不要倒得太满,不然端的时候会烫手}\end{exemple}\end{sous-entrée}

\end{entrée}

\begin{entrée}{ɯ-mphru}{}{ⓔɯ-mphru} 
\classe{np} 
\begin{définition}\pfra{à la suite de}\end{définition}
\begin{définition}\pcmn{……以后}\end{définition}
\begin{exemple}\pjya{ɯ-mphru pjɯ-ʑe-a tɕe pjɯ-ndɯn-a ŋu nɤ}\hspace{5pt}\pcmn{我从头开始读}\end{exemple}\sens{1}
\begin{définition}\pfra{continuer}\end{définition}
\begin{définition}\pcmn{继续}\end{définition}
\begin{exemple}\pjya{saχsɯ ɯ-qhu tɕe ɯ-mphru nɯ-mɟa-tɕi}\hspace{5pt}\pcmn{我们午后再继续}\end{exemple}\sens{2}
\begin{définition}\pfra{prendre la relève, remplacer}\end{définition}
\begin{définition}\pcmn{接班}\end{définition}
\begin{exemple}\pjya{ci nɯ kɯ ɯʑo ɯ-mphru pjɤ-mɟa}\hspace{5pt}\pcmn{那个人接过了他的班}\end{exemple}
\begin{sous-entrée}{ɯ-mphru,mɟa}{ⓔɯ-mphruⓝɯ-mphru,mɟa} 
\classe{n}
\classe{vt} \paradigme{dir}{pɯ-}\paradigme{dir}{nɯ-}\relationsémantique{Component 1}{\lien{ⓔɯ-mphru}{ɯ-mphru}}\relationsémantique{Component 2}{\lien{ⓔmɟa}{mɟa}}\end{sous-entrée}

\étymologie{ⁿpʰro (deɦi.ⁿpʰror)}\end{entrée}

\begin{entrée}{ɯ-mphɯsku}{}{ⓔɯ-mphɯsku} 
\classe{np} 
\begin{définition}\pfra{croupe (bovins, ovins)}\end{définition}
\begin{définition}\pcmn{臀部(牛、羊)}\end{définition}\relationsémantique{参考}{\lien{ⓔtɯ-mphɯz}{tɯ-mphɯz}}\relationsémantique{参考}{\lien{ⓔtɯ-ku}{tɯ-ku}}\end{entrée}

\begin{entrée}{ɯ-mthoŋ,nɯɬoʁ}{}{ⓔɯ-mthoŋ,nɯɬoʁ} 
\classe{np}
\classe{np}
\classe{vi} \paradigme{dir}{nɯ-}
\begin{définition}\pfra{se trahir, dévoiler ses défauts}\end{définition}
\begin{définition}\pcmn{露馅}\end{définition}
\begin{exemple}\pjya{a-mthoŋ nɯ-ɬoʁ ɲɯ-ŋu}\hspace{5pt}\pcmn{我要露馅了}\end{exemple}
\begin{exemple}\pjya{ɯ-mthoŋ pjɤ-nɯ-ɬoʁ}\hspace{5pt}\pcmn{他露馅了}\end{exemple}\relationsémantique{Component 1}{\lien{}{ɯ-mthoŋ}}\relationsémantique{Component 2}{\lien{ⓔnɯɬoʁ}{nɯɬoʁ}}\end{entrée}

\begin{entrée}{ɯ-mtsioʁ}{}{ⓔɯ-mtsioʁ} 
\classe{np} 
\begin{définition}\pfra{bec}\end{définition}
\begin{définition}\pcmn{鸟嘴}\end{définition}\end{entrée}

\begin{entrée}{ɯ-ndaŋ,lɤt}{}{ⓔɯ-ndaŋ,lɤt} 
\classe{np}
\classe{np}
\classe{vt} 
\begin{définition}\pfra{penser à}\end{définition}
\begin{définition}\pcmn{(为某人)着想}\end{définition}\paradigme{dir}{pɯ-}
\begin{exemple}\pjya{nɯ ɯ-ndaŋ pjɯ-lat-a ŋu}\hspace{5pt}\pcmn{我为他着想}\end{exemple}
\begin{exemple}\pjya{tɯ-zda ra nɯ-ndaŋ pjɯ́-wɣ-lɤt ra ma nɯ-ʁdɯxpa ɣɯ-βzu mɤ-βdi}\hspace{5pt}\pcmn{他考虑到别人,不妨碍别人}\end{exemple}\relationsémantique{Component 1}{\lien{}{ɯ-ndaŋ}}\relationsémantique{Component 2}{\lien{}{lɤt}}\relationsémantique{参考}{\lien{ⓔrɤndaŋ}{rɤndaŋ}}\relationsémantique{参考}{\lien{ⓔlɤtⓗ1}{lɤt₁}}\étymologie{ⁿdaŋ}\end{entrée}

\begin{entrée}{ɯ-ndɤcu}{}{ⓔɯ-ndɤcu} 
\classe{np} 
\begin{définition}\pfra{à l'ouest}\end{définition}
\begin{définition}\pcmn{在西方}\end{définition}\end{entrée}

\begin{entrée}{ɯ-ndo}{}{ⓔɯ-ndo} 
\classe{np} \sens{1}
\begin{définition}\pfra{bord, côté du champs en direction du fleuve}\end{définition}
\begin{définition}\pcmn{边缘,田地靠河流的那一边}\end{définition}
\begin{exemple}\pjya{tɯ-ŋga ɯ-ndo}\hspace{5pt}\pcmn{衣服的下边}\end{exemple}\sens{2}
\begin{définition}\pfra{fin}\end{définition}
\begin{définition}\pcmn{结尾}\end{définition}
\begin{exemple}\pjya{tu-kɯ-stu ra ma ɯ-ndo tɕe mɤ-pe}\hspace{5pt}\pcmn{要注意一点,不然到最后就没有好结果}\end{exemple}
\begin{exemple}\pjya{ɯ-ndo tɕe mɤʑɯ kɤ-sɤpe ra}\hspace{5pt}\pcmn{事情到了最后要做得更好}\end{exemple}
\begin{sous-entrée}{ɯ-ndoɕɯndo}{ⓔɯ-ndoⓢ2ⓝɯ-ndoɕɯndo}
\begin{définition}\pfra{le côté le plus lointain}\end{définition}
\begin{définition}\pcmn{最边缘}\end{définition}
\begin{exemple}\pjya{nɤmkha ɯ-ndoɕɯndo ʑo zɯ ʑŋgri kɯ-tʂɯ-tʂot ʑo tɯ-rdoʁ ɣɤʑu}\hspace{5pt}\pcmn{在天的最边缘有一颗很明亮的星星}\end{exemple}\end{sous-entrée}

\end{entrée}

\begin{entrée}{ɯ-ndzɯɣlɯz}{}{ⓔɯ-ndzɯɣlɯz} 
\classe{np} 
\begin{définition}\pfra{comportement}\end{définition}
\begin{définition}\pcmn{举止}\end{définition}
\begin{exemple}\pjya{ɯʑo ɯ-ndzɯɣlɯz mɤ-kɯ-βdi ɕti ma tɯrme pe ɯ-sɯm sna}\hspace{5pt}\pcmn{这个人虽然举止不好,但是心底很善良}\end{exemple}\end{entrée}

\begin{entrée}{ɯ-ndzɯɣ,maʁ}{}{ⓔɯ-ndzɯɣ,maʁ} 
\classe{np}
\classe{vs} \paradigme{dir}{pɯ-}
\begin{définition}\pfra{être terrible}\end{définition}
\begin{définition}\pcmn{厉害}\end{définition}
\begin{exemple}\pjya{a-mu kɯ ``a-ku ɯ-tɯ-mŋɤm kɯ ɯ-ndzɯɣ ʑo ɲɯ-maʁ"}\hspace{5pt}\pcmn{妈妈说“我头疼得很厉害”}\end{exemple}
\begin{exemple}\pjya{jisŋi kɤntɕhaʁ tɯrme ɯ-tɯ-dɤn kɯ ɯ-ndzɯɣ ɲɯ-maʁ}\hspace{5pt}\pcmn{今天街上人特别多}\end{exemple}
\begin{exemple}\pjya{khɤcɤl ɯ-ndzɯɣ to-ɣɤmaʁ-ndʑi ɕti}\hspace{5pt}\pcmn{他们谈了很久}\end{exemple}\relationsémantique{Component 1}{\lien{}{ɯ-ndzɯɣ}}\relationsémantique{Component 2}{\lien{ⓔmaʁⓗ2}{maʁ}}\end{entrée}

\begin{entrée}{ɯ-ndzɯndzoʁ}{}{ⓔɯ-ndzɯndzoʁ} 
\classe{np} 
\begin{définition}\pfra{suivant sans relâche}\end{définition}
\begin{définition}\pcmn{紧紧地跟着}\end{définition}\relationsémantique{参考}{\lien{ⓔndzoʁ}{ndzoʁ}}\end{entrée}

\begin{entrée}{ɯntɕe}{}{ⓔɯntɕe} 
\classe{cnj} 
\begin{définition}\pfra{ensuite}\end{définition}
\begin{définition}\pcmn{以后}\end{définition}\end{entrée}

\begin{entrée}{ɯ-ntɕhantɕhɯr}{}{ⓔɯ-ntɕhantɕhɯr} 
\classe{np}  
\grammaire{n.rdpl} 
\begin{définition}\pfra{morceaux, débris}\end{définition}
\begin{définition}\pcmn{碎片}\end{définition}\relationsémantique{参考}{\lien{ⓔtɯ-ntɕhɯr}{tɯ-ntɕhɯr}}\end{entrée}

\begin{entrée}{ɯnɯnɯ}{}{ⓔɯnɯnɯ} 
\classe{pro} 
\begin{définition}\pfra{celui-là}\end{définition}
\begin{définition}\pcmn{那个}\end{définition}\relationsémantique{参考}{\lien{ⓔnɯnɯⓗ2}{nɯnɯ₂}}\end{entrée}

\begin{entrée}{ɯnɯnɯra}{}{ⓔɯnɯnɯra} 
\classe{pro} 
\begin{définition}\pfra{ceux-là}\end{définition}
\begin{définition}\pcmn{那些}\end{définition}\relationsémantique{参考}{\lien{ⓔnɯra}{nɯra}}\end{entrée}

\begin{entrée}{ɯŋ}{}{ⓔɯŋ} 
\classe{intj} 
\begin{définition}\pfra{exprime l'hésitation}\end{définition}
\begin{définition}\pcmn{表示犹豫,不高兴}\end{définition}\end{entrée}

\begin{entrée}{ɯŋaj,βzu}{}{ⓔɯŋaj,βzu} 
\classe{np}
\classe{vt} \paradigme{dir}{nɯ-}
\begin{définition}\pfra{être autosatisfait}\end{définition}
\begin{définition}\pcmn{得意,原谅自己,自满,无视自己的过错}\end{définition}
\begin{exemple}\pjya{ɯŋaj ma-nɯ-tɯ-nɯ-βze kɯ tɤ-stu tɤ-mbat}\hspace{5pt}\pcmn{你不要这么得意,努力一点}\end{exemple}\relationsémantique{Component 1}{\lien{}{ɯ-ŋaj}}\relationsémantique{Component 2}{\lien{}{βzu}}\relationsémantique{参考}{\lien{ⓔβzuⓗ1}{βzu₁}}\end{entrée}

\begin{entrée}{ɯŋgu}{}{ⓔɯŋgu} 
\classe{adv} 
\begin{définition}\pfra{autrefois, d'abord}\end{définition}
\begin{définition}\pcmn{本来,首先}\end{définition}
\begin{exemple}\pjya{ɯʑo ɯŋgu jɤznɤ taʁndo kɯ-tso ci pjɤ-ŋu ri, ɯ-ndo tɕe taʁndo mɯ-ɲɤ-tso}\hspace{5pt}\pcmn{他本来很听话的,后来就不听话了}\end{exemple}\relationsémantique{反义词}{\lien{ⓔɯ-ndo}{ɯ-ndo}}\relationsémantique{参考}{\lien{ⓔnɯŋgu}{nɯŋgu}}\étymologie{ⁿgo}\end{entrée}

\begin{entrée}{ɯ-ŋgɤrmoz}{}{ⓔɯ-ŋgɤrmoz} 
\classe{np} 
\begin{définition}\pfra{dérangement}\end{définition}
\begin{définition}\pcmn{打搅,麻烦人}\end{définition}
\begin{exemple}\pjya{aʑo nɤʑo mɤ-ɣi-a ma nɤ-ŋgɤrmoz sɤβze-a}\hspace{5pt}\pcmn{我不来你家,我会打搅你的}\end{exemple}\end{entrée}

\begin{entrée}{ɯ-ŋgu,thon}{}{ⓔɯ-ŋgu,thon} 
\classe{np}
\classe{vs} \paradigme{dir}{thɯ-}
\begin{définition}\pfra{avoir une bonne situation, être aisé}\end{définition}
\begin{définition}\pcmn{家境好;富裕}\end{définition}
\begin{exemple}\pjya{ɯ-ŋgu mɯ́j-thon}\hspace{5pt}\pcmn{他什么都没有}\end{exemple}
\begin{exemple}\pjya{jiɕqha nɯ ɯ-ŋgu thon, mɤɕi}\hspace{5pt}\pcmn{他很有钱,很富有}\end{exemple}
\begin{exemple}\pjya{jiʑo ɕaŋtaʁ tɯ-ŋgu mɤ-kɯ-thon me}\hspace{5pt}\pcmn{没有比我们穷的人了}\end{exemple}\relationsémantique{Component 1}{\lien{}{ɯ-ŋgu}}\relationsémantique{Component 2}{\lien{}{thon}}\étymologie{mgo.tʰon}\end{entrée}

\begin{entrée}{ɯ-ŋgɯ}{}{ⓔɯ-ŋgɯ} 
\classe{n} 
\begin{définition}\pfra{dedans}\end{définition}
\begin{définition}\pcmn{里面}\end{définition}
\begin{exemple}\pjya{a-ŋga ɯ-ŋgɯɕɯŋgɯ nɯ ɯ-poloʁ me}\hspace{5pt}\pcmn{我最里层的衣服没有袖子}\end{exemple}\end{entrée}

\begin{entrée}{ɯ-ŋgɯmɤpɕi}{}{ⓔɯ-ŋgɯmɤpɕi} 
\classe{np} 
\begin{définition}\pfra{l'intérieur et l'extérieur}\end{définition}
\begin{définition}\pcmn{里里外外}\end{définition}\relationsémantique{参考}{\lien{ⓔɯ-pɕi}{ɯ-pɕi}}\end{entrée}

\begin{entrée}{ɯ-ŋgɯsni}{}{ⓔɯ-ŋgɯsni} 
\classe{np} 
\begin{définition}\pfra{au cœur même de}\end{définition}
\begin{définition}\pcmn{最里面}\end{définition}\relationsémantique{参考}{\lien{ⓔtɯ-sni}{tɯ-sni}}\end{entrée}

\begin{entrée}{ɯ-ɴqra}{}{ⓔɯ-ɴqra} 
\classe{np} 
\begin{définition}\pfra{délabré}\end{définition}
\begin{définition}\pcmn{破烂}\end{définition}\relationsémantique{参考}{\lien{ⓔkhɤɴqra}{khɤɴqra}}\relationsémantique{参考}{\lien{ⓔrɤɴqra}{rɤɴqra}}\end{entrée}

\begin{entrée}{ɯ-pa,ɕe}{}{ⓔɯ-pa,ɕe} 
\classe{np}
\classe{vi}
\classe{np}
\classe{vt} \paradigme{dir}{nɯ-}\relationsémantique{Component 1}{\lien{ⓔpaⓗ3ⓝɯ-pa}{ɯ-pa}}\relationsémantique{Component 2}{\lien{ⓔɕe}{ɕe}}\paradigme{dir}{nɯ-}
\begin{définition}\pfra{être accaparé par}\end{définition}
\begin{définition}\pcmn{被……拿去自己用}\end{définition}
\begin{définition}\pfra{s'accaparer des objets qui appartiennent à d'autres}\end{définition}
\begin{définition}\pcmn{归为私用,拿去自己用;占有}\end{définition}
\begin{exemple}\pjya{iʑora ji-kɤndzɤtshi ɯ-ro pɯ-dɤn ri, ɯʑo ɯ-pa ɲɤ-nɯ-ɕe}\hspace{5pt}\pcmn{我们在一起玩的时候没有吃完的食物都被他占有了}\end{exemple}\relationsémantique{Component 1}{\lien{ⓔpaⓗ3ⓝɯ-pa}{ɯ-pa}}\relationsémantique{Component 2}{\lien{ⓔsɯxɕe}{sɯxɕe}}
\begin{sous-entrée}{ɯ-pa,sɯxɕe}{ⓔɯ-pa,ɕeⓝɯ-pa,sɯxɕe}\end{sous-entrée}

\begin{exemple}\pjya{nɤj nɤ-pa ma-nɯ-tɯ-nɯ-sɯxɕe}\hspace{5pt}\pcmn{你不要拿去自己用}\end{exemple}\end{entrée}

\begin{entrée}{ɯ-palɤjlɯz}{}{ⓔɯ-palɤjlɯz} 
\classe{n} 
\begin{définition}\pfra{méthode, façon}\end{définition}
\begin{définition}\pcmn{办法;措施}\end{définition}\end{entrée}

\begin{entrée}{ɯ-pɤl}{}{ⓔɯ-pɤl} 
\classe{np} \sens{1}
\begin{définition}\pfra{paume}\end{définition}
\begin{définition}\pcmn{(手、脚)掌}\end{définition}
\begin{exemple}\pjya{tɯ-jaʁ ɯ-pɤl}\hspace{5pt}\pcmn{手掌}\end{exemple}
\begin{exemple}\pjya{tɯ-mi ɯ-pɤl}\hspace{5pt}\pcmn{脚掌}\end{exemple}\sens{2}
\begin{définition}\pfra{partie de la louche qui sert à contenir le liquide}\end{définition}
\begin{définition}\pcmn{勺子容水的部分}\end{définition}\end{entrée}

\begin{entrée}{ɯ-pɤrthɤβ}{}{ⓔɯ-pɤrthɤβ} 
\classe{np} 
\begin{définition}\pfra{entre}\end{définition}
\begin{définition}\pcmn{两个之间}\end{définition}
\begin{exemple}\pjya{ndʑi-pɤrthɤβ}\hspace{5pt}\pcmn{在他们俩之间}\end{exemple}\relationsémantique{参考}{\lien{ⓔɯ-thɤβ}{ɯ-thɤβ}}\étymologie{bar}\end{entrée}

\begin{entrée}{ɯ-pɕi}{}{ⓔɯ-pɕi} 
\classe{n} 
\begin{définition}\pfra{dehors}\end{définition}
\begin{définition}\pcmn{外面}\end{définition}
\begin{exemple}\pjya{ɯ-pɕi qale ɣɤʑu wo}\hspace{5pt}\pcmn{外面有风}\end{exemple}\relationsémantique{参考}{\lien{ⓔmɤpɕi}{mɤpɕi}}\étymologie{pʰʲi}\end{entrée}

\begin{entrée}{ɯ-phe}{}{ⓔɯ-phe} 
\classe{postp} 
\begin{définition}\pfra{datif}\end{définition}
\begin{définition}\pcmn{与格}\end{définition}\relationsémantique{同义词}{\lien{ⓔɯ-ɕki}{ɯ-ɕki}}\end{entrée}

\begin{entrée}{ɯ-phɯ}{}{ⓔɯ-phɯ} 
\classe{np} 
\begin{définition}\pfra{prix}\end{définition}
\begin{définition}\pcmn{价钱}\end{définition}\relationsémantique{参考}{\lien{ⓔrɤphɯ}{rɤphɯ}}\relationsémantique{参考}{\lien{ⓔnɯphɯ}{nɯphɯ}}\end{entrée}

\begin{entrée}{ɯ-phɯɣ}{}{ⓔɯ-phɯɣ} 
\classe{np} 
\begin{définition}\pfra{source (fleuve)}\end{définition}
\begin{définition}\pcmn{水源}\end{définition}\relationsémantique{参考}{\lien{ⓔtɕhɯphɯɣ}{tɕhɯphɯɣ}}\étymologie{pʰugs}\end{entrée}

\begin{entrée}{ɯ-phɯl}{}{ⓔɯ-phɯl} 
\classe{np} 
\begin{définition}\pfra{serti, incrusté de}\end{définition}
\begin{définition}\pcmn{镶着}\end{définition}
\begin{exemple}\pjya{mbrɯtɕɯ ɯ-phɯl rɯnbotɕhi kɤ-rku}\hspace{5pt}\pcmn{镶着宝石的刀}\end{exemple}\end{entrée}

\begin{entrée}{ɯ-phɯŋgɯ}{}{ⓔɯ-phɯŋgɯ} 
\classe{np} 
\begin{définition}\pfra{giron}\end{définition}
\begin{définition}\pcmn{怀里}\end{définition}\end{entrée}

\begin{entrée}{ɯ-phɯphi}{}{ⓔɯ-phɯphi} 
\classe{np} 
\begin{définition}\pfra{vagin (enfant)}\end{définition}
\begin{définition}\pcmn{(小孩子的)阴道}\end{définition}\end{entrée}

\begin{entrée}{ɯ-phɯphɯ}{}{ⓔɯ-phɯphɯ} 
\classe{np} 
\begin{définition}\pfra{ce que l'on mendie}\end{définition}
\begin{définition}\pcmn{乞讨的(东西、钱)}\end{définition}
\begin{exemple}\pjya{kɤ-nɤjɤm tɕe nɤ-phɯphɯ ju-ɣɯt-a}\hspace{5pt}\pcmn{你在这里等着,给你送东西来(对乞丐说的话)}\end{exemple}\relationsémantique{参考}{\lien{ⓔnɤphɯphɯ}{nɤphɯphɯ}}\end{entrée}

\begin{entrée}{ɯ-punaŋtɕa}{}{ⓔɯ-punaŋtɕa} 
\classe{np} 
\begin{définition}\pfra{organes internes}\end{définition}
\begin{définition}\pcmn{内脏}\end{définition}\end{entrée}

\begin{entrée}{ɯ-qhu}{}{ⓔɯ-qhu} 
\classe{np}
\classe{np}
\classe{vt} \paradigme{dir}{tɤ-}\paradigme{dir}{thɯ-}
\begin{définition}\pfra{arrière}\end{définition}
\begin{définition}\pcmn{后面}\end{définition}
\begin{définition}\pfra{défendre, soutenir, prendre le parti de}\end{définition}
\begin{définition}\pcmn{维护;为……做主}\end{définition}
\begin{exemple}\pjya{nɤʑo nɯ-tɯ-ɤnɯri ɯ-qhu aʑo ki pɯ-rat-a}\hspace{5pt}\pcmn{你回去了以后我就写了这些}\end{exemple}
\begin{exemple}\pjya{nɯ ɯ-qhu nɯ tɕe kɯmaʁ kɯrɯχpi ci rat-a ŋu}\hspace{5pt}\pcmn{下一次,我再写一个藏语故事}\end{exemple}
\begin{exemple}\pjya{tɯtʂaŋ kɤ-βzu ra ma, tɯ-pɕoʁ ɯ-qhu kɤ-βzu mɤ-khɯ}\hspace{5pt}\pcmn{要公平,不要维护一方}\end{exemple}
\begin{exemple}\pjya{jiʑora ji-qhu thɯ-βze ra ma mɤ-jɤɣ nɤ!}\hspace{5pt}\pcmn{一定要为我们做主!}\end{exemple}
\begin{exemple}\pjya{aʑo a-qhu tɤ-βze ra (thɯ-βze ra) = aʑo pjɯ-kɯ-zɣɤŋgi-a ra}\hspace{5pt}\pcmn{你要为我做主!}\end{exemple}\relationsémantique{参考}{\lien{ⓔmaqhu}{maqhu}}\relationsémantique{参考}{\lien{ⓔɯ-qhɤchu}{ɯ-qhɤchu}}\relationsémantique{参考}{\lien{ⓔqhaqhu}{qhaqhu}}\relationsémantique{参考}{\lien{ⓔtʂaqhu}{tʂaqhu}}\relationsémantique{参考}{\lien{ⓔqharu}{qharu}}\relationsémantique{参考}{\lien{ⓔtɕhɯqhu}{tɕhɯqhu}}\relationsémantique{同义词}{\lien{ⓔɣɤŋgiⓝzɣɤŋgi}{zɣɤŋgi}}\relationsémantique{Component 1}{\lien{ⓔɯ-qhu}{ɯ-qhu}}\relationsémantique{Component 2}{\lien{}{βzu}}
\begin{sous-entrée}{ɯ-qhu,βzu}{ⓔɯ-qhuⓝɯ-qhu,βzu}\end{sous-entrée}

\end{entrée}

\begin{entrée}{ɯ-qhɤchu}{}{ⓔɯ-qhɤchu} 
\classe{adv} 
\begin{définition}\pfra{arrière}\end{définition}
\begin{définition}\pcmn{背面,后面}\end{définition}
\begin{exemple}\pjya{nɤki tɯrme ɣɯ ɯ-qhɤchu nɯtɕu laχtɕha ata}\hspace{5pt}\pcmn{那个人的后面有个东西}\end{exemple}\relationsémantique{参考}{\lien{ⓔɯ-qhu}{ɯ-qhu}}\end{entrée}

\begin{entrée}{ɯ-qhoʁ}{}{ⓔɯ-qhoʁ} 
\classe{np} 
\begin{définition}\pfra{largeur des habits (tronc)}\end{définition}
\begin{définition}\pcmn{衣服的宽度(胸膛和肚子)}\end{définition}
\begin{exemple}\pjya{a-ŋga ɯ-qhoʁ ɲɯ-ŋgɤr tɕe, kɤ-ŋga mɯ́j-khɯ}\hspace{5pt}\pcmn{因为我那件衣服太小,穿不下}\end{exemple}\end{entrée}

\begin{entrée}{ɯ-qiɯ}{}{ⓔɯ-qiɯ} 
\classe{np} 
\begin{définition}\pfra{moitié}\end{définition}
\begin{définition}\pcmn{一半}\end{définition}
\begin{exemple}\pjya{ɯ-qiɯ pɯ-mtsham-a}\hspace{5pt}\pcmn{我听了一半}\end{exemple}\end{entrée}

\begin{entrée}{ɯ-qoʁ}{}{ⓔɯ-qoʁ} 
\classe{np} 
\begin{définition}\pfra{un an (enfant)}\end{définition}
\begin{définition}\pcmn{周岁(孩子)}\end{définition}
\begin{exemple}\pjya{ɯ-qoʁ jɤ-azɣɯt ɯ́-ŋu?}\hspace{5pt}\pcmn{(你儿子)满周岁了吗?}\end{exemple}\end{entrée}

\begin{entrée}{ɯ-ru}{}{ⓔɯ-ru} 
\classe{np} \paradigme{comit}{kɤ́rɯru}\paradigme{comit}{kɤɣɯrɯru}
\begin{définition}\pfra{tige}\end{définition}
\begin{définition}\pcmn{杆子}\end{définition}\relationsémantique{参考}{\lien{ⓔaɣɯrɯruⓗ2}{aɣɯrɯru₂}}\end{entrée}

\begin{entrée}{ɯ-raŋ}{}{ⓔɯ-raŋ} 
\classe{np} 
\begin{définition}\pfra{génération, au moment de}\end{définition}
\begin{définition}\pcmn{年代,正当那个时候}\end{définition}\étymologie{riŋ}\end{entrée}

\begin{entrée}{ɯ-rɤɣ}{}{ⓔɯ-rɤɣ} 
\classe{np} 
\begin{définition}\pfra{au moment prévu, au même moment}\end{définition}
\begin{définition}\pcmn{在预定的时间}\end{définition}
\begin{exemple}\pjya{stonka ɯ-rɤɣ ja-zɣɯt tɕe tɤ-rɤku chɯ-mda ɕti}\hspace{5pt}\pcmn{秋天到了,庄稼就会成熟}\end{exemple}\relationsémantique{参考}{\lien{ⓔarɤrɤɣ}{arɤrɤɣ}}\end{entrée}

\begin{entrée}{ɯ-rcharchɤβ}{}{ⓔɯ-rcharchɤβ} 
\classe{np} 
\begin{définition}\pfra{interstice}\end{définition}
\begin{définition}\pcmn{缝隙;之间}\end{définition}\relationsémantique{参考}{\lien{ⓔɯ-rchɤβ}{ɯ-rchɤβ}}\end{entrée}

\begin{entrée}{ɯ-rchɤβ}{}{ⓔɯ-rchɤβ} 
\classe{np} \sens{1}
\begin{définition}\pfra{interstice}\end{définition}
\begin{définition}\pcmn{缝隙}\end{définition}\sens{2}
\begin{définition}\pfra{milieu}\end{définition}
\begin{définition}\pcmn{之间}\end{définition}\relationsémantique{参考}{\lien{ⓔɯ-rcharchɤβ}{ɯ-rcharchɤβ}}\end{entrée}

\begin{entrée}{ɯ-rɕa,mŋɤm}{}{ⓔɯ-rɕa,mŋɤm} 
\classe{np}
\classe{vs} 
\begin{définition}\pfra{chérir}\end{définition}
\begin{définition}\pcmn{疼爱}\end{définition}
\begin{exemple}\pjya{a-ɣe a-rɕa mŋɤm}\hspace{5pt}\pcmn{我疼爱我的孙子}\end{exemple}\relationsémantique{参考}{\lien{ⓔnɤrɕɤmŋɤm}{nɤrɕɤmŋɤm}}\relationsémantique{Component 1}{\lien{}{ɯ-rɕa}}\relationsémantique{Component 2}{\lien{ⓔmŋɤm}{mŋɤm}}\end{entrée}

\begin{entrée}{ɯ-rɕa,tsha}{}{ⓔɯ-rɕa,tsha} 
\classe{vi}
\classe{np}
\classe{vs} \paradigme{dir}{kɤ-}
\begin{définition}\pfra{être délicat et prévenant}\end{définition}
\begin{définition}\pcmn{体贴}\end{définition}
\begin{exemple}\pjya{nɤʑo nɤ-rɕa wuma ɲɯ-tsha}\hspace{5pt}\pcmn{你很体贴人的}\end{exemple}\relationsémantique{Component 1}{\lien{}{ɯ-rɕa}}\relationsémantique{Component 2}{\lien{ⓔtsha}{tsha}}
\begin{sous-entrée}{ɯ-rɕa,ɣɤtsha}{ⓔɯ-rɕa,tshaⓝɯ-rɕa,ɣɤtsha} 
\classe{np}
\classe{vt} 
\begin{exemple}\pjya{nɤ-taʁ a-rɕa tu-ɣɤtshe-a ra}\hspace{5pt}\pcmn{我要对你体贴一点}\end{exemple}
\begin{exemple}\pjya{nɤ-mu nɤ-wa ndʑɪ-ɕki nɤ-rɕa tɤ-ɣɤtshe ra}\hspace{5pt}\pcmn{你要孝顺你父母}\end{exemple}\relationsémantique{Component 1}{\lien{}{ɯ-rɕa}}\relationsémantique{Component 2}{\lien{ⓔɣɤtsha}{ɣɤtsha}}\end{sous-entrée}

\end{entrée}

\begin{entrée}{ɯ-rɕa,χtɤt}{}{ⓔɯ-rɕa,χtɤt} 
\classe{vt}
\classe{np}
\classe{vt} \paradigme{dir}{kɤ-}
\begin{définition}\pfra{se concentrer}\end{définition}
\begin{définition}\pcmn{专心,集中}\end{définition}
\begin{exemple}\pjya{nɤ-rɕa kɤ-χtɤt ɲɯ-ra}\hspace{5pt}\pcmn{你要专心一点}\end{exemple}
\begin{exemple}\pjya{nɯ-rɕa kɤ-χtɤt mɤ-cha-nɯ tɕe kɯmɤlɤxso ɕti}\hspace{5pt}\pcmn{他们不能专心学习,就白费了}\end{exemple}\relationsémantique{Component 1}{\lien{}{ɯ-rɕa}}\relationsémantique{Component 2}{\lien{ⓔχtɤt}{χtɤt}}\end{entrée}

\begin{entrée}{ɯ-rdoʁ}{}{ⓔɯ-rdoʁ} 
\classe{np} 
\begin{définition}\pfra{nourriture pour animaux}\end{définition}
\begin{définition}\pcmn{整体的牲畜的粮食}\end{définition}\relationsémantique{参考}{\lien{ⓔtɯ-rdoʁ}{tɯ-rdoʁ}}\end{entrée}

\begin{entrée}{ɯ-rgu,sɯ}{}{ⓔɯ-rgu,sɯ} 
\classe{np}
\classe{np}
\classe{vs} 
\begin{définition}\pfra{robuste, fort (malgré les apparences)}\end{définition}
\begin{définition}\pcmn{有(出乎意料的)能力,体力}\end{définition}
\begin{exemple}\pjya{nɤ-rgu (ɯ-tɯ-sɯ) nɯ!}\hspace{5pt}\pcmn{(没有想到)你办得到}\end{exemple}
\begin{exemple}\pjya{tɤ-pɤtso ɯ-rgu ɲɯ-sɯ tɕe, ɲɯ-rkaŋ}\hspace{5pt}\pcmn{那个小孩子很有能力,很壮(不要小看他)}\end{exemple}\relationsémantique{Component 1}{\lien{}{ɯ-rgu}}\relationsémantique{Component 2}{\lien{ⓔsɯ}{sɯ}}\end{entrée}

\begin{entrée}{ɯ-rɟa}{}{ⓔɯ-rɟa} 
\classe{n} 
\begin{définition}\pfra{injure}\end{définition}
\begin{définition}\pcmn{咒人的话}\end{définition}\relationsémantique{同义词}{\lien{ⓔkhɤrma}{khɤrma}}\relationsémantique{参考}{\lien{ⓔrɯrɟa}{rɯrɟa}}\end{entrée}

\begin{entrée}{ɯ-rɟɤŋgo}{}{ⓔɯ-rɟɤŋgo} 
\classe{np} \sens{1}
\begin{définition}\pfra{douleur irradiante}\end{définition}
\begin{définition}\pcmn{放射痛}\end{définition}\sens{2}
\begin{définition}\pfra{complication (maladie)}\end{définition}
\begin{définition}\pcmn{并发症}\end{définition}\end{entrée}

\begin{entrée}{ɯ-rka,ŋɤn}{}{ⓔɯ-rka,ŋɤn} 
\classe{np}
\classe{vs}
\classe{np}
\classe{vt} 
\begin{définition}\pfra{être ingrat, mauvais}\end{définition}
\begin{définition}\pcmn{心底不好,不怀好意}\end{définition}
\begin{exemple}\pjya{nɤ-rka ɯ-tɯ-ŋɤn}\hspace{5pt}\pcmn{你心底不好}\end{exemple}
\begin{exemple}\pjya{ɯ-rka ɲɯ-ŋɤn ma ɯ-taʁ wuma ʑo pɯ-pe-a ɕti ri, tham tɕe ɯʑo kɯ mɯ́j-wɣ-nɯkon-a}\hspace{5pt}\pcmn{他心底不好,我原来对他很好,现在他却不理我}\end{exemple}\relationsémantique{Component 1}{\lien{}{ɯ-rka}}\relationsémantique{Component 2}{\lien{ⓔŋɤn}{ŋɤn}}\relationsémantique{Component 1}{\lien{}{ɯ-rka}}\relationsémantique{Component 2}{\lien{ⓔɣɤŋɤn}{ɣɤŋɤn}}
\begin{sous-entrée}{ɯ-rka,ɣɤŋɤn}{ⓔɯ-rka,ŋɤnⓝɯ-rka,ɣɤŋɤn}
\begin{exemple}\pjya{nɤ-rka ma-tɯ-ɣɤŋɤn}\hspace{5pt}\pcmn{不要起坏心}\end{exemple}\end{sous-entrée}

\end{entrée}

\begin{entrée}{ɯ-rkarkɯ}{}{ⓔɯ-rkarkɯ} 
\classe{np}  
\grammaire{n.rdpl} 
\begin{définition}\pfra{environs}\end{définition}
\begin{définition}\pcmn{边缘}\end{définition}\relationsémantique{同义词}{\lien{ⓔɯ-zarzɯr}{ɯ-zarzɯr}}\relationsémantique{参考}{\lien{ⓔɯ-rkɯ}{ɯ-rkɯ}}\end{entrée}

\begin{entrée}{ɯrkoz}{}{ⓔɯrkoz} 
\classe{adv} 
\begin{définition}\pfra{spécialement}\end{définition}
\begin{définition}\pcmn{专门}\end{définition}\end{entrée}

\begin{entrée}{ɯ-rkɯ}{}{ⓔɯ-rkɯ} 
\classe{np} 
\begin{définition}\pfra{côté}\end{définition}
\begin{définition}\pcmn{旁边;角落}\end{définition}\relationsémantique{参考}{\lien{ⓔɯ-rkarkɯ}{ɯ-rkarkɯ}}
\begin{sous-entrée}{ɯ-rkɯɕɯrkɯ}{ⓔɯ-rkɯⓝɯ-rkɯɕɯrkɯ}
\begin{définition}\pfra{le côté le plus au bord}\end{définition}
\begin{définition}\pcmn{最边缘}\end{définition}
\begin{exemple}\pjya{zɣɤmbu nɯ kha ɯ-rkɯɕɯrkɯ nɯtɕu ɲɯ́-wɣ-ta ra}\hspace{5pt}\pcmn{扫把要放在房子的最边缘}\end{exemple}\end{sous-entrée}

\end{entrée}

\begin{entrée}{ɯ-rkɯm}{}{ⓔɯ-rkɯm} 
\classe{np} 
\begin{définition}\pfra{cotylédon}\end{définition}
\begin{définition}\pcmn{子叶}\end{définition}
\begin{exemple}\pjya{lɤpɯɣ ɯ-rkɯm}\hspace{5pt}\pcmn{萝卜的子叶}\end{exemple}\relationsémantique{参考}{\lien{ⓔkarkɯm}{karkɯm}}\end{entrée}

\begin{entrée}{ɯ-rma}{}{ⓔɯ-rma} 
\classe{np} 
\begin{définition}\pfra{ferment}\end{définition}
\begin{définition}\pcmn{酵母;曲子}\end{définition}\end{entrée}

\begin{entrée}{ɯ-rmɯrɟa}{}{ⓔɯ-rmɯrɟa} 
\classe{np} 
\begin{définition}\pfra{sobriquet, surnom}\end{définition}
\begin{définition}\pcmn{外号(贬义)}\end{définition}\relationsémantique{参考}{\lien{ⓔɯ-rɟa}{ɯ-rɟa}}\relationsémantique{参考}{\lien{ⓔtɤ-rmi}{tɤ-rmi}}\end{entrée}

\begin{entrée}{ɯ-rnɤɣmbaj}{}{ⓔɯ-rnɤɣmbaj} 
\classe{np} 
\begin{définition}\pfra{côté de l'oreille}\end{définition}
\begin{définition}\pcmn{耳边}\end{définition}\relationsémantique{参考}{\lien{ⓔtɯ-rna}{tɯ-rna}}\end{entrée}

\begin{entrée}{ɯ-rnɤqhu}{}{ⓔɯ-rnɤqhu} 
\classe{np} 
\begin{définition}\pfra{derrière les oreilles}\end{définition}
\begin{définition}\pcmn{耳朵后面}\end{définition}\end{entrée}

\begin{entrée}{ɯ-rozre}{}{ⓔɯ-rozre} 
\classe{np} 
\begin{définition}\pfra{reste}\end{définition}
\begin{définition}\pcmn{剩余的;余留的残渣}\end{définition}
\begin{exemple}\pjya{tɯ-ŋga ɯ-rozre nɯra jɤ-tsɯm}\hspace{5pt}\pcmn{你把剩下的衣服带走}\end{exemple}\relationsémantique{参考}{\lien{ⓔtɤ-ro}{tɤ-ro}}\end{entrée}

\begin{entrée}{ɯ-rqɯ}{}{ⓔɯ-rqɯ} 
\classe{np} 
\begin{définition}\pfra{objet froid}\end{définition}
\begin{définition}\pcmn{冷的东西}\end{définition}\relationsémantique{参考}{\lien{ⓔtɯcɯrqɯ}{tɯcɯrqɯ}}\end{entrée}

\begin{entrée}{ɯ-rtɤβ}{}{ⓔɯ-rtɤβ} 
\classe{np} 
\begin{définition}\pfra{lanière ornée}\end{définition}
\begin{définition}\pcmn{带有装饰的带子}\end{définition}\end{entrée}

\begin{entrée}{ɯ-rti}{}{ⓔɯ-rti} 
\classe{np} 
\begin{définition}\pfra{embryon de poulain}\end{définition}
\begin{définition}\pcmn{马的胚胎}\end{définition}
\begin{exemple}\pjya{rgonma ɯ-rti kɯ-mbro}\hspace{5pt}\pcmn{快要生的母马}\end{exemple}
\begin{exemple}\pjya{mbro ɯ-rti kɯ-tu}\hspace{5pt}\pcmn{怀孕的母马}\end{exemple}\relationsémantique{参考}{\lien{ⓔrɤrti}{rɤrti}}\relationsémantique{参考}{\lien{ⓔtɯ-rti}{tɯ-rti}}\étymologie{rteɦu}\end{entrée}

\begin{entrée}{ɯ-rtsa,tɕɤt}{}{ⓔɯ-rtsa,tɕɤt} 
\classe{np}
\classe{vt} \paradigme{dir}{nɯ-}
\begin{définition}\pfra{rechercher la cause de}\end{définition}
\begin{définition}\pcmn{追究}\end{définition}
\begin{exemple}\pjya{ɯ-rtsa ɲɤ-tɕɤt}\hspace{5pt}\pcmn{他追究了}\end{exemple}\relationsémantique{参考}{\lien{ⓔnɯrtsa}{nɯrtsa}}\relationsémantique{Component 1}{\lien{}{ɯ-rtsa}}\relationsémantique{Component 2}{\lien{ⓔtɕɤt}{tɕɤt}}\end{entrée}

\begin{entrée}{ɯ-rtshɯ}{}{ⓔɯ-rtshɯ} 
\classe{np} 
\begin{définition}\pfra{écorce de légumineuse (pour nourrir les bovidés)}\end{définition}
\begin{définition}\pcmn{豆类的粗糠秕,喂牛}\end{définition}
\begin{exemple}\pjya{stoʁ rtshɯ}\hspace{5pt}\pcmn{胡豆的粗糠秕}\end{exemple}\end{entrée}

\begin{entrée}{ɯ-rtshɯm}{}{ⓔɯ-rtshɯm} 
\classe{np} 
\begin{définition}\pfra{section}\end{définition}
\begin{définition}\pcmn{一段(不完整)}\end{définition}\end{entrée}

\begin{entrée}{ɯ-rtsi}{}{ⓔɯ-rtsi} 
\classe{np} 
\begin{définition}\pfra{laque}\end{définition}
\begin{définition}\pcmn{漆;油}\end{définition}
\begin{exemple}\pjya{ɯ-rtsi chɤ-lɤt}\hspace{5pt}\pcmn{他上了漆}\end{exemple}
\begin{exemple}\pjya{rgɯnba ɯ-rtsi to-lɤt}\hspace{5pt}\pcmn{在庙里上了漆}\end{exemple}
\begin{exemple}\pjya{laχtɕha ɯ-rtsi to-lɤt}\hspace{5pt}\pcmn{他给家具上了漆}\end{exemple}\relationsémantique{参考}{\lien{ⓔsɯrtsi}{sɯrtsi}}\étymologie{rtsi}\end{entrée}

\begin{entrée}{ɯ-rtɯrtɤβ}{}{ⓔɯ-rtɯrtɤβ} 
\classe{np} 
\begin{définition}\pfra{personne collante}\end{définition}
\begin{définition}\pcmn{缠着别人不放}\end{définition}\relationsémantique{参考}{\lien{ⓔrtɤβ}{rtɤβ}}\end{entrée}

\begin{entrée}{ɯ-rɯɣ}{}{ⓔɯ-rɯɣ} 
\classe{np} 
\begin{définition}\pfra{nationalité, race}\end{définition}
\begin{définition}\pcmn{族}\end{définition}\étymologie{rigs}\end{entrée}

\begin{entrée}{ɯrɯruz}{}{ⓔɯrɯruz} 
\classe{adv} 
\begin{définition}\pfra{à ce moment}\end{définition}
\begin{définition}\pcmn{当时;眼前}\end{définition}
\begin{exemple}\pjya{ɯʑo tɤ-ngo tɕe ɯrɯruz nɯ wuma pɯ-sɤɣʑɯr, kɯ-maqhu ʁo tɕe to-mna.}\hspace{5pt}\pcmn{他生病的时候,当时有生命危险,最后还是痊愈了}\end{exemple}
\begin{exemple}\pjya{ɯrɯruz ɣɯ ɯ-ndaŋ ma ɯ-qhu ɣɯ ɯ-ndaŋ kɤ-lɤt mɯ́j-spe}\hspace{5pt}\pcmn{他只会考虑眼前的事,不会考虑后果}\end{exemple}\end{entrée}

\begin{entrée}{ɯ-rɯz}{}{ⓔɯ-rɯz} 
\classe{np} \sens{1}
\begin{définition}\pfra{sorte, espèce}\end{définition}
\begin{définition}\pcmn{类别;品种}\end{définition}
\begin{exemple}\pjya{nɤki rɤjndoʁ ɯ-rɯz ɲɯ-ŋu}\hspace{5pt}\pcmn{这是大头菜的一种}\end{exemple}\sens{2}
\begin{définition}\pfra{tour (travail)}\end{définition}
\begin{définition}\pcmn{轮到自己(办事)}\end{définition}
\begin{exemple}\pjya{jisŋi kɤ-rɤma a-rɯz ŋu}\hspace{5pt}\pcmn{今天轮到我上班}\end{exemple}
\begin{exemple}\pjya{a-rɯz jɤ-azɣɯt}\hspace{5pt}\pcmn{轮到我了}\end{exemple}\relationsémantique{同义词}{\lien{}{ɯ-βra}}\sens{3}
\begin{définition}\pfra{clairvoyance}\end{définition}
\begin{définition}\pcmn{预见}\end{définition}\étymologie{rigs}\end{entrée}

\begin{entrée}{ɯ-rwarwa}{}{ⓔɯ-rwarwa} 
\classe{np} 
\begin{définition}\pfra{crête}\end{définition}
\begin{définition}\pcmn{鸡冠}\end{définition}\end{entrée}

\begin{entrée}{ɯ-ʁɤri}{}{ⓔɯ-ʁɤri} 
\classe{np} 
\begin{définition}\pfra{avant}\end{définition}
\begin{définition}\pcmn{前面}\end{définition}\relationsémantique{参考}{\lien{ⓔastuⓝɯ-ʁɤri,astu}{ɯ-ʁɤri,astu}}\end{entrée}

\begin{entrée}{ɯ-ʁdɤz}{}{ⓔɯ-ʁdɤz} 
\classe{np} 
\begin{définition}\pfra{charge, souci}\end{définition}
\begin{définition}\pcmn{负担}\end{définition}
\begin{exemple}\pjya{ndʑi-ʁdɤz tɕɤt-tɕi}\hspace{5pt}\pcmn{我们加重你们俩的负担}\end{exemple}\relationsémantique{参考}{\lien{ⓔnaʁdɤz}{naʁdɤz}}\end{entrée}

\begin{entrée}{ɯ-ʁjoʁ}{}{ⓔɯ-ʁjoʁ} 
\classe{np} 
\begin{définition}\pfra{partie extérieure des vêtements}\end{définition}
\begin{définition}\pcmn{衣服的外层}\end{définition}
\begin{exemple}\pjya{tɯ-ŋga ɯ-ʁjoʁ}\hspace{5pt}\pcmn{衣服的外层}\end{exemple}\relationsémantique{反义词}{\lien{ⓔnaŋɕa}{naŋɕa}}\relationsémantique{参考}{\lien{ⓔsɯʁjoʁ}{sɯʁjoʁ}}\end{entrée}

\begin{entrée}{ɯ-ʁɟoʁɟe}{}{ⓔɯ-ʁɟoʁɟe} 
\classe{np} 
\begin{définition}\pfra{vin ou lait dilué dans l'eau}\end{définition}
\begin{définition}\pcmn{掺了水的酒或者牛奶}\end{définition}\relationsémantique{参考}{\lien{ⓔʁɟo}{ʁɟo}}\end{entrée}

\begin{entrée}{ɯ-ʁlu}{}{ⓔɯ-ʁlu} 
\classe{np} 
\begin{définition}\pfra{endroit concave}\end{définition}
\begin{définition}\pcmn{凹下去的地形}\end{définition}\relationsémantique{参考}{\lien{ⓔaʁloʁlu}{aʁloʁlu}}\end{entrée}

\begin{entrée}{ɯ-ʁlɤt}{}{ⓔɯ-ʁlɤt} 
\classe{np} 
\begin{définition}\pfra{canon}\end{définition}
\begin{définition}\pcmn{枪杆【枪肚子】}\end{définition}
\begin{exemple}\pjya{ɯ-ʁlɤt nɯ ɕɤmɯɣdɯ mɯzi sɤ-rkɯ ŋu, ɕɤmɯɣdɯ sna mɤ-sna nɯ ʁlɤt rɤmdzɯt}\hspace{5pt}\pcmn{枪肚子是用来装火药的洞,枪肚子决定枪的好与坏。}\end{exemple}\end{entrée}

\begin{entrée}{ɯ-ʁle}{}{ⓔɯ-ʁle} 
\classe{np} 
\begin{définition}\pfra{réputation}\end{définition}
\begin{définition}\pcmn{名声,面子}\end{définition}
\begin{exemple}\pjya{ɯ-ʁle ɣɤʑu}\hspace{5pt}\pcmn{他有好名声}\end{exemple}
\begin{exemple}\pjya{ɯ-ʁle ko-ru tɕe to-nɯŋgumtha}\hspace{5pt}\pcmn{他为了得到好名声就照顾他了}\end{exemple}
\begin{exemple}\pjya{nɤʑo ɯ-ʁle ma-kɤ-tɯ-ru kɯ koŋla tú-wɣ-sɤpe ra}\hspace{5pt}\pcmn{不要只顾名声,要干实际的事情}\end{exemple}\relationsémantique{参考}{\lien{ⓔraʁle}{raʁle}}\relationsémantique{参考}{\lien{ⓔqale}{qale}}\end{entrée}

\begin{entrée}{ɯ-ʁnawa}{}{ⓔɯ-ʁnawa} 
\classe{np} 
\begin{définition}\pfra{vacances}\end{définition}
\begin{définition}\pcmn{(请)假}\end{définition}
\begin{exemple}\pjya{ji-ŋgundʑɯɣ kɯ a-ʁnawa mɯ́j-nɤle}\hspace{5pt}\pcmn{我们领导不给我请假}\end{exemple}
\begin{exemple}\pjya{a-ʁnawa tɤ-thu-t-a}\hspace{5pt}\pcmn{我请了假}\end{exemple}\end{entrée}

\begin{entrée}{ɯ-ʁɲɤrŋa}{}{ⓔɯ-ʁɲɤrŋa} 
\classe{np} 
\begin{définition}\pfra{crosse}\end{définition}
\begin{définition}\pcmn{枪把}\end{définition}
\begin{exemple}\pjya{ɯ-ʁɲɤrŋa nɯ tɯ-rpaʁ ɯ-sɤ-χtɤt ŋu}\hspace{5pt}\pcmn{枪把是用来抵住肩膀的部件。}\end{exemple}\end{entrée}

\begin{entrée}{ɯ-ʁre}{}{ⓔɯ-ʁre} 
\classe{np} 
\begin{définition}\pfra{respect, prestige, authorité}\end{définition}
\begin{définition}\pcmn{威望}\end{définition}
\begin{exemple}\pjya{ɯ-ʁre ɣɤʑu (=ɲɯ-ɣɤʁre), ɯʑo ɯ-ʁre kɯ-tu ci ɲɯ-ŋu}\hspace{5pt}\pcmn{他是有威望的人}\end{exemple}\relationsémantique{参考}{\lien{ⓔsaʁre}{saʁre}}\relationsémantique{参考}{\lien{ⓔnaʁre}{naʁre}}\relationsémantique{参考}{\lien{ⓔɣɤʁre}{ɣɤʁre}}\end{entrée}

\begin{entrée}{ɯ-sɤɣjɤɣ}{}{ⓔɯ-sɤɣjɤɣ} 
\classe{np} 
\begin{définition}\pfra{fin}\end{définition}
\begin{définition}\pcmn{结尾}\end{définition}
\begin{exemple}\pjya{χpi ɯ-sɤɣjɤɣ}\hspace{5pt}\pcmn{故事的结尾}\end{exemple}\relationsémantique{参考}{\lien{ⓔjɤɣ}{jɤɣ}}\end{entrée}

\begin{entrée}{ɯ-sɤɣɬoʁ}{}{ⓔɯ-sɤɣɬoʁ} 
\classe{np} 
\begin{définition}\pfra{endroit où une plante pousse}\end{définition}
\begin{définition}\pcmn{生长的地方(植物、蘑菇)}\end{définition}\relationsémantique{参考}{\lien{ⓔɬoʁⓗ2}{ɬoʁ₂}}\end{entrée}

\begin{entrée}{ɯ-sɤpe}{}{ⓔɯ-sɤpe} 
\classe{np} 
\begin{définition}\pfra{avantage}\end{définition}
\begin{définition}\pcmn{好处}\end{définition}\relationsémantique{参考}{\lien{ⓔpe}{pe}}\end{entrée}

\begin{entrée}{ɯ-sɤʁjɯʁjit}{}{ⓔɯ-sɤʁjɯʁjit}\relationsémantique{参考}{\lien{ⓔʁjit}{ʁjit}}\end{entrée}

\begin{entrée}{ɯ-sɤsɤʑa}{}{ⓔɯ-sɤsɤʑa} 
\classe{np} 
\begin{définition}\pfra{début}\end{définition}
\begin{définition}\pcmn{开头}\end{définition}
\begin{exemple}\pjya{nɤ-kɤ-ti ɯ-sɤsɤʑa kɯ-tu maŋe}\hspace{5pt}\pcmn{你的说法无从说起(没有根据)}\end{exemple}\relationsémantique{参考}{\lien{}{ʑa}}\relationsémantique{参考}{\lien{ⓔsɤʑa}{sɤʑa}}\end{entrée}

\begin{entrée}{ɯ-sɤti}{}{ⓔɯ-sɤti} 
\classe{np} 
\begin{définition}\pfra{prétexte}\end{définition}
\begin{définition}\pcmn{借口}\end{définition}\relationsémantique{参考}{\lien{ⓔti}{ti}}\end{entrée}

\begin{entrée}{ɯ-scawa}{}{ⓔɯ-scawa} 
\classe{np} 
\begin{définition}\pfra{pauvre de ...}\end{définition}
\begin{définition}\pcmn{倒霉}\end{définition}
\begin{exemple}\pjya{nɤ-scawa ɲɯ-saχaʁ}\hspace{5pt}\pcmn{你很倒霉}\end{exemple}\étymologie{skʲo.ba}\end{entrée}

\begin{entrée}{ɯ-sci}{}{ⓔɯ-sci} 
\classe{np}
\classe{np}
\classe{vt} 
\begin{définition}\pfra{à la place de}\end{définition}
\begin{définition}\pcmn{替}\end{définition}
\begin{exemple}\pjya{nɤ-sci aʑo ju-ɕe-a}\hspace{5pt}\pcmn{我替你去}\end{exemple}
\begin{exemple}\pjya{aʑo a-ʁa maŋe tɕe, nɤʑo a-sci tu-tɯ-βze ɯ-tɯ-cha?}\hspace{5pt}\pcmn{我没有空,你可以代替我做吗?}\end{exemple}\relationsémantique{Component 1}{\lien{ⓔɯ-sci}{ɯ-sci}}\relationsémantique{Component 2}{\lien{}{βzu}}
\begin{sous-entrée}{ɯ-sci,βzu}{ⓔɯ-sciⓝɯ-sci,βzu}\end{sous-entrée}

\sens{1}
\begin{définition}\pfra{se venger de}\end{définition}
\begin{définition}\pcmn{报仇}\end{définition}\relationsémantique{同义词}{\lien{}{ɯ-rtsot,βzu}}\sens{2}
\begin{définition}\pfra{remplacer}\end{définition}
\begin{définition}\pcmn{代替}\end{définition}\relationsémantique{同义词}{\lien{}{ɯ-tshɤt,βzu}}\sens{3}
\begin{définition}\pfra{répondre}\end{définition}
\begin{définition}\pcmn{答复}\end{définition}
\begin{exemple}\pjya{ɯ-sci to-βzu (=ɯ-tshɤt to-βzu; ɯ-rtsot to-βzu; ɯ-lɤn to-βzu)}\hspace{5pt}\pcmn{他报复了他;他代替了他;他答复了他}\end{exemple}\relationsémantique{同义词}{\lien{}{ɯ-lɤn,βzu}}\relationsémantique{同义词}{\lien{ⓔtɯ-ntsiⓝɯ-ntsi,βzu}{ɯ-ntsi,βzu}}\étymologie{skʲi}\end{entrée}

\begin{entrée}{ɯ-sku}{}{ⓔɯ-sku} 
\classe{np} 
\begin{définition}\pfra{tiges et feuilles du navet}\end{définition}
\begin{définition}\pcmn{圆根的茎和叶子}\end{définition}\end{entrée}

\begin{entrée}{ɯ-smɤnjɯn}{}{ⓔɯ-smɤnjɯn} 
\classe{np} 
\begin{définition}\pfra{prix du traitement}\end{définition}
\begin{définition}\pcmn{药费}\end{définition}\relationsémantique{参考}{\lien{ⓔsmɤn}{smɤn}}\end{entrée}

\begin{entrée}{ɯ-spjɯŋ}{}{ⓔɯ-spjɯŋ} 
\classe{np} 
\begin{définition}\pfra{tige centrale}\end{définition}
\begin{définition}\pcmn{主心干}\end{définition}\end{entrée}

\begin{entrée}{ɯ-spɯɣ}{}{ⓔɯ-spɯɣ} 
\classe{np} 
\begin{définition}\pfra{partie proche du corps (membre)}\end{définition}
\begin{définition}\pcmn{靠近身体的部分;根部(肢体)}\end{définition}
\begin{exemple}\pjya{tɯ-jaʁ ɯ-spɯɣ}\hspace{5pt}\pcmn{肩膀}\end{exemple}
\begin{exemple}\pjya{tɯ-jaʁndzu ɯ-spɯɣ}\hspace{5pt}\pcmn{手指的根部}\end{exemple}\end{entrée}

\begin{entrée}{ɯ-sqar}{}{ⓔɯ-sqar} 
\classe{np} 
\begin{définition}\pfra{endroit où les fils se croisent (pendant le tissage)}\end{définition}
\begin{définition}\pcmn{线上下交叉的地方(织布的时候)}\end{définition}\end{entrée}

\begin{entrée}{ɯ-srɯβ}{}{ⓔɯ-srɯβ} 
\classe{np} 
\begin{définition}\pfra{fissure, interstice, couture}\end{définition}
\begin{définition}\pcmn{裂缝;针脚}\end{définition}\étymologie{srubs}\end{entrée}

\begin{entrée}{ɯ-stu}{₁}{ⓔɯ-stuⓗ1} 
\classe{np} 
\begin{définition}\pfra{vers l'avant, directement}\end{définition}
\begin{définition}\pcmn{对面的地方或方向,直接}\end{définition}
\begin{exemple}\pjya{ɯ-stu ʑo kɤ-ɕe tɕe tɤ-atɯɣ}\hspace{5pt}\pcmn{你直走就会遇到}\end{exemple}\relationsémantique{参考}{\lien{ⓔastu}{astu}}\end{entrée}

\begin{entrée}{ɯ-stu}{₂}{ⓔɯ-stuⓗ2} 
\classe{np} 
\begin{définition}\pfra{vrai}\end{définition}
\begin{définition}\pcmn{真心,准确,实话}\end{définition}
\begin{exemple}\pjya{nɤ-stu tɤ-fse}\hspace{5pt}\pcmn{你要注意一下,你要守规矩一点}\end{exemple}
\begin{exemple}\pjya{a-stu tu-ti-a ŋu ma}\hspace{5pt}\pcmn{我是说真心话}\end{exemple}
\begin{exemple}\pjya{ɯʑo kɯ ɯ-stu tu-ti ɲɯ-ŋu}\hspace{5pt}\pcmn{他是说真话}\end{exemple}
\begin{exemple}\pjya{kɤ-nɤma ɯ-stu tú-wɣ-nɤma ra ma mɤ-pe}\hspace{5pt}\pcmn{事要用心做,不然会不好的}\end{exemple}
\begin{exemple}\pjya{a-stu tu-ti-a}\hspace{5pt}\pcmn{我说实话}\end{exemple}
\begin{exemple}\pjya{nɤ-stu tɤ-ti}\hspace{5pt}\pcmn{你说实话吧}\end{exemple}
\begin{exemple}\pjya{nɤ-stu tu-tɯ-ti ɯ́-ŋu ?}\hspace{5pt}\pcmn{你说实话吗?}\end{exemple}
\begin{exemple}\pjya{nɤ-stu ɯ́-ŋu?}\hspace{5pt}\pcmn{你说的是不是实话?}\end{exemple}\relationsémantique{参考}{\lien{ⓔstuⓗ2}{stu₂}}\end{entrée}

\begin{entrée}{ɯ-sta}{}{ⓔɯ-sta} 
\classe{np} 
\begin{définition}\pfra{habitude}\end{définition}
\begin{définition}\pcmn{习惯}\end{définition}
\begin{exemple}\pjya{aʑɯɣ ɯ-sta a-nɯ-βze ra (=a-nɯ-ɕat-a ra)}\hspace{5pt}\pcmn{我要养成习惯}\end{exemple}
\begin{exemple}\pjya{aʑo kɤ-nɯmtɕi ɯ-sta na-βzu}\hspace{5pt}\pcmn{我早起惯了}\end{exemple}
\begin{exemple}\pjya{nɤʑo kɤ-nɯmtɕi ɯ-sta ɲɤ-k-ɤβzu-ci (=nɤʑo kɤ-nɯmtɕi ɲɤ-tɯ-ɕɤt)}\hspace{5pt}\pcmn{你早起惯了}\end{exemple}
\begin{exemple}\pjya{nɤki nɯ ɯ-kɯ-mŋɤm to-mna tɕe ɯ-sta ʑo to-fse}\hspace{5pt}\pcmn{那个人病好了,恢复了原状}\end{exemple}\relationsémantique{参考}{\lien{ⓔtɤ-sta}{tɤ-sta}}\relationsémantique{参考}{\lien{ⓔtɯ-sta}{tɯ-sta}}\relationsémantique{参考}{\lien{ⓔta}{ta}}\end{entrée}

\begin{entrée}{ɯ-stɤrju}{}{ⓔɯ-stɤrju} 
\classe{np} 
\begin{définition}\pfra{vérité}\end{définition}
\begin{définition}\pcmn{真话}\end{définition}\relationsémantique{同义词}{\lien{ⓔtʂaŋχtɤm}{tʂaŋχtɤm}}\relationsémantique{参考}{\lien{ⓔtɯ-rju}{tɯ-rju}}\relationsémantique{参考}{\lien{ⓔɯ-stuⓗ2}{ɯ-stu₂}}\end{entrée}

\begin{entrée}{ɯ-tɤjɯ}{}{ⓔɯ-tɤjɯ} 
\classe{np} \sens{1}
\begin{définition}\pfra{ajouté}\end{définition}
\begin{définition}\pcmn{填充的}\end{définition}
\begin{exemple}\pjya{ki nɤ-ŋga ɯ-tɤjɯ a-pɯ-ŋu ma tɯ-nɤndʐo}\hspace{5pt}\pcmn{再给你这件衣服,不然你会冷的}\end{exemple}
\begin{exemple}\pjya{kɯki kɯ mɤ-tɯ́-wɣ-ɕɯfka tɕe, nɤ-tɤjɯ a-pɯ-tu ra}\hspace{5pt}\pcmn{这一点东西吃不饱,就拿这个填肚子吧}\end{exemple}\relationsémantique{参考}{\lien{ⓔɣɤjɯ}{ɣɤjɯ}}\sens{2}
\begin{définition}\pfra{non seulement ... mais}\end{définition}
\begin{définition}\pcmn{不但……而且}\end{définition}
\begin{exemple}\pjya{ɲɯ-ɕɯmŋɤm ɯ-tɤjɯ tɕe ɲɯ-sɤzoŋzoŋ ʑo ŋu}\hspace{5pt}\pcmn{(荨麻)不但把人弄痛,而且让人发麻}\end{exemple}\relationsémantique{同义词}{\lien{ⓔalalaⓝʁo alala ri}{ʁo alala ri}}\relationsémantique{同义词}{\lien{ⓔraⓗ1ⓢ2ⓝmɤra ma}{mɤra ma}}\relationsémantique{同义词}{\lien{ⓔmaʁⓗ1ⓝmaʁ kɯ}{maʁ kɯ}}\end{entrée}

\begin{entrée}{ɯ-tɤmcar}{}{ⓔɯ-tɤmcar} 
\classe{np} 
\begin{définition}\pfra{percuteur}\end{définition}
\begin{définition}\pcmn{撞针}\end{définition}
\begin{exemple}\pjya{ɯ-tɤmcar nɯ pɯlthi kɯ-ndo smi sɤ-sthɤβ ŋu}\hspace{5pt}\pcmn{撞针是夹住火绳点火用的部件}\end{exemple}\relationsémantique{参考}{\lien{ⓔtɤmcar}{tɤmcar}}\end{entrée}

\begin{entrée}{ɯ-tɕhaʁ}{₁}{ⓔɯ-tɕhaʁⓗ1} 
\classe{np} 
\begin{définition}\pfra{fourrage (pour cheval)}\end{définition}
\begin{définition}\pcmn{马料(没有磨成粉)}\end{définition}
\begin{exemple}\pjya{mbro ɯ-tɕhaʁ tɤ-ta-t-a}\hspace{5pt}\pcmn{我喂了马}\end{exemple}\relationsémantique{参考}{\lien{ⓔnɯtɕhaʁ}{nɯtɕhaʁ}}\étymologie{tɕʰag}\end{entrée}

\begin{entrée}{ɯ-tɕhaʁ}{₂}{ⓔɯ-tɕhaʁⓗ2} 
\classe{np} 
\begin{définition}\pfra{handicap}\end{définition}
\begin{définition}\pcmn{残疾}\end{définition}
\begin{exemple}\pjya{a-ku tɤ-mna ri, a-mi ɯ-tɕhaʁ pɯ-ɬoʁ}\hspace{5pt}\pcmn{我的头愈好,但是脚成了残疾的}\end{exemple}
\begin{exemple}\pjya{ɯ-tɕhaʁ pa-tɕɤt}\hspace{5pt}\pcmn{他变成成了残疾人}\end{exemple}\relationsémantique{参考}{\lien{ⓔtɕhaʁ}{tɕhaʁ}}\end{entrée}

\begin{entrée}{ɯ-tɕhɤl}{}{ⓔɯ-tɕhɤl} 
\classe{np} 
\begin{définition}\pfra{amende, punition}\end{définition}
\begin{définition}\pcmn{罚款}\end{définition}
\begin{exemple}\pjya{a-tɕhɤl nɯ-kho-t-a}\hspace{5pt}\pcmn{我交了罚款}\end{exemple}\relationsémantique{参考}{\lien{ⓔnɯtɕhɤl}{nɯtɕhɤl}}\relationsémantique{参考}{\lien{ⓔtɕhɤtpa}{tɕhɤtpa}}\étymologie{tɕʰad}\end{entrée}

\begin{entrée}{ɯ-tɕhɯβ}{}{ⓔɯ-tɕhɯβ} 
\classe{np} \sens{1}
\begin{définition}\pfra{prendre en compte}\end{définition}
\begin{définition}\pcmn{考虑到……}\end{définition}
\begin{exemple}\pjya{a-tɕɯ ɯ-tɕhɯβ βze-a ɲɯ-ra}\hspace{5pt}\pcmn{我要考虑到我儿子的情况}\end{exemple}\sens{2}
\begin{définition}\pfra{afin de}\end{définition}
\begin{définition}\pcmn{便于}\end{définition}\relationsémantique{参考}{\lien{ⓔnɤxtɕhɯβ}{nɤxtɕhɯβ}}\end{entrée}

\begin{entrée}{ɯ-tɕhɯz}{}{ⓔɯ-tɕhɯz} 
\classe{np} 
\begin{définition}\pfra{éternuement}\end{définition}
\begin{définition}\pcmn{喷嚏}\end{définition}
\begin{exemple}\pjya{ɯ-tɕhɯz to-ɣi}\hspace{5pt}\pcmn{他打了喷嚏}\end{exemple}\relationsémantique{参考}{\lien{ⓔatɕhɯz}{atɕhɯz}}\end{entrée}

\begin{entrée}{ɯtɕɯn}{}{ⓔɯtɕɯn} 
\classe{n} 
\begin{définition}\pfra{énorme}\end{définition}
\begin{définition}\pcmn{巨大}\end{définition}\étymologie{tɕʰen}\end{entrée}

\begin{entrée}{ɯ-tɕɯtɕu}{}{ⓔɯ-tɕɯtɕu} 
\classe{np} 
\begin{définition}\pfra{pénis, zizi (enfant)}\end{définition}
\begin{définition}\pcmn{阴茎,(小孩的)小鸡鸡}\end{définition}\end{entrée}

\begin{entrée}{ɯte}{}{ⓔɯte} 
\classe{adv} 
\begin{définition}\pfra{au bout du compte}\end{définition}
\begin{définition}\pcmn{本来;归根到底}\end{définition}\end{entrée}

\begin{entrée}{ɯ-thaʁ}{}{ⓔɯ-thaʁ} 
\classe{np} 
\begin{définition}\pfra{verrou en bois}\end{définition}
\begin{définition}\pcmn{插销}\end{définition}
\begin{exemple}\pjya{ɯ-thaʁ pjɯ́-wɣ-rku ra / pjɯ́-wɣ-lɤt ra}\hspace{5pt}\pcmn{要插上插销}\end{exemple}\relationsémantique{参考}{\lien{ⓔrɟɤthaʁ}{rɟɤthaʁ}}\end{entrée}

\begin{entrée}{ɯ-tha,ɯ-scoz}{}{ⓔɯ-tha,ɯ-scoz} 
\classe{np} 
\begin{définition}\pfra{culture}\end{définition}
\begin{définition}\pcmn{文化}\end{définition}
\begin{exemple}\pjya{nɤʑo nɤ-tha nɤ-scoz ɲɯ-rnaʁ}\hspace{5pt}\pcmn{你文化水平高}\end{exemple}\end{entrée}

\begin{entrée}{ɯ-thɤβ}{}{ⓔɯ-thɤβ} 
\classe{np} 
\begin{définition}\pfra{au milieu}\end{définition}
\begin{définition}\pcmn{中间}\end{définition}
\begin{exemple}\pjya{aʑo ndʑi-thɤβ tu-βze-a}\hspace{5pt}\pcmn{我来调解你们之间的纠纷}\end{exemple}
\begin{exemple}\pjya{tɕi-thɤβ kɯ-dɤn me}\hspace{5pt}\pcmn{我们俩年龄相差不多,我们俩之间没有很远}\end{exemple}\relationsémantique{参考}{\lien{ⓔɯ-pɤrthɤβ}{ɯ-pɤrthɤβ}}\end{entrée}

\begin{entrée}{ɯ-thɤcu}{}{ⓔɯ-thɤcu} 
\classe{np} 
\begin{définition}\pfra{en aval}\end{définition}
\begin{définition}\pcmn{在下游}\end{définition}\end{entrée}

\begin{entrée}{ɯ-tho}{}{ⓔɯ-tho} 
\classe{np} 
\begin{définition}\pfra{pédoncule}\end{définition}
\begin{définition}\pcmn{花梗}\end{définition}\end{entrée}

\begin{entrée}{ɯ-thoβ}{}{ⓔɯ-thoβ} 
\classe{np} 
\begin{définition}\pfra{puissance}\end{définition}
\begin{définition}\pcmn{权力}\end{définition}
\begin{exemple}\pjya{rɟɤlpu ɣɯ ɯ-thoβ nɯ tɤru ɣɯ sɤznɤ kɯ-wxti ŋu}\hspace{5pt}\pcmn{国王的权力比头人的大}\end{exemple}\étymologie{tʰob}\end{entrée}

\begin{entrée}{ɯ-thoʁ}{}{ⓔɯ-thoʁ} 
\classe{np} 
\begin{définition}\pfra{sol}\end{définition}
\begin{définition}\pcmn{地上}\end{définition}\étymologie{tʰog (sa.tʰog)}\end{entrée}

\begin{entrée}{ɯ-thɯɣli}{}{ⓔɯ-thɯɣli} 
\classe{np} 
\begin{définition}\pfra{taches (sur le pelage)}\end{définition}
\begin{définition}\pcmn{花纹(带斑点)}\end{définition}
\begin{exemple}\pjya{mphrɯɣ ɣɯ ɯ-thɯɣli ɲɯ-tʂot}\hspace{5pt}\pcmn{氆氇的花纹很清晰}\end{exemple}
\begin{exemple}\pjya{kɯrtsɤɣ ɣɯ ɯ-thɯɣli ɲɯ-fkra}\hspace{5pt}\pcmn{豹子的斑点很清晰}\end{exemple}\end{entrée}

\begin{entrée}{ɯ-thɯm}{}{ⓔɯ-thɯm} 
\classe{np} 
\begin{définition}\pfra{bouchon au fond des jarres d'alcool}\end{définition}
\begin{définition}\pcmn{酒缸底部的塞子}\end{définition}\relationsémantique{参考}{\lien{ⓔtɕhɤrzɤthɯm}{tɕhɤrzɤthɯm}}\end{entrée}

\begin{entrée}{ɯ-tsa}{}{ⓔɯ-tsa} 
\classe{np} 
\begin{définition}\pfra{qui convient}\end{définition}
\begin{définition}\pcmn{适合别人的东西}\end{définition}
\begin{exemple}\pjya{a-mi ɯ-tsa ɲɯ-βze, mɯ́j-wxti, mɯ́j-xtɕi}\hspace{5pt}\pcmn{(那双鞋子)很适合我的脚,不大也不小}\end{exemple}
\begin{exemple}\pjya{ɯ-ŋga ɯ-tsa ʑo ɲɯ-βze}\hspace{5pt}\pcmn{这件衣服穿着很合身}\end{exemple}\end{entrée}

\begin{entrée}{ɯ-tshɤt}{}{ⓔɯ-tshɤt} 
\classe{np} \sens{1}
\begin{définition}\pfra{à la place de}\end{définition}
\begin{définition}\pcmn{代替}\end{définition}
\begin{exemple}\pjya{ki wɯɟa ki ndʑu ɯ-tshɤt ŋu}\hspace{5pt}\pcmn{调羹是可以代替筷子的}\end{exemple}
\begin{exemple}\pjya{a-tshɤt ɣɯ-tu-βze ɲɯ-ŋu}\hspace{5pt}\pcmn{他来替代我}\end{exemple}\sens{2}
\begin{définition}\pfra{tout juste, correctement}\end{définition}
\begin{définition}\pcmn{刚好}\end{définition}\end{entrée}

\begin{entrée}{ɯ-tshot}{}{ⓔɯ-tshot} 
\classe{np} 
\begin{définition}\pfra{qui convient tout juste}\end{définition}
\begin{définition}\pcmn{刚好合适}\end{définition}
\begin{exemple}\pjya{a-xtsa ɯ-tshot ɲɯ-βze}\hspace{5pt}\pcmn{我的鞋子刚刚合适}\end{exemple}
\begin{exemple}\pjya{aʑo ɯ-rkoz ɯ-tshot kɤ-ndo-t-a me ri, ɯʑo nɯ ma mɯ-chɯ-ɤnɯrɕo ɕti}\hspace{5pt}\pcmn{我不是故意带的刚刚好(巧克力粉的数量),自然就用了那么多}\end{exemple}\relationsémantique{同义词}{\lien{ⓔɯ-tsa}{ɯ-tsa}}\end{entrée}

\begin{entrée}{ɯ-tshɯɣa}{}{ⓔɯ-tshɯɣa} 
\classe{np} \sens{1}
\begin{définition}\pfra{forme}\end{définition}
\begin{définition}\pcmn{形状}\end{définition}\sens{2}
\begin{définition}\pfra{méthode}\end{définition}
\begin{définition}\pcmn{方法}\end{définition}
\begin{exemple}\pjya{kɤ-rɯɕmi kɯnɤ ɯ-tshɯɣa tɕe tu}\hspace{5pt}\pcmn{说话也是有方法的}\end{exemple}
\begin{exemple}\pjya{kɤ-ti ɯ-tshɯɣa mɤ-naχtɕɯɣ ri, ɯ-tun naχtɕɯɣ}\hspace{5pt}\pcmn{说法不一样,意思一样}\end{exemple}
\begin{exemple}\pjya{ndʑi-tɯ-rɯndzɤtshi ɯ-tshɯɣa ɲɯ-naχtɕɯɣ (kɯ-rɯndzɤtshi ndʑi-tshɯɣa)}\hspace{5pt}\pcmn{他们吃饭的样子是一样的}\end{exemple}\end{entrée}

\begin{entrée}{ɯ-tshɯɣrtsa}{}{ⓔɯ-tshɯɣrtsa} 
\classe{np} 
\begin{définition}\pfra{sens, contenu (d'un texte)}\end{définition}
\begin{définition}\pcmn{内容;意义}\end{définition}
\begin{exemple}\pjya{ki tɯ-rju ɯ-tshɯɣrtsa ɲɯ-rnaʁ}\hspace{5pt}\pcmn{这句话的含义很深奥}\end{exemple}\end{entrée}

\begin{entrée}{ɯ-tsi}{}{ⓔɯ-tsi} 
\classe{np} 
\begin{définition}\pfra{moment}\end{définition}
\begin{définition}\pcmn{时间}\end{définition}
\begin{exemple}\pjya{toʁde ɯ-tsi ʑo qhe kɤ-mto ɲɤ-me}\hspace{5pt}\pcmn{一瞬间就看不见了}\end{exemple}\end{entrée}

\begin{entrée}{ɯ-tsololot}{}{ⓔɯ-tsololot} 
\classe{np} 
\begin{définition}\pfra{pénis, zizi (enfant)}\end{définition}
\begin{définition}\pcmn{阴茎,(小孩的)小鸡鸡}\end{définition}\relationsémantique{同义词}{\lien{ⓔɯ-tɕɯtɕu}{ɯ-tɕɯtɕu}}\end{entrée}

\begin{entrée}{ɯ-tsɯ,rnaʁ}{}{ⓔɯ-tsɯ,rnaʁ} 
\classe{np}
\classe{vs} 
\begin{définition}\pfra{garder un secret}\end{définition}
\begin{définition}\pcmn{保守秘密}\end{définition}
\begin{exemple}\pjya{nɤ-tsɯ ɲɯ-rnaʁ}\hspace{5pt}\pcmn{你把秘密保守好}\end{exemple}
\begin{exemple}\pjya{nɤ-tsɯ a-kɤ-tɯ-ɣɤrnaʁ ra}\hspace{5pt}\pcmn{你要保守秘密!}\end{exemple}\relationsémantique{Component 1}{\lien{}{ɯ-tsɯ}}\relationsémantique{Component 2}{\lien{ⓔrnaʁ}{rnaʁ}}\relationsémantique{参考}{\lien{ⓔnɤtsɯ}{nɤtsɯ}}\end{entrée}

\begin{entrée}{ɯ-tʂɯmpɤri}{}{ⓔɯ-tʂɯmpɤri} 
\classe{n} 
\begin{définition}\pfra{lanière du tablier}\end{définition}
\begin{définition}\pcmn{围裙的带子}\end{définition}
\begin{exemple}\pjya{ɯ-tʂɯmpɤri ɲɤ-nɯrtɤβ}\hspace{5pt}\pcmn{他拴了带子}\end{exemple}
\begin{exemple}\pjya{ɯ-tʂɯmpɤri ra ltɕhɤltɕhɤt ʑo pjɤ-nɯ-ɕɯɴqoʁ}\hspace{5pt}\pcmn{围裙的带子吊着,小巧玲珑的。}\end{exemple}\relationsémantique{参考}{\lien{ⓔtʂɯmpa}{tʂɯmpa}}\relationsémantique{参考}{\lien{ⓔtɤ-ri}{tɤ-ri}}\end{entrée}

\begin{entrée}{ɯ-xso}{}{ⓔɯ-xso} 
\classe{np} \sens{1}
\begin{définition}\pfra{vide}\end{définition}
\begin{définition}\pcmn{空}\end{définition}
\begin{exemple}\pjya{tɤ-fkɯm ɯ-xso}\hspace{5pt}\pcmn{空口袋}\end{exemple}
\begin{exemple}\pjya{khɯtsa ɯ-xso}\hspace{5pt}\pcmn{空的碗}\end{exemple}
\begin{exemple}\pjya{nɯ-xso chɤ-nɯ-ɬoʁ-nɯ}\hspace{5pt}\pcmn{他们空手走出了}\end{exemple}\sens{2}
\begin{définition}\pfra{normal}\end{définition}
\begin{définition}\pcmn{随便,普通}\end{définition}
\begin{exemple}\pjya{ɯ-xso jɤ-ari-a ɕti}\hspace{5pt}\pcmn{我是随便去的}\end{exemple}
\begin{exemple}\pjya{a-rʑaβ maʁ, ɯ-xso a-βzaŋsa ɕti}\hspace{5pt}\pcmn{不是我的妻子,是个普通朋友}\end{exemple}\end{entrée}

\begin{entrée}{ɯ-xtɤfka}{}{ⓔɯ-xtɤfka} 
\classe{np} 
\begin{définition}\pfra{le ventre rempli}\end{définition}
\begin{définition}\pcmn{肚子饱}\end{définition}
\begin{exemple}\pjya{a-xtɤfka ʑo tɤ-nɯ-ndza-t-a}\hspace{5pt}\pcmn{我吃了个饱}\end{exemple}\relationsémantique{参考}{\lien{ⓔtɯ-xtu}{tɯ-xtu}}\relationsémantique{参考}{\lien{ⓔfkaⓗ2}{fka}}\end{entrée}

\begin{entrée}{ɯ-χaʁ}{}{ⓔɯ-χaʁ} 
\classe{np} 
\begin{définition}\pfra{malheur}\end{définition}
\begin{définition}\pcmn{遭殃}\end{définition}
\begin{exemple}\pjya{a-χaʁ ʑo pɯ-ari}\hspace{5pt}\pcmn{我遭殃了}\end{exemple}
\begin{exemple}\pjya{nɤ-χaʁ ʑo pa-lɤt}\hspace{5pt}\pcmn{你遭殃了}\end{exemple}
\begin{exemple}\pjya{pɯwɯ nɯ ɯ-χaʁ ʑo pɯ-tɯ-sɤɣri-t (pɯ-tɯ-ta-t)}\hspace{5pt}\pcmn{你让驴子遭殃了}\end{exemple}\end{entrée}

\begin{entrée}{ɯ-χcɤl}{}{ⓔɯ-χcɤl} 
\classe{np} 
\begin{définition}\pfra{milieu}\end{définition}
\begin{définition}\pcmn{中间}\end{définition}
\begin{exemple}\pjya{kɯ-sɤmtshi nɯ kɯ kɯ-rɟaʁ ra tɕhaʁla ɯ-χcɤl ʑo ka-tsɯm}\hspace{5pt}\pcmn{领舞者把舞蹈队伍带到了坝子中间}\end{exemple}\étymologie{dkʲil}\end{entrée}

\begin{entrée}{ɯ-χpoʁ}{}{ⓔɯ-χpoʁ} 
\classe{np} 
\begin{définition}\pfra{chapeau (champignon)}\end{définition}
\begin{définition}\pcmn{菌盖}\end{définition}
\begin{exemple}\pjya{zdɯmqe nɯnɯra, ɯ-χpoʁ cho ɯ-jɯ nɯra ɯ-grɤl me}\hspace{5pt}\pcmn{黑银耳那些,看不清楚哪里是盖盖,那里是茎}\end{exemple}\relationsémantique{同义词}{\lien{ⓔtɤ-fkaβ}{tɤ-fkaβ}}\end{entrée}

\begin{entrée}{ɯ-χsɤr}{}{ⓔɯ-χsɤr} 
\classe{np} 
\begin{définition}\pfra{calcul}\end{définition}
\begin{définition}\pcmn{数数}\end{définition}
\begin{exemple}\pjya{tɯrme thɤstɯɣ kɯ-tu nɯ ɯ-χsɤr ko-ndo}\hspace{5pt}\pcmn{他记下了有几个人}\end{exemple}\relationsémantique{同义词}{\lien{ⓔtɤ-rtsɯz}{tɤ-rtsɯz}}\relationsémantique{参考}{\lien{ⓔχsɤrⓗ1}{χsɤr₁}}\end{entrée}

\begin{entrée}{ɯ-χsɤrtoʁ}{}{ⓔɯ-χsɤrtoʁ} 
\classe{np} 
\begin{définition}\pfra{sommet pointu, excroissance pointue}\end{définition}
\begin{définition}\pcmn{尖顶}\end{définition}\étymologie{gser.tʰog}\end{entrée}

\begin{entrée}{ɯ-χto}{}{ⓔɯ-χto} 
\classe{np} 
\begin{définition}\pfra{encoche}\end{définition}
\begin{définition}\pcmn{插口}\end{définition}
\begin{exemple}\pjya{tɤ-jtsi stɤsmɤt komɤl ɯ-kɯ-ndo ɯ-spa pɯ-kɤ-saχaʁ ɯ-χto rmi.}\hspace{5pt}\pcmn{柱头两头用来支撑横梁的插口叫\lien{ⓔɯ-χto}{ɯ-χto}。}\end{exemple}
\begin{exemple}\pjya{tɯwɯ cho sarwɯ li ndʑi-χto tu, tɕeri kɯ-ɤβʑɯrdu maʁ, kɤ-kɤ-rkhe ŋu, ndʑu ɯ-phoŋbu kɤ-kɤ-znɯjɯn ŋu}\hspace{5pt}\pcmn{纺锤和搓杆也有\lien{ⓔɯ-χto}{ɯ-χto},但不是方形,是顺着木条的圆形刻着的}\end{exemple}\end{entrée}

\begin{entrée}{ɯ-χtɯ}{}{ⓔɯ-χtɯ} 
\classe{np} 
\begin{définition}\pfra{endroit difficile à voir}\end{définition}
\begin{définition}\pcmn{不容易被发现的地方}\end{définition}
\begin{exemple}\pjya{ki sɤtɕha ɯ-χtɯ ʑo ri ɕti tɕe, tɯrme kɯ-ɕe rkɯn}\hspace{5pt}\pcmn{这个地方不容易被别人发现,去的人很少}\end{exemple}\relationsémantique{参考}{\lien{ⓔaraχtɯ}{araχtɯ}}\end{entrée}

\begin{entrée}{ɯ-χtɯkrɤl}{}{ⓔɯ-χtɯkrɤl} 
\classe{np} 
\begin{définition}\pfra{comme les autres}\end{définition}
\begin{définition}\pcmn{跟其他人一样}\end{définition}
\begin{exemple}\pjya{nɤ-χtɯkrɤl nɤ-mɤ-tɯ-fse nɯ!}\hspace{5pt}\pcmn{你真的不正常!}\end{exemple}\relationsémantique{同义词}{\lien{ⓔɯ-χtɯrca}{ɯ-χtɯrca}}\end{entrée}

\begin{entrée}{ɯ-χtɯrca}{}{ⓔɯ-χtɯrca} 
\classe{adv} 
\begin{définition}\pfra{comme les autres, avec les autres}\end{définition}
\begin{définition}\pcmn{和别人一样,和别人一起}\end{définition}
\begin{exemple}\pjya{nɤ-χtɯrca tɤ-fse}\hspace{5pt}\pcmn{你要跟其他人一样}\end{exemple}
\begin{exemple}\pjya{ɯ-χtɯrca mɤ-fse}\hspace{5pt}\pcmn{他跟正常人不一样}\end{exemple}
\begin{exemple}\pjya{ɯ-χtɯrca mɤ-kɯ-ɤri}\hspace{5pt}\pcmn{不合群的人}\end{exemple}\relationsémantique{同义词}{\lien{ⓔɯ-χtɯkrɤl}{ɯ-χtɯkrɤl}}\end{entrée}

\begin{entrée}{ɯ-zarzɯr}{}{ⓔɯ-zarzɯr} 
\classe{np}  
\grammaire{n.rdpl} 
\begin{définition}\pfra{environs}\end{définition}
\begin{définition}\pcmn{边缘}\end{définition}
\begin{exemple}\pjya{tɯji ɯ-zarzɯr ra sɯjno dɤn}\hspace{5pt}\pcmn{地边杂草多}\end{exemple}\relationsémantique{同义词}{\lien{ⓔɯ-rkarkɯ}{ɯ-rkarkɯ}}\relationsémantique{参考}{\lien{ⓔɯ-zɯr}{ɯ-zɯr}}\end{entrée}

\begin{entrée}{ɯ-zbroŋ}{}{ⓔɯ-zbroŋ} 
\classe{np} 
\begin{définition}\pfra{motifs sur le bord des pains}\end{définition}
\begin{définition}\pcmn{馍馍边缘的花纹}\end{définition}
\begin{exemple}\pjya{qajɣi ɯ-zbroŋ kɯ-tu nɯ mpɕɤr}\hspace{5pt}\pcmn{有花纹的馍馍好看}\end{exemple}\end{entrée}

\begin{entrée}{ɯ-zdɤrca}{}{ⓔɯ-zdɤrca} 
\classe{adv} 
\begin{définition}\pfra{avec les autres}\end{définition}
\begin{définition}\pcmn{和别人}\end{définition}
\begin{exemple}\pjya{ɯ-zdɤrca mɤ-kɯ-ɤri}\hspace{5pt}\pcmn{不合群的人}\end{exemple}
\begin{exemple}\pjya{nɤ-zdɤrca jɤ-ɣi!}\hspace{5pt}\pcmn{你跟同伴一起来吧}\end{exemple}\relationsémantique{同义词}{\lien{ⓔɯ-χtɯrca}{ɯ-χtɯrca}}\relationsémantique{同义词}{\lien{ⓔɯ-χtɯkrɤl}{ɯ-χtɯkrɤl}}\relationsémantique{参考}{\lien{}{ɯ-zda}}\relationsémantique{参考}{\lien{ⓔtɤ-rca}{tɤ-rca}}\end{entrée}

\begin{entrée}{ɯ-zgɯr}{}{ⓔɯ-zgɯr} 
\classe{np} 
\begin{définition}\pfra{partie recourbée}\end{définition}
\begin{définition}\pcmn{卷起来的部分}\end{définition}
\begin{exemple}\pjya{tsuku tɯrme ra ɣɯ nɯ-mgɯr mɤ-kɯ-ɤstu tɕe, kɯ-ɤzgrɯ kɯ-fse, nɯ ɯ-zgɯr ɣɤʑu tu-kɯ-ti ŋgrɤl}\hspace{5pt}\pcmn{有些人背部不直,就可以说他有驼背}\end{exemple}\relationsémantique{参考}{\lien{ⓔazgɯr}{azgɯr}}\end{entrée}

\begin{entrée}{ɯ-zɯr}{}{ⓔɯ-zɯr} 
\classe{np} 
\begin{définition}\pfra{côté}\end{définition}
\begin{définition}\pcmn{旁边}\end{définition}
\begin{exemple}\pjya{ndʑu nɯ khɯtsa ɯ-zɯr nɯ-te}\hspace{5pt}\pcmn{把筷子放在碗的旁边}\end{exemple}\relationsémantique{同义词}{\lien{ⓔɯ-βzɯr}{ɯ-βzɯr}}
\begin{sous-entrée}{ɯ-zɯrɕɯzɯr}{ⓔɯ-zɯrⓝɯ-zɯrɕɯzɯr} 
\classe{np} 
\begin{définition}\pfra{le plus au bord}\end{définition}
\begin{définition}\pcmn{最边缘}\end{définition}\relationsémantique{参考}{\lien{ⓔɯ-zarzɯr}{ɯ-zarzɯr}}\end{sous-entrée}

\end{entrée}

\begin{entrée}{ɯ-ʑat}{}{ⓔɯ-ʑat} 
\classe{np} 
\begin{définition}\pfra{caractère propre}\end{définition}
\begin{définition}\pcmn{本性;自己的性格}\end{définition}
\begin{exemple}\pjya{ɯʑo tɤtɕɯpɯ ɲɯ-ɕti tɕe, ɯ-ʑat ci ɣɤʑu}\hspace{5pt}\pcmn{他既然是男孩子,调皮一点是自然的}\end{exemple}
\begin{exemple}\pjya{tɤjpa ko-lɤt tɕe ɯ-ʑat ci ɣɤʑu, tɕe tɯ-ŋga pjɯ-jaʁ ɲɯ-ra}\hspace{5pt}\pcmn{下了雪肯定会冷一点,要多穿一点衣服}\end{exemple}
\begin{exemple}\pjya{ftɕar jɤ-kɯ-zɣɯt nɯ ɯ-ʑat ci ɣɤʑu nɤ ma ɲɯ-sɤɕke}\hspace{5pt}\pcmn{天气热一点,那是因为春天到了的缘故}\end{exemple}\relationsémantique{参考}{\lien{ⓔaɣɯʑɯʑat}{aɣɯʑɯʑat}}\end{entrée}

\begin{entrée}{ɯ-ʑɤŋɤn}{}{ⓔɯ-ʑɤŋɤn} 
\classe{np} 
\begin{définition}\pfra{pour se venger de ...}\end{définition}
\begin{définition}\pcmn{为了报复……}\end{définition}
\begin{exemple}\pjya{nɤʑo taʁndo mɯ́j-tɯ-tso, nɤ-ʑɤŋɤn kɯ-ɴqa ɲɯ-ta-znɤma ŋu}\hspace{5pt}\pcmn{因为你不听话,为了报复你,就让你干重货}\end{exemple}\étymologie{ʑe.ŋan}\end{entrée}

\begin{entrée}{ɯ-ʑɤrʑɯr}{}{ⓔɯ-ʑɤrʑɯr} 
\classe{np} 
\begin{définition}\pfra{faire en même temps}\end{définition}
\begin{définition}\pcmn{一边……一边}\end{définition}
\begin{exemple}\pjya{aʑo pjɯ-ta-sɯxɕɤt ɯ-ʑɤrʑɯr lu-taʁ-a ŋu}\hspace{5pt}\pcmn{我一边教你,一边织衣服。}\end{exemple}\end{entrée}

\begin{entrée}{ɯʑo}{}{ⓔɯʑo} 
\classe{pro} 
\begin{définition}\pfra{lui}\end{définition}
\begin{définition}\pcmn{他}\end{définition}
\begin{exemple}\pjya{ɯʑo ʑo}\hspace{5pt}\pcmn{他自己}\end{exemple}\end{entrée}

\begin{entrée}{ɯʑo-sti}{}{ⓔɯʑo-sti} 
\classe{np} 
\begin{définition}\pfra{seul}\end{définition}
\begin{définition}\pcmn{独自一个人}\end{définition}
\begin{exemple}\pjya{aʑo-sti ma me-a}\hspace{5pt}\pcmn{只有我一个人}\end{exemple}\relationsémantique{参考}{\lien{ⓔmɯsti}{mɯsti}}\relationsémantique{参考}{\lien{ⓔnɯstɤraʁndo}{nɯstɤraʁndo}}\relationsémantique{参考}{\lien{ⓔstɯsti}{stɯsti}}\end{entrée}

\begin{entrée}{ɯʑoz}{}{ⓔɯʑoz} 
\classe{adv} 
\begin{définition}\pfra{à part}\end{définition}
\begin{définition}\pcmn{另外}\end{définition}\end{entrée}

\begin{entrée}{ɯʑɯʑur}{}{ⓔɯʑɯʑur} 
\classe{adv} \sens{1}
\begin{définition}\pfra{au moment de}\end{définition}
\begin{définition}\pcmn{……的时候}\end{définition}\sens{2}
\begin{définition}\pfra{en même temps que}\end{définition}
\begin{définition}\pcmn{一边……一边}\end{définition}\end{entrée}

\newpage\caractère{w}

\begin{entrée}{wajɯ}{}{ⓔwajɯ} 
\classe{n} 
\begin{définition}\pfra{petit de yak}\end{définition}
\begin{définition}\pcmn{牦牛犊}\end{définition}\end{entrée}

\begin{entrée}{waɟɯ}{}{ⓔwaɟɯ} 
\classe{n} 
\begin{définition}\pfra{tremblement de terre}\end{définition}
\begin{définition}\pcmn{地震}\end{définition}
\begin{exemple}\pjya{waɟɯ to-βzu}\hspace{5pt}\pcmn{发生了地震}\end{exemple}
\begin{exemple}\pjya{waɟɯ ɲɤ-nmu}\hspace{5pt}\pcmn{发生了地震}\end{exemple}\end{entrée}

\begin{entrée}{waŋtshaŋ}{}{ⓔwaŋtshaŋ} 
\classe{n} 
\begin{définition}\pfra{armoire}\end{définition}
\begin{définition}\pcmn{柜子}\end{définition}\end{entrée}

\begin{entrée}{wɤrwɤr}{}{ⓔwɤrwɤr}\relationsémantique{参考}{\lien{ⓔhwɤrhwɤr}{hwɤrhwɤr}}\end{entrée}

\begin{entrée}{wudɯŋ}{}{ⓔwudɯŋ} 
\classe{n} 
\begin{définition}\pfra{petite jarre}\end{définition}
\begin{définition}\pcmn{小坛子(大小像瓶子一样)}\end{définition}\end{entrée}

\begin{entrée}{wudzɯdzi}{}{ⓔwudzɯdzi} 
\classe{intj} 
\begin{définition}\pfra{exprime la peur}\end{définition}
\begin{définition}\pcmn{表示害怕(看到怪物的时候)}\end{définition}\end{entrée}

\begin{entrée}{wɣrum}{}{ⓔwɣrum} 
\classe{vs} \paradigme{dir}{tɤ-}\paradigme{dir}{thɯ-}
\begin{définition}\pfra{blanc}\end{définition}
\begin{définition}\pcmn{白}\end{définition}
\begin{sous-entrée}{sɯɣrum}{ⓔwɣrumⓝsɯɣrum} 
\classe{vt}  
\grammaire{caus} 
\begin{définition}\pfra{blanchir, rendre blanc}\end{définition}
\begin{définition}\pcmn{使其变白}\end{définition}\relationsémantique{参考}{\lien{ⓔaɣrɤɣrum}{aɣrɤɣrum}}\end{sous-entrée}

\end{entrée}

\begin{entrée}{wulaʁ}{}{ⓔwulaʁ} 
\classe{n} 
\begin{définition}\pfra{corvée}\end{définition}
\begin{définition}\pcmn{乌拉,徭役}\end{définition}\étymologie{ɦu.lag}\end{entrée}

\begin{entrée}{wum}{}{ⓔwum} 
\classe{vt} \paradigme{dir}{\_}
\begin{définition}\pfra{fermer (sac, parapluie)}\end{définition}
\begin{définition}\pcmn{收;收紧;合拢}\end{définition}
\begin{exemple}\pjya{san pɯ-wum-a}\hspace{5pt}\pcmn{我收了伞}\end{exemple}
\begin{exemple}\pjya{lʁa ɯ-mŋu kɤ-wum-a}\hspace{5pt}\pcmn{我收紧了口子}\end{exemple}
\begin{exemple}\pjya{lʁa ɯ-ŋgɯ laχtɕha thɯ-wum-a}\hspace{5pt}\pcmn{我把东西装在口袋里了}\end{exemple}\relationsémantique{参考}{\lien{ⓔawɯwum}{awɯwum}}
\begin{sous-entrée}{ʑɣɤwum}{ⓔwumⓝʑɣɤwum} 
\classe{ps}  
\grammaire{refl} 
\begin{définition}\pfra{se refermer}\end{définition}
\begin{définition}\pcmn{自己合拢;自然合拢}\end{définition}\end{sous-entrée}

\end{entrée}

\begin{entrée}{wuma}{}{ⓔwuma} 
\classe{adv} \sens{1}
\begin{définition}\pfra{très}\end{définition}
\begin{définition}\pcmn{很;非常}\end{définition}
\begin{exemple}\pjya{wuma ʑo rɤɣo ra pjɤ-mpɕɤr ɲɯ-ŋu}\hspace{5pt}\pcmn{歌非常好听}\end{exemple}\sens{2}
\begin{définition}\pfra{vrai}\end{définition}
\begin{définition}\pcmn{真正的}\end{définition}
\begin{exemple}\pjya{ɬɤndʐi wuma nɯ nɤʑo ɲɯ-tɯ-ŋu}\hspace{5pt}\pcmn{你才是真正的魔鬼}\end{exemple}
\begin{sous-entrée}{wuma tɤŋu tɕe}{ⓔwumaⓝwuma tɤŋu tɕe}
\begin{définition}\pfra{en fait, à vrai dire}\end{définition}
\begin{définition}\pcmn{说实话;实际上}\end{définition}
\begin{exemple}\pjya{wuma tɤŋu tɕe, aʑo kɤ-ndza a-ʁjiz mɯ́j-ɣi.}\hspace{5pt}\pcmn{说实话,我不像吃东西}\end{exemple}\end{sous-entrée}

\étymologie{ŋo.ma}\end{entrée}

\begin{entrée}{wuŋgru}{}{ⓔwuŋgru} 
\classe{n} 
\begin{définition}\pfra{vesce}\end{définition}
\begin{définition}\pcmn{野豌豆}\end{définition}
\begin{exemple}\pjya{wuŋgru nɯ sɯjno kɯ-mbɤr ŋu, ɯ-tshɯɣa nɯ staχpɯ fse, tɯ-ji ɯ-ŋgɯ tɯ-ji ɯ-rkɯ arɤndɯndɤt ʑo tu-ɬoʁ ɕti, ɯ-mɯntoʁ kɯ-ɤɣɯrnɯɕɯr ŋu, ɯ-ru nɯ kɯ-ɤβʑɯrdu ŋu, wuma ʑo nɤrko, fsapaʁndza sna, tɯrme kɤ-ndza mɤ-sna.}\hspace{5pt}\pcmn{野豌豆是矮小的植物,样子像豌豆,地里和地边到处都可以生长,花淡红色,茎四方形、非常坚实。可以喂牲畜,人不能吃。}\end{exemple}\end{entrée}

\begin{entrée}{wo}{}{ⓔwo} 
\classe{part} 
\begin{définition}\pfra{impératif intensifié}\end{définition}
\begin{définition}\pcmn{加强命令式的语气}\end{définition}
\begin{exemple}\pjya{kɯm nɯ-thɯ-pe wo}\hspace{5pt}\pcmn{你把门关上嘛}\end{exemple}\end{entrée}

\begin{entrée}{woɬaʁ}{}{ⓔwoɬaʁ} 
\classe{n} 
\begin{définition}\pfra{marâtre}\end{définition}
\begin{définition}\pcmn{继母}\end{définition}\end{entrée}

\begin{entrée}{worɟɤmchɤn}{}{ⓔworɟɤmchɤn} 
\classe{intj} 
\begin{définition}\pfra{exprime l'étonnement}\end{définition}
\begin{définition}\pcmn{表示自己感触很深,惊叹}\end{définition}
\begin{exemple}\pjya{worɟɤmchɤn, jisŋi tɯ-mɯ ɯ-tɯ-dɤn}\hspace{5pt}\pcmn{天啊,今天雨怎么这么多!}\end{exemple}
\begin{exemple}\pjya{worɟɤmchɤn ɯ-tɯ-mbro nɯ!}\hspace{5pt}\pcmn{天啊,你这么高!}\end{exemple}
\begin{exemple}\pjya{worɟɤmchɤn ɯ-tɯ-nɯɲɤmkhe nɯ!}\hspace{5pt}\pcmn{天啊,你这么瘦!}\end{exemple}\étymologie{ʔo.rgʲan.mkʰʲen}\end{entrée}

\begin{entrée}{wortɕhi}{}{ⓔwortɕhi} 
\classe{intj} 
\begin{définition}\pfra{je vous en prie}\end{définition}
\begin{définition}\pcmn{求您了}\end{définition}
\begin{exemple}\pjya{wortɕhi tu-kɯ-qur-a-nɯ ɲɯ-ntshi}\hspace{5pt}\pcmn{求你们帮一下我}\end{exemple}\étymologie{ɦor.tɕʰe}\end{entrée}

\begin{entrée}{wortɕhi wojɤr}{}{ⓔwortɕhi wojɤr} 
\classe{intj} 
\begin{définition}\pfra{je vous en prie}\end{définition}
\begin{définition}\pcmn{求您了}\end{définition}\étymologie{ɦor.tɕʰe}\end{entrée}

\begin{entrée}{wutɕhɯtɕhɯ}{}{ⓔwutɕhɯtɕhɯ} 
\classe{intj} 
\begin{définition}\pfra{exprime que le locuteur a froid}\end{définition}
\begin{définition}\pcmn{表示冷}\end{définition}\end{entrée}

\begin{entrée}{wɯdɯŋ}{}{ⓔwɯdɯŋ} 
\classe{n} 
\begin{définition}\pfra{petite jarre d'alcool}\end{définition}
\begin{définition}\pcmn{小酒坛}\end{définition}\end{entrée}

\begin{entrée}{wɯɟa}{}{ⓔwɯɟa} 
\classe{n} 
\begin{définition}\pfra{cuillère}\end{définition}
\begin{définition}\pcmn{调羹}\end{définition}\end{entrée}

\begin{entrée}{wɯrna}{}{ⓔwɯrna} 
\classe{n} 
\begin{définition}\pfra{poids du fuseau}\end{définition}
\begin{définition}\pcmn{纺锤的秤砣}\end{définition}\end{entrée}

\begin{entrée}{wɯwɯ}{}{ⓔwɯwɯ} 
\classe{n} 
\begin{définition}\pfra{bolet}\end{définition}
\begin{définition}\pcmn{牛肝菌}\end{définition}
\begin{exemple}\pjya{wɯwɯ jmɤɣ nɯ ɕkrɤz, tɯrgi, sɤjku ɯ-ŋgɯ ra tu-ɬoʁ ŋu, ɯ-mgɯrqhu nɯ kɯ-qandʐɯlu tɕe kɯ-nɤmbju ŋu, ɯ-rʑɯɣ me kú-wɣ-nɤmɯma tɕe kɯ-mpɯ-mpɯ ʑo ŋu, ɯ-mdoʁ nɯ ɯ-pa ɯ-pɕoʁ kɯ-wɣrum tu, kɯ-qarŋe tu, ɯ-pa kɯ-wɣrum nɯ kɤ-ndza sna, ɯ-pa kɯ-qarŋe nɯ kɤ-ndza mɤ-sna, ɯ-ru nɯ pjɯ́-wɣ-qlɯt tɕe mɤ-ndoʁ kɤ-saʁ khɯ}\hspace{5pt}\pcmn{牛肝菌长在青冈树林、杉木林和白桦树林里,背面是乌色的,光滑,没有菌褶,摸起来很软。它下部有的是白色,有的是黄色的,白色的那些可以吃,黄色不可以吃。菌干不脆,不能折,只能撕。}\end{exemple}\end{entrée}

\begin{entrée}{wxti}{}{ⓔwxti} 
\classe{vs} \paradigme{dir}{tɤ-}\paradigme{dir}{thɯ-}\paradigme{dir}{tɤ-}
\begin{définition}\pfra{grand}\end{définition}
\begin{définition}\pcmn{大}\end{définition}
\begin{définition}\pfra{rendre grand}\end{définition}
\begin{définition}\pcmn{弄大}\end{définition}
\begin{exemple}\pjya{tɤ-pɤtso chɤ-wxti}\hspace{5pt}\pcmn{小孩子大了}\end{exemple}
\begin{exemple}\pjya{tɯ-ci chɤ-wxti}\hspace{5pt}\pcmn{水涨了}\end{exemple}
\begin{exemple}\pjya{@dianhua ɯ-skɤt ci ci tu-wxti, ci ci ɲɯ-xtɕi ɲɯ-ŋu tɕe, koŋla mɯ́j-sɤmtshɤm}\hspace{5pt}\pcmn{电话的声音一会大,一会小,听得不完整}\end{exemple}
\begin{exemple}\pjya{tɤ-ɣɤwxti-t-a}\hspace{5pt}\pcmn{我弄大了}\end{exemple}\relationsémantique{参考}{\lien{ⓔrɯxtuxti}{rɯxtuxti}}\relationsémantique{反义词}{\lien{ⓔxtɕi}{xtɕi}}
\begin{sous-entrée}{ɣɤwxti}{ⓔwxtiⓝɣɤwxti} 
\classe{vt} \end{sous-entrée}

\end{entrée}

\newpage\caractère{x}

\begin{entrée}{xɤlxɤl}{}{ⓔxɤlxɤl} 
\classe{idph.2} 
\begin{définition}\pfra{qui porte des habits pas trop serrés}\end{définition}
\begin{définition}\pcmn{形容衣服穿得宽松的样子}\end{définition}
\begin{sous-entrée}{xɤlnɤxɤl}{ⓔxɤlxɤlⓝxɤlnɤxɤl}
\begin{exemple}\pjya{tɯ-ŋga kɯ-ɤŋgɤjom tsa ɲɯ́-wɣ-nɯ-βzu tɕe, tɤ-kɯ-ŋke tɕe xɤlnɤxɤl pa tɕe sɤscit}\hspace{5pt}\pcmn{衣服宽一点的话,穿起来走路方便,舒服}\end{exemple}\end{sous-entrée}

\end{entrée}

\begin{entrée}{xɤtxɤt}{}{ⓔxɤtxɤt} 
\classe{idph.2} 
\begin{définition}\pfra{énorme et somptueux}\end{définition}
\begin{définition}\pcmn{形容又宽大又豪华的样子(房子、建筑物)}\end{définition}
\begin{exemple}\pjya{kha xɤtxɤt ʑo ɲɯ-pa}\hspace{5pt}\pcmn{房子又宽大又豪华}\end{exemple}\end{entrée}

\begin{entrée}{xcaŋxcaŋ}{}{ⓔxcaŋxcaŋ} 
\classe{idph.2} 
\begin{définition}\pfra{gros et plat}\end{définition}
\begin{définition}\pcmn{又大又扁状}\end{définition}
\begin{exemple}\pjya{ɯ-xtsa ɲɯ-nɤwxti-a tɕe xcaŋxcaŋ mɯ́j-nɯɣɯŋga}\hspace{5pt}\pcmn{他的鞋子我穿太大了,不合适}\end{exemple}
\begin{sous-entrée}{xcaŋnɤxcaŋ}{ⓔxcaŋxcaŋⓝxcaŋnɤxcaŋ} 
\classe{idph.3} 
\begin{exemple}\pjya{xcaŋnɤxcaŋ ʑo ɲɯ-nɤŋkɯŋke}\hspace{5pt}\pcmn{他穿着太大的鞋子走路}\end{exemple}\end{sous-entrée}

\end{entrée}

\begin{entrée}{xcat}{}{ⓔxcat} 
\classe{vs} \paradigme{dir}{nɯ-}\paradigme{dir}{tɤ-}
\begin{définition}\pfra{nombreux}\end{définition}
\begin{définition}\pcmn{有很多}\end{définition}
\begin{exemple}\pjya{tɯrme xcat}\hspace{5pt}\pcmn{人很多}\end{exemple}
\begin{exemple}\pjya{a-ŋga xcat ʑo ɕti}\hspace{5pt}\pcmn{我有很多衣服}\end{exemple}\relationsémantique{同义词}{\lien{ⓔdɤn}{dɤn}}
\begin{sous-entrée}{sɯxcat}{ⓔxcatⓝsɯxcat} 
\classe{vt}  
\grammaire{caus} 
\begin{exemple}\pjya{tɯ-mɯ qale kɯ ɲɯ-ɤsɯ-sɯxcat ʑo}\hspace{5pt}\pcmn{下着狂风暴雨}\end{exemple}\end{sous-entrée}

\end{entrée}

\begin{entrée}{xcɤxcɤt}{}{ⓔxcɤxcɤt} 
\classe{idph.2} 
\begin{définition}\pfra{petit et vif, mignon}\end{définition}
\begin{définition}\pcmn{形容小巧玲珑的样子}\end{définition}
\begin{exemple}\pjya{tɤ-pɤtso ɯ-rŋa ra xcɤxcɤt ʑo pa tɕe ɲɯ-sɤjndɤt}\hspace{5pt}\pcmn{小孩子的脸又小又可爱}\end{exemple}\end{entrée}

\begin{entrée}{xchɯxcho}{}{ⓔxchɯxcho} 
\classe{idph.2} 
\begin{définition}\pfra{creux}\end{définition}
\begin{définition}\pcmn{形容空心的样子}\end{définition}\end{entrée}

\begin{entrée}{xchuxchu/\variante{xcuxcu}}{}{ⓔxchuxchu} 
\classe{idph.2} 
\begin{définition}\pfra{épais et résistant (pétales d'une fleur)}\end{définition}
\begin{définition}\pcmn{形容花等厚而结实的样子}\end{définition}
\begin{exemple}\pjya{ɯ-mɯntoʁ nɯ xchuxchu ʑo ɲɯ-pa}\hspace{5pt}\pcmn{它的花显得又厚又结实}\end{exemple}
\begin{exemple}\pjya{tɤ-pɤtso kɯ ɯ-mtɕhi xchuxchu ʑo to-stu}\hspace{5pt}\pcmn{小孩子嘟着嘴巴}\end{exemple}\end{entrée}

\begin{entrée}{xɕaj}{₁}{ⓔxɕajⓗ1} 
\classe{n} 
\begin{définition}\pfra{herbe}\end{définition}
\begin{définition}\pcmn{一种草}\end{définition}
\begin{exemple}\pjya{xɕaj ʁnɯ-tɯphu tu tɕe, si ci tu, sɯjno ci tu. sɯjno nɯ ɯ-jwaʁ nɯ kɯ-rɲɟi kɯ-tɕɤr lu-kɯ-ɤmtɕoʁ ŋu. ɯ-ru nɯ ɯ-rtsɤɣ lu-oʑɯrja ŋu, ɯ-rtsɤɣ nɯtɕu ɯ-jwaʁ ntsɯ ku-ndzoʁ ŋu. ɯ-jwaʁ ɯ-qa nɯtɕu, ɯ-ru nɯ ku-mphɯr kɯ-fse ŋu, tɕe ɯ-ru lɤ-zri tɕe ɯ-lɤcu tɕe, li ɯ-rtsɤɣ ɲɯ-βze tɕe, li ɯ-jwaʁ lu-ɬoʁ ŋu. xɕaj nɯ tɯ-phɯ ɯ-ŋgɯ, lɤŋɤtʂɤ-ldʑa ɲɯ-βze cha, ɯ-kɤχcɤl nɯ tɕu ɯ-mat kɯ-ɕnom kɯ-fse lu-βze ŋu. sɯjno xɕaj nɯ kɤntɕhɯ-tɯphɯ tu, pɣɤtɕɯxɕaj kɤ-ti tu, pɤŋɤxɕaj kɤ-ti tu, cɤmi xɕaj kɤ-ti tu, xsɤrɯ kɤ-ti tu, pɣɤjmɤt kɤ-ti tu, tɯ-ci xɕaj kɤ-ti tu, tɕeri nɯ-tshɯɣa ra kɯ-naχtɕɯɣ tsa ɕti, nɯ ɯ-ŋgɯ zɯ cɤmi xɕaj stu ʑo wxti, cɤmi tsa tu-ɬoʁ ŋu, tɯ-ci xɕaj kɯ-mbɯ-mbɤr ma me, tɕhɯtoʁ ku tu-ɬoʁ ŋu, ɯ-ro nɯ ra zgoku tsa tu-ɬoʁ ŋu. nɯŋa cho jla kɤ-mbi wuma ʑo pe.}\hspace{5pt}\pcmn{\lien{ⓔxɕajⓗ1}{xɕaj}有两种,一种是树,另一种是草。草的那种叶子细长,上面尖。茎由许多节组成,每一节上有长有叶子。叶子的根部好像是裹着茎长出来的,叶子裹着的茎长高了又长出节来,又长叶子。每一秆的可以分五、六根,在顶上抽穗结果。这种草有很多种,有\lien{}{pɣɤtɕɯxɕaj}、\lien{ⓔpɤŋɤxɕaj}{pɤŋɤxɕaj}、\lien{}{cɤmi xɕaj}、\lien{ⓔxsɤrɯ}{xsɤrɯ}、\lien{ⓔpɣɤjmɤt}{pɣɤjmɤt}、\lien{}{tɯ-ci xɕaj}六种,但它们的形状都差不多相同。其中\lien{}{cɤmi xɕaj}最大,生长在河坝,\lien{}{tɯ-ci xɕaj}很矮,长在水草地上,其它的都生长在高山上。是喂奶牛和犏牛的好饲料。}\end{exemple}\end{entrée}

\begin{entrée}{xɕaj}{₂}{ⓔxɕajⓗ2} 
\classe{n} 
\begin{définition}\pfra{une espèce d'arbre}\end{définition}
\begin{définition}\pcmn{乔木的一种}\end{définition}
\begin{exemple}\pjya{xɕaj ʁnɯ-tɯphu tu tɕe, si ci tu, sɯjno ci tu. si nɯ kɯ-mbɯ-mbro ci ŋu, ɯ-jpum tsa ɲɯ-βze cha, tɕe zgo kɯ-mbɤr tsa zɯ tu-ɬoʁ ŋu, ɯ-ru ɯ-pɕi nɯ kɯ-pɣi ŋu, ɯ-rtaʁ dɤn, ɯ-jwaʁ ʑmbri ɯ-jwaʁ tsa fse, ɯ-mɯntoʁ kɯ-ndɯ-ndɯβ ɕti khro mɤ-χsɤl, ɯ-mat nɯ ɯ-cɤβ chɯ-βze ŋu, tɕeri ɯ-rɣi nɯ pɯ-ŋgra tɕe tu-ɬoʁ mɤ-cha. ɯʑo ɯ-kɯ-sɯ-mphɯl nɯ, ɯ-zrɤm ɕti. ɯ-si ngɯt tɕe, laʁdɯn ɯ-jɯ kɤ-βzu pe. kɤ-nɯ-βlɯ kɯnɤ wuma ʑo pe ma thɯ́-wɣ-nɯ-βlɯ tɕe ɯ-khɯ mɤ-wxti. ɯ-smi sɤɕke. si ɯ-xɕaj nɯ li ʁnɯ-tɯphu tu. tɕe xɕaj ɲaʁ kɤ-ti ci tu, xɕaj wɣrum kɤ-ti ci tu. xɕaj ɯ-rqhu nɯ fsapaʁ ɯ-ɕɤrɯ tɤ-mtshɤz kɤ-kɯ-ndo ɣɯ ɯ-smɤn ɲɯ-ŋu khi.}\hspace{5pt}\pcmn{{xɕaj}有两种,一种是树,另一种是草。树的那种长得很高、比较粗,生长在下半山。树皮呈灰色,枝桠多。叶子类似柳树的叶子。花很小,看不清楚。结荚果,但种子掉了以后不能发芽,繁殖靠根。因为木质结实,可以作各种农具的把子。也是烧火的好柴,因为烧起来发出很少烟,火气又高。这种树也分成两种,一种叫\lien{}{xɕaj ɲaʁ},另一种叫\lien{}{xɕajɣrum} 。\lien{ⓔxɕajⓗ1}{xɕaj}的树皮是牲畜骨质增生病的良药。}\end{exemple}\end{entrée}

\begin{entrée}{xɕajɲaʁ}{}{ⓔxɕajɲaʁ} 
\classe{n} 
\begin{définition}\pfra{espèce d'arbre}\end{définition}
\begin{définition}\pcmn{乔木的一种}\end{définition}\end{entrée}

\begin{entrée}{xɕɤfsa}{}{ⓔxɕɤfsa} 
\classe{n} 
\begin{définition}\pfra{ficelle (faite de paille)}\end{définition}
\begin{définition}\pcmn{草绳}\end{définition}\relationsémantique{参考}{\lien{ⓔxɕajⓗ1}{xɕaj}}\end{entrée}

\begin{entrée}{xɕɤɣrum}{}{ⓔxɕɤɣrum} 
\classe{n} 
\begin{définition}\pfra{espèce d'arbre}\end{définition}
\begin{définition}\pcmn{乔木的一种}\end{définition}\relationsémantique{参考}{\lien{ⓔxɕajⓗ1}{xɕaj}}\end{entrée}

\begin{entrée}{xɕɤndʑu}{}{ⓔxɕɤndʑu} 
\classe{n} 
\begin{définition}\pfra{bâton très fin}\end{définition}
\begin{définition}\pcmn{又细又短的小木棒}\end{définition}\end{entrée}

\begin{entrée}{xɕelwi}{}{ⓔxɕelwi} 
\classe{n} 
\begin{définition}\pfra{tique}\end{définition}
\begin{définition}\pcmn{蜱【草虱】}\end{définition}
\begin{exemple}\pjya{xɕelwi nɯ qajɯ ci ŋu. ɯ-mdoʁ kɯ-ɣɯrni ŋu. ɯ-mɤlɤjaʁ kɯβdɤ-ldʑa tu, ɯ-ku kɯ-xtɕɯ-xtɕi ŋu, ɯ-mtɕhi ɲɯ-ɤmtɕoʁ, sɯŋgɯ kɯ-mbɤr cho stɤmku ra ɣɤʑu tɕe nɯŋa cho jla mbala nɯ ra nɯ-taʁ kɤ-ndzoʁ rga, tɕe nɯ-se ku-tshi tɕe tɤ-fka tɕe ɯʑo kɯ-wxtɯ-wxti ʑo ɲɯ-βze ŋu, fsapaʁ ɯ-se ku-tshi ɕɯŋgɯ staʁnɤ sqi jamar ɲɯ-wxti ɲɯ-ŋu. tɕe xɕelwi nɯ tɤ-fka tɕe pjɤ-tɤr ɲɯ-ɕti ma ɯ-jɤɣɤt sɤ-lɤt maŋe, tɕe tu-ndze nɤ tu-ndze ma nɯ ma kɤ-lɤt mɯ́j-khɯ. tɕe wuma ʑo dɤn, fsapaʁ tɯ-rdoʁ ɯ-taʁ kɤ-rtsi mɤ-kɯ-sɤcha ʑo ku-ndzoʁ ɲɯ-ɕti, tɕe fsapaʁ ɯ-taʁ wuma ʑo ʁnɤt, tɕe tham tɕe ɯ-smɤn ɣɤʑu tɕe, tú-wɣ-lɤt tɕe ɲɯ-ɣɤme ɲɯ-ŋu ri ɯ-qhu tɕe li ku-ndzoʁ ɲɯ-ɕti. tɯrme ɯ-taʁ kɯnɤ ku-ndzoʁ ŋgrɤl tɕe ɣɯ-phɯt tɤ-kha tɕe ɯ-rmi tu-βzu mɤ-βdi ma tɕe ɯ-phoŋbu nɯ kɤ-phɯt khɯ ma ɯ-ku nɯ ku-raʁ ŋu tɕe kɤ-phɯt mɤ-khɯ. mɤ-kɤ-nɤrmi ɲɯ́-wɣ-phɯt tɕe kɤ́kɯku kɤ-phɯt khɯ tu-kɯ-ti ɲɯ-ŋu.}\hspace{5pt}\pcmn{蜱是一种虫,是红色的。有四只脚,头部很小,嘴很尖。生活在灌木丛和草地上,喜欢爬在奶牛、犏牛和黄牛身上吸血。吸饱以后,它的身体就变得很大,是它吸血前的十倍以上,蜱吸饱了就会掉下来。因为身上没有肛门,只能吃了又吃,不能解便。蜱非常多,一头牛身上的蜱不计其数,对牲畜非常有害。现在有一种农药可以消灭它,但过了一段时间它会重新出现。蜱也会爬在人身上。据说拔掉的时候不能叫它的名字,不然只能拔掉它的身子,它的头部会卡在里面。不叫它的名字的话就可以连头一起拔掉。}\end{exemple}\end{entrée}

\begin{entrée}{xɕiri}{}{ⓔxɕiri} 
\classe{n} 
\begin{définition}\pfra{belette}\end{définition}
\begin{définition}\pcmn{黄鼠狼}\end{définition}
\begin{exemple}\pjya{xɕiri nɯ ɯ-ku lɯlu fse, ɯ-rna ra kɯnɤ lɯlu fse, ɯ-mi nɯ ra ɯ-ndzrɯ tu, ɯ-phoŋbu ra kɯ-ɤɣɯrnɯɕɯr ŋu, ɯ-jme jpum cho rɲɟi, ɯ-phoŋbu cho ɯ-jme ra ɯ-tɯ-jpum ɯ-tɯ-zri afsu. tɤ-rɤku mɤ-ndze, tɕeri ɯʑo staʁ kɯ-xtɕi ɣɯ pɣɤtɕɯ, βʑɯ nɯ ra tu-ndze ŋu. ci ci tɕe lɯlu kɯnɤ ku-mtsɯɣ ŋgrɤl, li ci ci tɕe tɤ-pɤtso nɤrŋi tsa ku-mtsɯɣ cho pjɯ-mɯrʁɯz ra ŋgrɤl, tɕe rɯdaʁ kɯ-ŋɤn tsa ci ŋu.}\hspace{5pt}\pcmn{黄鼠狼的头像猫,耳朵也像猫,脚上有爪子,身体是淡红色的,尾巴粗而长,身子的粗细长短与尾巴相同。不吃粮食,但是吃比自己小的鸟、老鼠等。有时候还会咬猫,咬伤和抓伤婴儿,是比较坏的动物。}\end{exemple}\end{entrée}

\begin{entrée}{xɕɯβ}{}{ⓔxɕɯβ} 
\classe{vs} \paradigme{dir}{nɯ-}\paradigme{dir}{pɯ-}\paradigme{dir}{lɤ-}
\begin{définition}\pfra{se dégonfler}\end{définition}
\begin{définition}\pcmn{瘪下去;缩下去}\end{définition}
\begin{exemple}\pjya{ɯ-ŋgɯ tɯ-ci ɲɤ-me tɕe ɲɤ-xɕɯβ}\hspace{5pt}\pcmn{里面的水没有了,缩下去了}\end{exemple}\relationsémantique{同义词}{\lien{ⓔɲchoʁ}{ɲchoʁ}}\end{entrée}

\begin{entrée}{xoŋnɤxoŋ}{}{ⓔxoŋnɤxoŋ} 
\classe{idph.2} 
\begin{définition}\pfra{sensation d'engourdissement dans la bouche (après avoir mangé du xanthoxyle)}\end{définition}
\begin{définition}\pcmn{形容吃了花椒以后麻的感觉}\end{définition}
\begin{sous-entrée}{sɤxoŋxoŋ}{ⓔxoŋnɤxoŋⓝsɤxoŋxoŋ} 
\classe{vt} 
\begin{définition}\pfra{engourdir la bouche (xanthoxyle)}\end{définition}
\begin{définition}\pcmn{发麻}\end{définition}
\begin{exemple}\pjya{tɕɣom a-kɯr ɯ-ŋgɯ na-sɤxoŋxoŋ ʑo}\hspace{5pt}\pcmn{花椒吃在嘴里麻得很}\end{exemple}\relationsémantique{参考}{\lien{ⓔɣɤzɯβzɯβ}{ɣɤzɯβzɯβ}}\end{sous-entrée}

\end{entrée}

\begin{entrée}{xoŋxoŋ}{}{ⓔxoŋxoŋ} 
\classe{idph.2} 
\begin{définition}\pfra{blanchâtre}\end{définition}
\begin{définition}\pcmn{形容灰白}\end{définition}
\begin{exemple}\pjya{tɯɣur xoŋxoŋ ʑo pjɤ-ta}\hspace{5pt}\pcmn{打了灰白的霜}\end{exemple}\end{entrée}

\begin{entrée}{xsar}{}{ⓔxsar} 
\classe{n} 
\begin{définition}\pfra{naemorhedus goral}\end{définition}
\begin{définition}\pcmn{岩羊【青羊】}\end{définition}
\begin{exemple}\pjya{xsar nɯ cɤmi praʁ ɯ-ŋgɯ ku-rɤʑi ɲɯ-ŋu, praʁ ɯ-taʁ kɤ-ŋke wuma kɯ-cha ŋu, tɯ-mɯ kɯ pɯ-pa-χtɕi tu-ŋke-a cha tu-ti ŋu tu-kɯ-ti ɲɯ-ŋu, ɯ-mdoʁ nɯ kɯ-pɣi ɲɯ-ŋu, tu-nɯsɯku ŋgrɤl, sɯjwaʁ tu-ndze, ɯ-ʁrɯ nɯ tɯ-tɕha kɯ-mtɕɯ-mtɕoʁ tu, khɯna kɤ-sat wuma kɯ-cha ŋu. khɯna kɯ praʁ ɯ-taʁ ku-roʁ ŋgrɤl tɕe, nɯ tɕu tɕe tu-tɕhi tɕe pjɤ-sat ŋgrɤl. qartsɯ tɤjpɣom kɤ-ta ɯ-raŋ tɕe, tshu, tɕe nɯ chɯ-nɯrmɤmbe ŋu. ɯ-qa taʁrɯ nɯ tshɤt cho naχtɕɯɣ}\hspace{5pt}\pcmn{青羊一般生活在河坝的岩石里,善于在岩石上行走。据说它自吹:“凡雨水能淋到的地方,我都能走”。青羊是灰色的,能爬树,吃树叶,有一对很尖的角,能把狗杀死。狗把它追到悬崖时,青羊就会用角把狗顶死。冬天结冰时,青羊会变得很肥,会脱毛。蹄子像山羊的蹄子一样。}\end{exemple}\end{entrée}

\begin{entrée}{xsɤndʐi}{}{ⓔxsɤndʐi} 
\classe{n} 
\begin{définition}\pfra{peau de goral}\end{définition}
\begin{définition}\pcmn{岩羊【青羊】皮子}\end{définition}\relationsémantique{参考}{\lien{ⓔxsar}{xsar}}\relationsémantique{参考}{\lien{ⓔtɯ-ndʐi}{tɯ-ndʐi}}\end{entrée}

\begin{entrée}{xsɤrɯ}{}{ⓔxsɤrɯ} 
\classe{n} 
\begin{définition}\pfra{une plante}\end{définition}
\begin{définition}\pcmn{植物的一种}\end{définition}
\begin{exemple}\pjya{xsɤrɯ nɯ tɯ-ji ɯ-rkɯ tu-kɯ-ɬoʁ ci sɯjno ŋu. ɯ-jwaʁ kɯ-mba kɯ-zri tsa ŋu, ɯ-zrɤm nɯ kɯ-wɣrum kɯ-ɤrɤrtsɯ-rtsɤɣ ci ŋu, rko. jla nɯŋa ra wuma rga-nɯ, tɤrɤku ɯ-ŋgɯ tɤ-ɬoʁ tɕe, mɤ-sɤpe. ɯ-lu tu.}\hspace{5pt}\pcmn{\lien{ⓔxsɤrɯ}{xsɤrɯ}是生长在田边的植物,叶子细长,根是白色的,有节且硬。犏牛、奶牛都喜欢吃。长在庄稼地里时,对庄稼不好。有乳汁。}\end{exemple}\end{entrée}

\begin{entrée}{xsɯr}{}{ⓔxsɯr} 
\classe{vt}  
\grammaire{apass} \paradigme{dir}{tɤ-}\paradigme{dir}{tɤ-}
\begin{définition}\pfra{poêler}\end{définition}
\begin{définition}\pcmn{炒}\end{définition}
\begin{exemple}\pjya{aʑo tɤjmɤɣ tɤ-xsɯr-a}\hspace{5pt}\pcmn{我把蘑菇炒了}\end{exemple}\relationsémantique{参考}{\lien{ⓔɕnɤxsɯr}{ɕnɤxsɯr}}
\begin{sous-entrée}{rɤxsɯr}{ⓔxsɯrⓝrɤxsɯr} 
\classe{vi} \end{sous-entrée}

\begin{exemple}\pjya{z-rɤxsɯr}\hspace{5pt}\pcmn{炒锅(用来炒菜的工具)}\end{exemple}\end{entrée}

\begin{entrée}{xsɯxsɯ}{}{ⓔxsɯxsɯ} 
\classe{n} 
\begin{définition}\pfra{type de sac}\end{définition}
\begin{définition}\pcmn{口袋的一种}\end{définition}
\begin{exemple}\pjya{xsɯxsɯ (khorca) nɯ laχtɕha fkɯm tu-kɤ-fkur ci ŋu, rgali pɯ ɯ-ndʐi nɯ kɯ-mdoʁmdi pjɯ́-wɣ-qaʁ tɕe tɤ-rom tɕe ta-mar ɲɯ́-wɣ-mar pjɯ́-wɣ-χtsɤβ tɕe, nɯ mpɯ ʑo tɕe, ɯ-rme nɯ ɯ-pɕi ɲɯ́-wɣ-ɕthɯz tɕe, ɯ-mɤlɤjaʁ kɯβde ɣɯ ɯ-ndʐi nɯ tɕu ɯ-sɤ-fkur ɯ-ri kú-wɣ-tshoʁ, thaχthi maʁ nɤ tɯ-ndʐi qase thɯ-kɤ-tɕɤt kú-wɣ-tshoʁ tɕe, nɯ ɯ-sɤ-fkur ŋu. ɯ-mke stu nɯ pjɯ́-wɣ-pri tɕe, nɯ tɤ-fkɯm ɣɯ ɯ-mŋu ɲɯ́-wɣ-βzu ŋu, tɕe nɯtɕu qase ci kú-wɣ-βraʁ tɕe tɤ-fkɯm ɯ-ŋgɯ laχtɕha thɯ́-wɣ-rku tɕe, qase nɯ kɯ ɯ-mŋu nɯ kú-wɣ-sɯ-xtɕɤr ŋu, ɯ-mŋu nɯ ɯ-taʁ pɕoʁ tú-wɣ-ɕthɯz tɕe tú-wɣ-fkur ŋu.}\hspace{5pt}\pcmn{\lien{ⓔxsɯxsɯ}{xsɯxsɯ}是装东西的口袋,可以背。把小牛的皮整块剥下来,干了以后,就擦上酥油,搓揉,等到揉得软了,毛向外翻,在四条腿的皮子上系上背带,或者用线制作的背带,或者用剖成的皮绳作背带。在脖颈部位破一条口子,作为口袋的开口,在那里扎一根皮绳,把东西装好后,就用皮绳捆住口子。背的时候,口袋的口子向上。}\end{exemple}\end{entrée}

\begin{entrée}{xʂɤxʂɤt}{}{ⓔxʂɤxʂɤt} 
\classe{idph.2} 
\begin{définition}\pfra{long et fin, flexible}\end{définition}
\begin{définition}\pcmn{形容苗条、纤细、柔软的样子}\end{définition}
\begin{exemple}\pjya{ɯ-xtu ra xʂɤxʂɤt to-stu}\hspace{5pt}\pcmn{她肚子很瘦。}\end{exemple}
\begin{exemple}\pjya{jiɕqha tɯrme ɯ-ŋga ɯ-tɯ-xtɕi kɯ xʂɤxʂɤt ɲɯ-pa}\hspace{5pt}\pcmn{那个人衣服穿得很紧}\end{exemple}
\begin{exemple}\pjya{kɯ-xtshɯm jnom ci xʂɤxʂɤt ɲɯ-ŋu}\hspace{5pt}\pcmn{又细又软}\end{exemple}
\begin{exemple}\pjya{ɯ-phoŋbu ra xʂɤxʂɤt ʑo kɯ-pa ci ɲɯ-ŋu}\hspace{5pt}\pcmn{他身材苗条}\end{exemple}\relationsémantique{参考}{\lien{ⓔχʂɤχʂɤt}{χʂɤχʂɤt}}\end{entrée}

\begin{entrée}{xtaŋxtaŋ}{}{ⓔxtaŋxtaŋ} 
\classe{ideo.2} 
\begin{définition}\pfra{gonflé}\end{définition}
\begin{définition}\pcmn{形容胀得很鼓的样子}\end{définition}
\begin{exemple}\pjya{ɯ-xtu xtaŋxtaŋ ʑo ɲɯ-nɤmbɤβ}\hspace{5pt}\pcmn{他肚子胀得很鼓}\end{exemple}\end{entrée}

\begin{entrée}{xtɤβxtɤβ}{}{ⓔxtɤβxtɤβ} 
\classe{idph.2} 
\begin{définition}\pfra{épais et court}\end{définition}
\begin{définition}\pcmn{形容粗而短的样子}\end{définition}
\begin{exemple}\pjya{tɤ-pɤtso ɯ-mi ɲɯ-tshu xtɤβxtɤβ ʑo}\hspace{5pt}\pcmn{小伙子的腿又粗又短}\end{exemple}\end{entrée}

\begin{entrée}{xtɤqa}{}{ⓔxtɤqa} 
\classe{n} 
\begin{définition}\pfra{bas-ventre}\end{définition}
\begin{définition}\pcmn{小肚子}\end{définition}\relationsémantique{参考}{\lien{ⓔtɯ-xtu}{tɯ-xtu}}\relationsémantique{参考}{\lien{ⓔtɯ-qa}{tɯ-qa}}\end{entrée}

\begin{entrée}{xtɤtshɤt}{}{ⓔxtɤtshɤt} 
\classe{n} 
\begin{définition}\pfra{contrôle de son appétit}\end{définition}
\begin{définition}\pcmn{饮食节制有度}\end{définition}
\begin{exemple}\pjya{tsuku lu-βzi-nɯ tɕe, cha kɤ-tshi lu-ɣɤtɕhom-nɯ tɕe, nɯ-ɣi ra kɯ tú-wɣ-nɤmqe-nɯ tɕe, ``nɤ-xtɤtshɤt ɯ-tɯ-me nɯ" tu-ti-nɯ ɲɯ-ŋu}\hspace{5pt}\pcmn{有些人喝酒喝太多,醉了,他们的家人骂他们说:“你完全是饮食无度啊”}\end{exemple}\relationsémantique{参考}{\lien{ⓔtɯ-xtu}{tɯ-xtu}}\end{entrée}

\begin{entrée}{xtɕɤr}{}{ⓔxtɕɤr} 
\classe{vt} \paradigme{dir}{kɤ-}\paradigme{dir}{tɤ-}
\begin{définition}\pfra{attacher}\end{définition}
\begin{définition}\pcmn{系}\end{définition}
\begin{exemple}\pjya{a-xtsa tɤ-xtɕar-a (ɯ-ri kɤ-lat-a)}\hspace{5pt}\pcmn{我系了鞋带}\end{exemple}
\begin{exemple}\pjya{tɤ-fkɯm kɤ-xtɕɤr}\hspace{5pt}\pcmn{你把口袋系一下}\end{exemple}\relationsémantique{参考}{\lien{ⓔxtsɤxtɕɤr}{xtsɤxtɕɤr}}\relationsémantique{参考}{\lien{ⓔmthɯxtɕɤr}{mthɯxtɕɤr}}\end{entrée}

\begin{entrée}{xtɕi}{}{ⓔxtɕi} 
\classe{vs} \paradigme{dir}{nɯ-}\paradigme{dir}{pɯ-}\paradigme{dir}{nɯ-}
\begin{définition}\pfra{petit}\end{définition}
\begin{définition}\pcmn{小}\end{définition}
\begin{définition}\pfra{rendre petit}\end{définition}
\begin{définition}\pcmn{弄小}\end{définition}
\begin{exemple}\pjya{wuma ɲɯ-xtɕi}\hspace{5pt}\pcmn{很小}\end{exemple}
\begin{exemple}\pjya{ɯ-phoŋbu ɲɯ-xtɕi}\hspace{5pt}\pcmn{他身体很小}\end{exemple}
\begin{exemple}\pjya{ɯ-lɯz kɯ-xtɕi}\hspace{5pt}\pcmn{他很年轻}\end{exemple}
\begin{exemple}\pjya{@shouji ɯ-skɤt ɲɯ-xtɕi tɕe, khro mɯ́j-mtsham-a}\hspace{5pt}\pcmn{手机的声音很低,我听不见}\end{exemple}
\begin{exemple}\pjya{nɤ-skɤt pɯ-xtɕi}\hspace{5pt}\pcmn{你的声音变小了}\end{exemple}
\begin{exemple}\pjya{nɤʑo nɤ-mu pɯ-xtɕi tɕe ɯ-pɯ́-mpɕɤr?}\hspace{5pt}\pcmn{你母亲小的时候漂亮吗?}\end{exemple}
\begin{exemple}\pjya{ɲɯ-dɤn tsa ɕti ri, pɯ-ɣɤxtɕi-t-a}\hspace{5pt}\pcmn{有点多,我弄小了}\end{exemple}\relationsémantique{反义词}{\lien{ⓔwxti}{wxti}}\relationsémantique{参考}{\lien{ⓔaxtɕɯxte}{axtɕɯxte}}
\begin{sous-entrée}{kɯxtɕɯxtɕi}{ⓔxtɕiⓝkɯxtɕɯxtɕi}
\begin{définition}\pfra{un peu}\end{définition}
\begin{définition}\pcmn{一点}\end{définition}
\begin{exemple}\pjya{ɯʑo kɯ kɯ-xtɕɯ-xtɕi ma na-nɤma me}\hspace{5pt}\pcmn{他只做了一点点}\end{exemple}\end{sous-entrée}

\begin{sous-entrée}{ɣɤxtɕi}{ⓔxtɕiⓝɣɤxtɕi} 
\classe{vt} \end{sous-entrée}

\end{entrée}

\begin{entrée}{xtɕɯxte}{}{ⓔxtɕɯxte} 
\classe{n} 
\begin{définition}\pfra{taille}\end{définition}
\begin{définition}\pcmn{大小}\end{définition}\relationsémantique{参考}{\lien{ⓔxtɕi}{xtɕi}}\relationsémantique{参考}{\lien{ⓔmɯxte}{mɯxte}}\relationsémantique{参考}{\lien{ⓔaxtɕɯxte}{axtɕɯxte}}\end{entrée}

\begin{entrée}{xthom}{}{ⓔxthom} 
\classe{vt}  
\grammaire{refl} \paradigme{dir}{nɯ-}\paradigme{dir}{lɤ-}
\begin{définition}\pfra{poser horizontalement}\end{définition}
\begin{définition}\pcmn{横着放;放平}\end{définition}
\begin{exemple}\pjya{ɕoŋtɕa na-xthom}\hspace{5pt}\pcmn{他把木料横着放了}\end{exemple}
\begin{exemple}\pjya{ɯ-ndɤcu @luyinji na-xthom}\hspace{5pt}\pcmn{他把录音机放平了}\end{exemple}
\begin{exemple}\pjya{laʁjɯɣ pɯ-xthom-a}\hspace{5pt}\pcmn{我把棍子放平了}\end{exemple}\relationsémantique{参考}{\lien{ⓔndom}{ndom}}
\begin{sous-entrée}{axthom}{ⓔxthomⓝaxthom} 
\classe{vi}  
\grammaire{pass} 
\begin{définition}\pfra{être posé}\end{définition}
\begin{définition}\pcmn{横放着}\end{définition}
\begin{exemple}\pjya{nɯre ri axthom tɕe ata ɕti}\hspace{5pt}\pcmn{在那里放着}\end{exemple}\end{sous-entrée}

\begin{sous-entrée}{ʑɣɤxthom}{ⓔxthomⓝʑɣɤxthom} 
\classe{vi} \end{sous-entrée}

\begin{définition}\pfra{s'allonger horizontalement}\end{définition}
\begin{définition}\pcmn{躺下}\end{définition}
\begin{exemple}\pjya{nɯ ɕɯmɯma ʑo ɯ-thoʁ nɯ tɕu lo-ʑɣɤxthom}\hspace{5pt}\pcmn{他马上就躺在地上}\end{exemple}\end{entrée}

\begin{entrée}{xtsu}{}{ⓔxtsu} 
\classe{vi} \paradigme{dir}{nɯ-}
\begin{définition}\pfra{fermenter}\end{définition}
\begin{définition}\pcmn{发酵}\end{définition}
\begin{exemple}\pjya{tɯ-ɣli ɲo-xtsu}\hspace{5pt}\pcmn{肥料发酵了}\end{exemple}
\begin{exemple}\pjya{cha ɲo-xtsu}\hspace{5pt}\pcmn{酒发酵了}\end{exemple}
\begin{exemple}\pjya{ɯ-sɯm rɯwɯrawi ɲɯ-xtsu}\hspace{5pt}\pcmn{心情很烦乱}\end{exemple}
\begin{exemple}\pjya{rzoŋlu ʑo ɲɯ-xtsu}\hspace{5pt}\pcmn{忙得不可开交}\end{exemple}\relationsémantique{参考}{\lien{ⓔsɯxtsuⓗ2}{sɯxtsu₂}}\end{entrée}

\begin{entrée}{xtsɤɕna}{}{ⓔxtsɤɕna} 
\classe{n} 
\begin{définition}\pfra{pointe recourbée des bottes}\end{définition}
\begin{définition}\pcmn{鞋子钩起的顶端}\end{définition}
\begin{exemple}\pjya{xtsɤɕna nɯ tɯ-xtsa ɣɯ ɯ-ʁɤri tɯ-ɕna kɯ-fse tɯ-kɯ-ŋgɤɣ nɯ ŋu tɕe xtsɤrkɯ ɯ-ʁɤri ku-kɯ-ɤndɯndo ɯ-stu nɯ ŋu, ɯ-qhuchu lu-kɯ-ɣe komɤr jaʁndzu χsɯm jamar kɯ-rɟum, tɯ-tɣa ro ro kɯ-rɲɟi nɯ li xtsɤɕna rmi.}\hspace{5pt}\pcmn{\lien{ⓔxtsɤɕna}{xtsɤɕna}(鞋鼻子)是鞋子前面钩着的部分,是\lien{ⓔxtsɤrkɯ}{xtsɤrkɯ}(鞋边)的接头部分,后面有一块三指宽、一拃多长的红皮子,这块皮子叫\lien{ⓔxtsɤɕna}{xtsɤɕna}(鞋鼻子)。}\end{exemple}\end{entrée}

\begin{entrée}{xtsɤku}{}{ⓔxtsɤku} 
\classe{n} 
\begin{définition}\pfra{partie de la botte qui recouvre les mollets}\end{définition}
\begin{définition}\pcmn{靴筒(靴子盖小腿的部分)}\end{définition}\end{entrée}

\begin{entrée}{xtsɤpɤl}{}{ⓔxtsɤpɤl} 
\classe{n} 
\begin{définition}\pfra{chaussure (ne dépasse pas la cheville)}\end{définition}
\begin{définition}\pcmn{鞋子(没有筒)}\end{définition}\end{entrée}

\begin{entrée}{xtsɤqar}{}{ⓔxtsɤqar} 
\classe{n} 
\begin{définition}\pfra{endroit où la semelle et la chaussure sont cousues ensemble}\end{définition}
\begin{définition}\pcmn{鞋底和鞋子的接头部分}\end{définition}\end{entrée}

\begin{entrée}{xtsɤqarmbe}{}{ⓔxtsɤqarmbe} 
\classe{n} 
\begin{définition}\pfra{vieille semelle de chaussure en cuir}\end{définition}
\begin{définition}\pcmn{被扔掉的皮鞋底}\end{définition}\end{entrée}

\begin{entrée}{xtsɤqɤr}{}{ⓔxtsɤqɤr} 
\classe{n} 
\begin{définition}\pfra{bordure de la semelle}\end{définition}
\begin{définition}\pcmn{鞋底子的周边}\end{définition}\relationsémantique{参考}{\lien{ⓔtɯ-xtsa}{tɯ-xtsa}}\end{entrée}

\begin{entrée}{xtsɤqɤrqaβ}{}{ⓔxtsɤqɤrqaβ} 
\classe{n} 
\begin{définition}\pfra{aiguille pour réparer les chaussures}\end{définition}
\begin{définition}\pcmn{补鞋子的针}\end{définition}\relationsémantique{参考}{\lien{}{qaβ}}\end{entrée}

\begin{entrée}{xtsɤrkɯ}{}{ⓔxtsɤrkɯ} 
\classe{n} 
\begin{définition}\pfra{partie de la chaussure recouvrant les pieds}\end{définition}
\begin{définition}\pcmn{鞋边(鞋子、靴子盖脚的部分)}\end{définition}
\begin{exemple}\pjya{xtsɤrkɯ nɯ tɯ-xtsa χchoʁe ɯ-pa mŋulɤn ɯ-sɤ-tshoʁ nɯ ŋu, xtsɤrkɯ ɯ-taʁ nɯ tɕu xtsɤku pjɯ́-wɣ-tshoʁ, ɯ-ŋgɯ ɯ-pɕoʁ nɯ tɕu xtsɤkɤŋgɯ pjɯ-tu ra, xtsɤkɤŋgɯ nɯ tɯ-ŋgar pjɯ-ŋu ra, xtsɤku nɯ cɤndʐi nɯ maʁ nɤ ɕkom ndʐi, nɯ maʁ nɤ qartshɤndʐi kɯ-mba pjɯ-ŋu ɲɯ-ra. tɯ-xtsa nɯ ɯ-qhu tɕe ɯ-srɯβzɤn pjɯ-tu ɲɯ-ra, tɯ-xtsa nɯ tú-wɣ-ŋga tɕe wuma ʑo mpja. tɯ-xtsa ʁnɯ-tɯphu tu, tɯ-tɯphu nɯ konaʁxtsa rmi, ɯ-xtsɤrkɯ nɯ komɤr thɯ-kɤ-sɯɣ-ɲaʁ ŋu, mɤʑɯ tɯ-tɯphu nɯ koscaxtsa rmi, tɕe ɯ-xtsɤrkɯ nɯ mɯ-pɯ-kɤ-sɯxtshwi ŋu.}\hspace{5pt}\pcmn{鞋帮的左右两边的下侧装鞋底,上面装靴筒,在内层要有内衬,是用羊毛织出来的布料。靴筒是用麝香鹿皮、麂子皮或是比较薄的鹿子皮做成的。在鞋后面要有\lien{ⓔsrɯβzɤn}{srɯβzɤn}(在缝合处夹的一块布料)。这种鞋子穿起来很暖和。鞋子分成两种,一种叫\lien{}{konaʁ xtsa},鞋帮是染成黑色的红皮子,另一种叫\lien{}{kosca xtsa},鞋帮是根本没有染色的皮子。}\end{exemple}\end{entrée}

\begin{entrée}{xtsɤrtɯm}{}{ⓔxtsɤrtɯm} 
\classe{n} 
\begin{définition}\pfra{botte en cuir}\end{définition}
\begin{définition}\pcmn{皮靴子}\end{définition}
\begin{exemple}\pjya{xtsɤrtɯm nɯ ɯ-xtsɤku nɯ tɯ-ndʐi ŋu tɕe, ɯ-ɕnɤku ɯ-stu nɯ lú-wɣ-phaʁ tɕe, nɯ ɯ-stu nɯ tɯ-ndʐi kɤ-βzɯχsɯm tú-wɣ-rku tɕe, ɯ-xtsɤrkɯ ɯ-ʑoz kɯ-me. xtsɤrtɯm nɯ li ʁnɯ-tɯphu tu tɕe, tɯ-tɯphu nɯ ɯ-pɕi tɯ-ndʐi ŋu, ɯ-ŋgɯ tɯ-ŋgar ŋu, li ci tɯ-tɯphu nɯ tɤ-mbextsa rmi, ɯ-pɕi nɯ tɯ-rtɯthɯ ŋu, ɯ-ŋgɯ nɯ li tɯ-ŋgar ŋu. ki ʁnɯ-tɯphu ki tɯ-xtsa ni ndʑi-tʂɯβ naχtɕɯɣ, ndʑi-mŋulɤn ra.}\hspace{5pt}\pcmn{皮靴子的靴筒是一块皮子,把那块皮子的前端(对着脚背的部分)破一个口,那里装上三角形的皮子,没有另外的鞋帮。这种靴子有两种,一种外层是皮子,内层是羊毛布,另一种叫 \lien{ⓔtɤmbextsa}{tɤmbextsa},外层是麻布,里面还是羊毛布。这两种的缝法一样,都需要鞋底。}\end{exemple}\end{entrée}

\begin{entrée}{xtsɤsoʁ}{}{ⓔxtsɤsoʁ} 
\classe{n} 
\begin{définition}\pfra{semelle en paille}\end{définition}
\begin{définition}\pcmn{鞋垫}\end{définition}\end{entrée}

\begin{entrée}{xtsɤxtɕɤr}{}{ⓔxtsɤxtɕɤr} 
\classe{n} 
\begin{définition}\pfra{lacet}\end{définition}
\begin{définition}\pcmn{鞋带}\end{définition}
\begin{exemple}\pjya{xtsɤxtɕɤr na-nɯ-rtɤβ}\hspace{5pt}\pcmn{他系了鞋带}\end{exemple}
\begin{exemple}\pjya{nɤ-xtsɤxtɕɤr ɲɤ-nɯ-ɬoʁ nɤ!}\hspace{5pt}\pcmn{你的鞋带解开了}\end{exemple}
\begin{exemple}\pjya{nɤ-xtsɤxtɕɤr kɤ-lɤt ma tɯ́-wɣ-tʂaβ}\hspace{5pt}\pcmn{你把鞋带系上,不然会摔跤的}\end{exemple}\end{entrée}

\begin{entrée}{xtshɯm}{}{ⓔxtshɯm} 
\classe{vs} \paradigme{dir}{nɯ-}\paradigme{dir}{thɯ-}
\begin{définition}\pfra{fin}\end{définition}
\begin{définition}\pcmn{细(直径)}\end{définition}
\begin{définition}\pfra{rendre fin}\end{définition}
\begin{définition}\pcmn{使变细}\end{définition}
\begin{exemple}\pjya{tɤ-ri kɯ-xtshɯm}\hspace{5pt}\pcmn{细线}\end{exemple}
\begin{exemple}\pjya{si kɯ-xtshɯm}\hspace{5pt}\pcmn{很细的树}\end{exemple}\relationsémantique{反义词}{\lien{ⓔjpum}{jpum}}
\begin{sous-entrée}{ɣɤxtshɯm}{ⓔxtshɯmⓝɣɤxtshɯm} 
\classe{vt} \end{sous-entrée}

\end{entrée}

\begin{entrée}{xtsɯ}{}{ⓔxtsɯ} 
\classe{vt} \paradigme{dir}{pɯ-}
\begin{définition}\pfra{piler}\end{définition}
\begin{définition}\pcmn{捣碎,砸碎,碾磨}\end{définition}
\begin{exemple}\pjya{tʂha pa-xtsɯ}\hspace{5pt}\pcmn{他把马茶捣碎了}\end{exemple}
\begin{exemple}\pjya{ɕom pa-xtsɯ}\hspace{5pt}\pcmn{他打了铁}\end{exemple}
\begin{exemple}\pjya{hajtsu pa-xtsɯ}\hspace{5pt}\pcmn{他把辣椒捣碎了}\end{exemple}
\begin{exemple}\pjya{mɯzi pa-xtsɯ}\hspace{5pt}\pcmn{他把黑火药捣碎了}\end{exemple}\end{entrée}

\begin{entrée}{xtsɯɣ}{}{ⓔxtsɯɣ} 
\classe{vt} \paradigme{dir}{tɤ-}\paradigme{dir}{tɤ-}
\begin{définition}\pfra{toucher, atteindre, frapper}\end{définition}
\begin{définition}\pcmn{打中}\end{définition}
\begin{définition}\pfra{atteindre avec}\end{définition}
\begin{définition}\pcmn{用……射中、打中}\end{définition}
\begin{exemple}\pjya{ta-xtsɯɣ}\hspace{5pt}\pcmn{他打中了}\end{exemple}
\begin{exemple}\pjya{tɤ-xtsɯɣa}\hspace{5pt}\pcmn{我打中了}\end{exemple}
\begin{exemple}\pjya{(pri) to-xtsɯɣ ri jo-nɯɕe}\hspace{5pt}\pcmn{虽然我打中了熊,它却回去了(逃走了 )}\end{exemple}
\begin{exemple}\pjya{rdɤstaʁ kɯ tɤ́-wɣ-sɯxtsɯɣ-a}\hspace{5pt}\pcmn{他扔石头打中了我}\end{exemple}
\begin{exemple}\pjya{ɯ-zgrɯ kɯ tɤ́-wɣ-sɯxtsɯɣ-a}\hspace{5pt}\pcmn{他用肘打中了我}\end{exemple}
\begin{exemple}\pjya{tɯdi kɯ tɤ́-wɣsɯxtsɯɣ-a}\hspace{5pt}\pcmn{他射箭射中了我}\end{exemple}
\begin{exemple}\pjya{kɯki mbrɯtɕɯ ɲɯ-mtɕoʁ tɕe, nɤ-jaʁ tɯ-sɯxtsɯɣ ma}\hspace{5pt}\pcmn{刀很锋利,小心不要割到手}\end{exemple}
\begin{sous-entrée}{sɯxtsɯɣ}{ⓔxtsɯɣⓝsɯxtsɯɣ} 
\classe{vt}  
\grammaire{caus} \end{sous-entrée}

\end{entrée}

\begin{entrée}{xtsɯsna}{}{ⓔxtsɯsna} 
\classe{n} 
\begin{définition}\pfra{toute sorte de}\end{définition}
\begin{définition}\pcmn{各种各样}\end{définition}\end{entrée}

\begin{entrée}{xtʂoŋxtʂoŋ}{}{ⓔxtʂoŋxtʂoŋ} 
\classe{idph.2} 
\begin{définition}\pfra{mou et gonflé}\end{définition}
\begin{définition}\pcmn{形容饱满而软的样子}\end{définition}
\begin{exemple}\pjya{@qiqiu nɯ qale xtʂoŋxtʂoŋ ʑo chɤ-mtshɤt}\hspace{5pt}\pcmn{气球吹得胀鼓鼓的}\end{exemple}\end{entrée}

\begin{entrée}{xtɯrkɯ}{}{ⓔxtɯrkɯ} 
\classe{n} 
\begin{définition}\pfra{cordes pour attacher la charrue au joug}\end{définition}
\begin{définition}\pcmn{牛皮绳【纤绳】}\end{définition}
\begin{exemple}\pjya{xtɯrkɯ nɯ mbɣɤru cho stuxsi ndʑi-kɯ-sɤthɤri tɯ-ndʐi kɯ-rɟum kɯ-zri tsa nɯ ŋu}\hspace{5pt}\pcmn{\lien{ⓔxtɯrkɯ}{xtɯrkɯ}是连接犁杆和牛轭的又宽又长的牛皮绳}\end{exemple}\end{entrée}

\begin{entrée}{xtɯrɲɟi}{}{ⓔxtɯrɲɟi} 
\classe{n} 
\begin{définition}\pfra{longueur}\end{définition}
\begin{définition}\pcmn{长度}\end{définition}\relationsémantique{参考}{\lien{ⓔxtɯtⓗ1}{xtɯt₁}}\relationsémantique{参考}{\lien{ⓔrɲɟi}{rɲɟi}}\end{entrée}

\begin{entrée}{xtɯt}{₂}{ⓔxtɯtⓗ2} 
\classe{n} 
\begin{définition}\pfra{chat sauvage}\end{définition}
\begin{définition}\pcmn{野猫}\end{définition}
\begin{exemple}\pjya{xtɯt nɯ lɯlu cho kɯ-naχtɕɯ-χtɕɯɣ ʑo ŋu, tɕeri xtɯt nɯ sɯŋgɯ, praʁ ɯ-rchɤβ ku-rɤʑi ŋu, ɯʑo sɤznɤ rɯdaʁ kɯ-xtɕi ra tu-ndze ɲɯ-ŋu, lɯlu sɤznɤ kɯ-xtɕɯ-xtɕi ɲɯ-wxti cho ɲɯ-rkaŋ. ɯ-mdoʁ nɯ kɯ-pɣi tɕe ɯ-taʁ kɯ-ɲaʁ kɯ-ɤkhra ɲɯ-ŋu. ci ci tɕe lɯlu cho ɲawa tu-βzu-ndʑi ɲɯ-ŋgrɤl.}\hspace{5pt}\pcmn{野猫和家猫一模一样,但是野猫生活在森林里和岩洞里,吃比自己小的动物,比家猫大一些,强一些。颜色是灰色,上面有黑色的斑纹。有时候会和家猫交配。}\end{exemple}\end{entrée}

\begin{entrée}{xtɯt}{₁}{ⓔxtɯtⓗ1} 
\classe{vs} \paradigme{dir}{nɯ-}\paradigme{dir}{nɯ-}\paradigme{dir}{\_}
\begin{définition}\pfra{court}\end{définition}
\begin{définition}\pcmn{短}\end{définition}
\begin{définition}\pfra{raccourcir}\end{définition}
\begin{définition}\pcmn{弄短}\end{définition}
\begin{exemple}\pjya{ɯ-phoŋbu ɲɯ-xtɯt}\hspace{5pt}\pcmn{他的身体很小}\end{exemple}
\begin{exemple}\pjya{ɕoŋtɕa ɲɯ-xtɯt}\hspace{5pt}\pcmn{木料很短}\end{exemple}
\begin{exemple}\pjya{ɯ-ŋga to-ɣɤxtɯt}\hspace{5pt}\pcmn{他衣服穿得很短}\end{exemple}
\begin{exemple}\pjya{nɤʑo tɕheme tɯ-ɕti tɕe, nɤ-ŋga kɤ-ɣɤxtɯt mɤ-ra}\hspace{5pt}\pcmn{你是女孩子,衣服不能穿得太短}\end{exemple}
\begin{exemple}\pjya{aʑo mɯ-to-rɯndzaŋspa-a tɕe, tɯ-ŋga kɤ-qrɯ ɲɤ-ɣɤxtɯt-a}\hspace{5pt}\pcmn{我不小心把衣服裁得很短}\end{exemple}\relationsémantique{参考}{\lien{ⓔnɤxtɯt}{nɤxtɯt}}\relationsémantique{反义词}{\lien{ⓔzri}{zri}}\relationsémantique{反义词}{\lien{ⓔrɲɟi}{rɲɟi}}\relationsémantique{参考}{\lien{ⓔxtɯrɲɟi}{xtɯrɲɟi}}
\begin{sous-entrée}{ɣɤxtɯt}{ⓔxtɯtⓗ1ⓝɣɤxtɯt} 
\classe{vt}  
\grammaire{caus} \end{sous-entrée}

\begin{sous-entrée}{zɣɤxtɯt}{ⓔxtɯtⓗ1ⓝzɣɤxtɯt} 
\classe{vt} \paradigme{dir}{nɯ-}
\begin{définition}\pfra{raccourcir avec, faire raccourcir}\end{définition}
\begin{définition}\pcmn{使弄短;用……弄短}\end{définition}
\begin{exemple}\pjya{a-ŋga ɲɯ-zri tɕe, nɯ-zɣɤxtɯt-a}\hspace{5pt}\pcmn{我的衣服太长,就请人弄短了}\end{exemple}\end{sous-entrée}

\begin{sous-entrée}{sɯxtɯt}{ⓔxtɯtⓗ1ⓝsɯxtɯt} 
\classe{vt}  
\grammaire{caus} 
\begin{définition}\pfra{raccourcir}\end{définition}
\begin{définition}\pcmn{弄短}\end{définition}
\begin{exemple}\pjya{tɤ-rʑaʁ kɤ-sɯxtɯt khɯ}\hspace{5pt}\pcmn{可以把时间缩短}\end{exemple}\end{sous-entrée}

\end{entrée}

\begin{entrée}{xɯβxɯβ}{}{ⓔxɯβxɯβ} 
\classe{idph.2} \sens{1}
\begin{définition}\pfra{rose}\end{définition}
\begin{définition}\pcmn{粉红状}\end{définition}\sens{2}
\begin{définition}\pfra{chaud}\end{définition}
\begin{définition}\pcmn{形容(天气) 暖暖的}\end{définition}
\begin{exemple}\pjya{kɯ-ɣɯrni xɯβxɯβ ci ɲɯ-ŋu}\hspace{5pt}\pcmn{是粉红色的}\end{exemple}
\begin{exemple}\pjya{tɯrme ɯ-rŋa kɯ-ɣɯrni xɯβxɯβ ci ɲɯ-ŋu}\hspace{5pt}\pcmn{那个人的脸是粉红的}\end{exemple}
\begin{exemple}\pjya{kha jɤ-azɣɯt-a tɕe, xɯβxɯβ ɲɯ-mpja}\hspace{5pt}\pcmn{我到家里了,很暖}\end{exemple}
\begin{sous-entrée}{xɯβ}{ⓔxɯβxɯβⓢ2ⓝxɯβ} 
\classe{idph.1} 
\begin{exemple}\pjya{ɯ-mbrɯ xɯβ ʑo tɤ-ŋgɯ}\hspace{5pt}\pcmn{他一下子就生气了}\end{exemple}\end{sous-entrée}

\begin{sous-entrée}{xɯβnɤxɯβ}{ⓔxɯβxɯβⓢ2ⓝxɯβnɤxɯβ} 
\classe{idph.3} 
\begin{définition}\pfra{douleur lancinante}\end{définition}
\begin{définition}\pcmn{一阵一阵地痛(没有出血)}\end{définition}
\begin{exemple}\pjya{a-βri xɯβnɤxɯβ ɲɯ-ti}\hspace{5pt}\pcmn{我身上一阵一阵地痛}\end{exemple}
\begin{exemple}\pjya{mtshalu kɯ kɤ́-wɣ-mtsɯɣ-a tɕe, a-βri xɯβnɤxɯβ ɲɯ-ti}\hspace{5pt}\pcmn{我被荨麻刺到了,一阵一阵地痛}\end{exemple}\end{sous-entrée}

\begin{sous-entrée}{xɯβnɤlɯβ}{ⓔxɯβxɯβⓢ2ⓝxɯβnɤlɯβ} 
\classe{idph.4} \end{sous-entrée}

\begin{sous-entrée}{xɯwɯwi}{ⓔxɯβxɯβⓢ2ⓝxɯwɯwi} 
\classe{idph.6} 
\begin{exemple}\pjya{xɯwɯwi ʑo ɲɯ-mpja}\hspace{5pt}\pcmn{慢慢地暖起来}\end{exemple}\end{sous-entrée}

\begin{sous-entrée}{ɣɤxɯβxɯβ}{ⓔxɯβxɯβⓢ2ⓝɣɤxɯβxɯβ} 
\classe{vi} 
\begin{exemple}\pjya{ɲɯ-mɤrtsaβ ɲɯ-ɣɤxɯβxɯβ}\hspace{5pt}\pcmn{很辣}\end{exemple}
\begin{exemple}\pjya{ɯ-mbrɯ ɲɯ-ɣɤxɯβxɯβ ʑo ɲɯ-ŋgɯ}\hspace{5pt}\pcmn{他很生气}\end{exemple}\end{sous-entrée}

\begin{sous-entrée}{sɤxɯβxɯβ}{ⓔxɯβxɯβⓢ2ⓝsɤxɯβxɯβ} 
\classe{vt} 
\begin{exemple}\pjya{qale ɲɯ-sɤxɯβxɯβ}\hspace{5pt}\pcmn{风在吹到脸上(很冷的感觉)}\end{exemple}\end{sous-entrée}

\end{entrée}

\begin{entrée}{xɯchɯcho}{}{ⓔxɯchɯcho} 
\classe{intj} 
\begin{définition}\pfra{soupir de fatigue}\end{définition}
\begin{définition}\pcmn{表示自己很累的感叹声}\end{définition}
\begin{exemple}\pjya{nɯ-kɯ-ɲat tɕe xɯchɯcho tu-kɯ-ti ŋu}\hspace{5pt}\pcmn{累了就说“\lien{ⓔxɯchɯcho}{xɯchɯcho}”}\end{exemple}\end{entrée}

\begin{entrée}{xɯŋxɯŋ}{}{ⓔxɯŋxɯŋ} 
\classe{idph.2} 
\begin{définition}\pfra{claire (pièce)}\end{définition}
\begin{définition}\pcmn{形容(房间)明亮}\end{définition}
\begin{exemple}\pjya{kha ɲɯ-fsoʁ kɯ xɯŋxɯŋ ʑo}\hspace{5pt}\pcmn{房子很明亮}\end{exemple}
\begin{exemple}\pjya{ɲɯ-qarŋe xɯŋxɯŋ ʑo}\hspace{5pt}\pcmn{很黄}\end{exemple}
\begin{exemple}\pjya{tɤŋe ko-ntɕhɤr, xɯŋxɯŋ ʑo kha ɲɯ-fsoʁ}\hspace{5pt}\pcmn{太阳发光,(照得)房间很明亮}\end{exemple}
\begin{sous-entrée}{xɯŋɯŋi}{ⓔxɯŋxɯŋⓝxɯŋɯŋi} 
\classe{idph.7} 
\begin{exemple}\pjya{tɤŋe xɯŋɯŋi ʑo pɯ-ɣe}\hspace{5pt}\pcmn{太阳慢慢地下山了,很明亮}\end{exemple}
\begin{exemple}\pjya{tɤŋe xɯŋɯŋi ʑo to-nɯ-ɬoʁ}\hspace{5pt}\pcmn{太阳慢慢地出来了,很明亮}\end{exemple}\relationsémantique{参考}{\lien{ⓔʂɯŋʂɯŋ}{ʂɯŋʂɯŋ}}\end{sous-entrée}

\end{entrée}

\begin{entrée}{xɯrxɯr}{}{ⓔxɯrxɯr} 
\classe{idph.2} 
\begin{définition}\pfra{rond}\end{définition}
\begin{définition}\pcmn{圆形}\end{définition}
\begin{exemple}\pjya{tɤŋe xɯrxɯr ʑo ɲɯ-ɤrtɯm}\hspace{5pt}\pcmn{太阳是圆的}\end{exemple}
\begin{exemple}\pjya{mbrɤsɤm xɯrxɯr ɲɯ-ɤrtɯm}\hspace{5pt}\pcmn{晒粮食的簸箕是圆形的}\end{exemple}
\begin{sous-entrée}{xɯrnɤxɯr}{ⓔxɯrxɯrⓝxɯrnɤxɯr} 
\classe{idph.3} 
\begin{définition}\pfra{qui tourne}\end{définition}
\begin{définition}\pcmn{在转动}\end{définition}\end{sous-entrée}

\begin{sous-entrée}{xɯrinɤxɯri}{ⓔxɯrxɯrⓝxɯrinɤxɯri} 
\classe{idph.8} 
\begin{définition}\pfra{qui tourne vite}\end{définition}
\begin{définition}\pcmn{旋转得很快的样子}\end{définition}
\begin{exemple}\pjya{xɯrinɤxɯri ʑo ko-mtɕɯr}\hspace{5pt}\pcmn{转得很快了}\end{exemple}\end{sous-entrée}

\begin{sous-entrée}{ɣɤxɯrxɯr}{ⓔxɯrxɯrⓝɣɤxɯrxɯr} 
\classe{vi} 
\begin{définition}\pfra{qui tourne}\end{définition}
\begin{définition}\pcmn{在转动}\end{définition}
\begin{exemple}\pjya{mkhɯrlu ɲɯ-ɣɤxɯrxɯr ɲɯ-mtɕɯr}\hspace{5pt}\pcmn{轮子在转动}\end{exemple}\relationsémantique{参考}{\lien{ⓔtɯ-tɤxɯr}{tɯ-tɤxɯr}}\relationsémantique{同义词}{\lien{ⓔsɯrsɯr}{sɯrsɯr}}\end{sous-entrée}

\end{entrée}

\begin{entrée}{xwɤrnɤxwɤr/\variante{xwaranɤxwara}}{}{ⓔxwɤrnɤxwɤr} 
\classe{idph.3} 
\begin{définition}\pfra{qui tourne vite}\end{définition}
\begin{définition}\pcmn{形容转得很快的样子}\end{définition}
\begin{exemple}\pjya{xwɤrxwɤr nɤ xwɤrxwɤr ʑo ɲɯ-mtɕɯr}\hspace{5pt}\pcmn{哗哗哗地飞快旋转}\end{exemple}\end{entrée}

\newpage\caractère{χ}

\begin{entrée}{χajaŋ}{}{ⓔχajaŋ} 
\classe{n} 
\begin{définition}\pfra{aluminium}\end{définition}
\begin{définition}\pcmn{铝}\end{définition}\étymologie{ha.jaŋ}\end{entrée}

\begin{entrée}{χajχaj}{}{ⓔχajχaj} 
\classe{idpd.2} 
\begin{définition}\pfra{attendre (sans bouger)}\end{définition}
\begin{définition}\pcmn{呆呆地瞪着}\end{définition}
\begin{exemple}\pjya{dɯxpa ma χajχaj ʑo ɲɯ́-wɣ-nɤjo-a}\hspace{5pt}\pcmn{我呆呆地盼着他}\end{exemple}\end{entrée}

\begin{entrée}{χaŋχaŋ}{}{ⓔχaŋχaŋ} 
\classe{idph.2} 
\begin{définition}\pfra{un peu orange}\end{définition}
\begin{définition}\pcmn{形容淡橘黄色}\end{définition}
\begin{exemple}\pjya{ɲɯ-qarŋe χaŋχaŋ ʑo}\hspace{5pt}\pcmn{是淡黄的}\end{exemple}
\begin{exemple}\pjya{tɯrmɯkha tɕe, prɤɲi χaŋχaŋ ɲɤ-k-ɤβzu-ci}\hspace{5pt}\pcmn{傍晚的时候,晚霞带有橘黄色}\end{exemple}\end{entrée}

\begin{entrée}{χawo}{}{ⓔχawo} 
\classe{intj} 
\begin{définition}\pfra{expression du regret, de l'espoir}\end{définition}
\begin{définition}\pcmn{表示惋惜、希望}\end{définition}
\begin{exemple}\pjya{χawo ʑo nɤ-ɕqhe a-nɯ-me kɯ!}\hspace{5pt}\pcmn{唉,真希望你的咳嗽会治好}\end{exemple}\end{entrée}

\begin{entrée}{χɤβ}{}{ⓔχɤβ} 
\classe{vt} \paradigme{dir}{kɤ-}\paradigme{dir}{lɤ-}
\begin{définition}\pfra{boire complètement}\end{définition}
\begin{définition}\pcmn{喝干;喝尽最后一滴}\end{définition}
\begin{exemple}\pjya{kɤ-χɤβ kɯ-fse ci ɲɯ-ɕti}\hspace{5pt}\pcmn{只剩下一点点}\end{exemple}
\begin{exemple}\pjya{a-sŋɯro lɤ-χaβ-a}\hspace{5pt}\pcmn{我吸了气}\end{exemple}
\begin{exemple}\pjya{cha kɤ-χaβ-a}\hspace{5pt}\pcmn{我把酒喝光了}\end{exemple}\end{entrée}

\begin{entrée}{χɤjnɤχɤj}{}{ⓔχɤjnɤχɤj} 
\classe{idph.3} 
\begin{définition}\pfra{essouflé}\end{définition}
\begin{définition}\pcmn{喘气}\end{définition}
\begin{exemple}\pjya{tɤ-rɟɯɣ-a tɕe χɤjnɤχɤj ʑo tɤ-tɯt-a pɯ-ra}\hspace{5pt}\pcmn{我跑得气喘吁吁的}\end{exemple}\relationsémantique{同义词}{\lien{ⓔχinɤχi}{χinɤχi}}\end{entrée}

\begin{entrée}{χɤlnɤχɤl}{}{ⓔχɤlnɤχɤl} 
\classe{idph.3} 
\begin{définition}\pfra{marcher d'un pas assuré}\end{définition}
\begin{définition}\pcmn{形容走路步伐稳健的样子}\end{définition}\end{entrée}

\begin{entrée}{χɤlχɤl}{}{ⓔχɤlχɤl} 
\classe{idph.2} \sens{1}
\begin{définition}\pfra{relâché, guéri, rassuré}\end{définition}
\begin{définition}\pcmn{形容捆绑以后解开的宽松感、痊愈、放心的感觉}\end{définition}
\begin{exemple}\pjya{khapa fsapaʁ pɯ-rɤʑi-nɯ ɕti tɕe, ɕɯ-nam-a pɯ-ŋu ri, khapa pɯ-azɣɯt-a tɕe, χɤlχɤl ʑo jo-ɕe-nɯ}\hspace{5pt}\pcmn{牲畜原来在楼下,我正要去赶它们的时候,它们就消失得无影无踪了}\end{exemple}
\begin{exemple}\pjya{jɯfɕɯr tɤ-pɤtso wuma ʑo pɯ-nɯzdɯɣ-a ri, jɤ-azɣɯt tɕe, a-sɯm χɤlχɤl ʑo ɲo-pa}\hspace{5pt}\pcmn{昨天我很担心小孩子,他到了之后我就放心了}\end{exemple}
\begin{exemple}\pjya{χɤlχɤl ʑo to-mna}\hspace{5pt}\pcmn{完全痊愈}\end{exemple}\sens{2}
\begin{définition}\pfra{disparaître}\end{définition}
\begin{définition}\pcmn{消失得无影无踪(比较嘈杂的人或者动物)}\end{définition}
\begin{exemple}\pjya{χɤlχɤl chɤ-k-ɤrɕo-ci}\hspace{5pt}\pcmn{完全用完了}\end{exemple}
\begin{exemple}\pjya{χɤlχɤl to-ɕkɯt}\hspace{5pt}\pcmn{他吃完了}\end{exemple}
\begin{exemple}\pjya{slama ra jɤ-anɯri-nɯ, sloχpɯn ɯ-rkɯ χɤlχɤl ʑo ɲɯ-pa}\hspace{5pt}\pcmn{当学生都走了之后,老师觉得心里空落落的}\end{exemple}
\begin{exemple}\pjya{a-kɯ-mŋɤm χɤlχɤl ʑo tɤ-pa}\hspace{5pt}\pcmn{我的病完全痊愈了}\end{exemple}\relationsémantique{参考}{\lien{ⓔɣɤχɤlχɤl}{ɣɤχɤlχɤl}}\end{entrée}

\begin{entrée}{χɤnku}{}{ⓔχɤnku} 
\classe{n} 
\begin{définition}\pfra{casserole en fer pour cuire la pâté des cochons}\end{définition}
\begin{définition}\pcmn{煮猪草的生铁锅}\end{définition}
\begin{exemple}\pjya{χɤnku nɯ tɯ-thɯ kɯ-wxti ci ŋu, ɯ-spa khru ɲɯ-ŋu, mba tɕe wxti ri mɤ-rʑi, paʁndza sɤ-sqa wuma ʑo pe. χɤnku nɯ ɯ-spa khro ɲɯ-ŋu tɕe, khru mɤ-ngɯt ma ɲɯ-ndoʁ tɕe kɤ-ta kɤ-mɟa ra ɣɯ-mdzoz tsa ra.}\hspace{5pt}\pcmn{\lien{ⓔχɤnku}{χɤnku}是一种大锅,用生铁铸成。因为很薄,所以大而轻,最适合于煮猪草。因为用生铁做成,所以不结实,很脆,所以放下拿起的时候要特别小心。}\end{exemple}\end{entrée}

\begin{entrée}{χɤnχɤn}{}{ⓔχɤnχɤn} 
\classe{idph.2} 
\begin{définition}\pfra{large et vide}\end{définition}
\begin{définition}\pcmn{形容又大又空的样子}\end{définition}
\begin{exemple}\pjya{χɤnχɤn ʑo ɲɯ-ɤz-nɤjo}\hspace{5pt}\pcmn{他呆呆地站在那里等着(望着一个方向不动)}\end{exemple}
\begin{exemple}\pjya{kha ɯ-tɯ-wxti kɯ χɤnχɤn ʑo ɲɯ-pa}\hspace{5pt}\pcmn{房子又大又空}\end{exemple}
\begin{exemple}\pjya{nɤ-khɯtsa ɯ-tɯ-wxti nɯ χɤnχɤn kɯ}\hspace{5pt}\pcmn{你的碗很大!}\end{exemple}
\begin{exemple}\pjya{χɤnχɤn ʑo ma-tɤ-tɯ-ʑɣɤstu}\hspace{5pt}\pcmn{不要傻乎乎的那样}\end{exemple}\relationsémantique{同义词}{\lien{ⓔχajχaj}{χajχaj}}\end{entrée}

\begin{entrée}{χɤpɤχɤle}{}{ⓔχɤpɤχɤle} 
\classe{n} 
\begin{définition}\pfra{extraverti}\end{définition}
\begin{définition}\pcmn{外向}\end{définition}
\begin{exemple}\pjya{nɯki tɯrme χɤpɤχɤle ci ɯ-qhoχpa kɤ-rku kɯ-me ci ɕti}\hspace{5pt}\pcmn{这个人比较外向,有什么事都不会放在心里的。}\end{exemple}\end{entrée}

\begin{entrée}{χcha}{}{ⓔχcha} 
\classe{n} 
\begin{définition}\pfra{droite}\end{définition}
\begin{définition}\pcmn{右边}\end{définition}\end{entrée}

\begin{entrée}{χchoʁe}{}{ⓔχchoʁe} 
\classe{adv} \paradigme{emphatic}{χchɯχchoʁɯʁe}
\begin{définition}\pfra{à droite et à gauche}\end{définition}
\begin{définition}\pcmn{左右}\end{définition}
\begin{exemple}\pjya{jɯfɕɯr a-jaʁ χchoʁe ʑo laχtɕha tɤ-ndo-t-a}\hspace{5pt}\pcmn{昨天我两只手都拿了东西。}\end{exemple}\end{entrée}

\begin{entrée}{χcoŋkroŋ}{}{ⓔχcoŋkroŋ} 
\classe{n} 
\begin{définition}\pfra{en tailleur}\end{définition}
\begin{définition}\pcmn{盘腿}\end{définition}
\begin{exemple}\pjya{χcoŋkroŋ ɲɤ-βzu}\hspace{5pt}\pcmn{他盘腿坐了}\end{exemple}
\begin{exemple}\pjya{tɤ-tɕɯ kɯ χcoŋkroŋ ɲɯ-βze ŋgrɤl, tɕheme kɯ ndzɯpe ɲɯ-βze ŋgrɤl}\hspace{5pt}\pcmn{男子盘腿坐,女子跪着坐}\end{exemple}
\begin{exemple}\pjya{thamtham tɕe, tɕheme ra kɯ χcoŋkroŋ ɲɯ-kɯ-βzu tu, ɕɯŋgɯ tɕe, tɕheme kɯ χcoŋkroŋ ɲɯ-βze mɯ-pjɤ-jɤɣ}\hspace{5pt}\pcmn{现在,有些女子会盘腿坐,以前是不允许的}\end{exemple}\end{entrée}

\begin{entrée}{χcrɯχcri/\variante{χcɯχcri}}{}{ⓔχcrɯχcri} 
\classe{idph.2} 
\begin{définition}\pfra{dilué, peu épais}\end{définition}
\begin{définition}\pcmn{形容流体稀}\end{définition}
\begin{exemple}\pjya{nɯŋa ɯ-qe χcɯχcri ʑo ɲɯ-pa}\hspace{5pt}\pcmn{牛屎很稀}\end{exemple}\relationsémantique{参考}{\lien{ⓔscrɯscri}{scrɯscri}}\relationsémantique{参考}{\lien{ⓔɲcriɲcri}{ɲcriɲcri}}\end{entrée}

\begin{entrée}{χcɯχcri}{}{ⓔχcɯχcri} 
\classe{idph.2} 
\begin{définition}\pfra{gras et mou}\end{définition}
\begin{définition}\pcmn{形容胖而软的样子}\end{définition}\relationsémantique{参考}{\lien{ⓔʁɟɯʁɟri}{ʁɟɯʁɟri}}\end{entrée}

\begin{entrée}{χɕu}{}{ⓔχɕu} 
\classe{vs} \paradigme{dir}{tɤ-}
\begin{définition}\pfra{fort, résistant}\end{définition}
\begin{définition}\pcmn{健壮;力量大;有耐性}\end{définition}
\begin{exemple}\pjya{jla ɲɯ-χɕu}\hspace{5pt}\pcmn{犏牛很强壮}\end{exemple}
\begin{exemple}\pjya{mkhɯrlu ɲɯ-χɕu}\hspace{5pt}\pcmn{汽车马力大}\end{exemple}
\begin{exemple}\pjya{pɯ-tɯ-χɕu}\hspace{5pt}\pcmn{谢谢}\end{exemple}\relationsémantique{参考}{\lien{ⓔχɕuχɕe}{χɕuχɕe}}\end{entrée}

\begin{entrée}{χɕaʁ}{}{ⓔχɕaʁ} 
\classe{n} 
\begin{définition}\pfra{morceau de bois coupé à la hache}\end{définition}
\begin{définition}\pcmn{劈好了的木料(房背、走缘当石板用)}\end{définition}\étymologie{gɕag}\end{entrée}

\begin{entrée}{χɕaʁ}{}{ⓔχɕaʁ} 
\classe{vi} \paradigme{dir}{nɯ-}
\begin{définition}\pfra{décéder (honorifique, réservé aux lamas et aux sprulsku)}\end{définition}
\begin{définition}\pcmn{圆寂(敬语)}\end{définition}\étymologie{gɕegs}\end{entrée}

\begin{entrée}{χɕɤβ}{}{ⓔχɕɤβ} 
\classe{vs} \paradigme{dir}{tɤ-}\sens{1}
\begin{définition}\pfra{excessif (parole)}\end{définition}
\begin{définition}\pcmn{夸张(话)}\end{définition}
\begin{exemple}\pjya{ɯ-rju ɲɯ-χɕɤβ}\hspace{5pt}\pcmn{他说的话很夸张}\end{exemple}\sens{2}
\begin{définition}\pfra{éclatante, vive (couleur)}\end{définition}
\begin{définition}\pcmn{鲜艳(颜色)}\end{définition}
\begin{exemple}\pjya{ɯ-mdoʁ ɲɯ-χɕɤβ}\hspace{5pt}\pcmn{颜色很鲜艳(红色、黄色)}\end{exemple}\sens{3}
\begin{définition}\pfra{fort (bruit)}\end{définition}
\begin{définition}\pcmn{响;吵}\end{définition}
\begin{exemple}\pjya{ɯ-zgra ɲɯ-χɕɤβ}\hspace{5pt}\pcmn{声音很响(很吵)}\end{exemple}\sens{4}
\begin{définition}\pfra{forte (odeur)}\end{définition}
\begin{définition}\pcmn{浓(气味)}\end{définition}
\begin{exemple}\pjya{ɯ-di ɲɯ-χɕɤβ}\hspace{5pt}\pcmn{气味很浓}\end{exemple}\relationsémantique{参考}{\lien{ⓔrɯχɕɯχɕɤβ}{rɯχɕɯχɕɤβ}}\end{entrée}

\begin{entrée}{χɕɤl}{}{ⓔχɕɤl} 
\classe{n} 
\begin{définition}\pfra{verre}\end{définition}
\begin{définition}\pcmn{玻璃}\end{définition}\étymologie{ɕel}\end{entrée}

\begin{entrée}{χɕɤlkara}{}{ⓔχɕɤlkara} 
\classe{n} 
\begin{définition}\pfra{sucre en morceau}\end{définition}
\begin{définition}\pcmn{冰糖}\end{définition}\étymologie{ɕel.dkar}\end{entrée}

\begin{entrée}{χɕɤlmdoŋ}{}{ⓔχɕɤlmdoŋ} 
\classe{n} 
\begin{définition}\pfra{lunette, télescope}\end{définition}
\begin{définition}\pcmn{望远镜}\end{définition}\end{entrée}

\begin{entrée}{χɕɤlmɯɣ}{}{ⓔχɕɤlmɯɣ} 
\classe{n} 
\begin{définition}\pfra{lunettes}\end{définition}
\begin{définition}\pcmn{眼镜}\end{définition}\paradigme{comit}{kɤ́χɕɤlmɯlmɯɣ}
\begin{exemple}\pjya{χɕɤlmɯɣ tɤ-nɯ-ta-t-a (tɤ-nɯ-tshoʁ-a)}\hspace{5pt}\pcmn{我戴上了眼镜}\end{exemple}\étymologie{ɕel.mig}\end{entrée}

\begin{entrée}{χɕɤlzgoŋ}{}{ⓔχɕɤlzgoŋ} 
\classe{n} 
\begin{définition}\pfra{miroir}\end{définition}
\begin{définition}\pcmn{镜子}\end{définition}\relationsémantique{同义词}{\lien{ⓔkɯsɤɣru}{kɯsɤɣru}}\étymologie{ɕel.sgo}\end{entrée}

\begin{entrée}{χɕitka}{}{ⓔχɕitka} 
\classe{n} 
\begin{définition}\pfra{printemps}\end{définition}
\begin{définition}\pcmn{春天}\end{définition}\étymologie{dpʲid.ka}\end{entrée}

\begin{entrée}{χɕoʁ}{}{ⓔχɕoʁ} 
\classe{vt} \paradigme{dir}{\_}
\begin{définition}\pfra{tirer}\end{définition}
\begin{définition}\pcmn{抽出}\end{définition}
\begin{exemple}\pjya{ndʑu tɤ-χɕoʁ-a}\hspace{5pt}\pcmn{我把筷子抽出来了}\end{exemple}
\begin{exemple}\pjya{si thɯ-χɕoʁ}\hspace{5pt}\pcmn{(从柴堆里)取一根}\end{exemple}
\begin{exemple}\pjya{scapa la-χɕoʁ}\hspace{5pt}\pcmn{他把剑抽出来了}\end{exemple}\end{entrée}

\begin{entrée}{χɕu,rnaʁ}{}{ⓔχɕu,rnaʁ} 
\classe{vs}
\classe{vs} 
\begin{définition}\pfra{merci beaucoup}\end{définition}
\begin{définition}\pcmn{万分感谢}\end{définition}
\begin{exemple}\pjya{pɯ-tɯ-χɕu pɯ-tɯ-rnaʁ}\hspace{5pt}\pcmn{感谢你了}\end{exemple}
\begin{exemple}\pjya{pɯ-χɕu pɯ-rnaʁ}\hspace{5pt}\pcmn{万分感谢他}\end{exemple}\relationsémantique{Component 1}{\lien{ⓔχɕu}{χɕu}}\relationsémantique{Component 2}{\lien{ⓔrnaʁ}{rnaʁ}}\end{entrée}

\begin{entrée}{χɕɯldɤn}{}{ⓔχɕɯldɤn} 
\classe{n} 
\begin{définition}\pfra{en sécurité}\end{définition}
\begin{définition}\pcmn{安全;安康}\end{définition}
\begin{exemple}\pjya{ɯʑo ɯ-kha ra mɤ-kɯ-pe ku-me tɕe, χɕɯldɤn ɕti}\hspace{5pt}\pcmn{他的家里人都好,平安无事}\end{exemple}\relationsémantique{参考}{\lien{ⓔaχɕɯldɤn}{aχɕɯldɤn}}\end{entrée}

\begin{entrée}{χɕɯn}{}{ⓔχɕɯn} 
\classe{vi} \sens{1}\paradigme{dir}{pɯ-}
\begin{définition}\pfra{sain et sauf}\end{définition}
\begin{définition}\pcmn{安全}\end{définition}
\begin{exemple}\pjya{kɤ-χɕɯn lo-zɣɯt-ndʑi}\hspace{5pt}\pcmn{他们俩安全到达了}\end{exemple}
\begin{exemple}\pjya{kɤ-χɕɯn kɤ-nɯʑɯβ}\hspace{5pt}\pcmn{安心睡觉吧!}\end{exemple}\sens{2}
\begin{définition}\pfra{être fini (travail)}\end{définition}
\begin{définition}\pcmn{完;结束(工作)}\end{définition}
\begin{exemple}\pjya{ta-ma pjɤ-χɕɯn}\hspace{5pt}\pcmn{工作完了}\end{exemple}\étymologie{gɕin}\end{entrée}

\begin{entrée}{χɕɯnrʑi}{}{ⓔχɕɯnrʑi} 
\classe{n} 
\begin{définition}\pfra{Yama}\end{définition}
\begin{définition}\pcmn{阎王}\end{définition}\étymologie{gɕen.rdʑe}\end{entrée}

\begin{entrée}{χɕɯχɕi}{}{ⓔχɕɯχɕi} 
\classe{idph.2} 
\begin{définition}\pfra{qui écoute en silence}\end{définition}
\begin{définition}\pcmn{形容悄悄地听的样子}\end{définition}
\begin{exemple}\pjya{nɯtɕu nɯ-rkɯ χɕɯχɕi ʑo pjɤ-tɯ-sɤŋo}\hspace{5pt}\pcmn{你在那里悄悄地听他们交谈}\end{exemple}\end{entrée}

\begin{entrée}{χɕuχɕe}{}{ⓔχɕuχɕe} 
\classe{vs} 
\begin{définition}\pfra{fort, robuste}\end{définition}
\begin{définition}\pcmn{身强力壮}\end{définition}
\begin{exemple}\pjya{thamtham ʁʑɯnɯ ɲɯ-ɕti tɕe, wuma ʑo ɲɯ-χɕuχɕe}\hspace{5pt}\pcmn{他是青年,非常强壮}\end{exemple}\relationsémantique{参考}{\lien{ⓔχɕu}{χɕu}}\end{entrée}

\begin{entrée}{χinɤχi}{}{ⓔχinɤχi} 
\classe{idph.3} 
\begin{définition}\pfra{éssouflé}\end{définition}
\begin{définition}\pcmn{形容气喘吁吁的样子}\end{définition}
\begin{exemple}\pjya{ɯʑo jo-rɟɯɣ pjɤ-ra tɕe, χinɤχi ʑo ɲɯ-ʑɣɤstu jɤ-azɣɯt}\hspace{5pt}\pcmn{因为他要跑过来,气喘吁吁地到了}\end{exemple}
\begin{exemple}\pjya{ɲɤ-nɤɴqa, ɯ-fkur pjɤ-rʑi tɕe, χinɤχi ʑo jɤ-azɣɯt}\hspace{5pt}\pcmn{因为他很辛苦,负担很重,气喘吁吁地到了}\end{exemple}\relationsémantique{同义词}{\lien{ⓔqhinɤqhi}{qhinɤqhi}}\end{entrée}

\begin{entrée}{χɲɤβχɲɤβ}{}{ⓔχɲɤβχɲɤβ} 
\classe{idph.2} 
\begin{définition}\pfra{mou et humide}\end{définition}
\begin{définition}\pcmn{形容软而潮湿的样子}\end{définition}\relationsémantique{同义词}{\lien{ⓔχɲɤχɲɤr}{χɲɤχɲɤr}}\end{entrée}

\begin{entrée}{χɲɤχɲɤr}{}{ⓔχɲɤχɲɤr} 
\classe{idph.2} 
\begin{définition}\pfra{mou et humide}\end{définition}
\begin{définition}\pcmn{形容软而潮湿的样子}\end{définition}
\begin{exemple}\pjya{a-ʑɯβ ɲɯ-ɣi tɕe a-phoŋbu ra χɲɤχɲɤr ʑo ɲɯ-pa}\hspace{5pt}\pcmn{我睡意来了,觉得浑身软趴趴的}\end{exemple}\relationsémantique{同义词}{\lien{ⓔχɲɤβχɲɤβ}{χɲɤβχɲɤβ}}\end{entrée}

\begin{entrée}{χɲɯχɲi}{}{ⓔχɲɯχɲi} 
\classe{idph.2} \sens{1}
\begin{définition}\pfra{mou et en bouillie}\end{définition}
\begin{définition}\pcmn{形容又软又稀;晒嫣了的食物}\end{définition}
\begin{exemple}\pjya{kɯ-mpɯ kɯ-fse χɲɯχɲi kɯ-pa nɯra mɤ-rga-a}\hspace{5pt}\pcmn{我不喜欢又的那些(食物)}\end{exemple}\sens{2}
\begin{définition}\pfra{sans force}\end{définition}
\begin{définition}\pcmn{形容没有精神,没有力气的样子}\end{définition}
\begin{exemple}\pjya{iɕqha tɯrme nɯ χɲɯχɲi kɯ-pa ci ɲɯ-ŋu}\hspace{5pt}\pcmn{这个人没有精神}\end{exemple}\end{entrée}

\begin{entrée}{χoŋ}{}{ⓔχoŋ} 
\classe{idph.1} 
\begin{définition}\pfra{qui tombe tout d'un coup dans un trou}\end{définition}
\begin{définition}\pcmn{突然踏空,掉进洞里}\end{définition}
\begin{exemple}\pjya{ɯ-mi χoŋ ʑo pjɤ-ɕqhlɤt}\hspace{5pt}\pcmn{他突然踏空了}\end{exemple}
\begin{sous-entrée}{χoŋχoŋ}{ⓔχoŋⓝχoŋχoŋ} 
\classe{idph.2} 
\begin{définition}\pfra{ayant un grand trou}\end{définition}
\begin{définition}\pcmn{有洞(洞口很大)}\end{définition}
\begin{exemple}\pjya{ɯ-ŋga χoŋχoŋ ʑo kɯ-pa ko-spoʁ}\hspace{5pt}\pcmn{他衣服上有个大窟窿}\end{exemple}\end{sous-entrée}

\sens{2}
\begin{définition}\pfra{aube}\end{définition}
\begin{définition}\pcmn{天亮}\end{définition}
\begin{exemple}\pjya{lo-fsoʁ χoŋχoŋ ri mɯ́j-tɯ-rɤru}\hspace{5pt}\pcmn{天亮了,你还不起床}\end{exemple}\end{entrée}

\begin{entrée}{χpa}{}{ⓔχpa} 
\classe{vs} \paradigme{dir}{tɤ-}\paradigme{dir}{thɯ-}
\begin{définition}\pfra{fier, plein de confiance en soi}\end{définition}
\begin{définition}\pcmn{自豪,很有自信}\end{définition}
\begin{exemple}\pjya{tɯrme kɯ-χpa ci ɲɯ-ŋu}\hspace{5pt}\pcmn{他是一个自豪的人}\end{exemple}
\begin{exemple}\pjya{ɯ-sɯm lo-χpa}\hspace{5pt}\pcmn{他变得骄傲了}\end{exemple}\relationsémantique{参考}{\lien{ⓔznɤχpɯχpa}{znɤχpɯχpa}}\étymologie{dpa}\end{entrée}

\begin{entrée}{χpaχtshɤt}{}{ⓔχpaχtshɤt} 
\classe{n} 
\begin{définition}\pfra{yojana}\end{définition}
\begin{définition}\pcmn{由旬,长度单位(非常远)}\end{définition}\étymologie{dpag.tsʰad}\end{entrée}

\begin{entrée}{χpɤlwi}{}{ⓔχpɤlwi} 
\classe{n} 
\begin{définition}\pfra{un motif bouddhique}\end{définition}
\begin{définition}\pcmn{一种佛教图纹}\end{définition}\étymologie{dpal.beɦu}\end{entrée}

\begin{entrée}{χphjɤrχphjɤr}{}{ⓔχphjɤrχphjɤr} 
\classe{idph.2} 
\begin{définition}\pfra{fade}\end{définition}
\begin{définition}\pcmn{形容没有香味的}\end{définition}
\begin{exemple}\pjya{tɤ-mthɯm kɤ́-wɣ-sqa a-kɤ-smi ɲɯ-ra ma nɯ maʁ nɤ tɤ́-wɣ-ndza tɕe χphjɤrχphjɤr ʑo ɲɯ-pa tɕe mɯ́j-mɯm}\hspace{5pt}\pcmn{煮肉的的时候要煮熟,不然吃起来一点香味也没有,不好吃}\end{exemple}\end{entrée}

\begin{entrée}{χpi}{}{ⓔχpi} 
\classe{n} \sens{1}
\begin{définition}\pfra{histoire}\end{définition}
\begin{définition}\pcmn{故事}\end{définition}
\begin{exemple}\pjya{ɯʑo kɯ χpi wuma kɯ-mpɕɤr ci ɲɤ-sɤβzu}\hspace{5pt}\pcmn{他编了一个很精彩的故事}\end{exemple}\sens{2}
\begin{définition}\pfra{exemple}\end{définition}
\begin{définition}\pcmn{例子}\end{définition}
\begin{exemple}\pjya{ɯ-χpi zɯ (= a-pɯ-ŋu nɤ)}\hspace{5pt}\pcmn{例如}\end{exemple}
\begin{exemple}\pjya{nɤ-χpi ci ku-te-a}\hspace{5pt}\pcmn{我给你举个例子}\end{exemple}\relationsémantique{参考}{\lien{ⓔta-χpi}{ta-χpi}}\relationsémantique{参考}{\lien{ⓔraχpi}{raχpi}}\étymologie{dpe}\end{entrée}

\begin{entrée}{χpjɤt}{}{ⓔχpjɤt} 
\classe{vt}  
\grammaire{refl} \paradigme{dir}{kɤ-}\paradigme{dir}{tɤ-}\paradigme{dir}{tɤ-}\paradigme{dir}{tɤ-}\paradigme{dir}{kɤ-}
\begin{définition}\pfra{observer}\end{définition}
\begin{définition}\pcmn{观察}\end{définition}
\begin{définition}\pfra{cela dépend de}\end{définition}
\begin{définition}\pcmn{随便……,由……决定,自己看着办}\end{définition}
\begin{exemple}\pjya{nɤ-zda kɤ-χpjɤt}\hspace{5pt}\pcmn{你观察一下你的同伴}\end{exemple}
\begin{exemple}\pjya{aʑo ku-kɯ-χpjat-a}\hspace{5pt}\pcmn{你在观察我}\end{exemple}
\begin{exemple}\pjya{nɯ kɤ-χpjɤt me nɤ}\hspace{5pt}\pcmn{那个是说不准的}\end{exemple}
\begin{exemple}\pjya{nɤʑo tɯ-ɕe mɤ-tɯ-ɕe, nɤʑo tɤ-nɯχpjɤt}\hspace{5pt}\pcmn{去不去由你}\end{exemple}
\begin{exemple}\pjya{nɤʑo ma-tɯ-ɤrju a-tɤ-nɯχpjɤt}\hspace{5pt}\pcmn{你不要说,由他自己决定}\end{exemple}
\begin{exemple}\pjya{nɯ-βde a-tɤ-naχpjɤt}\hspace{5pt}\pcmn{你放弃吧,没有必要坚持}\end{exemple}
\begin{exemple}\pjya{mɯ-ɲɯ-ɣi nɤ a-tɤ-naχpjɤt}\hspace{5pt}\pcmn{如果他不来就算了吧}\end{exemple}
\begin{exemple}\pjya{kɤ-naχpjɤt mɤ-βze}\hspace{5pt}\pcmn{不能这样就过去了(不能放弃)}\end{exemple}
\begin{sous-entrée}{nɯχpjɤt}{ⓔχpjɤtⓝnɯχpjɤt} 
\classe{vt} \end{sous-entrée}

\begin{sous-entrée}{naχpjɤt}{ⓔχpjɤtⓝnaχpjɤt}\end{sous-entrée}

\begin{sous-entrée}{aχpɯχpjɤt}{ⓔχpjɤtⓝaχpɯχpjɤt} 
\classe{vi}  
\grammaire{recip} 
\begin{définition}\pfra{se regarder les uns les autres}\end{définition}
\begin{définition}\pcmn{互相观察(行为)}\end{définition}
\begin{exemple}\pjya{ma-tɯ-ɤχpɯχpjɤt-nɯ}\hspace{5pt}\pcmn{你们不要互相学坏}\end{exemple}\end{sous-entrée}

\begin{sous-entrée}{ʑɣɤχpjɤt}{ⓔχpjɤtⓝʑɣɤχpjɤt} 
\classe{vi} \end{sous-entrée}

\paradigme{dir}{tɤ-}
\begin{définition}\pfra{s'observer}\end{définition}
\begin{définition}\pcmn{观察自己}\end{définition}
\begin{définition}\pfra{demander l'avis de}\end{définition}
\begin{définition}\pcmn{征求意见}\end{définition}
\begin{exemple}\pjya{nɤʑo kɤ-nɯ-ʑɣɤχpjɤt tɕe, tɤ-wa ɯ-ɲɯ-tɯ-fse kɯ?}\hspace{5pt}\pcmn{你观察一下自己,是不是像一个父亲的样子}\end{exemple}
\begin{exemple}\pjya{laχtɕha nɯ tú-wɣ-χtɯ ɕi kɯ tu-ta-sɯχpjɤt}\hspace{5pt}\pcmn{我征求一下你的意见,买不买这个东西}\end{exemple}
\begin{sous-entrée}{sɯχpjɤt}{ⓔχpjɤtⓝsɯχpjɤt} 
\classe{vt} \end{sous-entrée}

\end{entrée}

\begin{entrée}{χploʁploʁ}{}{ⓔχploʁploʁ} 
\classe{idph.2} 
\begin{définition}\pfra{en boule}\end{définition}
\begin{définition}\pcmn{形容球形}\end{définition}
\begin{exemple}\pjya{ɯ-rte nɯ χploʁploʁ kɯ-pa ci ɲɯ-ŋu}\hspace{5pt}\pcmn{他的帽子是球形的}\end{exemple}
\begin{exemple}\pjya{tɤjmɤɣ thamtɕɤt nɯ mɯ-tɤ-kɯ-qawɤr nɯ χploʁploʁ kɯ-pa tu-kɯ-ti ɕti}\hspace{5pt}\pcmn{菌子在开放之前都是球形的}\end{exemple}\relationsémantique{参考}{\lien{ⓔploʁploʁ}{ploʁploʁ}}\relationsémantique{参考}{\lien{ⓔɕploʁɕploʁ}{ɕploʁɕploʁ}}\end{entrée}

\begin{entrée}{χpɯn}{}{ⓔχpɯn} 
\classe{n} 
\begin{définition}\pfra{moine}\end{définition}
\begin{définition}\pcmn{和尚}\end{définition}
\begin{exemple}\pjya{χpɯn to-ndo}\hspace{5pt}\pcmn{他当了和尚}\end{exemple}
\begin{exemple}\pjya{ɯ-tɕɯ χpɯn lu-te-a ɲɯ-sɯsam-a}\hspace{5pt}\pcmn{我想让我儿子当和尚}\end{exemple}\relationsémantique{参考}{\lien{ⓔnɯχpɯn}{nɯχpɯn}}\relationsémantique{参考}{\lien{ⓔrɤχpɯn}{rɤχpɯn}}\étymologie{dpon}\end{entrée}

\begin{entrée}{χpɯnbu}{}{ⓔχpɯnbu} 
\classe{n} 
\begin{définition}\pfra{maître}\end{définition}
\begin{définition}\pcmn{主人}\end{définition}
\begin{exemple}\pjya{ɯʑo kɯ χpɯnbu la-ndo}\hspace{5pt}\pcmn{他掌权了}\end{exemple}\relationsémantique{参考}{\lien{ⓔnɯχpɯnbu}{nɯχpɯnbu}}\end{entrée}

\begin{entrée}{χsu}{}{ⓔχsu} 
\classe{vt} \paradigme{dir}{nɯ-}\paradigme{dir}{pɯ-}\paradigme{dir}{thɯ-}
\begin{définition}\pfra{élever}\end{définition}
\begin{définition}\pcmn{养;供他吃}\end{définition}
\begin{exemple}\pjya{pɣɤtɕɯ kɯ ɯ-pɯ ɲɯ-ɤsɯ-χsu}\hspace{5pt}\pcmn{鸟在喂它的小鸟}\end{exemple}
\begin{exemple}\pjya{nɯ-χsu-t-a}\hspace{5pt}\pcmn{我养了他}\end{exemple}
\begin{exemple}\pjya{nɯ́-wɣ-χsu-a}\hspace{5pt}\pcmn{他养了我}\end{exemple}
\begin{exemple}\pjya{paʁ pɯ-χsu-t-a}\hspace{5pt}\pcmn{我喂了猪}\end{exemple}
\begin{exemple}\pjya{rgali thɯ-χsu-t-a}\hspace{5pt}\pcmn{我喂了小奶牛}\end{exemple}\relationsémantique{同义词}{\lien{ⓔngu}{ngu}}
\begin{sous-entrée}{sɯχsu}{ⓔχsuⓝsɯχsu} 
\classe{vt} 
\begin{définition}\pfra{nourrir avec}\end{définition}
\begin{définition}\pcmn{用……喂}\end{définition}\end{sous-entrée}

\begin{sous-entrée}{ʑɣɤsɯχsu}{ⓔχsuⓝʑɣɤsɯχsu} 
\classe{vi}  
\grammaire{refl} 
\begin{définition}\pfra{se nourrir}\end{définition}
\begin{définition}\pcmn{自己喂自己}\end{définition}\relationsémantique{同义词}{\lien{ⓔngu}{ngu}}\end{sous-entrée}

\étymologie{gso}\end{entrée}

\begin{entrée}{χsɤβ}{}{ⓔχsɤβ} 
\classe{n} 
\begin{définition}\pfra{étalon}\end{définition}
\begin{définition}\pcmn{公马}\end{définition}\étymologie{gseb}\end{entrée}

\begin{entrée}{χsɤl}{₁}{ⓔχsɤlⓗ1} 
\classe{vt} \paradigme{dir}{tɤ-}\paradigme{dir}{kɤ-}
\begin{définition}\pfra{manger, boire (honorifique)}\end{définition}
\begin{définition}\pcmn{用膳(敬语)}\end{définition}
\begin{exemple}\pjya{βlama kɯ to-χsɤl}\hspace{5pt}\pcmn{喇嘛吃了}\end{exemple}\étymologie{gsol}\end{entrée}

\begin{entrée}{χsɤl}{₂}{ⓔχsɤlⓗ2} 
\classe{vi} \paradigme{dir}{tɤ-}
\begin{définition}\pfra{clair, évident}\end{définition}
\begin{définition}\pcmn{明显;清晰}\end{définition}
\begin{exemple}\pjya{fso tɕe a-kɤ-tɯ-lɤt tɕe a-tɤ-χsɤl je}\hspace{5pt}\pcmn{你明天打电话就会知道}\end{exemple}
\begin{exemple}\pjya{kɯki tɤ-scoz ki pjɯ́-wɣ-ndɯn ɲɯ-jɤɣ ma ɲɯ-χsɤl}\hspace{5pt}\pcmn{可以把这封信读出来,因为写得很清楚}\end{exemple}\relationsémantique{参考}{\lien{ⓔsaχsɤlⓗ1ⓝsaχsɤl}{saχsɤl}}\relationsémantique{参考}{\lien{ⓔɣɯχsɤl}{ɣɯχsɤl}}\étymologie{gsal}\end{entrée}

\begin{entrée}{χsɤlkawa}{}{ⓔχsɤlkawa} 
\classe{n} 
\begin{définition}\pfra{moine qui garde la chapelle}\end{définition}
\begin{définition}\pcmn{守佛堂的和尚}\end{définition}\étymologie{gsol.ka.ba}\end{entrée}

\begin{entrée}{χsɤltɕhɯ/\variante{ʑɤβtɕhɯ}\variante{zlɤβtɕhɯ}}{}{ⓔχsɤltɕhɯ} 
\classe{n} 
\begin{définition}\pfra{eau (honorifique)}\end{définition}
\begin{définition}\pcmn{水(敬语)}\end{définition}\étymologie{gsol.tɕʰu}\end{entrée}

\begin{entrée}{χsɤr}{₂}{ⓔχsɤrⓗ2} 
\classe{n} 
\begin{définition}\pfra{or}\end{définition}
\begin{définition}\pcmn{金子}\end{définition}\étymologie{gser}\end{entrée}

\begin{entrée}{χsɤr}{₁}{ⓔχsɤrⓗ1} 
\classe{vt} \paradigme{dir}{tɤ-}
\begin{définition}\pfra{compter}\end{définition}
\begin{définition}\pcmn{数}\end{définition}
\begin{exemple}\pjya{ji-nɯŋa thɤstɯɣ ɣɤʑu tɤ-χsar-a}\hspace{5pt}\pcmn{我数了一下我们家的牛有多少头}\end{exemple}
\begin{exemple}\pjya{tɤ-χsar-a ri kɯmŋu ɣɤʑu}\hspace{5pt}\pcmn{我数了一下,有五个}\end{exemple}\relationsémantique{同义词}{\lien{ⓔrtsi}{rtsi}}\relationsémantique{参考}{\lien{ⓔɯ-χsɤr}{ɯ-χsɤr}}\end{entrée}

\begin{entrée}{χsɤrɣɤt}{}{ⓔχsɤrɣɤt} 
\classe{n} 
\begin{définition}\pfra{sûtra pour les animaux}\end{définition}
\begin{définition}\pcmn{为牛羊念的经}\end{définition}\étymologie{gser.ɦod}\end{entrée}

\begin{entrée}{χsɤrmdoʁ}{}{ⓔχsɤrmdoʁ} 
\classe{n} 
\begin{définition}\pfra{doré}\end{définition}
\begin{définition}\pcmn{金色}\end{définition}\étymologie{gser.mdog}\end{entrée}

\begin{entrée}{χsɤrnaʁ}{}{ⓔχsɤrnaʁ} 
\classe{n} 
\begin{définition}\pfra{or noir}\end{définition}
\begin{définition}\pcmn{黑色的金}\end{définition}\étymologie{gser.nag}\end{entrée}

\begin{entrée}{χsɤrɲa}{}{ⓔχsɤrɲa} 
\classe{n} 
\begin{définition}\pfra{cyprin}\end{définition}
\begin{définition}\pcmn{金鱼}\end{définition}\étymologie{gser.ɲa}\end{entrée}

\begin{entrée}{χsɤrɲɟɤt}{}{ⓔχsɤrɲɟɤt} 
\classe{n} 
\begin{définition}\pfra{vermeil}\end{définition}
\begin{définition}\pcmn{镀金的白银}\end{définition}\end{entrée}

\begin{entrée}{χsɤrthu}{}{ⓔχsɤrthu} 
\classe{n} 
\begin{définition}\pfra{la fête de l'été}\end{définition}
\begin{définition}\pcmn{看花节}\end{définition}\end{entrée}

\begin{entrée}{χsɤrʑa}{}{ⓔχsɤrʑa} 
\classe{n} 
\begin{définition}\pfra{coiffe en or}\end{définition}
\begin{définition}\pcmn{金冠}\end{définition}\étymologie{gser.ʑwa}\end{entrée}

\begin{entrée}{χsjɯβ}{}{ⓔχsjɯβ} 
\classe{n} 
\begin{définition}\pfra{peau de serpent}\end{définition}
\begin{définition}\pcmn{蛇蜕的皮}\end{définition}
\begin{exemple}\pjya{qapri ɯ-χsjɯβ chɤ-βde}\hspace{5pt}\pcmn{蛇脱皮了}\end{exemple}\end{entrée}

\begin{entrée}{χsjɯβnɤχsjɯβ}{}{ⓔχsjɯβnɤχsjɯβ} 
\classe{idph.3} 
\begin{définition}\pfra{reniflement}\end{définition}
\begin{définition}\pcmn{形容用鼻吸气的声音}\end{définition}
\begin{exemple}\pjya{χsjɯβnɤχsjɯβ to-stu (=ɯ-ɕna to-sɤχsɯχsjɯβ)}\hspace{5pt}\pcmn{他用鼻吸了气}\end{exemple}\relationsémantique{参考}{\lien{ⓔsɤχsɯχsjɯβ}{sɤχsɯχsjɯβ}}\end{entrée}

\begin{entrée}{χsoz}{}{ⓔχsoz} 
\classe{vs} \sens{1}
\begin{définition}\pfra{fine (oreille)}\end{définition}
\begin{définition}\pcmn{敏锐(耳朵)}\end{définition}
\begin{exemple}\pjya{tɕhi rɯdaʁ kɯ-fse ɯ-rna kɯ-χsoz me}\hspace{5pt}\pcmn{什么动物的听觉是最敏锐的?}\end{exemple}
\begin{exemple}\pjya{tɕhi rɯdaʁ ɣɯ ɯ-rna stu ʑo χsoz}\hspace{5pt}\pcmn{什么动物的听觉是最敏锐的?}\end{exemple}\sens{2}
\begin{définition}\pfra{fin (odorat)}\end{définition}
\begin{définition}\pcmn{敏锐(嗅觉)}\end{définition}
\begin{exemple}\pjya{a-ɕna χsoz}\hspace{5pt}\pcmn{我的嗅觉很敏锐}\end{exemple}\sens{3}
\begin{définition}\pfra{pas bouché (paille)}\end{définition}
\begin{définition}\pcmn{畅通(管子)}\end{définition}
\begin{exemple}\pjya{chɤmdɤru ɲɯ-χsoz tɕe, chɤmda kɤ-sɯtshi ɲɯ-khu}\hspace{5pt}\pcmn{吸管是畅通的,可以喝坛坛酒}\end{exemple}\relationsémantique{参考}{\lien{ⓔnaχsoz}{naχsoz}}\end{entrée}

\begin{entrée}{χsɯm}{}{ⓔχsɯm} 
\classe{num} 
\begin{définition}\pfra{trois}\end{définition}
\begin{définition}\pcmn{三}\end{définition}\end{entrée}

\begin{entrée}{χsɯmba}{}{ⓔχsɯmba} 
\classe{n} 
\begin{définition}\pfra{troisième mois}\end{définition}
\begin{définition}\pcmn{三月}\end{définition}\étymologie{gsum.pa}\end{entrée}

\begin{entrée}{χsɯmdu}{}{ⓔχsɯmdu} 
\classe{n} 
\begin{définition}\pfra{carrefour en Y}\end{définition}
\begin{définition}\pcmn{三岔口}\end{définition}\relationsémantique{同义词}{\lien{ⓔtʂɤsɤɴɢɤt}{tʂɤsɤɴɢɤt}}\end{entrée}

\begin{entrée}{χsɯmkha}{}{ⓔχsɯmkha} 
\classe{n} 
\begin{définition}\pfra{la troisième fois}\end{définition}
\begin{définition}\pcmn{第三次}\end{définition}
\begin{exemple}\pjya{χsɯmkha to-mbri}\hspace{5pt}\pcmn{已经是第三次了}\end{exemple}\end{entrée}

\begin{entrée}{χsɯmsna}{}{ⓔχsɯmsna} 
\classe{n} 
\begin{définition}\pfra{lorsque l'on tue un animal, part donnée aux amis}\end{définition}
\begin{définition}\pcmn{宰动物的时候,分给好友的部分}\end{définition}\étymologie{gsum.sna}\end{entrée}

\begin{entrée}{χsɯskɤl}{}{ⓔχsɯskɤl} 
\classe{n} 
\begin{définition}\pfra{trois repas}\end{définition}
\begin{définition}\pcmn{三顿}\end{définition}
\begin{exemple}\pjya{zama χsɯskɤl nɯ koŋla tu-kɯ-rɯndzɤtshi ra}\hspace{5pt}\pcmn{一天三餐要吃好一点}\end{exemple}\end{entrée}

\begin{entrée}{χʂɤχʂɤt}{}{ⓔχʂɤχʂɤt} 
\classe{idph.2} \sens{1}
\begin{définition}\pfra{léger (habit)}\end{définition}
\begin{définition}\pcmn{形容(衣服)单薄,透风}\end{définition}
\begin{exemple}\pjya{nɤ-ŋga kɯ-mba ci χʂɤχʂɤt tɯ-asɯ-ŋga}\hspace{5pt}\pcmn{你穿的衣服很单薄}\end{exemple}
\begin{exemple}\pjya{ɯ-ŋga ɲɯ-mba ʑo χʂɤχʂɤt}\hspace{5pt}\pcmn{他的衣服很薄}\end{exemple}\sens{2}
\begin{définition}\pfra{qui a les traits fins, au regard intelligent}\end{définition}
\begin{définition}\pcmn{眉目清秀,聪明的样子}\end{définition}
\begin{exemple}\pjya{kɯ-mpɕɤr ci χʂɤχʂɤt ɲɯ-ŋu}\hspace{5pt}\pcmn{他眉目清秀(有双眼皮,眼睛又圆又亮)}\end{exemple}\relationsémantique{参考}{\lien{ⓔxʂɤxʂɤt}{xʂɤxʂɤt}}\end{entrée}

\begin{entrée}{χtɤr}{}{ⓔχtɤr} 
\classe{vt} \paradigme{dir}{thɯ-}\paradigme{dir}{pɯ-}
\begin{définition}\pfra{disperser}\end{définition}
\begin{définition}\pcmn{打散;撒}\end{définition}
\begin{exemple}\pjya{tɯjpu pjɤ-χtɤr}\hspace{5pt}\pcmn{他撒了粮食}\end{exemple}
\begin{exemple}\pjya{tɯ-rɣi tha-χtɤr}\hspace{5pt}\pcmn{他撒了种子}\end{exemple}
\begin{exemple}\pjya{tɯ-ɣndʑɤr pɯ-nɯ-χtar-a}\hspace{5pt}\pcmn{我把糌粑打散了}\end{exemple}
\begin{exemple}\pjya{khu ɯ-ɯ-sɤɣmu kɯ rɯdaʁ ra pjɤ́-wɣ-χtɤr-nɯ ʑo}\hspace{5pt}\pcmn{老虎可怕得让百兽四处逃窜了}\end{exemple}\relationsémantique{参考}{\lien{ⓔʁndɤr}{ʁndɤr}}\relationsémantique{参考}{\lien{ⓔrɯtɕhɯχtɤr}{rɯtɕhɯχtɤr}}\étymologie{gtor}\end{entrée}

\begin{entrée}{χtɤt}{}{ⓔχtɤt}\paradigme{dir}{kɤ-}\paradigme{dir}{pɯ-}\paradigme{dir}{nɯ-}
\begin{définition}\pfra{appuyer contre}\end{définition}
\begin{définition}\pcmn{靠}\end{définition}
\begin{exemple}\pjya{kɤ-ɤmdzɯ-a tɕe, a-mgɯr kɤ-nɯ-χtat-a}\hspace{5pt}\pcmn{我坐下的时候就靠了背}\end{exemple}
\begin{exemple}\pjya{znde ɯ-taʁ a-mgɯr kɤ-nɯ-χtat-a}\hspace{5pt}\pcmn{我把背靠在墙上了}\end{exemple}
\begin{sous-entrée}{tɯ-sɯm,χtɤt}{ⓔχtɤtⓝtɯ-sɯm,χtɤt}\sens{1}
\begin{définition}\pfra{être loyal envers}\end{définition}
\begin{définition}\pcmn{对……忠心}\end{définition}
\begin{exemple}\pjya{nɤ-ɕki a-sɯm ku-χtat-a ɕti}\hspace{5pt}\pcmn{我对你忠心耿耿}\end{exemple}\end{sous-entrée}

\sens{2}
\begin{définition}\pfra{se concentrer}\end{définition}
\begin{définition}\pcmn{专心}\end{définition}
\begin{exemple}\pjya{a-sɯm ku-χtat-a ʑo ku-rɤma-a}\hspace{5pt}\pcmn{我很专心地做事}\end{exemple}
\begin{exemple}\pjya{nɤ-sɯm kɤ-χtɤt ʑo pɯ-rɤβzjoz ra nɤ!}\hspace{5pt}\pcmn{你要专心学习}\end{exemple}
\begin{sous-entrée}{ɯ-rɕa,χtɤt}{ⓔχtɤtⓢ2ⓝɯ-rɕa,χtɤt}
\begin{définition}\pfra{se concentrer}\end{définition}
\begin{définition}\pcmn{专心}\end{définition}\end{sous-entrée}

\begin{sous-entrée}{ʑɣɤχtɤt}{ⓔχtɤtⓢ2ⓝʑɣɤχtɤt} 
\classe{vi}  
\grammaire{refl} 
\begin{définition}\pfra{s'appuyer sur}\end{définition}
\begin{définition}\pcmn{靠}\end{définition}\end{sous-entrée}

\begin{exemple}\pjya{si ɯ-taʁ ko-ʑɣɤχtɤt}\hspace{5pt}\pcmn{他靠在树上}\end{exemple}
\begin{exemple}\pjya{znde ɯ-taʁ ko-ʑɣɤχtɤt}\hspace{5pt}\pcmn{我靠在墙上}\end{exemple}
\begin{exemple}\pjya{ɯ-zda ɯ-taʁ ko-ʑɣɤχtɤt}\hspace{5pt}\pcmn{他依靠了他的伙伴}\end{exemple}
\begin{exemple}\pjya{aʑo ɲɯ-ɤɲat-a tɕe, ɯ-taʁ kɤ-ʑɣɤχtat-a tɕe tɤ-nɯna-a}\hspace{5pt}\pcmn{我累了所以靠在他身上休息了}\end{exemple}\relationsémantique{同义词}{\lien{ⓔʑɣɤta}{ʑɣɤta}}\étymologie{gtad}\end{entrée}

\begin{entrée}{χtɕɤnzɤn}{}{ⓔχtɕɤnzɤn} 
\classe{n} 
\begin{définition}\pfra{bête sauvage}\end{définition}
\begin{définition}\pcmn{野兽}\end{définition}\étymologie{gtɕan.gzan}\end{entrée}

\begin{entrée}{χtɕɤt}{}{ⓔχtɕɤt} 
\classe{n} 
\begin{définition}\pfra{exorcisme}\end{définition}
\begin{définition}\pcmn{驱妖的仪式;咒经}\end{définition}
\begin{exemple}\pjya{sprɯskɯ kɯ χtɕɤt pa-lɤt}\hspace{5pt}\pcmn{活佛念了咒经}\end{exemple}\étymologie{gtɕod}\end{entrée}

\begin{entrée}{χtɕɤz}{}{ⓔχtɕɤz} 
\classe{vs} \paradigme{dir}{pɯ-}
\begin{définition}\pfra{être chéri}\end{définition}
\begin{définition}\pcmn{受宠}\end{définition}
\begin{exemple}\pjya{ɯ-mɤ-tɯ-χtɕɤz nɯ dɯχpa!}\hspace{5pt}\pcmn{他很可怜,根本不受人宠爱}\end{exemple}
\begin{sous-entrée}{sɯχtɕɤz}{ⓔχtɕɤzⓝsɯχtɕɤz} 
\classe{vt} 
\begin{définition}\pfra{adorer}\end{définition}
\begin{définition}\pcmn{宠爱}\end{définition}
\begin{exemple}\pjya{ɯ-mu ɯ-wa ni kɯ ɲɯ-sɯχtɕɤz-ndʑi}\hspace{5pt}\pcmn{他父母很宠爱他}\end{exemple}\end{sous-entrée}

\étymologie{gtɕes}\end{entrée}

\begin{entrée}{χtɕɤzɤz}{}{ⓔχtɕɤzɤz} 
\classe{n} 
\begin{définition}\pfra{mets pour bien recevoir les invités}\end{définition}
\begin{définition}\pcmn{款待客人的食品}\end{définition}\étymologie{gtɕes.zas}\end{entrée}

\begin{entrée}{χtɕi}{}{ⓔχtɕi} 
\classe{vt} \sens{1}\paradigme{dir}{pɯ-}\paradigme{dir}{nɯ-}
\begin{définition}\pfra{laver}\end{définition}
\begin{définition}\pcmn{洗}\end{définition}
\begin{exemple}\pjya{nɤ-rŋa pɯ-χtɕi}\hspace{5pt}\pcmn{你洗脸吧}\end{exemple}
\begin{exemple}\pjya{nɤ-ŋga nɯ-χtɕi}\hspace{5pt}\pcmn{你洗衣服吧}\end{exemple}
\begin{exemple}\pjya{tɯ-ŋga nɯ-χtɕi-t-a}\hspace{5pt}\pcmn{我洗了衣服}\end{exemple}
\begin{exemple}\pjya{a-βri pɯ-χtɕi-t-a}\hspace{5pt}\pcmn{我洗了身体}\end{exemple}
\begin{exemple}\pjya{a-ɕɣa nɯ-χtɕi-t-a}\hspace{5pt}\pcmn{我刷了牙}\end{exemple}
\begin{exemple}\pjya{aʑo jɤ-azɣɯt-a nɯ tɕu, ɯʑo kɯ ɯ-jaʁ pjɤ-k-ɤsɯ-χtɕi-ci}\hspace{5pt}\pcmn{我到那里的时候,他正在洗手}\end{exemple}
\begin{exemple}\pjya{nɤʑo nɤ-ku ci pɯ-nɯ-χtɕi}\hspace{5pt}\pcmn{你洗一下头吧}\end{exemple}
\begin{exemple}\pjya{nɤ-jaʁ pɯ-nɯ-χtɕi}\hspace{5pt}\pcmn{你洗一下手吧}\end{exemple}
\begin{exemple}\pjya{kha thɯ-χtɕi}\hspace{5pt}\pcmn{洗一下房子吧}\end{exemple}\sens{2}\paradigme{dir}{pɯ-}
\begin{définition}\pfra{tremper (pluie)}\end{définition}
\begin{définition}\pcmn{淋湿(雨)}\end{définition}
\begin{exemple}\pjya{tɯ-mɯ kɯ pɯ́-wɣ-χtɕi-a}\hspace{5pt}\pcmn{我被雨淋湿了}\end{exemple}\relationsémantique{参考}{\lien{ⓔraχtɕɯʁɟo}{raχtɕɯʁɟo}}
\begin{sous-entrée}{raχtɕi}{ⓔχtɕiⓢ2ⓝraχtɕi} 
\classe{vi} \sens{1}\paradigme{dir}{pɯ-}
\begin{définition}\pfra{se laver le visage, se laver}\end{définition}
\begin{définition}\pcmn{洗脸;洗澡}\end{définition}
\begin{exemple}\pjya{ɯ-pɯ-tɯ-raχtɕi}\hspace{5pt}\pcmn{你洗了脸没有?}\end{exemple}
\begin{exemple}\pjya{pɯ-raχtɕi-a}\hspace{5pt}\pcmn{我洗了脸}\end{exemple}\end{sous-entrée}

\sens{2}\paradigme{dir}{nɯ-}
\begin{définition}\pfra{laver des choses}\end{définition}
\begin{définition}\pcmn{洗东西}\end{définition}
\begin{exemple}\pjya{jisŋi nɯ-raχtɕi-a}\hspace{5pt}\pcmn{今天洗了衣服}\end{exemple}
\begin{sous-entrée}{ʑɣɤχtɕi}{ⓔχtɕiⓢ2ⓝʑɣɤχtɕi} 
\classe{vi}  
\grammaire{refl} 
\begin{définition}\pfra{se laver}\end{définition}
\begin{définition}\pcmn{洗自己}\end{définition}\end{sous-entrée}

\end{entrée}

\begin{entrée}{χtɕoŋ}{}{ⓔχtɕoŋ} 
\classe{n} 
\begin{définition}\pfra{rhumatisme}\end{définition}
\begin{définition}\pcmn{风湿,关节疼}\end{définition}\étymologie{gtɕoŋ}\end{entrée}

\begin{entrée}{χtɕɯrɯ}{}{ⓔχtɕɯrɯ} 
\classe{n} 
\begin{définition}\pfra{nu}\end{définition}
\begin{définition}\pcmn{裸体}\end{définition}\relationsémantique{参考}{\lien{ⓔχtɕɯrɯpa}{χtɕɯrɯpa}}\relationsémantique{参考}{\lien{ⓔrɯχtɕɯrɯ}{rɯχtɕɯrɯ}}\relationsémantique{参考}{\lien{ⓔnɯχtɕɯrɯ}{nɯχtɕɯrɯ}}\end{entrée}

\begin{entrée}{χtɕɯrɯpa}{}{ⓔχtɕɯrɯpa} 
\classe{n} 
\begin{définition}\pfra{tout nu}\end{définition}
\begin{définition}\pcmn{裸体,不穿衣服}\end{définition}\relationsémantique{参考}{\lien{ⓔrɯχtɕɯrɯ}{rɯχtɕɯrɯ}}\étymologie{gtɕer.bu.pa}\end{entrée}

\begin{entrée}{χtorma}{}{ⓔχtorma} 
\classe{n} 
\begin{définition}\pfra{offrande au dieux}\end{définition}
\begin{définition}\pcmn{供奉鬼神的物品}\end{définition}\étymologie{gtor.ma}\end{entrée}

\begin{entrée}{χtsɤβ}{}{ⓔχtsɤβ} 
\classe{vt} \paradigme{dir}{nɯ-}\paradigme{dir}{pɯ-}
\begin{définition}\pfra{pétrir, tanner (peau)}\end{définition}
\begin{définition}\pcmn{揉}\end{définition}
\begin{exemple}\pjya{tɯ-ndʐi pa-χtsɤβ}\hspace{5pt}\pcmn{他揉了皮子}\end{exemple}
\begin{exemple}\pjya{pɯ́-wɣ-χtsaβ-a}\hspace{5pt}\pcmn{他揉了我}\end{exemple}
\begin{exemple}\pjya{tɤ-pɤtso, ma-pɯ-kɯ-χtsaβ-a}\hspace{5pt}\pcmn{小孩,你不要揉我(你不要一整天麻烦我)}\end{exemple}
\begin{exemple}\pjya{ɯ-χtsɤβ pɯ-ɣe}\hspace{5pt}\pcmn{(皮子)已经揉好了}\end{exemple}\end{entrée}

\begin{entrée}{χtshɤχtshɤt}{}{ⓔχtshɤχtshɤt} 
\classe{idph.2} 
\begin{définition}\pfra{sage et très actif (enfant, petit animal)}\end{définition}
\begin{définition}\pcmn{形容小孩子或者小动物看起来很乖很灵活的样子,小巧玲珑。}\end{définition}
\begin{exemple}\pjya{jiɕqha tɤ-pɤtso χtshɤχtshɤt ci ɲɯ-ŋu}\hspace{5pt}\pcmn{那个小孩子很灵活}\end{exemple}
\begin{exemple}\pjya{tshɯtho χtshɤχtshɤt ʑo ɲɯ-pa}\hspace{5pt}\pcmn{那头小羊羔很灵活}\end{exemple}
\begin{sous-entrée}{χtshɤnɤχtshɤt}{ⓔχtshɤχtshɤtⓝχtshɤnɤχtshɤt} 
\classe{idph.3} 
\begin{exemple}\pjya{tɤ-pɤtso nɯ χtshɤnɤχtshɤt ʑo ɯ-mu ɯ-phe ko-ɕe}\hspace{5pt}\pcmn{那个小孩子很敏捷地一下子就到了他母亲身边}\end{exemple}\end{sous-entrée}

\end{entrée}

\begin{entrée}{χtsiɯ}{}{ⓔχtsiɯ} 
\classe{n} 
\begin{définition}\pfra{unité de mesure}\end{définition}
\begin{définition}\pcmn{升}\end{définition}\end{entrée}

\begin{entrée}{χtso}{}{ⓔχtso} 
\classe{vs} \paradigme{dir}{tɤ-}
\begin{définition}\pfra{propre}\end{définition}
\begin{définition}\pcmn{干净(本质)}\end{définition}\paradigme{dir}{tɤ-}
\begin{définition}\pfra{rendre propre}\end{définition}
\begin{définition}\pcmn{弄干净}\end{définition}
\begin{exemple}\pjya{tɯ-ci ɲɯ-χtso}\hspace{5pt}\pcmn{水很干净}\end{exemple}
\begin{exemple}\pjya{ndzɤtshi ɲɯ-χtso}\hspace{5pt}\pcmn{食物很干净}\end{exemple}
\begin{sous-entrée}{ɣɤχtso}{ⓔχtsoⓝɣɤχtso} 
\classe{vt}  
\grammaire{caus} \end{sous-entrée}

\begin{sous-entrée}{naχtso}{ⓔχtsoⓝnaχtso} 
\classe{vt} 
\begin{définition}\pfra{trouver propre}\end{définition}
\begin{définition}\pcmn{觉得干净}\end{définition}
\begin{exemple}\pjya{mɤ-kɤ-naχtso nɯra s-chɯ́-wɣ-βde}\hspace{5pt}\pcmn{觉得不干净的东西就要扔掉}\end{exemple}\relationsémantique{同义词}{\lien{ⓔnaχɕɯn}{naχɕɯn}}\relationsémantique{反义词}{\lien{ⓔɴqhi}{ɴqhi}}\end{sous-entrée}

\étymologie{gtsaŋ}\end{entrée}

\begin{entrée}{χtsur}{}{ⓔχtsur} 
\classe{vs} 
\begin{définition}\pfra{important}\end{définition}
\begin{définition}\pcmn{重要}\end{définition}
\begin{sous-entrée}{raχtsur}{ⓔχtsurⓝraχtsur} 
\classe{vt} \paradigme{dir}{tɤ-}
\begin{définition}\pfra{considérer comme important}\end{définition}
\begin{définition}\pcmn{觉得重要}\end{définition}
\begin{exemple}\pjya{aʑo kɤ-nɤma ra tu-stu-a pɯ-ŋu tɕe, ŋgumdʑɯɣ ra kɯ tú-wɣ-raχtsur-a-nɯ pɯ-ŋu}\hspace{5pt}\pcmn{因为我工作得很努力,领导们很器重我}\end{exemple}
\begin{exemple}\pjya{kɤ-nɤma nɯ ra ri, kɤ-rɯndzɤtshi kɯnɤ kɤ-raχtsur ra}\hspace{5pt}\pcmn{工作是需要的,但是吃饭也重要}\end{exemple}\end{sous-entrée}

\end{entrée}

\begin{entrée}{χtsɯm}{}{ⓔχtsɯm} 
\classe{n} 
\begin{définition}\pfra{paille d'orge}\end{définition}
\begin{définition}\pcmn{青稞杆}\end{définition}\end{entrée}

\begin{entrée}{χtsɯmpapɯ}{}{ⓔχtsɯmpapɯ} 
\classe{n} 
\begin{définition}\pfra{paille d'orge en botte}\end{définition}
\begin{définition}\pcmn{一捆一捆的青稞杆}\end{définition}\relationsémantique{参考}{\lien{ⓔχtsɯm}{χtsɯm}}\end{entrée}

\begin{entrée}{χtsɯχtsri}{}{ⓔχtsɯχtsri} 
\classe{idph.2} 
\begin{définition}\pfra{les plumes / poils hérissés}\end{définition}
\begin{définition}\pcmn{形容(鸟;猫)把(羽)毛毛竖起来的样子}\end{définition}
\begin{exemple}\pjya{pɣɤtɕɯ nɯ to-sɤmbrɯ tɕe, ɯ-muj ra χtsɯχtsri ʑo to-sɯɣndzur.}\hspace{5pt}\pcmn{小鸟生气了就把羽毛竖起来了}\end{exemple}\end{entrée}

\begin{entrée}{χtʂɯɣdʑa}{}{ⓔχtʂɯɣdʑa} 
\classe{n} 
\begin{définition}\pfra{thé au beurre}\end{définition}
\begin{définition}\pcmn{酥油茶}\end{définition}\étymologie{dkrug.dʑa}\end{entrée}

\begin{entrée}{χtɯ}{}{ⓔχtɯ} 
\classe{vt} \paradigme{dir}{tɤ-}\paradigme{dir}{tɤ-}\paradigme{dir}{tɤ-}
\begin{définition}\pfra{acheter}\end{définition}
\begin{définition}\pcmn{买}\end{définition}
\begin{définition}\pfra{acheter des choses}\end{définition}
\begin{définition}\pcmn{买东西}\end{définition}
\begin{définition}\pfra{vendre à}\end{définition}
\begin{définition}\pcmn{使……买、卖给}\end{définition}
\begin{exemple}\pjya{mbarkhom lɤ-tɯ-ɣe ri @beimu tɤ-tɯ-χtɯ-t loβ, nɤ-ɕqhe ɯ-smɤn}\hspace{5pt}\pcmn{你上次来马尔康的时候,你买了贝母对吧,咳嗽的药}\end{exemple}
\begin{exemple}\pjya{jɯfɕɯr @cai mɯ-ɕ-tɤ-χtɯ-t-a}\hspace{5pt}\pcmn{我昨天没有去买菜}\end{exemple}
\begin{exemple}\pjya{aʑo kɤntɕhaʁ tɯ-ŋga ɯ-kɯ-χtɯ jɤ-ari-a}\hspace{5pt}\pcmn{我到街上去买衣服了(还没有买回来)}\end{exemple}
\begin{exemple}\pjya{tɤ-mthɯm ɕ-tɤ-χtɯ-t-a.}\hspace{5pt}\pcmn{我去买肉了(已经买回来了)}\end{exemple}
\begin{exemple}\pjya{@dianhua jɤ-tɯ-lɤt ri, aʑo pɯ-raχtɯ-a}\hspace{5pt}\pcmn{你打电话的时候我正在买东西}\end{exemple}
\begin{exemple}\pjya{nɤʑo jɤ-tɯ-ɤzɣɯt ri, aʑo tɤ-raχtɯ-a}\hspace{5pt}\pcmn{你到来的时候,我已经买好东西了}\end{exemple}
\begin{exemple}\pjya{ki tɯ-ŋga ki ɯʑo kɯ tɤ́-wɣ-sɯ-χtɯ-a ŋu (=a-ɕki na-ntsɣe)}\hspace{5pt}\pcmn{这件衣服是他卖给我的}\end{exemple}\relationsémantique{参考}{\lien{ⓔraχtɯtsɣe}{raχtɯtsɣe}}
\begin{sous-entrée}{raχtɯ}{ⓔχtɯⓝraχtɯ} 
\classe{vi}  
\grammaire{apass} \end{sous-entrée}

\begin{sous-entrée}{sɯχtɯ}{ⓔχtɯⓝsɯχtɯ} 
\classe{vt} \end{sous-entrée}

\end{entrée}

\begin{entrée}{χtɯɣ}{}{ⓔχtɯɣ} 
\classe{vt} \paradigme{dir}{tɤ-}
\begin{définition}\pfra{demander une arbitration (auprès de son supérieur)}\end{définition}
\begin{définition}\pcmn{(向上级)请求裁决;请上级评理}\end{définition}
\begin{exemple}\pjya{tɕiʑo kɤ-nɯkrɤz mɯ́j-cha-tɕi tɕe (tɕi-tɯkrɤz mɯ́j-ɣi tɕe), taʁ ɕ-tu-kɤ-χtɯɣ ɲɯ-ɬoʁ}\hspace{5pt}\pcmn{我们俩既然说不通,只好请上级评理}\end{exemple}
\begin{exemple}\pjya{nɤ-ɕki ɣɯ-tu-χtɯɣ-a ɲɯ-ɬoʁ}\hspace{5pt}\pcmn{我只好来求您了}\end{exemple}\étymologie{gtug}\end{entrée}

\begin{entrée}{χtɯmbrɤl}{}{ⓔχtɯmbrɤl} 
\classe{n} 
\begin{définition}\pfra{célébration}\end{définition}
\begin{définition}\pcmn{庆祝}\end{définition}\étymologie{rten.ⁿbrel}\end{entrée}

\begin{entrée}{χtɯn}{}{ⓔχtɯn} 
\classe{n} 
\begin{définition}\pfra{mortier}\end{définition}
\begin{définition}\pcmn{臼【坨坨】}\end{définition}\étymologie{gtun}\end{entrée}

\begin{entrée}{χɯχɯ}{}{ⓔχɯχɯ} 
\classe{idph.2} 
\begin{définition}\pfra{qui a des grandes narines}\end{définition}
\begin{définition}\pcmn{形容鼻孔很大的样子}\end{définition}
\begin{exemple}\pjya{ca kɯ ɯ-ɕna χɯχɯ ʑo to-stu}\hspace{5pt}\pcmn{麝香鹿把鼻孔弄得很大}\end{exemple}\end{entrée}

\begin{entrée}{χwara}{}{ⓔχwara} 
\classe{n} 
\begin{définition}\pfra{un type de tente}\end{définition}
\begin{définition}\pcmn{一种帐篷}\end{définition}\étymologie{sbra}\end{entrée}

\begin{entrée}{χwɤr}{}{ⓔχwɤr} 
\classe{n} 
\begin{définition}\pfra{Hor}\end{définition}
\begin{définition}\pcmn{霍尔}\end{définition}\étymologie{hor}\end{entrée}

\begin{entrée}{χwɤrχwɤr}{}{ⓔχwɤrχwɤr} 
\classe{idph.2} \sens{1}
\begin{définition}\pfra{abîmé et déchiré}\end{définition}
\begin{définition}\pcmn{形容破烂的样子}\end{définition}
\begin{exemple}\pjya{ɯ-ŋga mɯ́j-pe tɕe χwɤrχwɤr ʑo ɲɯ-pa}\hspace{5pt}\pcmn{我的衣服不好,破破烂烂的}\end{exemple}\sens{2}
\begin{définition}\pfra{ouvert (sac)}\end{définition}
\begin{définition}\pcmn{形容向外展开的样子}\end{définition}\end{entrée}

\begin{entrée}{χuχu}{}{ⓔχuχu} 
\classe{idph.2} 
\begin{définition}\pfra{qui a une petite ouverture}\end{définition}
\begin{définition}\pcmn{形容洞口很小}\end{définition}
\begin{exemple}\pjya{tɤ-pɤtso ɯ-mtɕhi ɲɯ-xtɕi χuχu ʑo ɲɯ-pa}\hspace{5pt}\pcmn{小孩子的嘴很小}\end{exemple}\end{entrée}

\newpage\caractère{z}

\begin{entrée}{zaŋ}{}{ⓔzaŋ} 
\classe{n} 
\begin{définition}\pfra{cuivre}\end{définition}
\begin{définition}\pcmn{红铜}\end{définition}\étymologie{zaŋs}\end{entrée}

\begin{entrée}{zaŋpoŋ}{}{ⓔzaŋpoŋ} 
\classe{n} 
\begin{définition}\pfra{ventouse}\end{définition}
\begin{définition}\pcmn{火罐(原来是用红铜铸的)}\end{définition}\étymologie{zaŋs.bum}\end{entrée}

\begin{entrée}{zaŋzaŋ}{}{ⓔzaŋzaŋ} 
\classe{idph.2} 
\begin{définition}\pfra{ébouriffés et longs (cheveux)}\end{définition}
\begin{définition}\pcmn{形容头发等乱蓬蓬样子}\end{définition}
\begin{exemple}\pjya{ɯ-ku tɤrpɯ zaŋzaŋ ʑo}\hspace{5pt}\pcmn{他的头发又长又脏,乱蓬蓬的}\end{exemple}\relationsémantique{参考}{\lien{ⓔzoŋzoŋ}{zoŋzoŋ}}\relationsémantique{参考}{\lien{ⓔdzaŋdzaŋ}{dzaŋdzaŋ}}\end{entrée}

\begin{entrée}{zaχtɤt}{}{ⓔzaχtɤt} 
\classe{n} 
\begin{définition}\pfra{nourriture pour les morts}\end{définition}
\begin{définition}\pcmn{供奉死人的食物}\end{définition}\relationsémantique{同义词}{\lien{ⓔzɤmpo}{zɤmpo}}\end{entrée}

\begin{entrée}{zɤjzɤj}{}{ⓔzɤjzɤj} 
\classe{idph.2} 
\begin{définition}\pfra{si petit qu'on a peine à le voir}\end{définition}
\begin{définition}\pcmn{形容小而难以看见的样子}\end{définition}
\begin{exemple}\pjya{tɕetu zgoku ri tɯrme ci zɤjzɤj ʑo ɲɯ-ndzur}\hspace{5pt}\pcmn{这座山上有人站着,模模糊糊地看不清楚}\end{exemple}\relationsémantique{同义词}{\lien{ⓔdzɤjdzɤj}{dzɤjdzɤj}}\end{entrée}

\begin{entrée}{zɤmpo}{}{ⓔzɤmpo} 
\classe{n} 
\begin{définition}\pfra{nourriture pour les morts}\end{définition}
\begin{définition}\pcmn{供奉死人的食物}\end{définition}\relationsémantique{参考}{\lien{ⓔnɯzɤmpo}{nɯzɤmpo}}\relationsémantique{同义词}{\lien{ⓔzaχtɤt}{zaχtɤt}}\end{entrée}

\begin{entrée}{zɤntshaʁ}{}{ⓔzɤntshaʁ} 
\classe{n} 
\begin{définition}\pfra{plat}\end{définition}
\begin{définition}\pcmn{食品}\end{définition}\end{entrée}

\begin{entrée}{zɤsna}{}{ⓔzɤsna} 
\classe{n} 
\begin{définition}\pfra{nourriture pour les morts}\end{définition}
\begin{définition}\pcmn{祭祀死人用的食物}\end{définition}
\begin{exemple}\pjya{tɯrme pɯ-si tɕe zɤsna chɯ́-wɣ-βde ŋu}\hspace{5pt}\pcmn{人去世了就要给他供奉一些食物}\end{exemple}\relationsémantique{参考}{\lien{ⓔzaχtɤt}{zaχtɤt}}\relationsémantique{参考}{\lien{ⓔnɯzɤsna}{nɯzɤsna}}\étymologie{zas.sna}\end{entrée}

\begin{entrée}{zɤt}{}{ⓔzɤt} 
\classe{vs} \paradigme{dir}{nɯ-}\sens{1}
\begin{définition}\pfra{s'émousser}\end{définition}
\begin{définition}\pcmn{磨损}\end{définition}
\begin{exemple}\pjya{qraʁ ɲɤ-zɤt}\hspace{5pt}\pcmn{铧磨损了}\end{exemple}\relationsémantique{同义词}{\lien{ⓔsa}{sa}}\sens{2}
\begin{définition}\pfra{disparaître}\end{définition}
\begin{définition}\pcmn{消失}\end{définition}
\begin{exemple}\pjya{jiʑo kɤmɲɯ-skɤt a-mɤ-nɯ-zɤt (a-mɤ-nɯ-mbrɤt, a-thɯ-ɤrɕo)}\hspace{5pt}\pcmn{但愿我们干木鸟话不会消失}\end{exemple}
\begin{sous-entrée}{sɯɣzɤt}{ⓔzɤtⓝsɯɣzɤt} 
\classe{vt} 
\begin{définition}\pfra{laisser disparaître}\end{définition}
\begin{définition}\pcmn{让……消失}\end{définition}
\begin{exemple}\pjya{kɤ-sɯɣzɤt nɤja tɕe a-mɤ-nɯ-zɤt}\hspace{5pt}\pcmn{令它消失太可惜,希望不会消失}\end{exemple}\end{sous-entrée}

\étymologie{zad}\end{entrée}

\begin{entrée}{zbaʁ}{}{ⓔzbaʁ} 
\classe{vs} \paradigme{dir}{tɤ-}\paradigme{dir}{tɤ-}
\begin{définition}\pfra{sec}\end{définition}
\begin{définition}\pcmn{干燥}\end{définition}
\begin{définition}\pfra{sécher}\end{définition}
\begin{définition}\pcmn{弄干}\end{définition}
\begin{exemple}\pjya{tɯ-nga to-zbaʁ}\hspace{5pt}\pcmn{衣服干了}\end{exemple}
\begin{exemple}\pjya{tɤ-ɕkho-t-a tɕe, to-zbaʁ}\hspace{5pt}\pcmn{我晾了就干了}\end{exemple}
\begin{exemple}\pjya{qale kɯ a-ŋga to-sɯzbaʁ}\hspace{5pt}\pcmn{风把衣服(吹干)了}\end{exemple}\relationsémantique{同义词}{\lien{ⓔrom}{rom}}\relationsémantique{同义词}{\lien{ⓔkhrɯ}{khrɯ}}\relationsémantique{反义词}{\lien{ⓔaci}{aci}}\relationsémantique{参考}{\lien{ⓔɣɤzbaʁⓗ1ⓝɣɤzbaʁ}{ɣɤzbaʁ}}
\begin{sous-entrée}{sɯzbaʁ}{ⓔzbaʁⓝsɯzbaʁ} 
\classe{vt}  
\grammaire{caus} \end{sous-entrée}

\end{entrée}

\begin{entrée}{zbaʁzbɯ}{}{ⓔzbaʁzbɯ} 
\classe{vs} 
\begin{définition}\pfra{sec}\end{définition}
\begin{définition}\pcmn{干}\end{définition}\relationsémantique{参考}{\lien{ⓔzbaʁ}{zbaʁ}}\end{entrée}

\begin{entrée}{zbraʁ}{}{ⓔzbraʁ} 
\classe{vt} \paradigme{dir}{nɯ-}\paradigme{dir}{kɤ-}
\begin{définition}\pfra{attacher sur quelque chose}\end{définition}
\begin{définition}\pcmn{绑在另外一个物体}\end{définition}
\begin{exemple}\pjya{tɤ-jtsi ɯ-taʁ ɲɤ-zbraʁ}\hspace{5pt}\pcmn{把他绑在柱子上了}\end{exemple}\relationsémantique{参考}{\lien{ⓔβraʁ}{βraʁ}}
\begin{sous-entrée}{azbraʁ}{ⓔzbraʁⓝazbraʁ} 
\classe{vi}  
\grammaire{pass} 
\begin{définition}\pfra{être attaché sur quelque chose}\end{définition}
\begin{définition}\pcmn{被绑在另外一个物体上}\end{définition}\end{sous-entrée}

\end{entrée}

\begin{entrée}{zbrilu}{}{ⓔzbrilu} 
\classe{n} 
\begin{définition}\pfra{année du serpent}\end{définition}
\begin{définition}\pcmn{蛇年}\end{définition}\étymologie{sbrul.lo}\end{entrée}

\begin{entrée}{zbɯ}{}{ⓔzbɯ} 
\classe{n}  
\grammaire{n.lieu} 
\begin{définition}\pfra{Zbu}\end{définition}
\begin{définition}\pcmn{日部乡}\end{définition}\end{entrée}

\begin{entrée}{zbɯɣ}{}{ⓔzbɯɣ} 
\classe{idph.1} 
\begin{définition}\pfra{bruit d'une falaise qui s'écroule}\end{définition}
\begin{définition}\pcmn{形容房间里突然密闭昏闷的感觉,或岩石突然倒塌的声音}\end{définition}
\begin{exemple}\pjya{kɯm zbɯɣ ʑo ta-stu}\hspace{5pt}\pcmn{他关了门,房间里一下子就闷了起来}\end{exemple}\end{entrée}

\begin{entrée}{zbɯwa}{}{ⓔzbɯwa} 
\classe{n} 
\begin{définition}\pfra{lanière pour porter les enfants sur le dos}\end{définition}
\begin{définition}\pcmn{用来背小孩的带子}\end{définition}\relationsémantique{参考}{\lien{ⓔbɯwa}{bɯwa}}\end{entrée}

\begin{entrée}{zdɤβ}{}{ⓔzdɤβ} 
\classe{vt} \paradigme{dir}{tɤ-}
\begin{définition}\pfra{plier}\end{définition}
\begin{définition}\pcmn{折叠;裹起来}\end{définition}
\begin{exemple}\pjya{tɯmbri ta-zdɤβ}\hspace{5pt}\pcmn{他把绳子折起来了(弄成双股)}\end{exemple}\relationsémantique{同义词}{\lien{ⓔltɤβ}{ltɤβ}}\étymologie{sdeb}\end{entrée}

\begin{entrée}{zdoŋbu}{}{ⓔzdoŋbu} 
\classe{n} 
\begin{définition}\pfra{tronc coupé}\end{définition}
\begin{définition}\pcmn{大木头}\end{définition}\étymologie{sdoŋ.po}\end{entrée}

\begin{entrée}{zdoʁzdoʁ}{}{ⓔzdoʁzdoʁ} 
\classe{idph.2} 
\begin{définition}\pfra{petit et vif}\end{définition}
\begin{définition}\pcmn{形容小而鲜活的样子}\end{définition}
\begin{exemple}\pjya{kumpɣɤtɕɯ nɯ zdoʁzdoʁ ʑo ɲɯ-pa}\hspace{5pt}\pcmn{麻雀小巧玲珑}\end{exemple}\relationsémantique{参考}{\lien{ⓔɣɤzdoʁloʁ}{ɣɤzdoʁloʁ}}\end{entrée}

\begin{entrée}{zdɯɣ}{}{ⓔzdɯɣ} 
\classe{vi} \paradigme{dir}{pɯ-}
\begin{définition}\pfra{pénible}\end{définition}
\begin{définition}\pcmn{辛苦}\end{définition}
\begin{exemple}\pjya{ɯʑo ɲɯ-zdɯɣ}\hspace{5pt}\pcmn{他很辛苦}\end{exemple}\étymologie{sdug}\end{entrée}

\begin{entrée}{zdɯɬa}{}{ⓔzdɯɬa} 
\classe{n} 
\begin{définition}\pfra{Paeonia sp.}\end{définition}
\begin{définition}\pcmn{芍药}\end{définition}
\begin{exemple}\pjya{zdɯɬa nɯ zgoku kɯ-mbro kɯ-mbɤr aʁɤndɯndɤt sɯŋgɯ tu-ɬoʁ cha. sɯjno mɤ-mbro tɯ-phɯ ɯ-ŋgɯ ɯ-ru lɤŋɤtʂɤ-ldʑa tu-ɬoʁ cha. tɯ-ldʑa ma kɯ-me ci tu. ɯ-mɯntoʁ kɯ-ɣɯrni ɲɯ-kɯ-lɤt ci tu, nɯ ɣɯ ɯ-qa ɯ-zrɤm nɯ ɣɯrni, mɤʑɯ tɯ-tɯphu tɕe, ɯ-mɯntoʁ kɯ-wɣrum ci tu tɕe, nɯ ɣɯ ɯ-qa ɯ-zrɤm nɯ wɣrum. ɯ-mɯntoʁ wxti, tɯ-rdoʁ tɯ-rdoʁ ɲɯ-lɤt ŋu. ɯ-jwaʁ nɯ thɯ-kɤ-rɤfɕɯfɕɤt fse, ʁnɯ-tɯphu ni ndʑi-jwaʁ ra naχtɕɯɣ. smɤn ɲɯ-sna. ɯ-di ɕŋaʁ ci tu.}\hspace{5pt}\pcmn{芍药生长在山上山下的森林里。这种草长得不高,有的一棵可以长出五、六根茎,有的只长一根茎。有的开红花,根也是红色的,有的开白花,根也是白色的。花很大,一朵一朵地开,叶子好像是被撕下来的一样。两种芍药的叶子相同。可以入药。有一股臭味。}\end{exemple}\end{entrée}

\begin{entrée}{zdɯm}{}{ⓔzdɯm} 
\classe{n} 
\begin{définition}\pfra{nuage, brume}\end{définition}
\begin{définition}\pcmn{云;雾}\end{définition}\end{entrée}

\begin{entrée}{zdɯmkhɤtɕhɯ}{}{ⓔzdɯmkhɤtɕhɯ} 
\classe{n} 
\begin{définition}\pfra{humidité de la brume}\end{définition}
\begin{définition}\pcmn{早上起雾时的水分}\end{définition}
\begin{exemple}\pjya{zdɯm lɤ-ɣe tɕe, zdɯmkhɤtɕhɯ ɣɤʑu}\hspace{5pt}\pcmn{起雾的时候空气中就有水分}\end{exemple}
\begin{exemple}\pjya{zdɯmkhɤtɕhɯ kɯ tɯ-ŋga ɲɤ-znɯrlɤn}\hspace{5pt}\pcmn{雾气把衣服弄湿了}\end{exemple}
\begin{exemple}\pjya{tɯ-mɯ zdɯmkhɤtɕhɯ jamar ɲɯ-ɤsɯ-lɤt}\hspace{5pt}\pcmn{下雨了,雨点又细又密}\end{exemple}\end{entrée}

\begin{entrée}{zdɯmlaʁrɯʁrɯ}{}{ⓔzdɯmlaʁrɯʁrɯ} 
\classe{n} 
\begin{définition}\pfra{escargot}\end{définition}
\begin{définition}\pcmn{蜗牛}\end{définition}
\begin{exemple}\pjya{zdɯmlaʁrɯʁrɯ nɯ qajɯ ci ŋu, ɯ-rqhu tu, ɯ-rqhu nɯ kɯ-ɤrtɯm tɕe ɯ-ŋgɯ kɯ-spoʁ ɲɯ-ŋu, ɯʑo nɯ ɯ-rqhu ɯ-ŋgɯ nɯ tɕu ku-rɤʑi ŋu, ftɕar tɕe ɯ-ŋgɯ chɯ-nɯɬoʁ tɕe, ɯ-rqhu nɯ ɯ-mgɯr ɯ-taʁ ɲɯ-ndzoʁ tɕe tu-nɯfkɯfkur ɲɯ-ŋu, ɯʑo ɯ-phoŋbu nɯ ra kɯ-mpɯ-mpɯ ʑo ɲɯ-ŋu, ɯ-ɕɤrɯ maŋe, ɯ-ʁrɯ kɯnɤ kɯ-mpɯ-mpɯ ɲɯ-ŋu, tɕe ci ci lu-tɕɤt, ci ci chɯ-sɯ-ɕqhlɤt ɲɯ-ŋu, qartsɯ tɕe ɯ-ŋgɯ chɯ-ɕqhlɤt tɕe, kɤ-mto maŋe, ɯ-rqhu nɯ cɯrmbɯ ɯ-ŋgɯ aʁɤndɯndɤt ɣɤʑu.}\hspace{5pt}\pcmn{蜗牛是一种虫子,有壳。壳是圆形的,有洞。它自己住在洞里面,夏天从壳里出来,把壳背来背去。它身子全是软软的,没有骨头,触角也是软软的,有时候伸出来,有时候收回去。冬天它缩进壳里,看不见。壳在石头堆里到处可以看到。}\end{exemple}\end{entrée}

\begin{entrée}{zdɯmqe}{}{ⓔzdɯmqe} 
\classe{n} 
\begin{définition}\pfra{une espèce de champignon}\end{définition}
\begin{définition}\pcmn{一种菌子}\end{définition}
\begin{exemple}\pjya{zdɯmqe nɯ ɕɤr tɯ-mɯ kɤ-lɤt tɕe, soz tɕe stɤmku xɕaj ɣɯ ɯ-rchɤβ kɯ-ɤʁɟa nɯ ra ku-ndzoʁ ŋu. kɯ-ɤlɤɣɯ ŋu ma tɯ-rdoʁ tɯ-rdoʁ maʁ, tɤjmɤɣ fse ri tɯ-mɯ tɤ-jɯm tɕe ɲɯ-me ɕti}\hspace{5pt}\pcmn{\lien{ⓔzdɯmqe}{zdɯmqe}生长在草地、草丛之间的空地上。一般晚上下雨,早上有雾和雾刚散的时候才会长出。它不是单独生长的,也没有规则形状,像菌子,但是天一晴就会消失。}\end{exemple}\end{entrée}

\begin{entrée}{zdɯxthɯɣ}{}{ⓔzdɯxthɯɣ} 
\classe{n} 
\begin{définition}\pfra{à la limite de l'acceptable}\end{définition}
\begin{définition}\pcmn{勉强可以,不理想}\end{définition}
\begin{exemple}\pjya{a-kɤ-nɤma zdɯxthɯɣ ʑo nɯ-sthɯt-a}\hspace{5pt}\pcmn{我工作完成得差强人意}\end{exemple}\étymologie{sdug.tʰug}\étymologie{sdug.tʰug}\end{entrée}

\begin{entrée}{zdɯzdu}{}{ⓔzdɯzdu} 
\classe{idph.2} 
\begin{définition}\pfra{petit, rond et dur}\end{définition}
\begin{définition}\pcmn{形容圆、硬而小的感觉}\end{définition}
\begin{exemple}\pjya{tɤ-pɤtso ɲɯ-sɤjndɤt ɯ-jaʁ ra zdɯzdu ʑo ɲɯ-pa}\hspace{5pt}\pcmn{小孩子很可爱,手又小又圆}\end{exemple}\end{entrée}

\begin{entrée}{zdɯzdɯr/\variante{zdɯrzdɯr}}{}{ⓔzdɯzdɯr} 
\classe{idph.2} 
\begin{définition}\pfra{objet ronds et petits}\end{définition}
\begin{définition}\pcmn{圆形,很细小的东西(如珠子、豌豆等)}\end{définition}\relationsémantique{参考}{\lien{ⓔɣɤzdɯzdɯr}{ɣɤzdɯzdɯr}}\relationsémantique{参考}{\lien{ⓔstɯrstɯr}{stɯrstɯr}}\end{entrée}

\begin{entrée}{zga}{₂}{ⓔzgaⓗ2} 
\classe{n} 
\begin{définition}\pfra{sauce}\end{définition}
\begin{définition}\pcmn{酱}\end{définition}\end{entrée}

\begin{entrée}{zga}{₁}{ⓔzgaⓗ1} 
\classe{vs} \paradigme{dir}{tɤ-}
\begin{définition}\pfra{être mûr (abcès)}\end{définition}
\begin{définition}\pcmn{成熟(脓包、粉刺)}\end{définition}
\begin{exemple}\pjya{tɯ-ɣmbɤβ to-zga tɕe kɤ-tɕɣaʁ to-mda}\hspace{5pt}\pcmn{脓包成熟了,可以挤了}\end{exemple}\end{entrée}

\begin{entrée}{zgɤr}{}{ⓔzgɤr} 
\classe{n} 
\begin{définition}\pfra{tente}\end{définition}
\begin{définition}\pcmn{帐篷(棉布制成)}\end{définition}
\begin{exemple}\pjya{mbroχpa kɯ zgɤr cho-thɯ}\hspace{5pt}\pcmn{牧民搭了帐篷}\end{exemple}
\begin{exemple}\pjya{zgɤr thɯ-tʂɯβ-i}\hspace{5pt}\pcmn{我们缝了帐篷}\end{exemple}\étymologie{sgar}\end{entrée}

\begin{entrée}{zgɤrɕaŋ}{}{ⓔzgɤrɕaŋ} 
\classe{n} 
\begin{définition}\pfra{mât de tente}\end{définition}
\begin{définition}\pcmn{帐篷杆子}\end{définition}\étymologie{sgar.ɕiŋ}\end{entrée}

\begin{entrée}{zgɤrtshoʁ}{}{ⓔzgɤrtshoʁ} 
\classe{n} 
\begin{définition}\pfra{piquet de tente}\end{définition}
\begin{définition}\pcmn{帐篷桩}\end{définition}\end{entrée}

\begin{entrée}{zgɤt}{}{ⓔzgɤt} 
\classe{vi.nh} 
\begin{définition}\pfra{devoir}\end{définition}
\begin{définition}\pcmn{应该}\end{définition}
\begin{exemple}\pjya{ɲɯ-khɤm zgɤt}\hspace{5pt}\pcmn{他应该给}\end{exemple}
\begin{exemple}\pjya{ɲɯ-kham-a zgɤt}\hspace{5pt}\pcmn{我应该给}\end{exemple}
\begin{exemple}\pjya{tu-zrɯwxtɯwxti-a zgɤt}\hspace{5pt}\pcmn{我应该尊重他}\end{exemple}
\begin{exemple}\pjya{ɯ-koŋ nɯ jamar ɲɯ-zgɤt}\hspace{5pt}\pcmn{这个东西值这个价}\end{exemple}\relationsémantique{同义词}{\lien{ⓔtʂaŋ}{tʂaŋ}}\end{entrée}

\begin{entrée}{zgo}{}{ⓔzgo} 
\classe{n} 
\begin{définition}\pfra{montagne}\end{définition}
\begin{définition}\pcmn{山}\end{définition}\relationsémantique{同义词}{\lien{ⓔtɤmbɤt}{tɤmbɤt}}\end{entrée}

\begin{entrée}{zgoco}{}{ⓔzgoco} 
\classe{n} 
\begin{définition}\pfra{vallée}\end{définition}
\begin{définition}\pcmn{山沟}\end{définition}\end{entrée}

\begin{entrée}{zgoku}{}{ⓔzgoku} 
\classe{n} 
\begin{définition}\pfra{pente}\end{définition}
\begin{définition}\pcmn{山坡}\end{définition}\end{entrée}

\begin{entrée}{zgomdʑo}{}{ⓔzgomdʑo} 
\classe{n} 
\begin{définition}\pfra{nom d'une fête}\end{définition}
\begin{définition}\pcmn{看花节}\end{définition}\end{entrée}

\begin{entrée}{zgoŋzgoŋ}{}{ⓔzgoŋzgoŋ} 
\classe{idph.2} 
\begin{définition}\pfra{courbé}\end{définition}
\begin{définition}\pcmn{形容弓起来的样子}\end{définition}
\begin{exemple}\pjya{rgɤtpu nɯ ɯ-phoŋbu zgoŋzgoŋ ʑo ɲɯ-pa}\hspace{5pt}\pcmn{老年人背是弓着的}\end{exemple}\end{entrée}

\begin{entrée}{zgoʁ}{}{ⓔzgoʁ} 
\classe{idph.1} 
\begin{définition}\pfra{tout d'un coup (s'agenouiller)}\end{définition}
\begin{définition}\pcmn{一下子(跪下)}\end{définition}
\begin{exemple}\pjya{ɯ-χpɯm zgoʁ ʑo pjɤ-tshoʁ}\hspace{5pt}\pcmn{他一下子跪下了(很恭敬的样子)}\end{exemple}\relationsémantique{同义词}{\lien{ⓔdzoʁ}{dzoʁ}}\relationsémantique{同义词}{\lien{ⓔgoʁ}{goʁ}}\end{entrée}

\begin{entrée}{zgotɕɯ}{}{ⓔzgotɕɯ} 
\classe{n} 
\begin{définition}\pfra{pente}\end{définition}
\begin{définition}\pcmn{小山坡}\end{définition}\end{entrée}

\begin{entrée}{zgrawa}{}{ⓔzgrawa} 
\classe{n} 
\begin{définition}\pfra{sac en cuir}\end{définition}
\begin{définition}\pcmn{用牛皮缝成的口袋}\end{définition}
\begin{exemple}\pjya{qartshaz ɯ-ndʐi zgrawa}\hspace{5pt}\pcmn{鹿皮制成的口袋}\end{exemple}\étymologie{sgra.ba}\end{entrée}

\begin{entrée}{zgri}{}{ⓔzgri} 
\classe{n} 
\begin{définition}\pfra{espèce d'herbe}\end{définition}
\begin{définition}\pcmn{草的一种}\end{définition}\relationsémantique{同义词}{\lien{ⓔmɯrkuj}{mɯrkuj}}\end{entrée}

\begin{entrée}{zgroʁ}{₂}{ⓔzgroʁⓗ2} 
\classe{n} 
\begin{définition}\pfra{bracelet}\end{définition}
\begin{définition}\pcmn{手镯}\end{définition}\étymologie{sgrog}\end{entrée}

\begin{entrée}{zgroʁ}{₁}{ⓔzgroʁⓗ1} 
\classe{vt} \paradigme{dir}{tɤ-}
\begin{définition}\pfra{attacher}\end{définition}
\begin{définition}\pcmn{绑}\end{définition}
\begin{exemple}\pjya{laχtɕha to-zgroʁ}\hspace{5pt}\pcmn{他把东西捆起来了}\end{exemple}
\begin{exemple}\pjya{ɯ-fkur to-zgroʁ}\hspace{5pt}\pcmn{他把背包捆起来了}\end{exemple}
\begin{exemple}\pjya{tó-wɣ-zgroʁ}\hspace{5pt}\pcmn{他被绑起来了}\end{exemple}
\begin{exemple}\pjya{sɤrŋgɯŋga tɤ-zgroʁ-a}\hspace{5pt}\pcmn{我把床单捆起来了(要出发的时候)}\end{exemple}
\begin{sous-entrée}{azgroʁ}{ⓔzgroʁⓗ1ⓝazgroʁ} 
\classe{vi}  
\grammaire{pass} 
\begin{définition}\pfra{être attaché}\end{définition}
\begin{définition}\pcmn{被绑}\end{définition}\end{sous-entrée}

\étymologie{sgrog}\end{entrée}

\begin{entrée}{zgrɯβ}{}{ⓔzgrɯβ} 
\classe{vi} \paradigme{dir}{nɯ-}
\begin{définition}\pfra{faire avec toute son énergie}\end{définition}
\begin{définition}\pcmn{一心一意地做一件事情}\end{définition}
\begin{exemple}\pjya{tɕhɤz kɤ-zgrɯβ}\hspace{5pt}\pcmn{修佛法}\end{exemple}\étymologie{sgrub}\end{entrée}

\begin{entrée}{zgrɯl}{}{ⓔzgrɯl} 
\classe{vt} \paradigme{dir}{nɯ-}
\begin{définition}\pfra{rouler entre les mains (sens inverse des aiguilles d'une montre)}\end{définition}
\begin{définition}\pcmn{搓线(逆时针)}\end{définition}
\begin{exemple}\pjya{tɤ-ri na-zgrɯl}\hspace{5pt}\pcmn{他搓了线}\end{exemple}
\begin{exemple}\pjya{tɯmbri na-zgrɯl}\hspace{5pt}\pcmn{他搓了绳子}\end{exemple}
\begin{exemple}\pjya{nɯ-zgrɯl-a}\hspace{5pt}\pcmn{我搓了}\end{exemple}\relationsémantique{参考}{\lien{ⓔrɤjɯɣ}{rɤjɯɣ}}\étymologie{sgril}\end{entrée}

\begin{entrée}{zgrɯtɕhɯ}{}{ⓔzgrɯtɕhɯ} 
\classe{n} 
\begin{définition}\pfra{coup de coude}\end{définition}
\begin{définition}\pcmn{一肘(打)}\end{définition}
\begin{exemple}\pjya{zgrɯtɕhɯ tɤ-lat-a}\hspace{5pt}\pcmn{我打了一肘}\end{exemple}
\begin{exemple}\pjya{zgrɯtɕhɯ ma-tɤ-tɯ-lɤt}\hspace{5pt}\pcmn{你不要用肘打人}\end{exemple}\relationsémantique{参考}{\lien{ⓔnɯzgrɯtɕhɯ}{nɯzgrɯtɕhɯ}}\relationsémantique{参考}{\lien{ⓔtɯ-zgrɯ}{tɯ-zgrɯ}}\relationsémantique{参考}{\lien{ⓔtɕhɯ}{tɕhɯ}}\end{entrée}

\begin{entrée}{zgɯrmɯɣ}{}{ⓔzgɯrmɯɣ} 
\classe{n} 
\begin{définition}\pfra{mousse}\end{définition}
\begin{définition}\pcmn{青苔【木路苏】}\end{définition}\end{entrée}

\begin{entrée}{zgɯrwɯ}{}{ⓔzgɯrwɯ} 
\classe{n} 
\begin{définition}\pfra{bosse}\end{définition}
\begin{définition}\pcmn{驼背}\end{définition}\étymologie{sgur.ba}\end{entrée}

\begin{entrée}{zgɯt}{}{ⓔzgɯt} 
\classe{vi} \paradigme{dir}{kɤ-}
\begin{définition}\pfra{rétrécir (habits)}\end{définition}
\begin{définition}\pcmn{缩水(衣服)}\end{définition}
\begin{exemple}\pjya{tɯ-ŋga nɯ-χtɕi-t-a ri ko-zgɯt}\hspace{5pt}\pcmn{我洗了衣服就缩水了}\end{exemple}\end{entrée}

\begin{entrée}{zɣa}{}{ⓔzɣa} 
\classe{vs} \paradigme{dir}{pɯ-}
\begin{définition}\pfra{normalement il devrait}\end{définition}
\begin{définition}\pcmn{按理来说应该……}\end{définition}
\begin{exemple}\pjya{jɯfɕɯr tɯ-mɯ kɯ-lɤt pɯ-zɣa}\hspace{5pt}\pcmn{按道理,昨天晚上应该下雨(结果没有下雨)}\end{exemple}
\begin{exemple}\pjya{qale kɯ-βzu pɯ-zɣa}\hspace{5pt}\pcmn{按道理,应该有风}\end{exemple}
\begin{exemple}\pjya{tɯ-mɯ kɯ-jɯm pɯ-zɣa}\hspace{5pt}\pcmn{按道理,应该天晴}\end{exemple}
\begin{exemple}\pjya{pjɯ-ɕaβ ɲɯ-zɣa ri mɯ́j-ɕaβ}\hspace{5pt}\pcmn{按道理应该够长,但是不够长}\end{exemple}
\begin{exemple}\pjya{tʂu mɤ-kɯ-zɣa ʑo pjɤ-mbɯt}\hspace{5pt}\pcmn{路莫名其妙地塌下来了}\end{exemple}\end{entrée}

\begin{entrée}{zɣɤβlo}{}{ⓔzɣɤβlo}\relationsémantique{参考}{\lien{ⓔɣɤβlo}{ɣɤβlo}}\end{entrée}

\begin{entrée}{zɣɤɕɯɴqoʁ}{}{ⓔzɣɤɕɯɴqoʁ}\relationsémantique{参考}{\lien{ⓔɕɯɴqoʁ}{ɕɯɴqoʁ}}\end{entrée}

\begin{entrée}{zɣɤdi}{}{ⓔzɣɤdi}\relationsémantique{参考}{\lien{ⓔɣɤdi}{ɣɤdi}}\end{entrée}

\begin{entrée}{zɣɤji}{}{ⓔzɣɤji}\relationsémantique{参考}{\lien{ⓔɣɤji}{ɣɤji}}\end{entrée}

\begin{entrée}{zɣɤmbu}{}{ⓔzɣɤmbu} 
\classe{n} 
\begin{définition}\pfra{balai}\end{définition}
\begin{définition}\pcmn{扫帚}\end{définition}\end{entrée}

\begin{entrée}{zɣɤngɯt}{}{ⓔzɣɤngɯt}\relationsémantique{参考}{\lien{ⓔɣɤngɯt}{ɣɤngɯt}}\end{entrée}

\begin{entrée}{zɣɤŋgi}{}{ⓔzɣɤŋgi}\relationsémantique{参考}{\lien{ⓔɣɤŋgi}{ɣɤŋgi}}\end{entrée}

\begin{entrée}{zɣɤrzɣɤr}{}{ⓔzɣɤrzɣɤr} 
\classe{idph.2} 
\begin{définition}\pfra{qui prend de la place mais qui n'est pas lourd}\end{définition}
\begin{définition}\pcmn{形容松软的东西(棉絮、泡沫塑料等)虽然不重,但体积很大,很占地方的样子}\end{définition}\end{entrée}

\begin{entrée}{zɣɤʁre}{}{ⓔzɣɤʁre}\relationsémantique{参考}{\lien{ⓔɣɤʁre}{ɣɤʁre}}\end{entrée}

\begin{entrée}{zɣɤtɕa}{}{ⓔzɣɤtɕa}\relationsémantique{参考}{\lien{ⓔɣɤtɕa}{ɣɤtɕa}}\end{entrée}

\begin{entrée}{zɣɤwu}{}{ⓔzɣɤwu}\relationsémantique{参考}{\lien{ⓔɣɤwu}{ɣɤwu}}\end{entrée}

\begin{entrée}{zɣoma}{}{ⓔzɣoma} 
\classe{n} 
\begin{définition}\pfra{lie}\end{définition}
\begin{définition}\pcmn{酒糟}\end{définition}\end{entrée}

\begin{entrée}{zɣɯmphrɯmphru/\variante{\_zɣɯmphɯmphru}}{}{ⓔzɣɯmphrɯmphru} 
\classe{vt} \paradigme{dir}{pɯ-}
\begin{définition}\pfra{faire en continu, à de nombreuses reprises}\end{définition}
\begin{définition}\pcmn{接二连三地做}\end{définition}
\begin{exemple}\pjya{tɯ-fkur tɕɤkɯ a-pɯ-ɤta, nɯ z-ɲɯ́-wɣ-zɣɯmphrɯmphru jɤɣ}\hspace{5pt}\pcmn{那一背(东西)放在那边,可以接着把它背过来}\end{exemple}
\begin{exemple}\pjya{sɯfkur ɕ-pɯ-zɣɯmprɯmphru-t-a}\hspace{5pt}\pcmn{我接二连三地去背了柴}\end{exemple}
\begin{exemple}\pjya{pɤjkhu kutɕu ɯ-skɤt kɤ-zɣɯmphɯmphru ʑo kɤ-ti mɯ́j-cha}\hspace{5pt}\pcmn{这里的话,我还不能讲得很流利}\end{exemple}\end{entrée}

\begin{entrée}{zɣɯmphɯmphru}{}{ⓔzɣɯmphɯmphru}\relationsémantique{参考}{\lien{ⓔaɣɯmphɯmphru}{aɣɯmphɯmphru}}\end{entrée}

\begin{entrée}{zɣɯŋgɯŋgɯ}{}{ⓔzɣɯŋgɯŋgɯ}\relationsémantique{参考}{\lien{ⓔaɣɯŋgɯŋgɯ}{aɣɯŋgɯŋgɯ}}\end{entrée}

\begin{entrée}{zɣɯqhu}{}{ⓔzɣɯqhu} 
\classe{n} 
\begin{définition}\pfra{partie du fardeau opposée au dos du porteur}\end{définition}
\begin{définition}\pcmn{柴捆子的后边部分,不接触人的背部(比较粗的木柴)}\end{définition}\relationsémantique{反义词}{\lien{ⓔzrɯβɟu}{zrɯβɟu}}\end{entrée}

\begin{entrée}{zɣɯrkɯrkɯ}{}{ⓔzɣɯrkɯrkɯ}\relationsémantique{参考}{\lien{ⓔaɣɯrkɯrkɯ}{aɣɯrkɯrkɯ}}\end{entrée}

\begin{entrée}{zɣɯrndi}{}{ⓔzɣɯrndi} 
\classe{n} 
\begin{définition}\pfra{offrandes rituelles}\end{définition}
\begin{définition}\pcmn{(经堂上的)贡品}\end{définition}\end{entrée}

\begin{entrée}{zɣɯrni}{}{ⓔzɣɯrni}\relationsémantique{参考}{\lien{ⓔɣɯrni}{ɣɯrni}}\end{entrée}

\begin{entrée}{zɣɯt}{}{ⓔzɣɯt} 
\classe{vi} \paradigme{dir}{\_}\paradigme{perfective stem (1st and 3th persons)}{azɣɯt}
\begin{définition}\pfra{arriver}\end{définition}
\begin{définition}\pcmn{到达}\end{définition}
\begin{exemple}\pjya{ʑa jo-tɯ-ʑɣɯt}\hspace{5pt}\pcmn{你早就到了}\end{exemple}
\begin{exemple}\pjya{ʑa jɤ-azɣɯt-a}\hspace{5pt}\pcmn{我早就到了}\end{exemple}
\begin{exemple}\pjya{nɯ kóʁmɯz lɤ-azɣɯt loβ}\hspace{5pt}\pcmn{他刚刚才到呢}\end{exemple}
\begin{exemple}\pjya{a-jaʁ jɤ-azɣɯt}\hspace{5pt}\pcmn{我收到了}\end{exemple}
\begin{exemple}\pjya{mɯ-ɕɯ-tɯ-zɣɯt nɯ-sɯso-t-a}\hspace{5pt}\pcmn{我怕你到达不了}\end{exemple}
\begin{exemple}\pjya{a-slama nɯ laʁnɤ-rʑaʁ tɕe nɤ-ɕki zɣɯt}\hspace{5pt}\pcmn{我的学生过几天就会到你那里}\end{exemple}
\begin{sous-entrée}{nɯzɣɯt}{ⓔzɣɯtⓝnɯzɣɯt} 
\classe{vi}  
\grammaire{vert} 
\begin{définition}\pfra{rentrer chez soi}\end{définition}
\begin{définition}\pcmn{安全回家}\end{définition}\end{sous-entrée}

\end{entrée}

\begin{entrée}{zɣɯtɕɯtɕɤβ}{}{ⓔzɣɯtɕɯtɕɤβ} 
\classe{vt} \paradigme{dir}{tɤ-}
\begin{définition}\pfra{entrecroiser, assortir l'un après l'autre}\end{définition}
\begin{définition}\pcmn{连续地搭配(不同的颜色)}\end{définition}
\begin{exemple}\pjya{ʑaka ɯ-mdoʁ khatoʁ tú-wɣ-zɣɯtɕɯtɕɤβ tɕe mpɕɤr}\hspace{5pt}\pcmn{把每个颜色搭配成一路一路就美观}\end{exemple}
\begin{exemple}\pjya{tɯ-kɯ-mŋɤm tɤ-tu tɕe, kɤ-nɯna cho kɤ-rɤma tú-wɣ-zɣɯtɕɯtɕɤβ tɕe pe}\hspace{5pt}\pcmn{生病的时候,最好把工作和休息调理好}\end{exemple}\end{entrée}

\begin{entrée}{zjaŋzjaŋ}{}{ⓔzjaŋzjaŋ} 
\classe{idph.2} 
\begin{définition}\pfra{haut}\end{définition}
\begin{définition}\pcmn{身子高(比其他人高)}\end{définition}
\begin{exemple}\pjya{mbro ɯ-taʁ zjaŋzjaŋ to-ɕe}\hspace{5pt}\pcmn{他骑上了马,显得很高}\end{exemple}
\begin{sous-entrée}{zjaŋnɤzjaŋ}{ⓔzjaŋzjaŋⓝzjaŋnɤzjaŋ} 
\classe{idph.3} 
\begin{exemple}\pjya{mbro ɯ-taʁ to-ɕe tɕe, zjaŋnɤzjaŋ jɤ-ari-ndʑi}\hspace{5pt}\pcmn{他骑上了马,就往上游去了,显得很高}\end{exemple}\end{sous-entrée}

\begin{sous-entrée}{phɯzjaŋ}{ⓔzjaŋzjaŋⓝphɯzjaŋ} 
\classe{idph.5} 
\begin{exemple}\pjya{phɯzjaŋ ʑo tɤ-ndzur}\hspace{5pt}\pcmn{他突然间站起来了,显得比别人高}\end{exemple}\end{sous-entrée}

\begin{sous-entrée}{mɤlɤzjaŋ}{ⓔzjaŋzjaŋⓝmɤlɤzjaŋ} 
\classe{idph.6} 
\begin{exemple}\pjya{a-ɣe mɤlɤzjaŋ ʑo thɯ-aβzu (=zjaŋzjaŋ ʑo kɯ-pa thɯ-aβzu)}\hspace{5pt}\pcmn{我的孙子战法得很高了}\end{exemple}\end{sous-entrée}

\begin{sous-entrée}{zjaŋɯŋi}{ⓔzjaŋzjaŋⓝzjaŋɯŋi} 
\classe{idph.7} 
\begin{exemple}\pjya{zjaŋɯŋi ʑo jɤ-ari}\hspace{5pt}\pcmn{他慢慢地走了(身子很高的人)}\end{exemple}\end{sous-entrée}

\begin{sous-entrée}{zjɯŋɯzjaŋi}{ⓔzjaŋzjaŋⓝzjɯŋɯzjaŋi} 
\classe{idph.8} 
\begin{exemple}\pjya{zjɯŋɯzjaŋi ɲɯ-xcat}\hspace{5pt}\pcmn{很多人在一起,高矮不一}\end{exemple}
\begin{exemple}\pjya{zgo zjɯŋɯzjaŋi ɲɯ-xcat}\hspace{5pt}\pcmn{山很多,高低不一}\end{exemple}\relationsémantique{参考}{\lien{ⓔɣɤzjaŋlaŋ}{ɣɤzjaŋlaŋ}}\relationsémantique{参考}{\lien{ⓔsɤzjaŋlaŋ}{sɤzjaŋlaŋ}}\relationsémantique{参考}{\lien{ⓔɣɤzjaŋlaŋⓝɣɤzjaŋzjaŋ}{ɣɤzjaŋzjaŋ}}\relationsémantique{参考}{\lien{ⓔɣɤzjaŋlaŋⓝsɤzjaŋzjaŋ}{sɤzjaŋzjaŋ}}\relationsémantique{参考}{\lien{ⓔɣɤzjaŋlaŋⓝnɯzjaŋ}{nɯzjaŋ}}\relationsémantique{参考}{\lien{ⓔzjɤɣzjɤɣ}{zjɤɣzjɤɣ}}\relationsémantique{参考}{\lien{ⓔtsjaŋtsjaŋ}{tsjaŋtsjaŋ}}\end{sous-entrée}

\end{entrée}

\begin{entrée}{zjɤɣzjɤɣ}{}{ⓔzjɤɣzjɤɣ} 
\classe{idph.2} \paradigme{dir}{tɤ-}\paradigme{dir}{tɤ-}
\begin{définition}\pfra{grand, élevé}\end{définition}
\begin{définition}\pcmn{形容个子高,物体因为数量多而堆得很高的样子;或形容人沉默不语的模样}\end{définition}
\begin{définition}\pfra{se balancer, se dandiner, se pas tenir en place sur sa chaise}\end{définition}
\begin{définition}\pcmn{不停地摇动,坐不住}\end{définition}
\begin{exemple}\pjya{tɯrme zjɤɣzjɤɣ ʑo ɲɯ-ɤmdzɯ ɲɯ-rɤʑi}\hspace{5pt}\pcmn{那个人沉默不语地在那里坐着}\end{exemple}
\begin{exemple}\pjya{tɯ-ɣli zjɤɣzjɤɣ ʑo to-rmbɯ-nɯ}\hspace{5pt}\pcmn{他们那肥料堆得很高(肥料多)}\end{exemple}
\begin{exemple}\pjya{tɤ-fkɯm ɯ-ŋgɯ zjɤɣzjɤɣ ʑo cho-rku}\hspace{5pt}\pcmn{口袋里装得很满}\end{exemple}
\begin{exemple}\pjya{tɕɤndi ɲɯ-nɤŋkɯŋke tɕe ɲɯ-ɣɤzjɤɣlɤɣ}\hspace{5pt}\pcmn{他在边走动}\end{exemple}
\begin{exemple}\pjya{ma-tɯ-ɣɤzjɤɣlɤɣ ntsɯ, phoʁphoʁ kɤ-ɤmdzɯ}\hspace{5pt}\pcmn{别动,坐好一点}\end{exemple}
\begin{exemple}\pjya{pri ɲɯ-ɣɤzjɤɣlɤɣ ntsɯ}\hspace{5pt}\pcmn{老熊在动来动去}\end{exemple}
\begin{exemple}\pjya{χpɯn kɤ-ndɯn ɯ-raŋ tɕe tu-ɣɤzjɤɣlɤɣ ntsɯ ŋu}\hspace{5pt}\pcmn{和尚念经的时候摇头}\end{exemple}
\begin{exemple}\pjya{si nɯ-ndʐaβ tɕe tɤ́-wɣ-sɤzjɤɣlɤɣ-a}\hspace{5pt}\pcmn{树倒过来了,差一点把我弄倒了}\end{exemple}\relationsémantique{参考}{\lien{ⓔzjɤɣzjɤɣ}{zjɤɣzjɤɣ}}
\begin{sous-entrée}{zjɤɣnɤzjɤɣ}{ⓔzjɤɣzjɤɣⓝzjɤɣnɤzjɤɣ} 
\classe{idph.3} 
\begin{exemple}\pjya{laχtɕha zjɤɣnɤzjɤɣ ɲɯ-ɤsɯ-fkur kɤ-ari}\hspace{5pt}\pcmn{他背着一大堆东西,走了}\end{exemple}
\begin{exemple}\pjya{zjɤɣnɤzjɤɣ ɲɯ-ŋke}\hspace{5pt}\pcmn{他走路一脚高一脚低的。}\end{exemple}\end{sous-entrée}

\begin{sous-entrée}{zjɤɣnɤlɤɣ}{ⓔzjɤɣzjɤɣⓝzjɤɣnɤlɤɣ} 
\classe{idph.4} 
\begin{exemple}\pjya{zjɤɣnɤlɤɣ ɲɯ-ʑɣɤstu}\hspace{5pt}\pcmn{他在扭动,到处东瞻西望}\end{exemple}\end{sous-entrée}

\begin{sous-entrée}{mɤlɤzjɤɣ}{ⓔzjɤɣzjɤɣⓝmɤlɤzjɤɣ} 
\classe{idph.5} 
\begin{exemple}\pjya{@yangyu mɤlɤzjɤɣ to-rku}\hspace{5pt}\pcmn{他把洋芋装得满满的}\end{exemple}\end{sous-entrée}

\begin{sous-entrée}{phɯzjɤɣ}{ⓔzjɤɣzjɤɣⓝphɯzjɤɣ} 
\classe{idph.6} 
\begin{exemple}\pjya{tɯrme phɯzjɤɣ tɤ-nɯɬoʁ}\hspace{5pt}\pcmn{突然冒出个人来}\end{exemple}\end{sous-entrée}

\begin{sous-entrée}{zjɤɣɯɣi}{ⓔzjɤɣzjɤɣⓝzjɤɣɯɣi} 
\classe{idph.7} 
\begin{exemple}\pjya{tɯrme zjɤɣɯɣi lɤ-ari}\hspace{5pt}\pcmn{那个人慢慢地离开了}\end{exemple}\end{sous-entrée}

\begin{sous-entrée}{ɣɤzjɤɣlɤɣ}{ⓔzjɤɣzjɤɣⓝɣɤzjɤɣlɤɣ} 
\classe{vi} \end{sous-entrée}

\begin{sous-entrée}{sɤzjɤɣlɤɣ}{ⓔzjɤɣzjɤɣⓝsɤzjɤɣlɤɣ} 
\classe{vt} \end{sous-entrée}

\end{entrée}

\begin{entrée}{zɟada}{}{ⓔzɟada} 
\classe{n} 
\begin{définition}\pfra{nain}\end{définition}
\begin{définition}\pcmn{矮人;侏儒}\end{définition}\end{entrée}

\begin{entrée}{zɟaŋzɟaŋ}{}{ⓔzɟaŋzɟaŋ} 
\classe{idph.2} 
\begin{définition}\pfra{mou et enflé}\end{définition}
\begin{définition}\pcmn{形容软而鼓起的样子}\end{définition}
\begin{exemple}\pjya{lʁa ɯ-ŋgɯ laχtɕha khro to-rku-nɯ zɟaŋzɟaŋ ʑo ɲɯ-pa}\hspace{5pt}\pcmn{他们在口袋里装了很多东西,显得鼓鼓囊囊的}\end{exemple}\relationsémantique{同义词}{\lien{ⓔzɟraŋzɟraŋ}{zɟraŋzɟraŋ}}\end{entrée}

\begin{entrée}{zɟɤɣzɟɤɣ}{}{ⓔzɟɤɣzɟɤɣ} 
\classe{idph.2} 
\begin{définition}\pfra{court et épais}\end{définition}
\begin{définition}\pcmn{形容粗而短的样子}\end{définition}\end{entrée}

\begin{entrée}{zɟi}{}{ⓔzɟi} 
\classe{n} 
\begin{définition}\pfra{sac en poils de yak}\end{définition}
\begin{définition}\pcmn{毛织布袋}\end{définition}\étymologie{sgʲe}\end{entrée}

\begin{entrée}{zɟoʁzɟoʁ}{}{ⓔzɟoʁzɟoʁ} 
\classe{idph.2} 
\begin{définition}\pfra{petit}\end{définition}
\begin{définition}\pcmn{形容身材矮小}\end{définition}
\begin{exemple}\pjya{tɤ-pɤtso nɯ zɟoʁzɟoʁ ʑo ɲɯ-pa}\hspace{5pt}\pcmn{那个孩子个子矮矮的}\end{exemple}\relationsémantique{反义词}{\lien{ⓔzjɤɣzjɤɣ}{zjɤɣzjɤɣ}}\relationsémantique{反义词}{\lien{ⓔzjaŋzjaŋ}{zjaŋzjaŋ}}\end{entrée}

\begin{entrée}{zɟraŋzɟraŋ}{}{ⓔzɟraŋzɟraŋ} 
\classe{idph.2} 
\begin{définition}\pfra{mou et enflé}\end{définition}
\begin{définition}\pcmn{形容软而鼓起的样子}\end{définition}
\begin{exemple}\pjya{ɯ-xtu zɟraŋzɟraŋ ʑo pa}\hspace{5pt}\pcmn{他肚子鼓起来,软乎乎的样子}\end{exemple}\relationsémantique{同义词}{\lien{ⓔzɟaŋzɟaŋ}{zɟaŋzɟaŋ}}\end{entrée}

\begin{entrée}{zɟɯɣ}{}{ⓔzɟɯɣ} 
\classe{idph.1} 
\begin{définition}\pfra{bruit d'un objet lourd qui tombe de très haut}\end{définition}
\begin{définition}\pcmn{形容重的物体从高处落地(震动地面)发出的声音}\end{définition}
\begin{exemple}\pjya{zɟɯɣ ɲɯ-ti pa-ɣɤrɤt}\hspace{5pt}\pcmn{他扔下去了,发出震动的声音}\end{exemple}
\begin{sous-entrée}{zɟɯɣnɤzɟɯɣ}{ⓔzɟɯɣⓝzɟɯɣnɤzɟɯɣ} 
\classe{idph.3} 
\begin{exemple}\pjya{ɕɤmɯɣdɯ zɟɯɣnɤzɟɯɣ ɲɯ-ɤsɯ-lɤt}\hspace{5pt}\pcmn{他在射枪,发出震动的声音}\end{exemple}\end{sous-entrée}

\end{entrée}

\begin{entrée}{zmɤŋgɯ}{}{ⓔzmɤŋgɯ}\relationsémantique{参考}{\lien{ⓔmɤŋgɯ}{mɤŋgɯ}}\end{entrée}

\begin{entrée}{zmɤpɕi}{}{ⓔzmɤpɕi}\relationsémantique{参考}{\lien{ⓔmɤpɕi}{mɤpɕi}}\end{entrée}

\begin{entrée}{zmɤrɤβ}{}{ⓔzmɤrɤβ} 
\classe{vt} \paradigme{dir}{tɤ-}
\begin{définition}\pfra{manger en mélangeant avec}\end{définition}
\begin{définition}\pcmn{合着吃}\end{définition}
\begin{exemple}\pjya{qajɣi cho tɤjko tɤ-zmɤraβ-a}\hspace{5pt}\pcmn{我把馍馍和菜和着吃了}\end{exemple}\end{entrée}

\begin{entrée}{zmɤrtsaβ}{}{ⓔzmɤrtsaβ}\relationsémantique{参考}{\lien{ⓔmɤrtsaβ}{mɤrtsaβ}}\end{entrée}

\begin{entrée}{zmɯjqha}{}{ⓔzmɯjqha} 
\classe{vt} \paradigme{dir}{tɤ-}
\begin{définition}\pfra{offenser}\end{définition}
\begin{définition}\pcmn{得罪}\end{définition}
\begin{exemple}\pjya{tɤ-zmɯjqha-t-a}\hspace{5pt}\pcmn{我得罪了他}\end{exemple}\relationsémantique{参考}{\lien{ⓔqha}{qha}}\end{entrée}

\begin{entrée}{zmɯjrɯ}{}{ⓔzmɯjrɯ}\relationsémantique{参考}{\lien{ⓔmɯjrɯ}{mɯjrɯ}}\end{entrée}

\begin{entrée}{zmɯnmu}{}{ⓔzmɯnmu}\relationsémantique{参考}{\lien{ⓔmɯnmu}{mɯnmu}}\end{entrée}

\begin{entrée}{zmɯrmbɯ}{}{ⓔzmɯrmbɯ}\relationsémantique{参考}{\lien{ⓔamɯrmbɯ}{amɯrmbɯ}}\end{entrée}

\begin{entrée}{znaʁjɯβ}{}{ⓔznaʁjɯβ}\relationsémantique{参考}{\lien{ⓔnaʁjɯβ}{naʁjɯβ}}\end{entrée}

\begin{entrée}{znaχɕɯχɕu}{}{ⓔznaχɕɯχɕu} 
\classe{vi}  
\grammaire{refl}
\grammaire{trop} 
\begin{définition}\pfra{se croire plus fort (que les autres)}\end{définition}
\begin{définition}\pcmn{自以为强}\end{définition}
\begin{exemple}\pjya{kɤ-znaχɕɯχɕu mɤ-tɯ-nɯ-cha ma nɤʑo sɤz kɯ-χɕu tu}\hspace{5pt}\pcmn{你用不着逞能,比你有能力的大有人在}\end{exemple}\relationsémantique{参考}{\lien{ⓔχɕu}{χɕu}}\relationsémantique{参考}{\lien{ⓔrɯqhaχɕu}{rɯqhaχɕu}}\end{entrée}

\begin{entrée}{znaχsoz}{}{ⓔznaχsoz}\relationsémantique{参考}{\lien{ⓔnaχsoz}{naχsoz}}\end{entrée}

\begin{entrée}{znaχtɕɯɣ}{}{ⓔznaχtɕɯɣ}\relationsémantique{参考}{\lien{ⓔnaχtɕɯɣ}{naχtɕɯɣ}}\end{entrée}

\begin{entrée}{znɤβʁaβʁa}{}{ⓔznɤβʁaβʁa}\relationsémantique{参考}{\lien{ⓔβʁa}{βʁa}}\end{entrée}

\begin{entrée}{znɤchacha}{}{ⓔznɤchacha} 
\classe{vi}  
\grammaire{refl}
\grammaire{trop} \paradigme{dir}{tɤ-}
\begin{définition}\pfra{fanfaronner}\end{définition}
\begin{définition}\pcmn{逞能}\end{définition}
\begin{exemple}\pjya{jiɕqha kɯ-znɤchacha ci ɲɯ-ŋu}\hspace{5pt}\pcmn{他是个逞能的人}\end{exemple}\relationsémantique{参考}{\lien{ⓔtʂhɤtⓗ1ⓝznɤtʂhɯtʂhɯt}{znɤtʂhɯtʂhɯt}}\end{entrée}

\begin{entrée}{znɤɕqa}{}{ⓔznɤɕqa}\relationsémantique{参考}{\lien{ⓔnɤɕqa}{nɤɕqa}}\end{entrée}

\begin{entrée}{znɤɕqɯɕqraʁ/\variante{znɤɕqraʁɕqraʁ}}{}{ⓔznɤɕqɯɕqraʁ} 
\classe{vi}  
\grammaire{refl}
\grammaire{trop} 
\begin{définition}\pfra{se croire intelligent}\end{définition}
\begin{définition}\pcmn{自以为聪明}\end{définition}
\begin{exemple}\pjya{kɯ-znɤɕqraʁɕqraʁ ci ɲɯ-ŋu}\hspace{5pt}\pcmn{他是个自以为聪明的人}\end{exemple}
\begin{exemple}\pjya{nɤʑo ɲɯ-tɯ-znɤɕqraʁɕqraʁ}\hspace{5pt}\pcmn{你自以为很聪明}\end{exemple}\relationsémantique{参考}{\lien{ⓔɕqraʁ}{ɕqraʁ}}\relationsémantique{同义词}{\lien{ⓔznɤchacha}{znɤchacha}}\relationsémantique{同义词}{\lien{ⓔrɯχparɤβ}{rɯχparɤβ}}\end{entrée}

\begin{entrée}{znɤftɕɤftɕɤl}{}{ⓔznɤftɕɤftɕɤl} 
\classe{vi}  
\grammaire{refl}
\grammaire{trop} \paradigme{dir}{tɤ-}
\begin{définition}\pfra{se créer des ennuis à soi-même}\end{définition}
\begin{définition}\pcmn{自作多情;自己给自己找麻烦}\end{définition}
\begin{exemple}\pjya{to-znɤftɕɤftɕal-a}\hspace{5pt}\pcmn{我自作多情}\end{exemple}
\begin{exemple}\pjya{ɯʑo ɲɯ-znɤftɕɤftɕɤl}\hspace{5pt}\pcmn{他自作多情}\end{exemple}
\begin{exemple}\pjya{ɯʑo kɯ-znɤftɕɤftɕɤl ci ŋu}\hspace{5pt}\pcmn{他是一个自作多情的人}\end{exemple}\end{entrée}

\begin{entrée}{znɤja}{}{ⓔznɤja} 
\classe{vt} \paradigme{dir}{nɯ-}
\begin{définition}\pfra{chérir}\end{définition}
\begin{définition}\pcmn{珍惜;不舍得}\end{définition}
\begin{exemple}\pjya{tɯ-ŋga kɤ-rɤmbi ɲɯ-znɤje-a}\hspace{5pt}\pcmn{我舍不得把衣服给别人}\end{exemple}\relationsémantique{参考}{\lien{ⓔnɤja}{nɤja}}\end{entrée}

\begin{entrée}{znɤjo}{}{ⓔznɤjo}\relationsémantique{参考}{\lien{ⓔnɤjo}{nɤjo}}\end{entrée}

\begin{entrée}{znɤjpɯjpe/\variante{znɤjpejpe}}{}{ⓔznɤjpɯjpe} 
\classe{vs}  
\grammaire{refl}
\grammaire{trop} \paradigme{dir}{thɯ-}
\begin{définition}\pfra{se considérer comme quelqu'un de bien, se trouver belle}\end{définition}
\begin{définition}\pcmn{自以为好;自以为漂亮}\end{définition}
\begin{exemple}\pjya{ɯʑo kɯ-znɤjpɯjpe ci ŋu}\hspace{5pt}\pcmn{他是一个自以为好的人}\end{exemple}
\begin{exemple}\pjya{ɲɯ-znɤjpɯjpe}\hspace{5pt}\pcmn{他自以为好}\end{exemple}
\begin{exemple}\pjya{cho-znɤjpɯjpe}\hspace{5pt}\pcmn{他现在觉得自己很好(以前没有这个习惯)}\end{exemple}\end{entrée}

\begin{entrée}{znɤkɤro}{}{ⓔznɤkɤro}\relationsémantique{参考}{\lien{ⓔnɤkɤro}{nɤkɤro}}\end{entrée}

\begin{entrée}{znɤkhɤzŋga}{}{ⓔznɤkhɤzŋga}\relationsémantique{参考}{\lien{ⓔnɤkhɤzŋga}{nɤkhɤzŋga}}\end{entrée}

\begin{entrée}{znɤkhɯ}{}{ⓔznɤkhɯ}\relationsémantique{参考}{\lien{ⓔnɤkhɯ}{nɤkhɯ}}\end{entrée}

\begin{entrée}{znɤkɯt}{}{ⓔznɤkɯt}\relationsémantique{参考}{\lien{ⓔnɤkɯt}{nɤkɯt}}\end{entrée}

\begin{entrée}{znɤltɕɤm}{}{ⓔznɤltɕɤm} 
\classe{vt} \paradigme{dir}{pɯ-}\paradigme{dir}{pɯ-}
\begin{définition}\pfra{couvrir}\end{définition}
\begin{définition}\pcmn{用自己衣服的一角盖在别人身上,跟别人分享}\end{définition}
\begin{définition}\pfra{se couvrir les uns les autres}\end{définition}
\begin{définition}\pcmn{用衣服互相盖}\end{définition}
\begin{exemple}\pjya{pɯ-znɤltɕam-a}\hspace{5pt}\pcmn{我帮他盖了(衣服)}\end{exemple}
\begin{exemple}\pjya{pɯ́-wɣ-znɤltɕam-a}\hspace{5pt}\pcmn{他帮我盖了(衣服)}\end{exemple}
\begin{exemple}\pjya{tɯ-ŋga ɯ-βzɯr nɯ ɯ-zda ɯ-taʁ zɯ pa-znɤltɕɤm}\hspace{5pt}\pcmn{他顺便把铺盖的一角帮朋友盖在身上}\end{exemple}
\begin{exemple}\pjya{pɯ-aznɤltɕɯltɕɤm-tɕi}\hspace{5pt}\pcmn{我们互相盖了}\end{exemple}
\begin{sous-entrée}{aznɤltɕɯltɕɤm}{ⓔznɤltɕɤmⓝaznɤltɕɯltɕɤm} 
\classe{vi}  
\grammaire{recip} \end{sous-entrée}

\end{entrée}

\begin{entrée}{znɤlɯli}{}{ⓔznɤlɯli} 
\classe{vi} \paradigme{dir}{tɤ-}
\begin{définition}\pfra{faire des caprices}\end{définition}
\begin{définition}\pcmn{撒娇}\end{définition}
\begin{exemple}\pjya{ɯ-mu ɯ-phe ɲɯ-znɤlɯli}\hspace{5pt}\pcmn{他向他母亲撒娇}\end{exemple}
\begin{exemple}\pjya{a-mu ɯ-ɕki tɤ-znɤlɯli-a}\hspace{5pt}\pcmn{我在母亲面前撒娇了}\end{exemple}\end{entrée}

\begin{entrée}{znɤmaʁmaʁ}{}{ⓔznɤmaʁmaʁ} 
\classe{vt} \paradigme{dir}{tɤ-}
\begin{définition}\pfra{cacher la vérité}\end{définition}
\begin{définition}\pcmn{掩盖真相;掩盖自己的行为}\end{définition}
\begin{exemple}\pjya{to-mɯrkɯ ri to-znɤmaʁmaʁ}\hspace{5pt}\pcmn{他偷了东西但是不承认}\end{exemple}
\begin{exemple}\pjya{pɯ-kɯ-fse nɯ tɤ-ti wo ma ma-tɤ-tɯ-znɤmaʁmaʁ}\hspace{5pt}\pcmn{你要把发生的事情说清楚,不要掩盖真相}\end{exemple}\relationsémantique{参考}{\lien{ⓔmaʁⓗ1}{maʁ₁}}\end{entrée}

\begin{entrée}{znɤmbju}{}{ⓔznɤmbju}\relationsémantique{参考}{\lien{ⓔnɤmbju}{nɤmbju}}\end{entrée}

\begin{entrée}{znɤmɲole}{}{ⓔznɤmɲole}\relationsémantique{参考}{\lien{ⓔnɤmɲole}{nɤmɲole}}\end{entrée}

\begin{entrée}{znɤmpɕɤmpɕɤr}{}{ⓔznɤmpɕɤmpɕɤr} 
\classe{vi}  
\grammaire{refl}
\grammaire{trop} \paradigme{dir}{thɯ-}
\begin{définition}\pfra{se croire belle}\end{définition}
\begin{définition}\pcmn{以为自己很漂亮}\end{définition}
\begin{exemple}\pjya{chɤ-tɯ-znɤmpɕɤmpɕɤr}\hspace{5pt}\pcmn{你以为自己很漂亮了}\end{exemple}\relationsémantique{参考}{\lien{ⓔmpɕɤr}{mpɕɤr}}\relationsémantique{参考}{\lien{ⓔrɤmpɕɤr}{rɤmpɕɤr}}\end{entrée}

\begin{entrée}{znɤmqrɯz}{}{ⓔznɤmqrɯz}\relationsémantique{参考}{\lien{ⓔɣɤmqrɯz}{ɣɤmqrɯz}}\end{entrée}

\begin{entrée}{znɤmɯma}{}{ⓔznɤmɯma} 
\classe{vt} \paradigme{dir}{tɤ-}
\begin{définition}\pfra{faire toutes sortes de choses}\end{définition}
\begin{définition}\pcmn{做各种事情}\end{définition}
\begin{exemple}\pjya{tɕhi tɤ́-wɣ-znɤmɯma ʑo tu-kɯ-rɯndzaŋspa ra}\hspace{5pt}\pcmn{不管做什么,一定要小心}\end{exemple}
\begin{exemple}\pjya{tɕhi tɤ́-wɣ-znɤmɯma ʑo mɯ́j-cha}\hspace{5pt}\pcmn{不管让他做什么都不行}\end{exemple}\end{entrée}

\begin{entrée}{znɤndɤɣ}{}{ⓔznɤndɤɣ}\relationsémantique{参考}{\lien{ⓔnɤndɤɣ}{nɤndɤɣ}}\end{entrée}

\begin{entrée}{znɤndɤɣri}{}{ⓔznɤndɤɣri}\relationsémantique{参考}{\lien{ⓔnɤndɤɣri}{nɤndɤɣri}}\end{entrée}

\begin{entrée}{znɤndɤr}{}{ⓔznɤndɤr}\relationsémantique{参考}{\lien{ⓔnɤndɤr}{nɤndɤr}}\end{entrée}

\begin{entrée}{znɤndɯndɤt}{}{ⓔznɤndɯndɤt}\relationsémantique{参考}{\lien{ⓔnɤndɯndɤt}{nɤndɯndɤt}}\end{entrée}

\begin{entrée}{znɤngɯt}{}{ⓔznɤngɯt}\relationsémantique{参考}{\lien{ⓔnɤngɯt}{nɤngɯt}}\end{entrée}

\begin{entrée}{znɤŋɤβ}{}{ⓔznɤŋɤβ}\relationsémantique{参考}{\lien{ⓔsɤŋɤβ}{sɤŋɤβ}}\end{entrée}

\begin{entrée}{znɤŋgɤr}{}{ⓔznɤŋgɤr}\paradigme{dir}{\_}
\begin{définition}\pfra{pousser vers un côté (par la foule)}\end{définition}
\begin{définition}\pcmn{挤过去(因为人多,很拥挤)}\end{définition}
\begin{exemple}\pjya{ɲɯ-kɯ-znɤŋgar-a}\hspace{5pt}\pcmn{你把我挤到那边去}\end{exemple}\relationsémantique{参考}{\lien{ⓔŋgɤr}{ŋgɤr}}\end{entrée}

\begin{entrée}{znɤŋgɯ}{}{ⓔznɤŋgɯ}\relationsémantique{参考}{\lien{ⓔnɤŋgɯ}{nɤŋgɯ}}\end{entrée}

\begin{entrée}{znɤŋɯŋu}{}{ⓔznɤŋɯŋu} 
\classe{vs}  
\grammaire{refl}
\grammaire{trop} 
\begin{définition}\pfra{être présomptueux}\end{définition}
\begin{définition}\pcmn{自以为是}\end{définition}\relationsémantique{参考}{\lien{ⓔŋu}{ŋu}}\end{entrée}

\begin{entrée}{znɤpɤri}{}{ⓔznɤpɤri}\relationsémantique{参考}{\lien{ⓔnɤpɤri}{nɤpɤri}}\end{entrée}

\begin{entrée}{znɤphɤtphɤt}{}{ⓔznɤphɤtphɤt}\relationsémantique{参考}{\lien{ⓔnɤphɤtphɤt}{nɤphɤtphɤt}}\end{entrée}

\begin{entrée}{znɤrɕu}{}{ⓔznɤrɕu}\relationsémantique{参考}{\lien{ⓔnɤrɕu}{nɤrɕu}}\end{entrée}

\begin{entrée}{znɤre}{}{ⓔznɤre}\relationsémantique{参考}{\lien{ⓔnɤreⓗ1ⓢ2ⓝnɤre}{nɤre}}\end{entrée}

\begin{entrée}{znɤrko}{}{ⓔznɤrko}\relationsémantique{参考}{\lien{}{nɤrko}}\end{entrée}

\begin{entrée}{znɤʁamɟa}{}{ⓔznɤʁamɟa} 
\classe{vs} \paradigme{dir}{tɤ-}
\begin{définition}\pfra{zêlé}\end{définition}
\begin{définition}\pcmn{勤快,抓紧时间(舍不得耽误时间)}\end{définition}
\begin{exemple}\pjya{ta-ma ɲɯ-znɤʁamɟa}\hspace{5pt}\pcmn{他工作很勤快}\end{exemple}
\begin{exemple}\pjya{pɯ-znɤʁamɟa-tɕi}\hspace{5pt}\pcmn{我们俩很勤快}\end{exemple}
\begin{exemple}\pjya{jiɕqha nɯ kɯ-znɤʁamɟa ci ɲɯ-ŋu}\hspace{5pt}\pcmn{他是个勤快的人}\end{exemple}
\begin{exemple}\pjya{nɤʑo kɤ-rɤβzjoz ɲɯ-tɯ-znɤʁamɟa}\hspace{5pt}\pcmn{你学习很勤快}\end{exemple}\relationsémantique{参考}{\lien{ⓔtɤ-ʁamɟa}{tɤ-ʁamɟa}}\end{entrée}

\begin{entrée}{znɤʁdɤn/\variante{znɯʁdɤn}}{}{ⓔznɤʁdɤn}\relationsémantique{参考}{\lien{ⓔnɤʁdɤn}{nɤʁdɤn}}\end{entrée}

\begin{entrée}{znɤscɤr}{}{ⓔznɤscɤr}\relationsémantique{参考}{\lien{ⓔnɤscɤr}{nɤscɤr}}\end{entrée}

\begin{entrée}{znɤtʂa}{}{ⓔznɤtʂa}\relationsémantique{参考}{\lien{ⓔnɤtʂa}{nɤtʂa}}\end{entrée}

\begin{entrée}{znɤtʂhɯtʂhɯt}{}{ⓔznɤtʂhɯtʂhɯt}\relationsémantique{参考}{\lien{ⓔtʂhɤtⓗ1}{tʂhɤt₁}}\end{entrée}

\begin{entrée}{znɤtʂɯntʂɯn}{}{ⓔznɤtʂɯntʂɯn} 
\classe{vs} 
\begin{définition}\pfra{qui aime se vanter de ses bonnes actions}\end{définition}
\begin{définition}\pcmn{炫耀自己的功劳}\end{définition}
\begin{exemple}\pjya{ɲɯ-sɯxtʂɯn ri, ɲɯ-znɤtʂɯntʂɯn}\hspace{5pt}\pcmn{他虽然对别人好,但是会炫耀自己}\end{exemple}\relationsémantique{参考}{\lien{ⓔtɯ-tʂɯn}{tɯ-tʂɯn}}\étymologie{drin}\end{entrée}

\begin{entrée}{znɤtɯɣ}{}{ⓔznɤtɯɣ}\relationsémantique{参考}{\lien{ⓔatɯɣ}{atɯɣ}}\end{entrée}

\begin{entrée}{znɤχɤmthi}{}{ⓔznɤχɤmthi}\relationsémantique{参考}{\lien{ⓔnɤχɤmthi}{nɤχɤmthi}}\end{entrée}

\begin{entrée}{znɤχpɯχpa}{}{ⓔznɤχpɯχpa} 
\classe{vs} \paradigme{dir}{thɯ-}
\begin{définition}\pfra{arrogant}\end{définition}
\begin{définition}\pcmn{傲慢}\end{définition}
\begin{exemple}\pjya{jiɕqha nɯ ɲɯ-znɤχpɯχpa}\hspace{5pt}\pcmn{那个人很傲慢}\end{exemple}\relationsémantique{参考}{\lien{ⓔχpa}{χpa}}\end{entrée}

\begin{entrée}{znɤzraʁ}{}{ⓔznɤzraʁ}\relationsémantique{参考}{\lien{ⓔnɤzraʁ}{nɤzraʁ}}\end{entrée}

\begin{entrée}{zndɤkɤlwa}{}{ⓔzndɤkɤlwa} 
\classe{n} 
\begin{définition}\pfra{pierre plate}\end{définition}
\begin{définition}\pcmn{防雨水的石板}\end{définition}\relationsémantique{参考}{\lien{ⓔzndeⓗ1}{znde₁}}\end{entrée}

\begin{entrée}{zndɤqa}{}{ⓔzndɤqa} 
\classe{n} 
\begin{définition}\pfra{bas du mur}\end{définition}
\begin{définition}\pcmn{墙脚}\end{définition}\end{entrée}

\begin{entrée}{zndɤrchɤβ}{}{ⓔzndɤrchɤβ} 
\classe{n} 
\begin{définition}\pfra{fissure sur le mur}\end{définition}
\begin{définition}\pcmn{墙上的缝隙}\end{définition}\end{entrée}

\begin{entrée}{zndɤtɕhaʁ}{}{ⓔzndɤtɕhaʁ} 
\classe{n} 
\begin{définition}\pfra{se rétrécir (mur)}\end{définition}
\begin{définition}\pcmn{收缩,变形(墙因受潮等原因)}\end{définition}
\begin{exemple}\pjya{zndɤtɕhaʁ pjɤ-ɕe}\hspace{5pt}\pcmn{墙(因为受了潮)收缩变形了。}\end{exemple}\relationsémantique{参考}{\lien{ⓔzndeⓗ1}{znde₁}}\relationsémantique{参考}{\lien{ⓔtɕhaʁ}{tɕhaʁ}}\end{entrée}

\begin{entrée}{znde}{₁}{ⓔzndeⓗ1} 
\classe{n} 
\begin{définition}\pfra{mur en pierre}\end{définition}
\begin{définition}\pcmn{石墙}\end{définition}
\begin{exemple}\pjya{znde tɤ-βzu-t-a}\hspace{5pt}\pcmn{我修了墙}\end{exemple}
\begin{exemple}\pjya{qajɯ znde ɯ-taʁ tu-xcat-nɯ ʑo ɲɯ-ŋu}\hspace{5pt}\pcmn{墙上有很多虫子}\end{exemple}\relationsémantique{参考}{\lien{ⓔzndeⓗ2}{znde₂}}\relationsémantique{参考}{\lien{ⓔrɤznde}{rɤznde}}\relationsémantique{参考}{\lien{ⓔzndɤtɕhaʁ}{zndɤtɕhaʁ}}\relationsémantique{参考}{\lien{ⓔzndɤrchɤβ}{zndɤrchɤβ}}\end{entrée}

\begin{entrée}{znde}{₂}{ⓔzndeⓗ2} 
\classe{vt} \paradigme{dir}{tɤ-}
\begin{définition}\pfra{réparer un mur à un endroit, empiler des briques là où le mur s'est abîmé}\end{définition}
\begin{définition}\pcmn{堵住石墙的缺口}\end{définition}
\begin{exemple}\pjya{tɤ-znde-t-a}\hspace{5pt}\pcmn{我垒起来了}\end{exemple}
\begin{exemple}\pjya{ki ɯ-stu ki ɲɯ-ɤχa tɕe tɤ-znde-t-a}\hspace{5pt}\pcmn{这个地方有个缺口,我就堵上了}\end{exemple}\relationsémantique{参考}{\lien{ⓔzndeⓗ1}{znde₁}}\relationsémantique{参考}{\lien{ⓔrɤznde}{rɤznde}}\end{entrée}

\begin{entrée}{znɯβdaʁ}{}{ⓔznɯβdaʁ}\relationsémantique{参考}{\lien{ⓔnɯβdaʁ}{nɯβdaʁ}}\end{entrée}

\begin{entrée}{znɯcaχto}{}{ⓔznɯcaχto}\relationsémantique{参考}{\lien{ⓔnɯcaχto}{nɯcaχto}}\end{entrée}

\begin{entrée}{znɯɕkhɤɣ}{}{ⓔznɯɕkhɤɣ} 
\classe{vt} \paradigme{dir}{nɯ-}
\begin{définition}\pfra{ne pas se préoccuper de}\end{définition}
\begin{définition}\pcmn{不在乎,不理}\end{définition}
\begin{exemple}\pjya{nɯ kɯ a-phe nɯ ɲɯ-ti ri, ɲɯ-znɯɕkhaɣ-a ɕti}\hspace{5pt}\pcmn{他对我说这些话,但是我不在乎}\end{exemple}
\begin{exemple}\pjya{tɯ-mɯ ɲɯ-ɤsɯ-lɤt ri aʑo ɲɯ-znɯɕqhaɣ-a ɕti}\hspace{5pt}\pcmn{虽然下雨,但是我不在乎}\end{exemple}\end{entrée}

\begin{entrée}{znɯɕqhɯɕqhu}{}{ⓔznɯɕqhɯɕqhu} 
\classe{vt} \paradigme{dir}{pɯ-}\sens{1}
\begin{définition}\pfra{s'opposer}\end{définition}
\begin{définition}\pcmn{违反;反对}\end{définition}
\begin{exemple}\pjya{pɯ-znɯɕqhɯɕqhu-t-a}\hspace{5pt}\pcmn{我反对了}\end{exemple}
\begin{exemple}\pjya{ɯ-stu mɯ́j-nɤme tɕe pjɯ-znɯɕqhɯɕqhe ntsɯ ɲɯ-ɕti}\hspace{5pt}\pcmn{他不是诚心想做,总是搞破坏}\end{exemple}
\begin{exemple}\pjya{tɤ-tɯt-a nɯ ma-pɯ-tɯ-znɯɕqhɯɕqhe}\hspace{5pt}\pcmn{你不要反对我所说的}\end{exemple}\sens{2}
\begin{définition}\pfra{revenir sur (sa parole)}\end{définition}
\begin{définition}\pcmn{违背;反悔}\end{définition}
\begin{exemple}\pjya{tɤ-tɯ-nɯ-tɯt nɯ ma-pɯ-tɯ-znɯɕqhɯɕqhe}\hspace{5pt}\pcmn{你不要反悔}\end{exemple}\relationsémantique{参考}{\lien{ⓔnɯɕqhu}{nɯɕqhu}}\end{entrée}

\begin{entrée}{znɯɕtar}{}{ⓔznɯɕtar}\relationsémantique{参考}{\lien{ⓔnɯɕtar}{nɯɕtar}}\end{entrée}

\begin{entrée}{znɯfsoʁspɤt}{}{ⓔznɯfsoʁspɤt} 
\classe{vt} \paradigme{dir}{lɤ-}
\begin{définition}\pfra{(faire) toute la nuit jusqu'au lever du jour}\end{définition}
\begin{définition}\pcmn{从晚上开始一直……到天亮}\end{définition}
\begin{exemple}\pjya{kɤ-rɤrɤt lɤ-znɯfsoʁspat-a ʑo pɯ-ra}\hspace{5pt}\pcmn{我只好一直写到天亮}\end{exemple}
\begin{exemple}\pjya{kɤ-rɤma lɤ-znɯfsoʁspat-i ʑo}\hspace{5pt}\pcmn{我们晚上开始工作,一直工作到天亮}\end{exemple}\relationsémantique{参考}{\lien{}{fsoʁ}}\end{entrée}

\begin{entrée}{znɯɣɟɯ}{}{ⓔznɯɣɟɯ}\relationsémantique{参考}{\lien{ⓔnɯɣɟɯ}{nɯɣɟɯ}}\end{entrée}

\begin{entrée}{znɯɣmaz}{}{ⓔznɯɣmaz}\relationsémantique{参考}{\lien{ⓔnɯɣmaz}{nɯɣmaz}}\end{entrée}

\begin{entrée}{znɯjɯn}{}{ⓔznɯjɯn} 
\classe{vt} \paradigme{dir}{tɤ-}
\begin{définition}\pfra{être conforme à, suivre, aller le long}\end{définition}
\begin{définition}\pcmn{顺着;依着;沿着}\end{définition}
\begin{exemple}\pjya{ta-znɯjɯn}\hspace{5pt}\pcmn{他跟着他去了}\end{exemple}
\begin{exemple}\pjya{tɤ́-wɣ-znɯjɯn-a}\hspace{5pt}\pcmn{他跟着我去了}\end{exemple}
\begin{exemple}\pjya{tʂu tɤ-znɯjɯn-a}\hspace{5pt}\pcmn{我沿着路去了}\end{exemple}\relationsémantique{同义词}{\lien{ⓔnɯɴqhu}{nɯɴqhu}}\end{entrée}

\begin{entrée}{znɯkhrɯm}{}{ⓔznɯkhrɯm}\relationsémantique{参考}{\lien{ⓔnɯkhrɯm}{nɯkhrɯm}}\end{entrée}

\begin{entrée}{znɯkro}{}{ⓔznɯkro} 
\classe{vt} \paradigme{dir}{tɤ-}
\begin{définition}\pfra{donner une part à}\end{définition}
\begin{définition}\pcmn{分东西给别人}\end{définition}
\begin{exemple}\pjya{aʑo paχɕi a-zda pɯ-znɯkro-t-a}\hspace{5pt}\pcmn{我把苹果分给我的朋友了}\end{exemple}
\begin{exemple}\pjya{aʑo paχɕi pjɯ-ta-znɯkro}\hspace{5pt}\pcmn{我把苹果分给你}\end{exemple}
\begin{exemple}\pjya{nɤ-paχɕi nɤ-zda pɯ-znɯkrɤm}\hspace{5pt}\pcmn{你把苹果分给你的朋友}\end{exemple}\relationsémantique{参考}{\lien{ⓔkro}{kro}}\end{entrée}

\begin{entrée}{znɯkrɯβ}{}{ⓔznɯkrɯβ}\relationsémantique{参考}{\lien{ⓔnɯkrɯβ}{nɯkrɯβ}}\end{entrée}

\begin{entrée}{znɯkɯlu}{}{ⓔznɯkɯlu}\relationsémantique{参考}{\lien{ⓔnɯkɯlu}{nɯkɯlu}}\end{entrée}

\begin{entrée}{znɯmgɯrjɯm/\variante{znɯmgɯrjɯβ}}{}{ⓔznɯmgɯrjɯm} 
\classe{vt} \paradigme{dir}{kɤ-}\paradigme{dir}{thɯ-}
\begin{définition}\pfra{chauffer au feu}\end{définition}
\begin{définition}\pcmn{烘干;烤(在火塘边)}\end{définition}
\begin{exemple}\pjya{smi ɯ-phe kɤ-znɯmgɯrjɯm-a}\hspace{5pt}\pcmn{我烤火了}\end{exemple}
\begin{exemple}\pjya{@yangyu kɤ-kɤ-sqa kɤ-znɯmgɯrjɯm-a}\hspace{5pt}\pcmn{把煮过的洋芋烤了一下}\end{exemple}
\begin{exemple}\pjya{qajɣi nɯ chɯ́-wɣ-znɯmgɯrjɯβ tɕe, nɯ kɯ-fse thɯ-kɯ-smi nɯ tú-wɣ-ndza tɕe mɯm}\hspace{5pt}\pcmn{把馍馍在火塘边烤熟很好吃}\end{exemple}\end{entrée}

\begin{entrée}{znɯmkɤqloʁ}{}{ⓔznɯmkɤqloʁ}\relationsémantique{参考}{\lien{ⓔnɯmkɤqloʁ}{nɯmkɤqloʁ}}\end{entrée}

\begin{entrée}{znɯmnɤl}{}{ⓔznɯmnɤl}\relationsémantique{参考}{\lien{ⓔnɯmnɤl}{nɯmnɤl}}\end{entrée}

\begin{entrée}{znɯna}{}{ⓔznɯna} 
\classe{vt} \paradigme{dir}{\_}
\begin{définition}\pfra{arrêter}\end{définition}
\begin{définition}\pcmn{停止}\end{définition}
\begin{exemple}\pjya{aʑo thamaka kɤ-sko tɤ-znɯna-t-a ma mɯ́j-pe}\hspace{5pt}\pcmn{我停止抽烟了}\end{exemple}
\begin{exemple}\pjya{ta-ma tɤ-znɯna-t-a}\hspace{5pt}\pcmn{我停止工作了}\end{exemple}\relationsémantique{参考}{\lien{ⓔnɯna}{nɯna}}\end{entrée}

\begin{entrée}{znɯndzɯ}{}{ⓔznɯndzɯ}\relationsémantique{参考}{\lien{ⓔnɯndzɯ}{nɯndzɯ}}\end{entrée}

\begin{entrée}{znɯndʐɯnbu}{}{ⓔznɯndʐɯnbu}\relationsémantique{参考}{\lien{ⓔnɯndʐɯnbu}{nɯndʐɯnbu}}\end{entrée}

\begin{entrée}{znɯnoʁ}{}{ⓔznɯnoʁ} 
\classe{vt} \paradigme{dir}{pɯ-}
\begin{définition}\pfra{mettre de la nourriture dans la sauce}\end{définition}
\begin{définition}\pcmn{蘸}\end{définition}
\begin{exemple}\pjya{ɕkɤfkri ɯ-ŋgɯ pɯ-znɯnoʁ-a}\hspace{5pt}\pcmn{我蘸了大蒜沾水}\end{exemple}
\begin{exemple}\pjya{tsha pɯ-znɯnoʁ-a}\hspace{5pt}\pcmn{我蘸了盐巴}\end{exemple}
\begin{exemple}\pjya{tɯnoʁ tɤ-βzu-t-a tɕe, tɤ-mthɯm pɯ-znɯnoʁ}\hspace{5pt}\pcmn{我做了汁,你就蘸肉吧}\end{exemple}\relationsémantique{参考}{\lien{ⓔtɯnoʁ}{tɯnoʁ}}\end{entrée}

\begin{entrée}{znɯntsho}{}{ⓔznɯntsho}\relationsémantique{参考}{\lien{ⓔnɯntsho}{nɯntsho}}\end{entrée}

\begin{entrée}{znɯntsɯɣ}{}{ⓔznɯntsɯɣ}\relationsémantique{参考}{\lien{ⓔnɯntsɯɣ}{nɯntsɯɣ}}\end{entrée}

\begin{entrée}{znɯɲco}{}{ⓔznɯɲco}\relationsémantique{参考}{\lien{ⓔnɯco}{nɯco}}\end{entrée}

\begin{entrée}{znɯŋgu}{}{ⓔznɯŋgu}\relationsémantique{参考}{\lien{ⓔnɯŋgu}{nɯŋgu}}\end{entrée}

\begin{entrée}{znɯŋgɤrkɯ}{}{ⓔznɯŋgɤrkɯ} 
\classe{vt} \paradigme{dir}{kɤ-}
\begin{définition}\pfra{envelopper les bébés dans des habits et les placer verticalement}\end{définition}
\begin{définition}\pcmn{用衣服把婴儿竖着包起来}\end{définition}\end{entrée}

\begin{entrée}{znɯŋgra}{}{ⓔznɯŋgra}\relationsémantique{参考}{\lien{ⓔnɯŋgra}{nɯŋgra}}\end{entrée}

\begin{entrée}{znɯɴɢɤt}{}{ⓔznɯɴɢɤt} 
\classe{vt} \paradigme{dir}{nɯ-}
\begin{définition}\pfra{séparer}\end{définition}
\begin{définition}\pcmn{分开}\end{définition}
\begin{exemple}\pjya{tɤ-pɤtso ni nɯ-znɯɴɢɤt-i ma ɲɯ-rɯŋɯŋɤn-ndʑi}\hspace{5pt}\pcmn{我们让那两个孩子分开,因为他们俩准备干坏事}\end{exemple}\relationsémantique{参考}{\lien{ⓔɴɢɤt}{ɴɢɤt}}\end{entrée}

\begin{entrée}{znɯɴqhu}{}{ⓔznɯɴqhu}\relationsémantique{参考}{\lien{ⓔnɯɴqhu}{nɯɴqhu}}\end{entrée}

\begin{entrée}{znɯpoʁ}{}{ⓔznɯpoʁ}\relationsémantique{参考}{\lien{ⓔnɯpoʁ}{nɯpoʁ}}\end{entrée}

\begin{entrée}{znɯqatɯkɯr}{}{ⓔznɯqatɯkɯr} 
\classe{vt} \paradigme{dir}{nɯ-}
\begin{définition}\pfra{donner de mauvais conseils}\end{définition}
\begin{définition}\pcmn{进行反面教育}\end{définition}
\begin{exemple}\pjya{na-znɯqatɯkɯr}\hspace{5pt}\pcmn{他对他进行了反面教育}\end{exemple}
\begin{exemple}\pjya{nɯ́-wɣ-znɯqatɯkɯr-a}\hspace{5pt}\pcmn{他对我进行了反面教育}\end{exemple}
\begin{exemple}\pjya{ɲɤ-znɯqatɯkɯr}\hspace{5pt}\pcmn{他对他进行了反面教育}\end{exemple}\end{entrée}

\begin{entrée}{znɯqhɤstɯstu}{}{ⓔznɯqhɤstɯstu}\relationsémantique{参考}{\lien{ⓔnɯqhɤstɯstu}{nɯqhɤstɯstu}}\end{entrée}

\begin{entrée}{znɯrɤʁaŋ}{}{ⓔznɯrɤʁaŋ} 
\classe{vt} 
\begin{définition}\pfra{qui a le droit de}\end{définition}
\begin{définition}\pcmn{有权利……;有资格……}\end{définition}
\begin{exemple}\pjya{nɤʑo aʑo kɤ-sɯxɕɤt ra mɤ-tɯ-znɯrɤʁaŋ (=nɤ-rɤʁaŋ me)}\hspace{5pt}\pcmn{你没有资格教我}\end{exemple}\relationsémantique{参考}{\lien{ⓔtɯ-rɤʁaŋ}{tɯ-rɤʁaŋ}}\end{entrée}

\begin{entrée}{znɯrdɯl}{}{ⓔznɯrdɯl}\relationsémantique{参考}{\lien{ⓔnɯrdɯl}{nɯrdɯl}}\end{entrée}

\begin{entrée}{znɯrɯrɯz}{}{ⓔznɯrɯrɯz}\relationsémantique{参考}{\lien{ⓔnɯrɯz}{nɯrɯz}}\end{entrée}

\begin{entrée}{znɯsɤlɤɣ}{}{ⓔznɯsɤlɤɣ}\relationsémantique{参考}{\lien{ⓔnɯsɤlɤɣ}{nɯsɤlɤɣ}}\end{entrée}

\begin{entrée}{znɯslɯɣ}{}{ⓔznɯslɯɣ}\relationsémantique{参考}{\lien{ⓔnɯslɯɣ}{nɯslɯɣ}}\end{entrée}

\begin{entrée}{znɯsmɤn}{}{ⓔznɯsmɤn}\relationsémantique{参考}{\lien{ⓔnɯsmɤn}{nɯsmɤn}}\end{entrée}

\begin{entrée}{znɯstu}{}{ⓔznɯstu}\relationsémantique{参考}{\lien{ⓔnɯstu}{nɯstu}}\end{entrée}

\begin{entrée}{znɯta}{}{ⓔznɯta}\relationsémantique{参考}{\lien{ⓔta}{ta}}\end{entrée}

\begin{entrée}{znɯtɕarloŋ}{}{ⓔznɯtɕarloŋ}\relationsémantique{参考}{\lien{ⓔnɯtɕarloŋ}{nɯtɕarloŋ}}\end{entrée}

\begin{entrée}{znɯtɕhɤl}{}{ⓔznɯtɕhɤl}\relationsémantique{参考}{\lien{ⓔnɯtɕhɤl}{nɯtɕhɤl}}\end{entrée}

\begin{entrée}{znɯtɕhɤtpa}{}{ⓔznɯtɕhɤtpa} 
\classe{vt} 
\begin{définition}\pfra{punir}\end{définition}
\begin{définition}\pcmn{惩罚}\end{définition}\relationsémantique{参考}{\lien{ⓔtɕhɤtpa}{tɕhɤtpa}}\relationsémantique{同义词}{\lien{ⓔnɯtɕhɤlⓝznɯtɕhɤl}{znɯtɕhɤl}}\end{entrée}

\begin{entrée}{znɯtɯfɕɤl}{}{ⓔznɯtɯfɕɤl}\relationsémantique{参考}{\lien{ⓔnɯtɯfɕɤl}{nɯtɯfɕɤl}}\end{entrée}

\begin{entrée}{znɯxpri}{}{ⓔznɯxpri} 
\classe{vt} \paradigme{dir}{nɯ-}
\begin{définition}\pfra{prendre ... comme prétexte}\end{définition}
\begin{définition}\pcmn{以……为借口}\end{définition}
\begin{exemple}\pjya{kɯ-raχtɯ nɯ-znɯxpri-t-a tɕe jɤ-ari-a}\hspace{5pt}\pcmn{我以买东西为借口去了那边}\end{exemple}
\begin{exemple}\pjya{ɯ-kɤ-znɯxpri ci pjɤ-tu}\hspace{5pt}\pcmn{他有个借口}\end{exemple}
\begin{exemple}\pjya{ɯ-znɯxpri ɲɤ-ɕar}\hspace{5pt}\pcmn{他找了个借口}\end{exemple}\relationsémantique{参考}{\lien{ⓔtɯpri}{tɯpri}}\end{entrée}

\begin{entrée}{znɯχamba}{}{ⓔznɯχamba}\relationsémantique{参考}{\lien{ⓔrɯχamba}{rɯχamba}}\end{entrée}

\begin{entrée}{znɯχcɤl}{}{ⓔznɯχcɤl} 
\classe{vt} \paradigme{dir}{tɤ-}
\begin{définition}\pfra{atteindre la cible}\end{définition}
\begin{définition}\pcmn{打中}\end{définition}
\begin{exemple}\pjya{tɤ-fsɯr ɯ-taʁ tɤ-znɯχcal-a ʑo tɤ-lat-a}\hspace{5pt}\pcmn{我对着靶子中心射了枪}\end{exemple}
\begin{exemple}\pjya{ɕɤmɯɣdɯ tɤ-znɯχcal-a pɯ-cha-a}\hspace{5pt}\pcmn{我射枪成功地射中了}\end{exemple}\relationsémantique{参考}{\lien{ⓔɯ-χcɤl}{ɯ-χcɤl}}\end{entrée}

\begin{entrée}{znɯχpi}{}{ⓔznɯχpi} 
\classe{vt} \paradigme{dir}{pɯ-}
\begin{définition}\pfra{imiter}\end{définition}
\begin{définition}\pcmn{模仿}\end{définition}
\begin{exemple}\pjya{a-tɕhemɤχti ɣɯ ɯ-tɯ-rɤt nɯ pɯ-znɯχpi-t-a}\hspace{5pt}\pcmn{我模仿了我女朋友的字}\end{exemple}\étymologie{dpe}\end{entrée}

\begin{entrée}{znɯχtɕɯrɯ}{}{ⓔznɯχtɕɯrɯ}\relationsémantique{参考}{\lien{ⓔnɯχtɕɯrɯ}{nɯχtɕɯrɯ}}\end{entrée}

\begin{entrée}{znɯzɤz}{}{ⓔznɯzɤz} 
\classe{vt} \paradigme{dir}{tɤ-}
\begin{définition}\pfra{appâter}\end{définition}
\begin{définition}\pcmn{用……引诱}\end{définition}
\begin{exemple}\pjya{nɤki ŋgumdʑɯɣ kɤ-znɯzɤz mɤ-khɯ}\hspace{5pt}\pcmn{那个领导不会被引诱}\end{exemple}\end{entrée}

\begin{entrée}{znɯzdɯɣ}{}{ⓔznɯzdɯɣ}\relationsémantique{参考}{\lien{ⓔnɯzdɯɣ}{nɯzdɯɣ}}\end{entrée}

\begin{entrée}{znɯzɟɯ}{}{ⓔznɯzɟɯ}\relationsémantique{参考}{\lien{ⓔnɯzɟɯ}{nɯzɟɯ}}\end{entrée}

\begin{entrée}{znɯʑɣɤʑɣɤt}{}{ⓔznɯʑɣɤʑɣɤt}\relationsémantique{参考}{\lien{ⓔsɤʑɣɤʑɣɤt}{sɤʑɣɤʑɣɤt}}\end{entrée}

\begin{entrée}{zɲɟa}{₂}{ⓔzɲɟaⓗ2} 
\classe{n} 
\begin{définition}\pfra{une espèce d'arbrisseau}\end{définition}
\begin{définition}\pcmn{【黄刺泡儿】}\end{définition}
\begin{exemple}\pjya{zɲɟa nɯ tɯ-ji mŋu ndo ra kɤ-ɬoʁ rga, ɯ-jwaʁ kɯ-ɤɲaʁndzɯm ŋu, ɯʑo mɤ-mbro, ɯ-jwaʁ ɯ-taʁ ɯ-ru ɯ-taʁ ɯ-mdzu dɤn, wuma ʑo mtɕoʁ, ɯ-mɯntoʁ kɯ-wɣrum ŋu, ɯ-mat thɯ-tɯt tɕe ʁmɤrsɤr ŋu, wuma ʑo chi. ɯ-mat kɯ-ndɯ-ndɯβ ʑo kɯ-ɤrtɯ-rtɯm ʑo boʁ boʁ ŋu}\hspace{5pt}\pcmn{黄刺泡儿一般生长在田边地角,叶子是暗绿色的,长得不高,叶子和茎上长满尖锐的刺,花是白色的,果实成熟后呈金黄色,很甜。果实是由聚集在一起的小球组成的。}\end{exemple}\relationsémantique{参考}{\lien{ⓔzɲɟɤsɯsi}{zɲɟɤsɯsi}}\end{entrée}

\begin{entrée}{zɲɟa}{₁}{ⓔzɲɟaⓗ1} 
\classe{vs} \paradigme{dir}{tɤ-}
\begin{définition}\pfra{stablement maintenu}\end{définition}
\begin{définition}\pcmn{夹得稳(夹子)}\end{définition}
\begin{exemple}\pjya{tamɢom ɲɯ-zɲɟa}\hspace{5pt}\pcmn{夹子夹得稳}\end{exemple}
\begin{exemple}\pjya{tamɢom nɯ kɯ a-mtɕhɯrme thɯ-sɯ-phɯt-a ri ɲɯ-zɲɟa tɕe ɲɯ-pe}\hspace{5pt}\pcmn{我用夹子拔了胡子,它夹得非常稳}\end{exemple}\end{entrée}

\begin{entrée}{zɲɟɤsɯsi}{}{ⓔzɲɟɤsɯsi} 
\classe{n} 
\begin{définition}\pfra{espèce de baie}\end{définition}
\begin{définition}\pcmn{黄刺泡儿的果子}\end{définition}\relationsémantique{参考}{\lien{ⓔzɲɟaⓗ2}{zɲɟa₂}}\end{entrée}

\begin{entrée}{zɲɟɤʑru}{}{ⓔzɲɟɤʑru} 
\classe{n} 
\begin{définition}\pfra{une plante}\end{définition}
\begin{définition}\pcmn{植物的一种}\end{définition}
\begin{exemple}\pjya{zɲɟɤʑru nɯ si kɯ-mbɤr ci ŋu, ʁnɯ-tɯphu tu, tɯ-tɯphu nɯ ɯ-ru kɯ-ɣɯrni ŋu, ɯ-jwaʁ ɯ-ru ɯ-taʁ chɯ-ɤʑɯrja ŋu, ɯ-mat nɯ ɯ-jwaʁ rca chɯ-ɤʑɯrja ŋu, tɕe thɯ-tɯt tɕe, ci kɯ-qarŋe ci tu, kɯ-ɲaʁ ci tu, tɕe kɯ-qarŋe nɯ jndʐɤz, kɯ-ɲaʁ nɯ ndɯβ, tú-wɣ-ndza tɕe chi, zɲɟɤʑru ɯ-ru cho ɯ-jwaʁ ɯ-taʁ ra ɯ-mdzu kɯ-ndɯβ tsa tu. mɤʑɯ tɯ-tɯphu nɯ ɯ-ru cho ɯ-jwaʁ tɯ-ɣndʑɤr thɯ-kɤ-mar ʑo kɯ-fse kɯ-wɣrum tu. ɯ-mat pjɯ-ɴqoʁ tsa ŋu. thɯ-tɯt tɕe qarŋe. kɤ-ndza mɯm. li ɯ-ru cho ɯ-jwaʁ ra ɯ-mdzu tu. nɯ kɯ tu-mbro tsa cha.}\hspace{5pt}\pcmn{\lien{ⓔzɲɟɤʑru}{zɲɟɤʑru}是一种矮小的树,分为两种。一种有红色的树干,叶子排列在树干上,果实也和叶子长在一起。成熟后,有的是黄色,有的是黑色的,黄色的较大,黑色的较小,吃起来很甜。\lien{ⓔzɲɟɤʑru}{zɲɟɤʑru}的树干和叶子上有小刺。还有一种,树干和叶子上好像涂了白粉一样。果实垂吊着,成熟后变黄。可以吃。树干和叶子上也长有刺。这种\lien{ⓔzɲɟɤʑru}{zɲɟɤʑru}长得比较高一些。}\end{exemple}\end{entrée}

\begin{entrée}{zo}{}{ⓔzo} 
\classe{vi} \paradigme{dir}{kɤ-}
\begin{définition}\pfra{se poser (oiseau)}\end{définition}
\begin{définition}\pcmn{停落(鸟)}\end{définition}
\begin{exemple}\pjya{pɣa ko-zo}\hspace{5pt}\pcmn{鸟停落了}\end{exemple}
\begin{exemple}\pjya{ɣʑo ko-zo}\hspace{5pt}\pcmn{蜜蜂停落了}\end{exemple}\end{entrée}

\begin{entrée}{zoŋzoŋ}{}{ⓔzoŋzoŋ} 
\classe{idph.2} 
\begin{définition}\pfra{ébouriffé, en désordre}\end{définition}
\begin{définition}\pcmn{形容人蓬头垢面,头发乱蓬蓬的样子,或形容动物的尾巴粗而毛发多}\end{définition}
\begin{exemple}\pjya{nɤ-ku pɯ-sɤɕɤt ma zoŋzoŋ ʑo ɲɯ-pa}\hspace{5pt}\pcmn{你梳一下头,你的头发乱蓬蓬的}\end{exemple}\relationsémantique{参考}{\lien{ⓔzaŋzaŋ}{zaŋzaŋ}}\end{entrée}

\begin{entrée}{zraʁrɯz}{}{ⓔzraʁrɯz}\relationsémantique{参考}{\lien{ⓔraʁrɯz}{raʁrɯz}}\end{entrée}

\begin{entrée}{zraχtɕi}{}{ⓔzraχtɕi} 
\classe{n} 
\begin{définition}\pfra{savon}\end{définition}
\begin{définition}\pcmn{肥皂}\end{définition}\end{entrée}

\begin{entrée}{zrɤβ}{}{ⓔzrɤβ} 
\classe{n} 
\begin{définition}\pfra{bouc}\end{définition}
\begin{définition}\pcmn{公山羊}\end{définition}\end{entrée}

\begin{entrée}{zrɤβraʁ}{}{ⓔzrɤβraʁ}\relationsémantique{参考}{\lien{ⓔrɤβraʁ}{rɤβraʁ}}\end{entrée}

\begin{entrée}{zrɤβzjoz}{}{ⓔzrɤβzjoz}\relationsémantique{参考}{\lien{ⓔrɤβzjoz}{rɤβzjoz}}\end{entrée}

\begin{entrée}{zrɤɣrɯ}{}{ⓔzrɤɣrɯ}\relationsémantique{参考}{\lien{ⓔrɤɣrɯ}{rɤɣrɯ}}\end{entrée}

\begin{entrée}{zrɤjroʁ}{}{ⓔzrɤjroʁ}\relationsémantique{参考}{\lien{ⓔrɤjroʁ}{rɤjroʁ}}\end{entrée}

\begin{entrée}{zrɤkrɯ}{}{ⓔzrɤkrɯ}\relationsémantique{参考}{\lien{ⓔrɤkrɯ}{rɤkrɯ}}\end{entrée}

\begin{entrée}{zrɤma}{}{ⓔzrɤma}\relationsémantique{参考}{\lien{ⓔrɤma}{rɤma}}\end{entrée}

\begin{entrée}{zrɤmgo}{}{ⓔzrɤmgo} 
\classe{vt} \paradigme{dir}{tɤ-}
\begin{définition}\pfra{mélanger une poudre avec un liquide et en faire des boules}\end{définition}
\begin{définition}\pcmn{把粉状的物体跟液体混在一起,揉成一坨一坨}\end{définition}
\begin{exemple}\pjya{tɯ-ci kɯ tɤjlu tɤ-zrɤmgo-t-a}\hspace{5pt}\pcmn{我在面粉里放了一点水,揉成一坨一坨}\end{exemple}\relationsémantique{参考}{\lien{ⓔtɯ-mgo}{tɯ-mgo}}\end{entrée}

\begin{entrée}{zrɤmpɕɤr}{}{ⓔzrɤmpɕɤr}\relationsémantique{参考}{\lien{ⓔrɤmpɕɤr}{rɤmpɕɤr}}\end{entrée}

\begin{entrée}{zrɤndzraʁ}{}{ⓔzrɤndzraʁ}\relationsémantique{参考}{\lien{ⓔrɤndzraʁ}{rɤndzraʁ}}\end{entrée}

\begin{entrée}{zrɤntɕɯ}{}{ⓔzrɤntɕɯ} 
\classe{n} 
\begin{définition}\pfra{haricot}\end{définition}
\begin{définition}\pcmn{绿豆}\end{définition}
\begin{exemple}\pjya{zrɤntɕɯ nɯ li tɤ-rɤku ci ŋu, tɯ-ji nɯ mɤ-kɯ-sna tsa lu-ji khɯ, ɯʑo kɯ-xtɕi ci ŋu, tu-wxti mɤ-cha, ɯ-tshɯɣa nɯ staχpɯ cho naχtɕɯɣ, ɯ-jwaʁ, ɯ-ru, ɯ-mɯntoʁ, ɯ-cɤβ nɯ ra lonba staχpɯ fsɯ-fse ʑo fse, staχpɯ wuma ʑɤ tu-rɲɟi cha, ɯ-rdoʁ ɯ-jndʐɤz artɯm rloʁrloʁ. zrɤntɕɯ ɯ-rdoʁ xtɕi cho aɕpɯɕpa tsa.}\hspace{5pt}\pcmn{绿豆是一种庄稼,可以种在贫瘠的地里。它较小,长不大,形状和豌豆一样,叶子、茎、花、荚果和豌豆一模一样。豌豆长得长,颗粒是球形的。绿豆的颗粒小而扁。}\end{exemple}\étymologie{sran}\end{entrée}

\begin{entrée}{zrɤpɯ}{}{ⓔzrɤpɯ}\relationsémantique{参考}{\lien{ⓔrɤpɯ}{rɤpɯ}}\end{entrée}

\begin{entrée}{zrɤru}{}{ⓔzrɤru}\relationsémantique{参考}{\lien{ⓔrɤru}{rɤru}}\end{entrée}

\begin{entrée}{zrɤrɤt}{}{ⓔzrɤrɤt}\relationsémantique{参考}{\lien{ⓔrɤt}{rɤt}}\end{entrée}

\begin{entrée}{zrɤrɟit}{}{ⓔzrɤrɟit}\relationsémantique{参考}{\lien{ⓔrɤrɟit}{rɤrɟit}}\end{entrée}

\begin{entrée}{zrɤrmbɣo}{}{ⓔzrɤrmbɣo}\relationsémantique{参考}{\lien{ⓔrɤrmbɣo}{rɤrmbɣo}}\end{entrée}

\begin{entrée}{zrɤsta}{}{ⓔzrɤsta}\relationsémantique{参考}{\lien{ⓔrɤsta}{rɤsta}}\end{entrée}

\begin{entrée}{zrɤtɕha}{}{ⓔzrɤtɕha} 
\classe{vt} 
\begin{définition}\pfra{déterminé à partir de}\end{définition}
\begin{définition}\pcmn{以……为标准}\end{définition}
\begin{exemple}\pjya{nɤ-sɯm ɕe mɤ-ɕe tɤ-zrɤtɕhe}\hspace{5pt}\pcmn{以你想不想(做)为标准}\end{exemple}
\begin{exemple}\pjya{ɯ-spa rtaʁ mɤ-rtaʁ tɤ-zrɤtɕhe}\hspace{5pt}\pcmn{(你衣服裁得多不多)看材料够不够}\end{exemple}
\begin{sous-entrée}{arɤtɕha}{ⓔzrɤtɕhaⓝarɤtɕha} 
\classe{vi} 
\begin{exemple}\pjya{ɯ-ngra pe mɤ-pe nɯ, kɤ-nɤma pe mɤ-pe arɤtɕha}\hspace{5pt}\pcmn{他的工钱高不高,看他工作做得好不好}\end{exemple}
\begin{exemple}\pjya{kɤ-zɣɯt tɯ-cha mɤ-tɯ-cha nɯ kɤ-rɟɯɣ tɯ-cha mɤ-tɯ-cha arɤtɕha}\hspace{5pt}\pcmn{你能不能早到,看你能不能跑步}\end{exemple}\end{sous-entrée}

\end{entrée}

\begin{entrée}{zrɤtshi}{}{ⓔzrɤtshi}\relationsémantique{参考}{\lien{ⓔarɤtshi}{arɤtshi}}\end{entrée}

\begin{entrée}{zrɤʑi}{}{ⓔzrɤʑi} 
\classe{vt} \paradigme{dir}{kɤ-}\sens{1}
\begin{définition}\pfra{faire habiter}\end{définition}
\begin{définition}\pcmn{使……住在}\end{définition}\sens{2}
\begin{définition}\pfra{laisser}\end{définition}
\begin{définition}\pcmn{留下}\end{définition}
\begin{exemple}\pjya{ɯʑo kɯre kɤ-zrɤʑi-t-a}\hspace{5pt}\pcmn{我让他待在这里了}\end{exemple}
\begin{exemple}\pjya{izora tɤ-pɤtso kha ɯ-ngɯ ɯʑosti kɤ-zrɤʑi-j}\hspace{5pt}\pcmn{我们把孩子一个人留在家里}\end{exemple}\end{entrée}

\begin{entrée}{zri}{}{ⓔzri} 
\classe{vs} \paradigme{dir}{thɯ-}
\begin{définition}\pfra{long}\end{définition}
\begin{définition}\pcmn{长}\end{définition}\relationsémantique{同义词}{\lien{ⓔrɲɟi}{rɲɟi}}\relationsémantique{反义词}{\lien{ⓔxtɯtⓗ2}{xtɯt}}\end{entrée}

\begin{entrée}{zrɯ}{₁}{ⓔzrɯⓗ1} 
\classe{n} 
\begin{définition}\pfra{parasite des bovins}\end{définition}
\begin{définition}\pcmn{牛的寄生虫}\end{définition}\end{entrée}

\begin{entrée}{zrɯ}{₂}{ⓔzrɯⓗ2} 
\classe{n} 
\begin{définition}\pfra{adret}\end{définition}
\begin{définition}\pcmn{山阳,向阳的山坡}\end{définition}\end{entrée}

\begin{entrée}{zrɯ}{₃}{ⓔzrɯⓗ3} 
\classe{vt} \paradigme{dir}{nɯ-}
\begin{définition}\pfra{s'accaparer}\end{définition}
\begin{définition}\pcmn{霸占,占用}\end{définition}
\begin{exemple}\pjya{sɤtɕha nɯ-zrɯ-t-a}\hspace{5pt}\pcmn{我占用了这个地方}\end{exemple}\end{entrée}

\begin{entrée}{zrɯβɟu}{}{ⓔzrɯβɟu} 
\classe{n} 
\begin{définition}\pfra{dans un fardeau de bois, la partie qui est en contact avec le dos du porteur}\end{définition}
\begin{définition}\pcmn{柴捆子里比较细的枝条,接触人的背部}\end{définition}\relationsémantique{反义词}{\lien{ⓔzɣɯqhu}{zɣɯqhu}}\end{entrée}

\begin{entrée}{zrɯɕmi}{}{ⓔzrɯɕmi}\relationsémantique{参考}{\lien{ⓔrɯɕmi}{rɯɕmi}}\end{entrée}

\begin{entrée}{zrɯɣ}{}{ⓔzrɯɣ} 
\classe{n} 
\begin{définition}\pfra{pou}\end{définition}
\begin{définition}\pcmn{虱子}\end{définition}
\begin{exemple}\pjya{zrɯɣ nɯ-nɤmbɣaʁlaʁ rdɯl mɤ-tɕɤt}\hspace{5pt}\pcmn{虱子打滚也不会起灰尘(没有什么可怕的)}\end{exemple}\relationsémantique{参考}{\lien{ⓔaɣɯzrɯɣ}{aɣɯzrɯɣ}}\end{entrée}

\begin{entrée}{zrɯɣnɤn}{}{ⓔzrɯɣnɤn}\relationsémantique{参考}{\lien{ⓔrɯɣnɤn}{rɯɣnɤn}}\end{entrée}

\begin{entrée}{zrɯɣndza}{}{ⓔzrɯɣndza} 
\classe{n} 
\begin{définition}\pfra{mante religieuse}\end{définition}
\begin{définition}\pcmn{螳螂}\end{définition}
\begin{exemple}\pjya{zrɯɣndza nɯ qajɯ ci ŋu, ɯ-mi kɯtʂɤ-ldʑa tu, ɯ-ku kɯ-xtɕi tɕe kɯ-ɤmtɕoʁ ci ŋu, ɯ-phoŋbu kɯ-wxti tsa ci ŋu, ɯ-smɤt tɕe chɯ-ɤmtɕoʁ tsa ŋu, tɕe ɯ-mgɯr ɯ-qhu ra rko, ɯ-xtɤpa ra mpɯ tɕe zrɯɣ kɤ-ndza wuma rga, kɯ-mɤku tɕe, zrɯɣ pha ɯ-phoŋbu tɯtɯrca chɯ-mqlaʁ ŋu, khro tsa ta-ndza tɕe, tu-fka ɲɯ-ŋu tɕe zrɯɣ nɯ-atɯɣ tɕe ɯ-se ku-tshi tɕe ɲɯ-βde ɲɯ-ŋu. pha ɯ-phoŋbu kɯ-ɤrŋi, sɯjno cho aɣɯmdoʁ.}\hspace{5pt}\pcmn{螳螂是一种虫,有六只脚,头小而尖,身子较大,尾部是尖的。背部硬,肚皮软。它爱吃虱子,开始是整个吃掉。吃了几个以后,饱了,再遇到虱子的时候,喝了血就扔了。螳螂全身是绿色的,和草的颜色一样。}\end{exemple}\end{entrée}

\begin{entrée}{zrɯɣru}{}{ⓔzrɯɣru} 
\classe{n} 
\begin{définition}\pfra{épouillage}\end{définition}
\begin{définition}\pcmn{捉虱子}\end{définition}\relationsémantique{参考}{\lien{ⓔzrɯɣ}{zrɯɣ}}\relationsémantique{参考}{\lien{ⓔruⓗ2}{ru₂}}\relationsémantique{参考}{\lien{ⓔnɯzrɯɣru}{nɯzrɯɣru}}\end{entrée}

\begin{entrée}{zrɯndzɤtshi}{}{ⓔzrɯndzɤtshi}\relationsémantique{参考}{\lien{ⓔrɯndzɤtshi}{rɯndzɤtshi}}\end{entrée}

\begin{entrée}{zrɯŋgrɤl}{}{ⓔzrɯŋgrɤl}\relationsémantique{参考}{\lien{ⓔŋgrɤl}{ŋgrɤl}}\end{entrée}

\begin{entrée}{zrɯstɯnmɯ}{}{ⓔzrɯstɯnmɯ}\relationsémantique{参考}{\lien{ⓔrɯstɯnmɯ}{rɯstɯnmɯ}}\end{entrée}

\begin{entrée}{zrɯxtar}{}{ⓔzrɯxtar} 
\classe{vt} \paradigme{dir}{tɤ-}
\begin{définition}\pfra{développer}\end{définition}
\begin{définition}\pcmn{使兴旺起来}\end{définition}
\begin{exemple}\pjya{jiɕqha ta-zrɯxtar-nɯ}\hspace{5pt}\pcmn{他们使它兴旺起来}\end{exemple}
\begin{exemple}\pjya{kha kɤ-zrɯxtar pjɤ-cha-nɯ}\hspace{5pt}\pcmn{他们成功地令自己的家庭兴旺起来了}\end{exemple}\relationsémantique{参考}{\lien{ⓔsɯxtar}{sɯxtar}}\end{entrée}

\begin{entrée}{zʁaʁzʁaʁ}{}{ⓔzʁaʁzʁaʁ} 
\classe{idph.2} 
\begin{définition}\pfra{correctement habillé}\end{définition}
\begin{définition}\pcmn{形容穿得很精干的样子}\end{définition}
\begin{exemple}\pjya{nɯ-ŋga ra tu-kɯ-ɣɤxtɯt nɯ zʁaʁzʁaʁ ʑo ɲɯ-pa}\hspace{5pt}\pcmn{穿高一点的衣服看起来很威武精干}\end{exemple}\relationsémantique{反义词}{\lien{ⓔlɲɯɣlɲɯɣ}{lɲɯɣlɲɯɣ}}\end{entrée}

\begin{entrée}{zʁɤɲcɯ}{}{ⓔzʁɤɲcɯ} 
\classe{n} 
\begin{définition}\pfra{fronde}\end{définition}
\begin{définition}\pcmn{投石带【石子带】}\end{définition}
\begin{exemple}\pjya{zʁɤɲcɯ ci to-lɤt tɕe pɣa to-sɯxtsɯɣ}\hspace{5pt}\pcmn{他用投石带射中了鸟}\end{exemple}
\begin{exemple}\pjya{zʁɤɲcɯ nɯ tɤ-fsɤri maʁ nɤ qase nɯ-kɤ-βzu ŋu tɕe ɯ-χcɤl nɯ tɕu rdɤstaʁ ɯ-sɤɣ-raʁ ci tú-wɣ-βzu tɕe, tú-wɣ-zdɤβ tɕe ɯ-ɕnɤz tɯka nɯ kú-wɣ-ndo, ɯ-χcɤl nɯ tɕu rdɤstaʁ kɯ-xtɕi tsa chɯ́-wɣ-sɯɣraʁ tɕe tɯ-ku ɯ-taʁ tɯ-jaʁ ntsi kɯ χsɯ-tɤxɯr jamar kú-wɣ-sɯ-sɯ-mtɕɯr tɕe ɯ-ɕnɤz tɯ-rdoʁ nɯ ɲɯ́-wɣ-ta tɕe ɲɯ́-wɣ-lɤt, tɕe rdɤstaʁ kɯ-ɤrqhi ʑo ju-ɕe cha. tɕe nɯ rdɤstaʁ sɤ-lɤt ɯ-spa ŋu.}\hspace{5pt}\pcmn{投石带用麻绳或者皮绳做成。中间做一个能卡住石子的(结),(从中间)叠一下。绳的两头拿在手上,在中间卡住小石子,然后在头上挥转三圈,之后再把绳子的一头放掉,打出去,这样石子就投得远一些。投石带是投掷石头的专用工具。}\end{exemple}\end{entrée}

\begin{entrée}{zʁɤzʁɤt}{}{ⓔzʁɤzʁɤt} 
\classe{idph.3} 
\begin{définition}\pfra{(enfant) habillé de façon correcte}\end{définition}
\begin{définition}\pcmn{形容(小孩子)穿得很整齐的样子}\end{définition}\end{entrée}

\begin{entrée}{zɯ}{}{ⓔzɯ} 
\classe{postp} 
\begin{définition}\pfra{locatif}\end{définition}
\begin{définition}\pcmn{在}\end{définition}\end{entrée}

\begin{entrée}{zɯɣzɯɣ}{}{ⓔzɯɣzɯɣ} 
\classe{idph.2} 
\begin{définition}\pfra{stable, immobile}\end{définition}
\begin{définition}\pcmn{稳定的状态,一动也不动}\end{définition}
\begin{exemple}\pjya{zɯɣzɯɣ ʑo ɲɯ-ɤsɯ-ndo}\hspace{5pt}\pcmn{他拿着不放,动也不动}\end{exemple}
\begin{exemple}\pjya{zɯɣzɯɣ ʑo ɲɯ-rɤʑi}\hspace{5pt}\pcmn{他坐在那里,动也不动}\end{exemple}\relationsémantique{同义词}{\lien{ⓔgrɯɣgrɯɣ}{grɯɣgrɯɣ}}\end{entrée}

\begin{entrée}{zɯm}{}{ⓔzɯm} 
\classe{n} 
\begin{définition}\pfra{seau}\end{définition}
\begin{définition}\pcmn{桶}\end{définition}\étymologie{zom}\end{entrée}

\begin{entrée}{zɯmbɯr}{}{ⓔzɯmbɯr} 
\classe{n} 
\begin{définition}\pfra{bouton d'argent}\end{définition}
\begin{définition}\pcmn{银盆}\end{définition}\end{entrée}

\begin{entrée}{zɯmi}{}{ⓔzɯmi} 
\classe{adv} 
\begin{définition}\pfra{presque}\end{définition}
\begin{définition}\pcmn{差一点;几乎}\end{définition}
\begin{exemple}\pjya{rdɤstaʁ ta-lɤt, zɯmi ʑo jɯ-tɤ́-wɣ-tsɯɣ-a}\hspace{5pt}\pcmn{差一点打到了我}\end{exemple}
\begin{exemple}\pjya{zɯmi ɲɯ-naχtɕɯɣ}\hspace{5pt}\pcmn{几乎一样}\end{exemple}\end{entrée}

\begin{entrée}{zɯmjɯ}{}{ⓔzɯmjɯ} 
\classe{n} 
\begin{définition}\pfra{lanière servant à porter les seaux d'eau}\end{définition}
\begin{définition}\pcmn{背水桶的带子}\end{définition}\end{entrée}

\begin{entrée}{zɯmzɯm}{}{ⓔzɯmzɯm} 
\classe{idph.2} 
\begin{définition}\pfra{coupé très fin}\end{définition}
\begin{définition}\pcmn{形容切得很细}\end{définition}
\begin{exemple}\pjya{pjɯ́-wɣ-rɤkrɯ zɯmzɯm ʑo ɲɯ-ra}\hspace{5pt}\pcmn{要切得很细}\end{exemple}\end{entrée}

\begin{entrée}{zɯn}{}{ⓔzɯn} 
\classe{n} 
\begin{définition}\pfra{argent de l'époque impériale}\end{définition}
\begin{définition}\pcmn{民国之前通行的货币}\end{définition}\end{entrée}

\begin{entrée}{zɯŋzɯŋ}{}{ⓔzɯŋzɯŋ} 
\classe{idph.2} 
\begin{définition}\pfra{complètement blanc}\end{définition}
\begin{définition}\pcmn{全白}\end{définition}
\begin{exemple}\pjya{rgɤtpu ɯ-ku cho-wɣrum zɯŋzɯŋ ʑo}\hspace{5pt}\pcmn{老头子的头发全变白了}\end{exemple}\end{entrée}

\begin{entrée}{zɯxtɕhɤl}{}{ⓔzɯxtɕhɤl} 
\classe{n} 
\begin{définition}\pfra{cymbales}\end{définition}
\begin{définition}\pcmn{钹}\end{définition}\étymologie{sbug.tɕʰal}\end{entrée}

\begin{entrée}{zwu}{}{ⓔzwu} 
\classe{n} 
\begin{définition}\pfra{maladie de l'œil}\end{définition}
\begin{définition}\pcmn{眼病}\end{définition}
\begin{exemple}\pjya{ɯ-mɲaʁ zwu to-ɣi}\hspace{5pt}\pcmn{他眼上长了痘痘}\end{exemple}\end{entrée}

\begin{entrée}{zwaʁnɤzwaʁ}{}{ⓔzwaʁnɤzwaʁ} 
\classe{idph.3} 
\begin{définition}\pfra{mou, pas ferme}\end{définition}
\begin{définition}\pcmn{形容物体绵软}\end{définition}
\begin{exemple}\pjya{tú-wɣ-ndza tɕe zwaʁnɤzwaʁ ɲɯ-ti}\hspace{5pt}\pcmn{吃起来软绵绵的}\end{exemple}\relationsémantique{反义词}{\lien{ⓔtɕʁɯznɤtɕʁɯz}{tɕʁɯznɤtɕʁɯz}}\end{entrée}

\begin{entrée}{zwɤɣrum}{}{ⓔzwɤɣrum} 
\classe{n} 
\begin{définition}\pfra{armoise blanche}\end{définition}
\begin{définition}\pcmn{白艾蒿}\end{définition}\relationsémantique{参考}{\lien{ⓔzwɤrⓗ2}{zwɤr₂}}\relationsémantique{参考}{\lien{ⓔwɣrum}{wɣrum}}\end{entrée}

\begin{entrée}{zwɤr}{₂}{ⓔzwɤrⓗ2} 
\classe{n} 
\begin{définition}\pfra{armoise}\end{définition}
\begin{définition}\pcmn{蒿}\end{définition}
\begin{exemple}\pjya{zwɤr nɯ ɯ-ru cho ɯ-jwaʁ kɤsɯfse kɯ-pɣi ŋu. ɯ-jwaʁ nɯ kɯ-ɤɣɯrʑɯɣʑɯɣ kɯ-fse ŋu. zwɤr ɯ-di χɕɤβ, kɤntɕhɯ-tɯphu tu, zwɤrqha kɯ-rmi ci tu, si ŋu. zwɤɣrum kɤ-ti ci tu tɕe nɯ aɣrɤɣrum. zwɤrɲaʁ kɤ-ti ci tu tɕe nɯ aɲaʁndzɯm. ɴqiazwɤr kɤ-ti ci tu tɕe nɯ ɯ-jwaʁ ra mba tsa ɯ-mdoʁ nɯ arŋi. kɤ-ndza sna. zwɤrqha, zwɤɣrum, zwɤrɲaʁ nɯ ra nɯ-jwaʁ ɯ-taʁ ɯ-rme sɯβsɯβ tu, ɴqiazwɤr ɯ-jwaʁ ɯ-taʁ ɯ-rme me.}\hspace{5pt}\pcmn{蒿的茎和叶子全部都是灰色的,叶子上有褶,香味浓。蒿分很多种。叫做\lien{ⓔzwɤrqha}{zwɤrqha}的是一种树。叫做\lien{}{zwɤrɣrum}的淡白色。叫做\lien{}{zwɤrɲaʁ}的颜色比较深。叫做\lien{}{qiaβzwɤr}(香蒿)的叶子薄,绿色,可以吃。前三种蒿叶子上有细毛,香蒿叶子没有毛。}\end{exemple}\relationsémantique{参考}{\lien{ⓔzwɤɣrum}{zwɤɣrum}}\relationsémantique{参考}{\lien{ⓔzwɤrqha}{zwɤrqha}}\relationsémantique{参考}{\lien{ⓔɴqiazwɤr}{ɴqiazwɤr}}\end{entrée}

\begin{entrée}{zwɤr}{₁}{ⓔzwɤrⓗ1} 
\classe{vt} \paradigme{dir}{kɤ-}\paradigme{dir}{tɤ-}
\begin{définition}\pfra{allumer}\end{définition}
\begin{définition}\pcmn{点火;点灯}\end{définition}
\begin{exemple}\pjya{smi ka-zwɤr, smi ta-zwɤr}\hspace{5pt}\pcmn{他点了火}\end{exemple}
\begin{exemple}\pjya{tɤtʂu kɤ-zwar-a}\hspace{5pt}\pcmn{我点了灯}\end{exemple}\relationsémantique{参考}{\lien{ⓔamɯzwɤr}{amɯzwɤr}}\étymologie{sbor}\end{entrée}

\begin{entrée}{zwɤrqha}{}{ⓔzwɤrqha} 
\classe{n} 
\begin{définition}\pfra{espèce d'armoise}\end{définition}
\begin{définition}\pcmn{艾蒿的一种}\end{définition}\relationsémantique{参考}{\lien{ⓔzwɤrⓗ2}{zwɤr₂}}\end{entrée}

\begin{entrée}{zwɤrqhɤjmɤɣ}{}{ⓔzwɤrqhɤjmɤɣ} 
\classe{n} 
\begin{définition}\pfra{une espèce de champignon}\end{définition}
\begin{définition}\pcmn{一种菌子}\end{définition}
\begin{exemple}\pjya{zwɤrqhɤjmɤɣ nɯ zwɤrqha ɯ-ŋgɯ tu-ɬoʁ ŋu, ɯ-taʁ ɯ-pa kɯ-fsɯ-fse ʑo kɯ-wɣrɯ-wɣrum ʑo ŋu, kɤ-ndza sna}\hspace{5pt}\pcmn{\lien{ⓔzwɤrqhɤjmɤɣ}{zwɤrqhɤjmɤɣ}长在\lien{ⓔzwɤrqha}{zwɤrqha} 树林里,上部和下部一样都是白色的,能吃。}\end{exemple}\end{entrée}

\newpage\caractère{ʑ}

\begin{entrée}{ʑu}{}{ⓔʑu} 
\classe{n} 
\begin{définition}\pfra{yaourt}\end{définition}
\begin{définition}\pcmn{酸奶}\end{définition}\étymologie{ʑo}\end{entrée}

\begin{entrée}{ʑa}{₁}{ⓔʑaⓗ1} 
\classe{vt} \paradigme{dir}{\_}
\begin{définition}\pfra{commencer}\end{définition}
\begin{définition}\pcmn{开始}\end{définition}
\begin{exemple}\pjya{ɯ-mphru pjɯ-ʑe-a tɕe pjɯ-ndɯn-a ŋu ŋɤ}\hspace{5pt}\pcmn{我紧接着(从头)开始读}\end{exemple}
\begin{exemple}\pjya{tɯ-nɯrɤɣo chɤ-ʑa}\hspace{5pt}\pcmn{他开始唱起歌来}\end{exemple}\relationsémantique{参考}{\lien{ⓔsɤʑa}{sɤʑa}}\end{entrée}

\begin{entrée}{ʑa}{₂}{ⓔʑaⓗ2} 
\classe{vs} \paradigme{dir}{nɯ-}
\begin{définition}\pfra{avoir une atrophie musculaire}\end{définition}
\begin{définition}\pcmn{肌肉萎缩}\end{définition}
\begin{exemple}\pjya{ɯ-mi ɲɤ-ʑa}\hspace{5pt}\pcmn{他脚的肌肉萎缩了}\end{exemple}
\begin{exemple}\pjya{ɯ-jaʁ ɲɤ-ʑa}\hspace{5pt}\pcmn{他手的肌肉萎缩了}\end{exemple}\end{entrée}

\begin{entrée}{ʑaka}{}{ⓔʑaka} 
\classe{n} 
\begin{définition}\pfra{chacun}\end{définition}
\begin{définition}\pcmn{各自}\end{définition}\end{entrée}

\begin{entrée}{ʑakastaka}{}{ⓔʑakastaka} 
\classe{n} 
\begin{définition}\pfra{chacun le sien}\end{définition}
\begin{définition}\pcmn{各自各地}\end{définition}
\begin{exemple}\pjya{mɯntoʁ ʑakastaka ɯ-mdoʁ xcat ʑo ɕti}\hspace{5pt}\pcmn{花有各种各样的颜色}\end{exemple}
\begin{exemple}\pjya{jiʑora ʑakastaka ji-ma nɯ-nɤma-j ra}\hspace{5pt}\pcmn{我们要各自办各自的事情}\end{exemple}\end{entrée}

\begin{entrée}{ʑala}{}{ⓔʑala} 
\classe{n} 
\begin{définition}\pfra{application de glaise sur les murs pour les rendre plus lisse}\end{définition}
\begin{définition}\pcmn{在墙壁上涂上水泥使其光滑【敷墙壁】}\end{définition}\end{entrée}

\begin{entrée}{ʑaŋɬa}{}{ⓔʑaŋɬa} 
\classe{n} 
\begin{définition}\pfra{silex que l'on place au milieu du champs}\end{définition}
\begin{définition}\pcmn{放在田地中间的燧石}\end{définition}\relationsémantique{参考}{\lien{ⓔqapiⓗ1}{qapi₁}}\étymologie{ʑiŋ.ɬa}\end{entrée}

\begin{entrée}{ʑaŋmu}{}{ⓔʑaŋmu} 
\classe{n} 
\begin{définition}\pfra{le plus grand champs}\end{définition}
\begin{définition}\pcmn{最大的田地}\end{définition}\étymologie{ʑiŋ.mo}\end{entrée}

\begin{entrée}{ʑaŋndza}{}{ⓔʑaŋndza} 
\classe{n} 
\begin{définition}\pfra{festin}\end{définition}
\begin{définition}\pcmn{宴会}\end{définition}
\begin{exemple}\pjya{ʑaŋndza chɤ-lɤt-nɯ}\hspace{5pt}\pcmn{他们办了宴会}\end{exemple}
\begin{exemple}\pjya{jiʑora ʑaŋndza thɯ-lɤt-i}\hspace{5pt}\pcmn{我们办了宴会}\end{exemple}\end{entrée}

\begin{entrée}{ʑaŋpjaʁ}{}{ⓔʑaŋpjaʁ} 
\classe{n} 
\begin{définition}\pfra{outil pour faire cuire les momo}\end{définition}
\begin{définition}\pcmn{【烙片】用来炕馍馍的用具}\end{définition}\end{entrée}

\begin{entrée}{ʑara}{}{ⓔʑara} 
\classe{pro} 
\begin{définition}\pfra{eux}\end{définition}
\begin{définition}\pcmn{他们}\end{définition}\end{entrée}

\begin{entrée}{ʑaʁ}{₂}{ⓔʑaʁⓗ2} 
\classe{n} 
\begin{définition}\pfra{pellicule de graisse}\end{définition}
\begin{définition}\pcmn{浮油}\end{définition}\étymologie{ʑag}\end{entrée}

\begin{entrée}{ʑaʁ}{₃}{ⓔʑaʁⓗ3} 
\classe{n} 
\begin{définition}\pfra{jour}\end{définition}
\begin{définition}\pcmn{天}\end{définition}
\begin{exemple}\pjya{ʑaʁ χsɤ-rʑaʁ}\hspace{5pt}\pcmn{三天}\end{exemple}
\begin{exemple}\pjya{ʑaʁ χsɯ-sŋi}\hspace{5pt}\pcmn{三个白天}\end{exemple}\étymologie{ʑag}\end{entrée}

\begin{entrée}{ʑatsa}{}{ⓔʑatsa} 
\classe{n} 
\begin{définition}\pfra{bientôt}\end{définition}
\begin{définition}\pcmn{快要}\end{définition}
\begin{exemple}\pjya{dian ʑatsa arɕo ɲɯ-ŋu}\hspace{5pt}\pcmn{快没有电了}\end{exemple}\end{entrée}

\begin{entrée}{ʑaʑa}{}{ⓔʑaʑa} 
\classe{n} \sens{1}
\begin{définition}\pfra{longtemps avant}\end{définition}
\begin{définition}\pcmn{早就}\end{définition}\sens{2}
\begin{définition}\pfra{pendant longtemps}\end{définition}
\begin{définition}\pcmn{很久}\end{définition}
\begin{exemple}\pjya{ʑaʑa ʑo jɤ-azɣɯt-a}\hspace{5pt}\pcmn{我早就到了}\end{exemple}
\begin{exemple}\pjya{ʑaʑa ʑo mɯ-jɤ-azɣɯt}\hspace{5pt}\pcmn{他很久都没有来}\end{exemple}\end{entrée}

\begin{entrée}{ʑɤn}{}{ⓔʑɤn} 
\classe{vs} 
\begin{définition}\pfra{moins bon}\end{définition}
\begin{définition}\pcmn{(比自己)差}\end{définition}
\begin{exemple}\pjya{ɯ-kɤ-spa rkɯn tɕe, aʑo sɤznɤ ʑɤn}\hspace{5pt}\pcmn{他会做的事情很少,他比我差}\end{exemple}\relationsémantique{反义词}{\lien{ⓔmna}{mna}}\end{entrée}

\begin{entrée}{ʑɤni}{}{ⓔʑɤni} 
\classe{pro} 
\begin{définition}\pfra{eux deux}\end{définition}
\begin{définition}\pcmn{他们俩}\end{définition}\end{entrée}

\begin{entrée}{ʑɤŋgɯz}{}{ⓔʑɤŋgɯz} 
\classe{adv} 
\begin{définition}\pfra{l'un à l'autre}\end{définition}
\begin{définition}\pcmn{互相}\end{définition}
\begin{exemple}\pjya{tɕiʑo ʑɤŋgɯz azɣɤʁrɯʁre-tɕi}\hspace{5pt}\pcmn{我们俩互相尊重}\end{exemple}\relationsémantique{参考}{\lien{ⓔɯ-ŋgɯ}{ɯ-ŋgɯ}}\end{entrée}

\begin{entrée}{ʑɤwu}{}{ⓔʑɤwu} 
\classe{n} 
\begin{définition}\pfra{boiteux}\end{définition}
\begin{définition}\pcmn{跛子}\end{définition}\relationsémantique{同义词}{\lien{ⓔɕkala}{ɕkala}}\relationsémantique{参考}{\lien{ⓔaʑɤwu}{aʑɤwu}}\étymologie{ʑa.bo}\end{entrée}

\begin{entrée}{ʑɤzdaŋ}{}{ⓔʑɤzdaŋ} 
\classe{n} 
\begin{définition}\pfra{envie, volonté de dépasser les autres}\end{définition}
\begin{définition}\pcmn{妒忌;有赶上别人的心}\end{définition}\étymologie{ʑe.sdaŋ}\end{entrée}

\begin{entrée}{ʑdraŋʑdraŋ}{}{ⓔʑdraŋʑdraŋ} 
\classe{idph.2} 
\begin{définition}\pfra{objet long et flexible}\end{définition}
\begin{définition}\pcmn{形容物体(衣服,软树枝等)又长又软的样子}\end{définition}
\begin{exemple}\pjya{tɯmbri ʑdraŋʑdraŋ ʑo ɲɯ-ɤta}\hspace{5pt}\pcmn{绳子放在那里,很凌乱的样子}\end{exemple}
\begin{sous-entrée}{ʑdraŋnɤʑdraŋ}{ⓔʑdraŋʑdraŋⓝʑdraŋnɤʑdraŋ} 
\classe{idph.3} 
\begin{définition}\pcmn{地上有旧的衣服,他带走了}\end{définition}
\begin{exemple}\pjya{tɯ-ŋgɤmbe ɯ-thoʁ ɲɯ-ɤta tɕe, ʑdraŋnɤʑdraŋ ka-tsɯm}\end{exemple}\end{sous-entrée}

\begin{sous-entrée}{sɤʑdraŋlaŋ}{ⓔʑdraŋʑdraŋⓝsɤʑdraŋlaŋ} 
\classe{vt} 
\begin{définition}\pfra{secouer légèrement (objet long et flexible)}\end{définition}
\begin{définition}\pcmn{抖动(又长又软的东西)}\end{définition}\relationsémantique{同义词}{\lien{ⓔɕtʂɯɣɕtʂɯɣⓝsɤɕtʂɯlɯɣ}{sɤɕtʂɯlɯɣ}}\relationsémantique{同义词}{\lien{ⓔɕtʂaŋɕtʂaŋⓝsɤɕtʂaŋlaŋ}{sɤɕtʂaŋlaŋ}}\end{sous-entrée}

\end{entrée}

\begin{entrée}{ʑdrɤβʑdrɤβ}{}{ⓔʑdrɤβʑdrɤβ} 
\classe{idph.2} 
\begin{définition}\pfra{objet long et mou}\end{définition}
\begin{définition}\pcmn{形容物体长而软的样子}\end{définition}
\begin{exemple}\pjya{razmbe ʑdrɤβʑdrɤβ ʑo ɲɯ-ɴqoʁ}\hspace{5pt}\pcmn{烂布条又脏又长地在那里挂着}\end{exemple}
\begin{exemple}\pjya{razmbe ʑdrɤβʑdrɤβ ʑo ɲɯ-ɤta}\hspace{5pt}\pcmn{烂布条又脏又长地放在那里}\end{exemple}
\begin{exemple}\pjya{nɤ-χsɯmχsoz jɤ-sɯɣe ʑdrɤβʑdrɤβ ʑo ma-tɯ-ʑɣɤstu}\hspace{5pt}\pcmn{你要打起精神来,不要无精打采的样子}\end{exemple}\relationsémantique{参考}{\lien{ⓔɕthrɤβɕthrɤβ}{ɕthrɤβɕthrɤβ}}\end{entrée}

\begin{entrée}{ʑdɯɣʑdɯɣ}{}{ⓔʑdɯɣʑdɯɣ} 
\classe{idph.2} 
\begin{définition}\pfra{compact, solide}\end{définition}
\begin{définition}\pcmn{形容物体紧凑或牢固}\end{définition}
\begin{exemple}\pjya{nɤki tʂɤm ɲɯ-ɤrku tɕe, ʑdɯɣʑdɯɣ ʑo ɲɯ-ɤstu}\hspace{5pt}\pcmn{板壁装得很紧凑}\end{exemple}
\begin{exemple}\pjya{ɯ-ɕɣa ɯ-tɯ-pe kɯ ʑdɯɣʑdɯɣ ʑo ɲɯ-pa}\hspace{5pt}\pcmn{他的牙齿很牢固}\end{exemple}
\begin{exemple}\pjya{kɯm ʑdɯɣʑdɯɣ ʑo pjɤ-nɯ-sɤtsa}\hspace{5pt}\pcmn{他把门锁得死死的}\end{exemple}\end{entrée}

\begin{entrée}{ʑgaʁ}{}{ⓔʑgaʁ} 
\classe{adv} 
\begin{définition}\pfra{tout juste}\end{définition}
\begin{définition}\pcmn{刚好}\end{définition}
\begin{exemple}\pjya{jisŋi tɕe, tɯ-sla ʑgaʁ ʑo tu-tsu ŋu}\hspace{5pt}\pcmn{到今天就刚好满一个月了}\end{exemple}
\begin{exemple}\pjya{kɯtʂɤɣ ɯ-taʁ kɯβde pjɯ́-wɣ-ta tɕe, sqi ʑgaʁ ŋu}\hspace{5pt}\pcmn{六加四正好等于十}\end{exemple}\end{entrée}

\begin{entrée}{ʑgɤβʑgɤβ}{}{ⓔʑgɤβʑgɤβ} 
\classe{idph.2} 
\begin{définition}\pfra{grand, maigre et bossu}\end{définition}
\begin{définition}\pcmn{形容高、瘦而驼背的样子}\end{définition}\relationsémantique{同义词}{\lien{ⓔɕkɤɣɕkɤɣ}{ɕkɤɣɕkɤɣ}}\end{entrée}

\begin{entrée}{ʑgrɤɣʑgrɤɣ}{}{ⓔʑgrɤɣʑgrɤɣ} 
\classe{idph.2} 
\begin{définition}\pfra{dur et froid (sensation lorsqu'on s'allonge sur le sol)}\end{définition}
\begin{définition}\pcmn{形容躺在地上又没有衣服盖的感觉,又冷又硬的感觉。}\end{définition}
\begin{exemple}\pjya{ʑgrɤɣʑgrɤɣ ʑo ɲɯ-rŋgɯ}\hspace{5pt}\pcmn{他躺在地上,感觉地面又硬又冷}\end{exemple}\relationsémantique{参考}{\lien{ⓔnɯʑgrɤɣ}{nɯʑgrɤɣ}}\end{entrée}

\begin{entrée}{ʑgrɯɣ/\variante{ʑgrɯ}}{}{ⓔʑgrɯɣ} 
\classe{n} 
\begin{définition}\pfra{certainement}\end{définition}
\begin{définition}\pcmn{一定}\end{définition}\end{entrée}

\begin{entrée}{ʑɣɤβde}{}{ⓔʑɣɤβde} 
\classe{vi}  
\grammaire{refl} \paradigme{dir}{thɯ-}
\begin{définition}\pfra{se suicider en se jetant à l'eau}\end{définition}
\begin{définition}\pcmn{投河自尽}\end{définition}
\begin{exemple}\pjya{ma-ɕ-thɯ-tɯ-ʑɣɤβde ma nɤ-mu ɲɯ-ɤkhu}\hspace{5pt}\pcmn{你不要去投河自尽,你母亲在叫你}\end{exemple}\relationsémantique{参考}{\lien{ⓔβde}{βde}}\end{entrée}

\begin{entrée}{ʑɣɤβʁum}{}{ⓔʑɣɤβʁum}\relationsémantique{参考}{\lien{ⓔβʁum}{βʁum}}\end{entrée}

\begin{entrée}{ʑɣɤβzɟɯr}{}{ⓔʑɣɤβzɟɯr}\relationsémantique{参考}{\lien{ⓔβzɟɯr}{βzɟɯr}}\end{entrée}

\begin{entrée}{ʑɣɤɕphɣo}{}{ⓔʑɣɤɕphɣo}\relationsémantique{参考}{\lien{ⓔɕphɣo}{ɕphɣo}}\end{entrée}

\begin{entrée}{ʑɣɤɕthɯz}{}{ⓔʑɣɤɕthɯz} 
\classe{vi}  
\grammaire{refl} \paradigme{dir}{kɤ-}
\begin{définition}\pfra{dévoiler sa réelle identité}\end{définition}
\begin{définition}\pcmn{亮相}\end{définition}
\begin{exemple}\pjya{fso tɕe ju-ɣi-a tɕe ɣɯ-ku-ʑɣɤɕthɯz-a}\hspace{5pt}\pcmn{我明天来亲自亮相(给别人看我的真面目)}\end{exemple}\relationsémantique{参考}{\lien{ⓔɕthɯz}{ɕthɯz}}\end{entrée}

\begin{entrée}{ʑɣɤɕtʂat}{}{ⓔʑɣɤɕtʂat}\relationsémantique{参考}{\lien{ⓔɕtʂat}{ɕtʂat}}\end{entrée}

\begin{entrée}{ʑɣɤɕɯɣmu}{}{ⓔʑɣɤɕɯɣmu}\relationsémantique{参考}{\lien{ⓔɕɯɣmu}{ɕɯɣmu}}\end{entrée}

\begin{entrée}{ʑɣɤɕɯmbɣom}{}{ⓔʑɣɤɕɯmbɣom}\relationsémantique{参考}{\lien{ⓔɕɯmbɣom}{ɕɯmbɣom}}\end{entrée}

\begin{entrée}{ʑɣɤɕɯnŋo}{}{ⓔʑɣɤɕɯnŋo}\relationsémantique{参考}{\lien{ⓔɕɯnŋo}{ɕɯnŋo}}\end{entrée}

\begin{entrée}{ʑɣɤɕɯrga}{}{ⓔʑɣɤɕɯrga}\relationsémantique{参考}{\lien{ⓔɕɯrga}{ɕɯrga}}\end{entrée}

\begin{entrée}{ʑɣɤfɕɤt}{}{ⓔʑɣɤfɕɤt}\relationsémantique{参考}{\lien{ⓔfɕɤtⓗ1}{fɕɤt₁}}\end{entrée}

\begin{entrée}{ʑɣɤfsraŋ}{}{ⓔʑɣɤfsraŋ}\relationsémantique{参考}{\lien{ⓔfsraŋ}{fsraŋ}}\end{entrée}

\begin{entrée}{ʑɣɤfstɯn}{}{ⓔʑɣɤfstɯn}\relationsémantique{参考}{\lien{ⓔfstɯn}{fstɯn}}\end{entrée}

\begin{entrée}{ʑɣɤɣɤβdi}{}{ⓔʑɣɤɣɤβdi}\relationsémantique{参考}{\lien{ⓔɣɤβdi}{ɣɤβdi}}\end{entrée}

\begin{entrée}{ʑɣɤɣɤkhe}{}{ⓔʑɣɤɣɤkhe}\relationsémantique{参考}{\lien{ⓔkhe}{khe}}\end{entrée}

\begin{entrée}{ʑɣɤɣɤla}{}{ⓔʑɣɤɣɤla}\relationsémantique{参考}{\lien{ⓔɣɤla}{ɣɤla}}\end{entrée}

\begin{entrée}{ʑɣɤɣɤme}{}{ⓔʑɣɤɣɤme}\relationsémantique{参考}{\lien{ⓔɣɤme}{ɣɤme}}\end{entrée}

\begin{entrée}{ʑɣɤɣɤntaβ}{}{ⓔʑɣɤɣɤntaβ}\relationsémantique{参考}{\lien{ⓔɣɤntaβ}{ɣɤntaβ}}\end{entrée}

\begin{entrée}{ʑɣɤɣɤŋgi}{}{ⓔʑɣɤɣɤŋgi}\relationsémantique{参考}{\lien{ⓔɣɤŋgi}{ɣɤŋgi}}\end{entrée}

\begin{entrée}{ʑɣɤɣɤrndi}{}{ⓔʑɣɤɣɤrndi}\relationsémantique{参考}{\lien{ⓔɣɤrndi}{ɣɤrndi}}\end{entrée}

\begin{entrée}{ʑɣɤɣɤtɕa}{}{ⓔʑɣɤɣɤtɕa}\relationsémantique{参考}{\lien{ⓔɣɤtɕa}{ɣɤtɕa}}\end{entrée}

\begin{entrée}{ʑɣɤɣɤtɕɤt}{}{ⓔʑɣɤɣɤtɕɤt}\relationsémantique{参考}{\lien{ⓔɣɤtɕɤt}{ɣɤtɕɤt}}\end{entrée}

\begin{entrée}{ʑɣɤɣɤʑo}{}{ⓔʑɣɤɣɤʑo}\relationsémantique{参考}{\lien{ⓔʑoⓗ1}{ʑo₁}}\end{entrée}

\begin{entrée}{ʑɣɤkho}{}{ⓔʑɣɤkho}\relationsémantique{参考}{\lien{ⓔkhoⓗ1}{kho₁}}\end{entrée}

\begin{entrée}{ʑɣɤkro}{}{ⓔʑɣɤkro}\relationsémantique{参考}{\lien{ⓔkro}{kro}}\end{entrée}

\begin{entrée}{ʑɣɤlɤrko/\variante{zɣɤnɤrko}}{}{ⓔʑɣɤlɤrko} 
\classe{vi}  
\grammaire{refl} \paradigme{dir}{tɤ-}
\begin{définition}\pfra{s'encourager soi-même, garder confiance}\end{définition}
\begin{définition}\pcmn{自己鼓励自己}\end{définition}
\begin{exemple}\pjya{nɤʑo tɤ-ʑɣɤlɤrko, tɕe ʑa a-tɤ-tɯ-mna}\hspace{5pt}\pcmn{你要坚持,就会早点痊愈}\end{exemple}\relationsémantique{参考}{\lien{ⓔnɤrkoⓗ1}{nɤrko₁}}\end{entrée}

\begin{entrée}{ʑɣɤmaŋlo}{}{ⓔʑɣɤmaŋlo}\relationsémantique{参考}{\lien{ⓔmaŋlo}{maŋlo}}\end{entrée}

\begin{entrée}{ʑɣɤmar}{}{ⓔʑɣɤmar}\relationsémantique{参考}{\lien{ⓔmar}{mar}}\end{entrée}

\begin{entrée}{ʑɣɤmdzoz}{}{ⓔʑɣɤmdzoz} 
\classe{vi} 
\begin{définition}\pfra{avoir de la dignité, ne pas agir de façon inconsidérée}\end{définition}
\begin{définition}\pcmn{自重,注意自己的言行}\end{définition}
\begin{exemple}\pjya{tɯrme sɤtɕha jɤ-kɯ-ɤri tɕe, tɯ-kɯ-ʑɣɤmdzoz ra}\hspace{5pt}\pcmn{出远门的时候要会自重}\end{exemple}\relationsémantique{参考}{\lien{ⓔmdzoz}{mdzoz}}\end{entrée}

\begin{entrée}{ʑɣɤmgɯ}{}{ⓔʑɣɤmgɯ}\relationsémantique{参考}{\lien{ⓔmgɯ}{mgɯ}}\end{entrée}

\begin{entrée}{ʑɣɤmɲo}{}{ⓔʑɣɤmɲo}\relationsémantique{参考}{\lien{ⓔmɲoⓗ1}{mɲo₁}}\end{entrée}

\begin{entrée}{ʑɣɤmphɯr}{}{ⓔʑɣɤmphɯr}\relationsémantique{参考}{\lien{ⓔmphɯr}{mphɯr}}\end{entrée}

\begin{entrée}{ʑɣɤmto}{}{ⓔʑɣɤmto}\relationsémantique{参考}{\lien{ⓔmtoⓝmto}{mto}}\end{entrée}

\begin{entrée}{ʑɣɤnaχsoz}{}{ⓔʑɣɤnaχsoz}\relationsémantique{参考}{\lien{ⓔnaχsoz}{naχsoz}}\end{entrée}

\begin{entrée}{ʑɣɤnɤmpɕɤr}{}{ⓔʑɣɤnɤmpɕɤr}\relationsémantique{参考}{\lien{ⓔnɤmpɕɤr}{nɤmpɕɤr}}\end{entrée}

\begin{entrée}{ʑɣɤnɤmqe}{}{ⓔʑɣɤnɤmqe}\relationsémantique{参考}{\lien{ⓔnɤmqe}{nɤmqe}}\end{entrée}

\begin{entrée}{ʑɣɤnɤmtshɤr}{}{ⓔʑɣɤnɤmtshɤr}\relationsémantique{参考}{\lien{ⓔnɤmtshɤr}{nɤmtshɤr}}\end{entrée}

\begin{entrée}{ʑɣɤnɤndʐo}{}{ⓔʑɣɤnɤndʐo}\relationsémantique{参考}{\lien{ⓔnɤndʐo}{nɤndʐo}}\end{entrée}

\begin{entrée}{ʑɣɤnɤrko}{}{ⓔʑɣɤnɤrko}\relationsémantique{参考}{\lien{ⓔnɤrkoⓗ1}{nɤrko₁}}\end{entrée}

\begin{entrée}{ʑɣɤnɤstu}{}{ⓔʑɣɤnɤstu}\relationsémantique{参考}{\lien{ⓔnɤstu}{nɤstu}}\end{entrée}

\begin{entrée}{ʑɣɤnɤtsɯ}{}{ⓔʑɣɤnɤtsɯ}\relationsémantique{参考}{\lien{ⓔnɤtsɯ}{nɤtsɯ}}\end{entrée}

\begin{entrée}{ʑɣɤnbaʁ}{}{ⓔʑɣɤnbaʁ}\relationsémantique{参考}{\lien{}{nbaʁ}}\end{entrée}

\begin{entrée}{ʑɣɤndzɯ}{}{ⓔʑɣɤndzɯ}\relationsémantique{参考}{\lien{ⓔndzɯ}{ndzɯ}}\end{entrée}

\begin{entrée}{ʑɣɤntsɣe}{}{ⓔʑɣɤntsɣe}\relationsémantique{参考}{\lien{ⓔntsɣe}{ntsɣe}}\end{entrée}

\begin{entrée}{ʑɣɤnɯβlu}{}{ⓔʑɣɤnɯβlu}\relationsémantique{参考}{\lien{ⓔnɯβlu}{nɯβlu}}\end{entrée}

\begin{entrée}{ʑɣɤnɯkhramba}{}{ⓔʑɣɤnɯkhramba}\relationsémantique{参考}{\lien{ⓔnɯkhramba}{nɯkhramba}}\end{entrée}

\begin{entrée}{ʑɣɤnɯkon}{}{ⓔʑɣɤnɯkon}\relationsémantique{参考}{\lien{ⓔnɯkon}{nɯkon}}\end{entrée}

\begin{entrée}{ʑɣɤnɯmbrɤpɯ}{}{ⓔʑɣɤnɯmbrɤpɯ}\relationsémantique{参考}{\lien{ⓔnɯmbrɤpɯ}{nɯmbrɤpɯ}}\end{entrée}

\begin{entrée}{ʑɣɤnɯmpa}{}{ⓔʑɣɤnɯmpa}\relationsémantique{参考}{\lien{ⓔnɯmpa}{nɯmpa}}\end{entrée}

\begin{entrée}{ʑɣɤnɯsmɤn}{}{ⓔʑɣɤnɯsmɤn}\relationsémantique{参考}{\lien{ⓔnɯsmɤn}{nɯsmɤn}}\end{entrée}

\begin{entrée}{ʑɣɤnɯtʂawku}{}{ⓔʑɣɤnɯtʂawku}\relationsémantique{参考}{\lien{ⓔnɯtʂawku}{nɯtʂawku}}\end{entrée}

\begin{entrée}{ʑɣɤpa}{₁}{ⓔʑɣɤpaⓗ1} 
\classe{vi}  
\grammaire{refl} \paradigme{dir}{tɤ-}\sens{1}\paradigme{construction}{participe sujet}
\begin{définition}\pfra{faire semblant}\end{définition}
\begin{définition}\pcmn{装做}\end{définition}
\begin{exemple}\pjya{pɯ-kɯ-maʁ ɲɯ-ʑɣɤpa}\hspace{5pt}\pcmn{他装作不是他}\end{exemple}\relationsémantique{同义词}{\lien{ⓔnɯɕpɯz}{nɯɕpɯz}}\sens{2}
\begin{définition}\pfra{être orgueilleux}\end{définition}
\begin{définition}\pcmn{傲慢}\end{définition}
\begin{exemple}\pjya{jiɕqha nɯ kɯ-ʑɣɤpa ci ɲɯ-ŋu}\hspace{5pt}\pcmn{他是个傲慢的人}\end{exemple}\relationsémantique{同义词}{\lien{ⓔznɤjpɯjpe}{znɤjpɯjpe}}\sens{3}
\begin{définition}\pfra{s'appeler soi-même}\end{définition}
\begin{définition}\pcmn{自称}\end{définition}
\begin{exemple}\pjya{ʁnɯ-xpa pjɤ-wxti qhe tɤ-pi to-ʑɣɤpa}\hspace{5pt}\pcmn{因为他大两岁,所以自称哥哥}\end{exemple}\relationsémantique{同义词}{\lien{ⓔrtsiⓝʑɣɤrtsi}{ʑɣɤrtsi}}\end{entrée}

\begin{entrée}{ʑɣɤpa}{₂}{ⓔʑɣɤpaⓗ2}\relationsémantique{参考}{\lien{ⓔpaⓗ1}{pa₁}}\end{entrée}

\begin{entrée}{ʑɣɤpɣaʁ}{}{ⓔʑɣɤpɣaʁ}\relationsémantique{参考}{\lien{ⓔpɣaʁ}{pɣaʁ}}\end{entrée}

\begin{entrée}{ʑɣɤqɤr}{}{ⓔʑɣɤqɤr} 
\classe{vi} \paradigme{dir}{nɯ-}
\begin{définition}\pfra{s'isoler}\end{définition}
\begin{définition}\pcmn{独居;跟其他人分开}\end{définition}\relationsémantique{参考}{\lien{ⓔqɤr}{qɤr}}\end{entrée}

\begin{entrée}{ʑɣɤraχtɕɤz}{}{ⓔʑɣɤraχtɕɤz}\relationsémantique{参考}{\lien{ⓔraχtɕɤz}{raχtɕɤz}}\end{entrée}

\begin{entrée}{ʑɣɤrɤβraʁ}{}{ⓔʑɣɤrɤβraʁ}\relationsémantique{参考}{\lien{ⓔrɤβraʁ}{rɤβraʁ}}\end{entrée}

\begin{entrée}{ʑɣɤrɤɕi}{}{ⓔʑɣɤrɤɕi}\relationsémantique{参考}{\lien{ⓔrɤɕi}{rɤɕi}}\end{entrée}

\begin{entrée}{ʑɣɤrɤphɯ}{}{ⓔʑɣɤrɤphɯ} 
\classe{vi} \paradigme{dir}{pɯ-}\paradigme{dir}{kɯ-}
\begin{définition}\pfra{mesurer ses forces}\end{définition}
\begin{définition}\pcmn{量力而行}\end{définition}
\begin{exemple}\pjya{tɯʑo pjɯ-kɯ-ʑɣɤrɤphɯ ra}\hspace{5pt}\pcmn{要量力而行}\end{exemple}
\begin{exemple}\pjya{nɤ-mtɕhi ma-tɯ-ɣɤχɤm kɯ koŋla kɤ-ʑɣɤrɤphɯ}\hspace{5pt}\pcmn{不要说大话,要量力而行}\end{exemple}\end{entrée}

\begin{entrée}{ʑɣɤrku}{}{ⓔʑɣɤrku}\relationsémantique{参考}{\lien{ⓔrku}{rku}}\end{entrée}

\begin{entrée}{ʑɣɤrpu}{}{ⓔʑɣɤrpu}\relationsémantique{参考}{\lien{ⓔrpu}{rpu}}\end{entrée}

\begin{entrée}{ʑɣɤrtsi}{}{ⓔʑɣɤrtsi}\relationsémantique{参考}{\lien{ⓔrtsi}{rtsi}}\end{entrée}

\begin{entrée}{ʑɣɤrɯxtuxti}{}{ⓔʑɣɤrɯxtuxti}\relationsémantique{参考}{\lien{ⓔrɯxtuxti}{rɯxtuxti}}\end{entrée}

\begin{entrée}{ʑɣɤrʑɣɤr}{}{ⓔʑɣɤrʑɣɤr} 
\classe{idph.2} 
\begin{définition}\pfra{avoir des brins qui dépassent (dans une touffe)}\end{définition}
\begin{définition}\pcmn{形容一束东西当中,有一两根凸出来,显得不整齐的样子}\end{définition}
\begin{exemple}\pjya{tu-ro ʑɣɤrʑɣɤr ʑo ɲɯ-ŋu}\hspace{5pt}\pcmn{有(一两根)凸出来}\end{exemple}\relationsémantique{参考}{\lien{ⓔʑɣɤʑɣɤt}{ʑɣɤʑɣɤt}}\end{entrée}

\begin{entrée}{ʑɣɤʁmɯɣ}{}{ⓔʑɣɤʁmɯɣ}\relationsémantique{参考}{\lien{ⓔʁmɯɣ}{ʁmɯɣ}}\end{entrée}

\begin{entrée}{ʑɣɤsaʁjɤr}{}{ⓔʑɣɤsaʁjɤr}\relationsémantique{参考}{\lien{ⓔaʁjɤrⓝsaʁjɤr}{saʁjɤr}}\end{entrée}

\begin{entrée}{ʑɣɤsɤɕke}{}{ⓔʑɣɤsɤɕke} 
\classe{vi}  
\grammaire{refl} 
\begin{définition}\pfra{se brûler}\end{définition}
\begin{définition}\pcmn{烫到自己}\end{définition}
\begin{exemple}\pjya{ma-pɯ-tɯ-ʑɣɤsɤɕke}\hspace{5pt}\pcmn{你不要烫到自己}\end{exemple}\relationsémantique{参考}{\lien{ⓔɕke}{ɕke}}\relationsémantique{参考}{\lien{}{sɤɕke}}\end{entrée}

\begin{entrée}{ʑɣɤsɤfɕu}{}{ⓔʑɣɤsɤfɕu}\relationsémantique{参考}{\lien{ⓔafɕu}{afɕu}}\end{entrée}

\begin{entrée}{ʑɣɤsɤjɤr}{}{ⓔʑɣɤsɤjɤr}\relationsémantique{参考}{\lien{ⓔajɤr}{ajɤr}}\end{entrée}

\begin{entrée}{ʑɣɤsɤnbaʁ}{}{ⓔʑɣɤsɤnbaʁ}\relationsémantique{参考}{\lien{ⓔanbaʁ}{anbaʁ}}\end{entrée}

\begin{entrée}{ʑɣɤsɤɲɟoʁ}{}{ⓔʑɣɤsɤɲɟoʁ}\relationsémantique{参考}{\lien{ⓔɲɟoʁ}{ɲɟoʁ}}\end{entrée}

\begin{entrée}{ʑɣɤsɤpɣaʁsci}{}{ⓔʑɣɤsɤpɣaʁsci}\relationsémantique{参考}{\lien{ⓔapɣaʁsci}{apɣaʁsci}}\end{entrée}

\begin{entrée}{ʑɣɤsɤri}{}{ⓔʑɣɤsɤri}\relationsémantique{参考}{\lien{ⓔsɤri}{sɤri}}\end{entrée}

\begin{entrée}{ʑɣɤsɤrmbat}{}{ⓔʑɣɤsɤrmbat}\relationsémantique{参考}{\lien{ⓔarmbat}{armbat}}\end{entrée}

\begin{entrée}{ʑɣɤsɤrmi}{}{ⓔʑɣɤsɤrmi}\relationsémantique{参考}{\lien{ⓔsɤrmi}{sɤrmi}}\end{entrée}

\begin{entrée}{ʑɣɤsɤrqhi}{}{ⓔʑɣɤsɤrqhi}\relationsémantique{参考}{\lien{ⓔarqhi}{arqhi}}\end{entrée}

\begin{entrée}{ʑɣɤsɤrʁɯrʁu}{}{ⓔʑɣɤsɤrʁɯrʁu}\relationsémantique{参考}{\lien{ⓔarʁɯrʁu}{arʁɯrʁu}}\end{entrée}

\begin{entrée}{ʑɣɤsɤstɤko}{}{ⓔʑɣɤsɤstɤko}\relationsémantique{参考}{\lien{ⓔsɤstɤko}{sɤstɤko}}\end{entrée}

\begin{entrée}{ʑɣɤsɤtsa}{}{ⓔʑɣɤsɤtsa}\relationsémantique{参考}{\lien{ⓔatsa}{atsa}}\end{entrée}

\begin{entrée}{ʑɣɤsɤtɯɣ}{}{ⓔʑɣɤsɤtɯɣ}\relationsémantique{参考}{\lien{ⓔatɯɣ}{atɯɣ}}\end{entrée}

\begin{entrée}{ʑɣɤsɤzɣɯt}{}{ⓔʑɣɤsɤzɣɯt}\relationsémantique{参考}{\lien{ⓔsɤzɣɯt}{sɤzɣɯt}}\end{entrée}

\begin{entrée}{ʑɣɤsprɯl}{}{ⓔʑɣɤsprɯl} 
\classe{vi}  
\grammaire{refl} \paradigme{dir}{nɯ-}
\begin{définition}\pfra{se déguiser, se transformer}\end{définition}
\begin{définition}\pcmn{装扮成,转变成}\end{définition}
\begin{exemple}\pjya{ɬɤndʐi nɯ tɯrme ɲɯ-ʑɣɤsprɯl ŋgrɤl}\hspace{5pt}\pcmn{鬼会把自己变成人}\end{exemple}\étymologie{sprul}\end{entrée}

\begin{entrée}{ʑɣɤstu}{}{ⓔʑɣɤstu} 
\classe{vi}  
\grammaire{refl} \paradigme{dir}{tɤ-}
\begin{définition}\pfra{faire en sorte de devenir ainsi}\end{définition}
\begin{définition}\pcmn{使自己变成那样}\end{définition}
\begin{exemple}\pjya{tɕhɣaʁtɕhɣaʁ ʑo tɤ-ʑɣɤstu}\end{exemple}
\begin{exemple}\pjya{phoʁphoʁ to-ʑɣɤstu}\hspace{5pt}\pcmn{他使自己变得很干净}\end{exemple}\relationsémantique{参考}{\lien{}{stu}}\end{entrée}

\begin{entrée}{ʑɣɤsɯβde}{}{ⓔʑɣɤsɯβde} 
\classe{vi}  
\grammaire{refl}
\grammaire{caus} \paradigme{dir}{pɯ-}
\begin{définition}\pfra{se faire jeter}\end{définition}
\begin{définition}\pcmn{被摔下来}\end{définition}
\begin{exemple}\pjya{mbro to-nɯmbrɤpɯ tɕe pjɤ-ʑɣɤsɯβde}\hspace{5pt}\pcmn{他骑了马,被摔下来了}\end{exemple}\relationsémantique{参考}{\lien{ⓔβde}{βde}}\end{entrée}

\begin{entrée}{ʑɣɤsɯβʁa}{}{ⓔʑɣɤsɯβʁa}\relationsémantique{参考}{\lien{ⓔβʁa}{βʁa}}\end{entrée}

\begin{entrée}{ʑɣɤsɯβzi}{}{ⓔʑɣɤsɯβzi}\relationsémantique{参考}{\lien{ⓔβzi}{βzi}}\end{entrée}

\begin{entrée}{ʑɣɤsɯɕqhlɤt}{}{ⓔʑɣɤsɯɕqhlɤt}\relationsémantique{参考}{\lien{ⓔɕqhlɤt}{ɕqhlɤt}}\end{entrée}

\begin{entrée}{ʑɣɤsɯɕqraʁ}{}{ⓔʑɣɤsɯɕqraʁ}\relationsémantique{参考}{\lien{ⓔɕqraʁ}{ɕqraʁ}}\end{entrée}

\begin{entrée}{ʑɣɤsɯfsaŋ}{}{ⓔʑɣɤsɯfsaŋ}\relationsémantique{参考}{\lien{ⓔsɯfsaŋ}{sɯfsaŋ}}\end{entrée}

\begin{entrée}{ʑɣɤsɯɣlɯɣ}{}{ⓔʑɣɤsɯɣlɯɣ}\relationsémantique{参考}{\lien{ⓔlɯɣ}{lɯɣ}}\end{entrée}

\begin{entrée}{ʑɣɤsɯɣɲaʁ}{}{ⓔʑɣɤsɯɣɲaʁ}\relationsémantique{参考}{\lien{ⓔsɯɣɲaʁ}{sɯɣɲaʁ}}\end{entrée}

\begin{entrée}{ʑɣɤsɯɣʑi}{}{ⓔʑɣɤsɯɣʑi}\relationsémantique{参考}{\lien{ⓔʑi}{ʑi}}\end{entrée}

\begin{entrée}{ʑɣɤsɯmphɯr}{}{ⓔʑɣɤsɯmphɯr}\relationsémantique{参考}{\lien{ⓔmphɯr}{mphɯr}}\end{entrée}

\begin{entrée}{ʑɣɤsɯmto}{}{ⓔʑɣɤsɯmto}\relationsémantique{参考}{\lien{ⓔmtoⓝmto}{mto}}\end{entrée}

\begin{entrée}{ʑɣɤsɯndo}{}{ⓔʑɣɤsɯndo}\relationsémantique{参考}{\lien{ⓔndo}{ndo}}\end{entrée}

\begin{entrée}{ʑɣɤsɯndo}{}{ⓔʑɣɤsɯndo}\relationsémantique{参考}{\lien{ⓔndo}{ndo}}\end{entrée}

\begin{entrée}{ʑɣɤsɯntshɤβ}{}{ⓔʑɣɤsɯntshɤβ}\relationsémantique{参考}{\lien{ⓔntshɤβ}{ntshɤβ}}\end{entrée}

\begin{entrée}{ʑɣɤsɯrku}{}{ⓔʑɣɤsɯrku}\relationsémantique{参考}{\lien{ⓔrku}{rku}}\end{entrée}

\begin{entrée}{ʑɣɤsɯrtoʁ}{}{ⓔʑɣɤsɯrtoʁ}\relationsémantique{参考}{\lien{ⓔrtoʁ}{rtoʁ}}\end{entrée}

\begin{entrée}{ʑɣɤsɯsat}{}{ⓔʑɣɤsɯsat}\relationsémantique{参考}{\lien{ⓔsat}{sat}}\end{entrée}

\begin{entrée}{ʑɣɤsɯsŋaʁ}{}{ⓔʑɣɤsɯsŋaʁ}\relationsémantique{参考}{\lien{ⓔsŋaʁⓗ1}{sŋaʁ₁}}\end{entrée}

\begin{entrée}{ʑɣɤsɯxɕɤt}{}{ⓔʑɣɤsɯxɕɤt}\relationsémantique{参考}{\lien{ⓔsɯxɕɤt}{sɯxɕɤt}}\end{entrée}

\begin{entrée}{ʑɣɤsɯxtshu}{}{ⓔʑɣɤsɯxtshu}\relationsémantique{参考}{\lien{ⓔtshu}{tshu}}\end{entrée}

\begin{entrée}{ʑɣɤsɯxtso}{}{ⓔʑɣɤsɯxtso}\relationsémantique{参考}{\lien{ⓔtso}{tso}}\end{entrée}

\begin{entrée}{ʑɣɤsɯxtɯɣ}{}{ⓔʑɣɤsɯxtɯɣ} 
\classe{vi}  
\grammaire{refl} \paradigme{dir}{\_}
\begin{définition}\pfra{se rapprocher et entrer en contact avec}\end{définition}
\begin{définition}\pcmn{靠拢}\end{définition}
\begin{exemple}\pjya{ma-thɯ-tɯ-ʑɣɤsɯxtɯɣ ma nɤ-ŋga sɯ-pɣi}\hspace{5pt}\pcmn{你不要靠拢,会把你衣服弄脏}\end{exemple}\relationsémantique{同义词}{\lien{ⓔʑɣɤχtɤt}{ʑɣɤχtɤt}}\end{entrée}

\begin{entrée}{ʑɣɤsɯχsu}{}{ⓔʑɣɤsɯχsu}\relationsémantique{参考}{\lien{ⓔχsu}{χsu}}\end{entrée}

\begin{entrée}{ʑɣɤsɯzdɯɣ}{}{ⓔʑɣɤsɯzdɯɣ}\relationsémantique{参考}{\lien{ⓔsɯzdɯɣ}{sɯzdɯɣ}}\end{entrée}

\begin{entrée}{ʑɣɤta}{}{ⓔʑɣɤta} 
\classe{vi}  
\grammaire{refl} \sens{1}\paradigme{dir}{\_}
\begin{définition}\pfra{s’adosser à, s'appuyer}\end{définition}
\begin{définition}\pcmn{靠}\end{définition}
\begin{exemple}\pjya{ɯ-zda ɯ-taʁ pjɤ-ʑɣɤta}\hspace{5pt}\pcmn{他躺在别人的身上了}\end{exemple}
\begin{exemple}\pjya{aʑo ɲɤ-ɲat-a tɕe, ɯ-taʁ kɤ-ʑɣɤta-a}\hspace{5pt}\pcmn{我很累,所以靠在他身上}\end{exemple}\sens{2}\paradigme{dir}{nɯ-}
\begin{définition}\pfra{rester et ne pas vouloir partir}\end{définition}
\begin{définition}\pcmn{留在别人家里不肯走}\end{définition}
\begin{exemple}\pjya{tɯrme ɯ-kha ɲɤ-ʑɣɤta}\hspace{5pt}\pcmn{他留在人家的屋子里不肯走}\end{exemple}
\begin{exemple}\pjya{ma-nɯ-tɯ-ʑɣɤta kɯ nɯɕe-tɕi}\hspace{5pt}\pcmn{你不要待在这里,我们走吧}\end{exemple}\relationsémantique{同义词}{\lien{}{zɣɤχtɤt}}\relationsémantique{参考}{\lien{ⓔta}{ta}}\end{entrée}

\begin{entrée}{ʑɣɤtshi}{}{ⓔʑɣɤtshi} 
\classe{vi}  
\grammaire{refl} \paradigme{dir}{tɤ-}
\begin{définition}\pfra{se pendre}\end{définition}
\begin{définition}\pcmn{上吊}\end{définition}
\begin{exemple}\pjya{to-ʑɣɤtshi}\hspace{5pt}\pcmn{他上吊自尽了}\end{exemple}\relationsémantique{参考}{\lien{ⓔtshiⓗ2}{tshi₂}}\end{entrée}

\begin{entrée}{ʑɣɤtʂaβ}{}{ⓔʑɣɤtʂaβ}\relationsémantique{参考}{\lien{ⓔtʂaβ}{tʂaβ}}\end{entrée}

\begin{entrée}{ʑɣɤwum}{}{ⓔʑɣɤwum}\relationsémantique{参考}{\lien{ⓔwum}{wum}}\end{entrée}

\begin{entrée}{ʑɣɤxthom}{}{ⓔʑɣɤxthom}\relationsémantique{参考}{\lien{ⓔxthom}{xthom}}\end{entrée}

\begin{entrée}{ʑɣɤχpjɤt}{}{ⓔʑɣɤχpjɤt}\relationsémantique{参考}{\lien{ⓔχpjɤt}{χpjɤt}}\end{entrée}

\begin{entrée}{ʑɣɤχtɤt}{}{ⓔʑɣɤχtɤt}\relationsémantique{参考}{\lien{ⓔχtɤt}{χtɤt}}\end{entrée}

\begin{entrée}{ʑɣɤχtɕi}{}{ⓔʑɣɤχtɕi}\relationsémantique{参考}{\lien{ⓔχtɕi}{χtɕi}}\end{entrée}

\begin{entrée}{ʑɣɤznɯɲɤmkhe}{}{ⓔʑɣɤznɯɲɤmkhe}\relationsémantique{参考}{\lien{ⓔnɯɲɤmkhe}{nɯɲɤmkhe}}\end{entrée}

\begin{entrée}{ʑɣɤʑɣɤt}{}{ⓔʑɣɤʑɣɤt} 
\classe{idph.2} 
\begin{définition}\pfra{avoir des brins qui dépassent (dans une touffe)}\end{définition}
\begin{définition}\pcmn{形容一束东西当中,有一两根凸出来,显得不整齐的样子}\end{définition}
\begin{exemple}\pjya{kuxtɕo ɯ-taʁ si nɯ ɲɯ-ro ʑɣɤʑɣɤt ʑo}\hspace{5pt}\pcmn{背篼里装的柴有(一根两根)凸出来}\end{exemple}\relationsémantique{参考}{\lien{ⓔʑɣɤrʑɣɤr}{ʑɣɤrʑɣɤr}}\end{entrée}

\begin{entrée}{ʑi}{}{ⓔʑi} 
\classe{vi} \paradigme{dir}{nɯ-}\paradigme{dir}{pɯ-}\paradigme{dir}{thɯ-}
\begin{définition}\pfra{se résorber, se calmer}\end{définition}
\begin{définition}\pcmn{平静下来;减轻;消肿}\end{définition}
\begin{exemple}\pjya{ɯ-kɯ-mŋɤm ɲɤ-ʑi}\hspace{5pt}\pcmn{他的痛减轻了}\end{exemple}
\begin{exemple}\pjya{ɯ-mbrɯ ɲɤ-ʑi}\hspace{5pt}\pcmn{他的气消了}\end{exemple}
\begin{exemple}\pjya{a-mtshi kɯ-mŋɤm daldaltsɯtsa ʑo nɯ-ʑi}\hspace{5pt}\pcmn{我的胃疼(肝)慢慢地减轻了}\end{exemple}
\begin{exemple}\pjya{tʂha ku-tshi-a qhe ɲɯ-ʑi ɕti}\hspace{5pt}\pcmn{喝了茶就会好一点(解渴,或者不再打瞌睡了)}\end{exemple}
\begin{exemple}\pjya{tɯ-mɯ ɲɤ-ʑi}\hspace{5pt}\pcmn{雨停了}\end{exemple}
\begin{exemple}\pjya{qale ɲɤ-ʑi}\hspace{5pt}\pcmn{风停了}\end{exemple}
\begin{exemple}\pjya{ɯ-tɯ-ɣmbɤβ pjɤ-ʑi}\hspace{5pt}\pcmn{他的脓肿消肿了}\end{exemple}
\begin{définition}\pfra{calmer}\end{définition}
\begin{définition}\pcmn{令…消肿、令…平静}\end{définition}
\begin{exemple}\pjya{ɯ-mbrɯ ra pjɤ-sɯɣʑi}\hspace{5pt}\pcmn{她平息了怒气}\end{exemple}
\begin{sous-entrée}{ʑɣɤsɯɣʑi}{ⓔʑiⓝʑɣɤsɯɣʑi} 
\classe{vi}  
\grammaire{refl}
\grammaire{caus} 
\begin{définition}\pfra{se calmer}\end{définition}
\begin{définition}\pcmn{平静下来}\end{définition}\end{sous-entrée}

\begin{sous-entrée}{sɯɣʑi}{ⓔʑiⓝsɯɣʑi} 
\classe{vt} \paradigme{dir}{pɯ-}\end{sous-entrée}

\étymologie{ʑi}\end{entrée}

\begin{entrée}{ʑiwarɯmtɕhɤt}{}{ⓔʑiwarɯmtɕhɤt} 
\classe{n} 
\begin{définition}\pfra{une célébration bouddhique}\end{définition}
\begin{définition}\pcmn{一种法事}\end{définition}\étymologie{ʑi.ba ri.mtɕʰod}\end{entrée}

\begin{entrée}{ʑmbɤr}{}{ⓔʑmbɤr} 
\classe{n} 
\begin{définition}\pfra{ulcère}\end{définition}
\begin{définition}\pcmn{疮}\end{définition}
\begin{exemple}\pjya{a-rŋa ʑmbɤr ɲɤ-ɬoʁ}\hspace{5pt}\pcmn{我脸上生了疮}\end{exemple}\end{entrée}

\begin{entrée}{ʑmbraʁlaʁli}{}{ⓔʑmbraʁlaʁli} 
\classe{n} 
\begin{définition}\pfra{une plante}\end{définition}
\begin{définition}\pcmn{植物的一种}\end{définition}
\begin{exemple}\pjya{ʑmbraʁlaʁli nɯ sɯjno ci ŋu, ɯ-ru kɯ-mpɯ tsa ŋu, kɯ-ɤlɯlju ŋu, ɯ-jwaʁ ʁnɯz ma ku-tshoʁ mɤ-cha. ɯ-jwaʁ ni ndʑi-pɤrthɤβ ri ɯ-mɯntoʁ ɲɯ-βze tɕe ɯ-jwaʁ nɯ nɯ-ɴɢɤt tɕe, ɯ-mɯntoʁ pjɯ-ŋgra tɕe ɯ-mat nɯ ɯ-jwaʁ ɯ-pa pjɤ-ɴqoʁ ŋu. ɯ-mat thɯ-tɯt tɕe ɣɯrni. ɯ-ŋgɯ ɯ-ci cho ɯ-rdoʁ ra kɯnɤ ɣɯrni. mɤ-sɤndɤɣ. tɤ-pɤtso ra kɯ tu-ndza-nɯ ŋgrɤl. nɯ tú-wɣ-ndza tɕe ``tɯ-rqo ʑmbraʁ mɤ-ɕe" tu-ti-nɯ ŋgrɤl.}\hspace{5pt}\pcmn{\lien{ⓔʑmbraʁlaʁli}{ʑmbraʁlaʁli} 是一种植物,茎有点柔软,呈圆柱形。只长两片叶子。花夹在两片叶子中间,叶子展开了以后,花凋落结成果实吊在叶子下面。果实成熟后变红。没有毒性。小孩子们经常吃这个果实。据说吃了它“青稞的芒不会进入喉咙”。}\end{exemple}\end{entrée}

\begin{entrée}{ʑmbri}{₂}{ⓔʑmbriⓗ2} 
\classe{n} 
\begin{définition}\pfra{saule}\end{définition}
\begin{définition}\pcmn{柳树}\end{définition}\end{entrée}

\begin{entrée}{ʑmbri}{₁}{ⓔʑmbriⓗ1} 
\classe{vt} \paradigme{dir}{thɯ-}
\begin{définition}\pfra{faire du bruit, jouer d'un instrument de musique}\end{définition}
\begin{définition}\pcmn{发出声音;演奏音乐}\end{définition}
\begin{exemple}\pjya{ɟuli thɯ-ʑmbri-t-a}\hspace{5pt}\pcmn{我吹了竹笛}\end{exemple}
\begin{exemple}\pjya{zɯxtɕhɤl ta-ʑmbri}\hspace{5pt}\pcmn{他打钹了}\end{exemple}
\begin{exemple}\pjya{mkhɤrŋa ta-ʑmbri}\hspace{5pt}\pcmn{他敲锣了}\end{exemple}\relationsémantique{参考}{\lien{ⓔmbriⓗ1}{mbri₁}}\end{entrée}

\begin{entrée}{ʑmbrijmɤɣ}{}{ⓔʑmbrijmɤɣ} 
\classe{n} 
\begin{définition}\pfra{Hericium erinaceus}\end{définition}
\begin{définition}\pcmn{猴头菌【杨柳菌】}\end{définition}
\begin{exemple}\pjya{ʑmbrijmɤɣ nɯ ʑmbri kɯ-wxti ɯ-taʁ tu-ɬoʁ ɲɯ-ŋu, kɯmaʁ tɤjmɤɣ ra cho nɯ-tshɯɣa mɯ́j-naχtɕɯɣ, ɯʑo kɯ-ɤrtɯm rloŋrloŋ ɲɯ-ŋu, ɯ-βri nɯ tɤ-rme kɯ-fse ʁɟa ɲɯ-ŋu, kɤ-ndza ɲɯ-sna, ɯ-mdoʁ kɯ-ɤqarŋɯrŋe tsa ɲɯ-ŋu.}\hspace{5pt}\pcmn{杨柳菌长在较高大的柳树上,样子和其他蘑菇不同。呈圆球形,全身长满毛,可以吃。颜色是淡黄色。}\end{exemple}\end{entrée}

\begin{entrée}{ʑmbroko}{}{ⓔʑmbroko} 
\classe{n} 
\begin{définition}\pfra{Sonchus oleraceus}\end{définition}
\begin{définition}\pcmn{苦苣菜【空洞菜】}\end{définition}
\begin{exemple}\pjya{ʑmbroko nɯ tɯ-xpa tu-kɯ-ɬoʁ sɯjno ŋu, ɯ-ru ɯ-ŋgɯ nɯ kɯ-so ŋu, pjɯ́-wɣ-qlɯt tɕe mpɯ, tɕe ɯ-lu tu, fsapaʁ ndza wuma pe ɯ-mɯntoʁ kɯ-qarŋe tɕe kɤ-rɯlaba ŋu}\hspace{5pt}\pcmn{苦苣菜是一年生的植物,茎是空心的,撇断的时候是嫩的,有乳汁。是很好的饲草。花黄色,喇叭形。}\end{exemple}\end{entrée}

\begin{entrée}{ʑmbrɯ}{}{ⓔʑmbrɯ} 
\classe{n} 
\begin{définition}\pfra{bateau}\end{définition}
\begin{définition}\pcmn{船}\end{définition}
\begin{exemple}\pjya{ʑmbrɯ kɤ-lat-a}\hspace{5pt}\pcmn{我划了船}\end{exemple}\étymologie{gru}\end{entrée}

\begin{entrée}{ʑmbrɯβɟaj}{}{ⓔʑmbrɯβɟaj} 
\classe{n} 
\begin{définition}\pfra{rame}\end{définition}
\begin{définition}\pcmn{桨}\end{définition}\end{entrée}

\begin{entrée}{ʑmbrɯɟoʁ}{}{ⓔʑmbrɯɟoʁ} 
\classe{n} 
\begin{définition}\pfra{clayonnage}\end{définition}
\begin{définition}\pcmn{杨柳枝条}\end{définition}\end{entrée}

\begin{entrée}{ʑmbrɯkɤlu}{}{ⓔʑmbrɯkɤlu} 
\classe{n} 
\begin{définition}\pfra{type de saule}\end{définition}
\begin{définition}\pcmn{柳树的一种(看起来被据掉一样)}\end{définition}\end{entrée}

\begin{entrée}{ʑmbrɯpɣa}{}{ⓔʑmbrɯpɣa} 
\classe{n} 
\begin{définition}\pfra{espèce d'oiseau}\end{définition}
\begin{définition}\pcmn{一种鸟}\end{définition}
\begin{exemple}\pjya{ʑmbrɯpɣa nɯ pɣa tɤŋkhɯt staʁnɤ wxti, ɯ-mi kɯ-qarŋe ŋu, ɯ-mtsioʁ nɯ ɲaʁ rɲɟi tsa ɯ-jme nɯ ɯ-phoŋbu sɤz rɲɟi, ɯ-muj ɯ-mdoʁ wuma mpɕɤr, kɯ-ɲaʁ ra kɯnɤ nɤmbju, ɯ-xtɤpa ra ʁmɤrsɤr ɯ-mdoʁ tu, ɯ-rqo pa cho ɯ-ʁar χchoʁe ra kɯ-wɣrum tɯ-snaʁ ka tu. tɯ-ji ɯ-ŋgɯ ju-ɣi mɤ-ŋgrɤl, stɤmku cho sɯŋgɯ ra ku-rɤʑi ɕti.}\hspace{5pt}\pcmn{\lien{ⓔʑmbrɯpɣa}{ʑmbrɯpɣa}是一种鸟,略大于拳头。脚黄色,嘴黑色,尾巴比身子长。羽毛颜色很美,黑色有光泽。腹部是橙色的,脖子和翅膀上各有一块白点。这种鸟不会来到庄稼地里,只是在草地和森林里生活。}\end{exemple}\end{entrée}

\begin{entrée}{ʑmbɯlɯm}{}{ⓔʑmbɯlɯm} 
\classe{n} 
\begin{définition}\pfra{une espèce de champignon}\end{définition}
\begin{définition}\pcmn{【油辣枯】}\end{définition}
\begin{exemple}\pjya{ʑmbɯlɯm nɯ sɤjku cho mbraj ɯ-ŋgɯ tu-ɬoʁ ŋu, ɯ-qhu nɯ kɯ-ɤqarŋɯ-rŋe tɕe kɯ-nɤmbju tsa ŋu, ɯ-rʑɯɣ cho ɯ-ru nɯ kɯ-wɣrum ŋu, tú-wɣ-ndza tɕe kɯ-xtɕɯ-xtɕi qiaβ cho mɤrtsaβ}\hspace{5pt}\pcmn{油辣枯长在白桦树和红桦树林里,背面带有黄色,有点光泽,下面菌褶和干是白色的。吃的时候,有点苦,有点辣。}\end{exemple}\end{entrée}

\begin{entrée}{ʑŋgu}{}{ⓔʑŋgu} 
\classe{vi} \paradigme{dir}{kɤ-}
\begin{définition}\pfra{passer une rivière en bateau}\end{définition}
\begin{définition}\pcmn{坐船渡河}\end{définition}
\begin{exemple}\pjya{tɯ-ci ɯ-taʁ ko-ʑŋgu (=tɯ-ʑŋgu ko-lɤt)}\hspace{5pt}\pcmn{他坐船渡了河}\end{exemple}
\begin{exemple}\pjya{kɤ-ʑŋgu-a}\hspace{5pt}\pcmn{我渡了河}\end{exemple}
\begin{exemple}\pjya{kɯ-ʑŋgu}\hspace{5pt}\pcmn{船夫}\end{exemple}\end{entrée}

\begin{entrée}{ʑŋga}{}{ⓔʑŋga} 
\classe{vt} \sens{1}\paradigme{dir}{tɤ-}
\begin{définition}\pfra{aider qqn à s'habiller}\end{définition}
\begin{définition}\pcmn{帮别人穿衣服}\end{définition}
\begin{exemple}\pjya{ɯ-ŋga tɤ-ʑŋga-t-a}\hspace{5pt}\pcmn{我帮他穿衣服了}\end{exemple}\sens{2}\paradigme{dir}{pɯ-}
\begin{définition}\pfra{border le lit à qqn}\end{définition}
\begin{définition}\pcmn{帮别人盖被子}\end{définition}
\begin{exemple}\pjya{pɯ-ʑŋga-t-a}\hspace{5pt}\pcmn{我帮他盖了被子}\end{exemple}\relationsémantique{参考}{\lien{ⓔŋga}{ŋga}}\end{entrée}

\begin{entrée}{ʑŋgi}{}{ⓔʑŋgi} 
\classe{vi} \paradigme{dir}{\_}
\begin{définition}\pfra{porter du bois}\end{définition}
\begin{définition}\pcmn{背柴}\end{définition}
\begin{exemple}\pjya{ɕ-pɯ-ʑŋgi-a}\hspace{5pt}\pcmn{我去背柴了}\end{exemple}\end{entrée}

\begin{entrée}{ʑŋgri}{}{ⓔʑŋgri} 
\classe{n} 
\begin{définition}\pfra{étoile}\end{définition}
\begin{définition}\pcmn{星星}\end{définition}
\begin{exemple}\pjya{ʑŋgri kɯ-nɯqambɯmbjom}\hspace{5pt}\pcmn{流星}\end{exemple}\end{entrée}

\begin{entrée}{ʑɴɢu}{}{ⓔʑɴɢu} 
\classe{vt} \paradigme{dir}{pɯ-}\paradigme{dir}{thɯ-}
\begin{définition}\pfra{éplucher, décortiquer}\end{définition}
\begin{définition}\pcmn{削;掰开}\end{définition}
\begin{exemple}\pjya{jima pɯ-ʑɴɢu-t-a}\hspace{5pt}\pcmn{我剥了玉米}\end{exemple}
\begin{exemple}\pjya{stoʁ pɯ-ʑɴɢu-t-a}\hspace{5pt}\pcmn{我掰了胡豆}\end{exemple}
\begin{exemple}\pjya{ʑɴɢɯloʁ pɯ-ʑɴɢu-t-a}\hspace{5pt}\pcmn{我掰了核桃}\end{exemple}
\begin{exemple}\pjya{rɤjndoʁ pɯ-ʑɴɢu-t-a}\hspace{5pt}\pcmn{我剥了圆根}\end{exemple}
\begin{sous-entrée}{nɯɣɯʑɴɢu}{ⓔʑɴɢuⓝnɯɣɯʑɴɢu} 
\classe{vs}  
\grammaire{facil} 
\begin{définition}\pfra{facile à éplucher}\end{définition}
\begin{définition}\pcmn{容易削}\end{définition}\end{sous-entrée}

\end{entrée}

\begin{entrée}{ʑɴɢoʁ}{}{ⓔʑɴɢoʁ} 
\classe{vt} \paradigme{dir}{kɤ-}\paradigme{dir}{kɤ-}
\begin{définition}\pfra{accrocher}\end{définition}
\begin{définition}\pcmn{钩住}\end{définition}
\begin{définition}\pfra{accrocher avec}\end{définition}
\begin{définition}\pcmn{用……钩住}\end{définition}
\begin{exemple}\pjya{a-tɤ-ri kɤ-ʑɴɢoʁ}\hspace{5pt}\pcmn{你帮我牵线吧(用两只手的手指把线分开)}\end{exemple}
\begin{exemple}\pjya{tɤ-jŋoʁ kɯ tɯ-ŋga ko-sɯʑɴɢoʁ}\hspace{5pt}\pcmn{他用钩子把衣服钩起来了}\end{exemple}
\begin{exemple}\pjya{tɯ-rju nɯ kɯ-kɯ-mɯ-maʁ ku-tɯ-sɯʑɴɢoʁ}\hspace{5pt}\pcmn{你把话题东拉西扯}\end{exemple}\relationsémantique{参考}{\lien{ⓔɴqoʁ}{ɴqoʁ}}
\begin{sous-entrée}{sɯʑɴɢoʁ}{ⓔʑɴɢoʁⓝsɯʑɴɢoʁ} 
\classe{vt}  
\grammaire{caus} \end{sous-entrée}

\begin{sous-entrée}{aʑɴɢɯʑɴɢoʁ}{ⓔʑɴɢoʁⓝaʑɴɢɯʑɴɢoʁ} 
\classe{vi} 
\begin{définition}\pfra{accroché les uns avec les autres}\end{définition}
\begin{définition}\pcmn{钩在一起}\end{définition}
\begin{exemple}\pjya{βʑɯxsɯr nɯ aʑɴɢɯʑɴɢoʁ}\hspace{5pt}\pcmn{牛蒡子的果子钩在一起}\end{exemple}\end{sous-entrée}

\end{entrée}

\begin{entrée}{ʑɴɢro}{}{ⓔʑɴɢro} 
\classe{n} 
\begin{définition}\pfra{guimbarde}\end{définition}
\begin{définition}\pcmn{口簧琴}\end{définition}
\begin{exemple}\pjya{ʑɴɢro nɯ-ʑmbri-t-a}\hspace{5pt}\pcmn{我吹了口簧琴}\end{exemple}
\begin{exemple}\pjya{ʑɴɢro ɲɯ-ɤsɯ-lɤt}\hspace{5pt}\pcmn{他在吹口簧琴}\end{exemple}\end{entrée}

\begin{entrée}{ʑɴɢɯloʁ}{}{ⓔʑɴɢɯloʁ} 
\classe{n} \sens{1}
\begin{définition}\pfra{noix}\end{définition}
\begin{définition}\pcmn{核桃}\end{définition}
\begin{exemple}\pjya{mbrɯtɕɯ kɯ ʑɴɢɯloʁ nɯ-sɯphaʁ-a}\hspace{5pt}\pcmn{我用刀子把核桃撬开了}\end{exemple}
\begin{exemple}\pjya{ʑɴɢɯloʁ nɯ si kɯ-wxtɯ-wxti ci ŋu, ɯ-mat nɯ ɯ-rqhu kɯ-rkɯ-rko ŋu, ɯ-ŋgɯ nɯ tɯ-rnoʁ ɯ-tshɯɣa fse, tú-wɣ-ndza tɕe wuma ʑo mɯm, ɯ-kri tu, ɯ-mat nɯ tɤ-rtɕi sna.}\hspace{5pt}\pcmn{核桃是一种高大的树,果子壳很硬,形状像大脑,吃起来很香,含有油脂,有滋补的作用。}\end{exemple}\sens{2}
\begin{définition}\pfra{Golok}\end{définition}
\begin{définition}\pcmn{果洛}\end{définition}\end{entrée}

\begin{entrée}{ʑo}{₂}{ⓔʑoⓗ2} 
\classe{part} 
\begin{définition}\pfra{particule emphatique}\end{définition}
\begin{définition}\pcmn{强调助词}\end{définition}\end{entrée}

\begin{entrée}{ʑo}{₁}{ⓔʑoⓗ1} 
\classe{vs}  
\grammaire{refl}
\grammaire{caus} \paradigme{dir}{tɤ-}\paradigme{dir}{tɤ-}
\begin{définition}\pfra{léger}\end{définition}
\begin{définition}\pcmn{轻}\end{définition}
\begin{exemple}\pjya{ɯ-fkur ɲɯ-ʑo}\hspace{5pt}\pcmn{他的负担很轻}\end{exemple}\relationsémantique{反义词}{\lien{ⓔrʑi}{rʑi}}
\begin{sous-entrée}{ʑɣɤɣɤʑo}{ⓔʑoⓗ1ⓝʑɣɤɣɤʑo} 
\classe{vi} \end{sous-entrée}

\begin{définition}\pfra{se rendre léger}\end{définition}
\begin{définition}\pcmn{令自己变轻}\end{définition}
\begin{exemple}\pjya{tɤ-muj jamar ʑo to-ʑɣɤɣɤʑo}\hspace{5pt}\pcmn{他令自己变得像羽毛一样轻}\end{exemple}\end{entrée}

\begin{entrée}{ʑru}{}{ⓔʑru} 
\classe{vs} \sens{1}\paradigme{dir}{tɤ-}\paradigme{dir}{thɯ-}
\begin{définition}\pfra{grand et fort}\end{définition}
\begin{définition}\pcmn{强壮}\end{définition}\sens{2}
\begin{définition}\pfra{de bonne qualité, précieux}\end{définition}
\begin{définition}\pcmn{质量好;优质;贵重}\end{définition}
\begin{exemple}\pjya{ɕoŋtɕa ɲɯ-ʑru}\hspace{5pt}\pcmn{木料质量很好}\end{exemple}
\begin{sous-entrée}{nɤʑru}{ⓔʑruⓢ2ⓝnɤʑru} 
\classe{vt} 
\begin{définition}\pfra{trouver précieux}\end{définition}
\begin{définition}\pcmn{觉得贵重}\end{définition}\end{sous-entrée}

\end{entrée}

\begin{entrée}{ʑur}{}{ⓔʑur} 
\classe{vs} \paradigme{dir}{tɤ-}
\begin{définition}\pfra{en quantité suffisante}\end{définition}
\begin{définition}\pcmn{足够}\end{définition}
\begin{exemple}\pjya{kɤ-znɯkro mɯ́j-ʑur}\hspace{5pt}\pcmn{不够分给人家}\end{exemple}
\begin{exemple}\pjya{kɤ-rɤmbi mɯ́j-ʑur}\hspace{5pt}\pcmn{不够送}\end{exemple}\relationsémantique{同义词}{\lien{ⓔrtaʁ}{rtaʁ}}\end{entrée}

\begin{entrée}{ʑro}{}{ⓔʑro} 
\classe{n} 
\begin{définition}\pfra{type d'arbrisseau}\end{définition}
\begin{définition}\pcmn{灌木的一种}\end{définition}\end{entrée}

\begin{entrée}{ʑʁɯnʑʁɯn}{}{ⓔʑʁɯnʑʁɯn} 
\classe{idph.2} 
\begin{définition}\pfra{haut sur ses pieds}\end{définition}
\begin{définition}\pcmn{形容脚高而细长的样子}\end{définition}\end{entrée}

\begin{entrée}{ʑɯ}{₁}{ⓔʑɯⓗ1} 
\classe{vs} \paradigme{dir}{tɤ-}
\begin{définition}\pfra{pas seulement}\end{définition}
\begin{définition}\pcmn{不只}\end{définition}
\begin{exemple}\pjya{a-ʁi tɤtɕɯ kɯmŋu nɯ mɤ-ʑɯ ma mɤʑɯ tɕheme ci tu}\hspace{5pt}\pcmn{不只五个弟弟,还有一个妹妹}\end{exemple}
\begin{exemple}\pjya{tɯrme tɯ-rdoʁ mɯ́j-ʑɯ ma laχsɯm ɣɤʑu-nɯ}\hspace{5pt}\pcmn{不只一个人,有两三个人}\end{exemple}
\begin{exemple}\pjya{paχɕi ɯ-tɯ-wxti kɯ tɤŋkhɯt mɯ́j-ʑɯ (mɤ-kɯ-ʑɯ) jamar ɲɯ-wxti}\hspace{5pt}\pcmn{苹果比拳头大一点}\end{exemple}\relationsémantique{参考}{\lien{ⓔmɤʑɯ}{mɤʑɯ}}\end{entrée}

\begin{entrée}{ʑɯβdaʁ}{}{ⓔʑɯβdaʁ} 
\classe{n} 
\begin{définition}\pfra{dieu de la montagne}\end{définition}
\begin{définition}\pcmn{山神}\end{définition}\étymologie{gʑi.bdag}\end{entrée}

\begin{entrée}{ʑɯβʑɯβ/\variante{ʑɯpʑɯp}}{}{ⓔʑɯβʑɯβ} 
\classe{idph.2} 
\begin{définition}\pfra{beaucoup de chose dressées, beaucoup de gens debout}\end{définition}
\begin{définition}\pcmn{形容很多人、东西、动物等密密麻麻地站着的样子}\end{définition}
\begin{exemple}\pjya{tɤ-rtsho ʑɯβʑɯβ ʑo ɲɯ-pa}\hspace{5pt}\pcmn{麦桩密密麻麻地立在那里}\end{exemple}
\begin{exemple}\pjya{tɯrme ʑɯβʑɯβ ʑo ɲɯ-ndzur-nɯ}\hspace{5pt}\pcmn{很多人在那里站着}\end{exemple}
\begin{exemple}\pjya{fsapaʁ ra zgoku zɯ ʑɯβʑɯβ ʑo ɲɯ-rɤʑi-nɯ}\hspace{5pt}\pcmn{山上牲畜很多,密密麻麻}\end{exemple}
\begin{sous-entrée}{ʑɯβnɤlɯβ}{ⓔʑɯβʑɯβⓝʑɯβnɤlɯβ} 
\classe{idph.4} 
\begin{définition}\pfra{beaucoup de gens allant dans tous les sens s'occupant chacun de leur tâche}\end{définition}
\begin{définition}\pcmn{形容很多人来回走动,各自做自己的事情的场面}\end{définition}
\begin{exemple}\pjya{tɯrme ra ʑɯβnɤlɯβ ʑo ɲɯ-rɤma-nɯ}\hspace{5pt}\pcmn{很多人来回走动,各自做自己的事情}\end{exemple}\relationsémantique{参考}{\lien{ⓔjɯβjɯβ}{jɯβjɯβ}}\relationsémantique{参考}{\lien{ⓔɟɯɣɟɯɣ}{ɟɯɣɟɯɣ}}\end{sous-entrée}

\end{entrée}

\begin{entrée}{ʑɯɣ}{₁}{ⓔʑɯɣⓗ1} 
\classe{vs} \paradigme{dir}{pɯ-}
\begin{définition}\pfra{avoir complètement pourri}\end{définition}
\begin{définition}\pcmn{完全腐烂掉}\end{définition}
\begin{exemple}\pjya{nɯŋa pjɤ-ndʐaβ tɕe, pjɤ-ʑɯɣ}\hspace{5pt}\pcmn{奶牛摔倒了,死了然后尸体完全腐烂掉了}\end{exemple}\relationsémantique{同义词}{\lien{ⓔrɲɯl}{rɲɯl}}\end{entrée}

\begin{entrée}{ʑɯɣ}{₂}{ⓔʑɯɣⓗ2} 
\classe{vt} 
\begin{définition}\pfra{dire bonsoir}\end{définition}
\begin{définition}\pcmn{请晚安}\end{définition}
\begin{exemple}\pjya{sɤrma a-tɤ-ʑɯɣ-nɯ}\hspace{5pt}\pcmn{他们要向他请安}\end{exemple}\étymologie{bʑugs}\end{entrée}

\begin{entrée}{ʑɯm}{}{ⓔʑɯm} 
\classe{vs} \paradigme{dir}{nɯ-}\sens{1}
\begin{définition}\pfra{bon à manger}\end{définition}
\begin{définition}\pcmn{好吃}\end{définition}
\begin{exemple}\pjya{jɯfɕɯr, @cai tɤ-ndza-t-a pɯ-ʑɯm}\hspace{5pt}\pcmn{昨天吃的菜,很好吃}\end{exemple}
\begin{exemple}\pjya{ki kɤ-ndza ki ɲɯ-mɯm tɕe, tú-wɣ-nɯ-ndza tɕe ɲɯ-ʑɯm}\hspace{5pt}\pcmn{这种食物很好吃,吃了感觉很好}\end{exemple}\sens{2}
\begin{définition}\pfra{désaltérant}\end{définition}
\begin{définition}\pcmn{解渴的}\end{définition}
\begin{exemple}\pjya{aʑo tɤ-lu kɤ-tshi-t-a, ɲɯ-ʑɯm}\hspace{5pt}\pcmn{我喝了牛奶,很解渴}\end{exemple}\sens{3}
\begin{définition}\pfra{agréable}\end{définition}
\begin{définition}\pcmn{舒服}\end{définition}
\begin{exemple}\pjya{ki tɯ-ŋga tɤ-ŋga-t-a tɕe, ɲɯ-mpja tɕe pɯ-ʑɯm}\hspace{5pt}\pcmn{我穿了这件衣服,很暖很舒服}\end{exemple}\end{entrée}

\begin{entrée}{ʑɯmkhɤm}{}{ⓔʑɯmkhɤm} 
\classe{n} \sens{1}
\begin{définition}\pfra{domaine}\end{définition}
\begin{définition}\pcmn{管辖区}\end{définition}\sens{2}
\begin{définition}\pfra{beaucoup, un long moment}\end{définition}
\begin{définition}\pcmn{很多;很长时间}\end{définition}
\begin{exemple}\pjya{ʑɯmkhɤm ɯ-xpa}\hspace{5pt}\pcmn{很多年}\end{exemple}
\begin{exemple}\pjya{tɯrme ʑɯmkhɤm ʑo jo-ɣi-nɯ}\hspace{5pt}\pcmn{来了很多人}\end{exemple}\sens{3}
\begin{définition}\pfra{un long moment}\end{définition}
\begin{définition}\pcmn{很长时间}\end{définition}\relationsémantique{参考}{\lien{ⓔʑɯŋkhɤm}{ʑɯŋkhɤm}}\étymologie{ʑiŋ.kʰams}\end{entrée}

\begin{entrée}{ʑɯnmar}{}{ⓔʑɯnmar} 
\classe{n} 
\begin{définition}\pfra{beurre clarifié}\end{définition}
\begin{définition}\pcmn{融酥(净化黄油)}\end{définition}\relationsémantique{参考}{\lien{ⓔta-mar}{ta-mar}}\étymologie{ʑun.mar}\end{entrée}

\begin{entrée}{ʑɯŋkhɤm}{}{ⓔʑɯŋkhɤm} 
\classe{n} 
\begin{définition}\pfra{la terre entière}\end{définition}
\begin{définition}\pcmn{全世界}\end{définition}
\begin{exemple}\pjya{ʑɯŋkhɤm ɯ-ku zɯ}\hspace{5pt}\pcmn{世界上}\end{exemple}\relationsémantique{参考}{\lien{ⓔʑɯmkhɤm}{ʑɯmkhɤm}}\étymologie{ʑiŋ.kʰams}\end{entrée}

\begin{entrée}{ʑɯrɯʑɤri}{}{ⓔʑɯrɯʑɤri} 
\classe{adv} 
\begin{définition}\pfra{progressivement}\end{définition}
\begin{définition}\pcmn{渐渐}\end{définition}
\begin{exemple}\pjya{ʑɯrɯʑɤri tɕe chɯ-chɯ-mɤɕi-ndʑi nɤ chɯ-chɯ-mɤɕi-ndʑi}\hspace{5pt}\pcmn{他们俩逐渐变得越来越富裕}\end{exemple}\relationsémantique{参考}{\lien{ⓔɯ-ʑɤrʑɯr}{ɯ-ʑɤrʑɯr}}\end{entrée}

\begin{entrée}{ʑɯxsa}{}{ⓔʑɯxsa} 
\classe{n} 
\begin{définition}\pfra{siège, place où l'on s'assoit (honorifique)}\end{définition}
\begin{définition}\pcmn{座位(敬语)}\end{définition}\étymologie{bʑugs.sa}\end{entrée}

\begin{entrée}{ʑɯxsɯr}{}{ⓔʑɯxsɯr} 
\classe{n} 
\begin{définition}\pfra{fruit de la bardane}\end{définition}
\begin{définition}\pcmn{牛蒡子}\end{définition}\relationsémantique{参考}{\lien{ⓔtɤtɕɯβraʁ}{tɤtɕɯβraʁ}}\end{entrée}

\begin{entrée}{ʑwɤʑwɤr}{}{ⓔʑwɤʑwɤr} 
\classe{idph.2} 
\begin{définition}\pfra{de travers (chapeau)}\end{définition}
\begin{définition}\pcmn{形容不周正的样子(帽子)}\end{définition}\relationsémantique{同义词}{\lien{ⓔjwɤjwɤr}{jwɤjwɤr}}
\begin{sous-entrée}{ʑwɤrnɤlɤr}{ⓔʑwɤʑwɤrⓝʑwɤrnɤlɤr} 
\classe{idph.4} 
\begin{définition}\pfra{gesticulant}\end{définition}
\begin{définition}\pcmn{形容指手画脚的样子}\end{définition}
\begin{exemple}\pjya{ʑwɤrnɤlɤr ʑo ma-tɯ-ʑɣɤstu kɯ phoʁphoʁ kɤ-rɤʑi}\hspace{5pt}\pcmn{你不要指手画脚,规矩一点}\end{exemple}\end{sous-entrée}

\end{entrée}

\end{multicols}
\end{document}